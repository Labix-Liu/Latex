\documentclass[a4paper]{article}

%=========================================
% Packages
%=========================================
\usepackage{mathtools}
\usepackage{amsfonts}
\usepackage{amsmath}
\usepackage{amssymb}
\usepackage{amsthm}
\usepackage[a4paper, total={6in, 8in}, margin=1in]{geometry}
\usepackage[utf8]{inputenc}
\usepackage{fancyhdr}
\usepackage[utf8]{inputenc}
\usepackage{graphicx}
\usepackage{physics}
\usepackage[listings]{tcolorbox}
\usepackage{hyperref}
\usepackage{tikz-cd}
\usepackage{adjustbox}
\usepackage{enumitem}
\usepackage[font=small,labelfont=bf]{caption}
\usepackage{subcaption}
\usepackage{wrapfig}
\usepackage{makecell}



\raggedright

\usetikzlibrary{arrows.meta}

\DeclarePairedDelimiter\ceil{\lceil}{\rceil}
\DeclarePairedDelimiter\floor{\lfloor}{\rfloor}

%=========================================
% Fonts
%=========================================
\usepackage{tgpagella}
\usepackage[T1]{fontenc}


%=========================================
% Custom Math Operators
%=========================================
\DeclareMathOperator{\adj}{adj}
\DeclareMathOperator{\im}{im}
\DeclareMathOperator{\nullity}{nullity}
\DeclareMathOperator{\sign}{sign}
\DeclareMathOperator{\dom}{dom}
\DeclareMathOperator{\lcm}{lcm}
\DeclareMathOperator{\ran}{ran}
\DeclareMathOperator{\ext}{Ext}
\DeclareMathOperator{\dist}{dist}
\DeclareMathOperator{\diam}{diam}
\DeclareMathOperator{\aut}{Aut}
\DeclareMathOperator{\inn}{Inn}
\DeclareMathOperator{\syl}{Syl}
\DeclareMathOperator{\edo}{End}
\DeclareMathOperator{\cov}{Cov}
\DeclareMathOperator{\vari}{Var}
\DeclareMathOperator{\cha}{char}
\DeclareMathOperator{\Span}{span}
\DeclareMathOperator{\ord}{ord}
\DeclareMathOperator{\res}{res}
\DeclareMathOperator{\Hom}{Hom}
\DeclareMathOperator{\Mor}{Mor}
\DeclareMathOperator{\coker}{coker}
\DeclareMathOperator{\Obj}{Obj}
\DeclareMathOperator{\id}{id}
\DeclareMathOperator{\GL}{GL}
\DeclareMathOperator*{\colim}{colim}

%=========================================
% Custom Commands (Shortcuts)
%=========================================
\newcommand{\CP}{\mathbb{CP}}
\newcommand{\GG}{\mathbb{G}}
\newcommand{\F}{\mathbb{F}}
\newcommand{\N}{\mathbb{N}}
\newcommand{\Q}{\mathbb{Q}}
\newcommand{\R}{\mathbb{R}}
\newcommand{\C}{\mathbb{C}}
\newcommand{\E}{\mathbb{E}}
\newcommand{\Prj}{\mathbb{P}}
\newcommand{\RP}{\mathbb{RP}}
\newcommand{\T}{\mathbb{T}}
\newcommand{\Z}{\mathbb{Z}}
\newcommand{\A}{\mathbb{A}}
\renewcommand{\H}{\mathbb{H}}
\newcommand{\K}{\mathbb{K}}

\newcommand{\mA}{\mathcal{A}}
\newcommand{\mB}{\mathcal{B}}
\newcommand{\mC}{\mathcal{C}}
\newcommand{\mD}{\mathcal{D}}
\newcommand{\mE}{\mathcal{E}}
\newcommand{\mF}{\mathcal{F}}
\newcommand{\mG}{\mathcal{G}}
\newcommand{\mH}{\mathcal{H}}
\newcommand{\mI}{\mathcal{I}}
\newcommand{\mJ}{\mathcal{J}}
\newcommand{\mK}{\mathcal{K}}
\newcommand{\mL}{\mathcal{L}}
\newcommand{\mM}{\mathcal{M}}
\newcommand{\mO}{\mathcal{O}}
\newcommand{\mP}{\mathcal{P}}
\newcommand{\mS}{\mathcal{S}}
\newcommand{\mT}{\mathcal{T}}
\newcommand{\mV}{\mathcal{V}}
\newcommand{\mW}{\mathcal{W}}

%=========================================
% Colours!!!
%=========================================
\definecolor{LightBlue}{HTML}{2D64A6}
\definecolor{ForestGreen}{HTML}{4BA150}
\definecolor{DarkBlue}{HTML}{000080}
\definecolor{LightPurple}{HTML}{cc99ff}
\definecolor{LightOrange}{HTML}{ffc34d}
\definecolor{Buff}{HTML}{DDAE7E}
\definecolor{Sunset}{HTML}{F2C57C}
\definecolor{Wenge}{HTML}{584B53}
\definecolor{Coolgray}{HTML}{9098CB}
\definecolor{Lavender}{HTML}{D6E3F8}
\definecolor{Glaucous}{HTML}{828BC4}
\definecolor{Mauve}{HTML}{C7A8F0}
\definecolor{Darkred}{HTML}{880808}
\definecolor{Beaver}{HTML}{9A8873}
\definecolor{UltraViolet}{HTML}{52489C}



%=========================================
% Theorem Environment
%=========================================
\tcbuselibrary{listings, theorems, breakable, skins}

\newtcbtheorem[number within = subsection]{thm}{Theorem}%
{	colback=Buff!3, 
	colframe=Buff, 
	fonttitle=\bfseries, 
	breakable, 
	enhanced jigsaw, 
	halign=left
}{thm}

\newtcbtheorem[number within=subsection, use counter from=thm]{defn}{Definition}%
{  colback=cyan!1,
    colframe=cyan!50!black,
	fonttitle=\bfseries, breakable, 
	enhanced jigsaw, 
	halign=left
}{defn}

\newtcbtheorem[number within=subsection, use counter from=thm]{axm}{Axiom}%
{	colback=red!5, 
	colframe=Darkred, 
	fonttitle=\bfseries, 
	breakable, 
	enhanced jigsaw, 
	halign=left
}{axm}

\newtcbtheorem[number within=subsection, use counter from=thm]{prp}{Proposition}%
{	colback=LightBlue!3, 
	colframe=Glaucous, 
	fonttitle=\bfseries, 
	breakable, 
	enhanced jigsaw, 
	halign=left
}{prp}

\newtcbtheorem[number within=subsection, use counter from=thm]{lmm}{Lemma}%
{	colback=LightBlue!3, 
	colframe=LightBlue!60, 
	fonttitle=\bfseries, 
	breakable, 
	enhanced jigsaw, 
	halign=left
}{lmm}

\newtcbtheorem[number within=subsection, use counter from=thm]{crl}{Corollary}%
{	colback=LightBlue!3, 
	colframe=LightBlue!60, 
	fonttitle=\bfseries, 
	breakable, 
	enhanced jigsaw, 
	halign=left
}{crl}

\newtcbtheorem[number within=subsection, use counter from=thm]{eg}{Example}%
{	colback=Beaver!5, 
	colframe=Beaver, 
	fonttitle=\bfseries, 
	breakable, 
	enhanced jigsaw, 
	halign=left
}{eg}

\newtcbtheorem[number within=subsection, use counter from=thm]{ex}{Exercise}%
{	colback=Beaver!5, 
	colframe=Beaver, 
	fonttitle=\bfseries, 
	breakable, 
	enhanced jigsaw, 
	halign=left
}{ex}

\newtcbtheorem[number within=subsection, use counter from=thm]{alg}{Algorithm}%
{	colback=UltraViolet!5, 
	colframe=UltraViolet, 
	fonttitle=\bfseries, 
	breakable, 
	enhanced jigsaw, 
	halign=left
}{alg}




%=========================================
% Hyperlinks
%=========================================
\hypersetup{
    colorlinks=true, %set true if you want colored links
    linktoc=all,     %set to all if you want both sections and subsections linked
    linkcolor=DarkBlue,  %choose some color if you want links to stand out
}


\pagestyle{fancy}
\fancyhf{}
\rhead{Labix}
\lhead{Statement of Purpose: Johns Hopkins}
\rfoot{\thepage}

\title{Statement of Purpose: Johns Hopkins}

\author{Labix}

\date{\today}
\begin{document}
My interest in maths first came when I was in high school. Ian Stewart’s books served as my light introduction to the world of maths. In his book “The Seventeen Equations that Changed the World”, emphasis was put on the historical meaning and the intuition of the mathematicians behind these equations. I was motivated by the book and engaged with textbooks far outside my high school curriculum and began reading ahead of the course. \\~\\

In my third year, I worked on an essay featuring Kähler differentials under Dr. Gallauer. The essay involves motivating the definition of the Kähler differentials, proving some properties and exact sequences of the module. It was fascinating to construct my own examples and to compute all kinds of modules structure and isomorphisms. I compared the module of Kähler differentials with the smooth 1-forms on a manifold and proved that these two constructs are not isomorphic using the real line as an example. \\~\\

For my 4th year research project, I decided to study the equivalence of categories between linear functors on spaces and spectra under Dr. Dotto. This is motivated by my interests in cohomology theories and the fact that Brown’s Representability Theorem means that spectra give rise to cohomology theories and vice versa. Passing on to linear functors may be able to reveal what conditions on a functor allows for a cohomology theory. \\~\\

For me, the beauty in maths lies in how it can make sense of abstract objects. This is especially true with Algebraic Topology, where abstract spaces can be assigned concrete algebraic invariants in which we can use to deduce properties. Thanks to the projects, I have become interested in applying homological / homotopical techniques to deducing information from different mathematical objects such as algebraic structures, spaces or even data structures, and I imagine myself working in such areas in the future. Throughout my journey, I discovered my passion for maths and therefore would like to pursue further studies. Not only would I be able to learn more about the inner workings of the subject, I am also excited to contribute to the field that I enjoyed learning, and this would greatly help my career path. \\~\\

Having a large community of Geometers and Topologists is beneficial to pursuing my PhD studies as it gives me the chance to participate in more events and converse with like-minded people. Oxford offers a unique opportunity due to its large Geometry and Topology groups and its focus on specific areas that align with my interests. This includes the application of algebraic topology to the study of algebraic structures. In this sense, K Theory provides exact sequences and isomorphisms that reveal the structure of rings, such as the isomorphism conjectures for group rings. My background with Kähler differentials also interests me in the intersection of Algebraic Geometry and Algebraic Topology. The weekly algebra and topology seminars will also be a great chance to widen my knowledge in related fields. \\~\\

Beyond technical knowledge, I have also honed soft skills for PhD research. My role as the Talks Coordinator in Warwick Maths Society meant that I had to organize weekly math talks. Conversing with professors across different disciplines had them impart much of their passion to me. I also co-organized the Warwick Imperial Maths Conferences, which included working with Imperial College London to contact speakers and arrange for rooms and refreshments for the talks. \\~\\

In this past summer, I have been researching LLMs for education as an intern under an IT company called edxtore. I gained research experience since I had to read papers related to LLMs and to implement them. Not given a timetable means that I had to track progress independently whilst avoiding getting burnt out. The time management skills and paper reading skills I learned will be critical in my PhD studies. \\~\\

I was given the opportunity to teach a class of gifted students on set theory from my high school. Helping other people understand the beauty of maths sparked an unknown passion of teaching for me. I was also hired by my university to supervise ten first year students and assist them with core math modules and have been enjoying my job. Being a teacher gave me plenty of chances to communicate mathematical concepts to students colloquially, a skill that will benefit me as I pursue research. \\~\\

Equipped with a wide variety of research and communication skills together with the foundational knowledge that I have been in pursuit of, I am in an excellent position to embark on a PhD in Algebraic Topology.






\end{document}