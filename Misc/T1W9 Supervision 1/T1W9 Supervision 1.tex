\documentclass[a4paper]{article}

\input{C:/Users/liula/Desktop/Latex/Headers V1.2.tex}

\pagestyle{fancy}
\fancyhf{}
\rhead{Labix}
\lhead{T1W9 Supervision 1}
\rfoot{\thepage}

\title{T1W9 Supervision 1}

\author{Labix}

\date{\today}
\begin{document}
\section*{Exercise Sheet for Week 9}
\subsection*{Question 1}
In each of the following rings, which of them are commutative? Which of them admits a multiplicative inverse for every non-zero element? 
\begin{itemize}
\item The integers $(\Z,+,\times)$
\item The rational numbers $(\Q,+,\times)$
\item The real numbers $(\R,+,\times)$
\item The complex numbers $(\C,+,\times)$
\item The matrix ring $(M_{n\times n}(\R),+,\times)$ for $n\in\N\setminus\{0\}$ (Hint: The answer to both questions may depend on $n$)
\item The ring of matrices of non-zero determinant $(GL_n(\R),+,\times)$ for $n\in\N\setminus\{0\}$
\item The congruent numbers $(\Z/n\Z,+,\times)$ for $n\in\N\setminus\{0\}$ (Hint: whether all non-zero elements have a multiplicative inverse depends on $n$)
\item The polynomial ring $\R[x]$
\end{itemize}

Upshot: To give you abundance of examples of rings (commutative and non-commutative), (fields / non-fields)

\subsection*{Question 2}
The question will provide you with a ring and a subset of the ring. Which of the subsets is an ideal of the ring? Which of the subsets is a subring of the ring? (Subtle question: what is the difference between an ideal and a subring?)

\begin{itemize}
\item The integers $\Z$ and the subset $n\Z=\{kn\;|\;k\in\Z\}$ for some $n\in\N\setminus\{0\}$
\item The rational numbers $\Q$ and the subset $\frac{5}{7}\Z=\{\frac{5}{7}k\;|\;k\in\Z\}$
\item The congruence group $\Z/6\Z$ and the subset $\{1+\Z,5+\Z\}$. 
\item The general linear group $GL_n(\R)$ for $n\in\N\setminus\{0\}$ and the subset $\{M\in GL_n(\R)\;|\;\det(M)=1\}$
\item The polynomial ring $\R[x]$ and the subset $\R$
\item The polynomial ring $\R[x]$ and the subset $\{x-a\in\R[x]\;|\;a\in\R\}$
\end{itemize}

Is every subring an ideal? Is every ideal a subring? Prove or give a counter example. \\

Upshot: Subsets of a ring may be an ideal, may be a subring, may be neither. 

\subsection*{Question 3}
Let $(R,+,\cdot)$ be a ring. Prove of find a counter example to the following statements. 
\begin{itemize}
\item $(R,+)$ is an abelian group
\item $(R,\cdot)$ is an abelian group
\item Let $I$ be an ideal of $R$. Then $(I,+)$ is an abelian group
\item Let $R^\times$ be the group of units in $R$. Then $(R^\times,+)$ is a group
\item Let $R^\times$ be the group of units in $R$. Then $(R^\times,\cdot)$ is an abelian group
\end{itemize}

Upshot: Just a sanity check

\subsection*{Question 4}
Find the multiplicative inverse of $17$ in $\Z/100\Z$. \\

Upshot: Bezout's lemma comes up everywhere!

\subsection*{Question 5}
Prove or find a counter example: Every irreducible polynomial in $\R[x]$ is irreducible in $\C[x]$. \\

Express $x^4+4x^3+5x^2-2x-8\in\R[x]$ as a product of irreducible polynomials in $\R[x]$. Now express the same polynomial as a product of irreducible polynomials in $\C[x]$. \\

Upshot: While factorization of a polynomial over a given polynomial ring is unique up to shuffling their factors, factorization considered over a different background ring may lead to different factorization. This is because some irreducible polynomials become reducible in a larger ambient background. 





\end{document}