\documentclass[a4paper]{article}

%=========================================
% Packages
%=========================================
\usepackage{mathtools}
\usepackage{amsfonts}
\usepackage{amsmath}
\usepackage{amssymb}
\usepackage{amsthm}
\usepackage[a4paper, total={6in, 8in}, margin=1in]{geometry}
\usepackage[utf8]{inputenc}
\usepackage{fancyhdr}
\usepackage[utf8]{inputenc}
\usepackage{graphicx}
\usepackage{physics}
\usepackage[listings]{tcolorbox}
\usepackage{hyperref}
\usepackage{tikz-cd}
\usepackage{adjustbox}
\usepackage{enumitem}
\usepackage[font=small,labelfont=bf]{caption}
\usepackage{subcaption}
\usepackage{wrapfig}
\usepackage{makecell}



\raggedright

\usetikzlibrary{arrows.meta}

\DeclarePairedDelimiter\ceil{\lceil}{\rceil}
\DeclarePairedDelimiter\floor{\lfloor}{\rfloor}

%=========================================
% Fonts
%=========================================
\usepackage{tgpagella}
\usepackage[T1]{fontenc}


%=========================================
% Custom Math Operators
%=========================================
\DeclareMathOperator{\adj}{adj}
\DeclareMathOperator{\im}{im}
\DeclareMathOperator{\nullity}{nullity}
\DeclareMathOperator{\sign}{sign}
\DeclareMathOperator{\dom}{dom}
\DeclareMathOperator{\lcm}{lcm}
\DeclareMathOperator{\ran}{ran}
\DeclareMathOperator{\ext}{Ext}
\DeclareMathOperator{\dist}{dist}
\DeclareMathOperator{\diam}{diam}
\DeclareMathOperator{\aut}{Aut}
\DeclareMathOperator{\inn}{Inn}
\DeclareMathOperator{\syl}{Syl}
\DeclareMathOperator{\edo}{End}
\DeclareMathOperator{\cov}{Cov}
\DeclareMathOperator{\vari}{Var}
\DeclareMathOperator{\cha}{char}
\DeclareMathOperator{\Span}{span}
\DeclareMathOperator{\ord}{ord}
\DeclareMathOperator{\res}{res}
\DeclareMathOperator{\Hom}{Hom}
\DeclareMathOperator{\Mor}{Mor}
\DeclareMathOperator{\coker}{coker}
\DeclareMathOperator{\Obj}{Obj}
\DeclareMathOperator{\id}{id}
\DeclareMathOperator{\GL}{GL}
\DeclareMathOperator*{\colim}{colim}

%=========================================
% Custom Commands (Shortcuts)
%=========================================
\newcommand{\CP}{\mathbb{CP}}
\newcommand{\GG}{\mathbb{G}}
\newcommand{\F}{\mathbb{F}}
\newcommand{\N}{\mathbb{N}}
\newcommand{\Q}{\mathbb{Q}}
\newcommand{\R}{\mathbb{R}}
\newcommand{\C}{\mathbb{C}}
\newcommand{\E}{\mathbb{E}}
\newcommand{\Prj}{\mathbb{P}}
\newcommand{\RP}{\mathbb{RP}}
\newcommand{\T}{\mathbb{T}}
\newcommand{\Z}{\mathbb{Z}}
\newcommand{\A}{\mathbb{A}}
\renewcommand{\H}{\mathbb{H}}
\newcommand{\K}{\mathbb{K}}

\newcommand{\mA}{\mathcal{A}}
\newcommand{\mB}{\mathcal{B}}
\newcommand{\mC}{\mathcal{C}}
\newcommand{\mD}{\mathcal{D}}
\newcommand{\mE}{\mathcal{E}}
\newcommand{\mF}{\mathcal{F}}
\newcommand{\mG}{\mathcal{G}}
\newcommand{\mH}{\mathcal{H}}
\newcommand{\mI}{\mathcal{I}}
\newcommand{\mJ}{\mathcal{J}}
\newcommand{\mK}{\mathcal{K}}
\newcommand{\mL}{\mathcal{L}}
\newcommand{\mM}{\mathcal{M}}
\newcommand{\mO}{\mathcal{O}}
\newcommand{\mP}{\mathcal{P}}
\newcommand{\mS}{\mathcal{S}}
\newcommand{\mT}{\mathcal{T}}
\newcommand{\mV}{\mathcal{V}}
\newcommand{\mW}{\mathcal{W}}

%=========================================
% Colours!!!
%=========================================
\definecolor{LightBlue}{HTML}{2D64A6}
\definecolor{ForestGreen}{HTML}{4BA150}
\definecolor{DarkBlue}{HTML}{000080}
\definecolor{LightPurple}{HTML}{cc99ff}
\definecolor{LightOrange}{HTML}{ffc34d}
\definecolor{Buff}{HTML}{DDAE7E}
\definecolor{Sunset}{HTML}{F2C57C}
\definecolor{Wenge}{HTML}{584B53}
\definecolor{Coolgray}{HTML}{9098CB}
\definecolor{Lavender}{HTML}{D6E3F8}
\definecolor{Glaucous}{HTML}{828BC4}
\definecolor{Mauve}{HTML}{C7A8F0}
\definecolor{Darkred}{HTML}{880808}
\definecolor{Beaver}{HTML}{9A8873}
\definecolor{UltraViolet}{HTML}{52489C}



%=========================================
% Theorem Environment
%=========================================
\tcbuselibrary{listings, theorems, breakable, skins}

\newtcbtheorem[number within = subsection]{thm}{Theorem}%
{	colback=Buff!3, 
	colframe=Buff, 
	fonttitle=\bfseries, 
	breakable, 
	enhanced jigsaw, 
	halign=left
}{thm}

\newtcbtheorem[number within=subsection, use counter from=thm]{defn}{Definition}%
{  colback=cyan!1,
    colframe=cyan!50!black,
	fonttitle=\bfseries, breakable, 
	enhanced jigsaw, 
	halign=left
}{defn}

\newtcbtheorem[number within=subsection, use counter from=thm]{axm}{Axiom}%
{	colback=red!5, 
	colframe=Darkred, 
	fonttitle=\bfseries, 
	breakable, 
	enhanced jigsaw, 
	halign=left
}{axm}

\newtcbtheorem[number within=subsection, use counter from=thm]{prp}{Proposition}%
{	colback=LightBlue!3, 
	colframe=Glaucous, 
	fonttitle=\bfseries, 
	breakable, 
	enhanced jigsaw, 
	halign=left
}{prp}

\newtcbtheorem[number within=subsection, use counter from=thm]{lmm}{Lemma}%
{	colback=LightBlue!3, 
	colframe=LightBlue!60, 
	fonttitle=\bfseries, 
	breakable, 
	enhanced jigsaw, 
	halign=left
}{lmm}

\newtcbtheorem[number within=subsection, use counter from=thm]{crl}{Corollary}%
{	colback=LightBlue!3, 
	colframe=LightBlue!60, 
	fonttitle=\bfseries, 
	breakable, 
	enhanced jigsaw, 
	halign=left
}{crl}

\newtcbtheorem[number within=subsection, use counter from=thm]{eg}{Example}%
{	colback=Beaver!5, 
	colframe=Beaver, 
	fonttitle=\bfseries, 
	breakable, 
	enhanced jigsaw, 
	halign=left
}{eg}

\newtcbtheorem[number within=subsection, use counter from=thm]{ex}{Exercise}%
{	colback=Beaver!5, 
	colframe=Beaver, 
	fonttitle=\bfseries, 
	breakable, 
	enhanced jigsaw, 
	halign=left
}{ex}

\newtcbtheorem[number within=subsection, use counter from=thm]{alg}{Algorithm}%
{	colback=UltraViolet!5, 
	colframe=UltraViolet, 
	fonttitle=\bfseries, 
	breakable, 
	enhanced jigsaw, 
	halign=left
}{alg}




%=========================================
% Hyperlinks
%=========================================
\hypersetup{
    colorlinks=true, %set true if you want colored links
    linktoc=all,     %set to all if you want both sections and subsections linked
    linkcolor=DarkBlue,  %choose some color if you want links to stand out
}


\pagestyle{fancy}
\fancyhf{}
\rhead{Labix}
\lhead{T1W9 Supervision 1}
\rfoot{\thepage}

\title{T1W9 Supervision 1}

\author{Labix}

\date{\today}
\begin{document}
\section*{Exercise Sheet for Week 9}
\subsection*{Question 1}
In each of the following rings, which of them are commutative? Which of them admits a multiplicative inverse for every non-zero element? 
\begin{itemize}
\item The integers $(\Z,+,\times)$
\item The rational numbers $(\Q,+,\times)$
\item The real numbers $(\R,+,\times)$
\item The complex numbers $(\C,+,\times)$
\item The matrix ring $(M_{n\times n}(\R),+,\times)$ for $n\in\N\setminus\{0\}$ (Hint: The answer to both questions may depend on $n$)
\item The congruent numbers $(\Z/n\Z,+,\times)$ for $n\in\N\setminus\{0\}$ (Hint: whether all non-zero elements have a multiplicative inverse depends on $n$)
\item The polynomial ring $\R[x]$
\end{itemize}

Upshot: To give you abundance of examples of rings (commutative and non-commutative), (fields / non-fields)

\subsection*{Question 2}
The question will provide you with a ring and a subset of the ring. Which of the subsets is an ideal of the ring? Which of the subsets is a subring of the ring? (Subtle question: what is the difference between an ideal and a subring?)

\begin{itemize}
\item The integers $\Z$ and the subset $n\Z=\{kn\;|\;k\in\Z\}$ for some $n\in\N\setminus\{0\}$
\item The rational numbers $\Q$ and the subset $\frac{5}{7}\Z=\{\frac{5}{7}k\;|\;k\in\Z\}$
\item The congruence group $\Z/6\Z$ and the subset $\{1+\Z,5+\Z\}$. 
\item The polynomial ring $\R[x]$ and the subset $\R$
\item The polynomial ring $\R[x]$ and the subset $\{x-a\in\R[x]\;|\;a\in\R\}$
\end{itemize}

Is every subring an ideal? Is every ideal a subring? Prove or give a counter example. \\

Upshot: Subsets of a ring may be an ideal, may be a subring, may be neither. 

\subsection*{Question 3}
Let $(R,+,\cdot)$ be a ring. Prove of find a counter example to the following statements. 
\begin{itemize}
\item $(R,+)$ is an abelian group
\item $(R,\cdot)$ is an abelian group
\item Let $I$ be an ideal of $R$. Then $(I,+)$ is an abelian group
\item Let $R^\times$ be the set of units in $R$. Then $(R^\times,+)$ is a group
\item Let $R^\times$ be the set of units in $R$. Then $(R^\times,\cdot)$ is an abelian group
\end{itemize}

Upshot: Just a sanity check

\subsection*{Question 4}
Find the multiplicative inverse of $17$ in $\Z/100\Z$. \\

Upshot: Bezout's lemma comes up everywhere!

\subsection*{Question 5}
Prove or find a counter example: Every irreducible polynomial in $\R[x]$ is irreducible in $\C[x]$. \\

Express $x^4+4x^3+5x^2-2x-8\in\R[x]$ as a product of irreducible polynomials in $\R[x]$ (Hint: $1$ is a root). Now express the same polynomial as a product of irreducible polynomials in $\C[x]$. \\

Upshot: While factorization of a polynomial over a given polynomial ring is unique up to shuffling their factors, factorization considered over a different background ring may lead to different factorization. This is because some irreducible polynomials become reducible in a larger ambient background. 

\pagebreak
\section*{Answers}
\subsection*{Question 1}
The following rings are commutative: $\Z,\Q,\R,\C,M_{1\times 1}(\R),\Z/n\Z,\R[x]$. \\

The following rings admit a multiplicative inverse for every non-zero element:: $\Q,\R,\C,M_{1\times 1}(\R),\Z/n\Z$ when $n$ is prime. 

\subsection*{Question 2}
The following pairs give a ring and an ideal: 
\begin{itemize}
\item $\Z$ and $n\Z$
\end{itemize}

The following pairs give a ring and a subring: 
\begin{itemize}
\item $\Z$ and $n\Z$
\item $\R[x]$ and $\R$
\end{itemize}

\subsection*{Question 3}
\begin{itemize}
\item True
\item Consider $R=M_{2\times 2}(\R)$
\item True
\item Consider $R=M_{2\times 2}(\R)$
\item Consider $R=M_{2\times 2}(\R)$
\end{itemize}

\subsection*{Question 4}
The inverse is $53\Z$

\subsection*{Question 5}
$x^2+1$ is irreducible in $\R[x]$ but reducible in $\C[x]$. \\

$x^4+4x^3+5x^2-2x-8=(x^2+3x+4)(x+2)(x-1)$ in $\R[x]$. \\
$1/4 (-2 i x + \sqrt{7} - 3 i) (2 i x + \sqrt{7} + 3 i) (x - 1) (x + 2)$ in $\C[x]$
\end{document}