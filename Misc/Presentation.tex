\documentclass[10pt]{beamer}
\usepackage{tikz-cd}
\usepackage{adjustbox}
\usetheme{Copenhagen}

\newcommand{\CP}{\mathbb{CP}}
\newcommand{\GG}{\mathbb{G}}
\newcommand{\F}{\mathbb{F}}
\newcommand{\N}{\mathbb{N}}
\newcommand{\Q}{\mathbb{Q}}
\newcommand{\R}{\mathbb{R}}
\newcommand{\C}{\mathbb{C}}
\newcommand{\E}{\mathbb{E}}
\newcommand{\Prj}{\mathbb{P}}
\newcommand{\RP}{\mathbb{RP}}
\newcommand{\T}{\mathbb{T}}
\newcommand{\Z}{\mathbb{Z}}
\newcommand{\A}{\mathbb{A}}
\renewcommand{\H}{\mathbb{H}}
\newcommand{\K}{\mathbb{K}}

\newcommand{\mC}{\mathcal{C}}


\usecolortheme{default}
%Information to be included in the title page:
\title{Algebraic Differential Forms}
\author{Labix}

\begin{document}

\frame{\titlepage}

\begin{frame}[fragile]
\frametitle{Motivation}

Let $M$ be a smooth manifold. \\~\\

For each point $p\in M$, we have the cotangent space $T_p^\ast M$. \\~\\

They organize into a vector bundle $T^\ast M$. \\~\\

Smooth sections of $T^\ast M$ are called smooth differential $1$-forms. Its collection organizes into a $\mathcal{C}^\infty(M)$-module denoted by $\Omega^1(M)$. \\~\\

\adjustbox{scale=1.0,center}{\begin{tikzcd}
	0 & {\mathcal{C}^\infty(M)} & {\Omega^1(M)} & {\Omega^2(M)} & \cdots
	\arrow[from=1-1, to=1-2]
	\arrow["d", from=1-2, to=1-3]
	\arrow["d", from=1-3, to=1-4]
	\arrow[from=1-4, to=1-5]
\end{tikzcd}}\\~\\

We would like to have a similar notion for varieties in algebraic geometry. 

\end{frame}

\begin{frame}[fragile]
\frametitle{Definitions}

\begin{block}{Derivations} Let $A$ be a ring and $B$ an $A$-algebra. Let $M$ be a $B$-module. An $A$-derivation of $B$ into $M$ is an $A$-module homomorphism $d:B\to M$ such that the Leibniz rule holds: $$d(b_1b_2)=b_1d(b_2)+d(b_1)b_2$$ for $b_1,b_2\in B$. 
\end{block}

Denote the set of all $A$-derivations from $B$ to $M$ by $$\text{Der}_A(B,M)=\{d:B\to M\;|\;d\text{ is an }A\text{ derivation }\}$$

\begin{examples} Let $M$ be a smooth manifold. Then $$T_p(M)=\text{Der}_\R(\mathcal{C}_{M,p}^\infty,\R)$$ where $\mathcal{C}_{M,p}^\infty$ is the germ of smooth functions at $p$. 
\end{examples}


\end{frame}

\begin{frame}[fragile]
\frametitle{Definitions}

\begin{block}{Kähler Differentials} Let $A$ be a ring and let $B$ be an $A$-algebra. A $B$-module $\Omega_{B/A}^1$ together with an $A$-derivation $d:B\to\Omega_{B/A}^1$ is said to be a module Kähler Differentials of $B$ over $A$ if it satisfies the following universal property: \\~\\
For any $B$-module $M$, and for any $A$-derivation $d':B\to M$, there exists a unique $B$-module homomorphism $f:\Omega_{B/A}^1\to M$ such that $d'=f\circ d$. In other words, the following diagram commutes: \\
\adjustbox{scale=1.0,center}{\begin{tikzcd}
B\arrow[r, "d"]\arrow[rd, "d'"'] & \Omega_{B/A}^1\arrow[d, "\exists!f", dashed]\\
& M
\end{tikzcd}}
\end{block}
\end{frame}

\begin{frame}[fragile]
\frametitle{Constructions}

\begin{block}{Kähler Differentials as the Quotient of a Free Module} Let $A$ be a ring and $B$ be an $A$-algebra. Let $F$ be the free $B$-module generated by the symbols $\{d(b)\;|\;b\in B\}$. Let $R$ be the submodule of $F$ generated by the following relations: 
\begin{itemize}
\item $d(a_1b_1+a_2b_2)-a_1d(b_1)-a_2d(b_2)$ for all $b_1,b_2\in B$ and $a_1,a_2\in A$
\item $d(b_1b_2)-b_1d(b_2)-b_2d(b_1)$ for all $b_1,b_2\in B$
\end{itemize}
Then $F/R$ is a module of Kähler Differentials for $B$ over $A$. 
\end{block}

\begin{block}{Kähler Differentials as a Kernel} Let $A$ be a ring and $B$ be an $A$-algebra. Let $f:B\otimes_A B\to B$ be a function defined to be $f(b_1\otimes_A b_2)=b_1b_2$. Let $I$ be the kernel of $f$. Then $(I/I^2,d)$ is a module of Kähler Differentials of $B$ over $A$, where the derivation is the homomorphism $d:B\to I/I^2$ defined by $db=1\otimes b-b\otimes1\;(\bmod\;I^2)$. 
\end{block}
\end{frame}

\begin{frame}[fragile]
\frametitle{Two Exact Sequences}

\begin{block}{First Exact Sequence}
Let $B,C$ be $A$-algebras and let $\phi:B\to C$ be an $A$-algebra homomorphism. Then the following sequence is an exact sequence of $C$-modules: \\
\adjustbox{scale=1.0,center}{\begin{tikzcd}
\Omega_{B/A}^1\otimes_BC\arrow[r] & \Omega_{C/A}^1\arrow[r] & \Omega_{C/B}^1\arrow[r] & 0
\end{tikzcd}} \\
\end{block}

\begin{block}{Second Exact Sequence}
Let $A$ be a ring and $B$ an $A$-algebra. Let $I$ be an ideal of $B$ and $C=B/I$. Then the following sequence is an exact sequence of $C$-modules: \\
\adjustbox{scale=1.0,center}{\begin{tikzcd}
I/I^2\arrow[r] & \Omega_{B/A}^1\otimes_B C\arrow[r] & \Omega_{C/A}^1\arrow[r] & 0
\end{tikzcd}}\\
\end{block}

\end{frame}
\begin{frame}[fragile]
\frametitle{Some properties of the module}
\begin{block}{Commutes with localization}
Let $B$ be an algebra over $A$. Let $S$ be a multiplicative subset of $B$. Then $$S^{-1}\Omega_{B/A}^1\cong\Omega_{S^{-1}B/A}^1$$
\end{block}

\begin{block}{Computing using the Jacobian}
Let $A$ be a field. Let $C=\frac{A[x_1,\dots,x_n]}{(f_1,\dots,f_r)}$. Let $J$ be the Jacobian of $F=(f_1,\dots,f_r)$. Then $$\Omega_{C/A}^1\cong\text{coker}(J)$$
\end{block}

\end{frame}

\begin{frame}[fragile]
\frametitle{Computing some examples}

\begin{examples} Let $A$ be a ring and $B=A[x_1,\dots,x_n]$ so that $B$ is an $A$-algebra. Then $$\Omega_{B/A}^1\cong\bigoplus_{i=1}^nBd(x_i)$$ In particular, the module $\Omega_{B/A}^1$ is a finitely generated $B$-module. 
\end{examples}

\begin{examples} Let $V=\mathbb{V}(y^2-x^3)\subseteq\A_\C^2$ be the vanishing locus of the cuspidal cubic. \\

Then $$\Omega_{\C[V]/\C}^1\cong\frac{\C[V]dx\oplus\C[V]dy}{((-3x^2)dx\oplus(2y)dy)}$$
\end{examples}

\end{frame}

\begin{frame}[fragile]
\frametitle{Computing some examples}

\begin{examples} Let $W=\mathbb{V}(4x^2+9y^2-36)\subseteq\A_\C^2$ be the vanishing locus of an ellipse. \\
Then $$\Omega_{(\C[W])/\C}^1\cong\frac{\C[W]dx\oplus\C[W]dy}{((8x)dx\oplus(18y)dy)}$$ 
\end{examples}

\begin{examples} Let $U=\mathbb{V}(x^2+y^2-z^2)\subset\A_\C^3$ be the vanishing locus of the double cone. \\
Then $$\Omega_{\C[U]/\C}^1\cong\frac{\C[U]dx\oplus\C[U]dy\oplus\C[U]dz}{(2xdx\oplus 2ydy\oplus -2zdz)}$$
\end{examples}

\end{frame}

\begin{frame}[fragile]
\frametitle{Cotangent space from the module}

\begin{block}{Recovering the Cotangent Space} Let $(B,m)$ be a local ring which contains a field $K$ that is isomorphic to $B/m$ the residue field. Then the second exact sequence induces a vector space isomorphism $$\frac{m}{m^2}\cong\Omega_{B/K}^1\otimes_B K$$
\end{block}

In particular, if $(B,m)$ is the local ring of a variety at a point, the module is just the cotangent space, up to a change of base ring to the residue field. 

\end{frame}

\begin{frame}[fragile]
\frametitle{Computing some examples}

Recall that $$\Omega_{\C[V]/\C}^1\cong\frac{\C[V]dx\oplus\C[V]dy}{(-3x^2dx,2ydy)}$$

\begin{block}{}
When $p=(p_1,p_2)\neq(0,0)$, $x,y\notin m_p$ and so are in invertible in the localization. Thus within this localization, we can write the in the quotient as $dy=\frac{3x^2}{2y}dx$. And so we are left with $$\left(\Omega_{\C[V]/\C}^1\right)_{m_p}\cong\C[V]_{m_p}dx$$ Clearly this is a free $\C[V]_{m_p}$-module of rank $1$. Then $$\frac{m_p}{m_p^2}\cong\left(\Omega_{\C[V]_{m_p}/\C}^1\right)\otimes_{\C[V]_{m_p}}\frac{\C[V]_{m_p}}{m_p}\cong\C dx$$
which shows that $\frac{m_p}{m_p^2}$ is a $1$-dimensional vector space over $\C$. 
\end{block}

\end{frame}

\begin{frame}[fragile]
\frametitle{Computing some examples}

\begin{block}{}
Consider the point $(0,0)$. There is a surjection $\left(\Omega_{\C[V]_{(x,y)}/\C}^1\right)\to\frac{\C[V]_{(x,y)}}{(x,y)}dx\oplus\frac{\C[V]_{(x,y)}}{(x,y)}dy$ with kernel precisely $$\left(\Omega_{\C[V]_{(x,y)}/\C}^1\right)x\oplus\left(\Omega_{\C[V]_{(x,y)}/\C}^1\right)y$$ 

Then $$\frac{m_{(0,0)}}{m_{(0,0)}^2}\cong\left(\Omega_{\C[V]_{(x,y)}/\C}^1\right)\otimes_{\C[V]_{(x,y)}}\frac{\C[V]_{(x,y)}}{(x,y)}\cong\C dx\oplus\C dy$$ which shows that $\frac{m_{(0,0)}}{m_{(0,0)}^2}$ is a vector space of dimension $2$ over $\C$. \\
\end{block}

\end{frame}

\begin{frame}[fragile]
\frametitle{Two notions of differentials}

From a differential geometry perspective, we may ask whether $\Omega_{\mC^\infty(M)/\R}^1$ and $\Omega^1(M)$ are the same thing. 

\begin{block}{The Two Modules Are Not Isomorphic In General} Consider $\R$ as a smooth manifold. Then $\Omega^1(\R)$ is not isomorphic to $\Omega_{C^\infty(\R)/\R}^1$. In particular, for $f(x)=x$ and $g(x)=e^x$, $d(e^x)=e^xd(x)$ in $\Omega^1(\R)$ but $d(e^x)$ and $d(x)$ are linearly independent in $\Omega_{C^\infty(\R)/\R}^1$. 
\end{block}

The Leibniz rule and linearity of $d$ can only be applied finitely many times. For $e^x=\sum_{k=0}^\infty\frac{x^k}{k!}$ there is no reason for its derivative to be the same as its term by term derivative. 

\end{frame}


\end{document}