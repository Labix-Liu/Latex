\documentclass[a4paper]{article}

\input{C:/Users/liula/Desktop/Latex/Headers V1.2.tex}

\pagestyle{fancy}
\fancyhf{}
\rhead{Labix}
\lhead{T2W2 Supervision 1}
\rfoot{\thepage}

\title{T2W2 Supervision 1}

\author{Labix}

\date{\today}
\begin{document}
\section*{Exercise Sheet for Week 2}
\subsection*{Question 1}
Find the radius of convergence of the following power series: 
\begin{itemize}
\item $\sum_{k=0}^\infty\frac{x^k}{k!}$
\item $\sum_{k=0}^\infty\frac{x^k}{r^k}$ for some $r\in\R$. 
\end{itemize}

\subsection*{Question 2}
Consider the power series $\sum_{k=0}^\infty a_kx^k$ for $a_k\in\R$. Prove (it is hard) that the radius of convergence of the power series is given by $$r=\lim_{n\to\infty}\left(\inf\left\{\abs{x_m}^{-1/m}:m\geq n\right\}\right)$$ Check that this formula works for all the power series you have seen. \\

Upshot: A hack that you are not allowed to use in assignments or exams unless they taught you this in the notes (you may use it for sanity check). 

\subsection*{Question 3}
Let $\sum_{k=0}^\infty a_kx^k$ be a power series. Let $L=\lim_{n\to\infty}\abs{\frac{a_{k+1}}{a_k}}$. Suppose that $L$ is finite and non-zero. Show that if $t\in\R$ is such that $\abs{t}<\frac{1}{L}$, then the infinite sum $\sum_{k=0}^\infty a_kt^k$ converges. \\

Use the above to show that the power series $\sum_{k=0}^\infty kx^k$ has radius of convergence $R=1$. 

\subsection*{Question 4}
Consider the equation $y^2=x^3+x+10$ defined for the region $y\geq 0$. Find a parameterization of the curve. Now parameterize it for the case when $y\leq 0$. Piece the two to find a parameterization of the entire curve $y^2=x^3+x+10$, making sure that it is continuous. \\

Upshot: Not all curves are paramterized by a single formula. 

\subsection*{Question 5}
Find two different parametrizations of the standard parabola $y=x^2$. Can you write an expression relating the two parametrizations?\\

Upshot: First question says that parametrization is not unique. The second question is the same as asking for a reparametrization from one curve to another. 

\subsection*{Question 6}
Find a parametrization for the intersection of the two planes whose equations are given by $x^2+y^2=4$ and $z=xy$. 

\subsection*{Question 7}
7(a): Consider the following three vectors of $\R^3$: $\begin{pmatrix}1 \\ 0 \\ 1\end{pmatrix},\begin{pmatrix}1 \\ 1 \\ 0\end{pmatrix},\begin{pmatrix}0 \\ 1 \\ 1\end{pmatrix}$. Prove that no vector is a linear combination of the other two. We call these vectors linearly independent from each other. \\

7(b): Now consider the same set of vectors. Consider the set of all possible linear combinations of the three vectors: $$S=\left\{a\begin{pmatrix}1 \\ 0 \\ 1\end{pmatrix}+b\begin{pmatrix}1 \\ 1 \\ 0\end{pmatrix}+c\begin{pmatrix}0 \\ 1 \\ 1\end{pmatrix}:a,b,c\in\R\right\}$$ Show that $S=\R^3$. We say that these vectors span $\R^3$. \\

7(c): Now choose your favourite vector out of the original three vectors, and take it away. Show that the remaining two vectors no longer span $\R^3$. 

\end{document}