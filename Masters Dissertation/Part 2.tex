\section{Homotopy Pushouts and Homotopy Pullbacks}
\subsection{Models for Homotopy Pushout and Homotopy Pullback}
Pushouts and pullbacks are not well suited for studying homotopy invariant properties. Most of the invariants in algebraic topology detect differences up to homotopy equivalence, but pushouts and pullbacks are more rigid than homotopy equivalence: 

\begin{itemize}
\item Their universal properties guarantee that they are unique up to homeomorphism. 
\item Pushouts and pullbacks are determined by three spaces and two maps. But if we supply homotopy equivalent spaces then the pushout / pullback is not homotopy equivalence. 
\end{itemize}

The following example is given in \cite{PHC}. 

\begin{eg}\label{Ex1} Consider the following commutative diagram:   
 \\~\\ \adjustbox{scale=0.8,center}{\begin{tikzcd}
	\ast & {S^n} & {D^{n+1}} \\
	\ast & {S^n} & \ast
	\arrow[from=1-1, to=2-1]
	\arrow[from=1-2, to=1-1]
	\arrow[hook, from=1-2, to=1-3]
	\arrow["{\text{id}}"', from=1-2, to=2-2]
	\arrow["\ast", from=1-3, to=2-3]
	\arrow[from=2-2, to=2-1]
	\arrow[from=2-2, to=2-3]
\end{tikzcd}} \\~\\
While all the vertical arrows are weak equivalences, the induced map of pushouts is given by $S^{n+1}\to\ast$ which is clearly not a weak equivalence. 
\end{eg}

Therefore we concern ourselves with a homotopy invariant version of this concept. The ordinary pushouts and pullbacks are unique up to homeomorphism. We can explicitly define a set and its topology and show that it satisfies a universal property. We take a similar approach here and first introduce a model for homotopy pushouts. Our main reference for this section is \cite{CHT}. 

\begin{defn}\label{defn:Hpull} Let $W,X,Y\in\bold{Top}$ be spaces. Let $f:W\to X$ and $g:W\to Y$ be maps. Define the standard homotopy pushout of $f$ and $g$ to be the quotient space $$\text{hocolim}(X\overset{f}{\leftarrow}W\overset{g}{\rightarrow}Y)=\frac{X\amalg(W\times I)\amalg Y}{\sim}$$ where $\sim$ is the equivalence relation generated by $f(w)\sim (w,0)$ and $g(w)\sim(w,1)$ for $w\in W$. 
\end{defn}

\begin{defn}\label{defn:Hpush} Let $X,Y,Z\in\bold{Top}$ be spaces. Let $f:X\to Z$ and $g:Y\to Z$ be maps. Define the standard homotopy pullback of $f$ and $g$ to be the subspace $$\text{holim}(X\overset{f}{\rightarrow}Z\overset{g}{\leftarrow}Y)=\{(x,\alpha,y)\in X\times\text{Map}(I,Z)\times Y\;|\;\alpha(0)=f(x),\alpha(1)=g(y)\}$$
\end{defn}

We will use the following two ways to recgonize homotopy pushouts and homotopy pullbacks as certain limits and colimits. 

\begin{prp}\label{prp:Replace1} Let $X,Y,Z\in\bold{Top}$ be spaces. Let $f:X\to Z$ and $g:Y\to Z$ be maps. Then the there is a homeomorphism $$\lim(X\overset{f}{\rightarrow}Z\leftarrow P_g)\cong\text{holim}(X\overset{f}{\rightarrow}Z\overset{g}{\leftarrow}Y)$$ given by the map $(x,(y,\gamma))\mapsto(x,\gamma,y)$. 
\begin{proof}
Recall that the limit on the left is given by $\{(x,(y,\gamma))\in X\times P_g\;|\;\gamma(1)=f(x)\}$. Also $(y,\gamma)\in P_g$ means that $\gamma(0)=g(y)$. So the map defined by $(x,(y,\gamma))\mapsto(x,\gamma,y)$ is a well defined map and is clearly a homeomorphism. \cite{CHT}
\end{proof}
\end{prp}

Notice that we could have equally replaced the space $X$ with $P_f$ to obtain a homeomorphism between the limit and homotopy pullback entirely by symmetry.  

Dually we have a homeomorphism $$\text{colim}(X\overset{f}{\leftarrow}Z\rightarrow M_g)\cong\text{hocolim}(X\overset{f}{\leftarrow}Z\overset{g}{\rightarrow}Y)$$ Moreover, since the diagram of spaces is symmetric, we could have equally replaced the space on the left by the mapping path space / mapping cylinder and we would yet again obtain a homeomorphism. 

\begin{prp}\label{prp:Replace2} Let $X,Y,Z\in\bold{Top}$ be spaces. Let $f:X\to Z$ and $g:Y\to Z$ be maps. Then the there is a homeomorphism $$\lim(P_f\rightarrow Z\leftarrow P_g)\cong\text{holim}(X\overset{f}{\rightarrow}Z\overset{g}{\leftarrow}Y)$$ given by the map $((x,\gamma_1),(y,\gamma_2))\mapsto(x,\gamma_1\cdot\overline{\gamma_2},y)$. 
\begin{proof}
Notice that the map is well defined because by definition, we have that $$\lim(P_f\rightarrow Z\leftarrow P_g)=\{((x,\gamma_1),(y,\gamma_2))\in P_f\times P_g\;|\;\gamma_1(0)=f(x),\gamma_2(0)=g(y),\gamma_1(1)=\gamma_2(1)\}$$ The evident inverse of the given map is then $(x,\gamma,y)\mapsto((x,\gamma|_{[0,1/2]}),(y,\gamma_{[1/2,1]}))$. \cite{CHT}
\end{proof}
\end{prp}

Dually we have a homeomorphism $$\text{colim}(M_f\leftarrow W\rightarrow M_g)\cong\text{hocolim}(X\overset{f}{\leftarrow}W\overset{g}{\rightarrow}Y)$$ 

\begin{prp}\label{prp:FibRecog} Let $X,Y,Z$ be spaces. Let $f:X\to Z$ and $g:Y\to Z$ be maps. If one of $f$ or $g$ is a fibration, then the canonical map $$\lim(X\rightarrow Z\leftarrow Y)\simeq\text{holim}(X\rightarrow Z\leftarrow Y)$$ is a homotopy equivalence. 
\begin{proof}
Without loss of generality suppose that $f$ is a fibration. Replace $g$ by a homotopy equivalence $h:Y\to P_g$ and a fibration $k:P_g\to Z$. We then obtain a commutative diagram:  
 \\~\\ \adjustbox{scale=0.8,center}{\begin{tikzcd}
	{\text{lim}(X\rightarrow Z\leftarrow Y)} & Y \\
	{\text{holim}(X\rightarrow Z\leftarrow Y)} & {P_g} \\
	X & Z
	\arrow[from=1-1, to=1-2]
	\arrow[from=1-1, to=2-1]
	\arrow["h", from=1-2, to=2-2]
	\arrow[from=2-1, to=2-2]
	\arrow[from=2-1, to=3-1]
	\arrow["k", from=2-2, to=3-2]
	\arrow["f"', from=3-1, to=3-2]
\end{tikzcd}} \\~\\
It is commutative because the map from the limit to $X$ is precisely the projection map to $X$. Since the outer rectangle and the bottom square are pullbacks, the pasting law \ref{prp:Pasting1} shows that the upper square is also a pullback. Now since $f$ is a fibration and the bottom square is a pullback, we conclude that the middle horizontal arrow is a fibration by \ref{prp:PullPFib}. Since $h$ is a homotopy equivalence, we conclude that the map from the limit to the homotopy limit is a homotopy equivalence by \ref{prp:PullFibPWeak}. Hence we conclude. 
\end{proof}
\end{prp}

We can compute some examples of homotopy pushouts and pullbacks. 

\begin{eg}\label{eg:Hofib} Let $f:X\to Y$ be a map. For any $y\in Y$ recall that the homotopy fiber of $f$ is defined to be the space $$\text{hofib}_y(f)=\{(x,\gamma)\in X\times\text{Map}(I,Y)\;|\;\gamma(0)=f(x),\gamma(1)=y\}$$ This is precisely the definition of $\text{holim}(X\overset{f}{\rightarrow}Y\leftarrow\{y\})$. Also note that if we take actual limits instead of homotopy limits then the same diagram would give us the fiber of $f$, thus justifying its name. By replacing the space $X$ with $P_f$, we a homeomorphism $$\text{hofib}_{y}(X\rightarrow Y\leftarrow\{y\})\cong\lim(P_f\rightarrow Y\leftarrow\{y\})=\text{fib}_{y}(P_f\rightarrow Y)$$ 

Similarly, the homotopy cofiber of $f$ is defined to be $$\text{hocofib}(f)=\frac{(X\times I)\amalg Y}{\sim}$$ where the relation is generated by $(x,1)\sim(x',1)$ and $(x,0)\sim f(x)$ for $x,x'\in X$. This is precisely the definition of $\text{hocolim}(\ast\leftarrow X\overset{f}{\rightarrow}Y)$. 
\end{eg}

\begin{eg}\label{eg:SusLoop} Let $X$ be a pointed space. There is a unique map $X\to\ast$ to the terminal object in $\bold{Top}$. The homotopy pushout of $\ast\leftarrow X\rightarrow\ast$ is given by $$\text{hocolim}(\ast\leftarrow X\rightarrow\ast)=\frac{\ast\amalg(X\times I)\amalg\ast}{\sim}$$ where $\sim$ is generated by $\ast\sim(x,0)$ and $\ast\sim(x,1)$. This is precisely the definition of $\Sigma X$. 

Similarly, there is a unique map $\ast\to X$ sending $\ast$ to the base point of $X$. The homotopy pullback of $\ast\rightarrow X\leftarrow\ast$ is given by 
\begin{align*}
\text{holim}(\ast\rightarrow X\leftarrow\ast)&=\{(\ast,\gamma,\ast)\in\ast\times\text{Map}(I,X)\times\ast\;|\;\gamma(0)=\ast=\gamma(1)\}\\
&\cong\{\gamma\in\text{Map}(I,X)\;|\;\gamma\text{ is a loop at the base point }\}
\end{align*}
This is precisely the definition of $\Omega X$. 
\end{eg}

\begin{eg}\label{eg:Identity} Let $f:X\to Y$ be a map. The homotopy pullback of $f$ and the identity map on $Y$ is given by 
\begin{align*}
\text{holim}(X\overset{f}{\rightarrow}Y\overset{\text{id}}{\leftarrow}Y)&=\{(x,\gamma,y)\in X\times\text{Map}(I,Y)\times Y\;|\;\gamma(0)=x,\gamma(1)=y\}\\
&\cong\{(x,\gamma)\in X\times\text{Map}(I,Y)\;|\;\gamma(0)=x\}\\
&=P_f
\end{align*}
In particular, the homotopy pullback is homotopy equivalent to $X$ by \ref{prp:Factorization}

Similarly, the homotopy pushout of $f$ and the identity map on $X$ is given by \begin{align*}
\text{hocolim}(Y\overset{f}{\leftarrow}X\overset{\text{id}}{\rightarrow}X)&=\frac{Y\amalg(X\times I)\amalg X}{f(x)\sim(x,0), x\sim(x,1)}\\
&\cong\frac{Y\amalg(X\times I)}{f(x)\sim(x,0)}\\
&=M_f
\end{align*}
In particular, the homotopy pushout is homotopy equivalent to $Y$. 
\end{eg}

\begin{eg}\label{eg:Wedge&Prod} Let $X,Y$ be pointed spaces. The homotopy pushout of the initial maps $\ast\to X$ and $\ast\to Y$ that maps the unique object to the base point is given by $$\text{hocolim}(X\leftarrow\ast\rightarrow Y)\cong\frac{X\amalg I\amalg Y}{x_0\sim 0,y_0\sim 1}$$ which is just the spaces $X$ and $Y$ together with the unit interval connecting the base points. Since $I$ deformation retracts to a point, we see that the homotopy colimit is homotopy equivalent to $X\vee Y$. Another way to see this is that because the two initial maps are cofibration, the homotopy pushout is homotopy equivalent to the ordinary limit.  

Dually, the homotopy pullback of the final maps $X\to\ast$ and $Y\to\ast$ is given by 
\begin{align*}
\text{holim}(X\rightarrow\ast\leftarrow Y)&=\{(x,\gamma,y)\in X\times\text{Map}(I,\ast)\times Y\;|\;\gamma(0)=\ast=\gamma(1)\}\\
&\cong X\times\text{Map}(I,\ast)\times Y\cong X\times Y
\end{align*}
\end{eg}

\begin{eg}\label{eg:Smash} Let $X,Y$ be spaces. There is an evident inclusion map $X\vee Y\to X\times Y$. We can ask for its homotopy pushout with the terminal map $X\vee Y\to\ast$. But $X\vee Y\to X\times Y$ is a closed inclusion and so is a cofibration (Refer to section 2.4 of \cite{MC}). Therefore the homotopy pushout is precisely the cofiber of this map, which is the quotient $X\wedge Y=\frac{X\times Y}{X\vee Y}$. 
\end{eg}

\begin{prp}\label{prp:3x3} Suppose that the following is a commutative diagram of spaces:  
 \\~\\ \adjustbox{scale=0.8,center}{\begin{tikzcd}
	{X_1} & {X_3} & {X_2} \\
	{Z_1} & {Z_3} & {Z_2} \\
	{Y_1} & {Y_3} & {Y_2}
	\arrow["{h_1}", from=1-1, to=1-2]
	\arrow["{f_1}"', from=1-1, to=2-1]
	\arrow["{f_3}"', from=1-2, to=2-2]
	\arrow["{h_2}"', from=1-3, to=1-2]
	\arrow["{f_2}", from=1-3, to=2-3]
	\arrow["{l_1}", from=2-1, to=2-2]
	\arrow["{l_2}"', from=2-3, to=2-2]
	\arrow["{g_1}", from=3-1, to=2-1]
	\arrow["{k_1}"', from=3-1, to=3-2]
	\arrow["{g_3}", from=3-2, to=2-2]
	\arrow["{g_2}"', from=3-3, to=2-3]
	\arrow["{k_2}", from=3-3, to=3-2]
\end{tikzcd}} \\~\\
Let $R_i$ be the homotopy pullback of the $i$th row, let $C_i$ be the homotopy pullback of the $i$th column. Then there is a homeomorphism $$\text{holim}(R_1\rightarrow R_3\leftarrow R_2)\cong\text{holim}(C_1\rightarrow C_3\leftarrow C_2)$$ 
\begin{proof}
Using the definition of the homotopy pullback we have $$\text{holim}(C_1\rightarrow C_3\leftarrow C_2)=\left\{((x_1,\alpha_1,y_1),\Gamma,(x_2,\alpha_2,y_2))\in C_1\times\text{Map}(I,C_3)\times C_2\;\bigg{|}\;\substack{\Gamma(0)=(f_1(x_1),l_1\circ\alpha_1,g_1(y_1))\\\Gamma(1)=(f_2(x_2),l_2\circ\alpha_2,g_2(y_2))}\right\}$$ Notice that the data of $\Gamma$ is equivalent to a homotopy $\Gamma:I\times I\to Z_3$, from $\Gamma(0,-):l_1\circ\alpha_1$ to $\Gamma(1,-):l_2\circ\alpha_2$. Similarly, we have $$\text{holim}(R_1\rightarrow R_3\leftarrow R_2)=\left\{((x_1,\beta_X,x_2),\Gamma,(y_1,\beta_Y,y_2))\in R_1\times\text{Map}(I,R_3)\times R_2\;\bigg{|}\;\substack{\Gamma(0)=(h_1(x_1),f_3\circ\beta_X,h_2(x_2))\\\Gamma(1)=(k_1(y_1),g_3\circ\beta_Y,k_2(y_2))}\right\}$$ and again the data of $\Gamma$ is equivalent to a homotopy $I\times I\to Z_3$, from $f_3\circ\beta_X$ to $g_3\circ\beta_Y$. Then the map $\text{holim}(C_1\rightarrow C_3\leftarrow C_2)\to\text{holim}(R_1\rightarrow R_3\leftarrow R_2)$ defined by $$((x_1,\alpha_1,y_1),\Gamma,(x_2,\alpha_2,y_2))\mapsto((x_1,\Gamma(-,0),x_2),\Gamma,(y_1,\Gamma(-,1),y_2))$$ gives the desired homeomorphism. 
\end{proof}
\end{prp}

The dual of the above is also true. Namely, if we have a commutative diagram of spaces:  
 \\~\\ \adjustbox{scale=0.8,center}{\begin{tikzcd}
	{X_1} & {X_3} & {X_2} \\
	{Z_1} & {Z_3} & {Z_2} \\
	{Y_1} & {Y_3} & {Y_3}
	\arrow[from=1-2, to=1-1]
	\arrow[from=1-2, to=1-3]
	\arrow[from=2-1, to=1-1]
	\arrow[from=2-1, to=3-1]
	\arrow[from=2-2, to=1-2]
	\arrow[from=2-2, to=2-1]
	\arrow[from=2-2, to=2-3]
	\arrow[from=2-2, to=3-2]
	\arrow[from=2-3, to=1-3]
	\arrow[from=2-3, to=3-3]
	\arrow[from=3-2, to=3-1]
	\arrow[from=3-2, to=3-3]
\end{tikzcd}} \\~\\
Then the homotopy pushout of the homotopy pushout of the rows is homeomorphic to the homotopy pushout of the homotopy pushout of the columns. Using this we can show the following: 

\begin{eg}\label{prp:WedgeCommHPull} Let $X,Y,Z,W$ be pointed spaces. Let $f:X\to Y$ and $g:X\to Z$ be maps. Then there is a homeomorphism $$W\wedge\text{hocolim}(Y\leftarrow X\rightarrow Z)\cong\text{holim}(W\wedge Y\leftarrow W\wedge X\rightarrow W\wedge Z)$$ To prove this, notice that there is a homeomorphism $$\text{holim}(W\times Y\leftarrow W\times X\rightarrow W\times Z)\cong W\times\text{hocolim}(Y\leftarrow X\rightarrow Z)$$ given by the identity map. 

We can also show that wedge sums commute with homotopy pushouts. To see this, consider the following diagram: \\~\\
\adjustbox{scale=1.0,center}{\begin{tikzcd}
	W & W & W \\
	\ast & \ast & \ast \\
	Y & X & Z
	\arrow[from=1-2, to=1-1]
	\arrow[from=1-2, to=1-3]
	\arrow[from=2-1, to=1-1]
	\arrow[from=2-1, to=3-1]
	\arrow[from=2-2, to=1-2]
	\arrow[from=2-2, to=2-1]
	\arrow[from=2-2, to=2-3]
	\arrow[from=2-2, to=3-2]
	\arrow[from=2-3, to=1-3]
	\arrow[from=2-3, to=3-3]
	\arrow[from=3-2, to=3-1]
	\arrow[from=3-2, to=3-3]
\end{tikzcd}}\\~\\
For each column, its homotopy pushout is precisely the wedge of $W$ and the corresponding space because the initial map $\ast\to W$ is a cofibration. Therefore the homotopy pushout of the homotopy pushout is the homotopy pushout of wedge sums. On the other hand, the homotopy pushout of the first row is $W$, the homotopy pushout of the second row is $\ast$ hence the homotopy pushout of the homotopy pushout of the rows is given by $W\vee\text{hocolim}(Y\leftarrow X\rightarrow Z)$. 

Then consider the commutative diagram: \\~\\
\adjustbox{scale=1.0,center}{\begin{tikzcd}
	{W\times Y} & {W\times X} & {W\times Z} \\
	{W\vee Y} & {W\vee X} & {W\vee Z} \\
	\ast & \ast & \ast
	\arrow[from=1-2, to=1-1]
	\arrow[from=1-2, to=1-3]
	\arrow[from=2-1, to=1-1]
	\arrow[from=2-1, to=3-1]
	\arrow[from=2-2, to=1-2]
	\arrow[from=2-2, to=2-1]
	\arrow[from=2-2, to=2-3]
	\arrow[from=2-2, to=3-2]
	\arrow[from=2-3, to=1-3]
	\arrow[from=2-3, to=3-3]
	\arrow[from=3-2, to=3-1]
	\arrow[from=3-2, to=3-3]
\end{tikzcd}}\\~\\
where the top vertical arrows are inclusions. The homotopy pushout of each column is precisely the corresponding smash product by \ref{eg:Smash}. The homotopy pushout of the homotopy pushout of the rows is given by $$\text{hocolim}(W\times\text{hocolim}(Y\leftarrow X\rightarrow Z)\leftarrow W\vee\text{hocolim}(Y\leftarrow X\rightarrow Z)\rightarrow\ast)$$ Therefore we have homeomorphisms 
\begin{align*}
W\wedge\text{hocolim}(Y\leftarrow X\rightarrow Z)&\cong\text{hocolim}(W\times\text{hocolim}(Y\leftarrow X\rightarrow Z)\leftarrow W\vee\text{hocolim}(Y\leftarrow X\rightarrow Z)\rightarrow\ast)\\
&\cong\text{hocolim}(W\wedge Y\leftarrow W\wedge X\rightarrow W\wedge Z)
\end{align*}
\end{eg}

By taking $W=S^1$, we also deduce that suspension commutes with taking standard homotopy pushouts. 

Recall that ordinary pullbacks are not well-suited for studying homotopy invariant concepts. The following theorem shows that homotopy pullbacks remedies the situation. 

\begin{prp}\label{prp:HpullPHomotopy} Suppose that we have a commutative diagram of spaces  
 \\~\\ \adjustbox{scale=0.8,center}{\begin{tikzcd}
	X & Z & Y \\
	{X'} & {Z'} & {Y'}
	\arrow["f", from=1-1, to=1-2]
	\arrow["{e_X}"', from=1-1, to=2-1]
	\arrow["{e_Z}"', from=1-2, to=2-2]
	\arrow["g"', from=1-3, to=1-2]
	\arrow["{e_Y}", from=1-3, to=2-3]
	\arrow["{f'}"', from=2-1, to=2-2]
	\arrow["{g'}", from=2-3, to=2-2]
\end{tikzcd}} \\~\\
in $\bold{Top}$. Define the map $$\phi_{X,Z,Y}^{X',Z',Y'}:\text{holim}(X\overset{f}{\rightarrow}Z\overset{g}{\leftarrow}Y)\to\text{holim}(X'\overset{f'}{\rightarrow}Z'\overset{g'}{\leftarrow}Y')$$ by the formula $(x,\gamma,y)\mapsto(e_X(x),e_Z\circ\gamma,e_Y(y))$. If each $e_X,e_Y,e_Z$ are homotopy equivalences, then $\phi$ is a homotopy equivalence. 
\begin{proof}
Consider the following commutative diagram:  
 \\~\\ \adjustbox{scale=0.8,center}{\begin{tikzcd}
	X & Z & Y \\
	X & {Z'} & Y \\
	{X'} & {Z'} & {Y'}
	\arrow["f", from=1-1, to=1-2]
	\arrow["{\text{id}_X}"', from=1-1, to=2-1]
	\arrow["{e_Z}", from=1-2, to=2-2]
	\arrow["g"', from=1-3, to=1-2]
	\arrow["{\text{id}_Y}", from=1-3, to=2-3]
	\arrow["{e_Z\circ f}", from=2-1, to=2-2]
	\arrow["{e_X}"', from=2-1, to=3-1]
	\arrow["{\text{id}_{Z'}}", from=2-2, to=3-2]
	\arrow["{e_Z\circ g}"', from=2-3, to=2-2]
	\arrow["{e_Y}", from=2-3, to=3-3]
	\arrow["{f'}"', from=3-1, to=3-2]
	\arrow["{g'}", from=3-3, to=3-2]
\end{tikzcd}} \\~\\
We prove that the homotopy pullback of the first row is homotopy equivalent to that of the second, and we prove that the homotopy pullback of the second row is homotopy equivalent to that of the third.  

Since $e_Z$ is a homotopy equivalence, we can find a homotopy inverse $k$ for $e_Z$ and a homotopy $H:Z\times I\to Z$ such that $H(-,0)=\text{id}_Z$ and $H(-,1)=k\circ e_Z$. Define a map $$\rho:\text{holim}(X\overset{f}{\rightarrow}Z'\overset{g}{\leftarrow}Y)\to\text{holim}(X\overset{e_Z\circ f}{\rightarrow}Z\overset{e_Z\circ g}{\leftarrow}Y)$$ by the formula $$(x,\gamma',y)\mapsto(x,H(f(x),-)\ast k(\gamma'(-))\ast\overline{H(g(y),-)}:I\to Z,y)$$ where $\ast$ denotes concatenation of paths. The path concatenation is well defined because we have that $H(f(x),1)=(k\circ e_Z\circ f)(x)=(k\circ\gamma')(0)$ and $k(\gamma'(1))=k(e_Z(g(y)))=H(g(y),1)$. This is well defined on the homotopy pullback because we have that 
\begin{itemize}
\item $H(f(x),-)\ast k(\gamma'(-))\ast\overline{H(g(y),-)}(0)=H(f(x),0)=\text{id}_Z(f(x))=f(x)$
\item $H(f(x),-)\ast k(\gamma'(-))\ast\overline{H(g(y),-)}(1)=H(g(y),0)=\text{id}_Z(g(y))=g(y)$
\end{itemize}
I claim that this map is the homotopy inverse to the map $\phi=\phi_{X,Y,Z}^{X,Y,Z'}$. We have that 
\begin{align*}
\rho(\phi(x,\gamma,y))&=\rho(x,e_Z\circ\gamma,y)\\
&=(x,H(f(x),-)\ast k(e_Z(\gamma(-))\ast\overline{H(g(y),-)},y)
\end{align*}
Now I claim that the middle path is homotopic to $\gamma$. For the first component of the concatenation, the path $H(f(x),t):I\to Z$ can be contracted to $H(f(x),0)=f(x)=\gamma(0)$ so you can homotope the traversal along $H(f(x),-)$ to the single point $f(x)=\gamma(0)$. For the third component of the concatenation, this is similar so we can homotope the traversal of $\overline{H(g(y),-)}$ to the single point $g(y)=\gamma(1)$. The middle part of the path is homotopic to $\gamma$ because $k\circ e_Z$ is homotopic to $\text{id}_Z$. Thus we conclude. 
\end{proof}
\end{prp}

We note that the above is still true if we replace homotopy equivalences by weak equivalences. Then the induced map between standard homotopy pushouts is a weak equivalence. This is given in \cite{CHT}. 

\subsection{Homotopy Pushout and Pullback Squares}
Recall that we want our new pushout to be a homotopy invariant, not a homeomorphic invariant. Any homeomorphic spaces has one unique way of writing it down set theoretically up to bijection of the underlying set and the topology, but homotopy equivalent spaces are not determined by its set theoretic representation. 

\begin{defn}\label{defn:Hcomm} Let $W,X,Y,Z\in\bold{Top}$ be spaces such that the following is a homotopy commutative diagram:  
 \\~\\ \adjustbox{scale=0.8,center}{\begin{tikzcd}
	W & Y \\
	X & Z
	\arrow["f", from=1-1, to=1-2]
	\arrow["g", from=1-1, to=2-1]
	\arrow["h", from=1-2, to=2-2]
	\arrow["k", from=2-1, to=2-2]
\end{tikzcd}} \\~\\
We say that the square is homotopy commutative if there exists homotopy from $h\circ f$ to $k\circ g$. 
\end{defn}

Since homotopy is an equivalent relation on maps we see that the existence of a homotopy from $k\circ g$ to $h\circ f$ also gives us a homotopy commutative square. 

\begin{eg}\label{eg:HpullIsHcomm} Consider the following square:  
 \\~\\ \adjustbox{scale=0.8,center}{\begin{tikzcd}
	\text{holim}(X\rightarrow Z\leftarrow Y) & Y \\
	X & Z
	\arrow["f", from=1-1, to=1-2]
	\arrow["g", from=1-1, to=2-1]
	\arrow["h", from=1-2, to=2-2]
	\arrow["k", from=2-1, to=2-2]
\end{tikzcd}} \\~\\
where the arrows from the homotopy pullback is given by projections. The diagram is not commutative because going through the two possible compositions only give points in $Z$ that are connected by a path. However, it is homotopy commutative if we consider the homotopy $H:\text{holim}(X\rightarrow Z\leftarrow Y)\times I\to Z$ defined by $$H((x,\gamma,y),t)=\gamma(t)$$ The situation is dual with the standard homotopy pushout. 
\end{eg}

In particular, a commutative square is also homotopy commutative since we can take the constant homotopy between the two maps. 

\begin{defn}\label{defn:HpullSq} Let $W,X,Y,Z\in\bold{Top}$ be spaces such that there is a homotopy commutative diagram  
 \\~\\ \adjustbox{scale=0.8,center}{\begin{tikzcd}
	W & Y \\
	X & Z
	\arrow["g", from=1-1, to=1-2]
	\arrow["f", from=1-1, to=2-1]
	\arrow["k", from=1-2, to=2-2]
	\arrow["h", from=2-1, to=2-2]
\end{tikzcd}} \\~\\
whose homotopy from $h\circ f$ to $k\circ g$ is witnessed by $H:W\times I\to Z$. Define the map $\alpha:\text{hocolim}(X\leftarrow W\rightarrow Y)\to Z$ by the formula $$\alpha(u)=\begin{cases}
h(u) & \text{ if }u\in X\\
k(u) & \text{ if }u\in Y\\
H(w,t) & \text{ if }u=(w,t)\in W\times I
\end{cases}$$ (Notice that it is well defined since $H(w,0)=h(f(w))$ and $H(W,1)=k(g(y))$). 
\begin{itemize}
\item We say that the square is a homotopy pushout square if $\alpha$ is a weak equivalence. 
\item We say that the diagram is $k$-cocartesian if $\alpha$ is $k$-connected. 
\end{itemize}
\end{defn}

The idea is similar to standard pushouts in the sense that if we have a commutative square  
 \\~\\ \adjustbox{scale=0.8,center}{\begin{tikzcd}
	W & Y \\
	X & Z
	\arrow[from=1-1, to=1-2]
	\arrow[from=1-1, to=2-1]
	\arrow[from=1-2, to=2-2]
	\arrow[from=2-1, to=2-2]
\end{tikzcd}} \\~\\
then by the universal property of pushouts one would obtain a comparison map $W\to\lim(X\rightarrow Z\leftarrow Y)$. When this is a homeomorphism we call $W$ a pushout of the diagram. Similarly we define homotopy pullback squares dually. 

\begin{defn}\label{defn:HpushSq} Let $W,X,Y,Z\in\bold{Top}$ be spaces such that there is a homotopy commutative diagram  
 \\~\\ \adjustbox{scale=0.8,center}{\begin{tikzcd}
	W & Y \\
	X & Z
	\arrow["g", from=1-1, to=1-2]
	\arrow["f", from=1-1, to=2-1]
	\arrow["k", from=1-2, to=2-2]
	\arrow["h", from=2-1, to=2-2]
\end{tikzcd}} \\~\\
whose homotopy from $h\circ f$ to $k\circ g$ is witnessed by $H:W\times I\to Z$. Let $\beta:W\to\text{holim}(X\rightarrow Z\leftarrow Y)$ be the map defined by $$\beta(w)=(f(w),H(w,-),g(y))$$ (notice that the map is well defined since $H(w,0)=h(f(w))$ and $H(w,1)=k(g(y))$). 
\begin{itemize}
\item We say that the diagram is a homotopy pullback if the $\beta$ is a weak equivalence. 
\item We say that the diagram is $k$-cartesian if $\beta$ is $k$-connected. 
\end{itemize}
\end{defn}

In the two definitions, I considered homotopy commutative square to be homotopy pushouts and homotopy pullbacks by considering the existence of a weak equivalence. One may ask to enforce the map to be a homotopy equivalence instead. While \cite{CHT} only requires the map is a weak equivalence, \cite{IHT} and \cite{MCHT} require them to be homotopy equivalences. Because every homotopy equivalence is a weak equivalence, using merely weak equivalences allows for a wider class of spaces to represent the same homotopy pushout or pullback. However the biggest drawback is that weak equivalences are not invertible in general, unless in specific examples such as CW complexes, in which case there is no difference between specifying the map to be a weak equivalence or a homotopy equivalence. 

Suppose the following is a commutative square in $\bold{Top}$:  
 \\~\\ \adjustbox{scale=0.8,center}{\begin{tikzcd}
	W & Y \\
	X & Z
	\arrow[from=1-1, to=1-2]
	\arrow[from=1-1, to=2-1]
	\arrow[from=1-2, to=2-2]
	\arrow[from=2-1, to=2-2]
\end{tikzcd}} \\~\\
Notice that the map $\alpha$ factors through the limit because we are taking the homotopy recording this homotopy commutative square to be the constant homotopy. The same is true for the comparison map for homotopy pullbacks. 

\cite{CHT} only compares homotopy pushouts / pullbacks with commutative squares and so defines a square to homotopy pushout / pullback if the comparison map is a weak equivalence. However the notion is quite limiting since the simplest examples fail to be a homotopy pushout square. For instance, the square  
 \\~\\ \adjustbox{scale=0.8,center}{\begin{tikzcd}
	W & Y \\
	X & {\text{hocolim}(X\leftarrow W\rightarrow Y)}
	\arrow[from=1-1, to=1-2]
	\arrow[from=1-1, to=2-1]
	\arrow[from=1-2, to=2-2]
	\arrow[from=2-1, to=2-2]
\end{tikzcd}} \\~\\
is not even an honest commutative square. 

\begin{prp}\label{prp:FibRecogSq} Let the following be a pullback square:  
 \\~\\ \adjustbox{scale=0.8,center}{\begin{tikzcd}
	W & Y \\
	X & Z
	\arrow[from=1-1, to=1-2]
	\arrow[from=1-1, to=2-1]
	\arrow[from=1-2, to=2-2]
	\arrow[from=2-1, to=2-2]
\end{tikzcd}} \\~\\
If one of $X\to Z$ or $Y\to Z$ is a fibration, then the square is a homotopy pullback square. 
\begin{proof}
By \ref{prp:FibRecog}, if one of $X\to Z$ or $Y\to Z$ is a fibration, then the canonical map from the limit to the homotopy limit is a homotopy equivalence. Hence we obtain a weak equivalence $W\cong\lim(X\rightarrow Z\leftarrow Y)\to\text{holim}(X\rightarrow Z\leftarrow Y)$. 
\end{proof}
\end{prp}

The following proposition shows that recognizing homotopy pullback squares can be done through other factorizations of maps into weak equivalences and fibrations. The theorem can be found in section 13.3 of \cite{MCL}. The following proof is original. 

\begin{prp}\label{prp:AnyFac} Let $W,X,Y,Z\in\bold{Top}$ be spaces such that there is a homotopy commutative diagram  
 \\~\\ \adjustbox{scale=0.8,center}{\begin{tikzcd}
	W & Y \\
	X & Z
	\arrow[from=1-1, to=1-2]
	\arrow[from=1-1, to=2-1]
	\arrow["g", from=1-2, to=2-2]
	\arrow["f", from=2-1, to=2-2]
\end{tikzcd}} \\~\\
Then the square is a homotopy pullback square if and only if for any factorization of $Y\to Z$ into a weak equivalence $Y\to M$ and a fibration $M\to Z$, the induced map $$W\to\lim(X\rightarrow Z\leftarrow M)$$ is a weak equivalence. 
\begin{proof}
Further factorize $Y\to M$ into a weak equivalence $Y\to E$ and fibration $E\to M$ so that we have a acyclic cofibration $Y\to E$ and a fibration $E\to Z$. This is possible by section 2 of \cite{TBER}, which in turn cites \cite{NC}. Then notice that we have the following commutative diagram:  
 \\~\\ \adjustbox{scale=0.8,center}{\begin{tikzcd}
	Y & {P_g} \\
	E & Z
	\arrow[from=1-1, to=1-2]
	\arrow[from=1-1, to=2-1]
	\arrow[from=1-2, to=2-2]
	\arrow[from=2-1, to=2-2]
\end{tikzcd}} \\~\\
because they are both factorizations of $Y\to Z$, the diagram is commutative. Since $Y\to E$ is an acyclic cofibration and $P_g\to Z$ is a fibration, there exists a lift $E\to P_g$ making the above diagram commute by \ref{prp:Lift}. Since $Y\to P_g$ and $Y\to E$ are weak equivalences, $E\to P_g$ is also a weak equivalence by the two out of three property in \ref{prp:2o3}. Consider the following diagram: 
 \\~\\ \adjustbox{scale=0.8,center}{\begin{tikzcd}
	W && Y \\
	& {\lim(P_f\rightarrow Z\leftarrow E)} & E \\
	& {\lim(P_f\rightarrow Z\leftarrow P)} & {P_g} \\
	X & {P_f} & Z
	\arrow[from=1-1, to=1-3]
	\arrow[from=1-1, to=2-2]
	\arrow[bend right=10, from=1-1, to=3-2]
	\arrow[from=1-1, to=4-1]
	\arrow[from=1-3, to=2-3]
	\arrow[from=2-2, to=2-3]
	\arrow[from=2-2, to=3-2]
	\arrow[from=2-3, to=3-3]
	\arrow[from=3-2, to=3-3]
	\arrow[from=3-2, to=4-2]
	\arrow[from=3-3, to=4-3]
	\arrow[from=4-1, to=4-2]
	\arrow[from=4-2, to=4-3]
\end{tikzcd}} \\~\\
where the maps from $W$ to the limits are induced by the universal property of pullbacks, and so the triangle with $W$ and the two limits is a commutative diagram. By \ref{prp:WeakLims}, $\lim(P_f\rightarrow Z\leftarrow E)\to\lim(P_f\rightarrow Z\leftarrow P_g)$ is a weak equivalence. By the two out of three property \ref{prp:2o3}, $W\to\lim(P_f\to Z\leftarrow P_g)\cong\text{holim}(X\rightarrow Z\leftarrow Y)$ is a weak equivalence if and only if $W\to\lim(P_f\to Z\leftarrow E)$ is a weak equivalence.  

Finally, notice that since $Y\to E\to M$ is a factorization of the map $Y\to M$, we know that $E\to M$ is a weak equivalence by the two out of three property \ref{prp:2o3}. By \ref{prp:WeakLims} we know that $\lim(P_f\rightarrow Z\leftarrow E)\to\lim(P_f\rightarrow Z\leftarrow M)$ is a weak equivalence. Therefore by the two out of three property \ref{prp:2o3}, $W\to\lim(P_f\rightarrow Z\leftarrow E)$ is a weak equivalence if and only if $W\to\lim(P_f\rightarrow Z\leftarrow M)$ is a weak equivalence. This shows that $W\to\text{holim}(X\rightarrow Z\leftarrow Y)$ is a weak equivalence if and only if $W\to\lim(P_f\rightarrow Z\leftarrow M)$ is a weak equivalence, proving our claim. 
\end{proof}
\end{prp}

The following proposition is a generalization of the one given in \cite{CHT} since it deals with homotopy commutative squares instead of strictly commutative squares. The proof is inspired by that in section 13.3 of \cite{MCL}

\begin{prp}\label{prp:Pasting2} Consider the following homotopy commutative square:  
 \\~\\ \adjustbox{scale=0.8,center}{\begin{tikzcd}
	{X_1} & {X_2} & {X_3} \\
	{Y_1} & {Y_2} & {Y_3}
	\arrow[from=1-1, to=1-2]
	\arrow[from=1-1, to=2-1]
	\arrow[from=1-2, to=1-3]
	\arrow[from=1-2, to=2-2]
	\arrow[from=1-3, to=2-3]
	\arrow[from=2-1, to=2-2]
	\arrow[from=2-2, to=2-3]
\end{tikzcd}} \\~\\
in $\bold{Top}$. Let the right square be a homotopy pullback square. Then the left square is a homotopy pullback if and only if the rectangle is a homotopy pullback square. 
\begin{proof}
Factorize $X_3\to Y_3$ into a weak equivalence $X_3\to P$ and a fibration $P\to Y_3$ where $P$ is the mapping path space of the map $X_3\to Y_3$. We then obtain the following commutative diagram:  
 \\~\\ \adjustbox{scale=0.8,center}{\begin{tikzcd}
	{X_1} & {X_2} & {X_3} \\
	{\lim(Y_1\rightarrow Y_3\leftarrow P)} & {\lim(Y_2\rightarrow Y_3\leftarrow P)} & P \\
	{Y_1} & {Y_2} & {Y_3}
	\arrow[from=1-1, to=2-1]
	\arrow[from=1-2, to=2-2]
	\arrow["\simeq", from=1-3, to=2-3]
	\arrow[from=2-1, to=2-2]
	\arrow[from=2-1, to=3-1]
	\arrow[from=2-2, to=2-3]
	\arrow[from=2-2, to=3-2]
	\arrow["{\text{fib}}", from=2-3, to=3-3]
	\arrow[from=3-1, to=3-2]
	\arrow[from=3-2, to=3-3]
\end{tikzcd}} \\~\\
where the two vertical maps on the top left is induced by the canonical map of the homotopy pullback square. We are given that $X_2\to\lim(Y_2\rightarrow Y_3\leftarrow P)$ is a weak equivalence. By the pasting law of pullback squares \ref{prp:Pasting1}, we know that the bottom left square is a pullback square. Since $P\to Y_3$ is a fibration and the bottom right square is a pullback square, the middle lower vertical map is a fibration by \ref{prp:PullPFib}. 

Notice that the map $X_2\to Y_2$ is factored into a weak equivalence $X_2\to\lim(Y_2\rightarrow Y_3\leftarrow P)$ followed by a fibration $\lim(Y_2\rightarrow Y_3\leftarrow P)\to Y_2$. Moreover by the pasting law, the bottom left square is a pullback. Therefore the left square in the original diagram is a homotopy pullback square if and only if $X_1\to\lim(Y_1\rightarrow Y_3\leftarrow P)$ is a weak equivalence by \ref{prp:AnyFac}. But also the same map is a weak equivalence if and only if the outer rectangle is a homotopy pullback square by definition. Hence we conclude.
\end{proof}
\end{prp}

The dual version for homotopy pushouts is also true. Namely, if the following is homotopy commutative: 
 \\~\\ \adjustbox{scale=0.8,center}{\begin{tikzcd}
	{X_1} & {X_2} & {X_3} \\
	{Y_1} & {Y_2} & {Y_3}
	\arrow[from=1-1, to=1-2]
	\arrow[from=1-1, to=2-1]
	\arrow[from=1-2, to=1-3]
	\arrow[from=1-2, to=2-2]
	\arrow[from=1-3, to=2-3]
	\arrow[from=2-1, to=2-2]
	\arrow[from=2-2, to=2-3]
\end{tikzcd}} \\~\\
Assume that the left square is a homotopy pushout square, then the right square is a homotopy pushout square if and only if the left square is a homotopy pushout square. A proof in the case where the diagram is strictly commutative can be found in proposition 3.7.26 of \cite{CHT}. In the proof of \ref{thm:BMT} we will use the strictly commutative version. 

\begin{prp}\label{prp:HpullSqPWeak} Let $W,X,Y,Z\in\bold{Top}$ be spaces such that there is a homotopy commutative diagram  
 \\~\\ \adjustbox{scale=0.8,center}{\begin{tikzcd}
	W & Y \\
	X & Z
	\arrow[from=1-1, to=1-2]
	\arrow[from=1-1, to=2-1]
	\arrow[from=1-2, to=2-2]
	\arrow["f", from=2-1, to=2-2]
\end{tikzcd}} \\~\\
Then the following are true. 
\begin{itemize}
\item If the square is a homotopy pullback, and $Y\to Z$ is a weak equivalence, then $W\to X$ is a weak equivalence. 
\item If $Y\to Z$ and $W\to X$ are weak equivalence, then the square is a homotopy pullback. 
\end{itemize} 
\begin{proof}
The important square to consider is the following pullback square:  
 \\~\\ \adjustbox{scale=0.8,center}{\begin{tikzcd}
	\text{holim}(X\rightarrow Z\leftarrow Y) & Y \\
	P_f & Z
	\arrow[from=1-1, to=1-2]
	\arrow[from=1-1, to=2-1]
	\arrow[from=1-2, to=2-2]
	\arrow[from=2-1, to=2-2]
\end{tikzcd}} \\~\\
Since $Y\to Z$ is a weak equivalence and $P_f\to Z$ is a fibration, \ref{prp:PullFibPWeak} implies that $\text{holim}(X\rightarrow Z\leftarrow Y)\to P_f$ is a weak equivalence. Since this map is given by the composition $\text{holim}(X\rightarrow Z\leftarrow Y)\to X\to P_f$ and $X\to P_f$ is a weak equivalence, the two out of three property \ref{prp:2o3} gives that $\text{holim}(X\rightarrow Z\leftarrow Y)\to X$ is a weak equivalence. Since we have a weak equivalence $W\to\text{holim}(X\rightarrow Z\leftarrow Y)$, we again apply the two out of three property \ref{prp:2o3} so that the composition $W\to X$ is a weak equivalence.  

Conversely, if $Y\to Z$ and $W\to X$ are weak equivalence, then the same reasoning shows that $\text{holim}(X\rightarrow Z\leftarrow Y)\to P_f$ is a weak equivalence and hence $\text{holim}(X\rightarrow Z\leftarrow Y)\to X$ is a weak equivalence by two out of three \ref{prp:2o3}. Since $W\to X$ is also a weak equivalence, by two out of three \ref{prp:2o3} we conclude that $W\to\text{holim}(X\rightarrow Z\leftarrow Y)$ is a weak equivalence and so the square in question is a homotopy pullback.
\end{proof}
\end{prp}

The statement dualizers into a version involving homotopy pushout squares: Namely if one has a square: 
 \\~\\ \adjustbox{scale=0.8,center}{\begin{tikzcd}
	W & Y \\
	X & Z
	\arrow[from=1-1, to=1-2]
	\arrow[from=1-1, to=2-1]
	\arrow[from=1-2, to=2-2]
	\arrow["f", from=2-1, to=2-2]
\end{tikzcd}} \\~\\
and $W\to X$ is a weak equivalence, then the square is a homotopy pushout if and only if $Y\to Z$ is a weak equivalence. The proof of this can be found in \cite{CHT}. 

\begin{prp}\label{prp:Hpull&Hfib} Let $W,X,Y,Z\in\bold{Top}$ be spaces such that following is a homotopy commutative square  
 \\~\\ \adjustbox{scale=0.8,center}{\begin{tikzcd}
	W & Y \\
	X & Z
	\arrow[from=1-1, to=1-2]
	\arrow["h", from=1-1, to=2-1]
	\arrow["g", from=1-2, to=2-2]
	\arrow["f", from=2-1, to=2-2]
\end{tikzcd}} \\~\\
The square is a homotopy pullback if and only if for all $x\in X$, the map $$\text{hofib}_x(h)\to\text{hofib}_{f(x)}(g)$$ is a weak equivalence. 
\begin{proof}
Consider the following homotopy commutative diagram. 
 \\~\\ \adjustbox{scale=0.8,center}{\begin{tikzcd}
	{\text{hofib}_{\gamma(f(x))}(\alpha)} & {\text{hofib}_{x}(h)} & W \\
	\ast & {\text{hofib}_{f(x)}(g)} & {\text{holim}(X\rightarrow Z\leftarrow Y)} & Y \\
	& \ast & X & Z
	\arrow[from=1-1, to=1-2]
	\arrow[from=1-1, to=2-1]
	\arrow[from=1-2, to=1-3]
	\arrow["\beta", from=1-2, to=2-2]
	\arrow["\alpha", from=1-3, to=2-3]
	\arrow[from=2-1, to=2-2]
	\arrow["\gamma", from=2-2, to=2-3]
	\arrow[from=2-2, to=3-2]
	\arrow[from=2-3, to=2-4]
	\arrow[from=2-3, to=3-3]
	\arrow["g", from=2-4, to=3-4]
	\arrow[from=3-2, to=3-3]
	\arrow[from=3-3, to=3-4]
\end{tikzcd}} \\~\\
Notice that by definition, the map $W\to\text{holim}(X\rightarrow Z\leftarrow Y)\to X$ is precisely the map $h$. By repeated application of \ref{prp:Pasting2} we conclude that all four squares above are homtopy pullback squares. 

If $\alpha$ is a weak equivalence, then since the top wide rectangle is a homotopy pullback we have $\text{hofib}_{\gamma(f(x))}(\alpha)$ is weakly equivalent to $\ast$. But the top left square is a homotopy pullback hence $\text{holim}_x(h)$ is weakly equivalent to $\text{holim}_{f(x)}(g)$. Conversely, suppose that $\beta$ is a weak equivalence for all $x$. Since the top left square is a homotopy pullback, this implies that $\text{hofib}_{\gamma(f(x))}(\alpha)$ is weakly contractible for all $x$. In particular it is $n$-connected for all $n$. Then this implies that $\alpha$ is $(n+1)$-connected for all $n$. Hence $\alpha$ is a weak equivalence. 
\end{proof}
\end{prp}

The above proposition is also true if we replace weak equivalence with $k$-cartesian and can be found in \cite{CHT}. 

\begin{prp}\label{prp:CofibRepHpush} Let $W,X,Y,Z\in\bold{Top}$ be spaces such that there is a homotopy pushout square  
 \\~\\ \adjustbox{scale=0.8,center}{\begin{tikzcd}
	W & Y \\
	X & Z
	\arrow[from=1-1, to=1-2]
	\arrow[from=1-1, to=2-1]
	\arrow[from=1-2, to=2-2]
	\arrow[from=2-1, to=2-2]
\end{tikzcd}} \\~\\
Then there exists a pushout square  
 \\~\\ \adjustbox{scale=0.8,center}{\begin{tikzcd}
	W' & Y' \\
	X' & Z'
	\arrow[from=1-1, to=1-2]
	\arrow[from=1-1, to=2-1]
	\arrow[from=1-2, to=2-2]
	\arrow[from=2-1, to=2-2]
\end{tikzcd}} \\~\\
such that $W'\to X'$, $W'\to Y'$, $\text{colim}(X'\leftarrow W'\rightarrow Y')\to Z$ are cofibrations, and a map of squares  
 \\~\\ \adjustbox{scale=0.8,center}{\begin{tikzcd}
	{W'} && {Y'} \\
	& W && Y \\
	{X'} && {Z'} \\
	& X && Z
	\arrow[from=1-1, to=1-3]
	\arrow[from=1-1, to=2-2]
	\arrow[from=1-1, to=3-1]
	\arrow[from=1-3, to=2-4]
	\arrow[from=1-3, to=3-3]
	\arrow[from=2-2, to=2-4]
	\arrow[from=2-2, to=4-2]
	\arrow[from=2-4, to=4-4]
	\arrow[from=3-1, to=3-3]
	\arrow[from=3-1, to=4-2]
	\arrow[from=3-3, to=4-4]
	\arrow[from=4-2, to=4-4]
\end{tikzcd}} \\~\\
where the maps $W'\to W$, $X'\to X$, $Y'\to Y$, $Z'\to Z$ are homotopy equivalences. 
\begin{proof}
Take $W'=W$, $Y'$ the mapping cylinder of $W\to Y$ and $X'$ the mapping cylinder of $W\to X$. Take $Z'$ to be the mapping cylinder of $$\text{colim}(M_{W\to X}\leftarrow W\rightarrow M_{W\to Y})\cong\text{holim}(X\leftarrow W\rightarrow Y)\to Z$$ It is clear that $W'\to X'$ and $W'\to Y'$ are cofibrations. Moreover, $\text{colim}(X'\leftarrow W'\rightarrow Y')\to Z'$ is also a cofibration is a same reasoning. 
\end{proof}
\end{prp}

\begin{prp}\label{prp:LESHpull} Let $W,X,Y,Z\in\bold{Top}$ be spaces such that there is a commutative homotopy pullback square  
 \\~\\ \adjustbox{scale=0.8,center}{\begin{tikzcd}
	W & Y \\
	X & Z
	\arrow[from=1-1, to=1-2]
	\arrow[from=1-1, to=2-1]
	\arrow[from=1-2, to=2-2]
	\arrow[from=2-1, to=2-2]
\end{tikzcd}} \\~\\
Then the homotopy fiber of the induced map $W\to X\times Y$ is weakly equivalent to $\Omega Z$. Moreover, the fiber sequence as in \ref{thm:FiberSeq} induces a long exact sequence of homotopy groups:
\\~\\ \adjustbox{scale=0.8,center}{\begin{tikzcd}
	\cdots & {\pi_{n+1}(Z)} & {\pi_n(W)} & {\pi_n(X)\times\pi_n(Y)} & \cdots & {\pi_0(X)\times\pi_0(Y)}
	\arrow[from=1-1, to=1-2]
	\arrow[from=1-2, to=1-3]
	\arrow[from=1-3, to=1-4]
	\arrow[from=1-4, to=1-5]
	\arrow[from=1-5, to=1-6]
\end{tikzcd}} 
\begin{proof}
Consider the following commutative diagram:
\\~\\ \adjustbox{scale=0.8,center}{\begin{tikzcd}
	X & Z & Y \\
	X & \ast & Y \\
	\ast & \ast & \ast
	\arrow[from=1-1, to=1-2]
	\arrow["{\text{id}}"', from=1-1, to=2-1]
	\arrow[from=1-2, to=2-2]
	\arrow[from=1-3, to=1-2]
	\arrow["{\text{id}}", from=1-3, to=2-3]
	\arrow[from=2-1, to=2-2]
	\arrow[from=2-3, to=2-2]
	\arrow[from=3-1, to=2-1]
	\arrow[from=3-1, to=3-2]
	\arrow[from=3-2, to=2-2]
	\arrow[from=3-3, to=2-3]
	\arrow[from=3-3, to=3-2]
\end{tikzcd}} \\~\\
Using \ref{eg:Hofib} and \ref{eg:Wedge&Prod}, we see that the homotopy limit of the homotopy limit of the rows is precisely the homotopy fiber of the map $\text{holim}(X\rightarrow Z\leftarrow Y)\to X\times Y$. The homotopy limit of the homotopy limit of the columns is homotopy equivalent to $\text{holim}(\ast\rightarrow Z\leftarrow\ast)\cong\Omega Z$ using \ref{eg:Identity} and \ref{eg:SusLoop}. Now the maps $W\to\text{holim}(X\rightarrow Z\leftarrow Y)\to X,Y$ is precisely the maps $W\to X,Y$ in the original diagram by definition. Therefore we obtain the following commutative diagram: 
\\~\\ \adjustbox{scale=0.8,center}{\begin{tikzcd}
	W & {\text{holim}(X\rightarrow Z\leftarrow Y)} \\
	{X\times Y} & {X\times Y}
	\arrow[from=1-1, to=1-2]
	\arrow[from=1-1, to=2-1]
	\arrow[from=1-2, to=2-2]
	\arrow["{\text{id}}"', from=2-1, to=2-2]
\end{tikzcd}} \\~\\
Since the top horizontal arrow is a weak equivalence, we conclude that this square is a homotopy pullback square by \ref{prp:HpullSqPWeak}. Then there is a weak equivalence $$\text{hofib}_{(x,y)}(W\to X\times Y)\to\text{hofib}_{(x,y)}(\text{holim}(X\rightarrow Z\leftarrow Y)\to X\times Y)$$ by \ref{prp:Hpull&Hfib}. Since the latter space is homotopy equivalent to $\Omega Z$, the long exact sequence in \ref{thm:FiberSeq} completes the proof. 
\end{proof}
\end{prp}

\section{Excisive Functors}
\subsection{The Failure of the Identity Functor to be Excisive}
The following definition is given in \cite{CHT}. 

\begin{defn}\label{defn:Excisive} Let $F:\bold{Top}_\ast\to\bold{Top}_\ast$ be a functor. Define the following terminology: 
\begin{itemize}
\item We say that $F$ is reduced if $F(\ast)$ is weakly contractible. 
\item We say that $F$ is a homotopy functor if $f$ is a weak equivalence implies that $F(f)$ is a weak equivalence. 
\item We say that $F$ is finitary if the following are true. If $I$ is a filtered category and $X:I\to\bold{Top}_\ast$ is a diagram, then $$\underset{i\in I}{\text{hocolim}}F(X_i)\to F\left(\underset{i\in I}{\text{hocolim}}X_i\right)$$ is a weak equivalence. 
\item We say that $F$ is excisive if $F$ sends homotopy pushout squares to homotopy pullback squares. 
\end{itemize}
\end{defn}

The finitary requirement ensures that the functor $F$ is determined by its value on a finite complexes. Then by taking homotopy colimits the value of $F$ on spaces generated by colimits of finite complexes are determined. For an explicit model of homotopy colimits, we refer to chapter 8 of \cite{CHT}. 

It is more interesting to ask a functor to send homotopy puishouts to homotopy pullbacks. Indeed, for any space $X$ that can be decomposed into a union of two subspaces $A$ and $B$, the following square with inclusion maps is a homotopy pushout by 3.7.5 of \cite{CHT}. 
 \\~\\ \adjustbox{scale=0.8,center}{\begin{tikzcd}
	{A\cap B} & A \\
	B & X
	\arrow[from=1-1, to=1-2]
	\arrow[from=1-1, to=2-1]
	\arrow[from=1-2, to=2-2]
	\arrow[from=2-1, to=2-2]
\end{tikzcd}} \\~\\
It is not a homotopy pullback square in general. However, we have seen from \ref{prp:LESHpull} that homotopy pullbacks work well with the homotopy groups. Therefore given an excisive functor $F$, it becomes possible to compute the homotopy groups of $F(X)$ using the homotopy groups of $F(A)$ and $F(B)$. 

It is natural to ask what kinds of functors are excisive. Our first non-example is given by the Blakers-Massey theorem, which implies that the identity functor is not excisive. 

We will make use of all the properties of homotopy pushout and pullback squares we developed to prove the theorem. For segments of the proof that does not involve homotopy pushout and pullback squares, we defer the proof to the appendix. The remainder of the section, and the proof of the Blakers-Massey theorem, takes reference from \cite{CHT} and \cite{ATTD}. 

\begin{prp}\label{prp:Setup} Let $X$ be a space. Let $e^{d_i}\cong D^{d_i}$ be a cell of dimension $d_i$ for $i=1,2$. Then the following diagram  
 \\~\\ \adjustbox{scale=0.8,center}{\begin{tikzcd}
	X & X\cup e^{d_1} \\
	X\cup e^{d_2} & X\cup e^{d_1}\cup e^{d_2}
	\arrow[from=1-1, to=1-2]
	\arrow[from=1-1, to=2-1]
	\arrow[from=1-2, to=2-2]
	\arrow[from=2-1, to=2-2]
\end{tikzcd}} \\~\\
given by inclusion maps is $(d_1+d_2-3)$-cartesian. 
\begin{proof}
Let $p_1\in e^{d_1}$ and $p_2\in e^{d_2}$ be interior points. Since $X\cup e^{d_2}$ is weakly equivalent to $X\cup e^{d_1}\cup e^{d_2}\setminus\{p_1\}$ by inclusion (and similarly for $X\cup e^{d_1}$), the above square admits a weak equivalence to the following square:  
 \\~\\ \adjustbox{scale=0.8,center}{\begin{tikzcd}
	{X\cup e^{d_1}\cup e^{d_2}\setminus\{p_1,p_2\}} & {X\cup e^{d_1}\cup e^{d_2}\setminus\{p_2\}} \\
	{X\cup e^{d_1}\cup e^{d_2}\setminus\{p_1\}} & {X\cup e^{d_1}\cup e^{d_2}}
	\arrow[hook, from=1-1, to=1-2]
	\arrow[hook, from=1-1, to=2-1]
	\arrow[hook, from=1-2, to=2-2]
	\arrow[hook, from=2-1, to=2-2]
\end{tikzcd}} \\~\\
Let $Y=X\cup e^{d_1}\cup e^{d_2}$. By \ref{prp:Hpull&Hfib}, showing that the square is $(d_1+d_2-3)$-cartesian is the same as showing the map $$\text{hofib}_y(Y\setminus\{p_1,p_2\}\to Y\setminus\{p_1\})\to\text{hofib}_y(Y\setminus\{p_2\}\to Y)$$ is $(d_1+d_2-3)$-cartesian. Let $$C=\text{hofib}_y(Y\setminus\{p_2\}\to Y\setminus\{p_2\})\simeq\ast$$ 

Now the intersections below can be computed to give $$C\cap\text{hofib}_y(Y\setminus\{p_1,p_2\}\to Y\setminus\{p_1\})=\text{hofib}_y(Y\setminus\{p_1,p_2\}\to Y\setminus\{p_1,p_2\})$$ and $$C\cap\text{hofib}_y(Y\setminus\{p_2\}\to Y)=\text{hofib}_y(Y\setminus\{p_2\}\to Y\setminus\{p_2\})\simeq\ast$$ (because $C$ and the homotopy fibers are all subspaces of $Y\times\text{Map}(I,Y)$). By example 3.7.5 of \cite{CHT}, we conclude that the square  
 \\~\\ \adjustbox{scale=0.8,center}{\begin{tikzcd}
	{C\cap\text{hofib}_y(Y\setminus\{p_1,p_2\}\to Y\setminus\{p_1\})} & {\text{hofib}_y(Y\setminus\{p_1,p_2\}\to Y\setminus\{p_1\})} \\
	C & {C\cup\text{hofib}_y(Y\setminus\{p_1,p_2\}\to Y\setminus\{p_1\})}
	\arrow[hook, from=1-1, to=1-2]
	\arrow[hook, from=1-1, to=2-1]
	\arrow[hook, from=1-2, to=2-2]
	\arrow[hook, from=2-1, to=2-2]
\end{tikzcd}} \\~\\
is a homotopy pushout. Since the left two terms are contractible, the map between them is a weak equivalence. By the dual of \ref{prp:HpullSqPWeak} we conclude that the map on the right is a weak equivalence. Similarly, we can consider the diagram  
 \\~\\ \adjustbox{scale=0.8,center}{\begin{tikzcd}
	{C\cap\text{hofib}_y(Y\setminus\{p_2\}\to Y)} & {\text{hofib}_y(Y\setminus\{p_2\}\to Y)} \\
	C & {C\cup\text{hofib}_y(Y\setminus\{p_2\}\to Y)}
	\arrow[hook, from=1-1, to=1-2]
	\arrow[hook, from=1-1, to=2-1]
	\arrow[hook, from=1-2, to=2-2]
	\arrow[hook, from=2-1, to=2-2]
\end{tikzcd}} \\~\\
which is a homotopy pullback. And the fact that the terms on the left are contractible imply that the map on the right (which is in fact equal) $$\text{hofib}_y(Y\setminus\{p_2\}\to Y)\overset{=}{\hookrightarrow}\text{hofib}_y(Y\setminus\{p_2\}\to Y)$$ is a weak equivalence by the dual of \ref{prp:HpullSqPWeak}. We now have a chain of maps  
 \\~\\ \adjustbox{scale=0.8,center}{\begin{tikzcd}
	{\text{hofib}_y(Y\setminus\{p_1,p_2\}\to Y\setminus\{p_1\})} & {C\cup\text{hofib}_y(Y\setminus\{p_1,p_2\}\to Y\setminus\{p_1\})} \\
	\\
	{C\cup \text{hofib}_y(Y\setminus\{p_2\}\to Y)} & {\text{hofib}_y(Y\setminus\{p_2\}\to Y)}
	\arrow["{\text{weak eq.}}", hook, from=1-1, to=1-2]
	\arrow[hook, from=1-2, to=3-1]
	\arrow["{\text{weak eq.}}", from=3-1, to=3-2]
\end{tikzcd}} \\~\\
where the middle map is inclusion. It remains to show that the middle map is $(d_1+d_2-3)$-connected. This is the same as showing that the pair $$(C\cup \text{hofib}_y(Y\setminus\{p_2\}\to Y),C\cup\text{hofib}_y(Y\setminus\{p_1,p_2\}\to Y\setminus\{p_1\}))$$ is $(d_1+d_2-3)$-connected. The remainder of the proof here is deferred to the appendix at \ref{ss:BMT} since it is no longer an argument of homotopy pushout and pullback squares, and instead is an argument of constructing homotopies. 
\end{proof}
\end{prp}

The above proposition completes the following proof of Blakers-Massey theorem. 

\begin{thm}\label{thm:BMT} Let $X_0,X_1,X_2,X_{12}\in\bold{Top}$ be spaces such that the square  
 \\~\\ \adjustbox{scale=0.8,center}{\begin{tikzcd}
	X_0 & X_1 \\
	X_2 & X_{12}
	\arrow[from=1-1, to=1-2]
	\arrow[from=1-1, to=2-1]
	\arrow[from=1-2, to=2-2]
	\arrow[from=2-1, to=2-2]
\end{tikzcd}} \\~\\
is a homotopy pushout. Suppose the map $X_0\to X_i$ is $k_i$-connected for $i=1,2$. Then the diagram is $(k_1+k_2-1)$-cartesian. Explicitly, this means that $$\alpha:X_0\to\text{holim}(X_1\rightarrow X_{12}\leftarrow X_2)$$ is $(k_1+k_2-1)$-connected. 
\begin{proof}
By \ref{prp:CofibRepHpush} we only need to consider squares of the form  
 \\~\\ \adjustbox{scale=0.8,center}{\begin{tikzcd}
	{X_0} & {X_1} \\
	{X_2} & {X_{12}=X_1\amalg_{X_0}X_2}
	\arrow[from=1-1, to=1-2]
	\arrow[from=1-1, to=2-1]
	\arrow[from=1-2, to=2-2]
	\arrow[from=2-1, to=2-2]
\end{tikzcd}} \\~\\
By theorem 4.16 of \cite{AT}, if $(X,A)$ is $k$-connected CW complex, then there exists a CW pair $(Z,A)$ such that $Z\setminus A$ only has cells of dimension $\geq k$ that is weakly equivalent to $(X,A)$. The set up of the theorem now simplifies to the following: $X_1$ is the obtained from $X_0$ by gluing cells of dimension $\geq k_1$, and likewise for $X_2$. Now showing that the above diagram is $(k_1+k_2-1)$-cartesian is the same as showing that the map of homotopy fibers of the vertical maps is $(k_1+k_2-1)$-connected by proposition 3.3.18 of \cite{CHT}. This is the same as saying $(\text{holim}(X_1\to X_{12}),\text{holim}(X_0\to X_1))$ is $(k_1+k_2)$-connected by \ref{prp:PairConnectCon}, which is the same as saying any map from $(I^n,\partial I^n)$ to the pair of space is homotopic to a map mapping $I^n$ into $\text{holim}(X_0\to X_1)$ relative boundary. Such a map, by the hom product adjunction, is the same as giving a map $I^n\times I\to X_{12}$ for which $I^n\times\{1\}$ lies in $X_1$. But $X_1$ is a CW complex and $I^n\times\{1\}$ is compact, which means that image of the map is contained in finitely many cells, and so WLOG we can take $X_1,X_2$ to be the union of finitely many cells of appropriate dimension. 

Then by the dual of \ref{prp:Pasting2} (see remark below the theorem) and \ref{prp:Setup}, the proof is complete by induction. 
\end{proof}
\end{thm}

The homotopy excision theorem follows from the Blakers-Massey theorem in the following way. 

\begin{crl}\label{crl:HET} Let $X$ be a CW complex and $A,B$ two subcomplexes with non-empty intersection and $X=A\cup B$. If $(A,A\cap B)$ is $k_1$-connected and $(B,A\cap B)$ is $k_2$-connected, then the inclusion map $(A,A\cap B)\to(X,B)$ is $(k_1+k_2)$-connected. 
\begin{proof}
Consider the following square of inclusions:  
 \\~\\ \adjustbox{scale=0.8,center}{\begin{tikzcd}
	A\cap B & A \\
	B & X
	\arrow[from=1-1, to=1-2]
	\arrow[from=1-1, to=2-1]
	\arrow[from=1-2, to=2-2]
	\arrow[from=2-1, to=2-2]
\end{tikzcd}} \\~\\
We have seen that such a square diagram is a homotopy pushout diagram. 
By assumption, the inclusion maps $A\cap B\to A$ and $A\cap B\to B$ are $k_1$ and $k_2$ connected respectively. Blaker's-Massey theorem \ref{thm:BMT} implies that $$\text{hofib}(A\cap B\to A)\to\text{hofib}(B,X)$$ is $(k_1+k_2-1)$-connected. Since any map is $n$-connected if and only if its homotopy fiber is $(n-1)$-connected by \ref{prp:MapConnectCon}, we conclude that the inclusion map $(A,A\cap B)\to(X,B)$ is $(k_1+k_2)$-connected. 
\end{proof}
\end{crl}

We can interpret Blakers-Massey theorem as saying the identity functor is not excisive, because we think of weak equivalences as the limiting notion of $k$-connectedness. The fact that the diagram is only $(k_1+k_2-1)$-cartesian means that the comparison map is not quite a weak equivalence. 

In particular, the Freudenthal suspension theorem stated in 1.1.10 of \cite{FSHT}, which says that the map $$X\to\Omega\Sigma X$$ that is adjoint to $\text{id}:\Sigma X\to\Sigma X$ is $(2k+1)$-connected for any $k$-connected space $X$, follows also from Blakers-Massey theorem since it makes use of the homotopy excision theorem. 

\begin{crl}\label{crl:AdjointConn} Let $X,Y$ be spaces such that $X$ is $n$-connected. Let $f:X\to\Omega Y$ be a map. Let $\widetilde{f}:\Sigma X\to Y$ be its adjoint. If $f$ is a weak equivalence, then $\widetilde{f}$ is $(2n+3)$-connected. 
\begin{proof}
Since $X\to\Omega\Sigma X$ is the unit map, we obtain a commutative diagram:  
 \\~\\ \adjustbox{scale=0.8,center}{\begin{tikzcd}
	X & {\Omega\Sigma X} \\
	& {\Omega Y}
	\arrow[from=1-1, to=1-2]
	\arrow["f"', from=1-1, to=2-2]
	\arrow["{\Omega\widetilde{f}}", from=1-2, to=2-2]
\end{tikzcd}} \\~\\
Since the unit map is $(2n+1)$-connected, $\Omega\widetilde{f}$ must be $(2n+2)$-connected. Using the isomorphism $\pi_k(\Omega Z)\cong\pi_{k+1}(Z)$ for any space $Z$, we deduce that $\widetilde{f}$ itself must be at least $(2n+3)$-connected. Hence $\widetilde{f}$ is $(2n+3)$-connected. 
\end{proof}
\end{crl}

More: Connectivity of suspension and loop maps. 

\subsection{Excisive Approximations}
Even the simplest examples of functors fail to be excisive, as seen from the Blakers-Massey theorem. However, using the machinery of Goodwillie we can produce an excisive approximation of any given functor. Here we outline the basics and refer the proofs to \cite{CHT} and \cite{GW3}. 

\begin{defn}\label{defn:T1}Let $F$ be a homotopy functor. Let $X$ be a space.  Define $$T_1F(X)=\text{holim}(F(CX)\rightarrow F(\Sigma X)\leftarrow F(CX))$$
\end{defn}

It is clear that given a map $X\to Y$, this induces maps $CX\to CY$ and $\Sigma X\to\Sigma Y$. And the definition of the homotopy pullback tells us that $T_1F$ is a functor. Any natural transformation $F\Rightarrow G$ gives maps $F(CX)\to G(CX)$ and $F(\Sigma X)\to G(\Sigma X)$ so that $T_1$ itself is a functor that sends functors to functors. Finally, there is a natural transformation $t_1(F):F\Rightarrow T_1F$ since for each $X$ there is a natural map $F(X)\to\text{holim}(F(CX)\rightarrow F(\Sigma X)\leftarrow F(CX))$. 

\begin{defn}\label{defn:P1} Let $F$ be a homotopy functor. Let $X$ be a space. Define $$P_1F(X)=\text{hocolim}(F(X)\overset{t_1(F)(X)}{\rightarrow}T_1F(X)\overset{t_1(T_1F)(X)}{\rightarrow}T_1(T_1F)(X)\rightarrow\cdots)$$
\end{defn}

Since $t_1$ is natural transformation, we again obtain appropriate commutative diagrams so that $P_1F$ becomes a functor. 

\begin{eg}\label{eg:RedEx} If $F$ is reduced, then there is an easy way to describe the two functors $T_1F$ and $P_1F$. Beginning with $T_1F$, we know that $CX$ is contractible so $F(CX)$ is weakly equivalent to a point. Then $T_1F(X)$ is weakly equivalent to $\Omega F(\Sigma X)$. Iterating, we obtain $$T_1(T_1(F))(X)\simeq\Omega (T_1F)(\Sigma X)=\Omega(\Omega F(\Sigma(\Sigma X)))=\Omega^2 F(\Sigma^2 X)$$ as well as $$P_1F(X)=\underset{n\in\N}{\text{hocolim }}\Omega^n F(\Sigma^n X)$$
\end{eg}

The important thing to take away is that given a functor of spaces $F$, $P_1F$ is excisive. The proof can be found in \cite{GW3}. 

\begin{eg}\label{eg:ExcApproxId} As shown from the Blakers-Massey theorem \ref{thm:BMT}, $\text{id}:\bold{Top}_\ast\to\bold{Top}_\ast$ is not an excisive functor. However, using example \ref{eg:RedEx} we see that its excisive approximation is the functor $$X\mapsto\underset{n}{\text{hocolim}}\Omega^n\Sigma^n X$$
\end{eg}

The identity functor is interesting here precisely because it is not excisive, and hence we can obtain its excisive approximation. We will see later that excisive functors remedies the fact that $\pi_n$ does not satisfy excision for homology theories, and so precomposing $\pi_n$ with the approximation finally produces a reduced homology theory. And in the case of the identity functor, taking homotopy groups of its excisive approximation gives the stable homotopy groups. 

\section{Equivalence between Spectra and Reduced and Excisive Functors}
In the previous section, we have seen that spectra gives us a large class of excisive functors. In fact, this completely classifies all excisive functors up to weak equivalence. To show this, one can either deduce a Quillen equivalence between excisive functors and spectra by giving them both a suitable model category structure, or to appeal to the language of infinity categories. In this section we will assume the basic properties of infinity categories and prove the desired results. For the model category version we refer the interested reader to \cite{Biedermann_2007}. 

For background in infinity categories, we refer to \cite{HTT} and \cite{Kerodon}. Most of statements and definitions in this section are taken from \cite{HA}. 

\subsection{Stable Infinity Categories}
\begin{defn}\label{defn:SIC} Let $\mC$ be an infinity category. We say that $\mC$ is a stable infinity category if the following are true. 
\begin{itemize}
\item $\mC$ has a zero object $0$ (Such an infinity category is said to be pointed)
\item $\mC$ admits all finite limits and colimits. 
\item A square in $\mC$ of the form  
 \\~\\ \adjustbox{scale=0.8,center}{\begin{tikzcd}
	X & Y \\
	Z & W
	\arrow[from=1-1, to=1-2]
	\arrow[from=1-1, to=2-1]
	\arrow[from=1-2, to=2-2]
	\arrow[from=2-1, to=2-2]
\end{tikzcd}} \\~\\
is a pushout if and only if it is a pullback. 
\end{itemize}
\end{defn}

The definition given here is not the same as that of \cite{HA}, but \cite{HA} does show that the two definitions are equivalent. 

\begin{defn}\label{defn:ISus} Let $\mC$ be an infinity category with a zero object $0$. Define the functor $\Sigma:\mC\to\mC$ by sending $X\in\mC$ to the pushout of $0\rightarrow X\leftarrow 0$. 
\end{defn}

Write $M^\Sigma$ as the full sub-category of $\text{Func}(\Delta^1\times\Delta^1,\mC)$ spanned by pushout squares of the form  
 \\~\\ \adjustbox{scale=0.8,center}{\begin{tikzcd}
	X & 0 \\
	0 & {Y}
	\arrow[from=1-1, to=1-2]
	\arrow[from=1-1, to=2-1]
	\arrow[from=1-2, to=2-2]
	\arrow[from=2-1, to=2-2]
\end{tikzcd}} \\~\\
The suspension is a functor because we can also define it to be the composition of the section $\mC\to M^\Sigma$ with projection $M^\Sigma\to\mC$ to the bottom right object. The section exists since projection $M^\Sigma\to\mC$ to the first component is a trivial Kan fibration. See the paragraph before remark 1.1.2.6 of \cite{HA}

\begin{defn}\label{ILoop} Let $\mC$ be an infinity category with a zero object $0$. Define the functor $\Omega:\mC\to\mC$ by sending $X\in\mC$ to the pullback of $0\leftarrow X\rightarrow0$. 
\end{defn}

We can guarantee that looping is a functor in a similar way to $\Sigma$. 

\begin{prp}\label{prp:SusLoopEquiv} Let $\mC$ be a stable infinity category. Then $\Omega$ and $\Sigma$ are both equivalence of infinity categories. 
\begin{proof}
Let $\mC$ be stable. Let $X\in\mC$. Then  
 \\~\\ \adjustbox{scale=0.8,center}{\begin{tikzcd}
	X & 0 \\
	0 & {\Sigma X}
	\arrow[from=1-1, to=1-2]
	\arrow[from=1-1, to=2-1]
	\arrow[from=1-2, to=2-2]
	\arrow[from=2-1, to=2-2]
\end{tikzcd}} \\~\\
is a pushout in $\mC$. Since it is also a pullback in $\mC$, we conclude that $\Omega\Sigma X$ and $X$ are equivalent. Similarly, let $Y\in\mC$. Then  
 \\~\\ \adjustbox{scale=0.8,center}{\begin{tikzcd}
	{\Omega Y} & 0 \\
	0 & Y
	\arrow[from=1-1, to=1-2]
	\arrow[from=1-1, to=2-1]
	\arrow[from=1-2, to=2-2]
	\arrow[from=2-1, to=2-2]
\end{tikzcd}} \\~\\
is a pullback and a pushout in $\mC$. We can conclude that $\Sigma\Omega Y$ is weakly equivalent to $Y$. This shows that $\Sigma$ is a homotopy inverse of $\Omega$ and vice versa. 
\end{proof}
\end{prp}

\subsection{Excisive Functors between Infinity Categories}
\begin{defn}\label{defn:ICExcisive} Let $\mC,\mD$ be pointed infinity categories. Let $F:\mC\to\mD$ be a functor. 
\begin{itemize}
\item We say that $F$ is reduced if $F(\ast)$ is a final object of $\mD$, where $\ast$ is a final object of $\mC$. 
\item We say that $F$ is excisive if $F$ sends pushout squares to pullback squares. 
\end{itemize}
\end{defn}

Stable infinity categories are infinity categories in which pushouts and pullbacks coincide. In particular, this means that the identity functor is excisive. 

We want to give an equivalent characterization of stable infinity categories. Before that we need a technical lemma. The following proof is given in \cite{AHKT}. 

\begin{lmm}\label{lmm:RedExiffEquiv} Let $\mC,\mD$ be pointed infinity categories. Suppose that $\mC$ admits all finite colimits and $\mD$ admits all finite limits. Then the following are equivalent. 
\begin{itemize}
\item $F$ is reduced and excisive. 
\item $F$ is reduced and it satisfies the following. For all $X\in\mC$, the comparison map $$\eta_X:F(X)\to\Omega_\mD(F(\Sigma_\mC X))$$ is an equivalence in $\mD$, where $\eta_X$ is given as follows. For any $X\in\mC$, the diagram  
 \\~\\ \adjustbox{scale=0.8,center}{\begin{tikzcd}
	X & \ast \\
	\ast & {\Sigma_\mC X}
	\arrow[from=1-1, to=1-2]
	\arrow[from=1-1, to=2-1]
	\arrow[from=1-2, to=2-2]
	\arrow[from=2-1, to=2-2]
\end{tikzcd}} \\~\\
is a pushout in $\mC$. Sending the diagram through $F$ gives the wanted comparison map $\eta_X$ by the universal property of limits. 
\end{itemize} 
\begin{proof}
Let $F$ be reduced and excisive. Notice that following diagram on the left  
 \\~\\ \adjustbox{scale=0.8,center}{\begin{tikzcd}
	X & \ast && {F(X)} & \ast \\
	\ast & {\Sigma_\mC X} && \ast & {F(\Sigma_\mC X)}
	\arrow[from=1-1, to=1-2]
	\arrow[from=1-1, to=2-1]
	\arrow[""{name=0, anchor=center, inner sep=0}, from=1-2, to=2-2]
	\arrow[from=1-4, to=1-5]
	\arrow[""{name=1, anchor=center, inner sep=0}, from=1-4, to=2-4]
	\arrow[from=1-5, to=2-5]
	\arrow[from=2-1, to=2-2]
	\arrow[from=2-4, to=2-5]
	\arrow["F", shorten <=14pt, shorten >=14pt, Rightarrow, from=0, to=1]
\end{tikzcd}} \\~\\
is a pushout diagram in $\mC$. Applying $F$ gives a pullback diagram on the right. On the other hand, we know that  
 \\~\\ \adjustbox{scale=0.8,center}{\begin{tikzcd}
	{\Omega_\mD F(\Sigma_\mC X)} & \ast \\
	\ast & {F(\Sigma_\mC X)}
	\arrow[from=1-1, to=1-2]
	\arrow[from=1-1, to=2-1]
	\arrow[from=1-2, to=2-2]
	\arrow[from=2-1, to=2-2]
\end{tikzcd}} \\~\\
is a pullback diagram. Therefore the comparison map $F(X)\to\Omega_\mD F(\Sigma_\mC X)$ is an equivalence.  

Now suppose that $F$ satisfies the second conditions. Let  
 \\~\\ \adjustbox{scale=0.8,center}{\begin{tikzcd}
	W & X \\
	Y & Z
	\arrow[from=1-1, to=1-2]
	\arrow[from=1-1, to=2-1]
	\arrow[from=1-2, to=2-2]
	\arrow[from=2-1, to=2-2]
\end{tikzcd}} \\~\\
be a pushout square. Consider the following diagram in $\mC$:  
 \\~\\ \adjustbox{scale=0.8,center}{\begin{tikzcd}
	W & X & 0 \\
	Y & {X\coprod_W Y} & {0\coprod_WY} & 0 \\
	0 & {X\coprod_W0} & {\Sigma_\mC W} & {\Sigma_\mC Y} \\
	& 0 & {\Sigma_\mC X} & {\Sigma_\mC(X\coprod_WY)}
	\arrow[from=1-1, to=1-2]
	\arrow[from=1-1, to=2-1]
	\arrow[from=1-2, to=1-3]
	\arrow[from=1-2, to=2-2]
	\arrow[from=1-3, to=2-3]
	\arrow[from=2-1, to=2-2]
	\arrow[from=2-1, to=3-1]
	\arrow[from=2-2, to=2-3]
	\arrow[from=2-2, to=3-2]
	\arrow[from=2-3, to=2-4]
	\arrow[from=2-3, to=3-3]
	\arrow[from=2-4, to=3-4]
	\arrow[from=3-1, to=3-2]
	\arrow[from=3-2, to=3-3]
	\arrow[from=3-2, to=4-2]
	\arrow[from=3-3, to=3-4]
	\arrow[from=3-3, to=4-3]
	\arrow[from=3-4, to=4-4]
	\arrow[from=4-2, to=4-3]
	\arrow[from=4-3, to=4-4]
\end{tikzcd}} \\~\\
By repeated application of \ref{prp:Pasting3}, we can conclude that each of the seven squares above are pushouts. By applying $F$, we obtain the following diagram:  
 \\~\\ \adjustbox{scale=0.8,center}{\begin{tikzcd}
	{F(W)} \\
	& {F(X)\times_{F(Z)}F(Y)} & {F(X)} & 0 \\
	& {F(Y)} & {F(Z)} & {F(0\coprod_WY)} & 0 \\
	& 0 & {F(X\coprod_W0)} & {F(\Sigma_\mC W)} & {F(\Sigma_\mC Y)} \\
	&& 0 & {F(\Sigma_\mC X)} & {F(\Sigma_\mC Z)}
	\arrow["\mu", from=1-1, to=2-2]
	\arrow[bend left = 20, from=1-1, to=2-3]
	\arrow[bend right = 20, from=1-1, to=3-2]
	\arrow[from=2-2, to=2-3]
	\arrow[from=2-2, to=3-2]
	\arrow[from=2-3, to=2-4]
	\arrow[from=2-3, to=3-3]
	\arrow[from=2-4, to=3-4]
	\arrow[from=3-2, to=3-3]
	\arrow[from=3-2, to=4-2]
	\arrow[from=3-3, to=3-4]
	\arrow[from=3-3, to=4-3]
	\arrow[from=3-4, to=3-5]
	\arrow[from=3-4, to=4-4]
	\arrow[from=3-5, to=4-5]
	\arrow[from=4-2, to=4-3]
	\arrow[from=4-3, to=4-4]
	\arrow[from=4-3, to=5-3]
	\arrow[from=4-4, to=4-5]
	\arrow[from=4-4, to=5-4]
	\arrow[from=4-5, to=5-5]
	\arrow[from=5-3, to=5-4]
	\arrow[from=5-4, to=5-5]
\end{tikzcd}} \\~\\
By considering the large square on the left, we obtain a comparison map $F(X)\times_{F(Z)}F(Y)\to\Omega_\mD F(\Sigma_\mC W)$ which will be called $\theta$. Let $\mu$ be the comparison map $F(W)\to F(X)\times_{F(Z)}F(Y)$. Finally, notice that the following diagram on the left sits in the bottom right of the above square, and that we can add $0$s to the map so that the limits of the two diagrams remain equivalent (coinitial):  
 \\~\\ \adjustbox{scale=0.8,center}{\begin{tikzcd}
	&& 0 &&& 0 & 0 & 0 \\
	&& {F(\Sigma_\mC Y)} & 0 & 0 & 0 \\
	0 & {F(\Sigma_\mC X)} & {F(\Sigma_\mC Z)} && {F(\Sigma_\mC X)} & {F(\Sigma_\mC Z)} & {F(\Sigma_\mC Y)}
	\arrow[from=1-3, to=2-3]
	\arrow[from=1-3, to=3-2]
	\arrow[from=1-6, to=1-7]
	\arrow[from=1-6, to=3-5]
	\arrow[from=1-7, to=3-6]
	\arrow[from=1-8, to=1-7]
	\arrow[from=1-8, to=3-7]
	\arrow[from=2-3, to=3-3]
	\arrow[from=2-4, to=2-5]
	\arrow[from=2-4, to=3-5]
	\arrow[from=2-5, to=3-6]
	\arrow[from=2-6, to=2-5]
	\arrow[from=2-6, to=3-7]
	\arrow[from=3-1, to=2-3]
	\arrow[from=3-1, to=3-2]
	\arrow[from=3-2, to=3-3]
	\arrow[from=3-5, to=3-6]
	\arrow[from=3-7, to=3-6]
\end{tikzcd}} \\~\\
Their limit is the pullback $$\Omega_\mD F(\Sigma_\mC X)\times_{\Omega_\mD F(\Sigma_\mC Z)}\Omega_\mD F(\Sigma_\mC Y)$$ Since all the diagrams map to each other, we obtain a commutative diagram:  
 \\~\\ \adjustbox{scale=0.8,center}{\begin{tikzcd}
	{F(W)} & {F(X)\times_{F(Z)}F(Y)} \\
	& {\Omega_\mD F(\Sigma_\mC W)} & {\Omega_\mD F(\Sigma_\mC X)\times_{\Omega_\mD F(\Sigma_\mC Z)}\Omega_\mD F(\Sigma_\mC Y)}
	\arrow["\mu", from=1-1, to=1-2]
	\arrow["\simeq"', from=1-1, to=2-2]
	\arrow["\theta", from=1-2, to=2-2]
	\arrow["\simeq", from=1-2, to=2-3]
	\arrow[from=2-2, to=2-3]
\end{tikzcd}} \\~\\
where by assumption we have equivalences $F(W)\simeq\Omega_\mD F(\Sigma_\mC W)$ hence there are also equivalences on pullbacks. Notice that this implies $\theta$ has a left and right homotopy inverse, and hence is an equivalence. By the two out of three property, $\mu$ is also an equivalence. Hence $F$ sends pushouts to pullbacks. 
\end{proof}
\end{lmm}

\begin{prp}\label{prp:SICEquivCon}Let $\mC$ be an infinity category. Then $\mC$ is a stable infinity category if and only if the following are true. 
\begin{itemize}
\item $\mC$ has a zero object $0$
\item $\mC$ admits all finite limits and colimits
\item The loop functor $\Omega:\mC\to\mC$ is an equivalence of infinity categories. 
\end{itemize} 
\begin{proof}
It suffices to show that the loop functor is an equivalence if and only if pushout squares and pullback squares coincide. By \ref{prp:SusLoopEquiv} that if $\mC$ is stable then $\Omega$ is an equivalence. 
Conversely, suppose that $\Omega$ is an equivalence of infinity categories. Since $\Sigma$ is adjoint to $\Omega$ by 1.1.2.8 of \cite{HA}, then $\Sigma$ and $\Omega$ are both fully faithful. Choosing the identity functor in \ref{lmm:RedExiffEquiv} shows that pushout squares and pullback squares coincide. 
\end{proof}
\end{prp}

\begin{prp}\label{prp:RedExiffLE} Let $\mC,\mD$ be stable infinity categories. Let $F:\mC\to\mD$ be a functor. Then the following are equivalent. 
\begin{itemize}
\item $F$ commutes with finite limits. 
\item $F$ is reduced and excisive. 
\end{itemize} 
\begin{proof}
Suppose that $F$ is commutes with finite limits. Given a pushout diagram in $\mC$, since $\mC$ is stable by we know that it is a pullback. Since $F$ commutes with finite limits, $F$ sends the pullback to a pullback. Hence $F$ is excisive. Let $\ast$ be a final object of $\mC$. Since $F$ is left exact and $\ast$ is the limit of the empty diagram, $F(\ast)$ is a final object of $\mD$. Hence $F$ is reduced. 

Suppose that $F$ is reduced and excisive. Given $X,Y\in\mC$, the product $X\times Y$ is a pullback so that the following is a pullback squares:  
 \\~\\ \adjustbox{scale=0.8,center}{\begin{tikzcd}
	{X\times Y} & X \\
	Y & \ast
	\arrow[from=1-1, to=1-2]
	\arrow[from=1-1, to=2-1]
	\arrow[from=1-2, to=2-2]
	\arrow[from=2-1, to=2-2]
\end{tikzcd}} \\~\\
Since $\mC$ is stable, the square is also a pushout. Then $F$ sends it to a pullback square:  
 \\~\\ \adjustbox{scale=0.8,center}{\begin{tikzcd}
	{F(X\times Y)} & {F(X)} \\
	{F(Y)} & \ast
	\arrow[from=1-1, to=1-2]
	\arrow[from=1-1, to=2-1]
	\arrow[from=1-2, to=2-2]
	\arrow[from=2-1, to=2-2]
\end{tikzcd}} \\~\\
where $F(\ast)\simeq\ast$ since $F$ is reduced. Thus we obtain an equivalence $F(X\times Y)\simeq F(X)\times F(Y)$.

Let $f,g:X\to Y$ be two maps in $\mC$. Let $E$ be the equalizer of $f$ and $g$. Since $E$ is the pullback of the following diagram,  
 \\~\\ \adjustbox{scale=0.8,center}{\begin{tikzcd}
	X & {X\times X} & {X\times_YX}
	\arrow["\Delta", from=1-1, to=1-2]
	\arrow[from=1-3, to=1-2]
\end{tikzcd}} \\~\\
where the map on the right is given by the universal property of $X\times X$, we obtain a pullback square  
 \\~\\ \adjustbox{scale=0.8,center}{\begin{tikzcd}
	E & {X\times_YX} \\
	X & {X\times X}
	\arrow[from=1-1, to=1-2]
	\arrow[from=1-1, to=2-1]
	\arrow[from=1-2, to=2-2]
	\arrow["\Delta"', from=2-1, to=2-2]
\end{tikzcd}} \\~\\
Since $\mC$ is stable, this is also a pushout square. Since $F$ is excisive, the following is a pullback square:  
 \\~\\ \adjustbox{scale=0.8,center}{\begin{tikzcd}
	{F(E)} & {F(X\times_YX)} \\
	{F(X)} & {F(X\times X)}
	\arrow[from=1-1, to=1-2]
	\arrow[from=1-1, to=2-1]
	\arrow[from=1-2, to=2-2]
	\arrow[from=2-1, to=2-2]
\end{tikzcd}} \\~\\
Since $F$ sends pullbacks to pullbacks, we obtain equivalences $F(X\times_YX)\simeq F(X)\times_{F(Y)}F(X)$ and $F(X\times X)\simeq F(X)\times F(X)$. Hence the pullback square  
 \\~\\ \adjustbox{scale=0.8,center}{\begin{tikzcd}
	{F(E)} & {F(X)\times_{F(Y)}F(X)} \\
	{F(X)} & {F(X)\times F(X)}
	\arrow[from=1-1, to=1-2]
	\arrow[from=1-1, to=2-1]
	\arrow[from=1-2, to=2-2]
	\arrow[from=2-1, to=2-2]
\end{tikzcd}} \\~\\
Therefore $F(E)$ is the equalizer of $F(f)$ and $F(g)$, and so $F$ preserves equalizers. Since $F$ preserves equalizers and products, we conclude that $F$ preserves all finite limits. 
\end{proof}
\end{prp}

\subsection{The Equivelence between Excisive Functors and Spectra}
Let $\mC$ be a simplicial category enriched in Kan Complexes. The homotopy coherent nerve $N_\bullet^\text{hc}(-)$ sends such a category to an infinity category. Following \cite{Kerodon}, we write $\mS=N_\bullet^{\text{hc}}(\bold{Kan})$ for the infinity category of spaces. This infinity category is equivalent to $N_\bullet^\text{hc}(\bold{CW})$, which is also equivalent to first inverting weak equivalences of $\bold{CGWH}$, and then taking homotopy coherent nerve. 

\begin{defn}\label{defn:ICSpec} Define the infinity category of spectra by $$\text{Sp}(\mS)=\lim(\cdots\rightarrow\mS\overset{\Omega}{\rightarrow}\mS\overset{\Omega}{\rightarrow}\mS)$$
\end{defn}

Our goal is to show that there is an equivalence between $\text{Sp}(\mS)$ and $\text{Exc}_\ast(\mS_\ast^\text{fin},\mS)$. For that we need to set up some lemmas. 

\begin{defn}\label{defn:Delooping} Let $\mC$ be an infinty category that admits all finite limits. Define the delooping functor of $\mC$ to be the evaluation functor $$\Omega_\mC^\infty:\text{Exc}_\ast(\mS_\ast^\text{fin},\mC)\to\mC$$ given on object by $(F:\mS_\ast^\text{fin}\to F(S^0)$
\end{defn}

\begin{prp}\label{prp:DeloopEquiv} Let $\mC$ be a pointed infinity category that admits all finite colimits. Let $\mD$ be an infinity category that admits all finite limits. Then post composition with $\Omega^\infty$ gives an equivalence of infinity categories $$\Omega_\mD^\infty\circ-:\text{Exc}_\ast(\mC,\text{Exc}_\ast(\mS_\ast^\text{fin},\mD))\overset{\simeq}{\rightarrow}\text{Exc}_\ast(\mC,\mD)$$ 
\begin{proof}
Notice that there is a canonical isomorphism 
\begin{align*}
\text{Exc}_\ast(\mC,\text{Exc}_\ast(\mS_\ast^\text{fin},\mD))&\simeq\text{Exc}_\ast(\mC\times\mS_\ast^\text{fin},\mD)\tag{Full subcat + adjoint}\\
&\simeq\text{Exc}_\ast(\mS^\text{fin},\text{Exc}_\ast(\mC,\mD))
\end{align*}
Under this identification, the functor $\Omega^\infty\circ -$ corresponds to the functor $\Omega_{\text{Exc}_\ast(\mC,\mD)}^\infty$ since $\Omega$ are computed term wise (like all limits). By 1.4.2.16 of \cite{HA}, $\text{Exc}_\ast(\mC,\mD)$ is stable. Hence by \ref{prp:SusLoopEquiv} we conclude that $\Omega_{\text{Exc}_\ast(\mC,\mD)}^\infty$ is an equivalence of infinity categories. Hence $\Omega_\mD^\infty\circ-$ is an equivalence of infinity categories. 
\end{proof}
\end{prp}

\begin{thm}\label{thm:SpEquivEx} There is an equivalence of infinity categories $$\text{Sp}(\mS)\simeq\text{Exc}_\ast(\mS_\ast^\text{fin},\mS)$$ 
\begin{proof}~\\
Since $\mS$ is presentable and the infinity category of presentable infinity categories admit all small limits, $\text{Sp}(\mS)$ is also presentable by 5.5.3.18 of \cite{HTT}. Every presentable infinity category admits all small limits and colimits by of \cite{HTT}. Since $\mS$ is pointed, $\text{Sp}(\mS)$ is also pointed. Since all limits are computed term-wise by \ref{prp:TermwiseLim}, we have that in particular $\Omega_{\text{Sp}(\mS)}$ is computed term wise. Given $X=\{X_n\;|\;n\in\N\}$ an object of $\text{Sp}(\mS)$, $\Omega_{\text{Sp}(\mS)}X$ is equivalent to $X$ because we have that $\Omega X_{n+1}$ is equivalent to $X_n$ for all $n$. By \ref{prp:SICEquivCon} we conclude that $\text{Sp}(\mS)$ is stable. 

Consider the canonical functor $F:\text{Sp}(\mS)\to\mS$ defined by recovering the first factor: $(X_0,X_1,\dots)\mapsto X_0$. It is clear that it commutes with finite limits since limits are computed term-wise by \ref{prp:TermwiseLim}. Hence $F$ is left exact. Using the equivalence of infinity categories $$\Omega^\infty\circ-:\text{Exc}_\ast^\text{L}(\text{Sp}(\mS),\text{Exc}_\ast(\mS_\ast^\text{fin},\mS))\to\text{Exc}_\ast (\text{Sp}(\mS),\mS)$$ we obtain a factorization  
 \\~\\ \adjustbox{scale=0.8,center}{\begin{tikzcd}
	{\text{Sp}(\mS)} && \mS \\
	& {\text{Exc}_\ast(\mS_\ast^\text{fin},\mS)}
	\arrow["F", from=1-1, to=1-3]
	\arrow["{G\circ -}"', from=1-1, to=2-2, dashed]
	\arrow["{\Omega^\infty}"', from=2-2, to=1-3]
\end{tikzcd}} \\~\\
Let $\mC$ be an arbitrary stable infinity category. By functoriality we obtain a similar factorization:  
 \\~\\ \adjustbox{scale=0.8,center}{\begin{tikzcd}
	{\text{Exc}_\ast (\mC,\text{Sp}(\mS))} && {\text{Exc}_\ast (\mC,\mS)} \\
	& {\text{Exc}_\ast (\mC,\text{Exc}_\ast(\mS_\ast^\text{fin},\mS))}
	\arrow["{F\circ -}", from=1-1, to=1-3]
	\arrow["{G\circ -}"', from=1-1, to=2-2]
	\arrow["{\Omega^\infty\circ -}"', from=2-2, to=1-3]
\end{tikzcd}} \\~\\ 
I claim that $F\circ -$ and $\Omega^\infty\circ -$ are equivalences so that $G\circ -$ is an equivalence. The case $\Omega^\infty\circ -$ is already clear. On the other hand, since $\Omega$ are computed term-wise by \ref{prp:TermwiseLim} and since $\text{Func}(\mC,-)$ is right adjoint to $\mC\times-$ we know that $\text{Func}$ commutes with finite limits. Thus we have that $$\text{Exc}_\ast(\mC,\text{Sp}(\mS))=\lim(\cdots\rightarrow\text{Exc}_\ast (\mC,\mS)\overset{\Omega\circ-}{\rightarrow}\text{Exc}_\ast (\mC,\mS)\overset{\Omega\circ-}{\rightarrow}\text{Exc}_\ast (\mC,\mS))$$ Now $\Omega_\mS\circ -$ is an equivalence because $\text{Exc}_\ast(\mC,\mS)$ is stable (because $\mC$ is stable and $\mS$ admits all finite (co)limits). We conclude that $\text{Exc}_\ast (\mC,\text{Sp}(\mS))\simeq\text{Exc}_\ast (\mC,\mS)$. Thus evaluation on the first factor $F\circ -:\text{Exc}_\ast (\mC,\text{Sp}(\mS))\to\text{Exc}_\ast (\mC,\mS)$ is an equivalence of infinity categories. 

From the fact that $G\circ -$ is an equivalence, we have an equivalence $$\text{Exc}_\ast(\mC,\text{Sp}(\mS))\simeq\text{Exc}_\ast(\mC,\text{Exc}_\ast(\mS_\ast^\text{fin},\mS))$$ for all stable infinity categories $\mC$. By the Yoneda lemma we conclude that $\text{Exc}_\ast(\mS_\ast^\text{fin},\mS)$ and $\text{Sp}(\mS)$ are equivalent. 
\end{proof}
\end{thm}

\section{Excisive Functors, Spectra and Homology Theories}
Spectra is well known to be related to cohomology. Explicitly, a modern reformuation of in \cite{FSHT} of Brown's representability theorem proves that every cohomology theory is represented by a spectra, while functors of the form $[-,K_n]$ for $\{K_n\;|\;n\in\N\}$ a spectrum defines a cohomology theory. In this final chapter, we will see that one can associate a spectra to a homology theory. In fact, it is more natural to think of such an association using the notion of excisive functors. 

In this section, we follows the brief outlines given in section 1 of \cite{GW1} which connects the notion of excisive functors, reduced homology theories and spectra. 

\subsection{From Excisive Functors to Homology}
The following version of reduced homology theory is given in both \cite{LNAT} and \cite{ATHH}. 

\begin{defn}\label{defn:RedHomology} A reduced homology theory is a collection of functors and natural trasnformations $$H_n:\bold{Top}_\ast\to\bold{Ab}\;\;\;\;\text{ and }s_n:H_n\Rightarrow H_{n+1}\circ\Sigma$$ such that the following axioms are satisfied. 
\begin{itemize}
\item If $f\simeq g:X\to Y$ are homotopic, then $H_n(f)=H_n(g)$
\item If $f:X\to Y$ is a map then there is an exact sequence $H_n(X)\overset{f_\ast}{\rightarrow}H_n(Y)\overset{j_\ast}{\rightarrow}H_n(C_f)$ where $j:X\to C_f$ is the inclusion. 
\item The natural transformation gives an isomorphism $s_n(X):H_n(X)\to H_{n+1}(\Sigma X)$ for all $n$. 
\item For any wedge sum $X=\bigvee_{i\in I}X_i$, the inclusion maps induces an isomorphism $\prod_{i\in I} H_n(X_{i\in I})\cong H_n(X)$
\item If $f:X\to Y$ is a weak homotopy equivalence, then $H_n(f)$ is an isomorphism. 
\end{itemize}
\end{defn}

The following theorem shows that by passing through an exicisve functor first and then apply the homotopy group functors, we obtain a reduced homology theory. I was unable to find a reference in the literature and most of the proof is original. 

\begin{thm}\label{thm:HomotopyGrpExHomology} Let $F:\bold{Top}_\ast\to\bold{Top}_\ast$ be a reduced, excisive, finitary and homotopy functor. Then $E_n(X)=\pi_n(F(X))$ defines a reduced homology theory. 
\begin{proof}
Firstly, note that $\pi_0(F(X))$ and $\pi_1(F(X))$ has the structure of an abelian group. To see this, recall that from \ref{eg:SusLoop} we have  
 \\~\\ \adjustbox{scale=0.8,center}{\begin{tikzcd}
	X & \ast \\
	\ast & {\Sigma X}
	\arrow[from=1-1, to=1-2]
	\arrow[from=1-1, to=2-1]
	\arrow[from=1-2, to=2-2]
	\arrow[from=2-1, to=2-2]
\end{tikzcd}} \\~\\
is a homotopy pushout. Since $F$ is excisive, we obtain a homotopy pullback square:  
 \\~\\ \adjustbox{scale=0.8,center}{\begin{tikzcd}
	{F(X)} & {F(\ast)} \\
	{F(\ast)} & {F(\Sigma X)}
	\arrow[from=1-1, to=1-2]
	\arrow[from=1-1, to=2-1]
	\arrow[from=1-2, to=2-2]
	\arrow[from=2-1, to=2-2]
\end{tikzcd}} \\~\\
and therefore we obtain a weak equivalence $F(X)\to\text{holim}(F(\ast)\rightarrow F(\Sigma X)\leftarrow F(\ast))$. On the other hand, we know that $\text{holim}(\ast\rightarrow F(\Sigma X)\leftarrow\ast)\cong\Omega F(\Sigma X)$ by \ref{eg:SusLoop}. Finally, from the commutative diagram:  
 \\~\\ \adjustbox{scale=0.8,center}{\begin{tikzcd}
	\ast & {F(\Sigma X)} & \ast \\
	{F(\ast)} & {F(\Sigma X)} & {F(\ast)}
	\arrow[from=1-1, to=1-2]
	\arrow["\simeq"', from=1-1, to=2-1]
	\arrow["{\text{id}}"', from=1-2, to=2-2]
	\arrow[from=1-3, to=1-2]
	\arrow["\simeq", from=1-3, to=2-3]
	\arrow[from=2-1, to=2-2]
	\arrow[from=2-3, to=2-2]
\end{tikzcd}} \\~\\
and \ref{prp:HpullPHomotopy}, we deduce that we have a weak equivalence:  
 \\~\\ \adjustbox{scale=0.8,center}{\begin{tikzcd}
	{\text{holim}(\ast\rightarrow F(\Sigma X)\leftarrow\ast)} & {\text{holim}(F(\ast)\rightarrow F(\Sigma X)\leftarrow F(\ast))}
	\arrow["\simeq", from=1-1, to=1-2]
\end{tikzcd}} \\~\\
Then we conclude that we have a chain of weak equivalences going the wrong way:  
 \\~\\ \adjustbox{scale=0.8,center}{\begin{tikzcd}
	{F(X)} & {\text{holim}(F(\ast)\rightarrow F(\Sigma X)\leftarrow F(\ast))} & {\text{holim}(\ast\rightarrow F(\Sigma X)\leftarrow\ast)} & {\Omega F(\Sigma X)}
	\arrow["\simeq", from=1-1, to=1-2]
	\arrow["\simeq"', from=1-3, to=1-2]
	\arrow["\cong"', from=1-4, to=1-3]
\end{tikzcd}} \\~\\
Regardless, we can still deduce a natural isomorphism $$\pi_n(F(X))\cong\pi_n(\Omega F(\Sigma X))$$ because $\pi_n$ sends weak equivalences to isomorphisms, and isomorphisms are invertible. We conclude that the natural bijection $\pi_0(F(X))\cong\pi_2(F(\Sigma^2 X))$ gives an abelian group structure on $\pi_0(F(X))$, and $\pi_1(F(X))\cong\pi_2(\Omega F(\Sigma^2 X)$ show that $\pi_1(F(X))$ is abelian. Now we prove that $\pi_n\circ F$ satisfies the axioms for being a reduced homology theory. 

To show that $\pi_n\circ F$ sends homotopic morphisms to equal morphisms, first consider the inclusion maps $i_0:X\times\{0\}\hookrightarrow X\times I$ and $i_1:X\times\{1\}\to X\times I$. They induces maps $\pi_n(F(i_0))$ and $\pi_n(F(i_1))$. I claim that these are the same map in $\bold{Ab}$. To see this, consider the post composition of both of these functions:  
 \\~\\ \adjustbox{scale=0.8,center}{\begin{tikzcd}
	{\pi_n(F(X))} && {\pi_n(F(X\times I))} && {\pi_n(F(X))}
	\arrow["{\pi_n(F(i_0))}", shift left, from=1-1, to=1-3]
	\arrow["{\pi_n(F(i_1))}"', shift right, from=1-1, to=1-3]
	\arrow["{\pi_n(F(\text{proj}))}", from=1-3, to=1-5]
\end{tikzcd}} \\~\\
where the second map is induced by projection on the first coordinate. Since $I$ is contractible, this projection map is a weak equivalence. Since $F$ preserves weak equivalences and $\pi_n$ sends weak equivalences to isomorphisms, we conclude that the latter map in the diagram is an isomorphism. Now consider the composite $\pi_n(F(\text{proj}))\circ\pi_n(F(i_0))=\pi_n(F(\text{proj}\circ i_0))$. Since $\text{proj}\circ i_0$ is a weak equivalence, $F$ of this composite is also a weak equivalence and hence is an isomorphism after passing through $\pi_n$. Since the composite and $\pi_n(F(\text{proj}))$ is an isomorphism, we conclude that $\pi_n(F(i_0))$ is an isomorphism. A similar argument shows that $\pi_n(F(i_1))$ is an isomorphism. 

Recall that we want to show that if $f$ and $g$ are homotopic, then $\pi_n(F(f))=\pi_n(F(g))$. Suppose that $H$ witnesses the homotopy from $f$ to $g$. Then we obtain a commutative diagram:  
 \\~\\ \adjustbox{scale=0.8,center}{\begin{tikzcd}
	{X\times\{0\}} \\
	& {X\times I} & Y \\
	{X\times\{1\}}
	\arrow["i_0", hook, from=1-1, to=2-2]
	\arrow["f", from=1-1, to=2-3, bend left = 20]
	\arrow["H", from=2-2, to=2-3]
	\arrow["i_1", hook, from=3-1, to=2-2]
	\arrow["g", from=3-1, to=2-3, bend right = 20]
\end{tikzcd}} \\~\\
where we identify $X\times\{0\}\cong X\cong X\times\{1\}$. Since $\pi_n\circ F$ is a functor, passing through $\pi_n\circ F$ gives the following commutative diagram:  
 \\~\\ \adjustbox{scale=0.8,center}{\begin{tikzcd}
	{\pi_n(F(X\times\{0\}))} \\
	& {\pi_n(F(X\times I))} & {\pi_n(F(Y))} \\
	{\pi_n(F(X\times\{1\}))}
	\arrow["\pi_n(F(i_0))"', from=1-1, to=2-2]
	\arrow["{\pi_n(F(f))}", from=1-1, to=2-3, bend left = 20]
	\arrow["{\pi_n(F(H))}", from=2-2, to=2-3]
	\arrow["\pi_n(F(i_1))", from=3-1, to=2-2]
	\arrow["{\pi_n(F(g))}"', from=3-1, to=2-3, bend right = 20]
\end{tikzcd}} \\~\\
Since the maps induced by $i_0$ and $i_1$ are isomorphisms and the diagram is commutative, we conclude that $\pi_n(F(f))=\pi_n(F(g))$. 

If $f:X\to Y$ is a map, then $X\to Y\to C_f$ is a cofiber sequence, and we have a homotopy pushout  
 \\~\\ \adjustbox{scale=0.8,center}{\begin{tikzcd}
	X & Y \\
	\ast & {C_f}
	\arrow["f", from=1-1, to=1-2]
	\arrow[from=1-1, to=2-1]
	\arrow[from=1-2, to=2-2]
	\arrow[from=2-1, to=2-2]
\end{tikzcd}} \\~\\
$F$ sends this to a homotopy pullback, so we obtain a fiber sequence $F(X)\to F(Y)\to F(C_f)$. Then $\pi_n$ sends the fiber sequence to the desired exact sequence. 

Recall that the wedge sum $\bigvee_{i\in I}X_i$ of the spaces $X_i$ is the colimit of the diagram consisting of the unique morphisms $\ast\to X_i$. Then the homotopy colimit is homeomorphic to the colimit as seen in chapter 8 of \cite{CHT}. Since $F$ is finitary by assumption, we have a homeomorphism $\bigvee_{i\in I}F(X_i)\cong F\left(\bigvee_{i\in I}X_i\right)$. We are left with trying to show that the canonical map $$\pi_n\left(\bigvee_{i\in I}F(X_i)\right)\to\prod_{i\in I}\pi_n\left(F(X_i)\right)$$ induced by projections is an isomorphism. Recall that $\pi_n(X)=[S^n,X]_\ast$. Since $S^n$ is compact for any $n$, if $[f]\in\pi_n(X)$ then $\im(f)$ is compact and so lies in finitely many $X_i$. Also, if $[f]$ is homotopic to the constant map, then the homotopy $H:S^n\times I\to X$ from $f$ to the constant map also has compact domain. Therefore the proof that the canonical map is an isomorphism can be reduced to the case of when $I$ is a finite indexing set. By induction we can reduce this to the case of the wedge of two spaces $X_1$ and $X_2$. Therefore it suffices to show that $\pi_n(F(X_1\vee X_2))\cong\pi_n(X_1)\times\pi_n(X_2)$. 

Consider the following homotopy pushout square:  
 \\~\\ \adjustbox{scale=0.8,center}{\begin{tikzcd}
	\ast & {X_1} \\
	{X_2} & {X_1\vee X_2}
	\arrow[from=1-1, to=1-2]
	\arrow[from=1-1, to=2-1]
	\arrow[from=1-2, to=2-2]
	\arrow[from=2-1, to=2-2]
\end{tikzcd}} \\~\\
Since $F$ is excisive, the following square is a homotopy pullback:  
 \\~\\ \adjustbox{scale=0.8,center}{\begin{tikzcd}
	F(\ast) & {F(X_1)} \\
	{F(X_2)} & {F(X_1\vee X_2)}
	\arrow[from=1-1, to=1-2]
	\arrow[from=1-1, to=2-1]
	\arrow[from=1-2, to=2-2]
	\arrow[from=2-1, to=2-2]
\end{tikzcd}} \\~\\
Now the inclusion $X_1\vee X_2\to X_1\times X_2$ followed by the two projections $X_1\times X_2\to X_1,X_2$ induces a map $F(X_1\vee X_2)\to F(X_1)\times F(X_2)$. Pre-composing further with inclusion maps $X_1,X_2\to X_1\vee X_2$ gives a map of homotopy pullbacks squares:  
 \\~\\ \adjustbox{scale=0.8,center}{\begin{tikzcd}
	{F(\ast)} && {F(X_1)} \\
	& {F(X_2)} && {F(X_1\vee X_2)} \\
	\ast && {F(X_1)} \\
	& {F(X_2)} && {F(X_1)\times F(X_2)}
	\arrow[from=1-1, to=1-3]
	\arrow[from=1-1, to=2-2]
	\arrow["\simeq"', from=1-1, to=3-1]
	\arrow[from=1-3, to=2-4]
	\arrow[from=1-3, to=3-3]
	\arrow[from=2-2, to=2-4]
	\arrow[from=2-2, to=4-2]
	\arrow[from=2-4, to=4-4]
	\arrow[from=3-1, to=3-3]
	\arrow[from=3-1, to=4-2]
	\arrow[from=3-3, to=4-4]
	\arrow[from=4-2, to=4-4]
\end{tikzcd}} \\~\\
By \ref{prp:LESHpull} we then obtain a map of long exact sequences:  
 \\~\\ \adjustbox{scale=0.6,center}{\begin{tikzcd}
	\cdots & {\pi_{n+1}(F(\ast))} & {\pi_{n+1}(F(X_1)\times F(X_2))} & {\pi_n((F(X_1)\vee F(X_2))} & {\pi_{n-1}(F(\ast))} & {\pi_{n-1}(F(X_1)\times F(X_2))} & \cdots & {\pi_0(F(X_1)\times F(X_2))} \\
	\cdots & {\pi_n(\ast)} & {\pi_n(F(X_1)\times F(X_2))} & {\pi_n(F(X_1)\times F(X_2))} & {\pi_{n-1}(\ast)} & {\pi_{n-1}(F(X_1)\times F(X_2))} & \cdots & {\pi_0(F(X_1)\times F(X_2))}
	\arrow[from=1-1, to=1-2]
	\arrow[from=1-2, to=1-3]
	\arrow["\cong"', from=1-2, to=2-2]
	\arrow[from=1-3, to=1-4]
	\arrow["{\text{id}}"', from=1-3, to=2-3]
	\arrow[from=1-4, to=1-5]
	\arrow[from=1-4, to=2-4]
	\arrow[from=1-5, to=1-6]
	\arrow["\cong", from=1-5, to=2-5]
	\arrow[from=1-6, to=1-7]
	\arrow["{\text{id}}", from=1-6, to=2-6]
	\arrow[from=1-7, to=1-8]
	\arrow["{\text{id}}", from=1-8, to=2-8]
	\arrow[from=2-1, to=2-2]
	\arrow[from=2-2, to=2-3]
	\arrow[from=2-3, to=2-4]
	\arrow[from=2-4, to=2-5]
	\arrow[from=2-5, to=2-6]
	\arrow[from=2-6, to=2-7]
	\arrow[from=2-7, to=2-8]
\end{tikzcd}} \\~\\
The five lemma implies isomorphisms $\pi_n(F(X_1\vee X_2))\cong\pi_n(F(X_1)\times F(X_2))\cong\pi_n(F(X_1))\times\pi_n(F(X_2))$ for $n\geq 1$. For the case $n=0$, we note that since $\Sigma (X_1\vee X_2)\cong\Sigma X_1\vee\Sigma X_2$ as given in \cite{AT}, we have that 
\begin{align*}
\pi_0(F(X_1\vee X_2))&\cong\pi_1(F(\Sigma(X_1\vee X_2)))\\
&\cong\pi_1(F(\Sigma X_1\vee\Sigma X_2))\\
&\cong\pi_1(F(\Sigma X_1)\vee F(\Sigma X_2))\\
&\cong\pi_1(F(\Sigma X_1))\times\pi_1(F(\Sigma X_2))\\
&\cong\pi_0(F(X_1))\times\pi_0(F(X_2))
\end{align*} and so we conclude. 
\end{proof}
\end{thm}

\begin{eg}\label{eg:SHG} The stable homotopy groups arise naturally as a reduced homology theory in the following way. The homotopy groups $\pi_\ast$ do not make up a reduced homology theory because they are missing the excisive property. This can be explained as the identity functor not quite being excisive by the Blakers-Massey theorem. 

While the identity functor is not excisive, we can take its excisive approximation $P_1F(X)=\underset{k}{\text{hocolim}}\Omega^k\Sigma^k X$ to give a reduced homology theory $$X\mapsto\pi_n(\underset{k}{\text{hocolim}}(\Omega^k\Sigma^kX))$$ Since $\pi_n$ sends directed homotopy colimits to directed colimits by \ref{prp:HomGrpPHocolim}, the group on the right is isomorphic to $$\lim_{k\to\infty}\pi_n(\Omega^k\Sigma^k X)\cong\lim_{k\to\infty}\pi_{n+k}(\Sigma^k X)$$ This is precisely the definition of the stable homotopy groups of $X$. 
\end{eg}

\begin{eg}\label{eg:RedHomlogy} Let $X$ be a space with base point $x_0$. For each $n\in\N$, the symmetric group $S^n$ acts on $X^n$ by permuting the coordinates. We can then form the orbit space of the action, denoted $SP_n(X)$. The inclusion $X^n\hookrightarrow X^{n+1}$ given by $(x_1,\dots,x_n)\mapsto(x_1,\dots,x_n,x_0)$ induces a well defined inclusion $SP_n(X)\to SP_{n+1}(X)$. We call the colimit as $n\to\infty$ by the infinite symmetric product $$SP(X)=\underset{n\to\infty}{\text{colim}}SP_n(X)$$ It turns out that $SP(-)$ is functorial, preserves weak equivalencs and is excisive. The proof of excision involves the use of 3-cubes and quasi-fibrations, which is not the focus of the paper. We refer to \cite{CHT} and \cite{ATHV} for a more detailed discussion. 

In any case, post composing with the homotopy groups give a reduced homology theory $\pi_\ast(SP(-))$ and it turns out that there are natural isomorphisms $$\pi_\ast(SP(X))\cong\widetilde{H}_{\text{Sing}}(X;\Z)$$ so that the reduced singular homology theory can be given by an exicisve functor in this manner. In fact, \cite{ATHV} defines singular homology using this method. 

We can interpret the Mayer-Vietoris theorem in the following way: For a space $X$ that we can decompose as a union $X=A\cup B$, the following square with inclusion maps is a homotopy pushout: 
 \\~\\ \adjustbox{scale=0.8,center}{\begin{tikzcd}
	{A\cap B} & A \\
	B & X
	\arrow[from=1-1, to=1-2]
	\arrow[from=1-1, to=2-1]
	\arrow[from=1-2, to=2-2]
	\arrow[from=2-1, to=2-2]
\end{tikzcd}} \\~\\
Since $SP(-)$ is excisive, the following square is a homotopy pullback: 
 \\~\\ \adjustbox{scale=0.8,center}{\begin{tikzcd}
	{SP(A\cap B)} & {SP(A)} \\
	{SP(B)} & {SP(X)}
	\arrow[from=1-1, to=1-2]
	\arrow[from=1-1, to=2-1]
	\arrow[from=1-2, to=2-2]
	\arrow[from=2-1, to=2-2]
\end{tikzcd}} \\~\\
Then using \ref{prp:LESHpull} we obtain a long exact sequence of homotopy groups, which is precisely the Mayer-Vietoris sequence. 
\end{eg}

\subsection{From Spectra to Homology}
\begin{defn}\label{defn:HomSpec} Let $K=\{K_n\;|\;n\in\N\}$ be a spectrum. Define a functor $E_n:\bold{Top}\to\bold{Ab}$ by $$E_n(X;K)=\lim_{k\to\infty}\pi_{n+k}(X\wedge K_k)$$ and natural transformations $s_n:E_n\Rightarrow E_{n+1}\circ\Sigma$ given by the structure maps of the spectrum. 
\end{defn}

Here, the maps defining the direct limit $\pi_{n+k}(X\wedge K_k)\to\pi_{n+1+k}(X\wedge K_{k+1})$ are given by $$\pi_{n+k}(X\wedge K_k)\overset{\Sigma}{\rightarrow}\pi_{n+1+k}(S^1\wedge X\wedge K_k)\overset{\text{id}_X\wedge\sigma_n}{\rightarrow}\pi_{n+1+k}(X\wedge K_{k+1})$$ Moreover, it is a functor since any map $f:X\to Y$ wedges with the identity to give a map $X\wedge K_k\to Y\wedge K_k$. The homotopy group functor gives a map of homotopy groups. And the universal property of direct limits give the induced map. 

\begin{prp}\label{prp:WedgeSpecExcisive} Let $\{K_n,\sigma_n\}$ be a spectrum. Then the functor $$X\mapsto\text{hocolim}(K_0\wedge X\rightarrow\Omega(K_1\wedge X)\rightarrow\Omega^2(K_2\wedge X)\rightarrow\cdots)$$ is an excisive functor. 
\begin{proof}
Any space is $(-1)$-connected. In particular $K_0$ is $(-1)$-connected. Since $K_0\to\Omega K_1$ is a weak equivalence, we know that the adjoint map $\Sigma K_0\to K_1$ is $1$-connected by \ref{crl:AdjointConn}. Since $K_0$ is $(-1)$-connected, we have that $\Sigma K_0$ is $0$-connected. Together with the fact that $\Sigma K_0\to K_1$ is $1$-connected means that $K_1$ is $0$-connected. By doing this argument inductively, we deduce that $K_n$ is $(n-1)$-connected. In particular, for any space $X$, $X$ is $(-1)$-connected, then by \ref{prp:SmashConn} we have that $K_n\wedge X$ is $(n-1)$-connected.  

Let the following diagram be a homotopy pushout:  
 \\~\\ \adjustbox{scale=0.8,center}{\begin{tikzcd}
	W & Y \\
	X & Z
	\arrow[from=1-1, to=1-2]
	\arrow[from=1-1, to=2-1]
	\arrow[from=1-2, to=2-2]
	\arrow[from=2-1, to=2-2]
\end{tikzcd}} \\~\\
Using the homeomorphism $K_n\wedge\text{hocolim}(X_1\leftarrow X_0\rightarrow X_2)\cong\text{hocolim}(K_n\wedge X_1\leftarrow K_n\wedge X_0\rightarrow K_n\wedge X_2)$ by \ref{prp:WedgeCommHPull}, we deduce that the following square is also a homotopy pushout:  
 \\~\\ \adjustbox{scale=0.8,center}{\begin{tikzcd}
	{K_n\wedge W} & {K_n\wedge Y} \\
	{K_n\wedge X} & {K_n\wedge Z}
	\arrow[from=1-1, to=1-2]
	\arrow[from=1-1, to=2-1]
	\arrow[from=1-2, to=2-2]
	\arrow[from=2-1, to=2-2]
\end{tikzcd}} \\~\\
The spaces $W,X,Y,Z$ are $(-1)$-connected, and so all spaces in the above square is $(n-1)$-cartesian. In particular, the maps $K_n\wedge W\to K_n\wedge X$ and $K_n\wedge W\to K_n\wedge Y$ are $(n-1)$-connected. By the Blakers-Massey theorem \ref{thm:BMT}, the same square is $(2n-3)$-connected. Thinking of $\Omega X$ as a homotopy pullback, by 8.5.5 of \cite{CHT} we know that $\Omega$ commutes with homotopy limits. The  isomorphism $\pi_n(X)\cong\pi_{n+1}(\Omega X)$ for all spaces $X$ means that applying $\Omega$ reduces the connectivity of the space by $1$. Therefore the diagram  
 \\~\\ \adjustbox{scale=0.8,center}{\begin{tikzcd}
	{\Omega^n(K_n\wedge W)} & {\Omega^n(K_n\wedge Y)} \\
	{\Omega^n(K_n\wedge X)} & {\Omega^n(K_n\wedge Z)}
	\arrow[from=1-1, to=1-2]
	\arrow[from=1-1, to=2-1]
	\arrow[from=1-2, to=2-2]
	\arrow[from=2-1, to=2-2]
\end{tikzcd}} \\~\\
is $(n-3)$-cartesian. By 8.5.5 \cite{CHT}, homotopy colimits commute with homotopy pushouts and as $n\to\infty$ we conclude that the functor is excisive. 
\end{proof}
\end{prp}

\begin{thm}\label{thm:SpecHomology} Let $K=\{K_n\;|\;n\in\N\}$ be a spectrum. Then $E_n(X)=\lim_{k\to\infty}\pi_{n+k}(X\wedge K_k)$ defines a reduced homology theory. 
\begin{proof}
We have seen that $X\mapsto\underset{k}{\text{hocolim}}\Omega^k(X\wedge K_k)$ is an excisive functor. It is reduced since the smash of any space with the one point space is the one point space, and that the loopspace of the one point space is the one point space. It is a homotopy functor by 8.3.7 of \cite{CHT}. 

We show that it is finitary. Let $I$ be a filtered category. Then we have that $$\underset{i\in I}{\text{hocolim}}(\underset{k}{\text{hocolim}}\Omega^k(X_i\wedge K_k))\cong\underset{k}{\text{hocolim}}(\underset{i\in I}{\text{hocolim}}\Omega^k(X_i\wedge K_k))$$ by 8.5.5 of \cite{CHT}. Since $\Omega$ is a homotopy pullback, it is in particular a finite homotopy limit. By 8.5.5 of \cite{CHT} we conclude that $$\underset{k}{\text{hocolim}}(\underset{i\in I}{\text{hocolim}}\Omega^k(X_i\wedge K_k))\cong\underset{k}{\text{hocolim}}\Omega^k(\underset{i\in I}{\text{hocolim}}(X_i\wedge K_k))$$ Finally, by 8.5.8 of \cite{CHT} we know that $$\underset{k}{\text{hocolim}}\Omega^k(\underset{i\in I}{\text{hocolim}}(X_i\wedge K_k))\cong\underset{k}{\text{hocolim}}\Omega^k(\underset{i\in I}{\text{hocolim}}(X_i)\wedge K_k)$$ Therefore the said functor is finitary and so $\pi_n(\underset{k}{\text{hocolim}}\Omega^k(X\wedge K_k))$ defines a reduced homology theory. But we also have
\begin{align*}
\pi_n(\underset{k}{\text{hocolim}}\Omega^k(X\wedge K_k))&=\lim_{k\to\infty}\pi_n(\Omega^k(X\wedge K_k))\\
&=\lim_{k\to\infty}\pi_{n+k}(X\wedge K_k)
\end{align*}
and this establishes the claim. 
\end{proof}
\end{thm}

\begin{eg}\label{eg:EMSpec} Recall the Eilenberg-Maclane spectrum for an abelian group $G$, denoted by $K=\{K(G,n)\;|\;n\in\N\}$. Its associated reduced homology theory is given by $$E_n(X;K)=\lim_{k\to\infty}\pi_{n+k}(X\wedge K(G,k))$$ Notice that we can compute $E_0(S^0;K)\cong G$. The remark below 8.28 of \cite{LNAT} then implies that $E_n(X;K)$ is naturally isomorphic to reduced singular homology $\widetilde{H}_n(-;G)$ with coefficients in $G$. 
\end{eg}

\begin{eg}\label{SphereSpec} We have seen that the excisive functor $X\mapsto\underset{n}{\text{hocolim}}\Omega^n\Sigma^nX$ gives rise to the stable homotopy groups. It is natural to ask what $\Omega$-spectrum it is associated to. Following the infinity categorical equivalence between excisive functors and spectra, we can try and evaluating the functor at the $k$-spheres for each $k$. It does not quite give an $\Omega$-spectrum because there are no weak equivalences between the sequence of spaces. However, referring to section 2.4 of \cite{FSHT}, one can take fibrant replacements to obtain an associated $\Omega$-spectrum. 

Now evaluating the excisive approximation at the $k$-spheres give $$\underset{n}{\text{hocolim}}\Omega^n\Sigma^nS^k\cong\underset{n}{\text{hocolim}}\Omega^n(S^k\wedge S^n)$$ One can recognize using \cite{FSHT} that this is precisely the fibrant replacement of the sphere spectrum as described in the reference. 
\end{eg}
