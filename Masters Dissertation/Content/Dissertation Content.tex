\documentclass[a4paper]{article}

%=========================================
% Packages
%=========================================
\usepackage{mathtools}
\usepackage{amsfonts}
\usepackage{amsmath}
\usepackage{amssymb}
\usepackage{amsthm}
\usepackage[a4paper, total={6in, 8in}, margin=1in]{geometry}
\usepackage[utf8]{inputenc}
\usepackage{fancyhdr}
\usepackage[utf8]{inputenc}
\usepackage{graphicx}
\usepackage{physics}
\usepackage[listings]{tcolorbox}
\usepackage{hyperref}
\usepackage{tikz-cd}
\usepackage{adjustbox}
\usepackage{enumitem}
\usepackage[font=small,labelfont=bf]{caption}
\usepackage{subcaption}
\usepackage{wrapfig}
\usepackage{makecell}



\raggedright

\usetikzlibrary{arrows.meta}

\DeclarePairedDelimiter\ceil{\lceil}{\rceil}
\DeclarePairedDelimiter\floor{\lfloor}{\rfloor}

%=========================================
% Fonts
%=========================================
\usepackage{tgpagella}
\usepackage[T1]{fontenc}


%=========================================
% Custom Math Operators
%=========================================
\DeclareMathOperator{\adj}{adj}
\DeclareMathOperator{\im}{im}
\DeclareMathOperator{\nullity}{nullity}
\DeclareMathOperator{\sign}{sign}
\DeclareMathOperator{\dom}{dom}
\DeclareMathOperator{\lcm}{lcm}
\DeclareMathOperator{\ran}{ran}
\DeclareMathOperator{\ext}{Ext}
\DeclareMathOperator{\dist}{dist}
\DeclareMathOperator{\diam}{diam}
\DeclareMathOperator{\aut}{Aut}
\DeclareMathOperator{\inn}{Inn}
\DeclareMathOperator{\syl}{Syl}
\DeclareMathOperator{\edo}{End}
\DeclareMathOperator{\cov}{Cov}
\DeclareMathOperator{\vari}{Var}
\DeclareMathOperator{\cha}{char}
\DeclareMathOperator{\Span}{span}
\DeclareMathOperator{\ord}{ord}
\DeclareMathOperator{\res}{res}
\DeclareMathOperator{\Hom}{Hom}
\DeclareMathOperator{\Mor}{Mor}
\DeclareMathOperator{\coker}{coker}
\DeclareMathOperator{\Obj}{Obj}
\DeclareMathOperator{\id}{id}
\DeclareMathOperator{\GL}{GL}
\DeclareMathOperator*{\colim}{colim}

%=========================================
% Custom Commands (Shortcuts)
%=========================================
\newcommand{\CP}{\mathbb{CP}}
\newcommand{\GG}{\mathbb{G}}
\newcommand{\F}{\mathbb{F}}
\newcommand{\N}{\mathbb{N}}
\newcommand{\Q}{\mathbb{Q}}
\newcommand{\R}{\mathbb{R}}
\newcommand{\C}{\mathbb{C}}
\newcommand{\E}{\mathbb{E}}
\newcommand{\Prj}{\mathbb{P}}
\newcommand{\RP}{\mathbb{RP}}
\newcommand{\T}{\mathbb{T}}
\newcommand{\Z}{\mathbb{Z}}
\newcommand{\A}{\mathbb{A}}
\renewcommand{\H}{\mathbb{H}}
\newcommand{\K}{\mathbb{K}}

\newcommand{\mA}{\mathcal{A}}
\newcommand{\mB}{\mathcal{B}}
\newcommand{\mC}{\mathcal{C}}
\newcommand{\mD}{\mathcal{D}}
\newcommand{\mE}{\mathcal{E}}
\newcommand{\mF}{\mathcal{F}}
\newcommand{\mG}{\mathcal{G}}
\newcommand{\mH}{\mathcal{H}}
\newcommand{\mI}{\mathcal{I}}
\newcommand{\mJ}{\mathcal{J}}
\newcommand{\mK}{\mathcal{K}}
\newcommand{\mL}{\mathcal{L}}
\newcommand{\mM}{\mathcal{M}}
\newcommand{\mO}{\mathcal{O}}
\newcommand{\mP}{\mathcal{P}}
\newcommand{\mS}{\mathcal{S}}
\newcommand{\mT}{\mathcal{T}}
\newcommand{\mV}{\mathcal{V}}
\newcommand{\mW}{\mathcal{W}}

%=========================================
% Colours!!!
%=========================================
\definecolor{LightBlue}{HTML}{2D64A6}
\definecolor{ForestGreen}{HTML}{4BA150}
\definecolor{DarkBlue}{HTML}{000080}
\definecolor{LightPurple}{HTML}{cc99ff}
\definecolor{LightOrange}{HTML}{ffc34d}
\definecolor{Buff}{HTML}{DDAE7E}
\definecolor{Sunset}{HTML}{F2C57C}
\definecolor{Wenge}{HTML}{584B53}
\definecolor{Coolgray}{HTML}{9098CB}
\definecolor{Lavender}{HTML}{D6E3F8}
\definecolor{Glaucous}{HTML}{828BC4}
\definecolor{Mauve}{HTML}{C7A8F0}
\definecolor{Darkred}{HTML}{880808}
\definecolor{Beaver}{HTML}{9A8873}
\definecolor{UltraViolet}{HTML}{52489C}



%=========================================
% Theorem Environment
%=========================================
\tcbuselibrary{listings, theorems, breakable, skins}

\newtcbtheorem[number within = subsection]{thm}{Theorem}%
{	colback=Buff!3, 
	colframe=Buff, 
	fonttitle=\bfseries, 
	breakable, 
	enhanced jigsaw, 
	halign=left
}{thm}

\newtcbtheorem[number within=subsection, use counter from=thm]{defn}{Definition}%
{  colback=cyan!1,
    colframe=cyan!50!black,
	fonttitle=\bfseries, breakable, 
	enhanced jigsaw, 
	halign=left
}{defn}

\newtcbtheorem[number within=subsection, use counter from=thm]{axm}{Axiom}%
{	colback=red!5, 
	colframe=Darkred, 
	fonttitle=\bfseries, 
	breakable, 
	enhanced jigsaw, 
	halign=left
}{axm}

\newtcbtheorem[number within=subsection, use counter from=thm]{prp}{Proposition}%
{	colback=LightBlue!3, 
	colframe=Glaucous, 
	fonttitle=\bfseries, 
	breakable, 
	enhanced jigsaw, 
	halign=left
}{prp}

\newtcbtheorem[number within=subsection, use counter from=thm]{lmm}{Lemma}%
{	colback=LightBlue!3, 
	colframe=LightBlue!60, 
	fonttitle=\bfseries, 
	breakable, 
	enhanced jigsaw, 
	halign=left
}{lmm}

\newtcbtheorem[number within=subsection, use counter from=thm]{crl}{Corollary}%
{	colback=LightBlue!3, 
	colframe=LightBlue!60, 
	fonttitle=\bfseries, 
	breakable, 
	enhanced jigsaw, 
	halign=left
}{crl}

\newtcbtheorem[number within=subsection, use counter from=thm]{eg}{Example}%
{	colback=Beaver!5, 
	colframe=Beaver, 
	fonttitle=\bfseries, 
	breakable, 
	enhanced jigsaw, 
	halign=left
}{eg}

\newtcbtheorem[number within=subsection, use counter from=thm]{ex}{Exercise}%
{	colback=Beaver!5, 
	colframe=Beaver, 
	fonttitle=\bfseries, 
	breakable, 
	enhanced jigsaw, 
	halign=left
}{ex}

\newtcbtheorem[number within=subsection, use counter from=thm]{alg}{Algorithm}%
{	colback=UltraViolet!5, 
	colframe=UltraViolet, 
	fonttitle=\bfseries, 
	breakable, 
	enhanced jigsaw, 
	halign=left
}{alg}




%=========================================
% Hyperlinks
%=========================================
\hypersetup{
    colorlinks=true, %set true if you want colored links
    linktoc=all,     %set to all if you want both sections and subsections linked
    linkcolor=DarkBlue,  %choose some color if you want links to stand out
}


\pagestyle{fancy}
\fancyhf{}
\rhead{Labix}
\lhead{Selected Topics}
\rfoot{\thepage}

\title{Selected Topics}

\author{Labix}

\date{\today}
\begin{document}
\maketitle
\begin{abstract}
\end{abstract}

\pagebreak
\tableofcontents

\pagebreak
\section{Excisive Functors between Spaces}
\subsection{Homotopy Pushouts and Homotopy Pullbacks}
Why we want this: pushouts dont preserve homotopies, as with any limits / colimts (therefore we have homotopy limits / colimits in model category)

\begin{defn}{Standard Model for Homotopy Pushouts}{}
\end{defn}

\begin{defn}{Standard Model for Homotopy Pullbacks}{}
\end{defn}

\begin{defn}{Homotopy Pushouts}{}
\end{defn}

\begin{defn}{Homotopy Pullbacks}{}
\end{defn}

\begin{eg}{}{} Suspension and loopspace. 
\end{eg}

\begin{prp}{}{}
\end{prp}

\begin{defn}{Excisive Functors}{}
\end{defn}

\subsection{The Failure of the Identity Functor to be Excisive}

\subsection{Excisive Functors Coming From Spectra}

\pagebreak
\section{Spectra as Reduced and Excisive Functors}
\subsection{Stable Infinity Categories}
\begin{defn}{Infinity Pushouts}{} Let $\mC$ be an infinity category. Let $F:\Delta^1\times\Delta^1\to\mC$ be a morphism of simplicial sets. Let $X\in\mC$ be an object. We say that $X$ is a pushout in $\mC$ if there exists a natural transformation $u:\Delta X\Rightarrow F$ such that there is a homotopy equivalence of Kan complexes: 
\end{defn}

\begin{defn}{Infinity Pullbacks}{}
\end{defn}

Why are these the correct analogue? 

\begin{defn}{Stable Infinity Categories}{}
\end{defn}

Example in mind: spectra in ordinary categories: pushout=pullback. 

\begin{defn}{Excisive Functors}{}
\end{defn}

\subsection{Suspension and Loop Functors}
Own notes: Higher algebra 1.4

trivial kan fibration -> section (Kerodon 1.5.5.5)

\subsection{Stable Infinity Categories}

Recall that $\mS=N_\bullet^{\text{hc}}(\bold{Top}_\ast)$ is the infinity category of spaces. 

\begin{prp}{}{} Let $\mC$ be a pointed infinity category that admits all finite colimits. Then $\text{Exc}_\ast(\mC,\mS)$ is stable. \tcbline
\begin{proof}
Let $F:\mC\to\mS$ be excisive and reduced. Then $\Sigma_{\text{Exc}_\ast(\mC,\mS)}(F)=F\circ\Sigma_\mC$. By definition of the suspension functor, \\~\\
\adjustbox{scale=1,center}{\begin{tikzcd}
	X & \ast \\
	\ast & {\Sigma_\mC(X)}
	\arrow[from=1-1, to=1-2]
	\arrow[from=1-1, to=2-1]
	\arrow[from=1-2, to=2-2]
	\arrow[from=2-1, to=2-2]
\end{tikzcd}}\\~\\
is a pushout in $\mC$. Since $F$ is excisive, \\~\\
\adjustbox{scale=1,center}{\begin{tikzcd}
	F(X) & \ast \\
	\ast & {(F\circ\Sigma_\mC)(X)}
	\arrow[from=1-1, to=1-2]
	\arrow[from=1-1, to=2-1]
	\arrow[from=1-2, to=2-2]
	\arrow[from=2-1, to=2-2]
\end{tikzcd}}\\~\\
is a pullback in $\mS$. On the other hand, $\Omega_{\text{Exc}_\ast(\mC,\mS)}(F)=\Omega_\mS\circ F$. By definition of the loop functor, \\~\\
\adjustbox{scale=1,center}{\begin{tikzcd}
	{(\Omega_\mS\circ F\circ\Sigma_\mC)(X)} & \ast \\
	\ast & {(F\circ\Sigma_\mC)(X)}
	\arrow[from=1-1, to=1-2]
	\arrow[from=1-1, to=2-1]
	\arrow[from=1-2, to=2-2]
	\arrow[from=2-1, to=2-2]
\end{tikzcd}}\\~\\
is a pullback in $\mS$ for any $X\in\mC$. Therefore $F(X)$ and $(\Omega_\mS\circ F\circ\Sigma_\mC)(X)$ are equivalent. Hence $F$ and $\Omega_{\text{Exc}_\ast(\mC,\mS)}(\Sigma_{\text{Exc}_\ast(\mC,\mS)}(F))$ are equivalent. 
\end{proof}
\end{prp}

\begin{thm}{}{} There is an equivalence of infinity categories $$\text{Sp}(\mS)\simeq\lim(\cdots\rightarrow\mS\overset{\Omega}{\rightarrow}\mS\overset{\Omega}{\rightarrow}\mS)=:\overline{\mS}$$ induced by the evaluation map $\text{ev}_{S^0}:\overline{\mS}\to\mS$. \tcbline
\begin{proof}~\\
Since $\mS$ is presentable and the infinity category of presentable infinity categories admit all small limits, $\overline{\mS}$ is also presentable. Every presentable infinity category admits all small limits and colimits. Since $\mS$ is pointed, $\overline{\mS}$ is also pointed. Since all limits are computed term-wise, we have that in particular $\Omega_{\overline{\mS}}$ is computed term wise. given $\{X_n\;|\;n\in\N\}$ an object of $\overline{\mS}$, $\{\Omega X_n\;|\;n\in\N\}$ is equivalent to $\{X_n\;|\;n\in\N\}$ because we have that $\Omega X_{n+1}$ is equivalent to $X_n$ for all $n$. By a prp we conclude that $\overline{\mS}$ is stable. \\~\\

Consider the canonical functor $G:\overline{\mS}\to\mS$ defined by recovering the first factor: $(X_0,X_1,\dots)\mapsto X_0$. It is clear that it commutes with finite limits since limits are computed term-wise. \\~\\

Let $\mC$ be an arbitrary stable infinity category. Any functor $\mC\to\mS$ is left exact if and only if it is exact so that $\text{Exc}_\ast(\mC,\mS)=\text{Exc}_\ast^\text{L}(\mC,\mS)$. 1.4.2.16 implies that $\text{Exc}_\ast^\text{L}(\mC,\mS)$ is a stable infinity category. Thus $\Omega_\mS\circ -$ is an equivalence. \\~\\

On the other hand, since $\Omega$ are computed term-wise (like all limits) and since $\text{Func}(\mC,\overline{\mS})$ is right adjoint to products we know that $\text{Func}$ commutes with finite limits . Thus we have that $$\text{Exc}_\ast^\text{L}(\mC,\overline{\mS})=\lim(\cdots\rightarrow\text{Exc}_\ast^{\text{L}}(\mC,\mS)\overset{\Omega\circ-}{\rightarrow}\text{Exc}_\ast^{\text{L}}(\mC,\mS)\overset{\Omega\circ-}{\rightarrow}\text{Exc}_\ast^{\text{L}}(\mC,\mS))$$ Since each $\Omega_{\overline{\mS}}\circ -$ is an equivalence of infinity categories, we conclude that $\text{Exc}_\ast^{\text{L}}(\mC,\overline{\mS})\simeq\text{Exc}_\ast^{\text{L}}(\mC,\mS)$. Thus evaluation on the first factor $G\circ -:\text{Exc}_\ast^{\text{L}}(\mC,\overline{\mS})\to\text{Exc}_\ast^{\text{L}}(\mC,\mS)$ is an equivalence of infinity categories. \\~\\

By a previous corollary, there is an equivalence of infinity categories given by $$\Omega^\infty\circ-:\text{Exc}_\ast^\text{L}(\overline{\mS},\text{Sp}(\mS))\to\text{Exc}_\ast^{\text{L}}(\overline{\mS},\mS)$$ The fact that $G$ is left exact means that there is a factorization \\~\\
\adjustbox{scale=1,center}{\begin{tikzcd}
	{\overline{\mS}} && \mS \\
	& {\text{Sp}(\mS)}
	\arrow["G", from=1-1, to=1-3]
	\arrow["{G'\circ -}"', from=1-1, to=2-2]
	\arrow["{\Omega^\infty}"', from=2-2, to=1-3]
\end{tikzcd}}\\~\\
By functoriality we obtain a similar factorization: \\~\\
\adjustbox{scale=1,center}{\begin{tikzcd}
	{\text{Exc}_\ast^{\text{L}}(\mC,\overline{\mS})} && {\text{Exc}_\ast^{\text{L}}(\mC,\mS)} \\
	& {\text{Exc}_\ast^{\text{L}}(\mC,\text{Sp}(\mS))}
	\arrow["{G\circ -}", from=1-1, to=1-3]
	\arrow["{G'\circ -}"', from=1-1, to=2-2]
	\arrow["{\Omega^\infty\circ -}"', from=2-2, to=1-3]
\end{tikzcd}}\\~\\
Since $G\circ -$ and $\Omega^\infty\circ -$ are both equivalence of infinity categories, we conclude that $G'\circ -$ is an equivalence of infinity categories. \\~\\

Since this is true for all stable infinity categories, the fact that $$\text{Exc}_\ast(\mC,\overline{\mS})=\text{Exc}_\ast^L(\mC,\overline{\mS})\simeq\text{Exc}_\ast^L(\mC,\text{Sp}(\mS))=\text{Exc}_\ast(\mC,\text{Sp}(\mS))$$ is an equivalence for all stable $\mC$ together with the Yoneda embedding implies that $\overline{\mS}$ and $\text{Sp}(\mS)$ is an equivalence of infinity categories. 
\end{proof}
\end{thm}

Beware that in the proof we also showed that $G\circ -$ is an equivalence of infinity categories for any stable infinity category $\mC$. But this does not imply that $\overline{\mS}$ and $\mS$ are equivalent because we are applying the Yoneda embedding on the category of stable infinity categories, and a priori $\mS$ is not stable. 

\pagebreak
\section{From Functors to Excisive Functors}
\subsection{Goodwillie Calculus}
\begin{defn}{}{} T1 and P1
\end{defn}

\begin{thm}{}{} P1 is excisive. 
\end{thm}

\subsection{Excisive Approximations}
\begin{eg}{}{} Id -> Infinite loop suspension
\end{eg}

\subsection{Spectra and (Co)Homology Theories}
\begin{thm}{Brown's Representability Theorem}{}
\end{thm}

\begin{defn}{Cohomology Theory Associated to Spectra}{}
\end{defn}

\begin{defn}{Spectra Associated to Cohomology Theory}{}
\end{defn}

\begin{eg}{Singular Cohomology}{}
\end{eg}

\begin{eg}{K theory}{}
\end{eg}

\begin{eg}{Landweber-exact Spectra}{}
\end{eg}

\begin{thm}{Landweber exact functor theorem}{}
\end{thm}

\subsection{A Map From Functors to (Co)Homology Theories}
\begin{eg}{}{} Identity Functor -> stable homotopy theory (it is a homology theory)
\end{eg}

\begin{eg}{}{} Excisive functor F -> {F(Sn)} -> corresponding cohomolog theory
\end{eg}






\end{document}