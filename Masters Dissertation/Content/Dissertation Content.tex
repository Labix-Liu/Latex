\documentclass[a4paper]{article}

%=========================================
% Packages
%=========================================
\usepackage{mathtools}
\usepackage{amsfonts}
\usepackage{amsmath}
\usepackage{amssymb}
\usepackage{amsthm}
\usepackage[a4paper, total={6in, 8in}, margin=1in]{geometry}
\usepackage[utf8]{inputenc}
\usepackage{fancyhdr}
\usepackage[utf8]{inputenc}
\usepackage{graphicx}
\usepackage{physics}
\usepackage[listings]{tcolorbox}
\usepackage{hyperref}
\usepackage{tikz-cd}
\usepackage{adjustbox}
\usepackage{enumitem}
\usepackage[font=small,labelfont=bf]{caption}
\usepackage{subcaption}
\usepackage{wrapfig}
\usepackage{makecell}



\raggedright

\usetikzlibrary{arrows.meta}

\DeclarePairedDelimiter\ceil{\lceil}{\rceil}
\DeclarePairedDelimiter\floor{\lfloor}{\rfloor}

%=========================================
% Fonts
%=========================================
\usepackage{tgpagella}
\usepackage[T1]{fontenc}


%=========================================
% Custom Math Operators
%=========================================
\DeclareMathOperator{\adj}{adj}
\DeclareMathOperator{\im}{im}
\DeclareMathOperator{\nullity}{nullity}
\DeclareMathOperator{\sign}{sign}
\DeclareMathOperator{\dom}{dom}
\DeclareMathOperator{\lcm}{lcm}
\DeclareMathOperator{\ran}{ran}
\DeclareMathOperator{\ext}{Ext}
\DeclareMathOperator{\dist}{dist}
\DeclareMathOperator{\diam}{diam}
\DeclareMathOperator{\aut}{Aut}
\DeclareMathOperator{\inn}{Inn}
\DeclareMathOperator{\syl}{Syl}
\DeclareMathOperator{\edo}{End}
\DeclareMathOperator{\cov}{Cov}
\DeclareMathOperator{\vari}{Var}
\DeclareMathOperator{\cha}{char}
\DeclareMathOperator{\Span}{span}
\DeclareMathOperator{\ord}{ord}
\DeclareMathOperator{\res}{res}
\DeclareMathOperator{\Hom}{Hom}
\DeclareMathOperator{\Mor}{Mor}
\DeclareMathOperator{\coker}{coker}
\DeclareMathOperator{\Obj}{Obj}
\DeclareMathOperator{\id}{id}
\DeclareMathOperator{\GL}{GL}
\DeclareMathOperator*{\colim}{colim}

%=========================================
% Custom Commands (Shortcuts)
%=========================================
\newcommand{\CP}{\mathbb{CP}}
\newcommand{\GG}{\mathbb{G}}
\newcommand{\F}{\mathbb{F}}
\newcommand{\N}{\mathbb{N}}
\newcommand{\Q}{\mathbb{Q}}
\newcommand{\R}{\mathbb{R}}
\newcommand{\C}{\mathbb{C}}
\newcommand{\E}{\mathbb{E}}
\newcommand{\Prj}{\mathbb{P}}
\newcommand{\RP}{\mathbb{RP}}
\newcommand{\T}{\mathbb{T}}
\newcommand{\Z}{\mathbb{Z}}
\newcommand{\A}{\mathbb{A}}
\renewcommand{\H}{\mathbb{H}}
\newcommand{\K}{\mathbb{K}}

\newcommand{\mA}{\mathcal{A}}
\newcommand{\mB}{\mathcal{B}}
\newcommand{\mC}{\mathcal{C}}
\newcommand{\mD}{\mathcal{D}}
\newcommand{\mE}{\mathcal{E}}
\newcommand{\mF}{\mathcal{F}}
\newcommand{\mG}{\mathcal{G}}
\newcommand{\mH}{\mathcal{H}}
\newcommand{\mI}{\mathcal{I}}
\newcommand{\mJ}{\mathcal{J}}
\newcommand{\mK}{\mathcal{K}}
\newcommand{\mL}{\mathcal{L}}
\newcommand{\mM}{\mathcal{M}}
\newcommand{\mO}{\mathcal{O}}
\newcommand{\mP}{\mathcal{P}}
\newcommand{\mS}{\mathcal{S}}
\newcommand{\mT}{\mathcal{T}}
\newcommand{\mV}{\mathcal{V}}
\newcommand{\mW}{\mathcal{W}}

%=========================================
% Colours!!!
%=========================================
\definecolor{LightBlue}{HTML}{2D64A6}
\definecolor{ForestGreen}{HTML}{4BA150}
\definecolor{DarkBlue}{HTML}{000080}
\definecolor{LightPurple}{HTML}{cc99ff}
\definecolor{LightOrange}{HTML}{ffc34d}
\definecolor{Buff}{HTML}{DDAE7E}
\definecolor{Sunset}{HTML}{F2C57C}
\definecolor{Wenge}{HTML}{584B53}
\definecolor{Coolgray}{HTML}{9098CB}
\definecolor{Lavender}{HTML}{D6E3F8}
\definecolor{Glaucous}{HTML}{828BC4}
\definecolor{Mauve}{HTML}{C7A8F0}
\definecolor{Darkred}{HTML}{880808}
\definecolor{Beaver}{HTML}{9A8873}
\definecolor{UltraViolet}{HTML}{52489C}



%=========================================
% Theorem Environment
%=========================================
\tcbuselibrary{listings, theorems, breakable, skins}

\newtcbtheorem[number within = subsection]{thm}{Theorem}%
{	colback=Buff!3, 
	colframe=Buff, 
	fonttitle=\bfseries, 
	breakable, 
	enhanced jigsaw, 
	halign=left
}{thm}

\newtcbtheorem[number within=subsection, use counter from=thm]{defn}{Definition}%
{  colback=cyan!1,
    colframe=cyan!50!black,
	fonttitle=\bfseries, breakable, 
	enhanced jigsaw, 
	halign=left
}{defn}

\newtcbtheorem[number within=subsection, use counter from=thm]{axm}{Axiom}%
{	colback=red!5, 
	colframe=Darkred, 
	fonttitle=\bfseries, 
	breakable, 
	enhanced jigsaw, 
	halign=left
}{axm}

\newtcbtheorem[number within=subsection, use counter from=thm]{prp}{Proposition}%
{	colback=LightBlue!3, 
	colframe=Glaucous, 
	fonttitle=\bfseries, 
	breakable, 
	enhanced jigsaw, 
	halign=left
}{prp}

\newtcbtheorem[number within=subsection, use counter from=thm]{lmm}{Lemma}%
{	colback=LightBlue!3, 
	colframe=LightBlue!60, 
	fonttitle=\bfseries, 
	breakable, 
	enhanced jigsaw, 
	halign=left
}{lmm}

\newtcbtheorem[number within=subsection, use counter from=thm]{crl}{Corollary}%
{	colback=LightBlue!3, 
	colframe=LightBlue!60, 
	fonttitle=\bfseries, 
	breakable, 
	enhanced jigsaw, 
	halign=left
}{crl}

\newtcbtheorem[number within=subsection, use counter from=thm]{eg}{Example}%
{	colback=Beaver!5, 
	colframe=Beaver, 
	fonttitle=\bfseries, 
	breakable, 
	enhanced jigsaw, 
	halign=left
}{eg}

\newtcbtheorem[number within=subsection, use counter from=thm]{ex}{Exercise}%
{	colback=Beaver!5, 
	colframe=Beaver, 
	fonttitle=\bfseries, 
	breakable, 
	enhanced jigsaw, 
	halign=left
}{ex}

\newtcbtheorem[number within=subsection, use counter from=thm]{alg}{Algorithm}%
{	colback=UltraViolet!5, 
	colframe=UltraViolet, 
	fonttitle=\bfseries, 
	breakable, 
	enhanced jigsaw, 
	halign=left
}{alg}




%=========================================
% Hyperlinks
%=========================================
\hypersetup{
    colorlinks=true, %set true if you want colored links
    linktoc=all,     %set to all if you want both sections and subsections linked
    linkcolor=DarkBlue,  %choose some color if you want links to stand out
}


\pagestyle{fancy}
\fancyhf{}
\rhead{Labix}
\lhead{Dissertation Content}
\rfoot{\thepage}

\title{Dissertation Content}

\author{Labix}

\date{\today}
\begin{document}
\maketitle
\begin{abstract}
\end{abstract}

\pagebreak
\tableofcontents

\pagebreak
\section{Excisive Functors between Spaces}
\subsection{Homotopy Pushouts and Homotopy Pullbacks}
Work in category CGWH
all spaces pointed
$\Sigma$= reduced suspension
Cubical: all spaces assumed well-pointed (1.1.5 cubical) (Cw complex are well based) (same as non-degenerate base point)
homotopy colimits should be as 8.2.14 (since well based then unpointed homotopy pullbacks is homotopy equivalent to pointed homotopy pullback )
suspension refers to reduced suspension (of based spaces) (for well based spaces reduced suspension is homotopy equivalent to unreduced suspension)
all categorical limits / colimits (such as products) are assumed to be k-ified. (equipped with CGWH topology. 

Let me begin by slightly rephrasing what we learned in classical algebraic topology. Recall that if $X$ is a space and $X=A\cup B$ are two open sets, then \\~\\
\adjustbox{scale=1.0,center}{\begin{tikzcd}
	{A\cap B} & A \\
	B & X
	\arrow[hook, from=1-1, to=1-2]
	\arrow[hook, from=1-1, to=2-1]
	\arrow[hook, from=1-2, to=2-2]
	\arrow[hook, from=2-1, to=2-2]
\end{tikzcd}}\\~\\
is a pushout in $\bold{Spaces}$. 
\begin{itemize}
\item Recall that amalgamated products of groups is precisely the pushout in $\bold{Grps}$ (when the two maps are embeddings). The Seifert-Van Kampen theorem then says that if $A\cap B$ is non-empty, then $\pi_1:\bold{Spaces}\to\bold{Grp}$ sends pushouts to pushouts. 
\item If we upgrade the fundamental group $\pi_1$ into the fundamental groupoid $\Pi_1$ developed by Brown, again $\Pi_1:\bold{Spaces}\to\bold{Spaces}\to\bold{Grpoids}$ sends pushouts to pushouts, without the assumption of connectedness of $A\cap B$, and removing the dependence of the base point. (6.7.2 R.Brown)
\item Applying the singular homology functor does not give a pushout, nor a pullback, but gives a long exact sequence in homology which is useful for computation. 
\item Similarly, applying the singular cohomology functor gives a useful computational skill. 
\end{itemize}

When we say that $X$ is a pushout of $A$ and $B$ along $A\cap B$, we mean that $X$ is built by the piece $A$ and $B$. The Mayer-Vietoris theorem is the prime example of why we care about this. It says that computing the homology of parts of the space can recover homological information of the overall space. This is similar for the Seifert-van Kampen theorem. But in general, homotopy groups do not enjoy any theorems reminiscent to these properties, unless we are given (co)fibrations in which case there are indeed long exact sequences. \\

It is natural to look for conditions to relax so that we obtain similar results. The approach that algebraic topologists takes considers the following viewpoint: Most of the invariants in algebraic topology detect differences up to homotopy equivalence, but pushouts and pullbacks are more rigid than homotopy equivalence: 

\begin{itemize}
\item Their universal properties guarantee that they are unique up to homeomorphism. 
\item Pushouts and pullbacks are determined by three spaces and two maps. But if we supply homotopy equivalent spaces then the pushout / pullback is not homotopy equivalence. 
\end{itemize}

This means that pushouts and pullbacks (and in general arbitrary limts and colimits!) are quite incompatible with homotopies. 

\begin{eg}{}{} Consider the following commutative diagram:  \\~\\
\adjustbox{scale=1.0,center}{\begin{tikzcd}
	\ast & {S^n} & {D^{n+1}} \\
	\ast & {S^n} & \ast
	\arrow[from=1-1, to=2-1]
	\arrow[from=1-2, to=1-1]
	\arrow[hook, from=1-2, to=1-3]
	\arrow["{\text{id}}"', from=1-2, to=2-2]
	\arrow["\ast", from=1-3, to=2-3]
	\arrow[from=2-2, to=2-1]
	\arrow[from=2-2, to=2-3]
\end{tikzcd}}\\~\\
While all the vertical arrows are weak equivalences, the induced map of pushouts is given by $S^{n+1}\to\ast$ which is clearly not a weak equivalence. (dduggar0 (refer)
\end{eg}

Therefore we concern ourselves with a homotopy invariant version of this concept. The ordinary pushouts and pullbacks are unique up to homeomorphism. We can explicitly define a set and its topology and show that it satisfies a universal property. We take a similar approach here and first introduce a model for homotopy pushouts. 

\begin{defn}{The Standard Homotopy Pushout}{} Let $X,Y,Z\in\bold{CGWH}$ be spaces. Let $f:Z\to X$ and $g:Z\to Y$ be maps. Define the standard homotopy pushout of $f$ and $g$ to be the quotient space $$\text{hocolim}(X\overset{f}{\leftarrow}Z\overset{g}{\rightarrow}Y)=\frac{X\amalg(Z\times I)\amalg Y}{\sim}$$ where $\sim$ is the equivalence relation generated by $f(z)\sim (z,0)$ and $g(z)\sim(z,1)$ for $z\in Z$. 
\end{defn}

Remember that we want our new pushout to be a homotopy invariant, not a homeomorphic invariant. Any homeomorphic spaces has one unique way of writing it down set theoretically up to bijection, but homotopy equivalent spaces are not determined by its set theoretic representation. Given an arbitrary square in $\bold{Spaces}$, we also want to know how to compare whether the square exhibits a homotopy pushout. In particular, we use the standard model as an anchor for comparison. 

\begin{defn}{The Canonical Map of Homotopy Pushouts}{} Let $X,Y,Z\in\bold{CGWH}$ be spaces. Let $f:Z\to X$ and $g:Z\to Y$ be maps. Define the canonical map of the homotopy pushout of the diagram to be the map $$s:\text{hocolim}(X\overset{f}{\leftarrow}Z\overset{g}{\rightarrow}Y)\to\text{colim}(X\overset{f}{\leftarrow}Z\overset{g}{\rightarrow}Y)$$ given by the formula $$u\mapsto\begin{cases}
u & \text{ if }u\in X\\
f(z)=g(z) & \text{ if }u=(z,t)\in Z\times I\\
u & \text{ if }u\in Y
\end{cases}$$
\end{defn}

\begin{defn}{The Standard Homotopy Pullback}{} Let $X,Y,Z\in\bold{CGWH}$ be spaces. Let $f:X\to Z$ and $g:Y\to Z$ be maps. Define the standard homotopy pullback of $f$ and $g$ to be the subspace $$\text{holim}(X\overset{f}{\rightarrow}Z\overset{g}{\leftarrow}Y)=\{(x,\alpha,y)\in X\times\text{Map}(I,Z)\times Y\;|\;\alpha(0)=f(x),\alpha(1)=g(y)\}$$
\end{defn}

\begin{defn}{The Canonical Map of Homotopy Pullbacks}{} Let $X,Y,Z\in\bold{CGWH}$ be spaces. Let $f:X\to Y$ and $g:Y\to Z$ be maps. Define the canonical map from the pullback to the homotopy pullback $$c:\lim(X\overset{f}{\rightarrow}Z\overset{g}{\leftarrow}Y)\to\text{holim}(X\overset{f}{\rightarrow}Z\overset{g}{\leftarrow}Y)$$ to be given by $(x,y)\mapsto(x,e_{f(x)=g(y)},y)$ where $e$ refers to the constant loop at $f(x)=g(y)$. 
\end{defn}

\begin{prp}{}{} Let $X,Y,Z\in\bold{CGWH}$ be spaces. Let $f:X\to Z$ and $g:Y\to Z$ be maps. Then the there is a homeomorphism $$\lim(X\overset{f}{\rightarrow}Z\leftarrow P_g)\cong\text{holim}(X\overset{f}{\rightarrow}Z\overset{g}{\leftarrow}Y)$$ given by the map $(x,(y,\gamma))\mapsto(x,\gamma,y)$. \tcbline
\begin{proof}
Recall that the limit on the left is given by $\{(x,(y,\gamma))\in X\times P_g\;|\;\gamma(1)=f(x)\}$. Also $(y,\gamma)\in P_g$ means that $\gamma(0)=g(y)$. So the map defined by $(x,(y,\gamma))\mapsto(x,\gamma,y)$ is a well defined map and is clearly a homeomorphism. 
\end{proof}
\end{prp}

Notice that we could have equally replaced the space $X$ with $P_f$ to obtain a homeomorphism between the limit and homotopy pullback entirely by symmetry. \\~\\

Dually we have a homeomorphism $$\text{colim}(X\overset{f}{\leftarrow}Z\rightarrow M_g)\cong\text{hocolim}(X\overset{f}{\leftarrow}Z\overset{g}{\rightarrow}Y)$$ Moreover, since the diagram of spaces is symmetric, we could have equally replaced the space on the left by the mapping path space / mapping cylinder and we would yet again obtain a homeomorphism. 

\begin{prp}{}{} Let $X,Y,Z\in\bold{CGWH}$ be spaces. Let $f:X\to Z$ and $g:Y\to Z$ be maps. Then the there is a homeomorphism $$\lim(P_f\rightarrow Z\leftarrow P_g)\cong\text{holim}(X\overset{f}{\rightarrow}Z\overset{g}{\leftarrow}Y)$$ given by the map $((x,\gamma_1),(y,\gamma_2))\mapsto(x,\gamma_1\cdot\overline{\gamma_2},y)$. \tcbline
\begin{proof}
Notice that the map is well defined because by definition, we have that $$\lim(P_f\rightarrow Z\leftarrow P_g)=\{((x,\gamma_1),(y,\gamma_2))\in P_f\times P_g\;|\;\gamma_1(0)=f(x),\gamma_2(0)=g(y),\gamma_1(1)=\gamma_2(1)\}$$ The evident inverse of the given map is then $(x,\gamma,y)\mapsto((x,\gamma|_{[0,1/2]}),(y,\gamma_{[1/2,1]}))$. 
\end{proof}
\end{prp}

By an entirely dual argument we deduce a homeomorphism $$\text{colim}(M_f\leftarrow W\rightarrow M_g)\cong\text{hocolim}(X\overset{f}{\leftarrow}W\overset{g}{\rightarrow}Y)$$

\begin{prp}{}{} Let $X,Y,Z$ be spaces. Let $f:X\to Z$ and $g:Y\to Z$ be maps. If one of $f$ or $g$ is a fibration, then the canonical map $$\lim(X\rightarrow Z\leftarrow Y)\simeq\text{holim}(X\rightarrow Z\leftarrow Y)$$ is a homotopy equivalence. \tcbline
\begin{proof}
Without loss of generality suppose that $f$ is a fibration. Replace $g$ by a homotopy equivalence $h:Y\to P_g$ and a fibration $k:P_g\to Z$. We then obtain a commutative diagram: \\~\\
\adjustbox{scale=1.0,center}{\begin{tikzcd}
	{\text{lim}(X\rightarrow Z\leftarrow Y)} & Y \\
	{\text{holim}(X\rightarrow Z\leftarrow Y)} & {P_g} \\
	X & Z
	\arrow[from=1-1, to=1-2]
	\arrow[from=1-1, to=2-1]
	\arrow["h", from=1-2, to=2-2]
	\arrow[from=2-1, to=2-2]
	\arrow[from=2-1, to=3-1]
	\arrow["k", from=2-2, to=3-2]
	\arrow["f"', from=3-1, to=3-2]
\end{tikzcd}}\\~\\
It is commutative because the map from the limit to $X$ is precisely the projection map to $X$. Since the outer rectangle and the bottom square are pullbacks, the pasting law shows that the upper square is also a pullback. Now since $f$ is a fibration and the bottom square is a pullback, we conclude that the middle horizontal arrow is a fibration. Since $h$ is a homotopy equivalence, we conclude that the map from the limit to the homotopy limit is a homotopy equivalence. Hence we conclude. 
\end{proof}
\end{prp}

Dually, we have the following statement: If one of $f:X\to Y$ or $g:X\to Z$ is a cofibration, then we have a homotopy equivalence $$\text{colim}(Y\leftarrow X\rightarrow Z)\simeq\text{hocolim}(Y\leftarrow X\rightarrow Z)$$ given by the canonical map. 

We can compute some examples of homotopy pushouts and pullbacks. 

\begin{eg}{}{} Let $f:X\to Y$ be a map. For any $y\in Y$ recall that the homotopy fiber of $f$ is defined to be the space $$\text{hofiber}_y(f)=\{(x,\gamma)\in X\times\text{Map}(I,Y)\;|\;\gamma(0)=f(x),\gamma(1)=y\}$$ In fact, it can easily be seen that this is precisely the definition of the homotopy pullback $\text{holim}(X\overset{f}{\rightarrow}Y\leftarrow\{y_0\})$ where $y_0$ is the base point of $Y$. Also note that if we take actual limits instead of homotopy limits then the same diagram would give us the fiber of $f$, thus justifying its name. \\~\\

By replacing the space $X$ with $P_f$, we have seen a homeomorphism $$\text{hofib}_{y_0}(X\rightarrow Y\leftarrow\{y_0\})\cong\lim(P_f\rightarrow Y\leftarrow\{y_0\})=\text{fib}_{y_0}(P_f\rightarrow Y)$$ again justifying its name. \\~\\

Similarly, the homotopy cofiber of $f$ is defined to be $$\text{hocofiber}(f)=\frac{(X\times I)\amalg Y}{\sim}$$ where the relation is generated by $(x,1)\sim(x',1)$ and $(x,0)\sim f(x)$ for $x,x'\in X$. Again this is precisely the definition of $\text{hocolim}(\ast\leftarrow X\overset{f}{\rightarrow}Y)$. 
\end{eg}

The motivation behind the homotopy fiber is not dissimilar to that of homotopy pullbacks. Namely, the fibers of a map for varying $y$ are in general not homotopic, but the homotopy fibers of a map are, and are moreover invariant under a homotopy. 

\begin{eg}{}{} Let $X$ be a pointed space. There is a unique map $X\to\ast$ to the terminal object in $\bold{Spaces}$. The homotopy pushout of $\ast\leftarrow X\rightarrow\ast$ is given by $$\text{hocolim}(\ast\leftarrow X\rightarrow\ast)=\frac{\ast\amalg(X\times I)\amalg\ast}{\sim}$$ where $\sim$ is generated by $\ast\sim(x,0)$ and $\ast\sim(x,1)$. This is precisely the definition of suspension. \\~\\

Similarly, there is a unique map $\ast\to X$ sending $\ast$ to the base point of $X$. The homotopy pullback of $\ast\rightarrow X\leftarrow\ast$ is given by 
\begin{align*}
\text{holim}(\ast\rightarrow X\leftarrow\ast)&=\{(\ast,\gamma,\ast)\in\ast\times\text{Map}(I,X)\times\ast\;|\;\gamma(0)=\ast=\gamma(1)\}\\
&\cong\{\gamma\in\text{Map}(I,X)\;|\;\gamma\text{ is a loop at the base point }\}
\end{align*}
This is precisely the definition of loopspace. 
\end{eg}

\begin{eg}{}{} Let $f:X\to Y$ be a map. The homotopy pushout of $f$ and the identity map on $X$ is given by $$$$ Similarly, the homotopy pullback of $f$ and the identity map on $Y$ is given by 
\begin{align*}
\text{holim}(X\overset{f}{\rightarrow}Y\overset{\text{id}}{\leftarrow}Y)&=\{(x,\gamma,y)\in X\times\text{Map}(I,Y)\times Y\;|\;\gamma(0)=x,\gamma(1)=y\}\\
&\cong\{(x,\gamma)\in X\times\text{Map}(I,Y)\;|\;\gamma(0)=x\}\\
&=P_f
\end{align*}
Since $X$ is a deformation retract of $P_f$ (why?), we see that the homotopy pullback is homotopy equivalent to $X$. 
\end{eg}

\begin{eg}{}{} Let $X,Y$ be pointed spaces. The homotopy pushout of the initial maps $\ast\to X$ and $\ast\to Y$ is given by $$\text{hocolim}(X\leftarrow\ast\rightarrow Y)\cong X\vee Y$$ This is because the initial map are cofibrations, so that the homotopy pushout is homotopy equivalent to the ordinary limit. 
\end{eg}

\begin{eg}{}{}
\end{eg}

\begin{defn}{Homotopy Pushout Squares}{} Let $W,X,Y,Z\in\bold{CGWH}$ be spaces such that there is a homotopy commutative diagram \\~\\
\adjustbox{scale=1.0,center}{\begin{tikzcd}
	W & Y \\
	X & Z
	\arrow[from=1-1, to=1-2]
	\arrow[from=1-1, to=2-1]
	\arrow[from=1-2, to=2-2]
	\arrow[from=2-1, to=2-2]
\end{tikzcd}}\\~\\
\begin{itemize}
\item We say that the square is a homotopy pushout square if there exists a weak equivalence $h:\text{hocolim}(X\leftarrow W\rightarrow Y)\to Z$ such that the following diagram commutes up to homotopy: \\~\\
\adjustbox{scale=1.0,center}{\begin{tikzcd}
	W & Y \\
	X & {\text{hocolim}(X\leftarrow W\rightarrow Y)} \\
	&& Z
	\arrow[from=1-1, to=1-2]
	\arrow[from=1-1, to=2-1]
	\arrow[from=1-2, to=2-2]
	\arrow[bend left = 22, from=1-2, to=3-3]
	\arrow[from=2-1, to=2-2]
	\arrow[bend right = 20, from=2-1, to=3-3]
	\arrow["\exists", dashed, from=2-2, to=3-3]
\end{tikzcd}}\\~\\
\item We say that the diagram is $k$-cocartesian if the map there exists a $k$-connected map $\text{hocolim}(X\leftarrow W\rightarrow Y)\to Z$ making the same diagram commute up to homotopy. 
\end{itemize}
\end{defn}

The idea is similar to standard pushouts in the sense that if we have a commutative square \\~\\
\adjustbox{scale=1.0,center}{\begin{tikzcd}
	W & Y \\
	X & Z
	\arrow[from=1-1, to=1-2]
	\arrow[from=1-1, to=2-1]
	\arrow[from=1-2, to=2-2]
	\arrow[from=2-1, to=2-2]
\end{tikzcd}}\\~\\
then by the universal property of pushouts one would obtain a comparison map $W\to\lim(X\rightarrow Z\leftarrow Y)$. When this is a homeomorphism we call $W$ a pushout of the diagram. Similarly we define homotopy pullback squares dually. \\~\\

Notice that if the diagram is strictly commutative, there is a canonical choice of map one can consider. Namely, the map $$\beta:\text{hocolim}(X\overset{f}{\leftarrow}W\overset{g}{\rightarrow}Y)\overset{s}{\longrightarrow}\text{colim}(X\overset{f}{\leftarrow}W\overset{g}{\rightarrow}Y)\to Z$$ This is the definition given in Munson. However the notion is quite limiting since the simplest examples fail to be a homotopy pushout square. For instance, the square \\~\\
\adjustbox{scale=1.0,center}{\begin{tikzcd}
	W & Y \\
	X & {\text{hocolim}(X\leftarrow W\rightarrow Y)}
	\arrow[from=1-1, to=1-2]
	\arrow[from=1-1, to=2-1]
	\arrow[from=1-2, to=2-2]
	\arrow[from=2-1, to=2-2]
\end{tikzcd}}\\~\\

is not even an honest commutative square. 

\begin{defn}{Homotopy Pullback Squares}{} Let $W,X,Y,Z\in\bold{CGWH}$ be spaces such that there is a (not necessarily commutative) diagram \\~\\
\adjustbox{scale=1.0,center}{\begin{tikzcd}
	W & Y \\
	X & Z
	\arrow[from=1-1, to=1-2]
	\arrow[from=1-1, to=2-1]
	\arrow[from=1-2, to=2-2]
	\arrow[from=2-1, to=2-2]
\end{tikzcd}}\\~\\
\begin{itemize}
\item We say that the diagram is a homotopy pullback if the map $$\alpha:W\to\lim(X\overset{f}{\rightarrow}Z\overset{g}{\leftarrow}Y)\overset{c}{\longrightarrow}\text{holim}(X\overset{f}{\rightarrow}Z\overset{g}{\leftarrow}Y)$$ is a weak equivalence. 
\item We say that the diagram is $k$-cartesian if $\alpha$ is $k$-connected. 
\end{itemize}
\end{defn}

In the two definitions, I considered homotopy commutative square to be homotopy pushouts / homotopy pullbacks by considering the existence of a weak equivalence. One may ask to enforce the map to be a homotopy equivalence instead. While Munson only requires the map is a weak equivalence, Hurwitz and strom require them to be homotopy equivalences. Because every homotopy equivalence is a weak equivalence, using merely weak equivalences allows for a wider class of spaces to represent the same homotopy pushout / pullback. However the biggest drawback is that weak equivalences are not invertible in general, unless in specific examples such as CW complexes, in which case there is no difference between specifying the map to be a weak equivalence or a homotopy equivalence. We following the approach of Munson mainly because we would like to prove the Blakers-Massey theorem using homotopy squares. \\~\\

From a model theoretic persepective, this is precisely the difference between using Quillen's model structure on spaces or using Strom's model structure. 

\begin{prp}{}{} Let the following be a pullback square: \\~\\
\adjustbox{scale=1.0,center}{\begin{tikzcd}
	W & Y \\
	X & Z
	\arrow[from=1-1, to=1-2]
	\arrow[from=1-1, to=2-1]
	\arrow[from=1-2, to=2-2]
	\arrow[from=2-1, to=2-2]
\end{tikzcd}}\\~\\
If one of $X\to Z$ or $Y\to Z$ is a fibration, then the square is a homotopy pullback square. \tcbline
\begin{proof}
By 1.1.8, if one of $X\to Z$ or $Y\to Z$ is a fibration, then the canonical map from the limit to the homotopy limit is a homotopy equivalence. Hence we obtain a weak equivalence $W\cong\lim(X\rightarrow Z\leftarrow Y)\to\text{holim}(X\rightarrow Z\leftarrow Y)$. 
\end{proof}
\end{prp}

\subsection{Basic Properties of Homotopy Pushouts and Homotopy Pullbacks}
Recall that ordinary pullbacks are not well-suited for studying homotopy invariant concepts. The following theorem shows that homotopy pullbacks remedies the situation. 

\begin{thm}{}{} Suppose that we have a commutative diagram of spaces \\~\\
\adjustbox{scale=1.0,center}{\begin{tikzcd}
	X & Z & Y \\
	{X'} & {Z'} & {Y'}
	\arrow["f", from=1-1, to=1-2]
	\arrow["{e_X}"', from=1-1, to=2-1]
	\arrow["{e_Z}"', from=1-2, to=2-2]
	\arrow["g"', from=1-3, to=1-2]
	\arrow["{e_Y}", from=1-3, to=2-3]
	\arrow["{f'}"', from=2-1, to=2-2]
	\arrow["{g'}", from=2-3, to=2-2]
\end{tikzcd}}\\~\\
in $\bold{CGWH}$. Define the map $$\phi_{X,Z,Y}^{X',Z',Y'}:\text{holim}(X\overset{f}{\rightarrow}Z\overset{g}{\leftarrow}Y)\to\text{holim}(X'\overset{f'}{\rightarrow}Z'\overset{g'}{\leftarrow}Y')$$ by the formula $(x,\gamma,y)\mapsto(e_X(x),e_Z\circ\gamma,e_Y(y))$. Then the following are true. 
\begin{itemize}
\item If each $e_X,e_Y,e_Z$ are homotopy equivalences, then $\phi$ is a homotopy equivalence. 
\item If each $e_X,e_Y,e_Z$ are weak equivalences, then $\phi$ is a weak equivalence. 
\end{itemize} \tcbline
\begin{proof}
We first prove the case for homotopy equivalence. Consider the following commutative diagram: \\~\\
\adjustbox{scale=1.0,center}{\begin{tikzcd}
	X & Z & Y \\
	X & {Z'} & Y \\
	{X'} & {Z'} & {Y'}
	\arrow["f", from=1-1, to=1-2]
	\arrow["{\text{id}_X}"', from=1-1, to=2-1]
	\arrow["{e_Z}", from=1-2, to=2-2]
	\arrow["g"', from=1-3, to=1-2]
	\arrow["{\text{id}_Y}", from=1-3, to=2-3]
	\arrow["{e_Z\circ f}", from=2-1, to=2-2]
	\arrow["{e_X}"', from=2-1, to=3-1]
	\arrow["{\text{id}_{Z'}}", from=2-2, to=3-2]
	\arrow["{e_Z\circ g}"', from=2-3, to=2-2]
	\arrow["{e_Y}", from=2-3, to=3-3]
	\arrow["{f'}"', from=3-1, to=3-2]
	\arrow["{g'}", from=3-3, to=3-2]
\end{tikzcd}}\\~\\
We prove that the homotopy pullback of the first row is homotopy equivalent to that of the second, and we prove that the homotopy pullback of the second row is homotopy equivalent to that of the third. \\~\\

Since $e_Z$ is a homotopy equivalence, we can find a homotopy inverse $k$ for $e_Z$ and a homotopy $H:Z\times I\to Z$ such that $H(-,0)=\text{id}_Z$ and $H(-,1)=k\circ e_Z$. Define a map $$\rho:\text{holim}(X\overset{f}{\rightarrow}Z'\overset{g}{\leftarrow}Y)\to\text{holim}(X\overset{e_Z\circ f}{\rightarrow}Z\overset{e_Z\circ g}{\leftarrow}Y)$$ by the formula $$(x,\gamma',y)\mapsto(x,H(f(x),-)\ast k(\gamma'(-))\ast\overline{H(g(y),-)}:I\to Z,y)$$ where $\ast$ denotes concatenation of paths. The path concatenation is well defined because we have that $H(f(x),1)=(k\circ e_Z\circ f)(x)=(k\circ\gamma')(0)$ and $k(\gamma'(1))=k(e_Z(g(y)))=H(g(y),1)$. This is well defined on the homotopy pullback because we have that 
\begin{itemize}
\item $H(f(x),-)\ast k(\gamma'(-))\ast\overline{H(g(y),-)}(0)=H(f(x),0)=\text{id}_Z(f(x))=f(x)$
\item $H(f(x),-)\ast k(\gamma'(-))\ast\overline{H(g(y),-)}(1)=H(g(y),0)=\text{id}_Z(g(y))=g(y)$
\end{itemize}
I claim that this map is the homotopy inverse to the map $\phi=\phi_{X,Y,Z}^{X,Y,Z'}$. We have that 
\begin{align*}
\rho(\phi(x,\gamma,y))&=\rho(x,e_Z\circ\gamma,y)\\
&=(x,H(f(x),-)\ast k(e_Z(\gamma(-))\ast\overline{H(g(y),-)},y)
\end{align*}
Now I claim that the middle path is homotopic to $\gamma$. For the first component of the concatenation, the path $H(f(x),t):I\to Z$ can be contracted to $H(f(x),0)=f(x)=\gamma(0)$ so you can homotope the traversal along $H(f(x),-)$ to the single point $f(x)=\gamma(0)$. For the third component of the concatenation, this is similar so we can homotope the traversal of $\overline{H(g(y),-)}$ to the single point $g(y)=\gamma(1)$. The middle part of the path is homotopic to $\gamma$ because $k\circ e_Z$ is homotopic to $\text{id}_Z$. Thus we conclude. 
\end{proof}
\end{thm}

In the theorem we considered a commutative diagram of spaces instead of a homotopy commutative one because we imagine such a result to be applied to a map of diagrams. More specifically, if $F$ is a functor from spaces to spaces, then $F$ induces a functor from diagram of spaces to diagram of spaces, and unravelling the entire diagram from source to target gives a strictly commutative square. 

Note: there is a similar result for pushouts. \\

\begin{prp}{}{} Let $W,X,Y,Z\in\bold{CGWH}$ be spaces such that there is a commutative diagram \\~\\
\adjustbox{scale=1.0,center}{\begin{tikzcd}
	W & Y \\
	X & Z
	\arrow[from=1-1, to=1-2]
	\arrow[from=1-1, to=2-1]
	\arrow[from=1-2, to=2-2]
	\arrow[from=2-1, to=2-2]
\end{tikzcd}}\\~\\
Then the square is a homotopy pullback square if and only if for any factorization of $Y\to Z$ into a weak equivalence $Y\to M$ and a fibration $M\to Z$, the induced map $$W\to\lim(X\rightarrow Z\leftarrow M)$$ is a weak equivalence. \tcbline
\begin{proof}
Further factorize $Y\to M$ into a weak equivalence $Y\to E$ and fibration $E\to M$ so that we have a acyclic cofibration $Y\to E$ and a fibration $E\to Z$. Let $P$ be the mapping path space of $Y\to Z$. Then notice that we have the following commutative diagram: \\~\\
\adjustbox{scale=1.0,center}{\begin{tikzcd}
	Y & P \\
	E & Z
	\arrow[from=1-1, to=1-2]
	\arrow[from=1-1, to=2-1]
	\arrow[from=1-2, to=2-2]
	\arrow[from=2-1, to=2-2]
\end{tikzcd}}\\~\\
because they are both factorizations of $Y\to Z$, the diagram is commutative. Since $Y\to E$ is an acyclic cofibration and $P\to Z$ is a fibration, there exists a lift $E\to P$ making the above diagram commute. Since $Y\to P$ and $Y\to E$ are weak equivalences, $E\to P$ is also a weak equivalence. By the above theorem, $\lim(P_f\rightarrow Z\leftarrow P)\to\lim(P_f\rightarrow Z\leftarrow E)$ is a weak equivalence where $f$ is the map $X\to Z$. By the two out of three property, $W\to\lim(P_f\to Z\leftarrow P)=\text{holim}(X\rightarrow Z\leftarrow Y)$ is a weak equivalence if and only if $W\to\lim(P_f\to Z\leftarrow E)$ is a weak equivalence. \\~\\

Finally, notice that since $Y\to E\to M=Y\to M$, we know that $E\to M$ is a weak equivalence. By appendix we know that $\lim(P_f\rightarrow Z\leftarrow E)\to\lim(P_f\rightarrow Z\leftarrow M)$ is a weak equivalence. Therefore by the two out of three property, $W\to\lim(P_f\rightarrow Z\leftarrow E)$ is a weak equivalence if and only if $W\to\lim(P_f\rightarrow Z\leftarrow M)$ is a weak equivalence. This shows that $W\to\text{holim}(X\rightarrow Z\leftarrow Y)$ is a weak equivalence if and only if $W\to\lim(P_f\rightarrow Z\leftarrow M)$ is a weak equivalence. 
\end{proof}
\end{prp}

https://math.stackexchange.com/questions/1124930/doubt-about-proof-of-factorization-f-pi-where-i-is-acyclic-cofibration-and

https://arxiv.org/abs/1204.5427

\begin{prp}{}{} Suppose that the following is a commutative diagram of spaces: \\~\\
\adjustbox{scale=1.0,center}{\begin{tikzcd}
	{X_1} & {X_3} & {X_2} \\
	{Z_1} & {Z_3} & {Z_2} \\
	{Y_1} & {Y_3} & {Y_2}
	\arrow["{h_1}", from=1-1, to=1-2]
	\arrow["{f_1}"', from=1-1, to=2-1]
	\arrow["{f_3}"', from=1-2, to=2-2]
	\arrow["{h_2}"', from=1-3, to=1-2]
	\arrow["{f_2}", from=1-3, to=2-3]
	\arrow["{l_1}", from=2-1, to=2-2]
	\arrow["{l_2}"', from=2-3, to=2-2]
	\arrow["{g_1}", from=3-1, to=2-1]
	\arrow["{k_1}"', from=3-1, to=3-2]
	\arrow["{g_3}", from=3-2, to=2-2]
	\arrow["{g_2}"', from=3-3, to=2-3]
	\arrow["{k_2}", from=3-3, to=3-2]
\end{tikzcd}}\\~\\
Let $R_i$ be the homotopy pullback of the $i$th row, let $C_i$ be the homotopy pullback of the $i$th column. Then there is a homeomorphism $$\text{holim}(R_1\rightarrow R_3\leftarrow R_2)\cong\text{holim}(C_1\rightarrow C_3\leftarrow C_2)$$ \tcbline
\begin{proof}
Using the definition of the homotopy pullback we have $$\text{holim}(C_1\rightarrow C_3\leftarrow C_2)=\left\{((x_1,\alpha_1,y_1),\Gamma,(x_2,\alpha_2,y_2))\in C_1\times\text{Map}(I,C_3)\times C_2\;\bigg{|}\;\substack{\Gamma(0)=(f_1(x_1),l_1\circ\alpha_1,g_1(y_1))\\\Gamma(1)=(f_2(x_2),l_2\circ\alpha_2,g_2(y_2))}\right\}$$ Notice that the data of $\Gamma$ is equivalent to a homotopy $\Gamma:I\times I\to Z_3$, from $\Gamma(0,-):l_1\circ\alpha_1$ to $\Gamma(1,-):l_2\circ\alpha_2$. \\~\\

Similarly, we have $$\text{holim}(R_1\rightarrow R_3\leftarrow R_2)=\left\{((x_1,\beta_X,x_2),\Gamma,(y_1,\beta_Y,y_2))\in R_1\times\text{Map}(I,R_3)\times R_2\;\bigg{|}\;\substack{\Gamma(0)=(h_1(x_1),f_3\circ\beta_X,h_2(x_2))\\\Gamma(1)=(k_1(y_1),g_3\circ\beta_Y,k_2(y_2))}\right\}$$ and again the data of $\Gamma$ is equivalent to a homotopy $I\times I\to Z_3$, from $f_3\circ\beta_X$ to $g_3\circ\beta_Y$. \\~\\

Then the map $\text{holim}(C_1\rightarrow C_3\leftarrow C_2)\to\text{holim}(R_1\rightarrow R_3\leftarrow R_2)$ defined by $$((x_1,\alpha_1,y_1),\Gamma,(x_2,\alpha_2,y_2))\mapsto((x_1,\Gamma(-,0),x_2),\Gamma,(y_1,\Gamma(-,1),y_2))$$ gives the desired homeomorphism. 
\end{proof}
\end{prp}

The dual of the above is also true. Namely, if we have a commutative diagram of spaces: \\~\\
\adjustbox{scale=1.0,center}{\begin{tikzcd}
	{X_1} & {X_3} & {X_2} \\
	{Z_1} & {Z_3} & {Z_2} \\
	{Y_1} & {Y_3} & {Y_3}
	\arrow[from=1-2, to=1-1]
	\arrow[from=1-2, to=1-3]
	\arrow[from=2-1, to=1-1]
	\arrow[from=2-1, to=3-1]
	\arrow[from=2-2, to=1-2]
	\arrow[from=2-2, to=2-1]
	\arrow[from=2-2, to=2-3]
	\arrow[from=2-2, to=3-2]
	\arrow[from=2-3, to=1-3]
	\arrow[from=2-3, to=3-3]
	\arrow[from=3-2, to=3-1]
	\arrow[from=3-2, to=3-3]
\end{tikzcd}}\\~\\
Then the homotopy pushout of the homotopy pushout of the rows is homeomorphic to the homotopy pushout of the homotopy pushout of the columns. 

\begin{eg}{}{} Let $X,Y,Z,W$ be pointed spaces. Let $f:X\to Y$ and $g:X\to Z$ be maps. Then there is a homeomorphism $$W\wedge\text{hocolim}(Y\leftarrow X\rightarrow Z)\cong\text{holim}(W\wedge Y\leftarrow W\wedge X\rightarrow W\wedge Z)$$ To prove this, notice that there is a homeomorphism $$\text{holim}(W\times Y\leftarrow W\times X\rightarrow W\times Z)\cong W\times\text{hocolim}(Y\leftarrow X\rightarrow Z)$$ given by the identity map. Since $W\times V\to W\times\text{hocolim}(Y\leftarrow X\rightarrow Z)$ is a weak equivalence, we conclude that products commute with homotopy pushouts. We can also show that wedge sums commute with homotopy pushouts. To see this, consider the following diagram: \\~\\
\adjustbox{scale=1.0,center}{\begin{tikzcd}
	W & W & W \\
	\ast & \ast & \ast \\
	Y & X & Z
	\arrow[from=1-2, to=1-1]
	\arrow[from=1-2, to=1-3]
	\arrow[from=2-1, to=1-1]
	\arrow[from=2-1, to=3-1]
	\arrow[from=2-2, to=1-2]
	\arrow[from=2-2, to=2-1]
	\arrow[from=2-2, to=2-3]
	\arrow[from=2-2, to=3-2]
	\arrow[from=2-3, to=1-3]
	\arrow[from=2-3, to=3-3]
	\arrow[from=3-2, to=3-1]
	\arrow[from=3-2, to=3-3]
\end{tikzcd}}\\~\\
For each column, its homotopy pushout is precisely the wedge of $W$ and the corresponding space because the initial map $\ast\to W$ is a cofibration. Therefore the homotopy pushout of the homotopy pushout is the homotopy pushout of wedge sums. On the other hand, the homotopy pushout of the first row is $W$, the homotopy pushout of the second row is $\ast$ hence the homotopy pushout of the homotopy pushout of the rows is given by $W\vee\text{hocolim}(Y\leftarrow X\rightarrow Z)$. \\~\\

Then consider the commutative diagram: \\~\\
\adjustbox{scale=1.0,center}{\begin{tikzcd}
	{W\times Y} & {W\times X} & {W\times Z} \\
	{W\vee Y} & {W\vee X} & {W\vee Z} \\
	\ast & \ast & \ast
	\arrow[from=1-2, to=1-1]
	\arrow[from=1-2, to=1-3]
	\arrow[from=2-1, to=1-1]
	\arrow[from=2-1, to=3-1]
	\arrow[from=2-2, to=1-2]
	\arrow[from=2-2, to=2-1]
	\arrow[from=2-2, to=2-3]
	\arrow[from=2-2, to=3-2]
	\arrow[from=2-3, to=1-3]
	\arrow[from=2-3, to=3-3]
	\arrow[from=3-2, to=3-1]
	\arrow[from=3-2, to=3-3]
\end{tikzcd}}\\~\\
where the top vertical arrows are inclusions. The homotopy pushout of the columns 
\end{eg}

\begin{prp}{}{} Consider the following homotopy commutative square: \\~\\
\adjustbox{scale=1.0,center}{\begin{tikzcd}
	{X_1} & {X_2} & {X_3} \\
	{Y_1} & {Y_2} & {Y_3}
	\arrow[from=1-1, to=1-2]
	\arrow[from=1-1, to=2-1]
	\arrow[from=1-2, to=1-3]
	\arrow[from=1-2, to=2-2]
	\arrow[from=1-3, to=2-3]
	\arrow[from=2-1, to=2-2]
	\arrow[from=2-2, to=2-3]
\end{tikzcd}}\\~\\
in $\bold{CGWH}$. Let the right square be a homotopy pullback square. Then the left square is a homotopy pullback if and only if the rectangle is a homotopy pullback square. \tcbline
\begin{proof}
Factorize $X_3\to Y_3$ into a weak equivalence $X_3\to P$ and a fibration $P\to Y_3$ where $P$ is the mapping path space of the map $X_3\to Y_3$. We then obtain the following commutative diagram: \\~\\
\adjustbox{scale=1.0,center}{\begin{tikzcd}
	{X_1} & {X_2} & {X_3} \\
	{\lim(Y_1\rightarrow Y_3\leftarrow P)} & {\lim(Y_2\rightarrow Y_3\leftarrow P)} & P \\
	{Y_1} & {Y_2} & {Y_3}
	\arrow[from=1-1, to=2-1]
	\arrow[from=1-2, to=2-2]
	\arrow["\simeq", from=1-3, to=2-3]
	\arrow[from=2-1, to=2-2]
	\arrow[from=2-1, to=3-1]
	\arrow[from=2-2, to=2-3]
	\arrow[from=2-2, to=3-2]
	\arrow["{\text{fib}}", from=2-3, to=3-3]
	\arrow[from=3-1, to=3-2]
	\arrow[from=3-2, to=3-3]
\end{tikzcd}}\\~\\
where the two vertical maps on the top left is induced by the canonical map of the homotopy pullback square. Since we assume the right square in the diagram in the proposition is a homotopy pullback, we know that $X_2\to\lim(Y_2\rightarrow Y_3\leftarrow P)$ is a weak equivalence. By the pasting law of pullback squares, we know that the bottom left square is a pullback square. Since $P\to Y_3$ is a fibration and the bottom right square is a pullback square, the middle lower vertical map is a fibration. \\~\\

Notice that the map $X_2\to Y_2$ is factored into a weak equivalence $X_2\to\lim(Y_2\rightarrow Y_3\leftarrow P)$ followed by a fibration $\lim(Y_2\rightarrow Y_3\leftarrow P)\to Y_2$. Moreover by the pasting law, the bottom left square is a pullback. Therefore the left square in the original diagram is a homotopy pullback square if and only if $X_1\to\lim(Y_1\rightarrow Y_3\leftarrow P)$ is a weak equivalence. But also the same map is a weak equivalence if and only if the outer rectangle is a homotopy pullback square by definition. Hence we conclude. 
\end{proof}
\end{prp}

Kerodon 3.4

 (6.3.3 Arkhowitz)

5.72 strom. 13.3.15 hirschorn

\begin{prp}{}{} Let $W,X,Y,Z\in\bold{CGWH}$ be spaces such that there is a (not necessarily commutative) diagram \\~\\
\adjustbox{scale=1.0,center}{\begin{tikzcd}
	W & Y \\
	X & Z
	\arrow[from=1-1, to=1-2]
	\arrow[from=1-1, to=2-1]
	\arrow[from=1-2, to=2-2]
	\arrow[from=2-1, to=2-2]
\end{tikzcd}}\\~\\
Then the following are true. 
\begin{itemize}
\item If the square is a homotopy pullback, and $Y\to Z$ is a weak equivalence, then $W\to X$ is a weak equivalence. 
\item If $Y\to Z$ and $W\to X$ are weak equivalence, then the square is a homotopy pullback. 
\end{itemize} \tcbline
\begin{proof}
The crucial square to consider is the following pullback square: \\~\\
\adjustbox{scale=1.0,center}{\begin{tikzcd}
	\text{holim}(X\rightarrow Z\leftarrow Y) & Y \\
	P_f & Z
	\arrow[from=1-1, to=1-2]
	\arrow[from=1-1, to=2-1]
	\arrow[from=1-2, to=2-2]
	\arrow[from=2-1, to=2-2]
\end{tikzcd}}\\~\\
where we replace the map $X\to Z$ via the map $X\to P_f\to Z$ so that the first map is an equivalence and the second map is a fibration. Since $Y\to Z$ is a weak equivalence and $P_f\to Z$ is a fibration, the above implies that $\text{holim}(X\rightarrow Z\leftarrow Y)\to P_f$ is a weak equivalence. Since this map is given by the composition $\text{holim}(X\rightarrow Z\leftarrow Y)\to X\to P_f$ and $X\to P_f$ is a weak equivalence, the two out of three property gives that $\text{holim}(X\rightarrow Z\leftarrow Y)\to X$ is a weak equivalence. Since we have a weak equivalence $W\to\text{holim}(X\rightarrow Z\leftarrow Y)$, we again apply the two out of three property so that the composition $W\to X$ is a weak equivalence. \\~\\

Conversely, if $Y\to Z$ and $W\to X$ are weak equivalence, then the same reasoning shows that $\text{holim}(X\rightarrow Z\leftarrow Y)\to P_f$ is a weak equivalence and hence $\text{holim}(X\rightarrow Z\leftarrow Y)\to X$ is a weak equivalence. Since $W\to X$ is also a weak equivalence, by the two out of three property we conclude that $W\to\text{holim}(X\rightarrow Z\leftarrow Y)$ is a weak equivalence and so the square in question is a homotopy pullback. 
\end{proof}
\end{prp}

\begin{prp}{}{} Let $W,X,Y,Z\in\bold{CGWH}$ be spaces such that following is a (not necessarily commutative) square \\~\\
\adjustbox{scale=1.0,center}{\begin{tikzcd}
	W & Y \\
	X & Z
	\arrow[from=1-1, to=1-2]
	\arrow[from=1-1, to=2-1]
	\arrow[from=1-2, to=2-2]
	\arrow["f", from=2-1, to=2-2]
\end{tikzcd}}\\~\\
The square is a homotopy pullback if and only if for all $x\in X$, the map $$\text{hofiber}_x(W\to X)\to\text{hofiber}_{f(x)}(Y\to Z)$$ is a weak equivalence. \tcbline
\begin{proof}
Begin with the homotopy pullback square \\~\\
\adjustbox{scale=1.0,center}{\begin{tikzcd}
	{\text{holim}(X\rightarrow Z\leftarrow Y)} & Y \\
	X & Z
	\arrow[from=1-1, to=1-2]
	\arrow[from=1-1, to=2-1]
	\arrow[from=1-2, to=2-2]
	\arrow[from=2-1, to=2-2]
\end{tikzcd}}\\~\\
We know that there is a homotopy pullback square given by \\~\\
\adjustbox{scale=1.0,center}{\begin{tikzcd}
	{\text{hofib}_{f(x)}(g)} && Y \\
	\ast & X & Z
	\arrow[from=1-1, to=1-3]
	\arrow[from=1-1, to=2-1]
	\arrow["g", from=1-3, to=2-3]
	\arrow[from=2-1, to=2-2]
	\arrow[from=2-2, to=2-3]
\end{tikzcd}}\\~\\
We can view this as a homotopy pullback square where we consider the composition $\text{holim}_y(g)\to\ast\to X$ as one single map. The gives a comparison of a homotopy pullback square with the standard homotopy pullback. Hence we obtain \\~\\
\adjustbox{scale=1.0,center}{\begin{tikzcd}
	{\text{hofib}_{f(x)}(g)} & {\text{holim}(X\rightarrow Z\leftarrow Y)} & Y \\
	\ast & X & Z
	\arrow[from=1-1, to=1-2]
	\arrow[from=1-1, to=2-1]
	\arrow[from=1-2, to=1-3]
	\arrow[from=1-2, to=2-2]
	\arrow["g", from=1-3, to=2-3]
	\arrow[from=2-1, to=2-2]
	\arrow[from=2-2, to=2-3]
\end{tikzcd}}\\~\\
By the above prp, we conclude that the square on the left is a homotopy pullback. By the same method, we can glue two more squares to obtain the homotopy commutative diagram: \\~\\
\adjustbox{scale=1.0,center}{\begin{tikzcd}
	{\text{hofib}_{u}(\alpha)} & {\text{hofib}_{x}(f)} & W \\
	\ast & {\text{hofib}_{f(x)}(g)} & {\text{holim}(X\rightarrow Z\leftarrow Y)} & Y \\
	& \ast & X & Z
	\arrow[from=1-1, to=1-2]
	\arrow[from=1-1, to=2-1]
	\arrow[from=1-2, to=1-3]
	\arrow["u", from=1-2, to=2-2]
	\arrow["\alpha", from=1-3, to=2-3]
	\arrow[from=2-1, to=2-2]
	\arrow[from=2-2, to=2-3]
	\arrow[from=2-2, to=3-2]
	\arrow[from=2-3, to=2-4]
	\arrow[from=2-3, to=3-3]
	\arrow["g", from=2-4, to=3-4]
	\arrow[from=3-2, to=3-3]
	\arrow[from=3-3, to=3-4]
\end{tikzcd}}\\~\\
for any point $u$ in the homotopy limit. If $\alpha$ is a weak equivalence, then since the top wide rectangle is a homotopy pullback we have $\text{holim}_u(\alpha)$ is weakly equivalent to $\ast$. But the top left square is a homotopy pullback hence $\text{holim}_x(f)$ is weakly equivalent to $\text{holim}_{f(x)}(g)$. Conversely, suppose that $u$ is a weak equivalence for all $x$. Since the top left square is a homotopy pullback, this implies that $\text{holim}_u(\alpha)$ is weakly contractible for all $u$. In articular it is $n$-connected for all $n$. Then this implies that $\alpha$ is $(n+1)$-connected for all $n$. Hence $\alpha$ is a weak equivalence. 
\end{proof}
\end{prp}

\begin{prp}{}{} Let $W,X,Y,Z\in\bold{CGWH}$ be spaces such that there is a homotopy pushout square \\~\\
\adjustbox{scale=1.0,center}{\begin{tikzcd}
	W & Y \\
	X & Z
	\arrow[from=1-1, to=1-2]
	\arrow[from=1-1, to=2-1]
	\arrow[from=1-2, to=2-2]
	\arrow[from=2-1, to=2-2]
\end{tikzcd}}\\~\\
Then there exists a pushout square \\~\\
\adjustbox{scale=1.0,center}{\begin{tikzcd}
	W' & Y' \\
	X' & Z'
	\arrow[from=1-1, to=1-2]
	\arrow[from=1-1, to=2-1]
	\arrow[from=1-2, to=2-2]
	\arrow[from=2-1, to=2-2]
\end{tikzcd}}\\~\\
such that $W'\to X'$, $W'\to Y'$, $\text{colim}(X'\leftarrow W'\rightarrow Y')\to Z$ are cofibrations, and a map of squares \\~\\
\adjustbox{scale=1.0,center}{\begin{tikzcd}
	{W'} && {Y'} \\
	& W && Y \\
	{X'} && {Z'} \\
	& X && Z
	\arrow[from=1-1, to=1-3]
	\arrow[from=1-1, to=2-2]
	\arrow[from=1-1, to=3-1]
	\arrow[from=1-3, to=2-4]
	\arrow[from=1-3, to=3-3]
	\arrow[from=2-2, to=2-4]
	\arrow[from=2-2, to=4-2]
	\arrow[from=2-4, to=4-4]
	\arrow[from=3-1, to=3-3]
	\arrow[from=3-1, to=4-2]
	\arrow[from=3-3, to=4-4]
	\arrow[from=4-2, to=4-4]
\end{tikzcd}}\\~\\
where the maps $W'\to W$, $X'\to X$, $Y'\to Y$, $Z'\to Z$ are homotopy equivalences. \tcbline
\begin{proof}
Take $W'=W$, $Y'$ the mapping cylinder of $W\to Y$ and $X'$ the mapping cylinder of $W\to X$. Take $Z'$ to be the mapping cylinder of $$\text{colim}(M_{W\to X}\leftarrow W\rightarrow M_{W\to Y})\cong\text{holim}(X\leftarrow W\rightarrow Y)\to Z$$ It is clear that $W'\to X'$ and $W'\to Y'$ are cofibrations. Moreover, $\text{colim}(X'\leftarrow W'\rightarrow Y')\to Z'$ is also a cofibration is a same reasoning. 
\end{proof}
\end{prp}

\begin{defn}{Excisive Functors}{} Let $F:\bold{Spaces}\to\bold{Spaces}$ be a functor. We say that $F$ is excisive if the following are true. 
\begin{itemize}
\item $F$ is a homotopy functor. This means that if $f$ is a weak equivalence, then $F(f)$ is a weak equivalence. 
\item $F$ is finitary. This means that if $I$ is a filtered category and $X:I\to\bold{Spaces}$ is a diagram, then $$\underset{i\in I}{\text{hocolim}}F(X_i)\to F\left(\underset{i\in I}{\text{hocolim}}X_i\right)$$ is a weak equivalence. 
\item $F$ sends homotopy pushouts to homotopy pullbacks. 
\end{itemize}
\end{defn}

The finitary requirement ensures that the functor $F$ is determined by its value on a finite complexes. Then by taking homotopy colimits the value of $F$ on spaces generated by colimits of finite complexes are determined. We will say something about this condition again in section 2. \\~\\

A reasonable question might be to ask why excisive functors send homotopy pushouts to homotopy pullbacks instead to homotopy pushouts. Intuitively, pushouts are how we assemble spaces using smaller pieces. But interestingly enough, homotopy groups work better with homotopy pullbacks rather than homotopy pushouts. 

\begin{thm}{}{} LES of homotopy groups
https://math.stackexchange.com/questions/1262049/long-exact-sequence-of-homotopy-groups-pi-n-for-a-pointed-homotopy-pullback-s
5.6.9 Martin Arkowitz
\end{thm}

\subsection{Comparing Homotopy Pullback Squares}
\begin{prp}{}{} Let the following be a diagram of spaces: \\~\\
\adjustbox{scale=1.0,center}{\begin{tikzcd}
	{W_1} && {X_1} \\
	& {W_2} && {X_2} \\
	{Y_1} && {Z_1} \\
	& {Y_2} && {Z_2}
	\arrow[from=1-1, to=1-3]
	\arrow["{f_W}", from=1-1, to=2-2]
	\arrow[from=1-1, to=3-1]
	\arrow["{f_X}", from=1-3, to=2-4]
	\arrow[from=1-3, to=3-3]
	\arrow[from=2-2, to=2-4]
	\arrow[from=2-2, to=4-2]
	\arrow[from=2-4, to=4-4]
	\arrow[from=3-1, to=3-3]
	\arrow["{f_Y}", from=3-1, to=4-2]
	\arrow["{f_Z}", from=3-3, to=4-4]
	\arrow[from=4-2, to=4-4]
\end{tikzcd}}\\~\\
If $f_W,f_X,f_Y,f_Z$ are weak equivalences, then the back square is a homotopy pullback square if and only if the front square is a homotopy pullback square. \tcbline
\begin{proof}
Let $P_1$ be the standard homotopy pullback of the back square and let $P_2$ be the standard homotopy pullback of the front square. By 1.2.1 we know that $P_1\to P_2$ is a weak equivalence. Then we have the following commutative diagram: \\~\\
\adjustbox{scale=1.0,center}{\begin{tikzcd}
	{W_1} & {P_1} \\
	{W_2} & {P_2}
	\arrow[from=1-1, to=1-2]
	\arrow["{f_W}"', from=1-1, to=2-1]
	\arrow[from=1-2, to=2-2]
	\arrow[from=2-1, to=2-2]
\end{tikzcd}}\\~\\
It is then clear that $W_1\to P_1$ is a weak equivalence if and only if $W_2\to P_2$ is a weak equivalence.  
\end{proof}
\end{prp}

\subsection{The Failure of the Identity Functor to be Excisive}
It is natural to ask what kinds of functors are excisive. Instead of giving an example of an excisive functor, allow me to give an example of a functor that is not excisive. There are two reasons for this. The main theorem - Blakers-Massey theorem will assist us in giving a plethora of excisive functors. Moreover, it demonstrates the possibly simplest functor in existence - the identity functor is not excisive. \\~\\

Blakers-Massey theorem itself is a powerful theorem that implies the connectivity bounds in the Freudenthal suspension theorem. Before the main theorem, we must present preparatory definitions and lemmas, and then prove a special case of the theorem, and finally generalize it to the arbitrary case. Let me first give the statement of the main theorem. 

\begin{thm}{Blakers-Massey Theorem}{} Let $X_0,X_1,X_2,X_{12}\in\bold{CGWH}$ be spaces such that the square \\~\\
\adjustbox{scale=1.0,center}{\begin{tikzcd}
	X_0 & X_1 \\
	X_2 & X_{12}
	\arrow[from=1-1, to=1-2]
	\arrow[from=1-1, to=2-1]
	\arrow[from=1-2, to=2-2]
	\arrow[from=2-1, to=2-2]
\end{tikzcd}}\\~\\
is a homotopy pushout. Suppose the map $X_0\to X_i$ is $k_i$-connected for $i=1,2$. Then the diagram is $(k_1+k_2-1)$-cartesian. Explicitly, this means that $$\alpha:X_0\to\text{holim}(X_1\rightarrow X_{12}\leftarrow X_2)$$ is $(k_1+k_2-1)$-connected. 
\end{thm}

Firstly, recall that a square is a homotopy pullback if the map $\alpha$ above is $n$-connected for all $n\in\N$. The theorem implies that the identity functor is not excisive because after giving connectedness assumptions on maps in the square, we only get an upper bound of the connectedness of $\alpha$, which means that the identity functor is excisive up to a certain dimension. 

\begin{defn}{(Degenerative) Cubes}{} Let $a=(a_1,\dots,a_n)\in\R^n$. Let $\delta>0$. Let $L\subseteq\{1,\dots,n\}$. A cube in $\R^n$ is a set of the form $$W=W(a,\delta,L)=\{x\in\R^n\;|\;a_i\leq x\leq a_i+\delta\text{ for }i\in L\text{ and }x_i=a_i\text{ for }i\notin L\}$$
\end{defn}

The notation is making the object more complicated than what it should look like. $a\in\R^n$ refers to the bottom left coordinate of the cube. $\delta$ is the length of the cube and $L$ refers to the number of non-degenerate faces of the cube. In particular, any cube in $\R^n$ is homeomorphic to the standard cube $I^k$ for some $k\leq n$. 
\begin{itemize}
\item When $n=3$, $W(0,1,\{1,2\})$ is the unit square on the $xy$-plane. 
\item When $n=3$, $W(0,1,\{1,2,3\})$ is the unit cube. 
\item When $n=4$, $W(0,1,\{1,2,3\})$ is the unit cube with nonzero first three coordinates and zero otherwise. 
\end{itemize}

\begin{defn}{Special Sub-cube of a Cube}{} Let $W=W(a,\delta,L)$ be a cube in $\R^n$. Let $j=1$ or $2$. Suppose that $1\leq p\leq\abs{L}$. Define $$K_p^j(W)=\left\{(x_1,\dots,x_n)\in W\;\bigg{|}\;\frac{\delta(j-1)}{2}+a_i<x_i<\frac{\delta j}{2}+a_i\text{ for at least }p\text{ values of }i\in L\right\}$$
\end{defn}

Again the notation is making the object more complicated. Taking $W=W(0,1,\{1,2,3\})=I^3$ in $\R^3$, we have 
\begin{itemize}
\item $K_3^1(W)=W(0,1/2,\{1,2,3\})$ is one eighth of the cube $I^3$ with bottom left corner at the origin. 
\item $K_3^2(W)=W((1/2,1/2,1/2),1/2,\{1,2,3\})$ is one eighth of the cube $I^3$ with bottom left corner at $(1/2,1/2,1/2)$
\item $K_2^1(W)$ allows for one coordinate to go beyond the bottom left one eighth of the cube, and is the union four of the 1/8-sub-cubes that are adjacent to the $xy$-plane, the $yz$-plane and the $xz$-plane. 
\item $K_1^1(W)$ allows for two coordinate to go beyond the bottom left one eighth of the cube, and is equal to $W\setminus K_3^2(W)$. 
\item $K_0^1(W)$ allows for all coordinate to go beyond the bottom left one eighth of the cube, so the condition becomes vacuous and is equal to $W$. 
\end{itemize}

Summarizing, we think of $K_p^j(W)$ as follows. Subdivide the $\abs{L}$-dimensional cube into $2^\abs{L}$ sub-cubes of equal volume. $K_p^1$ is the union of a number of sub-cubes closest to the bottom left sub-cube. $K_p^2$ is the union of a number of sub-cubes closed to the upper right sub-cube. 

\begin{lmm}{}{} Let $Y$ be a space. Let $B\subseteq Y$ be a subspace of $Y$. Let $W=W(a,\partial, L)$ be a cube in $\R^n$. Let $f:W\to Y$ be a map. Let $j=1$ or $2$. Suppose that there exists some $p\leq\abs{L}$ such that $$f^{-1}(B)\cap C\subset K_p^j(C)$$ for all cubes $C\subset\partial W$. Then there exists a map $g:W\to Y$ such that $g\overset{\partial W}{\simeq} f$ and $$g^{-1}(B)\subset K_p^j(W)$$ \tcbline
\begin{proof}
(Proof by Munson in Cubical Homotopy Theory)Firstly, notice that any cube $W$ is homeomorphic to $I^n$ for some $n$, so we can just prove the statement for when $W=I^n$. In this case, our parameters of the cube is given by $I^n=W(a=0,\delta=1,L=\{1,\dots,n\})$ and our $K_p^j(W)$ is given by $$K_p^j(W)=\left\{(x_1,\dots,x_n)\in C\;\bigg{|}\;\frac{j-1}{2}<x_i<\frac{j}{2}\text{ for at least }p\text{ values of }i\in\{1,\dots,n\}\right\}$$~\\

Let $p_j$ be the center of the sub-cube $\left[\frac{j-1}{2},\frac{ji}{2}\right]^n$ inside $I^n$ for $j=1,2$. Let $R$ be a ray with starting point $p_j$. Let $P(R,p_j)$ be the intersection of $R$ and $\partial\left[\frac{j-1}{2},\frac{ji}{2}\right]^n$. Let $Q(R,p_j)$ be the intersection of $R$ and $\partial I^n$. By construction, the points $p_j$, $P(R,p_j)$ and $Q(R,p_j)$ are collinear with $P(R,p_j)$ always being the mid point and $P(R,p_j)$ is possibly equal to $Q(R,p_j)$. Being a line, we can define a linear homotopy from the line $[p_j,P(R,p_j)]$ to the line $[p_j,Q(R,p_j)]$ that fixes the point $p_j$ and sends $P(R,p_j)$ to $Q(R,p_j)$. Denote the homotopy by $h(y,t)$ for $y\in[p_j,P(R,p_j)]$ and $t$ the time variable. \\~\\

Now we can define a homotopy $H_j:I^n\times I\to I^n$ as follows: For each $y\in I^n$, there exists a unique ray $R$ starting at $p_j$ and passing through $y$. Then we obtain a homotopy $h$ from $[p_j,P(R,p_j)]$ to $[p_j,Q(R,p_j)]$ as above. Define $H(y,t)=h(y,t)$. It is clear that $H_j(q,1)=q$ for all $q\in\partial I^n$ so that $H_j$ is a homotopy from the identity, relative to the boundary $\partial I^n$. \\~\\

Let $g=f\circ H_j(-,1)$. From the properties of the homotopy $H_j$, we notice that $f\circ H_j:I^n\times I\to Y$ is a homotopy from $f\circ H(-,0)=f\circ\text{id}=f$ to $f\circ H(-,1)=g$, relative to the boundary $\partial I^n$. Thus we now have a homotopy from $f$ to a map $g$ relative to the boundary. It remains to show that $g^{-1}(B)\subset K_p^j(C)$. \\~\\

Let $z=(z_1,\dots,z_n)\in g^{-1}(B)$. If $z\in\left[\frac{j-1}{2},\frac{j}{2}\right]^n$ then clearly $z\in K_p^j(C)$ is true. So suppose instead that $z=(z_1,\dots,z_n)\in g^{-1}(B)$ satisfies the fact that either $z_a\geq\frac{j}{2}$ or $z_b\leq\frac{j-1}{2}$ for $1\leq a,b\leq n$. Let $R$ be the ray from $p_j$ passing through $z$. Then the condition on $z$ means that $z\in[P(R,p_j),Q(R,p_j)]$. Hence under the homotopy $H$, $z$ is mapped to $\partial I^n$. But $\partial I^n$ is a union $n-1$ dimensional faces of $I^n$ which are cubes. So $H(z,1)$ lies in some cube $C\subseteq\partial I^n$. By construction of $g$, $g(z)=f(H(z,1))$ and $g(z)\in B$ implies that $H(z,1)\in f^{-1}(B)$. Then $H(z,1)\in f^{-1}(B)$ and $H(z,1)\in C$ implies that $$H(z,1)\in f^{-1}(B)\cap C\subseteq K_p^j(C)$$ by the assumption on $f$. Write $H(z,1)=(w_1,\dots,w_n)\in\partial I^n$. This means that $\frac{j-1}{2}<w_i<\frac{j}{2}$ for at least $p$ of the coordinates of $H(z,1)$. \\~\\

Now the ray starting at $p_j$ and passing through $z$ is parametrized by the line $p_j+t(z-p_j)$ for $t\geq 0$. Since $H(z,1)$ lies behind the two points $z$ and $p_j$, we can write $H(z,1)=p_j-t_0(z-p_j)$ for some $t_0\geq 1$. By definition, $p_j$ is the point given in coordinates by $\left(\frac{2j-1}{4},\dots,\frac{2j-1}{4}\right)$. Hence the $i$th coordinate of $H(z,1)$ can be written as $$w_i=\frac{2j-1}{4}+t_0\left(z_i-\frac{2j-1}{4}\right)$$ Recall that in the previous paragraph we found that $\frac{j-1}{2}<w_i<\frac{j}{2}$ for at least $p$ of the coordinates of $H(z,1)$. Substituting $w_i$ into the inequality and simplifying gives $$-\frac{1}{4t_0}+\frac{2j-1}{4}<z_i<\frac{1}{4t_0}+\frac{2j-1}{4}$$ Since $t_0\geq 1$, we get $$-\frac{1}{4}+\frac{2j-1}{4}<-\frac{1}{4t_0}+\frac{2j-1}{4}<z_i<\frac{1}{4t_0}+\frac{2j-1}{4}<\frac{1}{4}+\frac{2j-1}{4}$$ The leftmost and rightmost terms bound $z_i$ between $\frac{j-1}{2}$ and $\frac{j}{2}$ for at least $p$ amount of coordinates $z_i$ of $z$. Hence $z\in K_p^j(C)$. This completes the proof. 
\end{proof}
\end{lmm}

\begin{lmm}{}{} Let $X$ be a space. Let $X_0,X_1,X_2\subseteq X$ be subspaces of $X$ such that $$X=X_1\cup X_2$$ and $X_0=X_1\cap X_2$ is non-empty. Assume that for each $i=1,2$, $(X_i,X_0)$ is $k_i$-connected with $k_i\geq 0$. Let $f:I^n\to X$ be a map. Let $$I^n=\bigcup_k W_k$$ be the decomposition of $I^n$ into cubes $W_k$ such that $f(W_k)\subseteq X_i$ for one of $i=0,1,2$ by the Lebesgue covering lemma. Then there exists a homotopy $$H:I^n\times I\to X$$ such that the following are true. 
\begin{itemize}
\item $f(-)=H(-,0)$
\item If $f(W)\subset X_i$, then $H(W,t)\subset X_i$ for all $t\in I$. 
\item If $f(W)\subset X_0$, then $H(W,t)=f(W)$ for all $t\in I$. 
\item If $f(W)\subset X_i$, then $\left((H(-,1))^{-1}(X_i\setminus X_0)\right)\cap W\subset K_{k_i+1}^i(W)$. 
\end{itemize} \tcbline
\begin{proof}
Let $C^d$ be the union of all cubes of dimension $\leq d$. We induct on $d$, the existence of such a homotopy $H:C^d\times I\to X$ that holds the required conditions true for all cubes $W$ with dimension $\leq d$. \\~\\

We first construct the homotopy for all cubes of dimension $0$. When $\dim(W)=0$, there are two cases: 
\begin{itemize}
\item If $f(W)\subset X_0$, define $H|_{W\times I}$ by $H(w,t)=f(w)$
\item If $f(W)\subset X_j$ and $f(W)\not\subset X_i$ for $1\leq i\neq j\leq 2$, $(X_j,X_0)$ is $(k_j\geq 0)$-connected implies that there exists a path $\gamma:I\to X$ from $f(W)$ to a point in $X_0$. Define $H|_{W\times I}$ by $H(w,t)=\gamma(t)$ (again $W=\{w\}$ is a one point set). 
\end{itemize}
Thus we now have a well defined map $H:C^0\times I\to X$. We need to show that this map satisfies the required conditions. 
\begin{itemize}
\item For each $z\in C^0$, either $H(z,0)=f(z)$ from the first case or $H(z,0)=\gamma(0)=f(z)$. 
\item If $f(W)\subset X_i$, then by construction $H(W,t)\subset X_i$ from the second case. 
\item If $f(W)\subset X_0$, then $H(W,t)=f(W)$ by the first case. 
\item $K_{k_i+1}^i(W)=\{w\}$ is a one point set and $(H(-,1))^{-1}(X_i\setminus X_0)\cap W\subseteq W$ means that this condition is satisfied. 
\end{itemize}~\\

Now $H$ is built on three pieces: the union of cubes landing in $X_i$ for $i=0,1,2$. The second and third conditions guarantee that each of the three pieces define a homotopy on each piece respectively. Since $\partial W\hookrightarrow W$ is a cofibration, we can extend these pieces of homotopy from $0$-dimensional cubes to $1$-dimensional. Recursively we are able to define a homotopy for all cubes of all dimensions inside $I^n$ that satisfy the first three conditions. \\~\\

Therefore now we can invoke the inductive hypothesis, so that there exists a homotopy from $f$ so that the new function satisfy all our required conditions for all cubes of dimension $<d$. With abuse of notation, call the restriction of our newly acquired function to $C^{d-1}$ also by the name $f$. Let $W$ be a cube of dimension $d$. 
\begin{itemize}
\item If $f(W)\subseteq X_0$, define $H|_{W\times I}$ by $H(w,t)=f(w)$
\item If $f(W)\subset X_1$ and $f(W)\not\subset X_2$ and $\dim(W)=d\leq k_1$, $(X_j,X_0)$ is $(k_j\geq 0)$-connected implies there exists a homotopy $K:W\times I\to X$ from $f$ relative to $\partial W$ such that $K(W,1)\subseteq X_0$. Define $H|_{W\times I}$ by $H=K$. 
\item If $f(W)\subset X_1$ and $f(W)\not\subset X_2$ and $\dim(W)=d>k_1$, then by induction we have $$f^{-1}(X_1\setminus X_0)\cap W'\subset K_d^1(W')\subset K_{k_1+1}^1(W')$$ for all $W'\subset\partial W$ (induction is applicable since $\dim(W')<\dim(W)$). By the above lemma, there exists a map $g:W\to X$ such that $g$ and $f$ are homotopic relative to $\partial W$ such that $g^{-1}(X_1\setminus X_0)\subset K_{k_1+1}^1(W)$. Call this homotopy from $f$ to $g$ by $R:W\times I\to X$. Then we define $H|_{W\times I}$ by $H=R$. 
\end{itemize}
(WLOG the cases where $X_1$ is swapped with $X_2$ and $k_1$ is swapped with $k_2$ and $K_p^1(W)$ is swapped with $K_p^2(W)$ in the last two sub-cases has a symmetrical argument). Finally we show that our required conditions are satisfied. 
\begin{itemize}
\item In all cases, $H(-,0)=f$ as one can immediately see. 
\item The second condition holds for all cubes of dimension $<d$ by inductive hypothesis. It also holds for our first and second case since $X_0\subset X_1,X_2$. For the third case, $H$ is a homotopy relative to $\partial W$. Since by induction hypothesis $f(\partial W)\subseteq X_1$, we also have $g(\partial W)\subseteq X_1$. Since $g$ is continuous then $H(W,1)=g(W)\subseteq X_1$. 
\item The third condition holds for all cubes of dimension $<d$ by inductive hypothesis, and holds true for all cubes of dimension $d$ by the first case. 
\item The fourth condition holds true by our argument in the third case, and is vacuously true in the second case since $H(W,1)\subseteq X_0$ implies that $(H(-,1))^{-1}(X_1\setminus X_0)=\emptyset$. 
\end{itemize}
Thus the proof is complete. 
\end{proof}
\end{lmm}

We can now prove a weaker version of Blakers-Massey theorem. 

\begin{prp}{}{} Let $X$ be a space. Let $e^{d_i}$ be a cell of dimension $d_i$ for $i=1,2$. Then the following diagram \\~\\
\adjustbox{scale=1.0,center}{\begin{tikzcd}
	X & X\cup e^{d_1} \\
	X\cup e^{d_2} & X\cup e^{d_1}\cup e^{d_2}
	\arrow[from=1-1, to=1-2]
	\arrow[from=1-1, to=2-1]
	\arrow[from=1-2, to=2-2]
	\arrow[from=2-1, to=2-2]
\end{tikzcd}}\\~\\
given by inclusion maps is $(d_1+d_2-3)$-cartesian. \tcbline
\begin{proof}
(Proof is by Munson in Cubical Homotopy Theory) Let $p_1\in e^{d_1}$ and $p_2\in e^{d_2}$ be interior points. Since $X\cup e^{d_2}$ is weakly equivalent to $X\cup e^{d_1}\cup e^{d_2}\setminus\{p_1\}$ by inclusion (and similarly for $X\cup e^{d_1}$), the above square admits a weak equivalence to the following square: \\~\\
\adjustbox{scale=1.0,center}{\begin{tikzcd}
	{X\cup e^{d_1}\cup e^{d_2}\setminus\{p_1,p_2\}} & {X\cup e^{d_1}\cup e^{d_2}\setminus\{p_2\}} \\
	{X\cup e^{d_1}\cup e^{d_2}\setminus\{p_1\}} & {X\cup e^{d_1}\cup e^{d_2}}
	\arrow[hook, from=1-1, to=1-2]
	\arrow[hook, from=1-1, to=2-1]
	\arrow[hook, from=1-2, to=2-2]
	\arrow[hook, from=2-1, to=2-2]
\end{tikzcd}}\\~\\
Let $Y=X\cup e^{d_1}\cup e^{d_2}$. By thm???? to show that the square is $(d_1+d_2-3)$-cartesian is the same as showing the map $$\text{hofiber}_y(Y\setminus\{p_1,p_2\}\to Y\setminus\{p_1\})\to\text{hofiber}_y(Y\setminus\{p_2\}\to Y)$$ is $(d_1+d_2-3)$-cartesian. Let $$C=\text{hofiber}_y(Y\setminus\{p_2\}\to Y\setminus\{p_2\})\simeq\ast$$ 

Now the intersections below can be calculated to give $$C\cap\text{hofiber}_y(Y\setminus\{p_1,p_2\}\to Y\setminus\{p_1\})=\text{hofiber}_y(Y\setminus\{p_1,p_2\}\to Y\setminus\{p_1,p_2\})$$ and $$C\cap\text{hofiber}_y(Y\setminus\{p_2\}\to Y)=\text{hofiber}_y(Y\setminus\{p_2\}\to Y\setminus\{p_2\})\simeq\ast$$ (because $C$ and the homotopy fibers are all subspaces of $Y\times\text{Map}(I,Y)$). By thm??? we conclude that the square \\~\\
\adjustbox{scale=1.0,center}{\begin{tikzcd}
	{C\cap\text{hofiber}_y(Y\setminus\{p_1,p_2\}\to Y\setminus\{p_1\})} & {\text{hofiber}_y(Y\setminus\{p_1,p_2\}\to Y\setminus\{p_1\})} \\
	C & {C\cup\text{hofiber}_y(Y\setminus\{p_1,p_2\}\to Y\setminus\{p_1\})}
	\arrow[hook, from=1-1, to=1-2]
	\arrow[hook, from=1-1, to=2-1]
	\arrow[hook, from=1-2, to=2-2]
	\arrow[hook, from=2-1, to=2-2]
\end{tikzcd}}\\~\\
is a homotopy pullback. Since the left two terms are contractible, it is a weak equivalence. By thm??? we conclude that the map on the right is a weak equivalence. Similarly, we can consider the diagram \\~\\
\adjustbox{scale=1.0,center}{\begin{tikzcd}
	{C\cap\text{hofiber}_y(Y\setminus\{p_2\}\to Y)} & {\text{hofiber}_y(Y\setminus\{p_2\}\to Y)} \\
	C & {C\cup\text{hofiber}_y(Y\setminus\{p_2\}\to Y)}
	\arrow[hook, from=1-1, to=1-2]
	\arrow[hook, from=1-1, to=2-1]
	\arrow[hook, from=1-2, to=2-2]
	\arrow[hook, from=2-1, to=2-2]
\end{tikzcd}}\\~\\
which is a homotopy pullback. And the fact that the terms on the left are contractible imply that the map on the right (which is in fact equal) $$\text{hofiber}_y(Y\setminus\{p_2\}\to Y)\overset{=}{\hookrightarrow}\text{hofiber}_y(Y\setminus\{p_2\}\to Y)$$ is a weak equivalence by the same theorem. We now have a chain of maps \\~\\
\adjustbox{scale=1.0,center}{\begin{tikzcd}
	{\text{hofiber}_y(Y\setminus\{p_1,p_2\}\to Y\setminus\{p_1\})} & {C\cup\text{hofiber}_y(Y\setminus\{p_1,p_2\}\to Y\setminus\{p_1\})} \\
	\\
	{C\cup \text{hofiber}_y(Y\setminus\{p_2\}\to Y)} & {\text{hofiber}_y(Y\setminus\{p_2\}\to Y)}
	\arrow["{\text{weak eq.}}", hook, from=1-1, to=1-2]
	\arrow[hook, from=1-2, to=3-1]
	\arrow["{\text{weak eq.}}", from=3-1, to=3-2]
\end{tikzcd}}\\~\\
where the middle map is inclusion. It remains to show that the middle map is $(d_1+d_2-3)$-connected. This is the same as showing that the pair $$(C\cup \text{hofiber}_y(Y\setminus\{p_2\}\to Y),C\cup\text{hofiber}_y(Y\setminus\{p_1,p_2\}\to Y\setminus\{p_1\}))$$ is $(d_1+d_2-3)$-connected. \\~\\

To simplify notations let us write the pair as $(A,B)$. Let $\phi:(I^n,\partial I^n)\to(A,B)$ be a map. Recall that $A=\{(x,\phi)\in Y\times\text{Map}(I,Y)\;|\;\phi(0)=y,\phi(1)=x\}$. The first variable is determined by the end point of $\phi$ so giving a map $I^n\to A$ is the same as giving a map $I^n\to\text{Map}(I,Y)$ for which all paths in the image has starting point $y$ and ending point in $Y\setminus\{p_2\}$. By the hom-product adjunction, this is equivalent to giving a map $\psi:I^n\times I\to Y$ such that $\psi(z,0)=y$ is the base point and $\psi(z,1)\in Y\setminus\{p_2\}$. Similarly, we can consider the map $\phi:\partial I^n\to B$ and deduce that $\phi(z)$ is a path lying entirely in the codomain of the map of the homotopy fiber $C=\text{hofib}_y(Y\setminus\{p_2\}\to Y\setminus\{p_2\})$ or it is a path lying entirely in the codomain of the map of the homotopy fiber $\text{hofiber}_y(Y\setminus\{p_1,p_2\}\to Y\setminus\{p_1\}))$. By the same adjunction we conclude that our $\psi$ above must also satisfy for any fixed $z\in\partial I^n$, $\psi(z,t)$ lies entirely in either $Y\setminus\{p_1\}$ or $Y\setminus\{p_2\}$ (and conversely these information give a map $(I^n,\partial I^n)\to(A,B))$ by the adjunction). \\~\\

To summarize: we have a map $$\psi:I^n\times I\to Y$$ such that 
\begin{itemize}
\item $\psi(z,0)=y$ is the base point for all $z\in I^n$. 
\item $\psi(z,1)\in Y\setminus\{p_2\}$ for all $z\in I^n$. 
\item For any fixed $z\in\partial I^n$, $\psi(z,t)$ lies entirely in $Y\setminus\{p_1\}$ or $Y\setminus\{p_2\}$ for varying $t$. 
\end{itemize}
The goal is to make a homotopy from $\psi$ to a map whose third condition holds for any $z\in I^n$ when $n\leq d_1+d_2-3$. Then passing through the adjunction again we see that our original map $(I^n,\partial I^n)\to(A,B)$ is homotopic to the constant map as required. Apply 5.1.4 to obtain a homotopy $H:I^n\times I\times I\to Y$ from $\psi$ to a new map $\eta:I^n\times I\to Y$, such that we have a decomposition of $I^n\times I$ into cubes $W$ and the following are true. 
\begin{enumerate}
\item $\psi(W)\subset Y\setminus\{p_2\}$ implies $H(W,r)\subset Y\setminus\{p_2\}$ for all $r\in I$. 
\item $\psi(W)\subset Y\setminus\{p_1\}$ implies $H(W,r)\subset Y\setminus\{p_1\}$ for all $r\in I$. 
\item $\psi(W)\subset Y\setminus\{p_1,p_2\}$ implies $H(W,r)=\psi(W)$ for all $r\in I$. 
\item $\psi(W)\subset Y\setminus\{p_2\}$ then $(H(-,1)^{-1}(\{p_1\}))\cap W\subset K_{d_1}^1(W)$
\item $\psi(W)\subset Y\setminus\{p_1\}$ then $(H(-,1)^{-1}(\{p_2\}))\cap W\subset K_{d_2}^2(W)$
\end{enumerate}
We claim that $H(z,t,r)$ satisfies the three bullet points for all fixed $r$. 

Firstly, we already know that $\psi(z,0)=y$ is the base point for all $z\in I^n$. So for all cubes $W\subseteq I^n\times\{0\}$, we have $\psi(W)=\{y\}\subset Y\setminus\{p_1,p_2\}$. By 3., we conclude that $H(W,r)=\psi(W)=\{y\}$. Secondly, we know that $\psi(z,1)\in Y\setminus\{p_2\}$ for all $z\in I^n$. So for all cubes $W\subseteq I^n\times\{1\}$, we have $\psi(W)\subset Y\setminus\{p_2\}$. By 1., we conclude that $H(W,r)\subset Y\setminus\{p_2\}$ for all $r\in I$. Finally, according to the third bullet point, $\psi(z,I)$ lies entirely in $Y\setminus\{p_1\}$ or $Y\setminus\{p_2\}$ for $z\in\partial I^n$ WLOG lets say it lies entirely in $Y\setminus\{p_i\}$. Choose cubes $W_1,\dots,W_k$ in the decomposition of $I^n\times I$ so that it forms a minimal cover for $\{z\}\times I\subset W_1\cup\cdots\cup W_k$. By definition of the decomposition, these cubes firstly contain at least one point in $\{z\}\times I$, and $\psi(\{z\}times I)\subset Y\setminus\{p_j\}$ implies that $\psi(W_1),\dots,\psi(W_k)\subset Y\setminus\{p_j\}$. By 1., we conclude that $H(W_1,r),\dots,H(W_k,r)\subset Y\setminus\{p_j\}$ so that $H(W_1\cup\cdots\cup W_k,r)\subset Y\setminus\{p_j\}$. \\~\\

It remains to show that $\eta(-,-)=H(-,-,1)$ satisfies the stronger condition of the third bullet point as desired. Let $n\leq d_1+d_2-3$. We want to show that $\eta(z,I)\subset Y\setminus\{p_j\}$ for some $j$. I claim that this is equivalent to saying $$\text{proj}\left(\eta^{-1}(\{p_1\})\right)\cap\text{proj}\left(\eta^{-1}(\{p_2\})\right)=\emptyset$$ where $\text{proj}$ is the projection to the first coordinate. Indeed if $\eta(z,I)$ always lie inside one of $Y\setminus\{p_j\}$, $j=1,2$, then $\text{proj}\left(\eta^{-1}(\{p_1\})\right)=\{z\in I^n\;|\;\eta(z,I)\subset Y\setminus\{p_1\}\}$ and similarly for the other projection so that their intersection is empty. Conversely if one of $\eta(z,I)$ does not entirely in $Y\setminus\{p_2\}$ then the intersection is non-empty. \\~\\

So it suffices to prove that the intersection given above is empty. So suppose it is non-empty with an element $z_0$. Then there exists $t_1,t_2\in I$ such that $\eta(z_0,t_1)\in Y\setminus\{p_2\}$ and $\eta(z_0,t_2)\in Y\setminus\{p_1\}$. Choose cubes $W_1=W(a_1,\delta_1,L_1),W_2=W(a_2,\delta_2,L_2)$ in the given decomposition of $I^n\times I$ so that $(z_0,t_1)\in W_1$ and $(z_0,t_2)\in W_2$. By 4. and 5. we have $(z_0,t_1)\in\eta^{-1}(\{p_1\})\cap W_1\subset  K_{d_1}^1(W_1)$ and similarly for $(z_0,t_2)$. Then $(z_0,t_j)$ has at least $d_j$ coordinates satisfying the inequalities to lie in $K_{d_j}^1$. Hence $z_0=\text{proj}(z_0,t)$ has at least $d_j-1$ coordinates satisfying those inequalities for each $j=1,2$. For each $j$, $\text{proj}(W_j)$ is a cube containing $z_0$. Subdivide the cubes $W_1$ and $W_2$ further so that $\text{proj}(W_1)=\text{proj}(W_2)$. Since $\text{proj}(z_0,t)=z_0$, this means that $z_0$ has at least $d_j-1$ coordinates satisfying the inequalities of $K_{d_1}^1(W_1)$ and $K_{d_2}^2(W_2)$. But notice that the inequalities of $K_{d_1}^1(W_1)$ and $K_{d_2}^2(W_2)$ are disjoint (one concerns whether the points are at the front of the cube, the other at the back). So these conditions are disjoint and $z_0$ must have at least $d_1+d_2-2$ conditions on its coordinates. This is impossible if $z_0$ has less than $d_1+d_2-3$ coordinates. Hence we are done. \\~\\

Notice that the proof required $d_1,d_2\geq 1$. Assume WLOG that $d_2=0$, then right from the beginning we are considering the map of homotopy fibers $$\text{hofiber}_y(X\to X\cup e^{d_2})\to\text{hofiber}_y(X\cup e{d_1}\to X\cup e^{d_1}\cup e^{d_2})$$ where $X\cup e^{d_1}$ is now the disjoint union of $X$ with a base point. Then $\text{hofiber}_y(X\to X\amalg\ast)$ consists of pairs $(x,\phi)\in X\times\text{Map}(I,X\amalg\ast)$ such that $\phi(0)\in X$ and $\phi(1)=\ast$. But $\ast$ is disjoint from $X$ means that no pairs satisfy this conditions and the homotopy fiber is the empty set. Similarly for the target homotopy fiber. Hence the map of homotopy fibers is the identity and it is trivially true. The proof is similar for $d_1=0$. 
\end{proof}
\end{prp}

\begin{thm}{Blakers-Massey Theorem for Squares}{} Let $X_0,X_1,X_2,X_{12}\in\bold{CGWH}$ be spaces such that the square \\~\\
\adjustbox{scale=1.0,center}{\begin{tikzcd}
	X_0 & X_1 \\
	X_2 & X_{12}
	\arrow[from=1-1, to=1-2]
	\arrow[from=1-1, to=2-1]
	\arrow[from=1-2, to=2-2]
	\arrow[from=2-1, to=2-2]
\end{tikzcd}}\\~\\
is a homotopy pushout. Suppose the map $X_0\to X_i$ is $k_i$-connected for $i=1,2$. Then the diagram is $(k_1+k_2-1)$-cartesian. Explicitly, this means that $$\alpha:X_0\to\text{holim}(X_1\rightarrow X_{12}\leftarrow X_2)$$ is $(k_1+k_2-1)$-connected. \tcbline
\begin{proof}
By 3.7.29, we only need to consider squares of the form \\~\\
\adjustbox{scale=1.0,center}{\begin{tikzcd}
	{X_0} & {X_1} \\
	{X_2} & {X_{12}=X_1\amalg_{X_0}X_2}
	\arrow[from=1-1, to=1-2]
	\arrow[from=1-1, to=2-1]
	\arrow[from=1-2, to=2-2]
	\arrow[from=2-1, to=2-2]
\end{tikzcd}}\\~\\
By Hatcher 4.16, if $(X,A)$ is $k$-connected CW complex, then there exists a CW pair $(Z,A)$ such that $Z\setminus A$ only has cells of dimension $\geq k$ that is weakly equivalent to $(X,A)$. The set up of the theorem now simplifies to the following: $X_1$ is the obtained from $X_0$ by gluing cells of dimension $\geq k_1$, and likewise for $X_2$. Now showing that the above diagram is $(k_1+k_2-1)$-cartesian is the same as showing that the map of homotopy fibers of the vertical maps is $(k_1+k_2-1)$-connected. This is the same as saying $(\text{holim}(X_1\to X_{12}),\text{holim}(X_0\to X_1)$ is $(k_1+k_2)$-connected, which is the same as saying any map from $(I^n,\partial I^n)$ to the pair of space is homotopic to a map mapping $I^n$ into $\text{holim}(X_0\to X_1)$ relative boundary. By a similar method in the above lemma, such a map, by the hom product adjunction, is the same as giving a map $I^n\times I\to X_{12}$ for which $I^n\times\{1\}$ lies in $X_1$. But $X_1$ is a CW complex and $I^n\times\{1\}$ is compact, which means that image of the map is contained in finitely many cells, and so WLOG we can take $X_1,X_2$ to be the union of finitely many cells of appropriate dimension. \\~\\

Now by 1.1.10 and the above lemma we are done. 
\end{proof}
\end{thm}

This theorem directly generalizes the homotopy excision theorem in the following way. For $X$ a CW complex and $A,B$ two subcomplexes with non-empty intersection and $X=A\cup B$, consider the following square of inclusions: \\~\\
\adjustbox{scale=1.0,center}{\begin{tikzcd}
	A\cap B & A \\
	B & X
	\arrow[from=1-1, to=1-2]
	\arrow[from=1-1, to=2-1]
	\arrow[from=1-2, to=2-2]
	\arrow[from=2-1, to=2-2]
\end{tikzcd}}\\~\\
We have seen that such a square diagram is a homotopy pushout diagram. Now any inclusion map $W\hookrightarrow Z$ is $k$-connected if and only if $(Z,W)$ is $k$-connected. So $(A,A\cap B)$ is $k_1$-connected and $(X,B)$ is $k_2$-connected. Blaker's-Massey theorem implies that $$\text{hofiber}(A\cap B\to A)\to\text{hofiber}(B,X)$$ is $(k_1+k_2-1)$-connected. But by definition we have an isomorphism $\pi_k(\text{hofiber}(U\to V)\cong\pi_{k+1}(V,U)$. So we are really just saying that $\pi_k(A,A\cap B)\to\pi_k(X,B)$ given by the inclusion is $(k_1+k_2)$-connected. 

\begin{thm}{Freundenthal Suspension Theorem}{} Let $X$ be an $n$-connected space. Then the map $$X\to\Omega\Sigma X$$ that is adjoint to the identity map $\text{id}:\Sigma X\to\Sigma X$ is $(2n+1)$-connected. \tcbline
\begin{proof}
Notice that the following is a homotopy pushout square: \\~\\
\adjustbox{scale=1.0,center}{\begin{tikzcd}
	X & CX \\
	CX & {\Sigma X}
	\arrow[from=1-1, to=1-2]
	\arrow[from=1-1, to=2-1]
	\arrow[from=1-2, to=2-2]
	\arrow[from=2-1, to=2-2]
\end{tikzcd}}\\~\\
where all maps are inclusions. Since $X$ is $n$-connected, $(CX,X)$ is $(n+1)$-connected and so the map $X\to CX$ is $n$-connected. By Blakers-Massey theorem, the map $X\to\text{holim}(CX\rightarrow\Sigma X\leftarrow CX)$ given by $x\mapsto(x,e_x,x)$ is $(2n-1)$-connected. On the other hand, $CX$ is contractible so that $\text{holim}(CX\rightarrow\Sigma X\leftarrow CX)$ is homotopy equivalent to $\Omega\Sigma X$ via the map $(x,\gamma,y)\mapsto\gamma$. The composite $X\to\Omega\Sigma X$ is then precisely the map adjoint to $\text{id}:\Sigma X\to\Sigma X$, and hence the map $X\to\Omega\Sigma X$ is $(2n+1)$-connected. 
\end{proof}
\end{thm}

\begin{crl}{}{} Let $X,Y$ be spaces such that $X$ is $n$-connected. Let $f:X\to\Omega Y$ be a map. Let $\widetilde{f}:\Sigma X\to Y$ be its adjoint. If $f$ is a weak equivalence, then $\widetilde{f}$ is $(2n+3)$-connected. \tcbline
\begin{proof}
Since $X\to\Omega\Sigma X$ is the unit map, we obtain a commutative diagram: \\~\\
\adjustbox{scale=1.0,center}{\begin{tikzcd}
	X & {\Omega\Sigma X} \\
	& {\Omega Y}
	\arrow[from=1-1, to=1-2]
	\arrow["f"', from=1-1, to=2-2]
	\arrow["{\Omega\widetilde{f}}", from=1-2, to=2-2]
\end{tikzcd}}\\~\\
Since the unit map is $(2n+1)$-connected, $\Omega\widetilde{f}$ must be $(2n+2)$-connected. This means that $\pi_k(\Omega\Sigma X)\cong\pi_k(\Omega Y)$ are isomorphic via the induced map of $\Omega\widetilde{f}$. Using the isomorphism $\pi_k(\Omega Z)\cong\pi_{k+1}(Z)$ for any space $Z$, we deduce that $\widetilde{f}$ itself must be at least $(2n+3)$-connected. Hence $\widetilde{f}$ is $(2n+3)$-connected. 
\end{proof}
\end{crl}

More: Connectivity of suspension and loop maps. 

\subsection{Producing Excisive Approximations from Functors}
\begin{defn}{$T_1$ of a Functor}{} Let $F$ be a homotopy functor. Let $X$ be a space.  Define $$T_1F(X)=\text{holim}(F(CX)\rightarrow F(\Sigma X)\leftarrow F(CX))$$
\end{defn}

It is clear that given a map $X\to Y$, this induces maps $CX\to CY$ and $\Sigma X\to\Sigma Y$. And the definition of the homotopy pullback tells us that $T_1F$ is a functor. Any natural transformation $F\Rightarrow G$ gives maps $F(CX)\to G(CX)$ and $F(\Sigma X)\to G(\Sigma X)$ so that $T_1$ itself is a functor that sends functors to functors. Finally, there is a natural transformation $t_1(F):F\Rightarrow T_1F$ since for each $X$ there is a natural map $F(X)\to\text{holim}(F(CX)\rightarrow F(\Sigma X)\leftarrow F(CX))$. 

\begin{defn}{$P_1$ of a Functor}{} Let $F$ be a homotopy functor. Let $X$ be a space. Define $$P_1F(X)=\text{hocolim}(F(X)\overset{t_1(F)(X)}{\rightarrow}T_1F(X)\overset{t_1(T_1F)(X)}{\rightarrow}T_1(T_1F)(X)\rightarrow\cdots)$$
\end{defn}

Since $t_1$ is natural transformation, we again obtain appropriate commutative diagrams so that $P_1F$ becomes a functor. 

\begin{eg}{}{} If $F$ is reduced, then there is an easy way to describe the two functors $T_1F$ and $P_1F$. Beginning with $T_1F$, we know that $CX$ is contractible so $F(CX)$ is weakly equivalent to a point. Then $T_1F(X)$ is weakly equivalent to $\Omega F(\Sigma X)$. Iterating, we obtain $$T_1(T_1(F))(X)\simeq\Omega (T_1F)(\Sigma X)=\Omega(\Omega F(\Sigma(\Sigma X)))=\Omega^2 F(\Sigma^2 X)$$ as well as $$P_1F(X)=\underset{n\in\N}{\text{hocolim }}\Omega^n F(\Sigma^n X)$$
\end{eg}

We borrow terminology from Goodwillie

\begin{defn}{}{} Let $F:\bold{Spaces}\to\bold{Spaces}$ be a homotopy functor. We say that $F$ is $E(c,k)$ if the following are true. For all homotopy pushout squares \\~\\
\adjustbox{scale=1,center}{\begin{tikzcd}
	W & Y \\
	X & Z
	\arrow[from=1-1, to=1-2]
	\arrow[from=1-1, to=2-1]
	\arrow[from=1-2, to=2-2]
	\arrow[from=2-1, to=2-2]
\end{tikzcd}}\\~\\
such that the top horizontal map is $k_1$-connected and the left vertical map is $k_2$-connected for $k_1,k_2\geq k$, then $F$ of the square is $(k_1+k_2-c)$-cartesian. 
\end{defn}

Since any map is $(-1)$-connected, every homotopy functor is $E(-1,-1)$. \\
$\text{id}$ is $E(1,-1)$\\
If $F$ is $E(-\infty,-1)$ then $F$ is excisive. \\
If $F$ is $E(c,k)$, then $F$ is $E(c,k-1)$. This follows from the fact that every $n$-connected map is $(n-1)$-connected. 

\begin{prp}{}{} Let $F:\bold{Spaces}\to\bold{Spaces}$ be a homotopy functor. Suppose that $F$ is $E(c,k)$, then $T_1F$ is $E(c-1,\max\{k-1,-1\})$. 
\end{prp}

The maximum in the formula is to prevent $(-n)$-connected maps to appear in the discussion for large $n$. 

\begin{thm}{}{} Let $F:\bold{Spaces}\to\bold{Spaces}$ be a homotopy functor. Then $P_1F$ is excisive. \tcbline
\begin{proof}
All maps are trivially $(-1)$-connected. Hence $F$ is $E(-1,-1)$. Induct on $n$ to see that $T_1^nF$ is $E(-n,-1)$. Since $P_1F$ is the homotopy colimit of $T_1^nF$ as $n\to\infty$, we conclude that $P_1F$ is $E(-\infty,-1)$. 
\end{proof}
\end{thm}

\begin{eg}{}{} As shown from the Blakers-Massey theorem, $\text{id}$ is not an excisive functor. However, using example 1.3.3 we see that its excisive approximation is the functor $$X\mapsto\text{hocolim}_n\Omega^n\Sigma^n X$$
\end{eg}

The identity functor is interesting here precisely because it is not excisive, and hence we can obtain its excisive approximation. We will see later that excisive functors remedies the fact that $\pi_n$ does not satisfy excision for homology theories, and so precomposing $\pi_n$ with the approximation finally produces a reduced homology theory. And in the case of the identity functor, taking homotopy groups of its excisive approximation gives the stable homotopy groups. 

\subsection{Excisive Functors and Spectra}
While the identity functor is not excisive, there is still a wide class of functors that are exicisive, and these functors arise from spectra. 

\begin{defn}{Pre-Spectra}{} A pre-spectra consists of a collection of spaces $X_n$ and bonding maps $\sigma_n:\Sigma X_n\to X_{n+1}$ for $n\in\N$. 
\end{defn}

\begin{defn}{Spectra}{} A spectra is a pre-spectra such that the adjoints of the bonding maps are weak equivalences. 
\end{defn}

\begin{eg}{}{} Let $G$ be an abelian group. Recall (from Hatcher probably) that there are homotopy equivalences $K(G,n)$ and $\Omega K(G,n+1)$ so that we obtain the Eilenberg-Maclane spectrum $K=\{K(G,n)\}$. 
\end{eg}

\begin{eg}{}{} For each $n\in\N$, there are homeomorphisms $\Sigma S^n\to S^{n+1}$, but this does not mean that the adjoints of these maps $S^n\to\Omega S^{n+1}$ is a homeomorphism. It is not even a weak equivalence, and by Blakers-Massey theorem, these adjoint bonding maps are $(2n+1)$-connected. 
\end{eg}

Given a space $X$ and a spectrum $\{Y_n,\sigma_n\}$, we can smash $X$ with the spectrum at each level and smashing with the identity will give bonding maps $$\Sigma(Y_n\wedge X)=S^1\wedge Y_n\wedge X\overset{\sigma_n\wedge\text{id}_X}{\to} Y_{n+1}\wedge X$$ (we work in $\bold{CGWH}$) which are usually not weak equivalences.  (2.4.1 foundations of stable homotopy theory)

\begin{prp}{}{} Let $\{K_n,\sigma_n\}$ be a spectrum. Then the functor $$X\mapsto\text{hocolim}(K_0\wedge X\rightarrow\Omega(K_1\wedge X)\rightarrow\Omega^2(K_2\wedge X)\rightarrow\cdots)$$ is an excisive functor. \tcbline
\begin{proof}
Any space is $(-1)$-connected. In particular $K_0$ is $(-1)$-connected. Since $K_0\to\Omega K_1$ is a weak equivalence, we know that the adjoint map $\Sigma K_0\to K_1$ is $1$-connected. Since $K_0$ is $(-1)$-connected, we have that $\Sigma K_0$ is $0$-connected. Together with the fact that $\Sigma K_0\to K_1$ is $1$-connected means that $K_1$ is $0$-connected. By doing this argument inductively, we deduce that $K_n$ is $(n-1)$-connected. In particular, for any space $X$, $X$ is $(-1)$-connected and hence $K_n\wedge X$ is $(n-1)$-connected. \\~\\

Let the following diagram be a homotopy pushout: \\~\\
\adjustbox{scale=1,center}{\begin{tikzcd}
	W & Y \\
	X & Z
	\arrow[from=1-1, to=1-2]
	\arrow[from=1-1, to=2-1]
	\arrow[from=1-2, to=2-2]
	\arrow[from=2-1, to=2-2]
\end{tikzcd}}\\~\\
Using the homeomorphism $K_n\wedge\text{holim}(X_1\leftarrow X_0\rightarrow X_2)\cong\text{holim}(K_n\wedge X_1\leftarrow K_n\wedge X_0\rightarrow K_n\wedge X_2)$ we deduce that the following square is also a homotopy pushout: \\~\\
\adjustbox{scale=1,center}{\begin{tikzcd}
	{K_n\wedge W} & {K_n\wedge Y} \\
	{K_n\wedge X} & {K_n\wedge Z}
	\arrow[from=1-1, to=1-2]
	\arrow[from=1-1, to=2-1]
	\arrow[from=1-2, to=2-2]
	\arrow[from=2-1, to=2-2]
\end{tikzcd}}\\~\\
The spaces $W,X,Y,Z$ are $(-1)$-connected, and so all spaces in the above square is $(n-1)$-cartesian. In particular, the maps $K_n\wedge W\to K_n\wedge X$ and $K_n\wedge W\to K_n\wedge Y$ are $(n-1)$-connected. By the Blakers-Massey theorem, the same square is $(2n-3)$-connected. Now recall that $\Omega$ commutes with homotopy limits. The  isomorphism $\pi_n(X)\cong\pi_{n+1}(\Omega X)$ for all spaces $X$ means that applying $\Omega$ reduces the connectivity of the space by $1$. Therefore the diagram \\~\\
\adjustbox{scale=1,center}{\begin{tikzcd}
	{\Omega^n(K_n\wedge W)} & {\Omega^n(K_n\wedge Y)} \\
	{\Omega^n(K_n\wedge X)} & {\Omega^n(K_n\wedge Z)}
	\arrow[from=1-1, to=1-2]
	\arrow[from=1-1, to=2-1]
	\arrow[from=1-2, to=2-2]
	\arrow[from=2-1, to=2-2]
\end{tikzcd}}\\~\\
is $(n-3)$-cartesian. By 8.5.5 Munson, homotopy colimits commute with homotopy pushouts and as $n\to\infty$ we conclude that the functor is excisive. 
\end{proof}
\end{prp}

\pagebreak
\section{Spectra as Reduced and Excisive Functors}
\subsection{Stable Infinity Categories}
\begin{defn}{Stable Infinity Categories}{} Let $\mC$ be an infinity category. We say that $\mC$ is a stable infinity category if the following are true. 
\begin{itemize}
\item $\mC$ has a zero object $0$. 
\item $\mC$ admits all finite limits and colimits. 
\item A square in $\mC$ of the form \\~\\
\adjustbox{scale=1.0,center}{\begin{tikzcd}
	X & Y \\
	Z & W
	\arrow[from=1-1, to=1-2]
	\arrow[from=1-1, to=2-1]
	\arrow[from=1-2, to=2-2]
	\arrow[from=2-1, to=2-2]
\end{tikzcd}}\\~\\
is a pushout if and only if it is a pullback. 
\end{itemize}
\end{defn}

The definition given here is not the same as that of Lurie's, but Lurie does show that the two definitions are equivalent. 

Example in mind: spectra in ordinary categories: pushout=pullback. 

\begin{defn}{Suspension Functor}{} Let $\mC$ be an infinity category with a zero object $0$. Define the functor $\Sigma:\mC\to\mC$ by sending $X\in\mC$ to the pushout of $0\rightarrow X\leftarrow 0$. 
\end{defn}

Write $M^\Sigma$ as the full sub-category of $\text{Func}(\Delta^1\times\Delta^1,\mC)$ spanned by pushout squares of the form \\~\\
\adjustbox{scale=1.0,center}{\begin{tikzcd}
	X & 0 \\
	0 & {Y}
	\arrow[from=1-1, to=1-2]
	\arrow[from=1-1, to=2-1]
	\arrow[from=1-2, to=2-2]
	\arrow[from=2-1, to=2-2]
\end{tikzcd}}\\~\\
The suspension is a functor because we can also define it to be the composition of the section $\mC\to M^\Sigma$ with projection $M^\Sigma\to\mC$ to the bottom right object. The section exists since projection $M^\Sigma\to\mC$ to the first component is a trivial Kan fibration (Kerodon 1.5.5.5). (Lurie HA p.23)

\begin{defn}{Loop Functor}{} Let $\mC$ be an infinity category with a zero object $0$. Define the functor $\Omega:\mC\to\mC$ by sending $X\in\mC$ to the pullback of $0\leftarrow X\rightarrow0$. 
\end{defn}

We can guarantee that looping is a functor in a similar way to $\Sigma$. 

\begin{prp}{}{} Let $\mC$ be a stable infinity category. Then $\Omega$ and $\Sigma$ are both equivalence of infinity categories. \tcbline
\begin{proof}
Let $\mC$ be stable. Let $X\in\mC$. Then \\~\\
\adjustbox{scale=1.0,center}{\begin{tikzcd}
	X & 0 \\
	0 & {\Sigma X}
	\arrow[from=1-1, to=1-2]
	\arrow[from=1-1, to=2-1]
	\arrow[from=1-2, to=2-2]
	\arrow[from=2-1, to=2-2]
\end{tikzcd}}\\~\\
is a pushout in $\mC$. Since it is also a pullback in $\mC$, we conclude that $\Omega\Sigma X$ and $X$ are equivalent since (co)limts are unique up to equivalence. Similarly, let $Y\in\mC$. Then \\~\\
\adjustbox{scale=1.0,center}{\begin{tikzcd}
	{\Omega Y} & 0 \\
	0 & Y
	\arrow[from=1-1, to=1-2]
	\arrow[from=1-1, to=2-1]
	\arrow[from=1-2, to=2-2]
	\arrow[from=2-1, to=2-2]
\end{tikzcd}}\\~\\
is a pullback and a pushout in $\mC$. We can conclude that $\Sigma\Omega Y$ is weakly equivalent to $Y$. This shows that $\Sigma$ is a homotopy inverse of $\Omega$ and vice versa. 
\end{proof}
\end{prp}

\subsection{Excisive Functors between Infinity Categories}
\begin{defn}{Excisive Functors}{} Let $\mC,\mD$ be infinity categories. Let $F:\mC\to\mD$ be a functor. We say that $F$ is excisive if $F$ sends pushout squares to pullback squares. 
\end{defn}

Stable infinity categories are infinity categories in which pushouts and pullbacks coincide. In particular, this means that the identity functor is excisive. 

\begin{prp}{}{} Let $\mC,\mD$ be infinity categories. Suppose that $\mC$ is pointed and admits all finite colimits. Suppose that $\mD$ admits all finite limits. Then $$\text{Exc}_\ast(\mC,\mD)$$ is a stable infinity category. \tcbline
\begin{proof}
We first show that $\text{Exc}_\ast(\mC,\mD)$ is pointed. Let $\ast$ denote a final object of $\mC$. Since $\mD$ admits all finite limits and final objects are limits of the empty diagram, $\mD$ admits a final object $\ast'$. Let $X:\mC\to\mD$ be the constant functor given by $X(C)=\ast'$. Evidently it is reduced and exicisve and is the final object of $\text{Exc}_\ast(\mC,\mD)$. \\~\\

Now let $Y\in\text{Exc}_\ast(\mC,\mD)$. Since $X$ and $Y$ are reduced, the mapping space $\Hom_\mD(X(\ast),Y(\ast))$ is contractible. Consider the restriction map $$\Hom_{\text{Func}(\mC,\mD)}(X,Y)\to\Hom_\mD(X(\ast),Y(\ast))$$ given by sending a natural transformation $F:X\Rightarrow Y$ to $F(\ast):X(\ast)\to Y(\ast)$. Clearly \\~\\
\adjustbox{scale=1.0,center}{\begin{tikzcd}
	& \mC \\
	\ast && \mD
	\arrow["X", from=1-2, to=2-3]
	\arrow["{\text{incl.}}", hook, from=2-1, to=1-2]
	\arrow["{X|_\ast}"', from=2-1, to=2-3]
\end{tikzcd}}\\~\\
is a $2$-simplex so by prp7.3.6.1 (Kerodon), $\Hom_{\text{Func}(\mC,\mD)}(X,Y)\to\Hom_\mD(X(\ast),Y(\ast))$ is a homotopy equivalence and so $X$ is an initial object. \\~\\

prp1.4.2.16 HA
\end{proof}
\end{prp}

We want to give an equivalent characterization of stable infinity categories. Before that we need a technical lemma. 

\begin{prp}{}{} Let $\mC,\mD$ be pointed infinity categories. Suppose that $\mC$ admits all finite colimits and $\mD$ admits all finite limits. Then the following are equivalent. 
\begin{itemize}
\item $F$ is reduced and excisive. 
\item $F$ is reduced and it satisfies the following. For all $X\in\mC$, the comparison map $$\eta_X:F(X)\to\Omega_\mD(F(\Sigma_\mC X))$$ is an equivalence in $\mD$, where $\eta_X$ is given as follows. For any $X\in\mC$, the diagram \\~\\
\adjustbox{scale=1.0,center}{\begin{tikzcd}
	X & \ast \\
	\ast & {\Sigma_\mC X}
	\arrow[from=1-1, to=1-2]
	\arrow[from=1-1, to=2-1]
	\arrow[from=1-2, to=2-2]
	\arrow[from=2-1, to=2-2]
\end{tikzcd}}\\~\\
is a pushout in $\mC$. Sending the diagram through $F$ gives the wanted comparison map $\eta_X$ by the universal property of limits. 
\end{itemize} \tcbline
\begin{proof}
Let $F$ be reduced and excisive. Notice that following diagram on the left \\~\\
\adjustbox{scale=1.0,center}{\begin{tikzcd}
	X & \ast && {F(X)} & \ast \\
	\ast & {\Sigma_\mC X} && \ast & {F(\Sigma_\mC X)}
	\arrow[from=1-1, to=1-2]
	\arrow[from=1-1, to=2-1]
	\arrow[""{name=0, anchor=center, inner sep=0}, from=1-2, to=2-2]
	\arrow[from=1-4, to=1-5]
	\arrow[""{name=1, anchor=center, inner sep=0}, from=1-4, to=2-4]
	\arrow[from=1-5, to=2-5]
	\arrow[from=2-1, to=2-2]
	\arrow[from=2-4, to=2-5]
	\arrow["F", shorten <=14pt, shorten >=14pt, Rightarrow, from=0, to=1]
\end{tikzcd}}\\~\\
is a pushout diagram in $\mC$. Applying $F$ gives a pullback diagram on the right. On the other hand, we know that \\~\\
\adjustbox{scale=1.0,center}{\begin{tikzcd}
	{\Omega_\mD F(\Sigma_\mC X)} & \ast \\
	\ast & {F(\Sigma_\mC X)}
	\arrow[from=1-1, to=1-2]
	\arrow[from=1-1, to=2-1]
	\arrow[from=1-2, to=2-2]
	\arrow[from=2-1, to=2-2]
\end{tikzcd}}\\~\\
is a pullback diagram. Since limits in infinity category are unique up to equivalence, the comparison map $F(X)\to\Omega_\mD F(\Sigma_\mC X)$ is an equivalence. \\~\\

Now suppose that $F$ satisfies the second conditions. Let \\~\\
\adjustbox{scale=1.0,center}{\begin{tikzcd}
	W & X \\
	Y & Z
	\arrow[from=1-1, to=1-2]
	\arrow[from=1-1, to=2-1]
	\arrow[from=1-2, to=2-2]
	\arrow[from=2-1, to=2-2]
\end{tikzcd}}\\~\\
be a pushout square. Consider the following diagram in $\mC$: \\~\\
\adjustbox{scale=1.0,center}{\begin{tikzcd}
	W & X & 0 \\
	Y & {X\coprod_W Y} & {0\coprod_WY} & 0 \\
	0 & {X\coprod_W0} & {\Sigma_\mC W} & {\Sigma_\mC Y} \\
	& 0 & {\Sigma_\mC X} & {\Sigma_\mC(X\coprod_WY)}
	\arrow[from=1-1, to=1-2]
	\arrow[from=1-1, to=2-1]
	\arrow[from=1-2, to=1-3]
	\arrow[from=1-2, to=2-2]
	\arrow[from=1-3, to=2-3]
	\arrow[from=2-1, to=2-2]
	\arrow[from=2-1, to=3-1]
	\arrow[from=2-2, to=2-3]
	\arrow[from=2-2, to=3-2]
	\arrow[from=2-3, to=2-4]
	\arrow[from=2-3, to=3-3]
	\arrow[from=2-4, to=3-4]
	\arrow[from=3-1, to=3-2]
	\arrow[from=3-2, to=3-3]
	\arrow[from=3-2, to=4-2]
	\arrow[from=3-3, to=3-4]
	\arrow[from=3-3, to=4-3]
	\arrow[from=3-4, to=4-4]
	\arrow[from=4-2, to=4-3]
	\arrow[from=4-3, to=4-4]
\end{tikzcd}}\\~\\
Label the small squares $1$ to $7$ from left to right and top to bottom. By definition, $1$ is a pushout diagram. Since $1+2$ is a pushout diagram, by the pasting law $2$ is a diagram. Similarly, $3$ is a pushout diagram. Now $1+2+3+4$ is a pushout by definition. Since $1+3$ is a pushout, $2+4$ is a pushout diagram. Since $2$ is a pushout diagram, $4$ is a pushout diagram. Now $2+4+6$ is a pushout diagram by definition. Since $2+4$ is a pushout, $6$ is a pushout. Similarly,$3+4+5$ is a pushout. Since $3+4$ is a pushout then so is $5$. Finally, $4+5+6+7$ is a pushout diagram. Since $4+6$ is a pushout, so is $5+7$. Since $5$ is a pushout, then $7$ is a pushout. This proves that all squares $1$ to $7$ are pushouts. By applying $F$, we obtain the following diagram: \\~\\
\adjustbox{scale=1.0,center}{\begin{tikzcd}
	{F(W)} & {F(X)} & 0 \\
	{F(Y)} & {F(X\coprod_W Y)} & {F(0\coprod_WY)} & 0 \\
	0 & {F(X\coprod_W0)} & {F(\Sigma_\mC W)} & {F(\Sigma_\mC Y)} \\
	& 0 & {F(\Sigma_\mC X)} & {F(\Sigma_\mC(X\coprod_WY))}
	\arrow[from=1-1, to=1-2]
	\arrow[from=1-1, to=2-1]
	\arrow[from=1-2, to=1-3]
	\arrow[from=1-2, to=2-2]
	\arrow[from=1-3, to=2-3]
	\arrow[from=2-1, to=2-2]
	\arrow[from=2-1, to=3-1]
	\arrow[from=2-2, to=2-3]
	\arrow[from=2-2, to=3-2]
	\arrow[from=2-3, to=2-4]
	\arrow[from=2-3, to=3-3]
	\arrow[from=2-4, to=3-4]
	\arrow[from=3-1, to=3-2]
	\arrow[from=3-2, to=3-3]
	\arrow[from=3-2, to=4-2]
	\arrow[from=3-3, to=3-4]
	\arrow[from=3-3, to=4-3]
	\arrow[from=3-4, to=4-4]
	\arrow[from=4-2, to=4-3]
	\arrow[from=4-3, to=4-4]
\end{tikzcd}}\\~\\
Since $Z$ is equivalent to $X\coprod_WY$, we can remove the top left object and replace it with the pullback so that we obtain a commutative diagram: \\~\\
\adjustbox{scale=1.0,center}{\begin{tikzcd}
	{F(W)} \\
	& {F(X)\times_{F(Z)}F(Y)} & {F(X)} & 0 \\
	& {F(Y)} & {F(Z)} & {F(0\coprod_WY)} & 0 \\
	& 0 & {F(X\coprod_W0)} & {F(\Sigma_\mC W)} & {F(\Sigma_\mC Y)} \\
	&& 0 & {F(\Sigma_\mC X)} & {F(\Sigma_\mC Z)}
	\arrow["\mu", from=1-1, to=2-2]
	\arrow[bend left = 20, from=1-1, to=2-3]
	\arrow[bend right = 20, from=1-1, to=3-2]
	\arrow[from=2-2, to=2-3]
	\arrow[from=2-2, to=3-2]
	\arrow[from=2-3, to=2-4]
	\arrow[from=2-3, to=3-3]
	\arrow[from=2-4, to=3-4]
	\arrow[from=3-2, to=3-3]
	\arrow[from=3-2, to=4-2]
	\arrow[from=3-3, to=3-4]
	\arrow[from=3-3, to=4-3]
	\arrow[from=3-4, to=3-5]
	\arrow[from=3-4, to=4-4]
	\arrow[from=3-5, to=4-5]
	\arrow[from=4-2, to=4-3]
	\arrow[from=4-3, to=4-4]
	\arrow[from=4-3, to=5-3]
	\arrow[from=4-4, to=4-5]
	\arrow[from=4-4, to=5-4]
	\arrow[from=4-5, to=5-5]
	\arrow[from=5-3, to=5-4]
	\arrow[from=5-4, to=5-5]
\end{tikzcd}}\\~\\
By considering the large square on the left, we obtain a comparison map $F(X)\times_{F(Z)}F(Y)\to\Omega_\mD F(\Sigma_\mC W)$ which will be called $\theta$. Let $\mu$ be the comparison map $F(W)\to F(X)\times_{F(Z)}F(Y)$. Finally, notice that the following diagram on the left sits in the bottom right of the above square, and that we can add $0$s to the map so that the limits of the two diagrams remain equivalent (coinitial): \\~\\
\adjustbox{scale=1.0,center}{\begin{tikzcd}
	&& 0 &&& 0 & 0 & 0 \\
	&& {F(\Sigma_\mC Y)} & 0 & 0 & 0 \\
	0 & {F(\Sigma_\mC X)} & {F(\Sigma_\mC Z)} && {F(\Sigma_\mC X)} & {F(\Sigma_\mC Z)} & {F(\Sigma_\mC Y)}
	\arrow[from=1-3, to=2-3]
	\arrow[from=1-3, to=3-2]
	\arrow[from=1-6, to=1-7]
	\arrow[from=1-6, to=3-5]
	\arrow[from=1-7, to=3-6]
	\arrow[from=1-8, to=1-7]
	\arrow[from=1-8, to=3-7]
	\arrow[from=2-3, to=3-3]
	\arrow[from=2-4, to=2-5]
	\arrow[from=2-4, to=3-5]
	\arrow[from=2-5, to=3-6]
	\arrow[from=2-6, to=2-5]
	\arrow[from=2-6, to=3-7]
	\arrow[from=3-1, to=2-3]
	\arrow[from=3-1, to=3-2]
	\arrow[from=3-2, to=3-3]
	\arrow[from=3-5, to=3-6]
	\arrow[from=3-7, to=3-6]
\end{tikzcd}}\\~\\
Their limit is precisely computed vertically on each slice and hence is the pullback $$\Omega_\mD F(\Sigma_\mC X)\times_{\Omega_\mD F(\Sigma_\mC Z)}\Omega_\mD F(\Sigma_\mC Y)$$ Since all the diagrams map to each other, we obtain a commutative diagram: \\~\\
\adjustbox{scale=1.0,center}{\begin{tikzcd}
	{F(W)} & {F(X)\times_{F(Z)}F(Y)} \\
	& {\Omega_\mD F(\Sigma_\mC W)} & {\Omega_\mD F(\Sigma_\mC X)\times_{\Omega_\mD F(\Sigma_\mC Z)}\Omega_\mD F(\Sigma_\mC Y)}
	\arrow["\mu", from=1-1, to=1-2]
	\arrow["\simeq"', from=1-1, to=2-2]
	\arrow["\theta", from=1-2, to=2-2]
	\arrow["\simeq", from=1-2, to=2-3]
	\arrow[from=2-2, to=2-3]
\end{tikzcd}}\\~\\
where by assumption we have equivalences $F(W)\simeq\Omega_\mD F(\Sigma_\mC W)$ hence there are also equivalences on pullbacks. Notice that this implies $\theta$ has a left and right homotopy inverse, and hence is an equivalence. By the two out of three property, $\mu$ is also an equivalence. Hence $F$ sends pushouts to pullbacks. 
\end{proof}
\end{prp}

\begin{prp}{}{} Let $\mC$ be an infinity category. Then $\mC$ is a stable infinity category if and only if the following are true. 
\begin{itemize}
\item $\mC$ has a zero object $0$
\item $\mC$ admits all finite limits and colimits
\item The loop functor $\Omega:\mC\to\mC$ is an equivalence of infinity categories. 
\end{itemize} \tcbline
\begin{proof}
It suffices to show that the loop functor is an equivalence if and only if pushout squares and pullback squares coincide. Moreover, we have seen that if $\mC$ is stable then $\Omega$ is an equivalence. \\~\\

Conversely, suppose that $\Omega$ is an equivalence of infinity categories. Since $\Sigma$ is adjoint to $\Omega$ (HA Lurie p.24), then $\Sigma$ and $\Omega$ are both fully faithful. Choosing the identity functor in prp???? shows that pushout squares and pullback squares coincide. 
\end{proof}
\end{prp}

\begin{prp}{}{} Let $\mC,\mD$ be stable infinity categories. Let $F:\mC\to\mD$ be a functor. Then the following are equivalent. 
\begin{itemize}
\item $F$ commutes with finite limits. 
\item $F$ is reduced and excisive. 
\end{itemize} \tcbline
\begin{proof}~\\
\begin{itemize}
\item Suppose that $F$ is commutes with finite limits. Given a pushout diagram in $\mC$, since $\mC$ is stable by prp2.1.3 we know that it is a pullback. Since $F$ commutes with finite limits, $F$ sends the pullback to a pullback. Hence $F$ is excisive. Let $\ast$ be a final object of $\mC$. Since $F$ is left exact and $\ast$ is the limit of the empty diagram, $F(\ast)$ is a final object of $\mD$. Hence $F$ is reduced. \\~\\
\item Suppose that $F$ is reduced and excisive. Given $X,Y\in\mC$, the product $X\times Y$ is a pullback so that the following is a pullback squares: \\~\\
\adjustbox{scale=1,center}{\begin{tikzcd}
	{X\times Y} & X \\
	Y & \ast
	\arrow[from=1-1, to=1-2]
	\arrow[from=1-1, to=2-1]
	\arrow[from=1-2, to=2-2]
	\arrow[from=2-1, to=2-2]
\end{tikzcd}}\\~\\
Since $\mC$ is stable, the square is also a pushout. Then $F$ sends it to a pullback square: \\~\\
\adjustbox{scale=1,center}{\begin{tikzcd}
	{F(X\times Y)} & {F(X)} \\
	{F(Y)} & \ast
	\arrow[from=1-1, to=1-2]
	\arrow[from=1-1, to=2-1]
	\arrow[from=1-2, to=2-2]
	\arrow[from=2-1, to=2-2]
\end{tikzcd}}\\~\\
where $F(\ast)\simeq\ast$ since $F$ is reduced. Thus we obtain an equivalence $F(X\times Y)\simeq F(X)\times F(Y)$. \\~\\

Let $f,g:X\to Y$ be two maps in $\mC$. Let $E$ be the equalizer of $f$ and $g$. Since $E$ is the pullback of the following diagram, \\~\\
\adjustbox{scale=1,center}{\begin{tikzcd}
	X & {X\times X} & {X\times_YX}
	\arrow["\Delta", from=1-1, to=1-2]
	\arrow[from=1-3, to=1-2]
\end{tikzcd}}\\~\\
where the map on the right is given by the universal property of $X\times X$, we obtain a pullback square \\~\\
\adjustbox{scale=1,center}{\begin{tikzcd}
	E & {X\times_YX} \\
	X & {X\times X}
	\arrow[from=1-1, to=1-2]
	\arrow[from=1-1, to=2-1]
	\arrow[from=1-2, to=2-2]
	\arrow["\Delta"', from=2-1, to=2-2]
\end{tikzcd}}\\~\\
Since $\mC$ is stable, this is also a pushout square. Since $F$ is excisive, the following is a pullback square: \\~\\
\adjustbox{scale=1,center}{\begin{tikzcd}
	{F(E)} & {F(X\times_YX)} \\
	{F(X)} & {F(X\times X)}
	\arrow[from=1-1, to=1-2]
	\arrow[from=1-1, to=2-1]
	\arrow[from=1-2, to=2-2]
	\arrow[from=2-1, to=2-2]
\end{tikzcd}}\\~\\
Since $F$ sends pullbacks to pullbacks, we obtain equivalences $F(X\times_YX)\simeq F(X)\times_{F(Y)}F(X)$ and $F(X\times X)\simeq F(X)\times F(X)$. Hence the pullback square \\~\\
\adjustbox{scale=1,center}{\begin{tikzcd}
	{F(E)} & {F(X)\times_{F(Y)}F(X)} \\
	{F(X)} & {F(X)\times F(X)}
	\arrow[from=1-1, to=1-2]
	\arrow[from=1-1, to=2-1]
	\arrow[from=1-2, to=2-2]
	\arrow[from=2-1, to=2-2]
\end{tikzcd}}\\~\\
Therefore $F(E)$ is the equalizer of $F(f)$ and $F(g)$, and so $F$ preserves equalizers. Since $F$ preserves equalizers and products, we conclude that $F$ preserves all finite limits. 
\end{itemize}
\end{proof}
\end{prp}

\subsection{The Equivelence between Excisive Functors and Spectra}
Let $\mC$ be a simplicial category enriched in Kan Complexes. The homotopy coherent nerve $N_\bullet^\text{hc}(-)$ sends such a category to an infinity category. Following Lurie, we write $\mS=N_\bullet^{\text{hc}}(\bold{Kan})$ for the infinity category of spaces. This infinity category is equivalent to $N_\bullet^\text{hc}(\bold{CW})$, which is also equivalent to first inverting weak equivalences of $\bold{CGWH}$, and then taking homotopy coherent nerve. 

\begin{defn}{The Infinity Category of Spectra}{} Define the infinity category of spectra by $$\text{Sp}(\mS)=\lim(\cdots\rightarrow\mS\overset{\Omega}{\rightarrow}\mS\overset{\Omega}{\rightarrow}\mS)$$
\end{defn}

Our goal is to show that there is an equivalence between $\text{Sp}(\mS)$ and $\text{Exc}_\ast(\mS_\ast^\text{fin},\mS)$. For that we need to set up some lemmas. 

\begin{defn}{The Delooping Functor}{} Let $\mC$ be an infinty category that admits all finite limits. Define the delooping functor of $\mC$ to be the evaluation functor $$\Omega_\mC^\infty:\text{Exc}_\ast(\mS_\ast^\text{fin},\mC)\to\mC$$ given on object by $(F:\mS_\ast^\text{fin}\to F(S^0)$
\end{defn}

\begin{prp}{}{} Let $\mC$ be a pointed infinity category that admits all finite colimits. Let $\mD$ be an infinity category that admits all finite limits. Then post composition with $\Omega^\infty$ gives an equivalence of infinity categories $$\Omega_\mD^\infty\circ-:\text{Exc}_\ast(\mC,\text{Exc}_\ast(\mS_\ast^\text{fin},\mD))\overset{\simeq}{\rightarrow}\text{Exc}_\ast(\mC,\mD)$$ \tcbline
\begin{proof}
Notice that there is a canonical isomorphism 
\begin{align*}
\text{Exc}_\ast(\mC,\text{Exc}_\ast(\mS_\ast^\text{fin},\mD))&\simeq\text{Exc}_\ast(\mC\times\mS_\ast^\text{fin},\mD)\tag{Full subcat + adjoint}\\
&\simeq\text{Exc}_\ast(\mS^\text{fin},\text{Exc}_\ast(\mC,\mD))
\end{align*}
Under this identification, the functor $\Omega^\infty\circ -$ corresponds to the functor $\Omega_{\text{Exc}_\ast(\mC,\mD)}^\infty$ since $\Omega$ are computed term wise (like all limits). By prp2.2.4 $\text{Exc}_\ast(\mC,\mD)$ is stable. Hence by prp3.2.2 we conclude that $\Omega_{\text{Exc}_\ast(\mC,\mD)}^\infty$ is an equivalence of infinity categories. Hence $\Omega_\mD^\infty\circ-$ is an equivalence of infinity categories. 
\end{proof}
\end{prp}

\begin{thm}{}{} There is an equivalence of infinity categories $$\text{Sp}(\mS)\simeq\text{Exc}_\ast(\mS_\ast^\text{fin},\mS)$$ \tcbline
\begin{proof}~\\
Since $\mS$ is presentable and the infinity category of presentable infinity categories admit all small limits, $\text{Sp}(\mS)$ is also presentable. Every presentable infinity category admits all small limits and colimits. Since $\mS$ is pointed, $\text{Sp}(\mS)$ is also pointed. Since all limits are computed term-wise, we have that in particular $\Omega_{\text{Sp}(\mS)}$ is computed term wise. Given $X=\{X_n\;|\;n\in\N\}$ an object of $\text{Sp}(\mS)$, $\Omega_{\text{Sp}(\mS)}X$ is equivalent to $X$ because we have that $\Omega X_{n+1}$ is equivalent to $X_n$ for all $n$. By a prp we conclude that $\text{Sp}(\mS)$ is stable. \\~\\

Consider the canonical functor $F:\text{Sp}(\mS)\to\mS$ defined by recovering the first factor: $(X_0,X_1,\dots)\mapsto X_0$. It is clear that it commutes with finite limits since limits are computed term-wise. Hence $F$ is left exact. Using the equivalence of infinity categories $$\Omega^\infty\circ-:\text{Exc}_\ast^\text{L}(\text{Sp}(\mS),\text{Exc}_\ast(\mS_\ast^\text{fin},\mS))\to\text{Exc}_\ast^{\text{L}}(\text{Sp}(\mS),\mS)$$ we obtain a factorization \\~\\
\adjustbox{scale=1,center}{\begin{tikzcd}
	{\text{Sp}(\mS)} && \mS \\
	& {\text{Exc}_\ast(\mS_\ast^\text{fin},\mS)}
	\arrow["F", from=1-1, to=1-3]
	\arrow["{G\circ -}"', from=1-1, to=2-2, dashed]
	\arrow["{\Omega^\infty}"', from=2-2, to=1-3]
\end{tikzcd}}\\~\\
Let $\mC$ be an arbitrary stable infinity category. By functoriality we obtain a similar factorization: \\~\\
\adjustbox{scale=1,center}{\begin{tikzcd}
	{\text{Exc}_\ast^{\text{L}}(\mC,\text{Sp}(\mS))} && {\text{Exc}_\ast^{\text{L}}(\mC,\mS)} \\
	& {\text{Exc}_\ast^{\text{L}}(\mC,\text{Exc}_\ast(\mS_\ast^\text{fin},\mS))}
	\arrow["{F\circ -}", from=1-1, to=1-3]
	\arrow["{G\circ -}"', from=1-1, to=2-2]
	\arrow["{\Omega^\infty\circ -}"', from=2-2, to=1-3]
\end{tikzcd}}\\~\\

I claim that $F\circ -$ and $\Omega^\infty\circ -$ are equivalences so that $G\circ -$ is an equivalence. The case $\Omega^\infty\circ -$ is already clear. On the other hand, since $\Omega$ are computed term-wise (like all limits) and since $\text{Func}(\mC,\text{Sp}(\mS))$ is right adjoint to products we know that $\text{Func}$ commutes with finite limits . Thus we have that $$\text{Exc}_\ast^\text{L}(\mC,\text{Sp}(\mS))=\lim(\cdots\rightarrow\text{Exc}_\ast^{\text{L}}(\mC,\mS)\overset{\Omega\circ-}{\rightarrow}\text{Exc}_\ast^{\text{L}}(\mC,\mS)\overset{\Omega\circ-}{\rightarrow}\text{Exc}_\ast^{\text{L}}(\mC,\mS))$$ Now $\Omega_\mS\circ -$ is an equivalence because $\text{Exc}_\ast(\mC,\mS)$ is stable (because $\mC$ is stable and $\mS$ admits all finite (co)limits). We conclude that $\text{Exc}_\ast^{\text{L}}(\mC,\text{Sp}(\mS))\simeq\text{Exc}_\ast^{\text{L}}(\mC,\mS)$. Thus evaluation on the first factor $F\circ -:\text{Exc}_\ast^{\text{L}}(\mC,\text{Sp}(\mS))\to\text{Exc}_\ast^{\text{L}}(\mC,\mS)$ is an equivalence of infinity categories. \\~\\

From the fact that $G\circ -$ is an equivalence, we have an equivalence $$\text{Exc}_\ast(\mC,\text{Sp}(\mS))=\text{Exc}_\ast^L(\mC,\text{Sp}(\mS))\simeq\text{Exc}_\ast^L(\mC,\text{Exc}_\ast(\mS_\ast^\text{fin},\mS))=\text{Exc}_\ast(\mC,\text{Exc}_\ast(\mS_\ast^\text{fin},\mS))$$ for all stable infinity categories $\mC$. By the Yoneda lemma we conclude that $\text{Exc}_\ast(\mS_\ast^\text{fin},\mS)$ and $\text{Sp}(\mS)$ are equivalent. 
\end{proof}
\end{thm}

Here notice that we are taking excisive fucntors from $\mS_\ast^\text{fin}$ to $\mS$ instead of functors from $\mS$. This has the same effect as taking finitary functors in the ordinary-categorical case. In both cases we would like the functor to be determined on finite colimits. \\

More: $F$ in the proof is precisely $\Omega^\infty$ once we prove equivalence $\text{Sp}(\mS)\simeq\text{Exc}_\ast(\mS_\ast^\text{fin},\mS)$. It has an adjoint $\Sigma\infty$ sending $X$ to its suspension spectrum. Through the composition we see that $F\in\text{Exc}_\ast(\mS_\ast^\text{fin},\mS)$ corresponds to the spectrum $\Sigma^\infty F(S^0)\simeq\{F(S^n)\}$. 

\pagebreak
\section{The Relation Between Excisive Functors, Spectra and Homology Theories}
The heart of the exposition lies in the following commutative diagram: \\~\\
\adjustbox{scale=1,center}{\begin{tikzcd}
	&&&&&& {\bold{Cohomology}} \\
	\\
	\\
	&&& {\bold{Sp}} &&&&&& {\bold{Homology}} \\
	\\
	\\
	{\text{Exc}_\ast(\bold{Top}_\ast,\bold{Sp})} &&&&&& {\text{Exc}_\ast(\bold{Top}_\ast,\bold{Top}_\ast)} \\
	\\
	&&&&&& {\text{Func}(\bold{Top}_\ast,\bold{Top}_\ast)}
	\arrow["{\text{Brown}}", shift left, from=1-7, to=4-4]
	\arrow["{\text{S-Duality}?}", from=1-7, to=4-10]
	\arrow["{\{K_n\}\mapsto[X,K_n]}", shift left, from=4-4, to=1-7]
	\arrow["{\{K_n\}\mapsto\pi_n(\Omega^\infty\{K_n\wedge -\})}", from=4-4, to=4-10]
	\arrow["{\{K_n\}\mapsto\Omega^\infty\{K_n\wedge -\}}", shift left, from=4-4, to=7-7]
	\arrow["{G\mapsto G(S^0)}", from=7-1, to=4-4]
	\arrow["{G\mapsto G|_0}", shift left, from=7-1, to=7-7]
	\arrow["{F\mapsto R_\infty\{F(S^n)\}}", shift left, from=7-7, to=4-4]
	\arrow["{F\mapsto\pi_n\circ F}"', from=7-7, to=4-10]
	\arrow["{F\mapsto R_\infty\{F(S^n\wedge -)\}}", shift left, from=7-7, to=7-1]
	\arrow["{F\mapsto P_1F}", from=9-7, to=7-7]
\end{tikzcd}}\\~\\
	


The triangle on the right commutes by definition. The triangle on the left also commutes with by definition. The bijection $\bold{Sp}$ and $\bold{Homology}$ does not give an equivalence of categories. 

\subsection{From Excisive Functors to (Co)Homology}
\begin{defn}{Reduced Homology Theory}{} A reduced homology theory is a collection of functors and natural trasnformations $$H_n:\bold{CW}_\ast\to\bold{Ab}\;\;\;\;\text{ and }s_n:H_n\Rightarrow H_{n+1}\circ\Sigma$$ such that the following axioms are satisfied. 
\begin{itemize}
\item If $f\simeq g:X\to Y$ are homotopic, then $H_n(f)=H_n(g)$
\item If $f:X\to Y$ is a map then there is an exact sequence $$H_n(X)\overset{f_\ast}{\rightarrow}H_n(Y)\overset{j_\ast}{\rightarrow}H_n(C_f)$$ where $j:X\to C_f$ is the inclusion. 
\item The natural transformation gives an isomorphism $$s_n(X):H_n(X)\to H_{n+1}(\Sigma X)$$ for all $n$. 
\item For any wedge product $X=\bigvee_k X_k$, the inclusion maps induces an isomorphism $$\bigoplus_k H_n(X_k)\cong H_n(X)$$
\item If $f:X\to Y$ is a weak homotopy equivalence, then $H_n(f)$ is an isomorphism. 
\end{itemize}
\end{defn} (Davies / Switzer)

\begin{thm}{}{} Let $F$ be an excisive functor. Then $E_n(X)=\pi_n(F(X))$ defines a reduced homology theory. \tcbline
\begin{proof}
Firstly, note that $\pi_0(F(X))$ and $\pi_1(F(X))$ has the structure of an abelian group. To see this, recall that from ????? we have \\~\\
\adjustbox{scale=1,center}{\begin{tikzcd}
	X & \ast \\
	\ast & {\Sigma X}
	\arrow[from=1-1, to=1-2]
	\arrow[from=1-1, to=2-1]
	\arrow[from=1-2, to=2-2]
	\arrow[from=2-1, to=2-2]
\end{tikzcd}}\\~\\
is a homotopy pushout. Since $F$ is excisive, we obtain a homotopy pullback square: \\~\\
\adjustbox{scale=1,center}{\begin{tikzcd}
	{F(X)} & {F(\ast)} \\
	{F(\ast)} & {F(\Sigma X)}
	\arrow[from=1-1, to=1-2]
	\arrow[from=1-1, to=2-1]
	\arrow[from=1-2, to=2-2]
	\arrow[from=2-1, to=2-2]
\end{tikzcd}}\\~\\
and therefore we obtain a weak equivalence $F(X)\to\text{holim}(F(\ast)\rightarrow F(\Sigma X)\leftarrow F(\ast))$. On the other hand, we know that $\text{holim}(\ast\rightarrow F(\Sigma X)\leftarrow\ast)\cong\Omega F(\Sigma X)$. Finally, from the commutative diagram: \\~\\
\adjustbox{scale=1,center}{\begin{tikzcd}
	\ast & {F(\Sigma X)} & \ast \\
	{F(\ast)} & {F(\Sigma X)} & {F(\ast)}
	\arrow[from=1-1, to=1-2]
	\arrow["\simeq"', from=1-1, to=2-1]
	\arrow["{\text{id}}"', from=1-2, to=2-2]
	\arrow[from=1-3, to=1-2]
	\arrow["\simeq", from=1-3, to=2-3]
	\arrow[from=2-1, to=2-2]
	\arrow[from=2-3, to=2-2]
\end{tikzcd}}\\~\\
and the matching lemma, we deduce that we have a weak equivalence: \\~\\
\adjustbox{scale=1,center}{\begin{tikzcd}
	{\text{holim}(\ast\rightarrow F(\Sigma X)\leftarrow\ast)} & {\text{holim}(F(\ast)\rightarrow F(\Sigma X)\leftarrow F(\ast))}
	\arrow["\simeq", from=1-1, to=1-2]
\end{tikzcd}}\\~\\
Then we conclude that we have a chain of weak equivalences going the wrong way: \\~\\
\adjustbox{scale=1,center}{\begin{tikzcd}
	{F(X)} & {\text{holim}(F(\ast)\rightarrow F(\Sigma X)\leftarrow F(\ast))} & {\text{holim}(\ast\rightarrow F(\Sigma X)\leftarrow\ast)} & {\Omega F(\Sigma X)}
	\arrow["\simeq", from=1-1, to=1-2]
	\arrow["\simeq"', from=1-3, to=1-2]
	\arrow["\cong"', from=1-4, to=1-3]
\end{tikzcd}}\\~\\
Regardless, we can still deduce a natural isomorphism $$\pi_n(F(X))\cong\pi_n(\Omega F(\Sigma X))$$ because $\pi_n$ sends weak equivalences to isomorphisms, and isomorphisms are invertible. We conclude that the natural bijection $\pi_0(F(X))\cong\pi_2(F(\Sigma^2 X))$ gives an abelian group structure on $\pi_0(F(X))$, and $\pi_1(F(X))\cong\pi_2(\Omega F(\Sigma^2 X)$ show that $\pi_1(F(X))$ is abelian. Now we prove that $\pi_n\circ F$ satisfies the axioms for being a reduced homology theory. \\~\\

To show that $\pi_n\circ F$ is a homotopy functor, first consider the inclusion maps $i_0:X\times\{0\}\hookrightarrow X\times I$ and $i_1:X\times\{1\}\to X\times I$. They induces maps $\pi_n(F(i_0))$ and $\pi_n(F(i_1))$. I claim that these are the same map in $\bold{Ab}$. To see this, consider the post composition of both of these functions: \\~\\
\adjustbox{scale=1,center}{\begin{tikzcd}
	{\pi_n(F(X))} && {\pi_n(F(X\times I))} && {\pi_n(F(X))}
	\arrow["{\pi_n(F(i_0))}", shift left, from=1-1, to=1-3]
	\arrow["{\pi_n(F(i_1))}"', shift right, from=1-1, to=1-3]
	\arrow["{\pi_n(F(\text{proj}))}", from=1-3, to=1-5]
\end{tikzcd}}\\~\\
where the second map is induced by projection on the first coordinate. Since $I$ is contractible, this projection map is a weak equivalence. Since $F$ preserves weak equivalences and $\pi_n$ sends weak equivalences to isomorphisms, we conclude that the latter map in the diagram is an isomorphism. Now consider the composite $\pi_n(F(\text{proj}))\circ\pi_n(F(i_0))=\pi_n(F(\text{proj}\circ i_0))$. Since $\text{proj}\circ i_0$ is a weak equivalence, $F$ of this composite is also a weak equivalence and hence is an isomorphism after passing through $\pi_n$. Since the composite and $\pi_n(F(\text{proj}))$ is an isomorphism, we conclude that $\pi_n(F(i_0))$ is an isomorphism. A similar argument shows that $\pi_n(F(i_1))$ is an isomorphism. \\~\\

Going back to the main proof, recall that we want to show that if $f$ and $g$ are homotopic, then $\pi_n(F(f))=\pi_n(F(g))$. Suppose that $H$ witnesses the homotopy from $f$ to $g$. Then we obtain a commutative diagram: \\~\\
\adjustbox{scale=1,center}{\begin{tikzcd}
	{X\times\{0\}} \\
	& {X\times I} & Y \\
	{X\times\{1\}}
	\arrow["i_0", hook, from=1-1, to=2-2]
	\arrow["f", from=1-1, to=2-3, bend left = 20]
	\arrow["H", from=2-2, to=2-3]
	\arrow["i_1", hook, from=3-1, to=2-2]
	\arrow["g", from=3-1, to=2-3, bend right = 20]
\end{tikzcd}}\\~\\
where we identify $X\times\{0\}\cong X\cong X\times\{1\}$. Since $\pi_n\circ F$ is a functor, passing through $\pi_n\circ F$ gives the following commutative diagram: \\~\\
\adjustbox{scale=1,center}{\begin{tikzcd}
	{\pi_n(F(X\times\{0\}))} \\
	& {\pi_n(F(X\times I))} & {\pi_n(F(Y))} \\
	{\pi_n(F(X\times\{1\}))}
	\arrow["\pi_n(F(i_0))"', from=1-1, to=2-2]
	\arrow["{\pi_n(F(f))}", from=1-1, to=2-3, bend left = 20]
	\arrow["{\pi_n(F(H))}", from=2-2, to=2-3]
	\arrow["\pi_n(F(i_1))", from=3-1, to=2-2]
	\arrow["{\pi_n(F(g))}"', from=3-1, to=2-3, bend right = 20]
\end{tikzcd}}\\~\\
Since the maps induced by $i_0$ and $i_1$ are isomorphisms and the diagram is commutative, we conclude that $\pi_n(F(f))=\pi_n(F(g))$. \\~\\

If $f:X\to Y$ is a map, then $X\to Y\to C_f$ is a cofiber sequence, and we have a homotopy pushout \\~\\
\adjustbox{scale=1,center}{\begin{tikzcd}
	X & Y \\
	\ast & {C_f}
	\arrow["f", from=1-1, to=1-2]
	\arrow[from=1-1, to=2-1]
	\arrow[from=1-2, to=2-2]
	\arrow[from=2-1, to=2-2]
\end{tikzcd}}\\~\\
$F$ sends this to a homotopy pullback, so we obtain a fiber sequence $F(X)\to F(Y)\to F(C_f)$. Then $\pi_n$ sends the fiber sequence to the desired exact sequence. \\~\\

Recall that the wedge product is a coproduct of a diagram with no morphisms in $\bold{Spaces}$. Hence $\bigvee_{i\in I}F(X_i)$ is homotopy equivalent to $\text{hocolim}(F(X_i))$. Since $F$ is finitary and $\pi_n$ sends homotopy colimits to colimits, we have 
\begin{align*}
\pi_n\left(F\left(\bigvee_{i\in I}X_i\right)\right)&\cong\pi_n(F(\text{hocolim}_{i\in I}X_i))\\
&\cong\pi_n(\text{hocolim}_{i\in I}(F(X_i)))\\
&\cong\text{colim}_i\pi_n(F(X_i))\\
&\cong\bigoplus_{i\in I}\pi_n(F(X_i))
\end{align*}
since $\text{colim}$ of an empty diagram in $F$ the direct sum. \\~\\
\end{proof}
\end{thm}

\begin{eg}{}{} The stable homotopy groups arise naturally as a reduced homology theory in the following way. The homotopy groups $\pi_\ast$ do not make up a reduced homology theory because they are missing the excisive property. This can be explained as the identity functor not quite being excisive by the Blakers-Massey theorem. Indeed if $F$ is excisive, then $\pi_\ast\circ F$ does indeed form a reduced homology theory. \\~\\

While the identity functor is not excisive, we can take its excisive approximation $P_1F(X)=\text{hocolim}_k\Omega^k\Sigma^k X$ to give a reduced homology theory $$X\mapsto\pi_n(\text{hocolim}_k(\Omega^k\Sigma^kX))$$ Since $\pi_n$ sends directed homotopy colimits to directed colimits, the group on the right is isomorphic to $$\lim_{k\to\infty}\pi_n(\Omega^k\Sigma^k X)\cong\lim_{k\to\infty}\pi_{n+k}(\Sigma^k X)$$ This is precisely the definition of the stable homotopy groups of $X$. 
\end{eg}

\begin{eg}{Infinite Symmetric Product}{} Let $X$ be a space with base point $x_0$. For each $n\in\N$, the symmetric group $S^n$ acts on $X^n$ by permuting the coordinates. We can then form the orbit space of the action, denoted $SP_n(X)$. The inclusion $X^n\hookrightarrow X^{n+1}$ given by $(x_1,\dots,x_n)\mapsto(x_1,\dots,x_n,x_0)$ induces a well defined inclusion $SP_n(X)\to SP_{n+1}(X)$. We call the colimit as $n\to\infty$ by the infinite symmetric product $$SP(X)=\underset{n\to\infty}{\text{colim}}SP_n(X)$$ It turns out that $SP(-)$ is functorial, preserves weak equivalencs and is excisive. The proof of excision involves the use of 3-cubes and quasi-fibrations, which is not the focus of the paper. \\~\\

In any case, post composing with the homotopy groups give a reduced homology theory $$\pi_\ast(SP(-))$$ and it turns out that there are natural isomorphisms $$\pi_\ast(SP(X))\cong\widetilde{H}_{\text{Sing}}(X;\Z)$$ so that the reduced singular homology theory can be given by an exicisve functor in this manner. (ref?)
\end{eg}

\subsection{From Spectra to (Co)Homology}
\begin{defn}{Homology Theory Associated to Spectra}{} Let $K=\{K_n\;|\;n\in\N\}$ be a spectrum. Define a functor $E_n:\bold{Spaces}\to\bold{Ab}$ by $$E_n(X;K)=\lim_{k\to\infty}\pi_{n+k}(X\wedge K_k)$$ and natural transformations $s_n:E_n\Rightarrow E_{n+1}\circ\Sigma$ given by the structure maps of the spectrum. 
\end{defn}

Here, the maps defining the direct limit $\pi_{n+k}(X\wedge K_k)\to\pi_{n+1+k}(X\wedge K_{k+1})$ are given by $$\pi_{n+k}(X\wedge K_k)\overset{\Sigma}{\rightarrow}\pi_{n+1+k}(S^1\wedge X\wedge K_k)\overset{\text{id}_X\wedge\sigma_n}{\rightarrow}\pi_{n+1+k}(X\wedge K_{k+1})$$ (we are working in $\bold{CGWH}$) Moreover, it is a functor since any map $f:X\to Y$ wedges with the identity to give a map $X\wedge K_k\to Y\wedge K_k$. The homotopy group functor gives a map of homotopy groups. And the universal property of direct limits give the induced map. (Davies p.229)

\begin{thm}{}{} Let $K=\{K_n\;|\;n\in\N\}$ be a spectrum. Then $E_n(X)=\lim_{k\to\infty}\pi_{n+k}(X\wedge K_k)$ defines a reduced homology theory. \tcbline
\begin{proof}
We have seen that $X\mapsto\text{hocolim}_k\Omega^k(X\wedge K_k)$ is an excisive functor. Therefore $\pi_n(\text{hocolim}_k\Omega^k(X\wedge K_k))$ defines a reduced homology theory. But we also have
\begin{align*}
\pi_n(\text{hocolim}_k\Omega^k(X\wedge K_k))&=\lim_{k\to\infty}\pi_n(\Omega^k(X\wedge K_k))\\
&=\lim_{k\to\infty}\pi_{n+k}(X\wedge K_k)\\
\end{align*}
and this establishes the claim. 
\end{proof}
\end{thm}

\begin{eg}{Reduced Singular Homology Groups}{} Recall the Eilenberg-Maclane spectrum for an abelian group $G$, denoted by $K=\{K(G,n)\;|\;n\in\N\}$. Its associated reduced homology theory is given by $$E_n(X;K)=\lim_{n\to\infty}\pi_{n+k}(X\wedge K(G,k))$$ (Davies p.229) Since reduced homology theories are determined uniquely up to isomorphism by $E_0(S^0)$, we see that $E_0(S^0;K)\cong G$ implies that $E_n$ is naturally isomorphic to reduced singular homology $\widetilde{H}_n(-;G)$ with coefficients in $G$. \\~\\

Relation to infinite symmetric product?
\end{eg}

\begin{eg}{}{} We have seen that the excisive functor $X\mapsto\text{hocolim}_n\Omega^n\Sigma^nX$ gives rise to the stable homotopy groups. It is natural to ask what spectrum it is associated to. Following the infinity categorical equivalence between excisive functors and spectra, we recognize that evaluating at the $k$-spheres almost gives us the associated spectra. Almost because as mentioned (where?), there is no actual weak equivalences, and one must take fibrant replacement. \\~\\

Ignoring this issue for a moment, we can at least recognize evaluating the excisive approximation at the $k$-spheres give $$\text{hocolim}_n\Omega^n\Sigma^nS^k\cong\text{hocolim}_n\Omega^n(S^k\wedge S^n)$$ This is precisely the fibrant replacement of the sphere spectrum. 
\end{eg}

It is also curious to talk about the more classical association of spectra with cohomology. 

\begin{defn}{Reduced Cohomology Theory}{} A reduced cohomology theory is a collection of contravariant functors and natural trasnformations $$H^n:\bold{CW}_\ast\to\bold{Ab}\;\;\;\;\text{ and }s^n:H^{n+1}\circ\Sigma\Rightarrow H^n$$ such that the following axioms are satisfied. 
\begin{itemize}
\item If $f\simeq g:X\to Y$ are homotopic, then $H^n(f)=H^n(g)$
\item If $f:X\to Y$ is a map then there is an exact sequence $$H^n(C_f)\overset{j^\ast}{\rightarrow}H^n(Y)\overset{f^\ast}{\rightarrow}H^n(X)$$ where $j:X\to C_f$ is the inclusion. 
\item The natural transformation gives an isomorphism $$s^n(X):H^{n+1}(\Sigma X)\to H^n(X)$$ for all $n$. 
\item For any wedge product $X=\bigvee_k X_k$, the inclusion maps induces an isomorphism $$\prod_k H^n(X_k)\cong H^n(X)$$
\item If $f:X\to Y$ is a weak homotopy equivalence, then $H^n(f)$ is an isomorphism. 
\end{itemize}
\end{defn}

AGP

$[X,Y]_\ast$ means the set of pointed homotopy classes of maps from $X$ to $Y$. 

\begin{thm}{Brown's Representability Theorem}{} Let $H^n:\bold{CW}_\ast\to\bold{Ab}$ be a reduced cohomology theory. Then there exists a spectra $\{K_n\;|\;n\in\N\}$ such that there are natural isomorphisms $$H^n(X)\cong[X,K_n]_\ast$$ for all spaces $X$. 
\end{thm}

Bordism, Stable Homotopy and Adams Spectral Sequences. 

https://mathoverflow.net/questions/63974/is-every-homology-theory-given-by-a-spectrum

switzer 14.35

\begin{prp}{}{} Let $\{K_n\;|\;n\in\N\}$ be a spectra. Then $$E^n(-)=[-,K_n]_\ast$$ defines a reduced cohomology theory whose maps $s^n:E^{n+1}(\Sigma X)\to E^n(X)$ are given by $$[\Sigma X,K_{n+1}]_\ast\cong[X,
\Omega K_{n+1}]_\ast\cong[X, K_n]_\ast$$
\end{prp}

While there is an association between spectra and cohomology theories, it is well known that their categories are not equivalent, mainly because of the existence of phantom maps: Non-trivial maps of spectra that give the zero map when associating it to cohomology. 

\begin{eg}{Landweber-exact Spectra}{}
\end{eg}

\begin{thm}{Landweber exact functor theorem}{}
\end{thm}

S-Duality?

\subsection{Excisive Functors Taking Values in Infinite Loop Spaces}



\pagebreak
\section{Appendix}
\subsection{Homotopical Connectivity}
2.6 cubical homotopy theory
\begin{defn}{$n$-Connected Spaces}{} Let $X$ be a space. Let $n\in\N$. We say that $X$ is $n$-connected if for all $-1\leq k\leq n$, any map $S^k\to X$ is homotopic to the constant map to the base point. 
\end{defn}

Every non-empty space is $(-1)$-connected. 

\begin{defn}{$n$-Connected Spaces}{} Let $(X,A)$ be a pointed pair of spaces. Let $n\in\N$. We say that $(X,A)$ is $n$-connected if for all $-1\leq k\leq n$, any map $(D^k,\partial D^k)\to(X,A)$ is homotopic to a map $(D^k,\partial D^k)\to (A,A)$ relative to boundary. 
\end{defn}

\begin{defn}{$n$-Connected Maps}{} Let $X,Y$ be spaces. Let $f:X\to Y$ be a map. We say that $f$ is $n$-connected if $(M_f,X)$ is $n$-connected. 
\end{defn}

\begin{prp}{}{} Let $X$ be a space. Let $n\in\N$. Then the following are equivalent. 
\begin{itemize}
\item $X$ is $n$-connected. 
\item $\pi_k(X)=0$ for all $0\leq k\leq n$. 
\item For all $-1\leq k\leq n$, every map $S^k\to X$ extends to a map $D^{k+1}\to X$. 
\item $(CX,X)$ is $(n+1)$-connected. 
\end{itemize}
\end{prp}

\begin{prp}{}{} Let $(X,A)$ be a space. Let $n\in\N$. Then the following are equivalent. 
\begin{itemize}
\item $(X,A)$ is $n$-connected
\item For all $0<k\leq n$, $\pi_k(X,A)=0$ and $\pi_0(A)\to\pi_0(X)$ is surjective. 
\item $\iota:A\hookrightarrow X$ is $n$-connected. 
\end{itemize}
\end{prp}

\begin{prp}{}{} Let $X,Y$ be spaces. Let $f:X\to Y$ be a map. Let $n\in\N$. If $X$ is not empty, then the following are equivalent. 
\begin{itemize}
\item $f$ is $n$-connected
\item $\text{hofiber}_y(f)$ is $n$-connected for all $y\in Y$. 
\item For all $0\leq k<n$ and all $x\in X$, the map $$\pi_k(f):\pi_k(X,x)\to\pi_k(Y,f(x))$$ is an isomorphism and $\pi_n(f):\pi_k(X,x)\to\pi_k(Y,f(x))$ is surjective. 
\end{itemize}
\end{prp}

Every map is $(-1)$-connected since every non-empty space is $(-1)$-connected. 

\begin{lmm}{}{} Let $X,Y,Z$ be spaces. Let $f:X\to Y$ and $g:Y\to Z$ be maps. Let $n\in\N$. If $f$ is $(n-1)$-connected and $g\circ f$ is $n$-connected, then $g$ is $n$-connected. \tcbline
\begin{proof}
We know that $g$ must be at least $(n-1)$-connected. It remains to show that $\pi_{n-1}(g)$ is an isomorphism and $\pi_n(g)$ is a surjection for all base points. \\~\\

Let $y\in Y$. If $y=f(x)$ for some $x\in X$, then $\pi_{n-1}(g\circ f):\pi_{n-1}(X,x)\to\pi_{n-1}(Z,g(f(x)))$ is an isomorphism. By funtoriality, $\pi_{n-1}(f):\pi_{n-1}(X,x)\to\pi_{n-1}(Y,f(x))$ is injective, and hence an isomorphism. Hence $\pi_{n-1}(g)$ is an isomorphism. Also, since $\pi_n(g\circ f)$ is a surjection, we know that $\pi_n(g)$ is a surjection. \\~\\

We now treat the cases where $y$ is not in the image of $f$. If $n\geq 1$, then $\pi_0(f)$ is a surjection. For any $y\in Y$, there exists $x\in X$ such that $f(x)$ and $y$ lie in the same path component. Then consider the following diagram: \\~\\
\adjustbox{scale=1,center}{\begin{tikzcd}
	{\pi_{n-1}(Y,f(x))} & {\pi_{n-1}(Z,g(f(x)))} \\
	{\pi_{n-1}(Y,y)} & {\pi_{n-1}(Z,g(y))}
	\arrow[from=1-1, to=1-2]
	\arrow["\cong"', from=1-1, to=2-1]
	\arrow["\cong", from=1-2, to=2-2]
	\arrow[from=2-1, to=2-2]
\end{tikzcd}}\\~\\
whose vertical maps are given by conjugation, and hence isomorphisms. By the above paragraph, the top arrow is an isomorphism, and hence the bottom arrow is also an isomorphism. The same method shows that $\pi_n(g)$ is surjective when the base point $y$ is not in the image of $f$. \\~\\

It remains to treat the case $n=0$. But notice that $\pi_0(g\circ f):\pi_0(X)\to\pi_0(Z)$ is surjective implies that $\pi_0(g)$ is surjective by functoriality of $\pi_0$. So we are done. 
\end{proof}
\end{lmm}

\begin{prp}{}{} Let $X,Y$ be spaces such that $X$ is $n$-connected and $Y$ is $m$-connected for $m,n\geq -1$. Then $X\wedge Y$ is $(n+m+1)$-connected. 
\end{prp}

\subsection{Weak Equivalences}
\begin{defn}{Weak Equivalences}{} Let $(X,x_0),(Y,y_0)$ be pointed spaces. Let $f:X\to Y$ be a pointed map. We say that $f$ is a weak equivalence if the following are true. 
\begin{itemize}
\item $\pi_n(f):\pi_n(X,x_0)\to\pi_n(Y,y_0)$ is an isomorphism for all $n\geq 1$. 
\item $\pi_0(f):\pi_0(X)\to\pi_0(Y)$ is a bijection. 
\end{itemize}
\end{defn}

\begin{prp}{}{} Let $X,Y$ be pointed spaces. Let $f:X\to Y$ be a map. Then $f$ is a weak equivalence if and only if for all CW complexes $C$, the induced map $$[C,X]_\ast\to[C,Y]_\ast$$ is a bijection. 
\end{prp}

2.6.17 highly non-trivial. 

Clearly $f$ is a weak equivalence if and only if $f$ is $n$-connected for all $n$. 

\begin{eg}{No Going Backwards}{} $S^1$ with pesudocircle. 
\end{eg}

\begin{prp}{Two-Out-Of-Three Property}{} Let $X,Y,Z$ be spaces. Let $f:X\to Y$ and $g:Y\to Z$ be maps. If any two of $f,g,g\circ f$ are weak equivalences, then so is the third one. \tcbline
\begin{proof}
If $f$ and $g$ are weak equivalences then it is clear that $g\circ f$ is a weak equivalence. If $g$ and $g\circ f$ are weak equivalence but $f$ is not, then there exists $n$ such that $\pi_n(f):\pi_n(X,x_0)\to\pi_n(Y,f(x_0))$ is not an isomorphism for some $x_0\in X$. Since $\pi_n(g):\pi_n(Y,f(x_0))\to\pi_n(Z,g(f(x_0)))$ is an isomorphism, this means that $\pi_n(g)\circ\pi_n(f)$ is not an isomorphism. This is a contradiction since $\pi_n(g\circ f)$ is a weak equivalence. \\~\\

Finally suppose that $f$ and $g\circ f$ are weak equivalences. If $f$ is surjective, suppose for a contradiction that $\pi_n(g):\pi_n(Y,y_0)\to\pi_n(Z,g(y_0))$ is not an isomorphism for some $n$ and $y_0\in Y$. Then $\pi_n(f)$ being an isomorphism implies that for any $x_0\in f^{-1}(y_0)$ ($f$ is surjective) then $\pi_n(g)\circ\pi_n(f)$ is not an isomorphism. This is a contradiction since $\pi_n(g\circ f)$ is an isomorphism. For the case that $f$ is not surjective, there exists $y_0\in Y$ that does not lie in $\im(f)$. Since $\pi_0(f)$ is an isomorphism, there exists some $x\in X$ such that $f(x)$ and $y$ lie in the same path component. Let $\gamma:I\to Y$ be a path from $f(x)$ to $y$. Then the following diagram commutes: \\~\\
\adjustbox{scale=1,center}{\begin{tikzcd}
	{\pi_n(Y,f(x))} & {\pi_n(Z,g(f(x)))} \\
	{\pi_n(Y,y_0)} & {\pi_n(Z,g(y_0))}
	\arrow[from=1-1, to=1-2]
	\arrow["\cong"', from=1-1, to=2-1]
	\arrow["\cong", from=1-2, to=2-2]
	\arrow[from=2-1, to=2-2]
\end{tikzcd}}\\~\\
where the vertical maps are induced by conjugation with $\gamma$, and therefore the diagram is commutative. From the case where we assumed that $f$ is surjective, we can conclude that if $f$ and $g\circ f$ are weak equivalences, then $g|_{\im(f)}$ is a weak equivalence. Hence the top arrow is an isomorphism. Since three out of the four arrows in the commutative diagram is an isomorphism, the last arrow is also an isomorphism. Hence $g$ is a weak equivalence. 
\end{proof}
\end{prp}

\begin{thm}{Restatement of Feudenthal's Suspension Theorem}{} Let $X$ be a space. Let $n\in\N$. Then for all $k>n+1$, the suspension map induces isomorphisms $$\pi_{n+k}(\Sigma^kX)\cong\pi_{n+k+1}(\Sigma^{k+1}X)$$
\end{thm}

\subsection{Fibrations and Cofibrations}
\begin{defn}{Fibrations}{}
\end{defn}

Replacing maps with fibrations: 

\begin{prp}{}{}
\end{prp}

Fibers of fibrations are homotopy equivalent, and all such are homotopy equivalent to the homotopy fiber. 

\begin{lmm}{}{}
\end{lmm}

\begin{prp}{}{} Let $W,X,Y,Z$ be spaces such that the following diagram is a pullback: \\~\\
\adjustbox{scale=1,center}{\begin{tikzcd}
	W & Y \\
	X & Z
	\arrow[from=1-1, to=1-2]
	\arrow[from=1-1, to=2-1]
	\arrow[from=1-2, to=2-2]
	\arrow[from=2-1, to=2-2]
\end{tikzcd}}\\~\\
If $Y\to Z$ is a fibration, then $W\to X$ is a fibration. 
\end{prp}

\begin{prp}{}{} Let $p:(E,e_0)\to(B,b_0)$ be a fibration with fiber $(F=f^{-1}(b_0),e_0)$. Then there exists a natural long exact sequence: \\~\\
\adjustbox{scale=0.85,center}{\begin{tikzcd}
	\cdots & {\pi_n(F,e_0)} & {\pi_n(E,e_0)} & {\pi_n(B,b_0)} & {\pi_{n-1}(F,e_0)} & \cdots & {\pi_0(E,e_0)} & {\pi_0(B,b_0)}
	\arrow[from=1-1, to=1-2]
	\arrow[from=1-2, to=1-3]
	\arrow[from=1-3, to=1-4]
	\arrow[from=1-4, to=1-5]
	\arrow[from=1-5, to=1-6]
	\arrow[from=1-6, to=1-7]
	\arrow[from=1-7, to=1-8]
\end{tikzcd}}\\~\\
(need to define exactness of sets). 
\end{prp}

\begin{prp}{}{} Let $W,X,Y,Z$ be spaces such that the following diagram is a pullback: \\~\\
\adjustbox{scale=1,center}{\begin{tikzcd}
	W & Y \\
	X & Z
	\arrow[from=1-1, to=1-2]
	\arrow[from=1-1, to=2-1]
	\arrow[from=1-2, to=2-2]
	\arrow[from=2-1, to=2-2]
\end{tikzcd}}\\~\\
Let $Y\to Z$ be a fibration. If $X\to Z$ is a weak equivalence, then $W\to Y$ is a weak equivalence. 
\end{prp}

\begin{prp}{}{} Let $W,X,Y,Z$ be spaces such that the following diagram is a pullback: \\~\\
\adjustbox{scale=1,center}{\begin{tikzcd}
	W & Y \\
	X & Z
	\arrow[from=1-1, to=1-2]
	\arrow[from=1-1, to=2-1]
	\arrow[from=1-2, to=2-2]
	\arrow[from=2-1, to=2-2]
\end{tikzcd}}\\~\\
Let $Y\to Z$ be a fibration. If $X\to Z$ is a homotopy equivalence, then $W\to Y$ is a homotopy equivalence. 
\end{prp}

\begin{prp}{}{} Let $X,Y,Z,V$ be spaces such that we have the following diagram: \\~\\
\adjustbox{scale=1,center}{\begin{tikzcd}
	& X \\
	& Y \\
	V & Z
	\arrow[from=1-2, to=2-2]
	\arrow[from=2-2, to=3-2]
	\arrow[from=3-1, to=3-2]
\end{tikzcd}}\\~\\
If $V\to Z$ is a fibration and $X\to Y$ is a weak equivalence, then the canonical map $$\lim(V\rightarrow Z\leftarrow X)\to\lim(V\rightarrow Z\leftarrow Y)$$ induced by the universal property is a weak equivalence. \tcbline
\begin{proof}
Follows from 4.3.6. 
\end{proof}
\end{prp}

\subsection{Sequential Homotopy Colimits}
Recall that the standard homotopy pushout of $X\leftarrow W\rightarrow Y$ can be equivalent defined as the pushout of $M_{W\to X}\leftarrow W\rightarrow M_{W\to Y}$. Here we have replaced the maps with cofibrations. Motivated by this, we take the following as our standard model for sequantial homotopy colimits. 

\begin{defn}{A Model for Sequential Homotopy Colimits}{} Let the following be a sequence of spaces and maps: \\~\\
\adjustbox{scale=1,center}{\begin{tikzcd}
	{X_0} & {X_1} & {X_2} & \cdots
	\arrow["{f_0}", from=1-1, to=1-2]
	\arrow["{f_1}", from=1-2, to=1-3]
	\arrow[from=1-3, to=1-4]
\end{tikzcd}}\\~\\
Define $M_n$ inductively as follows. 
\begin{itemize}
\item When $n=0$, take $M_0=X_0$. 
\item When $n=1$, take $M_1=M_{f_0}$. Notice that we obtain a map $M_{f_0}\to X_1$ through the factorization of $f_0$ through $M_{f_0}$. Moreover we have a commutative diagram: \\~\\
\adjustbox{scale=1,center}{\begin{tikzcd}
	{M_0} & {M_1} \\
	{X_0} & {X_1} & {X_2} & \cdots
	\arrow[from=1-1, to=1-2]
	\arrow[from=1-1, to=2-1]
	\arrow[from=1-2, to=2-2]
	\arrow[from=2-1, to=2-2]
	\arrow[from=2-2, to=2-3]
	\arrow[from=2-3, to=2-4]
\end{tikzcd}}\\~\\
Also notice that we have a map $X_1\to M_1=M_{f_0}$ given by inclusion. 
\item Assume that $M_k$ has been constructed so that we have commutative diagrams: \\~\\
\adjustbox{scale=1,center}{\begin{tikzcd}
	{M_0} & {M_1} & \cdots & {M_{k-1}} & {M_k} \\
	{X_0} & {X_1} & \cdots & {X_{k-1}} & {X_k} & {X_{k+1}} & \cdots
	\arrow[from=1-1, to=1-2]
	\arrow[from=1-1, to=2-1]
	\arrow[from=1-2, to=1-3]
	\arrow[from=1-2, to=2-2]
	\arrow[from=1-3, to=1-4]
	\arrow[from=1-4, to=1-5]
	\arrow[from=1-4, to=2-4]
	\arrow[from=1-5, to=2-5]
	\arrow[from=2-1, to=2-2]
	\arrow[from=2-2, to=2-3]
	\arrow[from=2-3, to=2-4]
	\arrow[from=2-4, to=2-5]
	\arrow[from=2-5, to=2-6]
	\arrow[from=2-6, to=2-7]
\end{tikzcd}}\\~\\
Also assume we have maps $M_k\to X_k$, and assume that $M_{f_{k-1}}\to M_k\to X_k$ is equal to the map $M_{f_{k-1}}\to X_k$ in the factorization of $f_k$. Define $M_{k+1}$ to be the pushout $$M_{k+1}=\lim(M_k\leftarrow X_k\rightarrow M_{f_k})$$ where $X_k\to M_k$ is the map given by the composite of the inclusion $X_k\to M_{f_{k-1}}$ with the induced map $M_{f_{k-1}}\to M_k$ given by the property that $M_k$ is a pushout. Notice that we obtain a commutative diagram \\~\\
\adjustbox{scale=1,center}{\begin{tikzcd}
	{M_k} & {M_{k+1}} \\
	{X_k} & {X_{k+1}}
	\arrow[from=1-1, to=1-2]
	\arrow[from=1-1, to=2-1]
	\arrow[from=1-2, to=2-2]
	\arrow[from=2-1, to=2-2]
\end{tikzcd}}\\~\\
The map $M_{k+1}\to X_{k+1}$ is given by the universal property of $M_{k+1}$ as a pushout, since the map $X_k\to M_{f_{k-1}}\to M_k\to X_k\to X_{k+1}$ is assumed to be equal to the map $X_k\to M_{f_k}\to X_{k+1}$ by inductive hypothesis, thus giving us a commutative diagram: \\~\\
\adjustbox{scale=1,center}{\begin{tikzcd}
	{X_k} & {M_{f_k}} \\
	{M_{f_{k-1}}} \\
	{M_k} & {M_{k+1}} \\
	&& {X_{k+1}}
	\arrow[from=1-1, to=1-2]
	\arrow[from=1-1, to=2-1]
	\arrow[from=1-2, to=3-2]
	\arrow[bend left = 20, from=1-2, to=4-3]
	\arrow[from=2-1, to=3-1]
	\arrow[from=3-1, to=3-2]
	\arrow[bend right = 20, from=3-1, to=4-3]
	\arrow["\exists"{description}, dashed, from=3-2, to=4-3]
\end{tikzcd}}\\~\\
The commutativity of the square in question follows also from this diagram since the map $M_k\to X_{k+1}$ factors through $X_k$. 

To complete induction, we want to prove that $M_{f_k}\to M_{k+1}\to X_{k+1}$ is equal to the map $M_{f_k}\to X_k$ in the factorization of $f_k$. But the above same diagram proves it (by considering the triangle involving $M_{f_k}, M_{k+1}$ and $X_{k+1}$). 
\end{itemize}
Define the homotopy colimit as $$\text{hocolim}(X_0\to X_1\to X_2\to\cdots)=\text{colim}(M_0\to M_1\to M_2\to\cdots)$$
\end{defn}

This definition is inspired by the standard model for homotopy pushotus in the following way. Notice that $X_0=M_0\to M_1=M_{f_0}$ is a cofibration by construction. Moreover, since \\~\\
\adjustbox{scale=1,center}{\begin{tikzcd}
	{X_k} & {M_{f_k}} \\
	{M_k} & {M_{k+1}}
	\arrow[from=1-1, to=1-2]
	\arrow[from=1-1, to=2-1]
	\arrow[from=1-2, to=2-2]
	\arrow[from=2-1, to=2-2]
\end{tikzcd}}\\~\\
is a pushout diagram, $X_k\to M_{f_k}$ being a cofibration implies that $M_k\to M_{k+1}$ is a cofibration. \\~\\

Also, notice that the maps $M_0\to X_0$ and $M_{f_0}=M_1\to X_1$ are weak equivalences. Now assume that $M_{f_{k-1}}\to M_k$ is a weak equivalence for some $k$, consider the same pushout diagram: \\~\\
\adjustbox{scale=1,center}{\begin{tikzcd}
	{X_k} & {M_{f_k}} \\
	{M_k} & {M_{k+1}}
	\arrow[from=1-1, to=1-2]
	\arrow[from=1-1, to=2-1]
	\arrow[from=1-2, to=2-2]
	\arrow[from=2-1, to=2-2]
\end{tikzcd}}\\~\\
Recall that the map $X_k\to M_k$ factors through $M_{f_{k-1}}$ and the map $X_k\to M_{f_{k-1}}$ is given by inclusion. Hence it is precisely the homotopy inverse of the homotopy equivalence $M_{f_{k-1}}\to X_k$. Since we assume that $M_{f_{k-1}}\to M_k$ is a weak equivalence, the two out of three property implies that $X_k\to M_k$ is a weak equivalence. Moreover, since $X_k\to M_{f_k}$ is a cofibration, we conclude that $M_{f_k}\to M_{k+1}$ is a weak equivalence. By construction the weak equivalence $M_{f_k}\to X_{k+1}$ factors through the map $M_{f_k}\to M_{k+1}$, which is a weak equivalence. Therefore by the universal property, the map $M_{f_k}\to X_k$ is a weak equivalence. \\~\\

All this is to say that we have replaced the sequence of spaces $X_0\to X_1\to X_2\to\cdots$ into a sequence of cofibrations $M_0\to M_1\to M_2\to\cdots$ equipped with weak equivalences $M_n\to X_n$ making the obvious diagram commute. 

\begin{prp}{}{} Let the following be a sequence of spaces and maps: \\~\\
\adjustbox{scale=1,center}{\begin{tikzcd}
	{X_0} & {X_1} & {X_2} & \cdots
	\arrow["{f_0}", from=1-1, to=1-2]
	\arrow["{f_1}", from=1-2, to=1-3]
	\arrow[from=1-3, to=1-4]
\end{tikzcd}}\\~\\
Then there is an isomorphism $$\pi_n(\underset{k}{\text{hocolim}}X_k)\cong\underset{k}{\text{colim}}\pi_n(X_k)$$ \cite{FSHT}
\end{prp}

\begin{prp}{}{} hocolim commutes with $\Omega$. 
\end{prp}

\subsection{Some Facts About Infinity Categories}
\begin{prp}{}{} The infinity categories $\mS$ and $\bold{Cat}_\infty$ is complete and cocomplete. 
\end{prp}

Fabian I.35\\~\\

Limits of functors can be computed pointwise: 

\begin{prp}{}{} Let $\mC,\mD,\mE$ be infinity categories. Let $F:\mD\to\mE$ be a functor. Then the induced functor $$-\circ F:\text{Func}(\mE,\mC)\to\text{Func}(\mD,\mC)$$ preserves limits and colimits. 
\end{prp}

In particular, by choosing the inclusion functor $\{d\}\hookrightarrow\mD$, we obtain the (co)limit preserving functor $$\text{ev}_d:\text{Func}(\mD,\mC)\to\text{Func}(\{d\},\mC)\simeq\mC$$ Now given any diagram $X:K\to\text{Func}(\mD,\mC)$ that admits a limit, $\lim_K X$ is an object $\text{Func}(\mD,\mC)$. To compute the value of this functor at $d\in\mD$, we use the evaluation map to get $$\left(\lim_K X\right)(d)=\text{ev}_d\left(\lim_KX\right)=\lim_K(\text{ev}_d\circ X)$$ where the limit on the right is now an object in $\mC$, and the diagram of the limit is given on objects by $F(d)$ for all $F$ in the image of $X$. (I.39 Fabian)





\end{document}
