\documentclass[a4paper]{article}

\input{C:/Users/liula/Desktop/Latex/Headers.tex}

\pagestyle{fancy}
\fancyhf{}
\rhead{Labix}
\lhead{Set Theory}
\rfoot{\thepage}

\title{Set Theory}

\author{Labix}

\date{\today}
\begin{document}
\maketitle
\begin{abstract}
These notes aim to develop basic notions of sets and logics following closely to ZFC set theory, as well as introducing a few examples and provide proofs for theorems that require careful inspection. Theorems that with proofs ommitted are expected to have readers be able to prove them. Although that there is a wide variety of axiomatic set theories that are accepted by different mathematicians, ZFC set theory is considered to be one of the most widely recognized theories among them all. \\~\\
Beware that while set theory has no formal prequisites in terms of its content (except perhaps logic theory), the mathematical concepts are not at all easy to understand. In fact the more foundational the mathematics, the harder the proofs and the more abstract the content becomes. First year university students should aim to complete up to chapter $3$ in order to develop the suitable language for further mathematical content in their degrees. \\~\\
Unfortunately these notes are not up to date: the Axiom Schema of Replacement is missing and descriptions and explanations for chapter 3 onwards is incomplete. 
Famous mathematicians who contributed to this area in mathematics include Ernst Zermelo and Abraham Fraenkel, Kurt Gödel and Georg Cantor and many more. \\~\\
\textbf{References}
\begin{itemize}
\item Naive Set Theory by Paul R. Halmos
\end{itemize}
\end{abstract}
\pagebreak
\tableofcontents
\pagebreak
\input{C:/Users/liula/Desktop/Latex/Sets and Numbers/Set Theory/Set Theory Content.tex}
\end{document}