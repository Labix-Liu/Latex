\documentclass[a4paper]{article}

\input{C:/Users/liula/Desktop/Latex/Headers V1.2.tex}

\pagestyle{fancy}
\fancyhf{}
\rhead{Labix}
\lhead{Enriched Category Theory}
\rfoot{\thepage}

\title{Enriched Category Theory}

\author{Labix}

\date{\today}
\begin{document}
\maketitle
\begin{abstract}
\end{abstract}
\pagebreak
\tableofcontents

\pagebreak
\section{Algebraic Objects in a Category}
\subsection{Group Objects}
\begin{defn}{Group Objects}{} Let $\mC$ be a category with finite products. We say that $G\in\mC$ is a group object if there exists three morphisms 
\begin{itemize}
\item Multiplication: $m:G\times G\to G$
\item Identity: $e:\ast\to G$ where $\ast$ is the terminal object
\item Inverse: $\text{inv}:G\to G$
\end{itemize}
such that the following diagrams commute. 
\begin{itemize}
\item Associativity: \\~\\
\adjustbox{scale=1.0,center}{\begin{tikzcd}
	{G\times G\times G} & {G\times G} \\
	{G\times G} & G
	\arrow["{m\times\text{id}_G}", from=1-1, to=1-2]
	\arrow["{\text{id}_G\times m}"', from=1-1, to=2-1]
	\arrow["m", from=1-2, to=2-2]
	\arrow["m"', from=2-1, to=2-2]
\end{tikzcd}}\\~\\
\item Identity: \\~\\
\adjustbox{scale=1.0,center}{\begin{tikzcd}
	G & {G\times G} \\
	{G\times G} & G
	\arrow["{(e,\text{id}_G)}", from=1-1, to=1-2]
	\arrow["{(\text{id}_G,e)}"', from=1-1, to=2-1]
	\arrow["{\text{id}_G}"{description}, from=1-1, to=2-2]
	\arrow["m", from=1-2, to=2-2]
	\arrow["m"', from=2-1, to=2-2]
\end{tikzcd}}\\~\\
\item Inverse: \\~\\
\adjustbox{scale=1.0,center}{\begin{tikzcd}
	G & {G\times G} \\
	{G\times G} & G
	\arrow["{(\text{inv},\text{id}_G)}", from=1-1, to=1-2]
	\arrow["{(\text{id}_G,\text{inv})}"', from=1-1, to=2-1]
	\arrow["e"{description}, from=1-1, to=2-2]
	\arrow["m", from=1-2, to=2-2]
	\arrow["m"', from=2-1, to=2-2]
\end{tikzcd}}\\~\\
\end{itemize}
\end{defn}

\begin{prp}{}{} A group object in the category $\bold{Set}$ of sets is a group in the usual sense. 
\end{prp}

\begin{prp}{}{} A group object in the category $\bold{Grp}$ of groups is an abelian group. 
\end{prp}

\pagebreak
\section{Monoidal Categories}
\subsection{Strict and Weak Monoidal Categories}
\begin{defn}{Strict Monoidal Categories}{} A strict monoidal category is a category $\mA$ consisting of a bifunctor $\otimes:\mA\times\mA\to\mA$ together with an object $I\in\mA$ such that the following are true. 
\begin{itemize}
\item Associativity: $(A\otimes B)\otimes C=A\otimes(B\otimes C)$
\item Identity: $I\otimes A=A$ and $A\otimes I=A$
\end{itemize}
\end{defn}

Notice that we require strict equality in the associativity and identity laws. Since we usually only consider objects up to isomorphism in a category, strict monoidal categories may seem quite rare in practise. 

\begin{defn}{Weak Monoidal Category}{} A weak monoidal category is a category $\mA$ consisting of a bifunctor $\otimes:\mA\times\mA\to\mA$ together with an object $I\in\mA$ such that the following are true. 
\begin{itemize}
\item Associativity: There are isomorphisms $$\alpha_{A,B,C}:(A\otimes B)\otimes C\overset{\cong}{\longrightarrow} A\otimes(B\otimes C)$$ that is natural in $A$, $B$ and $C$
\item Identity: There are isomorphisms $$\lambda_A:I\otimes A\overset{\cong}{\longrightarrow} A\;\;\;\;\text{ and }\;\;\;\;\rho_A:A\otimes I\overset{\cong}{\longrightarrow} A$$ that are both natural in $A$
\end{itemize}
Such natural isomorphisms must also satisfy the following commutative laws: 
\begin{itemize}
\item The pentagon identity: \\~\\
\adjustbox{scale=1.0,center}{\begin{tikzcd}
	& {(A\otimes B)\otimes(C\otimes D)} \\
	\\
	{((A\otimes B)\otimes C)\otimes D} && {A\otimes (B\otimes(C\otimes D))} \\
	\\
	\\
	{(A\otimes(B\otimes C))\otimes D} && {A\otimes((B\otimes C)\otimes D)}
	\arrow["{\alpha_{A,B,C\otimes D}}", from=1-2, to=3-3]
	\arrow["{\alpha_{A\otimes B,C,D}}", from=3-1, to=1-2]
	\arrow["{\alpha_{A,B,C}\otimes 1_D}"', from=3-1, to=6-1]
	\arrow["{\alpha_{A,B\otimes C,D}}"', from=6-1, to=6-3]
	\arrow["{1_A\otimes\alpha_{B,C,D}}"', from=6-3, to=3-3]
\end{tikzcd}}\\~\\
\item The triangle identity: \\~\\
\adjustbox{scale=1.0,center}{\begin{tikzcd}
	{(A\otimes I)\otimes B} && {A\otimes (I\otimes B)} \\
	\\
	& {A\otimes B}
	\arrow["{\alpha_{A,I,B}}", from=1-1, to=1-3]
	\arrow["{\rho_A\otimes 1_B}"', from=1-1, to=3-2]
	\arrow["{1_A\otimes\lambda_B}", from=1-3, to=3-2]
\end{tikzcd}}\\~\\
\end{itemize}
\end{defn}

It is clear that every strict monoidal category is also a weak monoidal category. 

\begin{lmm}{}{} Every category $\mC$ with finite products is a monoidal category with product $\times:\mC\times\mC\to\mC$ and identity $\ast$ the terminal object. 
\end{lmm}

\begin{prp}{}{} For any commutative ring $R$, the category $\bold{Mod}_R$ of $R$-modules is a monoidal category with the tensor product $\otimes$ and the identity object $R$. 
\end{prp}

\begin{defn}{Symmetric Monoidal Category}{} Let $\mC$ be a category. We say that $\mC$ is a symmetric monoidal category if $\mC$ is a weak monoidal category together with isomorphisms $$s_{A,B}:A\otimes B\overset{\cong}{\longrightarrow} B\otimes A$$ that are natural in $A$ and $B$ such that the following are satisfied: 
\begin{itemize}
\item Unit coherence: If $I$ is the distinguished object of $\mC$ as a weak monoidal category, then the following diagram commutes: \\~\\
\adjustbox{scale=1.0,center}{\begin{tikzcd}
	{A\otimes I} && {I\otimes A} \\
	& A
	\arrow["{s_{A,I}}", from=1-1, to=1-3]
	\arrow["{\lambda_A}"', from=1-1, to=2-2]
	\arrow["{\rho_A}", from=1-3, to=2-2]
\end{tikzcd}}\\~\\
\item The associativity coherence: For any $A,B,C\in\mC$, the following diagram commutes: \\~\\
\adjustbox{scale=1.0,center}{\begin{tikzcd}
	{(A\otimes B)\otimes C} && {(B\otimes A)\otimes C} \\
	{A\otimes(B\otimes C)} && {B\otimes(A\otimes C)} \\
	{(B\otimes C)\otimes A} && {B\otimes(C\otimes A)}
	\arrow["{s_{A,B}\otimes\text{id}_C}", from=1-1, to=1-3]
	\arrow["{\alpha_{A,B,C}}"', from=1-1, to=2-1]
	\arrow["{\alpha_{B,A,C}}", from=1-3, to=2-3]
	\arrow["{s_{A,B\otimes C}}"', from=2-1, to=3-1]
	\arrow["{\text{id}_B\otimes s_{A,C}}", from=2-3, to=3-3]
	\arrow["{\alpha_{B,C,A}}"', from=3-1, to=3-3]
\end{tikzcd}}\\~\\
\item The inverse law: For any $A,B\in\mC$, the following diagram commutes: \\~\\
\adjustbox{scale=1.0,center}{\begin{tikzcd}
	& {B\otimes A} \\
	{A\otimes B} && {A\otimes B }
	\arrow["{s_{B,A}}", from=1-2, to=2-3]
	\arrow["{s_{A,B}}", from=2-1, to=1-2]
	\arrow["{\text{id}_{A\otimes B}}"', from=2-1, to=2-3]
\end{tikzcd}}\\~\\
\end{itemize}
\end{defn}

\subsection{Closed Categories}

\pagebreak
\section{Enriched Categories}




















\end{document}