\documentclass[a4paper]{article}

%=========================================
% Packages
%=========================================
\usepackage{mathtools}
\usepackage{amsfonts}
\usepackage{amsmath}
\usepackage{amssymb}
\usepackage{amsthm}
\usepackage[a4paper, total={6in, 8in}, margin=1in]{geometry}
\usepackage[utf8]{inputenc}
\usepackage{fancyhdr}
\usepackage[utf8]{inputenc}
\usepackage{graphicx}
\usepackage{physics}
\usepackage[listings]{tcolorbox}
\usepackage{hyperref}
\usepackage{tikz-cd}
\usepackage{adjustbox}
\usepackage{enumitem}
\usepackage[font=small,labelfont=bf]{caption}
\usepackage{subcaption}
\usepackage{wrapfig}
\usepackage{makecell}



\raggedright

\usetikzlibrary{arrows.meta}

\DeclarePairedDelimiter\ceil{\lceil}{\rceil}
\DeclarePairedDelimiter\floor{\lfloor}{\rfloor}

%=========================================
% Fonts
%=========================================
\usepackage{tgpagella}
\usepackage[T1]{fontenc}


%=========================================
% Custom Math Operators
%=========================================
\DeclareMathOperator{\adj}{adj}
\DeclareMathOperator{\im}{im}
\DeclareMathOperator{\nullity}{nullity}
\DeclareMathOperator{\sign}{sign}
\DeclareMathOperator{\dom}{dom}
\DeclareMathOperator{\lcm}{lcm}
\DeclareMathOperator{\ran}{ran}
\DeclareMathOperator{\ext}{Ext}
\DeclareMathOperator{\dist}{dist}
\DeclareMathOperator{\diam}{diam}
\DeclareMathOperator{\aut}{Aut}
\DeclareMathOperator{\inn}{Inn}
\DeclareMathOperator{\syl}{Syl}
\DeclareMathOperator{\edo}{End}
\DeclareMathOperator{\cov}{Cov}
\DeclareMathOperator{\vari}{Var}
\DeclareMathOperator{\cha}{char}
\DeclareMathOperator{\Span}{span}
\DeclareMathOperator{\ord}{ord}
\DeclareMathOperator{\res}{res}
\DeclareMathOperator{\Hom}{Hom}
\DeclareMathOperator{\Mor}{Mor}
\DeclareMathOperator{\coker}{coker}
\DeclareMathOperator{\Obj}{Obj}
\DeclareMathOperator{\id}{id}
\DeclareMathOperator{\GL}{GL}
\DeclareMathOperator*{\colim}{colim}

%=========================================
% Custom Commands (Shortcuts)
%=========================================
\newcommand{\CP}{\mathbb{CP}}
\newcommand{\GG}{\mathbb{G}}
\newcommand{\F}{\mathbb{F}}
\newcommand{\N}{\mathbb{N}}
\newcommand{\Q}{\mathbb{Q}}
\newcommand{\R}{\mathbb{R}}
\newcommand{\C}{\mathbb{C}}
\newcommand{\E}{\mathbb{E}}
\newcommand{\Prj}{\mathbb{P}}
\newcommand{\RP}{\mathbb{RP}}
\newcommand{\T}{\mathbb{T}}
\newcommand{\Z}{\mathbb{Z}}
\newcommand{\A}{\mathbb{A}}
\renewcommand{\H}{\mathbb{H}}
\newcommand{\K}{\mathbb{K}}

\newcommand{\mA}{\mathcal{A}}
\newcommand{\mB}{\mathcal{B}}
\newcommand{\mC}{\mathcal{C}}
\newcommand{\mD}{\mathcal{D}}
\newcommand{\mE}{\mathcal{E}}
\newcommand{\mF}{\mathcal{F}}
\newcommand{\mG}{\mathcal{G}}
\newcommand{\mH}{\mathcal{H}}
\newcommand{\mI}{\mathcal{I}}
\newcommand{\mJ}{\mathcal{J}}
\newcommand{\mK}{\mathcal{K}}
\newcommand{\mL}{\mathcal{L}}
\newcommand{\mM}{\mathcal{M}}
\newcommand{\mO}{\mathcal{O}}
\newcommand{\mP}{\mathcal{P}}
\newcommand{\mS}{\mathcal{S}}
\newcommand{\mT}{\mathcal{T}}
\newcommand{\mV}{\mathcal{V}}
\newcommand{\mW}{\mathcal{W}}

%=========================================
% Colours!!!
%=========================================
\definecolor{LightBlue}{HTML}{2D64A6}
\definecolor{ForestGreen}{HTML}{4BA150}
\definecolor{DarkBlue}{HTML}{000080}
\definecolor{LightPurple}{HTML}{cc99ff}
\definecolor{LightOrange}{HTML}{ffc34d}
\definecolor{Buff}{HTML}{DDAE7E}
\definecolor{Sunset}{HTML}{F2C57C}
\definecolor{Wenge}{HTML}{584B53}
\definecolor{Coolgray}{HTML}{9098CB}
\definecolor{Lavender}{HTML}{D6E3F8}
\definecolor{Glaucous}{HTML}{828BC4}
\definecolor{Mauve}{HTML}{C7A8F0}
\definecolor{Darkred}{HTML}{880808}
\definecolor{Beaver}{HTML}{9A8873}
\definecolor{UltraViolet}{HTML}{52489C}



%=========================================
% Theorem Environment
%=========================================
\tcbuselibrary{listings, theorems, breakable, skins}

\newtcbtheorem[number within = subsection]{thm}{Theorem}%
{	colback=Buff!3, 
	colframe=Buff, 
	fonttitle=\bfseries, 
	breakable, 
	enhanced jigsaw, 
	halign=left
}{thm}

\newtcbtheorem[number within=subsection, use counter from=thm]{defn}{Definition}%
{  colback=cyan!1,
    colframe=cyan!50!black,
	fonttitle=\bfseries, breakable, 
	enhanced jigsaw, 
	halign=left
}{defn}

\newtcbtheorem[number within=subsection, use counter from=thm]{axm}{Axiom}%
{	colback=red!5, 
	colframe=Darkred, 
	fonttitle=\bfseries, 
	breakable, 
	enhanced jigsaw, 
	halign=left
}{axm}

\newtcbtheorem[number within=subsection, use counter from=thm]{prp}{Proposition}%
{	colback=LightBlue!3, 
	colframe=Glaucous, 
	fonttitle=\bfseries, 
	breakable, 
	enhanced jigsaw, 
	halign=left
}{prp}

\newtcbtheorem[number within=subsection, use counter from=thm]{lmm}{Lemma}%
{	colback=LightBlue!3, 
	colframe=LightBlue!60, 
	fonttitle=\bfseries, 
	breakable, 
	enhanced jigsaw, 
	halign=left
}{lmm}

\newtcbtheorem[number within=subsection, use counter from=thm]{crl}{Corollary}%
{	colback=LightBlue!3, 
	colframe=LightBlue!60, 
	fonttitle=\bfseries, 
	breakable, 
	enhanced jigsaw, 
	halign=left
}{crl}

\newtcbtheorem[number within=subsection, use counter from=thm]{eg}{Example}%
{	colback=Beaver!5, 
	colframe=Beaver, 
	fonttitle=\bfseries, 
	breakable, 
	enhanced jigsaw, 
	halign=left
}{eg}

\newtcbtheorem[number within=subsection, use counter from=thm]{ex}{Exercise}%
{	colback=Beaver!5, 
	colframe=Beaver, 
	fonttitle=\bfseries, 
	breakable, 
	enhanced jigsaw, 
	halign=left
}{ex}

\newtcbtheorem[number within=subsection, use counter from=thm]{alg}{Algorithm}%
{	colback=UltraViolet!5, 
	colframe=UltraViolet, 
	fonttitle=\bfseries, 
	breakable, 
	enhanced jigsaw, 
	halign=left
}{alg}




%=========================================
% Hyperlinks
%=========================================
\hypersetup{
    colorlinks=true, %set true if you want colored links
    linktoc=all,     %set to all if you want both sections and subsections linked
    linkcolor=DarkBlue,  %choose some color if you want links to stand out
}


\pagestyle{fancy}
\fancyhf{}
\rhead{Labix}
\lhead{Operad Theory}
\rfoot{\thepage}

\title{Operad Theory}

\author{Labix}

\date{\today}
\begin{document}
\maketitle
\begin{abstract}
\begin{itemize}
\end{itemize}
\end{abstract}
\pagebreak
\tableofcontents
\pagebreak

\section{The Theory of Operads}
\subsection{General Operads}
\begin{defn}{(Planar) Operads}{} Let $(\mC,\otimes,I)$ be a symmetric monoidal category. An operad in $\mC$ consists of the following data
\begin{itemize}
\item A sequence $P=\{P(n)\;|\;n\in\N\}$ of objects in $\mC$
\item A composition function $$\gamma:P(n)\otimes P(k_1)\otimes\cdots\otimes P(k_n)\to P(k_1+\cdots+k_n)$$
\item An unit $\mu:I\to P(1)$
\end{itemize}
such that the following compatibility conditions are satisfied. 
\begin{itemize}
\item Associativity in the first symmetric product: \\~\\
\adjustbox{scale=0.8,center}{\begin{tikzcd}
	{P(n)\otimes\left(\bigotimes_{k=1}^n\left(P(i_k)\otimes\bigotimes_{t=1}^{i_k}P(j_{k,t})\right)\right)} & {P(n)\otimes\left(\bigotimes_{k=1}^nP\left(\sum_{t=1}^{i_k}j_{k,t}\right)\right)} \\
	&& {P\left(\sum_{k=1}^n\sum_{t=1}^{i_k}j_{k,t}\right)} \\
	{P(n)\otimes\left(\bigotimes_{k=1}^nP(j_k)\right)\otimes\left(\bigotimes_{k=1}^n\bigotimes_{t=1}^{t_k}P(j_{k,t})\right)} & {P\left(\sum_{u=1}^ki_u\right)\otimes\left(\bigotimes_{k=1}^n\bigotimes_{t=1}^{i_k}P(j_{k,t})\right)}
	\arrow["{\text{id}_{P(n)}\otimes\gamma}"', from=1-1, to=1-2]
	\arrow["\cong", from=1-1, to=3-1]
	\arrow["\gamma"', from=1-2, to=2-3]
	\arrow["{\gamma\otimes\text{id}_{\otimes\otimes}}", from=3-1, to=3-2]
	\arrow["\gamma", from=3-2, to=2-3]
\end{tikzcd}}\\~\\
\item Unitality: \\~\\
\adjustbox{scale=1,center}{\begin{tikzcd}
	{P(n)\otimes P(1)^{\otimes n}} & {P(n)\otimes I^{\otimes n}} & {I\otimes P(n)} & {P(1)\otimes P(n)} \\
	& {P(n)} & {P(n)}
	\arrow["\gamma"', from=1-1, to=2-2]
	\arrow["{\text{id}_{P(n)}\otimes\mu^{\otimes n}}"', from=1-2, to=1-1]
	\arrow["\cong", from=1-2, to=2-2]
	\arrow["{\mu\otimes\text{id}_{P(n)}}", from=1-3, to=1-4]
	\arrow["\cong"', from=1-3, to=2-3]
	\arrow["\gamma", from=1-4, to=2-3]
\end{tikzcd}}\\~\\
\end{itemize}
\end{defn}

TBA: An operad is a monoid in the monoidal category $(\text{Func}(\bold{S},\mC),\circ,I)$

\begin{defn}{Symmetric Operads}{} Let $(\mC,\otimes)$ be a symmetric monoidal category. A symmetric operad is an operad $(P,\gamma,\mu)$ on $\mC$ such that the following are true. 
\begin{itemize}
\item Each $P(n)$ is an $S_n$-module for each $n\in\N$. 
\item $\gamma:P(n)\otimes P(k_1)\otimes\cdots\otimes P(k_n)\to P(k_1+\cdots+k_n)$ is equivariant in the following sense: 
\begin{align*}
\gamma(c\cdot\sigma,d_1,\dots,d_n)&=\gamma(c,d_{\sigma^{-1}(1)},\dots,d_{\sigma^{-1}(n)})\cdot\sigma(k_1,\dots,k_n)\\
\gamma(c,d_1\cdot\tau_1,\dots,d_n\cdot\tau_n)&=\gamma(c,d_1,\dots,d_n)\cdot(\tau_1\oplus\cdots\oplus\tau_n)
\end{align*}
\end{itemize}
\end{defn}

The symmetric monoidal category one usually considers are algebraic. For instance, ${_R}\bold{Mod}$, $\bold{Ch}(R)_{\geq 0}$

\begin{defn}{Morphisms of Operads}{} Let $(\mC,\otimes,I)$ be a symmetric monoidal category. Let $(P,\gamma,\mu)$ and $(P,\gamma',\mu')$ be operads in $\mC$. A morphism $\varphi:P\to P'$ of operads consists of the following data. 
\begin{itemize}
\item For each $n\in\N$, a map $\varphi_n:P(n)\to P'(n)$ in $\mC$
\item Compatible with composition. This means that for any $n,k_1,\dots,k_n\in\N$, the following diagram commutes: \\~\\
\adjustbox{scale=1,center}{\begin{tikzcd}
	{P(n)\otimes P(k_1)\otimes\cdots\otimes P(k_n)} &&& {P'(n)\otimes P'(k_1)\otimes\cdots\otimes P'(k_n)} \\
	{P(k_1+\dots+k_n)} &&& {P'(k_1+\dots+k_n)}
	\arrow["{\varphi_n\otimes\varphi_{k_1}\otimes\cdots\otimes\varphi_{k_n}}", from=1-1, to=1-4]
	\arrow["\gamma"', from=1-1, to=2-1]
	\arrow["{\gamma'}", from=1-4, to=2-4]
	\arrow["{\varphi_{k_1+\cdots+k_n}}", from=2-1, to=2-4]
\end{tikzcd}}\\~\\
\item Compatible with the identity. This means that the following diagram holds: \\~\\
\adjustbox{scale=1,center}{\begin{tikzcd}
	& I \\
	{P(1)} && {P'(1)}
	\arrow["\mu"', from=1-2, to=2-1]
	\arrow["{\mu'}", from=1-2, to=2-3]
	\arrow["{\varphi_1}"', from=2-1, to=2-3]
\end{tikzcd}}\\~\\
\end{itemize}
\end{defn}

\subsection{Operads as a Monoid in a Monoidal Category}

\subsection{The Endomorphism Operad and Algebra Operads}
\begin{defn}{The Endomorphism Operad}{} Let $(\mC,\otimes,I)$ be a symmetric monoidal category. Let $X\in\mC$ be an object. Define the endomorphism operad $\mE\text{nd}(X)$ on $X$ to consist of the following data. 
\begin{itemize}
\item For each $n\in\N$, $\mE\text{nd}(X)(n)=\Hom_\mC(X^{\otimes n},X)$
\item Composition is given by the following: \\~\\
\adjustbox{scale=0.74,center}{\begin{tikzcd}
	{\Hom_\mC(X^{\otimes n},X)\otimes\bigotimes_{i=1}^n\Hom_\mC(X^{\otimes k_i},X)} & {\Hom_\mC(X^{\otimes n},X)\otimes\Hom_\mC(X^{\otimes k_1+\dots+k_n},X^{\otimes n})} && {\Hom_\mC(X^{\otimes k_1+\cdots+k_n},X)}
	\arrow["{\text{id}\otimes\bigotimes}", from=1-1, to=1-2]
	\arrow["{\text{Composition}}", from=1-2, to=1-4]
\end{tikzcd}}\\~\\
\item The unit map $\mu:I\to\Hom_\mC(X,X)$ is the image of identity map $\text{id}_X$ under the adjunction $$\Hom_\mC(X,X)\cong\Hom_\mC(I,\Hom_\mC(X,X))$$
\end{itemize}
\end{defn}

\begin{defn}{The (Non-Unital) Planar Associative Operad}{} Let $(\mC,\otimes,I)$ be a symmetric monoidal category with initial object $\emptyset$. Define the planar associative operad $\mA\text{ss}$ of $\mC$ to consist of the following data. 
\begin{itemize}
\item For each $n\in\N\setminus\{0\}$, $\mA\text{ss}(n)=\{I\}$ and $\mA\text{ss}(0)=\emptyset$
\item The composition function is uniquely determined by the unique map of the terminal object (in other words the identity $\text{id}_I$)
\item The unit $\{\ast\}\to\mA\text{ss}(1)=\{\ast\}$ is also uniquely determined by the unique object
\end{itemize}
\end{defn}

\begin{defn}{The Symmetric Associative Operad}{} Define the planar associative operad $\mA\text{ss}$ to consist of the following data. 
\begin{itemize}
\item For each $n\in\N$, $\mA\text{ss}(n)=S_n$ where $S_n$ is the symmetric group
\item 
\end{itemize}
\end{defn}

\begin{defn}{The Commutative Operad}{}
\end{defn}

\pagebreak
\section{Coloured Operads}
\subsection{Coloured Operads}
\begin{defn}{Coloured Operads}{} Let $(\mC,\otimes,I)$ be a symmetric monoidal category. A coloured operad in $\mC$ consists of the following data. 
\begin{itemize}
\item A set $S$ of objects in $\mC$ called colours
\item For each $n\in\N$, and each $C_1,\dots,C_n,C\in S$, an object $P(C_1,\dots,C_n,C)\in\mC$
\item For each $(n+1)$-tuple $C_1,\dots,C_n,C$ and $n$ other tuples $(D_{1,1},\dots,D_{1,k_1}),\dots,(D_{n,1},\dots,D_{n,k_n})$, there is a morphism $$P(C_1,\dots,C_n,C)\otimes\bigotimes_{i=1}^nP(D_{i,1},\dots,D_{i,k_i},C_i)\to P(D_{1,1},\dots,D_{n,k_n},C)$$ called the composition operation
\item For each $C\in S$, a morphism $$1_C:I\to P(C,C)$$ called the identity
\end{itemize}
such that composition is associativity and unital. 
\end{defn}

\subsection{The Category of Operators}
Every coloured operad defines and is defined by a category called the category of operators. Recall the category of finite sets to consist of objects of the form $[n]=\{0,\dots,n\}$. 

\begin{defn}{The Category of Pointed Finite Sets}{} Define the category of pointed finite sets $$\bold{Fin}_\ast$$ to consist of the following data. 
\begin{itemize}
\item The objects consists of $\langle n\rangle=\{\ast,1,\dots,n\}$ together with the chosen point $\ast$
\item For $\langle n\rangle$ and $\langle m\rangle $ in $\bold{Fin}_\ast$, a morphism is a function $f:\langle n\rangle\to\langle m\rangle$ such that $f(\ast)=\ast$. 
\item Composition is given by the composition of functions. 
\end{itemize}
\end{defn}

\begin{defn}{The Category of Operators}{} Let $A$ be a symmetric coloured operad in $\bold{Set}$. Define the category of operators $\mC_A$ as follows. 
\begin{itemize}
\item The objects are finite sequences $C_1,\dots,C_n$ of colours in $A$
\item For two tuples $(C_1,\dots,C_n)$ and $(D_1,\dots,D_m)$, a morphism $F:(C_1,\dots,C_n)\to(D_1,\dots,D_m)$ is given as a tuple $(\phi,f_1,\dots,f_m)$ where $\phi:\langle n\rangle\to\langle m\rangle$ is a morphism in $\bold{Fin}_\ast$ and each $f_i$ for $1\leq i\leq m$ an operation $$f_i\in\Hom_A((C_k)_{k\in\phi^{-1}(i)},D_i)$$ in $A$
\item Composition is given component wise in $\bold{Fin}_\ast$ and in $A$. 
\end{itemize}
This is often paired with the canonical forgetful functor $p:\mC_A\to\bold{Fin}_\ast$. 
\end{defn}

We can reconstruct the operad $A$ from the category of operators in the following way. 

\pagebreak
\section{Infinity Operads}
\











\end{document}