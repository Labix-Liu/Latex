\documentclass[a4paper]{article}

\input{C:/Users/liula/Desktop/Latex/Headers V1.2.tex}

\pagestyle{fancy}
\fancyhf{}
\rhead{Labix}
\lhead{Operad Theory}
\rfoot{\thepage}

\title{Operad Theory}

\author{Labix}

\date{\today}
\begin{document}
\maketitle
\begin{abstract}
\begin{itemize}
\end{itemize}
\end{abstract}
\pagebreak
\tableofcontents
\pagebreak

\section{The Theory of Operads}
\subsection{General Operads}
\begin{defn}{Operads}{} Let $(\mC,\otimes,I)$ be a symmetric monoidal category. An operad in $\mC$ consists of the following data
\begin{itemize}
\item A sequence $P=\{P(n)\;|\;n\in\N\}$ of objects in $\mC$
\item A composition function $$\gamma:P(n)\otimes P(k_1)\otimes\cdots\otimes P(k_n)\to P(k_1+\cdots+k_n)$$
\item An unit $\mu:I\to P(1)$
\end{itemize}
such that the following compatibility conditions are satisfied. 
\begin{itemize}
\item Associativity in the first symmetric product: \\~\\
\adjustbox{scale=0.8,center}{\begin{tikzcd}
	{P(n)\otimes\left(\bigotimes_{k=1}^n\left(P(i_k)\otimes\bigotimes_{t=1}^{i_k}P(j_{k,t})\right)\right)} & {P(n)\otimes\left(\bigotimes_{k=1}^nP\left(\sum_{t=1}^{i_k}j_{k,t}\right)\right)} \\
	&& {P\left(\sum_{k=1}^n\sum_{t=1}^{i_k}j_{k,t}\right)} \\
	{P(n)\otimes\left(\bigotimes_{k=1}^nP(j_k)\right)\otimes\left(\bigotimes_{k=1}^n\bigotimes_{t=1}^{t_k}P(j_{k,t})\right)} & {P\left(\sum_{u=1}^ki_u\right)\otimes\left(\bigotimes_{k=1}^n\bigotimes_{t=1}^{i_k}P(j_{k,t})\right)}
	\arrow["{\text{id}_{P(n)}\otimes\gamma}"', from=1-1, to=1-2]
	\arrow["\cong", from=1-1, to=3-1]
	\arrow["\gamma"', from=1-2, to=2-3]
	\arrow["{\gamma\otimes\text{id}_{\otimes\otimes}}", from=3-1, to=3-2]
	\arrow["\gamma", from=3-2, to=2-3]
\end{tikzcd}}\\~\\
\item Unitality: \\~\\
\adjustbox{scale=1,center}{\begin{tikzcd}
	{P(n)\otimes P(1)^{\otimes n}} & {P(n)\otimes I^{\otimes n}} & {I\otimes P(n)} & {P(1)\otimes P(n)} \\
	& {P(n)} & {P(n)}
	\arrow["\gamma"', from=1-1, to=2-2]
	\arrow["{\text{id}_{P(n)}\otimes\mu^{\otimes n}}"', from=1-2, to=1-1]
	\arrow["\cong", from=1-2, to=2-2]
	\arrow["{\mu\otimes\text{id}_{P(n)}}", from=1-3, to=1-4]
	\arrow["\cong"', from=1-3, to=2-3]
	\arrow["\gamma", from=1-4, to=2-3]
\end{tikzcd}}\\~\\
\end{itemize}
\end{defn}

TBA: An operad is a monoid in the monoidal category $(\text{Func}(\bold{S},\mC),\circ,I)$

\begin{defn}{Symmetric Operads}{} Let $(\mC,\otimes)$ be a symmetric monoidal category. A symmetric operad is an operad $(P,\gamma,\mu)$ on $\mC$ such that the following are true. 
\begin{itemize}
\item Each $P(n)$ is an $S_n$-module for each $n\in\N$. 
\item $\gamma:P(n)\otimes P(k_1)\otimes\cdots\otimes P(k_n)\to P(k_1+\cdots+k_n)$ is equivariant in the following sense: 
\begin{align*}
\gamma(c\cdot\sigma,d_1,\dots,d_n)&=\gamma(c,d_{\sigma^{-1}(1)},\dots,d_{\sigma^{-1}(n)})\cdot\sigma(k_1,\dots,k_n)\\
\gamma(c,d_1\cdot\tau_1,\dots,d_n\cdot\tau_n)&=\gamma(c,d_1,\dots,d_n)\cdot(\tau_1\oplus\cdots\oplus\tau_n)
\end{align*}
\end{itemize}
\end{defn}

The symmetric monoidal category one usually considers are algebraic. For instance, ${_R}\bold{Mod}$, $\bold{Ch}(R)_{\geq 0}$

\subsection{Operads as a Monoid in a Monoidal Category}










\end{document}