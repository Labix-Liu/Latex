\documentclass[a4paper]{article}

\input{C:/Users/liula/Desktop/Latex/Headers V1.2.tex}

\pagestyle{fancy}
\fancyhf{}
\rhead{Labix}
\lhead{Commutative Algebra 1}
\rfoot{\thepage}

\title{Commutative Algebra 1}

\author{Labix}

\date{\today}
\begin{document}
\maketitle
\begin{abstract}
\end{abstract}
\pagebreak
\tableofcontents
\pagebreak

\pagebreak
\section{Basic Notions of Commutative Rings}
\subsection{Local Rings}
\begin{defn}{Local Rings}{} Let $R$ be a commutative ring. We say that $R$ is a local ring if it has a unique maximal ideal $m$. In this case, we say that $R/m$ is the residue field of $R$. 
\end{defn}

\begin{eg}{}{} Consider the following commutative rings. 
\begin{itemize}
\item $\Z/6\Z$ is not a local ring. 
\item $\Z/8\Z$ is a local ring. 
\item $\Z/24\Z$ is not a local ring. 
\item $\R[x]$ is not a local ring. 
\end{itemize} \tcbline
\begin{proof}~\\
\begin{itemize}
\item The only ideals of $\Z/6\Z$ are $(2+6\Z)$ and $(3+6\Z)$. They do not contain each other and so they are both maximal. 
\item The only ideals of $\Z/8\Z$ are $(2+8\Z)$ and $(4+8\Z)$. But $(2+8\Z)\supseteq(4+8\Z)$. Hence $\Z/8\Z$ has a unique maximal ideal. 
\item A similar proof as above ensues. 
\item Any irreducible polynomial $f\in\R[x]$ is such that $(f)$ is a maximal ideal. Indeed the evaluation homomorphism gives an isomorphism $\frac{\R[x]}{(f)}\cong\R$. 
\end{itemize}
\end{proof}
\end{eg}

\begin{prp}{}{} Let $R$ be a ring and $I$ an ideal of $R$. Then $I$ is the unique maximal ideal of $R$ if and only if $I$ is the set containing all non-units of $R$. \tcbline
\begin{proof}
Let $I$ be the unique maximal ideal of $R$. Clearly $I$ does not contain any unit else $I=R$. Now suppose that $r$ is a non-unit. Suppose that $r\notin I$. Define $J=\{sr|s\in R\}$ Clearly $J$ is an ideal. It must be contained in some maximal ideal. Since $I$ is the unique maximal ideal, $J\subseteq I$. But this means that $r\in I$, a contradiction. Thus every non-unit is in $I$. \\~\\
Suppose that $I$ contains all non-units of $R$. Let $r\notin I$. Then there exists $s\notin I$ such that $rs=1$. Then $(r+I)(s+I)=1+I$ in $R/I$. This means that every element of $R/I$ has a multiplicative inverse which means that $R/I$ is a field and thus $I$ is a maximal ideal. Now let $J\neq I$ be another maximal ideal. Then $J$ contains some unit $r$. This implies that $J=R$ and thus $I$ is the unique maximal ideal. 
\end{proof}
\end{prp}

\begin{eg}{}{} Let $k$ be a field. Then the ring of power series $k[[x]]$ is a local ring. \tcbline
\begin{proof}
Let $M$ be the set of all non-units of $k[[x]]$. I first show that $f\in M$ if and only if the constant term of $f$ is non-zero. Let $g$ be a power series. Then the $n$th coefficient of $f\cdot g$ is given by $$c_n=\sum_{k=0}^na_kb_{n-k}$$ If the constant term of $f$ is $0$, then $c_0=0$ and so $f\cdot g\neq 1$. Now if the constant term of $f$ is $a_0\neq 0$, then set $b_0=\frac{1}{a_0}$. Now we can use the formula $0=c_n$ to deduce $$b_n=-\frac{\sum_{k=1}^na_kb_{n-k}}{a_0}$$. This is such that $a_n\cdot b_n=0$. Define $g=\sum_{k=0}^\infty b_kx^k$. Then $f\cdot g=1$. Thus $f$ is a unit. \\~\\

By the above proposition, we conclude that $M$ is the unique maximal ideal of $k[[x]]$. 
\end{proof}
\end{eg}

We will discuss more of local rings in the topic of localizations. 

\subsection{Hilbert's Basis Theorem}
\begin{thm}{Hilbert's Basis Theorem}{} Let $R$ be a commutative ring. If $R$ is Noetherian, then $$R[x_1,\dots,x_n]$$ is a Noetherian ring. 
\end{thm}

\begin{prp}{}{} Let $R$ be a commutative ring. Let $I$ be an ideal of $R$. If $R$ is Noetherian then $R/I$ is Noetherian. 
\end{prp}

\begin{thm}{}{} Let $R=\bigoplus_{i=1}^nR_i$ be a graded ring. Then $R$ is Noetherian if and only if $R_0$ is Noetherian and $R$ is finitely generated as an $R_0$-module. 
\end{thm}

\subsection{Spectra of a Ring}
\begin{defn}{Max Spectrum of a Ring}{} Let $A$ be a commutative ring. Define the max spectrum of $A$ to be $$\text{maxSpec}(A)=\{m\subseteq A\;|\;m\text{ is a maximal ideal of }A\}$$
\end{defn}

\begin{defn}{Spectrum of a Ring}{} Let $A$ be a commutative ring. Define the spectrum of $A$ to be $$\text{Spec}(A)=\{p\subseteq A\;|\;p\text{ is a prime ideal of }A\}$$
\end{defn}

\begin{eg}{}{} Consider the following commutative rings. 
\begin{itemize}
\item $\text{Spec}(\Z/6\Z)=\{(2+6\Z),(3+6\Z)\}$
\item $\text{Spec}(\Z/8\Z)=\{(2+8\Z)\}$
\item $\text{Spec}(\Z/24\Z)=\{(2+24\Z),(3+24\Z)\}$
\item $\text{Spec}(\R[x])=\{(f)\;|\;f\text{ is irreducible }\}$
\end{itemize} \tcbline
\begin{proof}~\\
\begin{itemize}
\item The only ideals of $\Z/6\Z$ are $(2+6\Z)$ and $(3+6\Z)$. We need to find which ones are prime ideals. Now $\Z/6\Z\setminus(2+6\Z)$ consists of $1+6\Z$, $3+6\Z$ and $5+6\Z$. No multiplication of these elements give an element of $(2+6\Z)$. So any two elements in $\Z/6\Z$ which multiply to an element of $(2+6\Z)$ must contain one element that lie in $(2+6\Z)$. Hence $(2+6\Z)$ is prime. This is similar for $(3+6\Z)$. Hence $\text{Spec}(\Z/6\Z)=\{(2+6\Z),(3+6\Z)\}$. 
\item The only ideals of $\Z/8\Z$ are $(2+8\Z)$ and $(4+8\Z)$. A similar argument as above shows that $(2+8\Z)$ is a prime ideal. However, $6+8\Z\notin(4+8\Z)$ while $(6+8\Z)^2=4+8\Z\in(4+8\Z)$ which shows that $(4+8\Z)$ is not a prime ideal. 
\item A similar proof as above ensues. 
\item Recall that $\R[x]$ is a principal ideal domain. Let $I=(f)$ be a prime ideal of $\R[x]$. Then $f$ is irreducible. Thus every prime ideal of $\R[x]$ is of the form $(f)$ for $f$ an irreducible polynomial. 
\end{itemize}
\end{proof}
\end{eg}

\begin{lmm}{}{} Let $R,S$ be commutative rings. Let $f_1:R\times S\to R$ and $f_2:R\times S\to S$ denote the projection maps. Then the map $$f_1^\ast\amalg f_2^\ast:\text{Spec}(R)\amalg\text{Spec}(S)\to\text{Spec}(R\times S)$$ is a bijection. \tcbline
\begin{proof}
The core of the proof is the fact that $P$ is a prime ideal of $R\times S$ if and only if $P=R\times Q$ or $P=V\times S$ for either a prime ideal $Q$ of $P$ or a prime ideal $V$ of $S$. It is clear that if $Q$ is a prime ideal of $S$ and $V$ is a prime ideal of $R$, then $R\times Q$ and $V\times S$ are both prime ideals of $R\times S$. \\~\\

So suppose that $P$ is a prime ideal in $R\times S$. Let $e_1=(1,0)$ and $e_2=(0,1)$. Since $P\neq R$, at least one of $e_1$ or $e_2$ is not in $P$. Without loss of generality assume that $e_1\notin P$. But $e_1e_2=0\in P$ and $P$ being prime implies that $e_2\in P$. Since $e_2$ is the identity of $\{0\}\times S\cong S$, we conclude that $\{0\}\times S\subseteq P$. By the correspondence theorem, the projection map $f_1:R\times S\to R$ gives a bijection between prime ideals of $R\times S$ that contain $\{0\}\times S$ and prime ideals of $R$. So $f_1(P)$ is a prime ideal of $R$. Thus $P=f_1(P)\times S$ which is exactly what we wanted. \\~\\

Now the bijection is clear. $f_1^\ast\amalg f_2^\ast$ sends a prime ideal $P$ of $R$ to $P\times S$ and it sends a prime ideal $Q$ of $S$ to $R\times Q$. This map is surjective by the above argument. It is injective by inspection. 
\end{proof}
\end{lmm}

\pagebreak
\section{Ideals Of a Commutative Ring}
\subsection{Operations on Ideals}
\begin{prp}{}{} Let $R$ be a commutative ring. Let $S,T\subseteq R$ be subsets of $R$. Then $$\langle S\cup T\rangle=\langle S\rangle +\langle T\rangle$$
\end{prp}

\begin{prp}{}{} Let $R$ be a commutative ring. Let $I,J$ be ideals of $R$. Suppose that $I\subseteq J$. Let $\overline{J}$ denote the ideal of $R/I$ corresponding to $J$ under the correspondence theorem. Then there is an isomorphism $$\frac{R/I}{\overline{J}}\cong\frac{R}{I+J}$$ given by the formula $(r+I)+\overline{J}\mapsto r+(I+J)$. 
\end{prp}

\begin{eg}{}{} There is an isomorphism given by $$\frac{\Z[x]}{(x+1,x^2+2)}\cong\Z/3\Z$$ \tcbline
\begin{proof}
Using the above propositions, we have that 
\begin{align*}
\frac{\Z[x]}{(x+1,x^2+2)}&=\frac{\Z[x]}{(x+1)+(x^2+2)}\\
&\cong\frac{\Z[x]/(x+1)}{(3)}
\end{align*}
Indeed, the ideal $(x^2+2)$ corresponds to the ideal $(3)$ in $\frac{\Z[x]}{(x+1)}$ because the remainder of $x^2+2$ divided by $(x+1)$ is $(3)$. Now $\Z[x]/(x+1)\cong\Z$ by the evaluation homomorphism. Thus quotieting by the ideal $(3)$ gives the field $\Z/3\Z$. 
\end{proof}
\end{eg}

Some more important results from Groups and Rings and Rings and Modules include: 
\begin{itemize}
\item If $I$ and $J$ are coprime, then $IJ=I\cap J$
\item Chinese Remainder Theorem: If $I$ and $J$ are coprime, then there is an isomorphism $$\frac{R}{I\cap J}\cong\frac{R}{I}\times\frac{R}{J}$$
\end{itemize}

\subsection{Radical Ideals}
The radical of an ideal is a very different notion from the radical of module. 

\begin{defn}{Radical of an Ideal}{} Let $I$ be an ideal of a ring $R$. Define the radical of $I$ to be $$\sqrt{I}=\{r\in R|r^n\in I\text{ for some }n\in\N\}$$
\end{defn}

\begin{prp}{}{} Let $R$ be a commutative ring. Let $I$ be an ideal. Then the following are true. 
\begin{itemize}
\item $I\subseteq\sqrt{I}$
\item $\sqrt{\sqrt{I}}=\sqrt{I}$
\item $\sqrt{I^m}=\sqrt{I}$ for all $m\geq 1$
\item $\sqrt{I}=R$ if and only if $I=R$
\end{itemize} \tcbline
\begin{proof}~\\
\begin{itemize}
\item Let $r\in I$. Then $r^1\in I$ Thus by choosing $n=1$ we shows that $r^n\in I$. Thus $r\in\sqrt{I}$. 
\item By the above, we already know that $\sqrt{I}\subseteq\sqrt{\sqrt{I}}$. So let $r\in\sqrt{\sqrt{I}}$. Then there exists some $n\in\N$ such that $r^n\in\sqrt{I}$. But $r^n\in\sqrt{I}$ means that there exists some $m\in\N$ such that $(r^n)^m\in I$. But $nm\in\N$ is a natural number such that $r^{nm}\in I$. Hence $r\in\sqrt{I}$ and so we conclude. 
\end{itemize}
\end{proof}
\end{prp}

\begin{prp}{}{} Let $R$ be a commutative ring. Let $I,J$ be ideals of $R$. Then the following are true. 
\begin{itemize}
\item If $I\subseteq J$ then $\sqrt{I}\subseteq\sqrt{J}$
\item $\sqrt{IJ}=\sqrt{I\cap J}$
\item $\sqrt{I+J}=\sqrt{\sqrt{I}+\sqrt{J}}$
\end{itemize} \tcbline
\begin{proof}~\\
\begin{itemize}
\item Let $x\in\sqrt{IJ}$. Then $x^n\in IJ$. This means that there exists $i\in I$ and $j\in J$ such that $x^n=ij$. Since $I$ and $J$ are two sided ideals, we can conclude that $x^n=ij\in I,J$. Hence $x^n=ij\in I\cap J$. We conclude that $x\in\sqrt{I\cap J}$. Now let $x\in\sqrt{I\cap J}$. Then there exists $n\in\N$ such that $x^n\in I\cap J$. Then $x^n\in I$ and $x^n\in J$ implies that $x^{2n}=x^n\cdot x^n\in IJ$. We conclude that $x\in\sqrt{IJ}$. 
\end{itemize}
\end{proof}
\end{prp}

\begin{prp}{}{} Let $R$ be a commutative ring. Let $I$ be an ideal. Then $$\sqrt{I}=\bigcap_{\substack{p\text{ a prime ideal}\\I\subseteq p\subseteq R}}p$$
\end{prp}

\begin{defn}{Radical Ideals}{} Let $R$ be a commutative ring. Let $I$ be an ideal of $R$. We say that $I$ is radical if $$\sqrt{I}=I$$
\end{defn}

In particular, by the above lemma it follows that the radical of an ideal is a radical ideal. 

\begin{lmm}{}{} Let $R$ be a ring. Let $P$ be a prime ideal of $R$. Then $P$ is radical. 
\end{lmm}

We conclude that there is an inclusion of types of ideal in which each inclusion is strict: $$\substack{\text{Maximal}\\\text{ideals}}\subset\substack{\text{Prime}\\\text{ideals}}\subset\substack{\text{Radical}\\\text{ideals}}$$

\begin{thm}{}{} Let $R$ be a commutative ring. Let $I$ be an ideal of $R$. Denote $\varphi$ to be the inclusion preserving one-to-one bijection $$\left\{\substack{\text{Ideals of }R\\\text{containing }I}\right\}\;\;\overset{1:1}{\longleftrightarrow}\;\;\left\{\text{Ideals of }R/I\right\}$$ from the correspondence theorem for rings. In other words, $\varphi(A)=A/I$. Let $J\subseteq R$ be an ideal containing $I$. Then the following are true. 
\begin{itemize}
\item $J$ is a radical ideal if and only if $\varphi(J)=J/I$ is a radical ideal. 
\item $J$ is a prime ideal if and only if $\varphi(J)=J/I$ is a prime ideal. 
\item $J$ is a maximal ideal if and only if $\varphi(J)=J/I$ is a maximal ideal. 
\end{itemize} \tcbline
\begin{proof}~\\
\begin{itemize}
\item Let $J$ be a radical ideal. Suppose that $r+I\in\sqrt{J/I}$. This means that $(r+I)^n=r^n+I\in J/I$ for some $n\in\N$. But this means that $r^n\in J$. This implies that $r\in\sqrt{J}=J$. Thus $r+I\in J/I$ and we conclude that $\sqrt{J/I}\subseteq J/I$. Since we also have $J/I\subseteq\sqrt{J/I}$, we conclude. \\~\\

Now suppose that $J/I$ is a radical ideal. Let $r\in\sqrt{J}$. This means that $r^n\in J$ for some $n\in\N$. Now $r^n+I=(r+I)^n\in J/I$ implies that $r+I\in\sqrt{J/I}=J/I$. Hence $r\in J$ and so $\sqrt{J}\subseteq J$. Since we also have that $J\subseteq\sqrt{J}$, we conclude. 

\item Let $J$ be a prime ideal. Then $R/J$ is an integral domain. By the second isomorphism theorem, we have that $R/J\cong(R/I)/(J/I)$ and hence $(R/I)/(J/I)$ is also an integral domain. Hence $J/I$ is a prime ideal. The converse is also true. 

\item Let $J$ be a maximal ideal. Then $R/J$ is a field. By the second isomorphism theorem, we have that $R/J\cong(R/I)/(J/I)$ and hence $(R/I)/(J/I)$ is also a field. Hence $J/I$ is a maximal ideal. The converse is also true. 
\end{itemize}
\end{proof}
\end{thm}

Another way to write the bijections is via spectra: $$\text{Spec}(R/I)\;\;\overset{1:1}{\longleftrightarrow}\;\;\{P\in\text{Spec}(R)\;|\;I\subseteq P\}$$ and $$\text{maxSpec}(R/I)\;\;\overset{1:1}{\longleftrightarrow}\;\;\{m\in\text{maxSpec}(R)\;|\;I\subseteq m\}$$

\subsection{Nilradical and Jacobson Ideals}
Let $R$ be a ring. Recall that an element $r\in R$ is nilpotent if $r^n=0_R$ for some $n\in\N$. When $R$ is commutative, we can form an ideal out of nilpotent elements. 

\begin{defn}{Nilradicals}{} Let $R$ be a ring. Define the nilradical of $R$ to be $$N(R)=\{r\in R\;|\;r\text{ is nilpotent}\}$$
\end{defn}

Note that this is different from nilpotent ideals, as nilpotency is a property of an ideal. However the Nilradical ideal is a nil ideal and every sub-ideal of the nilradical is a nil ideal. 

\begin{prp}{}{} Let $R$ be a ring and $N(R)$ its nilradical. Then the following are true. 
\begin{itemize}
\item $N(R)$ is an ideal of $R$
\item $N(R/N(R))=0$
\end{itemize}\tcbline
\begin{proof}~\\
\begin{itemize}
\item Suppose that $r,s$ are nilpotent, meaning that $r^n=0$ and $s^m=0$. Then $(r+s)^{n+m}=0$. Moreover, if $t\in R$ then $t\cdot r$ is also nilpotent
\item Let $r\notin N(R)$. Every element $r+N(R)\in R/N(R)$ has the property that $r^n\neq 0$. Consider $(r+N(R))^n=r^n+N(R)$. If $r^n\in N(R)$ then $r^n=u$ for some nilpotent $u$, which means that $r^n$ is nilpotent and thus $r$ is nilpotent, a contradiction. This means that $r+N(R)\notin N(R/N(R))$ for all $r\notin N(R)$ and thus $N(R/N(R))=0$
\end{itemize}
\end{proof}
\end{prp}

\begin{prp}{}{} Let $R$ be a commutative ring. The nilradical of $R$ is the intersection of all prime ideals of $R$. \tcbline
\begin{proof}
We want to show that $$N(R)=\bigcap_{P\in\text{Spec}(R)}P$$
Trivially $N(R)$ is a prime ideal. Now suppose that $r\in R$ is in the intersection of all prime ideals. Then $r^n$ also lies in every prime ideal. 
\end{proof}
\end{prp}

\begin{eg}{}{} Consider the ring $$R=\frac{\C[x,y]}{(x^2-y,xy)}$$ Then its nilradical is given by $N(R)=(x,y)$. \tcbline
\begin{proof}
Notice that in the ring $R$, $x^3=x(x^2)=xy=0$ and $y^3=x^6=(x^3)^2=0$ and hence $x$ and $y$ are both nilpotent elements of $R$. By definition of the nilradical, we conclude that $(x,y)\subseteq N(R)$. Now $(x,y)$ is a maximal ideal of $\C[x,y]$ because $\C[x,y]/(x,y)\cong\C$. Also notice that $(x,y)\supseteq(x^2-y,xy)$ because for any element $f(x)(x^2-y)+g(x)(xy)\in(x^2-y,xy)$, we have that 
\begin{align*}
f(x)(x^2-y)+g(x)(xy)\in(x^2-y,xy)&=(xf(x))x-f(x)y+(g(x)x)y\\
&=(xf(x))x+(xg(x)-f(x))y\in (x,y)
\end{align*}
By the correspondence theorem, $(x,y)/(x^2-y)$ is an maximal ideal of $R$. In particular, $(x,y)$ is also a prime ideal. But the $N(R)$ is the intersection of all prime ideals and hence $N(R)\subseteq(x,y)$. We conclude that $N(R)=(x,y)$. 
\end{proof}
\end{eg}

\begin{defn}{Reduced Rings}{} Let $R$ be a commutative ring. We say that $R$ is reduced if $N(R)=0$. 
\end{defn}

\begin{prp}{}{} Let $R$ be a commutative ring. Let $I$ be an ideal of $R$. Then $R/I$ is reduced if and only if $I$ is a radical ideal. 
\end{prp}

So radical, prime and maximal ideals all have characterizations using the quotient ring: 
\begin{itemize}
\item $I$ is maximal if and only if $R/I$ is a field. 
\item $I$ is prime if and only if $R/I$ is an integral domain. 
\item $I$ is radical if and only if $R/I$ is reduced. 
\end{itemize}

Recall the notion of the Jacobson radical from Rings and Modules. Let $R$ be a ring. The Jacobson radical of $R$ is the radical $$J(R)=\text{rad}(R)=\bigcap_{\substack{S\trianglelefteq R\\R\text{ is cosimple}}}S$$ of $R$ considered as a left $R$-module. But when $R$ is a commutative ring, this description can be simplified. 

\begin{prp}{}{} Let $R$ be a commutative ring. Then $$J(R)=\bigcap_{m\in\text{maxSpec}(R)}m$$ \tcbline
\begin{proof}
Submodules of $R$ are precisely ideals of $R$ and cosimple ideals are ideals $I$ of $R$ for which $R/I$ is simple. But if $R/I$ is simple, then $R/I$ contains no ideals which means that $R/I$ is a field. So $I$ is a maximal ideal. 
\end{proof}
\end{prp}

Recall some properties of the Jacobson radical from Rings and Modules. For a (not necessarily commutative ring $R$), 
\begin{itemize}
\item $J(R/J(R))=0$
\end{itemize}

\begin{prp}{}{} Let $R$ be a commutative ring. Then $x\in J(R)$ if and only if $1-xy\in R^\times$ for all $y\in R$. \tcbline
\begin{proof}
\end{proof}
\end{prp}

\subsection{Extensions and Contractions of Ideals}
\begin{defn}{Extension of Ideals}{} Let $R,S$ be commutative rings. Let $f:R\to S$ be a ring homomorphism. Let $I$ be an ideal of $R$. Define the extension $I^e$ of $I$ to $S$ to be the ideal $$I^e=\langle f(i)\;|\;i\in I\rangle$$
\end{defn}

\begin{prp}{}{} Let $R,S$ be commutative rings. Let $f:R\to S$ be a ring homomorphism. Let $I,I_1,I_2$ be an ideal of $R$. Then the following are true regarding the extension of ideals. 
\begin{itemize}
\item Closed under sum: $(I_1+I_2)^e=I_1^e+I_2^e$
\item $(I_1\cap I_2)^e\subseteq I_1^e\cap I_2^e$
\item Closed under products: $(I_1I_2)^e=I_1^eI_2^e$
\item $(I_1/I_2)^e\subseteq I_1^e/I_2^e$
\item $\text{rad}(I)^e\subseteq\text{rad}(I^e)$
\end{itemize}
\end{prp}

\begin{defn}{Contraction of Ideals}{} Let $R,S$ be commutative rings. Let $f:R\to S$ be a ring homomorphism. Let $J$ be an ideal of $S$. Define the contraction $J^c$ of $J$ to $R$ to be the ideal $$J^c=f^{-1}(J)$$
\end{defn}

\begin{prp}{}{} Let $R,S$ be commutative rings. Let $f:R\to S$ be a ring homomorphism. Let $J,J_1,J_2$ be an ideal of $S$. Then the following are true regarding the extension of ideals. 
\begin{itemize}
\item $(J_1+J_2)^e\supseteq J_1^e+J_2^e$
\item Closed under intersections: $(J_1\cap J_2)^e=J_1^e\cap J_2^e$
\item $(J_1J_2)^e\supseteq J_1^eJ_2^e$
\item $(J_1/J_2)^e\subseteq J_1^e/J_2^e$
\item Closed under taking radicals: $\text{rad}(J)^e=\text{rad}(J^e)$
\end{itemize}
\end{prp}

\begin{prp}{}{} Let $R,S$ be commutative rings. Let $f:R\to S$ be a ring homomorphism. Let $I$ be an ideal of $R$ and let $J$ be an ideal of $S$. Then the following are true. 
\begin{itemize}
\item $I\subseteq I^{ec}$
\item $J^{ce}\subseteq J$
\item $I^e=I^{ece}$
\item $J^c=J^{cec}$
\end{itemize}
\end{prp}

\subsection{Revisiting the Polynomial Ring}
\begin{prp}{}{} Let $R$ be a commutative ring. Then we have $$N(R[x])=N(R)[x]$$ \tcbline
\begin{proof}
Let $f=\sum_{k=0}^na_kx^k\in N(R)[x]$. Then each $a_k$ is nilpotent in $R$, and there exists $n_k\in\N$ such that $a_k^{n_k}=0$. This also proves that $a_kx^k$ is nilpotent. Since the sum of nilpotents is a nilpotent, we conclude that $f$ is nilpotent. \\~\\

Now suppose that $f\in N(R[x])$. We induct on the degree of $f$. Let $\deg(f)=0$. Then $f$ is nilpotent and $f$ lies in $R$. Thus $f\in N(R)[x]$. Now suppose that the claim is true for $\deg(f)\leq n-1$. Let $\deg(g)=n$ with leading coefficient $b_n$. Since $g$ is nilpotent in $R[x]$, there exists $m\in\N$ such that $g^m=0$. Then in particular, $b_n^m=0$ so that $b_n$ is nilpotent. Then $b_nx^n$ is also nilpotent. Now since $N(R[x])$ is an ideal of $R[x]$, we have that $g-b_nx^n\in N(R[x])$. By inductive hypothesis, $g-b_nx^n\in N(R)[x]$. Since $N(R)$ is an ideal of $R$, we have that $N(R)[x]$ is an ideal of $R[x]$. So $g=(g-b_nx^n)+b_nx^n\in N(R)[x]$. Thus we are done. 
\end{proof}
\end{prp}

Some more important results from Groups and Rings and Rings and Modules include: 
\begin{itemize}
\item If $R$ is an integral domain, then $R[x]$ is an integral domain. 
\item $R$ is a UFD if and only if $R[x]$ is a UFD
\item If $F$ is a field, then $F[x]$ is an Euclidean domain, a PID and a UFD
\item If $F$ is a field, then the ideal generated by $p$ is maximal if and only if $p$ is irreducible. 
\end{itemize}

Regarding ideals of the polynomial ring, the following maybe useful: 
\begin{itemize}
\item $I[x]$ is an ideal of $R$
\item There is an isomorphism $\frac{R[x]}{I[x]}\cong\frac{R}{I}[x]$ given by the map $$\left(f=\sum_{k=0}^na_kx^k+I[x]\right)\mapsto\left(\sum_{k=0}^n(a_k+I)x^k\right)$$
\item If $I$ is a prime ideal of $R$, then $I[x]$ is a prime ideal of $R[x]$. 
\end{itemize}

\pagebreak
\section{Simplifying Generators of an Ideal}
\subsection{Ordering on the Monomials}
Recall that a monomial in $R[x_1,\dots,x_n]$ is an element in the polynomial ring of the form $x_1^{a_1}\cdots x_n^{a_n}$. For simplicity we write this as $x^{(a_1,\dots,a_n)}$. 

\begin{defn}{Monomial Ordering}{} A monomial ordering on a polynomial ring $k[x_1,\dots,x_n]$ is a relation $>$ on $\N^n$. This means that the following are true. 
\begin{itemize}
\item $>$ is a total ordering on $\N^n$
\item If $a>b$ and $c\in\N^n$ then $a+c>b+c$
\item $>$ is a well ordering on $\N^n$ (any nonempty subset of $\N^n$ has a smallest element)
\end{itemize}
\end{defn}

\begin{defn}{Lexicographical Order}{} Let $a=(a_1,\dots,a_n)$ and $b=(b_1,\dots,b_n)$ in $\N^n$. We say that $a>_{\text{lex}}b$ if in the first nonzero entry of $a-b$ is positive. 
\end{defn}

In practise this means that the we value more powers of $x_1$

\begin{defn}{Graded Lex Order}{} Let $a=(a_1,\dots,a_n)$ and $b=(b_1,\dots,b_n)$ in $\N^n$. We say that $a>_{\text{grlex}}b$ if either of the following holds. 
\begin{itemize}
\item $\abs{a}=\sum_{k=1}^na_k>\sum_{k=1}^nb_k=\abs{b}$
\item $\abs{a}=\abs{b}$ and $a>_{\text{lex}}b$
\end{itemize}
\end{defn}

\begin{defn}{Graded Lex Order}{} Let $a=(a_1,\dots,a_n)$ and $b=(b_1,\dots,b_n)$ in $\N^n$. We say that $a>_{\text{grlex}}b$ if either of the following holds. 
\begin{itemize}
\item $\abs{a}=\sum_{k=1}^na_k>\sum_{k=1}^nb_k=\abs{b}$
\item $\abs{a}=\abs{b}$ and the last nonzero entry of $a-b$ is negative. 
\end{itemize}
\end{defn}

In practise we value lower powers of the last variable $x_n$. 

\begin{prp}{}{} The above three orders are all monomial orderings of $k[x_1,\dots,x_n]$. 
\end{prp}

\begin{defn}{Multidegree}{} Let $f\in k[x_1,\dots,x_n]$ be a polynomial in the form $f=\sum_{v\in\N^n}c_vx^v$. Define the multidegree of $f$ to be $$\text{multideg}(f)=\max_>\{v\in\N^n|a_v\neq 0\}$$ where $>$ is a monomial ordering on $k[x_1,\dots,x_n]$. 
\end{defn}

\begin{defn}{Leading Objects}{} Let $f\in k[x_1,\dots,x_n]$ be a polynomial in the form $f=\sum_{v\in\N^n}c_vx^v$. 
\begin{itemize}
\item Define the leading coefficient of $f$ to be $\text{LC}(f)=c_\text{multideg(f)}\in k$
\item Define the leading monomial of $f$ to be $\text{LM}(f)=c_\text{multideg(f)}\in k$
\item Define the leading term of $f$ to be $\text{LT}=\text{LC}(f)\cdot\text{LM}(f)$
\end{itemize}
\end{defn}

\begin{prp}{Division Algorithm in $k[x_1,\dots,x_n]$}{}
\end{prp}

\subsection{Monomial Ideals}
\begin{defn}{Monomial Ideals}{} An ideal $I\subset k[x_1,\dots,x_n]$ is said to be a monomial ideal if $I$ is generated by a set of monomials $\{x^v|v\in A\}$ for some $A\subset\N^n$. In this case we write $$I=\langle x^v|v\in A\rangle$$
\end{defn}

\begin{lmm}{}{} Let $I=\langle x^v|v\in A\rangle$ be an ideal of $k[x_1,\dots,x_n]$. Then a monomial $x^w$ lies in $I$ if and only if $x^v|x^w$ for some $v\in A$. Moreover, if $f=\sum_{w\in\N^n}c_wx^w\in k[x_1,\dots,x_n]$ lies in $I$, then each $x^w$ is divisible by $x^v$ for some $v\in A$. 
\end{lmm}

\begin{thm}{Dickson's Lemma}{} Every monomial ideal is finitely generated. In particular, every monomial ideal $I=\langle x^v|v\in A\rangle$ is of the form $$I=\langle x^{v_1},\dots,x^{v_n}\rangle$$ where $v_1,\dots,v_n\in A$. 
\end{thm}

\subsection{Groebner Bases}

\pagebreak
\section{Modules over a Commutative Ring}
Recall from Rings and Modules that a module consists of an abelian group $M$ and a ring $R$ such that there is a binary operation $\cdot:R\times M\to M$ that mimic the notion of a group action: 
\begin{itemize}
\item For $r,s\in R$, $s\cdot(r\cdot m)=(sr)\cdot m$ for all $m\in M$. 
\item For $1_R\in R$ the multiplicative identity, $1_R\cdot m=m$ for all $m\in M$. 
\end{itemize}

When $R$ is a commutative ring, the first axiom is relaxed so that the resulting element of $M$ makes no difference whether you apply $r$ first or $s$ first. This makes module act even more similarly than fields (although one still need the notion of a basis, which appears in free modules). Therefore the first section concerns transferring techniques in linear algebra such as the Cayley Hamilton theorem to module over a ring that mimic the notion of vector spaces. 

\subsection{Cayley-Hamilton Theorem}
\begin{defn}{Characteristic Polynomial}{} Let $R$ be a commutative ring. Let $A\in M_{n\times n}(R)$ be a matrix. Define the characteristic polynomial of $A$ to be the polynomial $$c_A(x)=\det(A-xI)$$
\end{defn}

\begin{thm}{Cayley-Hamilton Theorem}{} Let $R$ be a commutative ring. Let $A\in M_{n\times n}(R)$ be a matrix. Then $c_A(A)=0$. 
\end{thm}

\begin{crl}{}{} Let $R$ be a commutative ring. Let $M$ be a finitely generated $R$-module. Let $I$ be an ideal of $R$. Let $\varphi\in\text{End}_R(M)$. If $\varphi(M)\subseteq IM$, then there exists $a_1,\dots,a_n\in I$ such that $$\varphi^n+a_1\varphi^{n-1}+\dots+a_{n-1}\varphi+\text{id}_M=0:M\to M$$ \tcbline
\begin{proof}
Suppose that $M$ is generated by $x_1,\dots,x_n$. There exists a surjective map $\rho:R^n\to M$ given by $(r_1,\dots,r_n)\mapsto\sum_{k=1}^nr_kx_n$. Since $\varphi(M)\subseteq IM$, we havt that $$\varphi(x_k)=\sum_{i=1}^nr_{ki}x_i$$ for some $r_{ki}\in I$. Write $A$ to be the matrix $A=(a_{ki})$. We now have a commutative diagram: \\~\\
In other words, we have the diagram: \\
\adjustbox{scale=1.0,center}{\begin{tikzcd}
	{R^n} & M \\
	{R^n} & M
	\arrow["\rho", two heads, from=1-1, to=1-2]
	\arrow["A"', from=1-1, to=2-1]
	\arrow["\varphi", from=1-2, to=2-2]
	\arrow["\rho"', two heads, from=2-1, to=2-2]
\end{tikzcd}}\\~\\
By Cayley-Hamilton theorem, we have that $c_A(A)=0$ is the zero function. For all $x\in R^n$, we have that 
\begin{align*}
c_A(A)(x)&=0\\
c_A(Ax)&=0\\
\rho(c_A(Ax))&=\rho(0)\\
c_A(\rho(Ax))&=0\tag{$\rho$ is $R$-linear}\\
c_A(\varphi(\rho(x)))&=0\tag{Diagram is commutative}
\end{align*}
Since $\rho$ is surjective, we conclude that for any $m\in M$, the above calculation gives $c_A(\varphi(m))=0$ so that $c_A(\varphi)$ is the zero map. 
\end{proof}
\end{crl}

\subsection{Nakayama's Lemma}
\begin{lmm}{Nakayama's Lemma I}{} Let $R$ be a commutative ring. Let $M$ be a finitely generated $R$-module. Let $I$ be an ideal of $R$. If $IM=M$, then there exists $r\in R$ such that $rM=0$ and $r-1\in I$. \tcbline
\begin{proof}
Choose $\varphi=\text{id}_M$. Then $\varphi$ is surjective so that $M=\varphi(M)\subseteq IM$. By crl 4.1.3, there exists $r_1,\dots,r_n\in I$ such that $(1+r_1+\dots+r_n)M=0$. By choosing $r=1+r_1+\dots+r_n$, we see that $rM=0$ and $r-1\in I$ so that we conclude. 
\end{proof}
\end{lmm}

\begin{lmm}{Nakayama's Lemma II}{} Let $R$ be a commutative ring. Let $M$ be a finitely generated $R$-module. Let $I$ be an ideal of $R$ such that $I\subseteq J(R)$ and $IM=M$. Then $M=0$. \tcbline
\begin{proof}
By Nakayama's lemma I, there exists $r\in R$ such that $rM=0$ and $r-1\in I\subseteq J(R)$. By 2.3.8, we have that $1-(r-1)(-1)=r\in R^\times$. This means that $r$ is invertible. Hence $rM=0$ implies $M=r^{-1}rM=0$. 
\end{proof}
\end{lmm}

\begin{crl}{}{} Let $R$ be a commutative ring. Let $M$ be a finitely generated $R$-module. Let $I$ be an ideal of $R$ such that $I\subseteq J(R)$. Let $N$ be an $R$-submodule of $M$. If $$M=IM+N$$ then $M=N$. \tcbline
\begin{proof}
Since quotients of finitely generated modules are finitely generated, we know that $M/N$ is finitely generated. Define the map $$\phi:IM+N\to I\frac{M}{N}$$ by $\phi(im+n)=i(m+N)$. This map is clearly surjective. Now I claim that $\ker(\phi)=N$. For any $im+n\in\ker(\phi)$, we see that $i(m+N)=N$ means that $im\in N$. Hence $im+n\in N$. On the other hand, if $im+n\in N$ then $im\in N$. But this means that $im+N=N$. Hence $im+n\in\ker(\phi)$. By the first isomorphism theorem for modules, we conclude that $$\frac{M}{N}=\frac{IM+N}{N}\cong I\frac{M}{N}$$ We can now apply Nakayama's lemma II to conclude that $M/N=0$ so that $M=N$. 
\end{proof}
\end{crl}

\begin{crl}{}{} Let $(R,m)$ be a local ring. Let $M$ be a finitely generated $R$-module. Then the following are true. 
\begin{itemize}
\item $M/mM$ is a finite dimensional vector space over $R/m$. 
\item $a_1,\dots,a_n\in M$ generates $M$ as an $R$-module if and only if $a_1+mM,\dots,a_n+mM$ generates $M/mM$ as a $R/m$ vector space. 
\end{itemize} \tcbline
\begin{proof}
For the first part, we already know that $M/mM$ is an $R$-module. We notice that for any $k\in m$ and $t+mM\in M/mM$ we have that $k(t+mM)=kt+kmM$. But $kt\in m$ means that $kt+kmM=mM$. Hence $M/mM$ is well defined as an $R/m$-module. Now suppose that $M$ is finitely generated by the elements $a_1,\dots,a_n$. Let $x+mM\in M/mM$. Then there exists $r_k\in R$ such that $x=r_1a_1+\dots+r_na_n$. But this means that $$x+mM=r_1(a_1+mM)+\dots+r_n(a_n+mM)$$ This means that $M/mM$ is generated by $a_1+mM,\dots,a_n+mM$. We conclude that $M/mM$ is finite dimensional. \\~\\

Suppose that $a_1,\dots,a_n\in M$ generates $M$ as an $R$-module. By the same argument as above, we can see that $a_1+mM,\dots,a_n+mM$ is a set of generators for $M/mM$. For the other direction, suppose that $a_1+mM,\dots,a_n+mM$ generates $M/mM$ as an $R/m$-vector space. Define $N=Ra_1+\dots+Ra_n\leq M$. Set $I=J(R)=m$. We want to show that $M=IM+N$. It is clear that $IM+N\leq M$. If $x\in M$, then there exists $r_k\in R$ such that $x+mM=r_1(a_1+mM)+\dots+r_n(a_n+M)$. In particular, this means that $$x-\sum_{k=1}^nr_ka_k\in mM$$ Hence $x\in IM+N$. We can now apply the above corollary to deduce that $M=N=Ra_1+\dots+Ra_n$ so that $M$ is generated by $a_1,\dots,a_n$. And so we are done. 
\end{proof}
\end{crl}

\subsection{Change of Rings}
\begin{defn}{Extension of Scalars}{} Let $R,S$ be commutative rings. Let $\varphi:R\to S$ be a ring homomorphism. Let $M$ be an $R$-module. Define the extension of $M$ to the ring $S$ to be the $S$-module $$S\otimes_R M$$
\end{defn}

\begin{defn}{Restriction of Scalars}{} Let $R,S$ be commutative rings. Let $\varphi:R\to S$ be a ring homomorphism. Let $M$ be an $S$-module. Define the restriction of $M$ to the ring $R$ to be the $R$-module $M$ equipped with the action $$r\cdot_R m=\varphi(r)\cdot_S m$$ for all $r\in R$. 
\end{defn}

\begin{thm}{}{} Let $R,S$ be commutative rings. Let $\varphi:R\to S$ be a ring homomorphism. Then there is an isomorphism $$\Hom_S(S\otimes_R M,N)\cong\Hom_R(M,N)$$ for any $R$-module $M$ and $S$-module $N$ given as follows. 
\begin{itemize}
\item For $f\in\Hom_S(S\otimes_R M,N)$, define the map $f^+\in\Hom_R(M,N)$ by $$f^+(m)=f(1\otimes m)$$
\item For $g\in\Hom_R(M,N)$, define the map $g^-\in\Hom_S(S\otimes_RM,N)$ by $$g^-(s\otimes m)=s\cdot g(m)$$
\end{itemize}
\end{thm}

\pagebreak
\section{Exact Sequences of Modules over Commutative Rings}
\subsection{Properties of the Hom Set}
Let $R$ be a ring. Let $M,N$ be $R$-modules. Recall that in Rings and Modules that $\Hom_R(M,N)$ is a $Z(R)$-modules. When $R$ is commutative, $Z(R)=R$ so that the Hom set becomes an $R$-module. 

\begin{prp}{}{} Let $R$ be a commutative ring. Let $M,N$ be $R$-modules. Then $$\Hom_R(M,N)$$ is an $R$-module with the following binary operations. 
\begin{itemize}
\item For $\phi,\varphi:M\to N$ two $R$-module homomorphisms, define $\phi+\varphi:M\to N$ by $(\phi+\varphi)(m)=\phi(m)+\varphi(m)$ for all $m\in M$
\item For $\phi:M\to N$ an $R$-module homomorphism and $r R$, define $r\phi:M\to N$ by $(r\phi)(m)=r\cdot\phi(m)$ for all $m\in M$. 
\end{itemize} 
In particular, it is an abelian group. \tcbline
\begin{proof}
We first show that the addition operation gives the structure of a group. 
\begin{itemize}
\item Since $M$ is associative as an additive group, associativity follows
\item Clearly the zero map $0\in\Hom_R(M,N)$ acts as the additive inverse since for any $\phi\in\Hom_R(M,N)$, we have that $\phi(m)+0=0+\phi(m)=\phi(m)$ since $0$ is the additive identity for $M$
\item For every $\phi\in\Hom_R(M,N)$, the map taking $m$ to $-\phi(m)$ also lies in $\Hom_R(M,N)$. Since $-\phi(m)$ is the inverse of $\phi(m)$ in $M$ for each $m\in M$, we have that $-\phi$ is the inverse of $\phi$
\end{itemize}
We now show that 
\begin{itemize}
\item Let $r,s\in R$, we have that $((sr)\phi)(m)=(sr)\cdot\phi(m)=s\cdot(r\cdot\phi(m))=s(r(\phi))(m)$ and hence we showed associativity. 
\item It is clear that $1_R\in R$ acts as the identity of the operation. 
\end{itemize}
Thus we are done. 
\end{proof}
\end{prp}

\begin{prp}{}{} Let $R$ be a ring. Let $I$ be an indexing set. Let $M_i,N$ be $R$-modules for $i\in I$. Then the following are true. 
\begin{itemize}
\item There is an isomorphism $$\Hom\left(\bigoplus_{i\in I}M_i,N\right)\cong\bigoplus_{i\in I}\Hom(M_i,N)$$
\item There is an isomorphism $$\Hom\left(\prod_{i\in I}M_i,N\right)\cong\prod_{i\in I}\Hom(M_i,N)$$
\end{itemize} \tcbline
\end{prp}

\begin{defn}{Induced Map of Hom}{} Let $R$ be a commutative ring. Let $M_1,M_2,N$ be $R$-modules. Let $f:M_1\to M_2$ be an $R$-module homomorphism. Define the induced map $$f^\ast:\Hom_R(M_2,N)\to\Hom(M_1,N)$$ by the formula $\varphi\mapsto\varphi\circ f$
\end{defn}

\begin{lmm}{}{} Let $R$ be a commutative ring. Let $M_1,M_2,N$ be $R$-modules. Let $f:M_1\to M_2$ be an $R$-module homomorphism. Then the induced map $$f^\ast:\Hom(M_2,N)\to\Hom(M_1,N)$$ is an $R$-module homomorphism. 
\end{lmm}

\subsection{Applying Hom and Tensor to Exact Sequences}
\begin{prp}{}{} Let $R$ be a commutative ring. Let the following be an exact sequence of $R$-modules. \\~\\
\adjustbox{scale=1.0,center}{\begin{tikzcd}
	0 & M_1 & M_2 & M_3 & 0
	\arrow[from=1-1, to=1-2]
	\arrow["f", from=1-2, to=1-3]
	\arrow["g", from=1-3, to=1-4]
	\arrow[from=1-4, to=1-5]
\end{tikzcd}}\\~\\ Let $N$ be an $R$-module. Then the following sequence \\~\\
\adjustbox{scale=1,center}{\begin{tikzcd}
	0 & {\Hom_R(M_3.N)} & {\Hom_R(M_2,N)} & {\Hom_R(M_1,N)}
	\arrow[from=1-1, to=1-2]
	\arrow["{g^\ast}", from=1-2, to=1-3]
	\arrow["{f^\ast}", from=1-3, to=1-4]
\end{tikzcd}} \\~\\ 
is exact. \tcbline
\begin{proof}~\\
\begin{itemize}
\item We first show that $g^\ast$ is injective. Let $\phi,\rho\in\Hom(C,G)$ such that $g^\ast(\phi)=g^\ast(\rho)$. This means that $\phi\circ g=\rho\circ g$. Let $c\in C$. Since $g$ is surjective, there exists $b\in B$ such that $g(b)=c$. Then $$\phi(c)=\phi(g(b))=\rho(g(b))=\rho(c)$$ Hence $\phi=\rho$. \\~\\

Now we show that $\im(g^\ast)\subseteq\ker(f^\ast)$. Let $g^\ast(\phi)\in\Hom(B,G)$ for $\phi\in\Hom(C,G)$. We want to show that $f^\ast(g^\ast(\phi))=0$. But we have that $$(\phi\circ g\circ f)(a)=\phi(g(f(a))=\phi(0)=0$$ since $\im(f)=\ker(g)$. Thus we conclude. \\~\\

Finally we show that $\ker(f^\ast)\subseteq\im(g^\ast)$. Let $f^\ast(\phi)=0$ for $\phi\in\Hom(B,G)$. This means that $\phi\circ f=0$ or in other words, $\im(f)\subseteq\ker(\phi)$.
Since $\phi(k)=0$ for all $k\in\im(f)$, $\phi$ descends to a map $\overline{\phi}:\frac{B}{\im(f)}\to G$. But $\im(f)=\ker(g)$ hence this is equivalent to a map $\overline{\phi}:\frac{B}{\ker(g)}\to G$. But by the first isomorphism theorem and the fact that $g$ is surjective, we conclude that $\overline{g}:\frac{B}{\ker(g)}\overset{g}{\cong} C$, where $b+\ker(g)\mapsto g(b)$. Thus we have constructed a map $\overline{\phi}\circ\overline{g}^{-1}:C\to G$ given by $g(b)\mapsto b+\ker(g)\mapsto\phi(b)$. But now $g^\ast(\overline{\phi}\circ\overline{g}^{-1})$ is the map defined by $$b\mapsto g(b)\mapsto b+\ker(g)\mapsto\phi(b)$$ and so this map is exactly $\phi$. Thus $\phi\in\im(g^\ast)$. 
\end{itemize}
\end{proof}
\end{prp}

\begin{prp}{}{} Let $R$ be a ring. Let the following be an exact sequence of $R$-modules. \\~\\
\adjustbox{scale=1.0,center}{\begin{tikzcd}
	0 & M_1 & M_2 & M_3 & 0
	\arrow[from=1-1, to=1-2]
	\arrow["f", from=1-2, to=1-3]
	\arrow["g", from=1-3, to=1-4]
	\arrow[from=1-4, to=1-5]
\end{tikzcd}}\\~\\ Let $N$ be an $R$-module. Then the following sequence \\~\\
\adjustbox{scale=1,center}{\begin{tikzcd}
	{M_1\otimes N} & {M_2\otimes N} & {M_3\otimes N} & 0
	\arrow["{f\otimes\text{id}_N}", from=1-1, to=1-2]
	\arrow["{g\otimes\text{id}_N}", from=1-2, to=1-3]
	\arrow[from=1-3, to=1-4]
\end{tikzcd}} \\~\\
is exact. 
\end{prp}

However, one can observe that we did not imply that $M_1\otimes N\to M_2\otimes N$ is                                                             injective. Indeed, this is because tensoring does not preserve injections. 

\pagebreak
\section{Algebra Over a Commutative Ring}
\subsection{Commutative Algebras}
\begin{defn}{Commutative Algebras}{} Let $R$ be a commutative ring. A commutative $R$-algebra is an $R$-algebra $A$ that is commutative. 
\end{defn}

\begin{prp}{}{} Let $R$ be a commutative ring. Then the following are equivalent characterizations of a commutative $R$-algebra. 
\begin{itemize}
\item $A$ is a commutative $R$-algebra
\item $A$ is a commutative ring together with a ring homomorphism $f:R\to A$
\end{itemize}\tcbline
\begin{proof}
Suppose that $A$ is an $R$-algebra. Then define a map $f:R\to A$ by $f(r)=r\cdot 1$ where $r\cdot 1$ is the module operation on $A$. Then clearly this is a ring homomorphism. \\~\\
Suppose that $A$ is a commutative ring together with a ring homomorphism $f:R\to A$. Define an action $\cdot:R\times A\to A$ by $r\cdot a=f(r)a$. Then this action clearly allows $A$ to be an $R$-module. 
\end{proof}
\end{prp}

Under the correspondence of associative algebra, the above proposition gives a another correspondence between the first one. $$\left\{(A,R)\;\bigg{|}\;\substack{A\text{ is a commutative }\\R\text{-algebra}}\right\}\;\;\overset{1:1}{\longleftrightarrow}\;\;\left\{\phi:R\to A\;\bigg{|}\;\substack{\phi\text{ is a ring homomorphism}\\\text{ such that }f(R)\subseteq Z(A)=A}\right\}\;\;$$ In particular, the construction above are inverses of each other so that it gives the one-to-one correspondence. 

\subsection{Finitely Generated Algebra}
\begin{defn}{Finitely Generated Algebra}{} Let $A$ be a commutative algebra over a ring $R$. We say that $A$ is a finitely generated algebra if there exists a finite set of elements $a_1,\dots,a_n$ such that $A$ is generated by $a_1,\dots,a_n$. Explicitly, this means that for all $a\in A$, there exists $c_{i_1,\dots,i_n}\in R$ for $i_1,\dots,i_n\in\N$ such that $$a=\sum_{i_1,\dots,i_n} c_{i_1,\dots,i_n}a_1^{i_1}\cdots a_n^{i_n}$$
\end{defn}

Finitely generated algebras are also called algebra of finite type. 

\begin{thm}{}{} Let $A$ be a commutative algebra over a ring $R$. Then the following are equivalent. 
\begin{itemize}
\item $A$ is a finitely generated algebra over $R$
\item There exists elements $a_1,\dots,a_n\in A$ such that the evaluation homomorphism $$\phi:R[x_1,\dots,x_n]\to A$$ given by $\phi(f)=f(a_1,\dots,a_n)$ is a surjection
\item There is an isomorphism $$A\cong\frac{R[x_1,\dots,x_n]}{I}$$ for some ideal $I$
\end{itemize}
\end{thm}

\begin{defn}{Finitely Presented Algebra}{} Let $R$ be a ring. Let $A=R[x_1,\dots,x_n]/I$ be a finitely generated algebra over $R$ for some ideal $I$. We say that $A$ is finitely presented if $I$ is finitely generated. 
\end{defn}

\begin{lmm}{}{} Let $R$ be a ring, considered as an algebra over $\Z$. If $R$ is finitely generated over $\Z$, then $R$ is finitely presented. \tcbline
\begin{proof}
Trivial since $\Z$ is a principal ideal domain. 
\end{proof}
\end{lmm}

\pagebreak
\section{Localization}
\subsection{Localization of a Ring}
\begin{defn}{Multiplicative Set}{} Let $R$ be a commutative ring. $S\subseteq R$ is a multiplicative set if $1\in S$ and $S$ is closed under multiplication: $x,y\in S$ implies $xy\in S$
\end{defn}

\begin{defn}{Localization of a Ring}{} Let $R$ be a commutative ring and $S\subseteq R$ be a multiplicative set. Define the ring of fractions of $R$ with respect to $S$ by $$S^{-1}R=\left\{\frac{r}{s}|r\in R,s\in S\right\}/\sim$$ where $\sim$ is defined by $$\frac{r}{s}\sim\frac{r'}{s'}\text{ if and only if }\exists v\in S\text{ such that }v(ru'-r'u)=0$$
If $S=\{1,f,f^2,\dots\}$ then we write $S^{-1}R=R_f=R[1/f]$. 
\end{defn}

\begin{prp}{}{} Let $S^{-1}R$ be a ring of fractions. 
\begin{itemize}
\item $\sim$ as defined in the ring of fractions is an equivalence relation
\item $(S^{-1}R,+,\times)$ is a ring
\item The map $\phi:R\to S^{-1}R$ defined by $\phi(r)\to\frac{r}{1}$ is a ring homomorphism
\end{itemize}\tcbline
\begin{proof}~\\
\begin{itemize}
\item Trivial
\item Define addition by $\frac{r}{s}+\frac{r'}{s'}=\frac{rs'+r's}{ss'}$ and multiplication by $\frac{r}{s}\cdot\frac{r'}{s'}=\frac{rr'}{ss'}$. Clearly addition is abelian, and has identity $\frac{0}{1}$ and inverse $\frac{-r}{s}$ for any $\frac{r}{s}\in S^{-1}R$. Multiplication also has identity $\frac{1}{1}$. 
\item We have that $\phi(r+s)=\frac{r+s}{1}=\frac{r}{1}+\frac{s}{1}=\phi(r)+\phi(s)$ and $\phi(rs)=\frac{rs}{1}=\frac{r}{1}\cdot\frac{s}{1}=\phi(r)\cdot\phi(s)$ for any $r,s\in R$. 
\end{itemize}
\end{proof}
\end{prp}

\begin{thm}{Universal Property}{} Let $g:A\to B$ be a ring homomorphism such that $g(s)$ is a unit in $B$ for all $s\in S$. Then there exists a unique ring homomorphism $h:S^{-1}A\to B$ such that $g=h\circ\phi$. In other words, the following diagram commutes: \\~\\
\adjustbox{scale=1.1,center}{\begin{tikzcd}
A\arrow[r, "\phi"]\arrow[rd, "g"] & S^{-1}A\arrow[d, "\exists!h", dashed]\\
&B
\end{tikzcd}}
\end{thm}

\subsection{Localization at a Prime Ideal}
\begin{lmm}{}{} Let $R$ be a ring and $P$ a prime ideal of $R$. Then $R\setminus P$ is a multiplicative set. \tcbline
\begin{proof}
By definition, $xy\in P$ implies $x\in P$ or $y\in P$, since $R\setminus P$ removes all these elements, we have that $x\notin P$ and $y\notin P$ implies that $xy\notin P$. 
\end{proof}
\end{lmm}

\begin{defn}{Localization on Prime Ideals}{} Let $R$ be a commutative ring. Let $P$ be a prime ideal. Denote $$R_p=(R\setminus P)^{-1}R$$ the localization of $R$ at $P$. 
\end{defn}

\begin{lmm}{}{} Let $R$ be an integral domain. Then the localization $$(R\setminus(0))^{-1}R$$ is exactly the field of fractions of $R$. 
\end{lmm}

\begin{prp}{}{} Let $R$ be a ring and let $p$ be a prime ideal of $R$. Then $R_p$ is a local ring. \tcbline
\begin{proof}
Let $I$ be the set of all non-units of $R_p$. It is sufficient to show that $I$ is an ideal by the above lemma. Clearly if $i\in I$ then $r\cdot i$ is also not invertible. Explicitly, we have $$I=\left\{\frac{r}{s}\in R_p\bigg{|}r\in p\right\}$$ Let $\frac{r_1}{s_1},\frac{r_2}{s_2}\in I$, then $\frac{r_1}{s_1}+\frac{r_2}{s_2}=\frac{r_1s_2+r_2s_1}{s_1s_2}$ is in $I$ since $r_1,r_2\in P$ and $P$ being an ideal implies $r_1s_2+r_2s_1\in P$. 
\end{proof}
\end{prp}

Be wary that in general localizations does not result in a local ring. This happens only when we are localizing with respect to a prime ideal. The importance of prime ideals is not explicit in the above because only using prime ideals $P$ can $R\setminus P$ be a multiplicative set which ultimately allows localization to make sense. 

\subsection{Properties of Localization}
\begin{prp}{}{} Localization commutes with direct sum of modules and quotient modules. 
\end{prp}

\subsection{Localization of a Module}
\begin{defn}{Localization of a Module}{} Let $R$ be a commutative ring and $S\subseteq R$ be a multiplicative set Let $M$ be a $R$-module. Define the ring of fractions of $M$ with respect to $S$ by $$S^{-1}M=\left\{\frac{m}{s}|m\in M,s\in S\right\}/\sim$$ where $\sim$ is defined by $$\frac{m}{s}\sim\frac{m'}{s'}\text{ if and only if }\exists v\in S\text{ such that }v(mu'-m'u)=0$$
If $S=\{1,f,f^2,\dots\}$ then we write $S^{-1}M=M_f=M[1/f]$. 
\end{defn}

\begin{prp}{}{} Let $S$ be a multiplicative set of a ring $R$. Then localization at $S$ preservers exact sequences. 
\end{prp}

\begin{prp}{}{} Let $M$ be an $A$-module. Then the $S^{-1}A$ modules $S^{-1}M$ is isomorphic to $S^{-1}A\otimes_AM$. More precisely, there exists a unique isomorphism $f:S^{-1}A\otimes_AM\to S^{-1}M$ such that $$f((a/s)\otimes m)=am/s$$
\end{prp}

\pagebreak
\section{Primary Decomposition}
\subsection{Support of a Module}
\begin{defn}{Support of a Module}{} Let $A$ be a commutative ring. Let $M$ be an $A$-module. The support of $M$ is the subset $$\text{Supp}(M)=\{P\text{ a prime ideal of }A\;|\;M_P\neq 0\}$$
\end{defn}

\subsection{Associated Prime}
\begin{defn}{Associated Prime}{} Let $M$ be an $A$-module. An associated prime $P$ of $M$ is a prime ideal of $A$ such that there exists some $m\in M$ such that $P=\text{Ann}(m)$. 
\end{defn}

\subsection{Primary Ideals}
\begin{defn}{Primary Ideals}{} Let $R$ be a commutative ring. Let $Q$ be a proper ideal of $R$. We say that $Q$ is a primary ideal of $R$ if $fg\in Q$ implies $f\in Q$ or $g^m\in Q$ for some $m>0$. 
\end{defn}

\begin{lmm}{}{} Let $A$ be a commutative ring. Let $Q$ be a primary ideal of $A$. Then $\sqrt{Q}$ is the smallest prime ideal containing $Q$. 
\end{lmm}

\begin{lmm}{}{} Let $R$ be a Noetherian ring and $I$ be a proper ideal that is not primary. Then $$I=J_1\cap J_2$$ for some ideals $J_1,J_2\neq I$. 
\end{lmm}

\begin{defn}{P-Primary Ideals}{} Let $A$ be a commutative ring. Let $P$ be a prime ideal. Let $Q$ be an ideal. We say that $Q$ is a $P$-primary ideal of $A$ if $$Q=\sqrt{P}$$
\end{defn}

\begin{thm}{}{} Let $A$ be a Noetherian ring and $Q$ an ideal of $A$. Then $Q$ is $P$-primary if and only if $\text{Ann}(A/Q)=\{P\}$. 
\end{thm}

\subsection{Primary Decomposition}
We want to express ideal $I$ in $R$ as $I=P_1^{e_1}\cdots P_n^{e_n}$ similar to a factorization of natural numbers, for some prime ideals $P_1,\dots,P_n$. However this notion fails and thus we have the following new type of ideal. 

\begin{defn}{Primary Decompositions}{} Let $A$ be a commutative ring. Let $I$ be an ideal of $A$. A primary decomposition $I$ consists of primary ideals $Q_1,\dots,Q_r$ of $A$ such that $$I=Q_1\cap\cdots\cap Q_r$$
\end{defn}

\begin{defn}{Minimal Primary Decompositions}{} Let $A$ be a commutative ring. Let $I$ be an ideal of $A$. Let $$I=Q_1\cap\cdots\cap Q_r$$ be a primary decomposition of $I$. We say that the decomposition is minimal if the following are true. 
\begin{itemize}
\item Each $\sqrt{Q_i}$ are distinct for $1\leq i\leq r$
\item Removing a primary ideal changes the intersection. This means that for any $i$, $I\neq\bigcap_{j\neq i}Q_j$
\end{itemize}
\end{defn}

\begin{thm}{}{} Every proper ideal in a Noetherian ring has a primary decomposition. 
\end{thm}

\begin{lmm}{}{} Let $\phi:R\to S$ be a ring homomorphism and $Q$ be a primary ideal in $S$. Then $\phi^{-1}(Q)$ is primary in $R$. 
\end{lmm}

\pagebreak
\section{Integral Dependence}
\subsection{Integral Extensions}
\begin{defn}{Integral Elements}{} Let $B$ be a ring and let $A\subseteq B$ be a subring. Let $b\in B$. We say that $b$ is integral over $A$ if there exists a monic polynomial $p(x)=x^n+a_{n-1}x^{n-1}+\dots+a_0\in A[x]$ such that $p(b)=0$. 
\end{defn}

\begin{prp}{}{} Let $B$ be a ring and let $A\subseteq B$. Let $b\in B$. Then the following are equivalent. 
\begin{itemize}
\item $b$ is integral over $A$
\item The subring $A[b]\subseteq B$ is finite over $A$
\item There exists an $A$ sub-algebra $A'\subseteq B$ such that $A[b]\subseteq A'$ and $A'$ is finite over $A$. 
\end{itemize}
\end{prp}

\begin{prp}{}{} Let $B$ be a ring and let $A\subseteq B$ be a subring. Let $b_1,b_2\in B$ be integral over $A$. Then $b_1+b_2$ and $b_1b_2$ are both integral over $A$. 
\end{prp}

\begin{defn}{Integral Extensions}{} Let $B$ be a ring and let $A\subseteq B$ be a subring. We say that $B$ is integral over $A$ if all elements of $B$ are integral over $A$. 
\end{defn}

\begin{lmm}{}{} Let $A\subseteq B\subseteq C$ be rings. If $C$ is integral over $B$ and $B$ is integral over $A$, then $C$ is integral over $A$. 
\end{lmm}

\begin{defn}{Integral Closure}{} Let $B$ be an $A$-algebra. Define the subring $$\overline{A}=\{b\in B|b\text{ is integral over }A\}$$ to be the integral closure of $A$ in $B$. If $\overline{A}=A$, then we say that $A$ is integrally closed in $B$. 
\end{defn}

\begin{lmm}{}{} Let $B$ be a ring and let $A\subseteq B$ be a subring. Then $\overline{A}$ is an integral extension of $A$. 
\end{lmm}

\begin{defn}{Normal Domains}{} Let $R$ be a domain. We say that $R$ is normal (intergrally closed) if $A$ is integrally closed in its field of fractions. \\~\\
The integral closure of $R$ in $\text{Frac}(R)$ is called the normalization of $R$. 
\end{defn}

\subsection{The Going-Up and Going-Down Theorems}

\subsection{Dedekind Domains}
\begin{defn}{Dedekind Domains}{} Let $R$ be a ring. We say that $R$ is a dedekind domain if the following are true. 
\begin{itemize}
\item $R$ is an integral domain
\item $R$ is an integrally closed
\item $R$ is Noetherian
\item Every non-zero prime ideal of $R$ is maximal
\end{itemize}
\end{defn}


\pagebreak
\section{Discrete Valuation Rings}
\subsection{Discrete Valuation Rings}
\begin{defn}{Totally Ordered Group}{} A totally ordered group is a group $G$ with a total order "$\leq$" such that it is
\begin{itemize}
\item a left ordered group: $a\leq b$ implies $ca\leq cb$ for all $a,b,c\in G$
\item a right ordered group: $a\leq b$ implies $ac\leq bc$ for all $a,b,c\in G$
\end{itemize}
\end{defn}

\begin{defn}{Valuation on a Field}{} Let $K$ be a field. Let $G$ be a totally ordered abelian group. A valuation on $K$ with values in $G$ is a map $v:K\setminus\{0\}\to G$ such that for all $x,y\in K^\ast$, we have 
\begin{itemize}
\item $v(xy)=v(x)+v(y)$
\item $v(x+y)\geq\min\{v(x),v(y)\}$
\end{itemize}
We use the convention that $v(0)=\infty$. \\~\\
$v$ is said to be a discrete valuation if $G=\Z$. 
\end{defn}

\begin{prp}{}{} Let $K$ be a field and $v:K\to\Z$ a discrete valuation. Then $$\{x\in K|v(x)\geq 0\}$$ is a subring of $K$. 
\end{prp}

\begin{defn}{Discrete Valuation Rings}{} The discrete valuation ring of a discrete valuation $v:K\to\Z$ is the subset $$A=\{x\in K|v(x)\geq 0\}$$
Alternatively, any ring isomorphic to a discrete valuation ring of some discrete valuation is also called a discrete valuation. 
\end{defn}

\begin{prp}{}{} Let $R$ be a discrete valuation ring with respect to the valuation $v$. Let $t\in R$ be such that $v(t)=1$. Then the following are true. 
\begin{itemize}
\item A nonzero element $u\in R$ is a unit if and only if $v(u)=0$
\item Every non-zero ideal of $R$ is a principal ideal of the form $(t^n)$ for some $n\geq 0$
\item Every $r\in R\setminus\{0\}$ can be written in the form $r=ut^n$ for some unit $u$ and $n\geq 0$. 
\end{itemize}\tcbline
\begin{proof}~\\
\begin{itemize}
\item Let $R$ be a discrete valuation ring. Suppose that $x\in R$ is a unit. Then $v(x^{-1})=-v(x)$. Then $-v(x),v(x)\geq 0$ implies $v(x)=0$. Now if $v(y)>0$ , suppose for contradiction that $u\in R$ is an inverse of $y$, then $$0=v(1)=v(uy)=v(u)+v(y)$$ But $v(y)>0$ implies that $v(u)<0$ which implies that $u\notin R$, a contradiction. 
\item Let $t\in R$ such that $v(t)=1$. Let $x\in m$ where $v(x)=n>0$. Then $v(x)=nv(t)=v(t^n)$ means that every $x\in m$ is of the form $t^n$. Thus $m=(t)$. Since every ideal $I$ is a subset of this maximal ideal, any ideal is of the form $I=(t^n)$ for some $n>0$. 
\item Follows from the fact that $(t^n)$ is the unique maximal ideal. 
\end{itemize}
\end{proof}
\end{prp}

\begin{prp}{}{} Let $R$ be an integral domain. Then the following are equivalent. 
\begin{itemize}
\item $R$ is a discrete valuation ring
\item $R$ is a UFD with a unique irreducible element up to multiplication of a unit
\item $R$ is a Noetherian local ring with a principal maximal ideal
\end{itemize} \tcbline
\begin{proof}~\\
\begin{itemize}
\item $(1)\implies(3)$: We have seen that the set of non-units is precisely the set $m=\{x\in K|v(x)>0\}$. We show that this is an ideal. Clearly $x,y\in m$ implies $v(x+y)=\min\{v(x),v(y)\}>0$. Let $u\in R$. Then $v(ux)=v(u)+v(x)>0$ since $v(x)>0$ and $v(u)\geq 0$. \\~\\
We have seen that every ideal is of the form $(t^n)$ for some $n>0$. Thus every ascending chains of ideal must be of the form $$(t^{n_1})\subset(t^{n_2})\subset\dots$$ for $n_1>n_2>\dots$. Since $n_1,n_2,\dots$ is strictly decreasing, the chain must eventually stabilizes. This proves that $R$ is Noetherian and has principal maximal ideal. 
\item $(1)\implies(3)$:
\end{itemize}
\end{proof}
\end{prp}


\pagebreak
\section{Dimension Theory for Rings}
\subsection{Dimension and Height}
\begin{defn}{Krull Dimension}{} Let $R$ be a commutative ring. Define the Krull dimension of $R$ to be $$\dim(R)=\sup\{t\in\N|p_0\subset\dots\subset p_t\text{ for }p_0,\dots,p_t\text{ prime ideals }\}$$
\end{defn}

\begin{defn}{Height of a Prime Ideal}{} Let $p$ be a prime ideal in a ring $R$. Define the height of $p$ to be $$\text{ht}(p)=\sup\{t\in\N|p_0\subset\dots\subset p_t=p\text{ for }p_0,\dots,p_t\text{ prime ideals }\}$$
\end{defn}

\begin{lmm}{}{} Let $p$ be a prime ideal in a ring $R$. Then $$\text{ht}(p)=\dim(R_p)$$ 
\end{lmm}

\begin{thm}{Krull's Principal Ideal Theorem}{} Let $R$ be a Noetherian ring. Let $I$ be a proper and principal ideal of $R$. Let $p$ be the smallest prime ideal containing $I$. Then $$\text{ht}_R(p)\leq 1$$
\end{thm}

\subsection{Length of a Module}
\begin{defn}{Length of a Module}{} Let $R$ be a ring and let $M$ be an $R$-module. Define the length of $M$ to be $$l_R(M)=\text{sup}\{n\in\N\;|\;0=M_0\subset M_1\subset\cdots\subset M_n=M\}$$
\end{defn}

\begin{lmm}{}{} Let $R$ be a ring. Let $0\to M'\to M\to M''\to 0$ be a short exact sequence of $R$-modules. Then $$l_R(M)=l_R(M')+l_R(M'')$$
\end{lmm}

\begin{lmm}{}{} Let $(A,m)$ be a local ring and let $M$ be an $A$-module. If $mM=0$, then $$l_A(M)=\dim_{A/m}(M)$$
\end{lmm}

\begin{prp}{}{} Let $R$ be a ring and let $M$ be an $R$-module. Then the following are equivalent. 
\begin{itemize}
\item $M$ is simple
\item $l_R(M)=1$
\item $M\cong A/m$ for some maximal ideal $m$ of $A$
\end{itemize}
\end{prp}

\subsection{The Hilbert Polynomial}
\begin{defn}{The Hilbert Polynomial}{} Let $R=\bigoplus_{k=0}^\infty R_k$ be a Noetherian graded ring. Let $M=\bigoplus_{k=0}^\infty M_k$ be a graded $R$-module. Define the Hilbert function $H_M:\N\to\N$ of $R$ to be the function defined by $$H_M(n)=l_{R_0}(M_n)$$
\end{defn}

\begin{defn}{The Hilbert Series}{} Let $R=\bigoplus_{k=0}^\infty R_k$ be a Noetherian graded ring. Let $M=\bigoplus_{k=0}^\infty M_k$ be a graded $R$-module. Define the Hilbert series $HS_M\in\Z[[t]]$ of $M$ to be the formal series $$HS_M(t)=\sum_{k=0}^\infty H_M(k)t^k=\sum_{k=0}^\infty l_{R_0}(M_k)t^k$$
\end{defn}

\begin{thm}{}{} Let $R=\bigoplus_{k=0}^\infty R_k$ be a Noetherian graded ring such that $R_0$ is Artinian. Let $M=\bigoplus_{k=0}^\infty M_k$ be a graded $R$-module. Let $\lambda:\{M_i\;|\;i\in I\}\to\Z$ be an additive function Then the function $$g(t)=\sum_{k=0}^\infty\lambda(M_k)t^k$$ is a rational function and can be written in the form $$g(t)=\frac{f(t)}{\prod_{i=1}^r(1-t^{d_i})}$$ for some $f(t)\in\Z[t]$ and $d_i\in\N$. 
\end{thm}

\begin{thm}{The Fundamental Theorem of Dimension Theory}{} Let $(R,m)$ be a local Noetherian ring. Let $I$ be an $m$-primary ideal. Then the following numbers are equal. 
\begin{itemize}
\item Let $J=\bigoplus_{k=0}^\infty\frac{I^k}{I^{k+1}}$. The order of the pole at $1$ of the rational function $HS_J$. 
\item The minimum number of elements of $R$ that can generate an $m$-primary ideal of $R$
\item The dimension $\dim_{R/m}(R)$
\end{itemize}
\end{thm}

The following is a generalization of Krull's principal ideal theorem. Both of the theorems can actually be deduced directly from the fundamental theorem. 

\begin{thm}{Krull's Height Theorem}{} Let $R$ be a Noetherian ring. Let $I$ be a proper ideal generated by $n$ elements. Let $p$ be the smallest prime ideal containing $I$. Then $$\text{ht}_R(p)\leq n$$
\end{thm}

\begin{thm}{}{} Let $(R,m)$ be a Noetherian local ring and let $k=R/m$ be the residue field. Then $$\dim(R)\leq\dim_k(m/m^2)$$
\end{thm}

\subsection{Global Dimension of a Ring}







\end{document}