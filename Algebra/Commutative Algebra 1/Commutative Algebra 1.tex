\documentclass[a4paper]{article}

\input{C:/Users/liula/Desktop/Latex/Headers V1.2.tex}

\pagestyle{fancy}
\fancyhf{}
\rhead{Labix}
\lhead{Commutative Algebra 1}
\rfoot{\thepage}

\title{Commutative Algebra 1}

\author{Labix}

\date{\today}
\begin{document}
\maketitle
\begin{abstract}
\end{abstract}
\pagebreak
\tableofcontents

\pagebreak
\section{Ideals Of a Commutative Ring}
\subsection{Basic Operations on Ideals}
Let $R$ be a commutative ring. Recall that an ideal of $R$ is a subset $I\subseteq R$ such that
\begin{itemize}
\item If $a,b\in I$, then $a+b\in I$. 
\item If $r\in R$ and $a\in I$, then $ra\in I$. 
\end{itemize}

\begin{prp}{}{} Let $R$ be a commutative ring. Let $I_1,\dots,I_n$ be ideals of $R$. Let $P_1,\dots,P_k$ be prime ideals of $R$. 
\begin{itemize}
\item Let $I$ be an ideal of $R$. If $I\subseteq\bigcup_{i=1}^kP_i$, then $I\subseteq P_i$ for some $i$. 
\item Let $P$ be an ideal of $R$. If $P\subseteq\bigcap_{i=1}^nI_i$, then $I_i\subseteq P$ for some $i$. 
\item Let $P$ be an ideal of $R$. If $P=\bigcap_{i=1}^nI_i$, then $I_i=P$ for some $i$. 
\end{itemize} \tcbline
\begin{proof}~\\
\begin{itemize}
\item We prove the contrapositive by induction $k$. When $k=1$, the case is clear. Suppose that $I\not\subseteq P_i$ for $1\leq i\leq k-1$ implies $I\not\subseteq\bigcup_{i=1}^{k-1}P_i$. Now suppose that $I\not\subseteq P_i$ for $1\leq i\leq k$. By induction hypothesis, for each $i$, there exists $x_j\in I$ such that $x_j\notin\bigcup_{i\neq j}P_i$. So $x_j\notin P_i$ for $j\neq i$. There are two cases. If $x_j\notin P_j$ for some $j$, then $x_j\notin\bigcup_{j\neq i}P_i\cup P_j=\bigcup_{i=1}^kP_i$ so we are done. If $x_j\in P_j$ for all $j$, then consider the element $y=\sum_{i=1}^k\prod_{j\neq i}x_j\in I$. Notice that $x_j\in P_j$ for $j\neq i$ implies that $\prod_{j\neq i}x_j$ lie in $P_k$ for any $k\neq i$. It is not an element of $P_i$ because $P_i$ is prime and $x_j\notin P_i$ for $j\neq i$. Then we conclude that $y$ does not lie in $P_i$ for any $i$. Hence $y\notin\bigcup_{i=1}^kP_i$ and we are done. 
\item We prove the contrapositive. Suppose that $I_i\not\subseteq P$ for all $i$. Then for each $i$, there exists $x_i\in I_i$ such that $x_i\notin P$. Then $\prod_{i=1}^nx_i\in\bigcap_{i=1}^nI_i$ is not an element of $P$ since $P$ is a prime ideal. Hence we are done. 
\item By the above, we have that $P=\bigcap_{i=1}^nI_i$ implies that $I_i\subseteq P$ for some $i$. Then $P=\bigcap_{i=1}^nI_i\subseteq I_i$ implies that $P=I_i$. 
\end{itemize}
\end{proof}
\end{prp}

\begin{prp}{}{} Let $R$ be a commutative ring. Let $I,J$ be ideals of $R$. Then there is an isomorphism of rings $$\frac{R}{I+J}\cong\frac{R/I}{(I+J)/I}$$ given by $r+(I+J)\mapsto(r+I)+((I+J)/I)$. \tcbline
\begin{proof}
Follows from the third isomorphism theorem of rings. 
\end{proof}
\end{prp}

\begin{eg}{}{} There is an isomorphism given by $$\frac{\Z[x]}{(x+1,x^2+2)}\cong\Z/3\Z$$ \tcbline
\begin{proof}
Using the above propositions, we have that 
\begin{align*}
\frac{\Z[x]}{(x+1,x^2+2)}&=\frac{\Z[x]}{(x+1)+(x^2+2)}\\
&\cong\frac{\Z[x]/(x+1)}{(3)}
\end{align*}
Indeed, the ideal $(x^2+2)$ corresponds to the ideal $(3)$ in $\frac{\Z[x]}{(x+1)}$ because the remainder of $x^2+2$ divided by $(x+1)$ is $(3)$. Now $\Z[x]/(x+1)\cong\Z$ by the evaluation homomorphism. Thus quotieting by the ideal $(3)$ gives the field $\Z/3\Z$. 
\end{proof}
\end{eg}

\begin{prp}{}{} Let $R$ be a commutative ring. Let $I,J$ be ideals of $R$. Then $\frac{R}{I}\cong\frac{R}{J}$ as $R$-modules if and only if $I=J$. \tcbline
\begin{proof}
When $I=J$ it is clear that $R/I\cong R/J$. Conversely, suppose that $\phi:R/I\to R/J$ is an $R$-module isomorphism. For any $r\in J$, we have $$\phi(r+I)=(r+J)\phi(1+I)=(r+J)(1+J)=(r+J)=0$$ Since $\phi$ is an isomorphism, we conclude that $r+I=I$, so that $r\in I$. This shows that $J\subseteq I$. Similarly one can show that $I\subseteq J$. 
\end{proof}
\end{prp}

\begin{defn}{Product of Ideals}{} Let $R$ be a commutative ring. Let $I,J$ be ideals of $R$. Define the product of $I$ and $J$ to be $$IJ=(ij\;|\;i\in I, j\in J)=\left\{\sum_{k=1}^na_kb_k\;|\;a_k\in I,b_k\in J\right\}$$
\end{defn}

\begin{lmm}{}{} Let $R$ be a commutative ring. Let $I_1,I_2$ be ideals of $R$. Then the following are true. 
\begin{itemize}
\item $I_1I_2$ is an ideal of $R$. 
\item $I_1I_2\subseteq I_1,I_2$ is an $R$-submodule of $I_1$ and $I_2$. 
\item If $I_1$ and $I_2$ are coprime, then $I_1I_2=I_1\cap I_2$. 
\end{itemize}
\end{lmm}

\begin{lmm}{}{} Let $R$ be a commutative ring. Let $I_1,I_2,I_3$ be ideals of $R$. Then the following are true. 
\begin{itemize}
\item Product is Commutative: $I_1I_2=I_2I_1$. 
\item Product is Associative: $(I_1I_2)I_3=I_1(I_2I_3)$. 
\item Distrbutivity: $I_1(I_2+I_3)=I_1I_2+I_1I_3$. 
\end{itemize}
\end{lmm}

Some more important results from Groups and Rings and Rings and Modules include: 
\begin{itemize}
\item Chinese Remainder Theorem: If $I$ and $J$ are coprime, then there is an isomorphism $$\frac{R}{IJ}=\frac{R}{I\cap J}\cong\frac{R}{I}\times\frac{R}{J}$$
\end{itemize}

\subsection{The Radical of an Ideal}
The radical of an ideal is a very different notion from the radical of module. 

\begin{defn}{Radical of an Ideal}{} Let $I$ be an ideal of a ring $R$. Define the radical of $I$ to be $$\sqrt{I}=\{r\in R\;|\;r^n\in I\text{ for some }n\in\N\}$$
\end{defn}

\begin{prp}{}{} Let $R$ be a commutative ring. Let $I$ be an ideal. Then the following are true. 
\begin{itemize}
\item $I\subseteq\sqrt{I}$
\item $\sqrt{\sqrt{I}}=\sqrt{I}$
\item $\sqrt{I^m}=\sqrt{I}$ for all $m\geq 1$
\item $\sqrt{I}=R$ if and only if $I=R$
\end{itemize} \tcbline
\begin{proof}~\\
\begin{itemize}
\item Let $r\in I$. Then $r^1\in I$ Thus by choosing $n=1$ we shows that $r^n\in I$. Thus $r\in\sqrt{I}$. 
\item By the above, we already know that $\sqrt{I}\subseteq\sqrt{\sqrt{I}}$. So let $r\in\sqrt{\sqrt{I}}$. Then there exists some $n\in\N$ such that $r^n\in\sqrt{I}$. But $r^n\in\sqrt{I}$ means that there exists some $m\in\N$ such that $(r^n)^m\in I$. But $nm\in\N$ is a natural number such that $r^{nm}\in I$. Hence $r\in\sqrt{I}$ and so we conclude. 
\item Since $I^m\subseteq I$, we know that $\sqrt{I^m}\subseteq\sqrt{I}$. Let $x\in\sqrt{I}$. Then $x^n\in I$ for some $n\in\N$. Then we have $(x^n)^m=x^{n+m}\in I^m$ so that $x\in\sqrt{I^m}$. 
\item Clearly if $I=R$ then $I\subseteq\sqrt{I}$ implies that $\sqrt{I}=R$. Conversely, $\sqrt{I}=R$ implies that $1\in\sqrt{I}$ and hence $1\in I$. Hence $I=R$. 
\end{itemize}
\end{proof}
\end{prp}

\begin{prp}{}{} Let $R$ be a commutative ring. Let $I,J$ be ideals of $R$. Then the following are true. 
\begin{itemize}
\item If $I\subseteq J$ then $\sqrt{I}\subseteq\sqrt{J}$
\item $\sqrt{IJ}=\sqrt{I\cap J}=\sqrt{I}\cap\sqrt{J}$
\item $\sqrt{I+J}=\sqrt{\sqrt{I}+\sqrt{J}}$
\end{itemize} \tcbline
\begin{proof}~\\
\begin{itemize}
\item Let $x\in\sqrt{IJ}$. Then $x^n\in IJ$. This means that there exists $i\in I$ and $j\in J$ such that $x^n=ij$. Since $I$ and $J$ are two sided ideals, we can conclude that $x^n=ij\in I,J$. Hence $x^n=ij\in I\cap J$. We conclude that $x\in\sqrt{I\cap J}$. Now let $x\in\sqrt{I\cap J}$. Then there exists $n\in\N$ such that $x^n\in I\cap J$. Then $x^n\in I$ and $x^n\in J$ implies that $x^{2n}=x^n\cdot x^n\in IJ$. We conclude that $x\in\sqrt{IJ}$. 
\item Since $IJ\subseteq I\cap J\subseteq I,J$, we already have $\sqrt{IJ}\subseteq\sqrt{I\cap J}\subseteq\sqrt{I}\cap\sqrt{J}$. Let $x\in\sqrt{I}\cap\sqrt{J}$. Then there exists $n,m\in\N$ such that $x^n\in I$ and $x^m\in J$. Then $x^n\cdot x^m=x^{n+m}\in IJ$ implies that $x\in\sqrt{IJ}$. 
\item Since $I,J\subseteq I+J$, we have $\sqrt{I}+\sqrt{J}\subseteq\sqrt{I+J}$ so that $\sqrt{\sqrt{I}+\sqrt{J}}\subseteq\sqrt{I+J}$. On the other hand, $I\subseteq\sqrt{I}$ and $J\subseteq\sqrt{J}$ implies that $I+J\subseteq\sqrt{I}+\sqrt{J}$. Then $\sqrt{I+J}\subseteq\sqrt{\sqrt{I}+\sqrt{J}}$ and so we are done. 
\end{itemize}
\end{proof}
\end{prp}

\begin{prp}{}{} Let $R$ be a commutative ring. Let $I$ be an ideal. Then $$\sqrt{I}=\bigcap_{\substack{p\text{ a prime ideal}\\I\subseteq p\subseteq R}}p$$
\end{prp}

\begin{defn}{Radical Ideals}{} Let $R$ be a commutative ring. Let $I$ be an ideal of $R$. We say that $I$ is radical if $$\sqrt{I}=I$$
\end{defn}

In particular, by the above lemma it follows that the radical of an ideal is a radical ideal. 

\begin{lmm}{}{} Let $R$ be a ring. Let $P$ be a prime ideal of $R$. Then $P$ is radical. \tcbline
\begin{proof}
We already know that $P\subseteq\sqrt{P}$. Let $x\in\sqrt{P}$. Then $x^n\in P$ for some $n\in\N$. Since $P$ is prime, by inducting downwards we deduce that $x\in P$. Thus $P$ is radical. 
\end{proof}
\end{lmm}

We conclude that there is an inclusion of types of ideal in which each inclusion is strict: $$\substack{\text{Maximal}\\\text{ideals}}\subset\substack{\text{Prime}\\\text{ideals}}\subset\substack{\text{Radical}\\\text{ideals}}$$

\subsection{The Nilradical of Commutative Rings}
Let $R$ be a ring. Recall that an element $r\in R$ is nilpotent if $r^n=0_R$ for some $n\in\N$. When $R$ is commutative, we can form an ideal out of nilpotent elements. 

\begin{defn}{Nilradicals}{} Let $R$ be a ring. Define the nilradical of $R$ to be $$N(R)=\{r\in R\;|\;r\text{ is nilpotent}\}$$
\end{defn}

Note that this is different from nilpotent ideals, as nilpotency is a property of an ideal. However the Nilradical ideal is a nil ideal and every sub-ideal of the nilradical is a nil ideal. 

\begin{prp}{}{} Let $R$ be a ring and $N(R)$ its nilradical. Then the following are true. 
\begin{itemize}
\item $N(R)$ is an ideal of $R$
\item $N(R/N(R))=0$
\end{itemize}\tcbline
\begin{proof}~\\
\begin{itemize}
\item Suppose that $r,s$ are nilpotent, meaning that $r^n=0$ and $s^m=0$. Then $(r+s)^{n+m}=0$. Moreover, if $t\in R$ then $t\cdot r$ is also nilpotent
\item Let $r\notin N(R)$. Every element $r+N(R)\in R/N(R)$ has the property that $r^n\neq 0$. Consider $(r+N(R))^n=r^n+N(R)$. If $r^n\in N(R)$ then $r^n=u$ for some nilpotent $u$, which means that $r^n$ is nilpotent and thus $r$ is nilpotent, a contradiction. This means that $r+N(R)\notin N(R/N(R))$ for all $r\notin N(R)$ and thus $N(R/N(R))=0$
\end{itemize}
\end{proof}
\end{prp}

\begin{prp}{}{} Let $R$ be a commutative ring. Then we have $$N(R)=\bigcap_{\substack{P\text{ is a prime}\\\text{ideal of }R}}P=\sqrt{(0)}$$ \tcbline
\begin{proof}
Let $x\in N(R)$. Let $P$ be an arbitrary prime ideal. Since $x$ is nilpotent, $x^n=0$ for some $n\in\N$. If $x\notin P$, then $x^2\notin P$ since $P$ is a prime ideal. Recursively we see that $x^k\notin P$ for all $k\in N\setminus\{0\}$. But $x^n=0\in P$ is a contradiction. Hence $N(R)\subseteq\bigcap_{P\in\text{Spec}(R)}P$. \\~\\

Now suppose that $x\in R$ is not nilpotent. Consider the set $$\Sigma=\{I\trianglelefteq R\;|\;x^k\notin I\text{ for all }k\geq 1\}$$ Notice that $(0)\in\Sigma$ and hence it is non-empty. Let $I_1\subseteq I_2\subseteq\cdots$ be a chain in $\Sigma$. Define $I=\bigcup_{k=1}^\infty I_k$. I claim that $I\in\Sigma$. First of all if $a,b\in I$ and $r\in R$, then $a\in I_m$ and $b\in I_n$ for some $m,n\geq 1$. Then $a,b\in I_{\max\{m,n\}}$ so that $a+b\in I_{\max\{m,n\}}\subseteq I$. Also $ra\in I_m\subseteq I$ since $I_m$ is an ideal. Hence $I$ itself is an ideal of $R$. Suppose for a contradiction that $x^n\in I$ for some $n$. Then $x^n\in I_k$ for some $k$. This is a contradiction since $I_k\in\Sigma$. Thus we know that $I\in\Sigma$. In particular, $I$ is an upper bound of $I_1\subseteq I_2\subseteq\cdots$. By Zorn's lemma, we conclude that $\Sigma$ has a maximal element, say $P$. \\~\\

Suppose for a contradiction that $P$ is not a prime ideal. Let $ab\in P$ and $a,b\notin P$. Then $P\subset P+(a),P+(b)$. Since $P$ is maximal in $\Sigma$, $P+(a)$ and $P+(b)$ cannot be in $\Sigma$, and there exists $x^m\in P+(a)$ and $x^n\in P+(b)$ for some $m,n$. Then $$x^{m+n}=x^m\cdot x^n\in (P+(a))(P+(b))=P+(ab)$$ Hence $P+(ab)\notin\Sigma$. But $ab\in P$ implies that $P+(ab)=P$. We have reached a contradiction. Thus $P$ is a prime ideal that does not contain $x$. We show that $x\notin N(R)$ implies $x\notin P$ for some prime ideal $P$. The contrapositive of this statement is $x\in P$ for all prime ideals $P$ implies $x\in N(R)$. Hence we are done. 
\end{proof}
\end{prp}

\begin{eg}{}{} Consider the ring $$R=\frac{\C[x,y]}{(x^2-y,xy)}$$ Then its nilradical is given by $N(R)=(x,y)$. \tcbline
\begin{proof}
Notice that in the ring $R$, $x^3=x(x^2)=xy=0$ and $y^3=x^6=(x^3)^2=0$ and hence $x$ and $y$ are both nilpotent elements of $R$. By definition of the nilradical, we conclude that $(x,y)\subseteq N(R)$. Now $(x,y)$ is a maximal ideal of $\C[x,y]$ because $\C[x,y]/(x,y)\cong\C$. Also notice that $(x,y)\supseteq(x^2-y,xy)$ because for any element $f(x)(x^2-y)+g(x)(xy)\in(x^2-y,xy)$, we have that 
\begin{align*}
f(x)(x^2-y)+g(x)(xy)\in(x^2-y,xy)&=(xf(x))x-f(x)y+(g(x)x)y\\
&=(xf(x))x+(xg(x)-f(x))y\in (x,y)
\end{align*}
By the correspondence theorem, $(x,y)/(x^2-y)$ is an maximal ideal of $R$. In particular, $(x,y)$ is also a prime ideal. But the $N(R)$ is the intersection of all prime ideals and hence $N(R)\subseteq(x,y)$. We conclude that $N(R)=(x,y)$. 
\end{proof}
\end{eg}

\begin{defn}{Reduced Rings}{} Let $R$ be a commutative ring. We say that $R$ is reduced if $N(R)=0$. 
\end{defn}

\begin{prp}{}{} Let $R$ be a commutative ring. Let $I$ be an ideal of $R$. Then $R/I$ is reduced if and only if $I$ is a radical ideal. 
\end{prp}

So radical, prime and maximal ideals all have characterizations using the quotient ring: 
\begin{itemize}
\item $I$ is maximal if and only if $R/I$ is a field. 
\item $I$ is prime if and only if $R/I$ is an integral domain. 
\item $I$ is radical if and only if $R/I$ is reduced. 
\end{itemize}

\subsection{The Jacobson Radical of Commutative Rings}
Let $R$ be a commutative ring. Recall that the Jacobson radical of a ring is defined to be $$J(R)=\bigcap_{m\text{ a maximal ideal}}m$$ since left and right maximal ideals coincide in $R$. Properties of the Jacobson radical include: 
\begin{itemize}
\item $J(R/J(R))=0$. 
\end{itemize}

\begin{lmm}{}{} Let $R$ be a commutative ring. Then $x\in J(R)$ if and only if $1-xy\in R^\times$ for all $y\in R$. \tcbline
\begin{proof}
Suppose that $x\notin J(R)$. Then $x\notin m$ for some maximal ideal $m$. Then $R=m+(x)$ since $m$ is maximal. Then there exists $p\in m$ and $y\in R$ such that $1=p+xy$. Then $1-xy=p\in m\notin R^\times$. \\~\\

Suppose that $1-xy\notin R^\times$ for some $y\in R$. Then $(1-xy)$ is a proper ideal of $R$. Then there exists a maximal ideal $m$ such that $(1-xy)\subseteq m$. Since $1\notin m$ and $y$ is arbitrary, we must have that $x\notin m$. Hence $x\notin J(R)$. 
\end{proof}
\end{lmm}

\begin{lmm}{}{} Let $R$ be a commutative ring. Then $x\in R$ is a unit if and only if $[x]\in R/J(R)$ is a unit. \tcbline
\begin{proof}
Suppose that $x\in R$ is a unit. Then there exists $y\in R$ such that $xy=1$. Then $[x][y]=[1]$ so we are done. Now suppose that $[x][y]=[1]$ for some $y\in R$. Then there exists $m\in J(R)$ such that $xy=1+m$. By the above lemma, $1+m$ is a unit hence $x$ is a unit. 
\end{proof}
\end{lmm}

\subsection{The Correspondence between Ideals and the Quotient}
\begin{defn}{Max Spectrum of a Ring}{} Let $A$ be a commutative ring. Define the max spectrum of $A$ to be $$\text{maxSpec}(A)=\{m\subseteq A\;|\;m\text{ is a maximal ideal of }A\}$$
\end{defn}

\begin{defn}{Spectrum of a Ring}{} Let $A$ be a commutative ring. Define the spectrum of $A$ to be $$\text{Spec}(A)=\{p\subseteq A\;|\;p\text{ is a prime ideal of }A\}$$
\end{defn}

\begin{eg}{}{} Consider the following commutative rings. 
\begin{itemize}
\item $\text{Spec}(\Z/6\Z)=\{(2+6\Z),(3+6\Z)\}$
\item $\text{Spec}(\Z/8\Z)=\{(2+8\Z)\}$
\item $\text{Spec}(\Z/24\Z)=\{(2+24\Z),(3+24\Z)\}$
\item $\text{Spec}(\R[x])=\{(f)\;|\;f\text{ is irreducible }\}$
\end{itemize} \tcbline
\begin{proof}~\\
\begin{itemize}
\item The only ideals of $\Z/6\Z$ are $(2+6\Z)$ and $(3+6\Z)$. We need to find which ones are prime ideals. Now $\Z/6\Z\setminus(2+6\Z)$ consists of $1+6\Z$, $3+6\Z$ and $5+6\Z$. No multiplication of these elements give an element of $(2+6\Z)$. So any two elements in $\Z/6\Z$ which multiply to an element of $(2+6\Z)$ must contain one element that lie in $(2+6\Z)$. Hence $(2+6\Z)$ is prime. This is similar for $(3+6\Z)$. Hence $\text{Spec}(\Z/6\Z)=\{(2+6\Z),(3+6\Z)\}$. 
\item The only ideals of $\Z/8\Z$ are $(2+8\Z)$ and $(4+8\Z)$. A similar argument as above shows that $(2+8\Z)$ is a prime ideal. However, $6+8\Z\notin(4+8\Z)$ while $(6+8\Z)^2=4+8\Z\in(4+8\Z)$ which shows that $(4+8\Z)$ is not a prime ideal. 
\item A similar proof as above ensues. 
\item Recall that $\R[x]$ is a principal ideal domain. Let $I=(f)$ be a prime ideal of $\R[x]$. Then $f$ is irreducible. Thus every prime ideal of $\R[x]$ is of the form $(f)$ for $f$ an irreducible polynomial. 
\end{itemize}
\end{proof}
\end{eg}

\begin{lmm}{}{} Let $R,S$ be commutative rings. Let $f_1:R\times S\to R$ and $f_2:R\times S\to S$ denote the projection maps. Then the map $$f_1^\ast\amalg f_2^\ast:\text{Spec}(R)\amalg\text{Spec}(S)\to\text{Spec}(R\times S)$$ is a bijection. \tcbline
\begin{proof}
The core of the proof is the fact that $P$ is a prime ideal of $R\times S$ if and only if $P=R\times Q$ or $P=V\times S$ for either a prime ideal $Q$ of $P$ or a prime ideal $V$ of $S$. It is clear that if $Q$ is a prime ideal of $S$ and $V$ is a prime ideal of $R$, then $R\times Q$ and $V\times S$ are both prime ideals of $R\times S$. \\~\\

So suppose that $P$ is a prime ideal in $R\times S$. Let $e_1=(1,0)$ and $e_2=(0,1)$. Since $P\neq R$, at least one of $e_1$ or $e_2$ is not in $P$. Without loss of generality assume that $e_1\notin P$. But $e_1e_2=0\in P$ and $P$ being prime implies that $e_2\in P$. Since $e_2$ is the identity of $\{0\}\times S\cong S$, we conclude that $\{0\}\times S\subseteq P$. By the correspondence theorem, the projection map $f_1:R\times S\to R$ gives a bijection between prime ideals of $R\times S$ that contain $\{0\}\times S$ and prime ideals of $R$. So $f_1(P)$ is a prime ideal of $R$. Thus $P=f_1(P)\times S$ which is exactly what we wanted. \\~\\

Now the bijection is clear. $f_1^\ast\amalg f_2^\ast$ sends a prime ideal $P$ of $R$ to $P\times S$ and it sends a prime ideal $Q$ of $S$ to $R\times Q$. This map is surjective by the above argument. It is injective by inspection. 
\end{proof}
\end{lmm}

\begin{thm}{}{} Let $R$ be a commutative ring. Let $I$ be an ideal of $R$. Denote $\varphi$ to be the inclusion preserving one-to-one bijection $$\left\{\substack{\text{Ideals of }R\\\text{containing }I}\right\}\;\;\overset{1:1}{\longleftrightarrow}\;\;\left\{\text{Ideals of }R/I\right\}$$ from the correspondence theorem for rings. In other words, $\varphi(A)=A/I$. Let $J\subseteq R$ be an ideal containing $I$. Then the following are true. 
\begin{itemize}
\item $J$ is a radical ideal if and only if $\varphi(J)=J/I$ is a radical ideal. 
\item $J$ is a prime ideal if and only if $\varphi(J)=J/I$ is a prime ideal. 
\item $J$ is a maximal ideal if and only if $\varphi(J)=J/I$ is a maximal ideal. 
\end{itemize} \tcbline
\begin{proof}~\\
\begin{itemize}
\item Let $J$ be a radical ideal. Suppose that $r+I\in\sqrt{J/I}$. This means that $(r+I)^n=r^n+I\in J/I$ for some $n\in\N$. But this means that $r^n\in J$. This implies that $r\in\sqrt{J}=J$. Thus $r+I\in J/I$ and we conclude that $\sqrt{J/I}\subseteq J/I$. Since we also have $J/I\subseteq\sqrt{J/I}$, we conclude. \\~\\

Now suppose that $J/I$ is a radical ideal. Let $r\in\sqrt{J}$. This means that $r^n\in J$ for some $n\in\N$. Now $r^n+I=(r+I)^n\in J/I$ implies that $r+I\in\sqrt{J/I}=J/I$. Hence $r\in J$ and so $\sqrt{J}\subseteq J$. Since we also have that $J\subseteq\sqrt{J}$, we conclude. 

\item Let $J$ be a prime ideal. Then $R/J$ is an integral domain. By the second isomorphism theorem, we have that $R/J\cong(R/I)/(J/I)$ and hence $(R/I)/(J/I)$ is also an integral domain. Hence $J/I$ is a prime ideal. The converse is also true. 

\item Let $J$ be a maximal ideal. Then $R/J$ is a field. By the second isomorphism theorem, we have that $R/J\cong(R/I)/(J/I)$ and hence $(R/I)/(J/I)$ is also a field. Hence $J/I$ is a maximal ideal. The converse is also true. 
\end{itemize}
\end{proof}
\end{thm}

Another way to write the bijections is via spectra: $$\text{Spec}(R/I)\;\;\overset{1:1}{\longleftrightarrow}\;\;\{P\in\text{Spec}(R)\;|\;I\subseteq P\}$$ and $$\text{maxSpec}(R/I)\;\;\overset{1:1}{\longleftrightarrow}\;\;\{m\in\text{maxSpec}(R)\;|\;I\subseteq m\}$$

\subsection{Extensions and Contractions of Ideals}
\begin{defn}{Extension of Ideals}{} Let $R,S$ be commutative rings. Let $f:R\to S$ be a ring homomorphism. Let $I$ be an ideal of $R$. Define the extension $I^e$ of $I$ to $S$ to be the ideal $$I^e=\langle f(i)\;|\;i\in I\rangle$$
\end{defn}

\begin{prp}{}{} Let $R,S$ be commutative rings. Let $f:R\to S$ be a ring homomorphism. Let $I,I_1,I_2$ be an ideal of $R$. Then the following are true regarding the extension of ideals. 
\begin{itemize}
\item If $I_1\subseteq I_2$, then $I_1^e\subseteq I_2^e$. 
\item Closed under sum: $(I_1+I_2)^e=I_1^e+I_2^e$
\item $(I_1\cap I_2)^e\subseteq I_1^e\cap I_2^e$
\item Closed under products: $(I_1I_2)^e=I_1^eI_2^e$
\item $(\sqrt{I})^e\subseteq\sqrt{I^e}$
\end{itemize} \tcbline
\begin{proof}~\\
\begin{itemize}
\item Let $x\in I_1^e$. Then $x=\sum s_kf(i_k)$ for some $i_k\in I_1$. Then $i_k\in I_2$ implies that $x\in I_2^e$. 
\item Since $I_1,I_2\subseteq I_1+I_2$, we have $I_1^e+I_2^e\subseteq(I_1+I_2)^e$. Conversely, let $x,\in(I_1+I_2)^e$. Then $x=\sum s_kf(i_k)$ for $i_k\in I_1+I_2$. Then we have $$x=\sum_{i_k\in I_1}s_kf(i_k)+\sum_{i_k\in I_2}s_kf(i_k)\in I_1^e+I_2^e$$ so we conclude. 
\item Since $I_1\cap I_2\subseteq I_1,I_2$ we are done. 
\item It suffices to check the generators lie in each other. Let $x\in I_1I_2$. Then $x=\sum i_kj_k$ for some $i_k\in I_1$ and $j_k\in I_2$. Then $f(x)=\sum f(i_k)f(j_k)$. Since $f(i_k)\in I_1^e$ and $f(j_k)^e$, then $f(x)\in I_1^eI_2^e$ so we conclude that $(I_1I_2)^e\subseteq I_1^eI_2^e$. Conversely, suppose that $x\in I_1^eI_2^e$. Then $x=\sum f(i_k)(j_k)$ for $i_k\in I_1$ and $j_k\in I_2$. Since $f$ is a ring homomorphism, we have that $$x=\sum f(i_k)f(j_k)=f\left(\sum i_kj_k\right)$$ Since $\sum i_kj_k\in I_1I_2$, we conclude that $x\in I_1^eI_2^e$. 
\item We have that $$
(\sqrt{I})^e=\left(f(i)\;\bigg{|}\;i\in\bigcap_{\substack{P\text{ prime}\\I\subseteq P}}P\right)\subseteq f\left(\bigcap_{\substack{P\text{ prime}\\I\subseteq P}}f(P)\right)\subseteq f\left(\bigcap_{\substack{Q\text{ prime}\\I^e\subseteq Q}}f(f^{-1}(Q))\right)$$ The last inclusion follows since for $I^e\subseteq Q$, we must have that $I\subseteq f^{-1}(Q)$. Then we have that $$(\sqrt{I})^e=f\left(\bigcap_{\substack{Q\text{ prime}\\I^e\subseteq Q}}Q\right)=\sqrt{I^e}$$ and so we are done. 
\end{itemize}
\end{proof}
\end{prp}

\begin{defn}{Contraction of Ideals}{} Let $R,S$ be commutative rings. Let $f:R\to S$ be a ring homomorphism. Let $J$ be an ideal of $S$. Define the contraction $J^c$ of $J$ to $R$ to be the ideal $$J^c=f^{-1}(J)$$
\end{defn}

\begin{prp}{}{} Let $R,S$ be commutative rings. Let $f:R\to S$ be a ring homomorphism. Let $J,J_1,J_2$ be an ideal of $S$. Then the following are true regarding the extension of ideals. 
\begin{itemize}
\item If $J_1\subseteq J_2$, then $J_1^c\subseteq J_2^c$. 
\item $(J_1+J_2)^c\supseteq J_1^c+J_2^c$
\item Closed under intersections: $(J_1\cap J_2)^c=J_1^c\cap J_2^c$
\item $(J_1J_2)^c\supseteq J_1^cJ_2^c$
\item Closed under taking radicals: $\text{rad}(J)^c=\text{rad}(J^c)$
\end{itemize} \tcbline
\begin{proof}~\\
\begin{itemize}
\item Clear since $f^{-1}(J_1)\subseteq f^{-1}(J_2)$ for $J_1\subseteq J_2$. 
\item Since $J_1,J_2\subseteq J_1+J_2$, we have that $J_1^c+J_2^c\subseteq(J_1+J_2)^c$. 
\item Since $J_1\cap J_2\subseteq J_1,J_2$, we have that $(J_1\cap J_2)^c\subseteq J_1^c\cap J_2^c$. Let $x\in J_1^c\cap J_2^c$. Then we have $f(x)\in J_1,J_2$ so that $f(x)\in J_1\cap J_2$. Hence $x\in(J_1\cap J_2)^c$. 
\item Suppose that $x\in J_1^c$ and $y\in J_2^c$. Then $f(xy)=f(x)f(y)\in J_1^cJ_2^c$. Hence $xy\in J_1^cJ_2^c$. 
\item 
\end{itemize}
\end{proof}
\end{prp}

\begin{prp}{}{} Let $R,S$ be commutative rings. Let $f:R\to S$ be a ring homomorphism. Let $I$ be an ideal of $R$ and let $J$ be an ideal of $S$. Then the following are true. 
\begin{itemize}
\item $I\subseteq I^{ec}$
\item $J^{ce}\subseteq J$
\item $I^e=I^{ece}$
\item $J^c=J^{cec}$
\end{itemize} \tcbline
\begin{proof}~\\
\begin{itemize}
\item Let $x\in I$. Then $f(x)\in I^e$. Thus $x\in f^{-1}(I^e)$. 
\item Since $J^{ce}$ is generated by $f(x)$ for all $x\in J^c$, it suffices to check that $f(x)\in J$ for all $x\in J^c$. But $x\in J^c$ implies that $f(x)\in J$ so we are done. 
\item Since $I\subseteq I^{ec}$, we know that $I^e\subseteq I^{ece}$. Also, from the second item we take $J=I^e$ to get $I^{ece}\subseteq I^e$. 
\item From the first item, take $I=J^c$ to get $J^c\subseteq J^{cec}$. Also, since $J^{ce}\subseteq J$, we have that $J^{cec}\subseteq J^c$. 
\end{itemize}
\end{proof}
\end{prp}

\subsection{Minimal Prime Ideals}
\begin{defn}{Minimal Prime Ideals}{} Let $R$ be a commutative ring. Let $I$ be an ideal of $R$. Let $P$ be a prime ideal of $R$. We say that $P$ is a minimal prime ideal over $I$ if for any other prime ideal $Q\supseteq I$ containing $I$, we have $P\subseteq Q$. 
\end{defn}

\begin{prp}{}{} Let $R$ be a commutative ring. Let $I$ be an ideal of $R$. Then a minimal prime ideal over $I$ exists. 
\end{prp}

\subsection{Revisiting the Polynomial Ring}
\begin{prp}{}{} Let $R$ be a commutative ring. Then we have $$N(R[x])=N(R)[x]$$ \tcbline
\begin{proof}
Let $f=\sum_{k=0}^na_kx^k\in N(R)[x]$. Then each $a_k$ is nilpotent in $R$, and there exists $n_k\in\N$ such that $a_k^{n_k}=0$. This also proves that $a_kx^k$ is nilpotent. Since the sum of nilpotents is a nilpotent, we conclude that $f$ is nilpotent. \\~\\

Now suppose that $f\in N(R[x])$. We induct on the degree of $f$. Let $\deg(f)=0$. Then $f$ is nilpotent and $f$ lies in $R$. Thus $f\in N(R)[x]$. Now suppose that the claim is true for $\deg(f)\leq n-1$. Let $\deg(g)=n$ with leading coefficient $b_n$. Since $g$ is nilpotent in $R[x]$, there exists $m\in\N$ such that $g^m=0$. Then in particular, $b_n^m=0$ so that $b_n$ is nilpotent. Then $b_nx^n$ is also nilpotent. Now since $N(R[x])$ is an ideal of $R[x]$, we have that $g-b_nx^n\in N(R[x])$. By inductive hypothesis, $g-b_nx^n\in N(R)[x]$. Since $N(R)$ is an ideal of $R$, we have that $N(R)[x]$ is an ideal of $R[x]$. So $g=(g-b_nx^n)+b_nx^n\in N(R)[x]$. Thus we are done. 
\end{proof}
\end{prp}

Some more important results from Groups and Rings and Rings and Modules include: 
\begin{itemize}
\item If $R$ is an integral domain, then $R[x]$ is an integral domain. 
\item $R$ is a UFD if and only if $R[x]$ is a UFD
\item If $F$ is a field, then $F[x]$ is an Euclidean domain, a PID and a UFD
\item If $F$ is a field, then the ideal generated by $p$ is maximal if and only if $p$ is irreducible. 
\end{itemize}

Regarding ideals of the polynomial ring, the following maybe useful: 
\begin{itemize}
\item $I[x]$ is an ideal of $R$
\item There is an isomorphism $\frac{R[x]}{I[x]}\cong\frac{R}{I}[x]$ given by the map $$\left(f=\sum_{k=0}^na_kx^k+I[x]\right)\mapsto\left(\sum_{k=0}^n(a_k+I)x^k\right)$$
\item If $I$ is a prime ideal of $R$, then $I[x]$ is a prime ideal of $R[x]$. 
\end{itemize}

\pagebreak
\section{Basic Notions of Commutative Rings}
\subsection{Noetherian Commutative Rings}
We recall some facts about Noetherian rings. In the following, let $R$ be a commutative ring, although they are also true if $R$ is non-commutative if we take all modules defined below to be left (right) $R$-modules. 

\begin{itemize}
\item If we have a short exact sequence of $R$-modules: \\~\\
\adjustbox{scale=1.0,center}{\begin{tikzcd}
	0 & {M_1} & {M_2} & {M_3} & 0
	\arrow[from=1-1, to=1-2]
	\arrow["f", from=1-2, to=1-3]
	\arrow["g", from=1-3, to=1-4]
	\arrow[from=1-4, to=1-5]
\end{tikzcd}}\\~\\
Then $M_2$ is Noetherian if and only if $M_1$ and $M_3$ are Noetherian. 
\item If $M$ and $N$ are $R$-modules, then $M\oplus N$ is Noetherian if and only if $M$ and $N$ are Noetherian. 
\item If $M$ is an $R$-module and $N$ is an $R$-submodule of $M$, then $M$ is Noetherian if and only if $N$ and $M/N$ are Noetherian.
\item If $R$ is Noetherian and $I$ is an ideal of $R$, then $R/I$ is Noetherian. 
\item Later when once has seen localization, we can also prove that: If $R$ is Noetherian then $S^{-1}R$ is Noetherian for any multiplicative subset $S$ of $R$. 
\end{itemize}

\begin{thm}{Hilbert's Basis Theorem}{} Let $R$ be a commutative ring. If $R$ is Noetherian, then $$R[x_1,\dots,x_n]$$ is a Noetherian ring. 
\end{thm}

\begin{prp}{}{} Let $R$ be a Noetherian commutative ring. Let $I$ be an ideal of $R$. Then there exists $n\in\N$ such that $$\sqrt{I}^n\subseteq I\subseteq\sqrt{I}$$ \tcbline
\begin{proof}
It is clear that $I\subseteq\sqrt{I}$. Since $R$ is Noetherian, $\sqrt{I}$ is finitely generated by say $x_1,\dots,x_n$. Then $x_i^{n_i}\in I$ for some $n_i\in\N$. Let $m=1+\sum_{i=1}^n(n_i-1)$. Then $\sqrt{I}^m$ is generated by $x_1^{r_1}\cdots x_n^{r_n}$ for $\sum_{i=1}^nr_i=m$. If $r_i<n_i$ for $i$ then $$m=\sum_{i=1}^nr_i\leq\sum_{i=1}^n(n_i-1)<m$$ is a contradiction. Hence there exists some $i$ for which $r_i\geq n_i$. Thus $x_1^{r_1}\cdots x_n^{r_n}\in I$. Thus $\sqrt{I}^m\subseteq I$. 
\end{proof}
\end{prp}

\begin{prp}{}{} Let $R$ be a Noetherian commutative ring. Then $N(R)$ is a nilpotent ideal. \tcbline
\begin{proof}
By the above, there exists $n\in\N$ such that $(N(R))^n=\sqrt{(0)}^n\subseteq (0)\subseteq\sqrt{(0)}$. Hence $(N(R))^n=(0)$ for some $n\in\N$. 
\end{proof}
\end{prp}

\subsection{Artinian Commutative Rings}
We recall some facts about Artinian modules. 

\begin{itemize}
\item If we have a short exact sequence of $R$-modules: \\~\\
\adjustbox{scale=1.0,center}{\begin{tikzcd}
	0 & {M_1} & {M_2} & {M_3} & 0
	\arrow[from=1-1, to=1-2]
	\arrow["f", from=1-2, to=1-3]
	\arrow["g", from=1-3, to=1-4]
	\arrow[from=1-4, to=1-5]
\end{tikzcd}}\\~\\
Then $M_2$ is Artinian if and only if $M_1$ and $M_3$ are Artinian. 
\item If $M$ and $N$ are $R$-modules, then $M\oplus N$ is Artinian if and only if $M$ and $N$ are Artinian. 
\item If $M$ is an $R$-module and $N$ is an $R$-submodule of $M$, then $M$ is Artinian if and only if $N$ and $M/N$ are Artinian.
\end{itemize}

Let $R$ be a (not necessarily commutative ring). If $R$ is left Artinian, then the following are true. 

\begin{itemize}
\item If $I$ is an ideal of $R$, then $R/I$ is Artinian. 
\item Every prime ideal of $R$ is maximal. 
\item $R$ only has finitely many maximal ideals. 
\item $J(R)$ is a nilpotent ideal. 
\item $R$ is Noetherian. 
\end{itemize}

There are also properties of Artinian rings that only commutative rings can realize. 

\begin{prp}{}{} Let $R$ be an integral domain. Then $R$ is Artinian if and only if $R$ is a field. \tcbline
\begin{proof}
It is clear that every field is Artinian. Conversely, let $R$ be Artinian. Consider the following descending chain of ideals in $R$: $$R\supseteq(x)\supseteq(x^2)\supseteq$$ for any $0\neq x\in R$. Since $R$ is Artinian, the chain terminates and $(x^n)=(x^{n+1})$ for some $n\in\N$. Then there exists $y\in R$ such that $x^n=yx^{n+1}$. This means that $x^n(1-yx)=0$. Since $R$ is an integral domain, $R$ has no nilpotents. Hence $x^n$ is non-zero and $1=xy$. Thus $x$ has an inverse so that $R$ is a field. 
\end{proof}
\end{prp}

\begin{prp}{}{} Let $R$ be a commutative ring. Let $R$ be Artinian. Then every prime ideal in $R$ is maximal. \tcbline
\begin{proof}
Let $P$ be a prime ideal. Since quotients of Artinian rings are Artinian, $R/P$ is Artinian. Since $R/P$ is also an integral domain, we conclude by the above that $R/P$ is a field. Hence $P$ is maximal. 
\end{proof}
\end{prp}

\begin{prp}{}{} Let $R$ be a commutative ring. If $R$ is Artinian, then $$N(R)=J(R)$$ \tcbline
\begin{proof}
Since every prime ideal in $R$ is maximal, we have that $$N(R)=\bigcap_{P\text{ a prime ideal}}P=\bigcap_{P\text{ a maximal ideal}}P=J(R)$$ and so we conclude. 
\end{proof}
\end{prp}

\begin{prp}{}{} Let $R$ be a commutative ring. If $R$ is Artinian, then $R$ has finitely many maximal ideals. \tcbline
\begin{proof}
Consider the collection $$\{m_1\cap\cdots\cap m_k\;|\;m_1,\dots,m_k\text{ are maximal ideals of }R\}$$ of $R$-submodules of $R$. Since $R$ is Artinian, every collection of $R$-submodules of $R$ has a minimal element. Hence this collection also has a minimal element, say $m_1\cap\cdots\cap m_k$. Let $m$ be another maximal ideal of $R$. Then $$m\cap m_1\cap\cdots\cap m_k\subseteq m_1\cap\cdots\cap m_k$$ Since $m_1\cap\cdots\cap m_k$ is minimal, they are equal. By prp1.1.1, we conclude that $m\supseteq m_i$ for some $i$. Since they are maximal, we have $m=m_i$. Hence $m_1,\dots,m_k$ gives the full list of distinct maximal ideals of $R$. 
\end{proof}
\end{prp}

\subsection{Local Rings}
\begin{defn}{Local Rings}{} Let $R$ be a commutative ring. We say that $R$ is a local ring if it has a unique maximal ideal $m$. In this case, we say that $R/m$ is the residue field of $R$. 
\end{defn}

\begin{eg}{}{} Consider the following commutative rings. 
\begin{itemize}
\item $\Z/6\Z$ is not a local ring. 
\item $\Z/8\Z$ is a local ring. 
\item $\Z/24\Z$ is not a local ring. 
\item $\R[x]$ is not a local ring. 
\end{itemize} \tcbline
\begin{proof}~\\
\begin{itemize}
\item The only ideals of $\Z/6\Z$ are $(2+6\Z)$ and $(3+6\Z)$. They do not contain each other and so they are both maximal. 
\item The only ideals of $\Z/8\Z$ are $(2+8\Z)$ and $(4+8\Z)$. But $(2+8\Z)\supseteq(4+8\Z)$. Hence $\Z/8\Z$ has a unique maximal ideal. 
\item A similar proof as above ensues. 
\item Any irreducible polynomial $f\in\R[x]$ is such that $(f)$ is a maximal ideal. Indeed the evaluation homomorphism gives an isomorphism $\frac{\R[x]}{(f)}\cong\R$. 
\end{itemize}
\end{proof}
\end{eg}

\begin{prp}{}{} Let $R$ be a ring and $I$ an ideal of $R$. Then $I$ is the unique maximal ideal of $R$ if and only if $I$ is the set containing all non-units of $R$. \tcbline
\begin{proof}
Let $I$ be the unique maximal ideal of $R$. Clearly $I$ does not contain any unit else $I=R$. Now suppose that $r$ is a non-unit. Suppose that $r\notin I$. Define $J=\{sr|s\in R\}$ Clearly $J$ is an ideal. It must be contained in some maximal ideal. Since $I$ is the unique maximal ideal, $J\subseteq I$. But this means that $r\in I$, a contradiction. Thus every non-unit is in $I$. \\~\\
Suppose that $I$ contains all non-units of $R$. Let $r\notin I$. Then there exists $s\notin I$ such that $rs=1$. Then $(r+I)(s+I)=1+I$ in $R/I$. This means that every element of $R/I$ has a multiplicative inverse which means that $R/I$ is a field and thus $I$ is a maximal ideal. Now let $J\neq I$ be another maximal ideal. Then $J$ contains some unit $r$. This implies that $J=R$ and thus $I$ is the unique maximal ideal. 
\end{proof}
\end{prp}

\begin{eg}{}{} Let $k$ be a field. Then the ring of power series $k[[x]]$ is a local ring. \tcbline
\begin{proof}
Let $M$ be the set of all non-units of $k[[x]]$. I first show that $f\in M$ if and only if the constant term of $f$ is non-zero. Let $g$ be a power series. Then the $n$th coefficient of $f\cdot g$ is given by $$c_n=\sum_{k=0}^na_kb_{n-k}$$ If the constant term of $f$ is $0$, then $c_0=0$ and so $f\cdot g\neq 1$. Now if the constant term of $f$ is $a_0\neq 0$, then set $b_0=\frac{1}{a_0}$. Now we can use the formula $0=c_n$ to deduce $$b_n=-\frac{\sum_{k=1}^na_kb_{n-k}}{a_0}$$ This is such that $a_n\cdot b_n=0$. Define $g=\sum_{k=0}^\infty b_kx^k$. Then $f\cdot g=1$. Thus $f$ is a unit. \\~\\

By the above proposition, we conclude that $M$ is the unique maximal ideal of $k[[x]]$. 
\end{proof}
\end{eg}

\begin{prp}{}{} Let $R$ be a commutative ring. Then the following are equivalent. 
\begin{itemize}
\item $R$ has exactly one prime ideal. (It is given by $N(R)$).
\item Every element of $R$ is either a unit or nilpotent. 
\item $N(R)$ is a maximal ideal. 
\end{itemize}
Under these equivalent assumptions, $(R,N(R))$ is a local ring. \tcbline
\begin{proof}~\\
\begin{itemize}
\item $(1)\implies(2)$: We know that $N(R)$ is a prime ideal, hence it is the unique prime ideal and unique maximal ideal. Thus $R$ is a local ring. By the above, elements of $R\setminus N(R)$ are units and element of $N(R)$ are nilpotent. 
\item $(2)\implies(3)$: It is clear that every nilpotent is a non-unit. By assumption, non-units of $R$ are nilpotents. Hence $N(R)$ is the set of all non-units. Since $N(R)$ is an ideal, by the above we conclude that $(R,N(R))$ is a local ring. In particular, $N(R)$ is the unique maximal ideal of $R$. 
\item $(3)\implies(1)$: Suppose that $N(R)$ is a maximal ideal. Let $P\neq R$ be a prime ideal of $R$. Since $N(R)$ is the intersection of all prime ideals, we have $N(R)\subseteq P$. By the correspondence theorem, $P$ corresponds to a prime ideal of $R/N(R)$. But $R/N(R)$ is a field, and since $P\neq R$ we must have that $P=N(R)$. Thus $N(R)$ is the unique prime ideal of $R$. 
\end{itemize}
\end{proof}
\end{prp}

\begin{prp}{}{} Let $R$ be a Noetherian commutative ring. Then the following are equivalent. 
\begin{itemize}
\item $R$ is an Artinian local ring. 
\item $R$ has a nilpotent maximal ideal. 
\item $R$ has a unique proper radical ideal. 
\item $R$ has a unique prime ideal. 
\item $N(R)$ is a maximal ideal of $R$. 
\end{itemize} \tcbline
\begin{proof}~\\
\begin{itemize}
\item $(1)\implies(2)$: Let $R$ be Artinian and local. By 2.1.4 we have $N(R)=J(R)=m$ since $J(R)$ is the intersection of all maximal ideals. Since $R$ is Noetherian, by 2.1.3 $N(R)=m$ is nilpotent. 
\end{itemize}
\end{proof}
\end{prp}

Since every Artinian ring is Noetherian, the above proposition implies the following. 

\begin{crl}{}{} Let $R$ be an Artinian commutative ring. Then the following are true. 
\begin{itemize}
\item $R$ is local. 
\item $N(R)$ is the unique maximal ideal of $R$. 
\item $N(R)$ is the unique prime ideal of $R$. 
\item $N(R)$ is the unique radical ideal of $R$. 
\item $N(R)$ is a nilpotent ideal. 
\end{itemize}
\end{crl}

We will discuss more of local rings in the topic of localizations. 

\pagebreak
\section{Modules over a Commutative Ring}
Recall from Rings and Modules that a module consists of an abelian group $M$ and a ring $R$ such that there is a binary operation $\cdot:R\times M\to M$ that mimic the notion of a group action: 
\begin{itemize}
\item For $r,s\in R$, $s\cdot(r\cdot m)=(sr)\cdot m$ for all $m\in M$. 
\item For $1_R\in R$ the multiplicative identity, $1_R\cdot m=m$ for all $m\in M$. 
\end{itemize}

When $R$ is a commutative ring, the first axiom is relaxed so that the resulting element of $M$ makes no difference whether you apply $r$ first or $s$ first. This makes module act even more similarly than fields (although one still need the notion of a basis, which appears in free modules). Therefore the first section concerns transferring techniques in linear algebra such as the Cayley Hamilton theorem to module over a ring that mimic the notion of vector spaces. 

\subsection{Cayley-Hamilton Theorem}
\begin{defn}{Characteristic Polynomial}{} Let $R$ be a commutative ring. Let $A\in M_{n\times n}(R)$ be a matrix. Define the characteristic polynomial of $A$ to be the polynomial $$c_A(x)=\det(A-xI)$$
\end{defn}

\begin{thm}{Cayley-Hamilton Theorem for Rings}{} Let $R$ be a commutative ring. Let $A\in M_{n\times n}(R)$ be a matrix. Then $c_A(A)=0$. 
\end{thm}

\begin{thm}{Cayley-Hamiliton Theorem for Modules}{} Let $R$ be a commutative ring. Let $M$ be a finitely generated $R$-module. Let $I$ be an ideal of $R$. Let $\varphi\in\text{End}_R(M)$. If $\varphi(M)\subseteq IM$, then there exists $a_1,\dots,a_{n-1}\in I$ such that $$\varphi^n+a_1\varphi^{n-1}+\dots+a_{n-1}\varphi+\text{id}_M=0:M\to M$$ \tcbline
\begin{proof}
Suppose that $M$ is generated by $x_1,\dots,x_n$. There exists a surjective map $\rho:R^n\to M$ given by $(r_1,\dots,r_n)\mapsto\sum_{k=1}^nr_kx_n$. Since $\varphi(M)\subseteq IM$, we havt that $$\varphi(x_k)=\sum_{i=1}^nr_{ki}x_i$$ for some $r_{ki}\in I$. Write $A$ to be the matrix $A=(a_{ki})$. We now have a commutative diagram: \\~\\
In other words, we have the diagram: \\
\adjustbox{scale=1.0,center}{\begin{tikzcd}
	{R^n} & M \\
	{R^n} & M
	\arrow["\rho", two heads, from=1-1, to=1-2]
	\arrow["A"', from=1-1, to=2-1]
	\arrow["\varphi", from=1-2, to=2-2]
	\arrow["\rho"', two heads, from=2-1, to=2-2]
\end{tikzcd}}\\~\\
By Cayley-Hamilton theorem, we have that $c_A(A)=0$ is the zero function. For all $x\in R^n$, we have that 
\begin{align*}
c_A(A)(x)&=0\\
c_A(Ax)&=0\\
\rho(c_A(Ax))&=\rho(0)\\
c_A(\rho(Ax))&=0\tag{$\rho$ is $R$-linear}\\
c_A(\varphi(\rho(x)))&=0\tag{Diagram is commutative}
\end{align*}
Since $\rho$ is surjective, we conclude that for any $m\in M$, the above calculation gives $c_A(\varphi(m))=0$ so that $c_A(\varphi)$ is the zero map. 
\end{proof}
\end{thm}

\begin{prp}{}{} Let $R$ be a commutative ring. Let $M$ be a finitely generated $R$-module. Let $\phi:M\to M$ be a surjective $R$-module homomorphism. Then $\phi$ is an isomorphism. \tcbline
\begin{proof}
Consider $M$ as an $R[\phi]$-module via the action $\phi\cdot m=\phi(m)$. Notice that $(\phi)M=M$ since $\phi$ is surjective. By the Cayley-Hamilton theorem, there exists $\alpha_1,\dots,\alpha_{n-1}\in R$ such that $$\text{id}^n+\alpha_1\phi\text{id}^{n-1}+\dots+\alpha_{n-1}\phi\text{id}+\text{id}=0:M\to M$$ This simplifies to the equation $$(\alpha_1+\dots+\alpha_{n-1})\phi(m)+m=0$$ for all $m\in M$. \\~\\

We want to show that $\phi$ is injective. Suppose that $\phi(m)=0$ for some $m\in M$. From the above equation, we see that $m=0$. Hence $\phi$ is an isomorphism. 
\end{proof}
\end{prp}

\subsection{Nakayama's Lemma}
\begin{lmm}{Nakayama's Lemma I}{} Let $R$ be a commutative ring. Let $M$ be a finitely generated $R$-module. Let $I$ be an ideal of $R$. If $IM=M$, then there exists $r\in R$ such that $rM=0$ and $r-1\in I$. \tcbline
\begin{proof}
Choose $\varphi=\text{id}_M$. Then $\varphi$ is surjective so that $M=\varphi(M)\subseteq IM$. By crl 4.1.3, there exists $r_1,\dots,r_n\in I$ such that $(1+r_1+\dots+r_n)M=0$. By choosing $r=1+r_1+\dots+r_n$, we see that $rM=0$ and $r-1\in I$ so that we conclude. 
\end{proof}
\end{lmm}

\begin{lmm}{Nakayama's Lemma II}{} Let $R$ be a commutative ring. Let $M$ be a finitely generated $R$-module. Let $I$ be an ideal of $R$ such that $I\subseteq J(R)$ and $IM=M$. Then $M=0$. \tcbline
\begin{proof}
By Nakayama's lemma I, there exists $r\in R$ such that $rM=0$ and $r-1\in I\subseteq J(R)$. By 2.3.8, we have that $1-(r-1)(-1)=r\in R^\times$. This means that $r$ is invertible. Hence $rM=0$ implies $M=r^{-1}rM=0$. 
\end{proof}
\end{lmm}

\begin{crl}{}{} Let $R$ be a commutative ring. Let $M$ be a finitely generated $R$-module. Let $I$ be an ideal of $R$ such that $I\subseteq J(R)$. Let $N$ be an $R$-submodule of $M$. If $$M=IM+N$$ then $M=N$. \tcbline
\begin{proof}
Since quotients of finitely generated modules are finitely generated, we know that $M/N$ is finitely generated. Define the map $$\phi:IM+N\to I\frac{M}{N}$$ by $\phi(im+n)=i(m+N)$. This map is clearly surjective. Now I claim that $\ker(\phi)=N$. For any $im+n\in\ker(\phi)$, we see that $i(m+N)=N$ means that $im\in N$. Hence $im+n\in N$. On the other hand, if $im+n\in N$ then $im\in N$. But this means that $im+N=N$. Hence $im+n\in\ker(\phi)$. By the first isomorphism theorem for modules, we conclude that $$\frac{M}{N}=\frac{IM+N}{N}\cong I\frac{M}{N}$$ We can now apply Nakayama's lemma II to conclude that $M/N=0$ so that $M=N$. 
\end{proof}
\end{crl}

\begin{crl}{}{} Let $R$ be a commutative ring. Let $m$ be a maximal ideal of $R$. Let $M$ be a finitely generated $R$-module. Then the following are true. 
\begin{itemize}
\item $M/mM$ is a finite dimensional vector space over $R/m$. 
\item $a_1,\dots,a_n\in M$ generates $M$ as an $R$-module if and only if $a_1+mM,\dots,a_n+mM$ generates $M/mM$ as a $R/m$ vector space. 
\end{itemize} \tcbline
\begin{proof}
For the first part, we already know that $M/mM$ is an $R$-module. We notice that for any $k\in m$ and $t+mM\in M/mM$ we have that $k(t+mM)=kt+kmM$. But $kt\in m$ means that $kt+kmM=mM$. Hence $M/mM$ is well defined as an $R/m$-module. Now suppose that $M$ is finitely generated by the elements $a_1,\dots,a_n$. Let $x+mM\in M/mM$. Then there exists $r_k\in R$ such that $x=r_1a_1+\dots+r_na_n$. But this means that $$x+mM=r_1(a_1+mM)+\dots+r_n(a_n+mM)$$ This means that $M/mM$ is generated by $a_1+mM,\dots,a_n+mM$. We conclude that $M/mM$ is finite dimensional. \\~\\

Suppose that $a_1,\dots,a_n\in M$ generates $M$ as an $R$-module. By the same argument as above, we can see that $a_1+mM,\dots,a_n+mM$ is a set of generators for $M/mM$. For the other direction, suppose that $a_1+mM,\dots,a_n+mM$ generates $M/mM$ as an $R/m$-vector space. Define $N=Ra_1+\dots+Ra_n\leq M$. Set $I=J(R)=m$. We want to show that $M=IM+N$. It is clear that $IM+N\leq M$. If $x\in M$, then there exists $r_k\in R$ such that $x+mM=r_1(a_1+mM)+\dots+r_n(a_n+M)$. In particular, this means that $$x-\sum_{k=1}^nr_ka_k\in mM$$ Hence $x\in IM+N$. We can now apply the above corollary to deduce that $M=N=Ra_1+\dots+Ra_n$ so that $M$ is generated by $a_1,\dots,a_n$. And so we are done. 
\end{proof}
\end{crl}

\begin{prp}{}{} Let $(R,m)$ be a local ring. Let $M$ be a finitely generated $R$-module. Then $a_1,\dots,a_n\in M$ is a minimal set of generators of $M$ as an $R$-module if and only if $a_1+mM,\dots,a_n+mM$ is a basis for $M/mM$ as a $R/m$ vector space. \tcbline
\begin{proof}
Suppose that $a_1,\dots,a_n$ generate $M$. The above shows that $a_1+mM,\dots,a_n+mM$ spans $M/mM$. So suppose for a contradiction that $a_1,\dots,a_n$ is a minimal generating set but $a_1+mM,\dots,a_n+mM$ is not a basis for $m/m^2$. This means that after relabelling, $a_1+mM,\dots,a_{n-1}+mM$ spans $M/mM$. By the above, this means that $a_1,\dots,a_{n-1}$ generate $M$. This is a contradiction of the minimality of the generating set $a_1,\dots,a_n$. Hence $a_1+mM,\dots,a_n+mM$ is a basis for $m/m^2$. \\~\\

Now suppose that $a_1+mM,\dots,a_n+mM$ is a basis for $m/m^2$. We have seen above that $a_1,\dots,a_n$ generate $M$. If this is not minimal, then there is some smaller generating set $b_1,\dots,b_k$ that still generates $M$ where $k<n$. By the above, $b_1+mM,\dots,b_k+mM$ spans $m/m^2$ hence $n=\dim_{R/m}(m/m^2)\leq k$. This is a contradiction since $k<n$. Hence we are done. 
\end{proof}
\end{prp}

\subsection{Change of Rings}
\begin{defn}{Extension of Scalars}{} Let $R,S$ be commutative rings. Let $\varphi:R\to S$ be a ring homomorphism. Let $M$ be an $R$-module. Define the extension of $M$ to the ring $S$ to be the $S$-module $$S\otimes_R M$$
\end{defn}

\begin{defn}{Restriction of Scalars}{} Let $R,S$ be commutative rings. Let $\varphi:R\to S$ be a ring homomorphism. Let $M$ be an $S$-module. Define the restriction of $M$ to the ring $R$ to be the $R$-module $M$ equipped with the action $$r\cdot_R m=\varphi(r)\cdot_S m$$ for all $r\in R$. 
\end{defn}

\begin{thm}{}{} Let $R,S$ be commutative rings. Let $\varphi:R\to S$ be a ring homomorphism. Then there is an isomorphism $$\Hom_S(S\otimes_R M,N)\cong\Hom_R(M,N)$$ for any $R$-module $M$ and $S$-module $N$ given as follows. 
\begin{itemize}
\item For $f\in\Hom_S(S\otimes_R M,N)$, define the map $f^+\in\Hom_R(M,N)$ by $$f^+(m)=f(1\otimes m)$$
\item For $g\in\Hom_R(M,N)$, define the map $g^-\in\Hom_S(S\otimes_RM,N)$ by $$g^-(s\otimes m)=s\cdot g(m)$$
\end{itemize}
\end{thm}

\subsection{Properties of the Hom Set}
Let $R$ be a ring. Let $M,N$ be $R$-modules. Recall that in Rings and Modules that $\Hom_R(M,N)$ is a $Z(R)$-modules. When $R$ is commutative, $Z(R)=R$ so that the Hom set becomes an $R$-module. 

\begin{prp}{}{} Let $R$ be a commutative ring. Let $M,N$ be $R$-modules. Then $$\Hom_R(M,N)$$ is an $R$-module with the following binary operations. 
\begin{itemize}
\item For $\phi,\varphi:M\to N$ two $R$-module homomorphisms, define $\phi+\varphi:M\to N$ by $(\phi+\varphi)(m)=\phi(m)+\varphi(m)$ for all $m\in M$
\item For $\phi:M\to N$ an $R$-module homomorphism and $r R$, define $r\phi:M\to N$ by $(r\phi)(m)=r\cdot\phi(m)$ for all $m\in M$. 
\end{itemize} \tcbline
\begin{proof}
We first show that the addition operation gives the structure of a group. 
\begin{itemize}
\item Since $M$ is associative as an additive group, associativity follows
\item Clearly the zero map $0\in\Hom_R(M,N)$ acts as the additive inverse since for any $\phi\in\Hom_R(M,N)$, we have that $\phi(m)+0=0+\phi(m)=\phi(m)$ since $0$ is the additive identity for $M$
\item For every $\phi\in\Hom_R(M,N)$, the map taking $m$ to $-\phi(m)$ also lies in $\Hom_R(M,N)$. Since $-\phi(m)$ is the inverse of $\phi(m)$ in $M$ for each $m\in M$, we have that $-\phi$ is the inverse of $\phi$
\end{itemize}
We now show that 
\begin{itemize}
\item Let $r,s\in R$, we have that $((sr)\phi)(m)=(sr)\cdot\phi(m)=s\cdot(r\cdot\phi(m))=s(r(\phi))(m)$ and hence we showed associativity. 
\item It is clear that $1_R\in R$ acts as the identity of the operation. 
\end{itemize}
Thus we are done. 
\end{proof}
\end{prp}

\begin{prp}{}{} Let $R$ be a ring. Let $I$ be an indexing set. Let $M_i,N$ be $R$-modules for $i\in I$. Then the following are true. 
\begin{itemize}
\item There is an isomorphism $$\Hom\left(\bigoplus_{i\in I}M_i,N\right)\cong\bigoplus_{i\in I}\Hom(M_i,N)$$
\item There is an isomorphism $$\Hom\left(\prod_{i\in I}M_i,N\right)\cong\prod_{i\in I}\Hom(M_i,N)$$
\end{itemize} \tcbline
\end{prp}

\begin{defn}{Induced Map of Hom}{} Let $R$ be a commutative ring. Let $M_1,M_2,N$ be $R$-modules. Let $f:M_1\to M_2$ be an $R$-module homomorphism. Define the induced map $$f^\ast:\Hom_R(M_2,N)\to\Hom(M_1,N)$$ by the formula $\varphi\mapsto\varphi\circ f$
\end{defn}

\begin{lmm}{}{} Let $R$ be a commutative ring. Let $M_1,M_2,N$ be $R$-modules. Let $f:M_1\to M_2$ be an $R$-module homomorphism. Then the induced map $$f^\ast:\Hom(M_2,N)\to\Hom(M_1,N)$$ is an $R$-module homomorphism. 
\end{lmm}

\subsection{Failure of Exactness of Hom and Tensoring}
\begin{prp}{}{} Let $R$ be a commutative ring. Let the following be an exact sequence of $R$-modules. \\~\\
\adjustbox{scale=1.0,center}{\begin{tikzcd}
	0 & M_1 & M_2 & M_3 & 0
	\arrow[from=1-1, to=1-2]
	\arrow["f", from=1-2, to=1-3]
	\arrow["g", from=1-3, to=1-4]
	\arrow[from=1-4, to=1-5]
\end{tikzcd}}\\~\\ Let $N$ be an $R$-module. Then the following two sequences \\~\\
\adjustbox{scale=1,center}{\begin{tikzcd}
	0 & {\Hom_R(M_3,N)} & {\Hom_R(M_2,N)} & {\Hom_R(M_1,N)} & {}\\
	{} & {\Hom_R(N,M_1)} & {\Hom_R(N,M_2)} & {\Hom_R(N,M_3)} & 0
	\arrow[from=1-1, to=1-2]
	\arrow[from=1-2, to=1-3]
	\arrow[from=1-3, to=1-4]
	\arrow[from=2-2, to=2-3]
	\arrow[from=2-3, to=2-4]
	\arrow[from=2-4, to=2-5]
\end{tikzcd}} \\~\\ 
are exact. \tcbline
\begin{proof}~\\
\begin{itemize}
\item We first show that $g^\ast$ is injective. Let $\phi,\rho\in\Hom(C,G)$ such that $g^\ast(\phi)=g^\ast(\rho)$. This means that $\phi\circ g=\rho\circ g$. Let $c\in C$. Since $g$ is surjective, there exists $b\in B$ such that $g(b)=c$. Then $$\phi(c)=\phi(g(b))=\rho(g(b))=\rho(c)$$ Hence $\phi=\rho$. \\~\\

Now we show that $\im(g^\ast)\subseteq\ker(f^\ast)$. Let $g^\ast(\phi)\in\Hom(B,G)$ for $\phi\in\Hom(C,G)$. We want to show that $f^\ast(g^\ast(\phi))=0$. But we have that $$(\phi\circ g\circ f)(a)=\phi(g(f(a))=\phi(0)=0$$ since $\im(f)=\ker(g)$. Thus we conclude. \\~\\

Finally we show that $\ker(f^\ast)\subseteq\im(g^\ast)$. Let $f^\ast(\phi)=0$ for $\phi\in\Hom(B,G)$. This means that $\phi\circ f=0$ or in other words, $\im(f)\subseteq\ker(\phi)$.
Since $\phi(k)=0$ for all $k\in\im(f)$, $\phi$ descends to a map $\overline{\phi}:\frac{B}{\im(f)}\to G$. But $\im(f)=\ker(g)$ hence this is equivalent to a map $\overline{\phi}:\frac{B}{\ker(g)}\to G$. But by the first isomorphism theorem and the fact that $g$ is surjective, we conclude that $\overline{g}:\frac{B}{\ker(g)}\overset{g}{\cong} C$, where $b+\ker(g)\mapsto g(b)$. Thus we have constructed a map $\overline{\phi}\circ\overline{g}^{-1}:C\to G$ given by $g(b)\mapsto b+\ker(g)\mapsto\phi(b)$. But now $g^\ast(\overline{\phi}\circ\overline{g}^{-1})$ is the map defined by $$b\mapsto g(b)\mapsto b+\ker(g)\mapsto\phi(b)$$ and so this map is exactly $\phi$. Thus $\phi\in\im(g^\ast)$. 
\end{itemize}
\end{proof}
\end{prp}

\begin{prp}{}{} Let $R$ be a commutative ring. Let the following be an exact sequence of $R$-modules. \\~\\
\adjustbox{scale=1.0,center}{\begin{tikzcd}
	0 & M_1 & M_2 & M_3 & 0
	\arrow[from=1-1, to=1-2]
	\arrow["f", from=1-2, to=1-3]
	\arrow["g", from=1-3, to=1-4]
	\arrow[from=1-4, to=1-5]
\end{tikzcd}}\\~\\ Let $N$ be an $R$-module. Then the following sequence \\~\\
\adjustbox{scale=1,center}{\begin{tikzcd}
	{M_1\otimes N} & {M_2\otimes N} & {M_3\otimes N} & 0
	\arrow["{f\otimes\text{id}_N}", from=1-1, to=1-2]
	\arrow["{g\otimes\text{id}_N}", from=1-2, to=1-3]
	\arrow[from=1-3, to=1-4]
\end{tikzcd}} \\~\\
is exact. 
\end{prp}

However, one can observe that we did not imply that $M_1\otimes N\to M_2\otimes N$ is injective. Indeed, this is because tensoring does not preserve injections. 

\pagebreak
\section{Algebra Over a Commutative Ring}
\subsection{Commutative Algebras}
\begin{defn}{Commutative Algebras}{} Let $R$ be a commutative ring. A commutative $R$-algebra is an $R$-algebra $A$ that is commutative. 
\end{defn}

\begin{prp}{}{} Let $R$ be a commutative ring. Then the following are equivalent characterizations of a commutative $R$-algebra. 
\begin{itemize}
\item $A$ is a commutative $R$-algebra
\item $A$ is a commutative ring together with a ring homomorphism $f:R\to A$
\end{itemize}\tcbline
\begin{proof}
Suppose that $A$ is an $R$-algebra. Then define a map $f:R\to A$ by $f(r)=r\cdot 1$ where $r\cdot 1$ is the module operation on $A$. Then clearly this is a ring homomorphism. \\~\\
Suppose that $A$ is a commutative ring together with a ring homomorphism $f:R\to A$. Define an action $\cdot:R\times A\to A$ by $r\cdot a=f(r)a$. Then this action clearly allows $A$ to be an $R$-module. 
\end{proof}
\end{prp}

Under the correspondence of associative algebra, the above proposition gives a another correspondence between the first one. $$\left\{(A,R)\;\bigg{|}\;\substack{A\text{ is a commutative }\\R\text{-algebra}}\right\}\;\;\overset{1:1}{\longleftrightarrow}\;\;\left\{\phi:R\to A\;\bigg{|}\;\substack{\phi\text{ is a ring homomorphism}\\\text{ such that }f(R)\subseteq Z(A)=A}\right\}\;\;$$ In particular, the construction above are inverses of each other so that it gives the one-to-one correspondence. 

\subsection{Free Commutative Algebras}
Let $R$ be a commutative ring. Let $X$ be a set. Recall that we defined $R\langle X\rangle$ to be the free (non-commutative) $R$-algebra over $X$. Explicitly, if $W=\{x_1\cdots x_n\;|\;x_1,\dots,x_n\in X\}$ is the set of words on $X$, then $$R\langle X\rangle=\bigoplus_{w\in W}R\cdot w$$ together with multiplication defined by $(x_1\cdots x_n)\cdot(y_1\cdots y_n)=x_1\cdots x_n\cdot y_1\cdots y_m$. 

\begin{defn}{Free Commutative Algebra over a Ring}{} Let $R$ be a commutative ring. Let $X$ be a set. Define the free commutative $R$-algebra over $X$ to be the quotient $$\text{Free}_R(X)=\frac{R\langle X\rangle}{\langle x_ix_j-x_jx_i\;|\;x_i,x_j\in X\rangle}$$
\end{defn}

\begin{prp}{Universal Property of Free Commutative Algebras}{} Let $R$ be a commutative ring. Let $X$ be a set. The free commutative algebra $\text{Free}_R(X)$ satisfies the following universal property. If $A$ is a commutative $R$-algebra, then for every $f:X\to A$ a map of sets, there exists a unique homomorphism of algebras $\varphi:\text{Free}_R(X)\to A$ such that $\varphi(x_i)=f(x_i)$ for each $x_i\in X$. In other words, the following diagram commutes: \\~\\
\adjustbox{scale=1.0,center}{\begin{tikzcd}
	X & {\text{Free}_R(X)} \\
	& A
	\arrow["\iota", hook, from=1-1, to=1-2]
	\arrow["f"', from=1-1, to=2-2]
	\arrow["{\exists!\varphi}", dashed, from=1-2, to=2-2]
\end{tikzcd}}\\~\\
where $\iota:X\to\text{Free}_R(X)$ is the inclusion. 
\end{prp}

\begin{prp}{}{} Let $R$ be a commutative ring. Let $X$ be a set. Then there is an $R$-algebra homomorphism $$\text{Free}_R(X)\cong R[X]$$ with the polynomial ring over $X$. 
\end{prp}

\subsection{Finiteness Properties of Algebras}
\begin{defn}{Finitely Generated Algebras}{} Let $R$ be a commutative ring. Let $A$ be a commutative $R$-algebra. We say that $A$ is finitely generated if there exists $a_1,\dots,a_n\in A$ such that every element $a\in A$ can be written as a polynomial in $a_1,\dots,a_n$. This means that $$a=\sum_{i_1,\dots,i_n}r_{i_1,\dots,i_n}a_1^{i_1}\cdots a_n^{i_n}$$
\end{defn}

Finitely generated algebras are also called algebra of finite type. 

\begin{thm}{}{} Let $A$ be a commutative algebra over a ring $R$. Then the following are equivalent. 
\begin{itemize}
\item $A$ is a finitely generated algebra over $R$
\item There exists elements $a_1,\dots,a_n\in A$ such that the evaluation homomorphism $$\phi:R[x_1,\dots,x_n]\to A$$ given by $\phi(f)=f(a_1,\dots,a_n)$ is a surjection
\item There is an isomorphism $$A\cong\frac{R[x_1,\dots,x_n]}{I}$$ for some ideal $I$
\end{itemize}
\end{thm}

\begin{defn}{Finitely Presented Algebra}{} Let $R$ be a ring. Let $A=R[x_1,\dots,x_n]/I$ be a finitely generated algebra over $R$ for some ideal $I$. We say that $A$ is finitely presented if $I$ is finitely generated. 
\end{defn}

\begin{lmm}{}{} Let $R$ be a ring, considered as an algebra over $\Z$. If $R$ is finitely generated over $\Z$, then $R$ is finitely presented. \tcbline
\begin{proof}
Trivial since $\Z$ is a principal ideal domain. 
\end{proof}
\end{lmm}

\begin{defn}{Finite Algebras}{} Let $R$ be a commutative ring. Let $A$ be an $R$-algebra. We say that $A$ is finite if $A$ is finitely generated as an $R$-module. 
\end{defn}

\begin{eg}{}{} Let $R$ be a commutative ring. Then $R[x]$ is a finitely generated algebra over $R$ but is not a finite $R$-algebra. 
\end{eg}

\subsection{Zariski's Lemma}
\begin{lmm}{}{} Let $F$ be a field. Let $f\in F[x]$. Then the localization $F[x]_f$ is not a field. 
\end{lmm}

\begin{thm}{Zariski's Lemma}{} Let $F$ be a field. Let $K$ be a field that is also a finitely generated algebra over $F$. Then $K$ is a finite algebra. In particular, $K$ is a finite dimensional vector space over $F$. 
\end{thm}

\begin{crl}{}{} Let $F$ be an algebraically closed field. Let $K$ be a field that is also a finitely generated algebra over $F$. Then the inclusion homomorphism $F\hookrightarrow K$ is an $F$-algebra isomorphism. 
\end{crl}

\begin{crl}{}{} Let $F$ be an algebraically closed field. Then every maximal ideal of $F[x_1,\dots,x_n]$ is of the form $(x_1-a_1,\dots,x_n-a_n)$ for some $a_1,\dots,a_n\in F$. 
\end{crl}

\pagebreak
\section{Localization}
\subsection{Localization of a Ring}
\begin{defn}{Multiplicative Set}{} Let $R$ be a commutative ring. $S\subseteq R$ is a multiplicative set if $1\in S$ and $S$ is closed under multiplication: $x,y\in S$ implies $xy\in S$
\end{defn}

\begin{defn}{Localization of a Ring}{} Let $R$ be a commutative ring and $S\subseteq R$ be a multiplicative set. Define the ring of fractions of $R$ with respect to $S$ by $$S^{-1}R=\left\{\frac{r}{s}\;\bigg{|}\;r\in R,s\in S\right\}/\sim$$ where we say that $r/s\sim r'/s'$ if there exists $t\in S$ such that $t(rs'-r's)=0$. 
\end{defn}

\begin{lmm}{}{} Let $R$ be a commutative ring. Let $f\in R$ be non-zero. Then the set $\{f^n\;|\;n\in\N\}$ is a multiplicative set. 
\end{lmm}

\begin{defn}{Localization at an Element}{} Let $R$ be a commutative ring. Let $f\in R$ be non-zero. Define the localization of $R$ at $f$ to be the ring $$R_f=\{f^n\;|\;n\in\N\}^{-1}R$$ It is also denoted as $R[1/f]$. 
\end{defn}

\begin{prp}{}{} Let $S^{-1}R$ be a ring of fractions. 
\begin{itemize}
\item $\sim$ as defined in the ring of fractions is an equivalence relation
\item $(S^{-1}R,+,\times)$ is a ring
\item The map $k:R\to S^{-1}R$ defined by $r\mapsto r/1$ is a ring homomorphism, called the localization map. 
\end{itemize}\tcbline
\begin{proof}~\\
\begin{itemize}
\item Trivial
\item Define addition by $\frac{r}{s}+\frac{r'}{s'}=\frac{rs'+r's}{ss'}$ and multiplication by $\frac{r}{s}\cdot\frac{r'}{s'}=\frac{rr'}{ss'}$. Clearly addition is abelian, and has identity $\frac{0}{1}$ and inverse $\frac{-r}{s}$ for any $\frac{r}{s}\in S^{-1}R$. Multiplication also has identity $\frac{1}{1}$. 
\end{itemize}
\end{proof}
\end{prp}

\begin{prp}{Universal Property}{} Let $R$ be a commutative ring. Let $S$ be a multiplicative set. Then $S^{-1}R$ and the localization map $k:R\to S^{-1}R$ satisfies the following universal property. \\~\\

For any commutative ring $B$ and ring homomorphism $\phi:R\to B$ such that $\phi(s)\in B^\times$ for all $s\in S$, there exists a unique ring homomorphism $\phi:S^{-1}R\to B$ such that the following diagram commutes: \\~\\
\adjustbox{scale=1.0,center}{\begin{tikzcd}
	R & {S^{-1}R} \\
	& B
	\arrow["k", from=1-1, to=1-2]
	\arrow["\phi"', from=1-1, to=2-2]
	\arrow["{\exists !\psi}", dashed, from=1-2, to=2-2]
\end{tikzcd}} \\~\\
Moreover, $S^{-1}R$ is the unique commutative ring (up to unique isomorphism) that has such a property. 
\end{prp}

\begin{lmm}{}{} Let $R$ be a commutative ring. Let $S\subseteq R$ be a multiplicative subset of $R$. If $R$ is Noetherian, then $S^{-1}R$ is Noetherian. 
\end{lmm}

\subsection{Localization Away from Prime Ideals}
\begin{lmm}{}{} Let $R$ be a ring and $P$ a prime ideal of $R$. Then $R\setminus P$ is a multiplicative set. \tcbline
\begin{proof}
By definition, $xy\in P$ implies $x\in P$ or $y\in P$, since $R\setminus P$ removes all these elements, we have that $x\notin P$ and $y\notin P$ implies that $xy\notin P$. 
\end{proof}
\end{lmm}

\begin{defn}{Localization at Prime Ideals}{} Let $R$ be a commutative ring. Let $P$ be a prime ideal. Denote $$R_p=(R\setminus P)^{-1}R$$ the localization of $R$ at $P$. 
\end{defn}

\subsection{Localization of a Module}
\begin{defn}{Localization of a Module}{} Let $R$ be a commutative ring and $S\subseteq R$ be a multiplicative set Let $M$ be a $R$-module. Define the ring of fractions of $M$ with respect to $S$ by $$S^{-1}M=\left\{\frac{m}{s}|m\in M,s\in S\right\}/\sim$$ where $\sim$ is defined by $$\frac{m}{s}\sim\frac{m'}{s'}\text{ if and only if }\exists v\in S\text{ such that }v(mu'-m'u)=0$$ If $S=\{1,f,f^2,\dots\}$ then we write $$S^{-1}M=M_f=M[1/f]$$
\end{defn}

\begin{lmm}{}{} Let $R$ be a commutative ring. Let $M$ be an $R$-module. Let $S\subseteq R$ be a multiplicative subset. Then $S^{-1}M$ is an $S^{-1}R$-module with operation given by $$\left(\frac{r}{s_1},\frac{m}{s_2}\right)\mapsto\frac{r\cdot m}{s_1s_2}$$
\end{lmm}

\begin{defn}{Induced Map of Localization}{} Let $R$ be a commutative ring. Let $S\subseteq R$ be a multiplicative subset. Let $M,N$ be $R$-modules. Let $\phi:M\to N$ be an $R$-module homomorphism. Define the induced map $$S^{-1}\phi:S^{-1}M\to S^{-1}N$$ by the formula $\frac{m}{s}\mapsto\frac{\phi(n)}{s}$. 
\end{defn}

\begin{lmm}{}{} Let $R$ be a commutative ring. Let $S\subseteq R$ be a multiplicative subset. Let $M,N$ be $R$-modules. Let $\phi:M\to N$ be an $R$-module homomorphism. Then the induced map $$S^{-1}\phi:S^{-1}M\to S^{-1}N$$ is a well defined ring homomorphism. 
\end{lmm}

\begin{prp}{}{} Let $R$ be a commutative ring. Let $S\subseteq R$ be a multiplicative subset. Let the following be an exact sequence of $R$-modules. \\~\\
\adjustbox{scale=1.0,center}{\begin{tikzcd}
	0 & M_1 & M_2 & M_3 & 0
	\arrow[from=1-1, to=1-2]
	\arrow["f", from=1-2, to=1-3]
	\arrow["g", from=1-3, to=1-4]
	\arrow[from=1-4, to=1-5]
\end{tikzcd}}\\~\\
Then the following is an exact sequence of $S^{-1}R$-modules. \\~\\
\adjustbox{scale=1.0,center}{\begin{tikzcd}
	0 & S^{-1}M_1 & S^{-1}M_2 & S^{-1}M_3 & 0
	\arrow[from=1-1, to=1-2]
	\arrow["S^{-1}f", from=1-2, to=1-3]
	\arrow["S^{-1}g", from=1-3, to=1-4]
	\arrow[from=1-4, to=1-5]
\end{tikzcd}}\\~\\
\end{prp}

\begin{crl}{}{} Let $R$ be a commutative ring. Let $S\subseteq R$ be a multiplicative subset. Let $M$ be an $R$-module. Then the following are true. 
\begin{itemize}
\item If $N_1,N_2$ are $R$-submodules of $M$, then $$S^{-1}(N_1+N_2)=S^{-1}N_1+S^{-1}N_2$$ as $S^{-1}R$-submodules of $S^{-1}M$. 
\item If $N_1,N_2$ are $R$-submodules of $M$, then $$S^{-1}(N_1\cap N_2)=S^{-1}N_1\cap S^{-1}N_2$$ as $S^{-1}R$-submodules of $S^{-1}M$. 
\item If $N$ is an $R$-submodule of $M$, then $$S^{-1}\frac{M}{N}\cong\frac{S^{-1}M}{S^{-1}N}$$ as $S^{-1}R$-modules. 
\item If $N$ is an $R$-module, then $$S^{-1}(M\oplus N)\cong S^{-1}M\oplus S^{-1}N$$ as $S^{-1}R$-modules. 
\end{itemize}
\end{crl}

\begin{prp}{}{} Let $R$ be a commutative ring. Let $M$ be an $R$-module. Then there is an isomorphism $$S^{-1}M\cong S^{-1}R\otimes_RM$$ of $S^{-1}R$-modules given by $\frac{m}{s}\mapsto\frac{1}{s}\otimes m$. 
\end{prp}

\begin{lmm}{}{} Let $R$ be a commutative ring. Let $S\subseteq R$ be a multiplicative subset. Let $M,N$ be $R$-modules. Let $\phi:M\to N$ be an $R$-module homomorphism. Then the following are true. 
\begin{itemize}
\item Localization commutes with kernels: $$S^{-1}\ker(\phi)\cong\ker(S^{-1}\phi)$$
\item Localization commutes with images: $$S^{-1}(\im\phi)\cong\im(S^{-1}\phi)$$
\item Localization commutes with cokernels: $$S^{-1}\frac{N}{\im(\phi)}\cong\frac{S^{-1}N}{\im(S^{-1}\phi)}$$
\end{itemize}
\end{lmm}

\subsection{Localization of Integral Domains}
\begin{lmm}{}{} Let $R$ be a commutative ring. Let $S$ be a multiplicative subset of $R$. If $R$ is an integral domain, then then following are true. 
\begin{itemize}
\item The localization map $R\to S^{-1}R$ is injective. 
\item If $0\notin S$, then $S^{-1}R$ is an integral domain. 
\end{itemize} \tcbline
\begin{proof}
Suppose that $0=\frac{a}{s}\cdot\frac{b}{t}$. By the equivalence relation this is the same as saying that $uab=0$ for some $u\in S$. Since $R$ is an integral domain and $0\neq S$, we conclude that $u\notin S$ so that $ab=0$. Again since $R$ is an integral domain this implies that $a=0$ or $b=0$. Hence either $a/s=0$ or $b/t=0$ in $S^{-1}R$. Hence $S^{-1}R$ is an integral domain. 
\end{proof}
\end{lmm}

\begin{prp}{}{} Let $R$ be an integral domain. Then the following are true. 
\begin{itemize}
\item $\text{Frac}(R)=R_{(0)}$
\item $R=\bigcap_{m\text{ a maximal ideal}}R_m$
\end{itemize}
\end{prp}

\subsection{Ideals of a Localization}
\begin{defn}{Ideals Closed Under Division}{} Let $R$ be a commutative ring. Let $I$ be an ideal of $R$. Let $S\subseteq R$ be a multiplicative subset. We say that $I$ is closed under division by $s$ if for all $s\in S$ and $a\in R$ such that $sa\in I$, we have $a\in I$. 
\end{defn}

\begin{lmm}{}{} Let $R$ be a commutative ring. Let $I$ be an ideal of $R$. Let $S\subseteq R$ be a multiplicative subset. Then we have $$I^e=S^{-1}I$$
\end{lmm}

\begin{prp}{}{} Let $R$ be a commutative ring. Let $S$ be a multiplicative subset of $R$. Let $P$ be a prime ideal of $R$. Then the following are true. 
\begin{itemize}
\item $S^{-1}P$ is a prime ideal of $S^{-1}R$ if and only if $S\cap P=\emptyset$. 
\item $S^{-1}P=S^{-1}R$ if and only if $S\cap P\neq\emptyset$. 
\end{itemize} \tcbline
\begin{proof}
Recall that $R/P$ is an integral domain if $P$ is prime. Since $S^{-1}$ commutes with quotients, we have that $$\frac{S^{-1}R}{S^{-1}P}\cong S^{-1}\frac{R}{P}$$~\\

If $S\cap P=\emptyset$, then $0\in P$ implies that $0\notin S$. This means that $0\notin\phi(S)$. By 5.3.1 we conclude that $S^{-1}(R/P)$ is an integral domain. Hence $S^{-1}P$ is a prime ideal. If $S\cap P\neq\emptyset$, suppose that $x\in S\cap P$. Then ?????
\end{proof}
\end{prp}

\begin{thm}{}{} Let $R$ be a commutative ring. Let $I$ be an ideal of $R$. Let $S\subseteq R$ be a multiplicative subset. Let $\phi:R\to S^{-1}R$ denote the localization map. Then there is a one-to-one bijection $$\{J\;|\;J\text{ is an ideal of }S^{-1}R\}\;\;\overset{1:1}{\longleftrightarrow}\;\;\left\{I\;|\;\substack{I\text{ is an ideal of }R\text{ and }\\I\text{ is closed under division by }S}\right\}$$ whose map is given by $J\mapsto J^c=\phi^{-1}(J)$ and inverse is given by $I\mapsto I^e=S^{-1}I$. \tcbline
\begin{proof}
We first show that our map of sets are well defined. Let $J$ be an ideal of $S^{-1}R$. We first show that $\phi^{-1}(J)$ is closed under division by $S$. Suppose that $s\in S$ and $r\in R$ such that $sr\in\phi^{-1}(J)$. Then $sr/1\in J$. Now since $J$ is an ideal of $S^{-1}R$, we know that $1/s\cdot sr/1\in J$. But $1/s\cdot sr/1=r/1=\phi(r)$. This means that $\phi(r)\in J$ and hence $r\in\phi^{-1}(J)$. Thus $\phi^{-1}(J)$ is an ideal closed under division by $S$. \\~\\

Now let $I$ be an ideal of $R$ closed under division. I claim that $S^{-1}I$ is an ideal of $S^{-1}R$. Let $a/s,b/t\in S^{-1}I$. Then $a/s+b/t=(at+bs)/st$. Since $I$ is an ideal, we know that $at+bs\in I$. Also since $S$ is a multiplicative subset, $st\in S$. Hence $(at+bs)/st\in I$. Now let $a/s\in S^{-1}I$ and $r/t\in S^{-1}R$. Then $(a/s)\cdot(r/t)=ar/st$. Since $I$ is an ideal, $ar\in I$. Thus $ar/st\in S^{-1}I$ so that $I$ is an ideal. \\~\\

It remains to show that the two maps are inverses of each other. Let $J$ be an ideal of $S^{-1}R$. We want to show that $J=S^{-1}(\phi^{-1}(J))$. Let $a/s\in J$. Since $J$ is an ideal, we have $\phi(a)=a/1=1/s\cdot a/s\in J$. Hence $a\in\phi^{-1}J$ so that $a/s\in S^{-1}\phi^{-1}(J)$. Thus $J\subseteq S^{-1}(\phi^{-1}(J))$. Now by 1.5.5 the extension of the contraction of $J$ is a subset of $J$. Hence we conclude. \\~\\

On the other hand, we also want to show that $I=\phi^{-1}(S^{-1}I)$. Again by 1.5.5 we know that $I\subseteq\phi^{-1}(S^{-1}I)$. Conversely, let $x\in\phi^{-1}(S^{-1}I)$. Then $\phi(x)=x/1\in S^{-1}I$. This means that $x/1=b/t$ for some $b\in I$ and $t\in S$. Then there exists $u\in S$ such that $uxt=ub$. Since $b\in I$, $ub\in I$ hence $uxt\in I$. Since $ut\in S$ and $I$ is closed under division, we have $x\in I$. \\~\\

This shows that $S^{-1}(-)$ and $\phi^{-1}(-)$ are mutual inverses of each others. Thus we conclude. 
\end{proof}
\end{thm}

Using the theorem we conclude that every ideal of $S^{-1}R$ is of the form $S^{-1}I$ for some ideal $I$ of $R$ such that $I$ is closed under division by $S$. 

\begin{prp}{}{} Let $R$ be a commutative ring. Let $I$ be an ideal of $R$. Let $S\subseteq R$ be a multiplicative subset. Then the above bijection restricts to the following bijection $$\{J\;|\;J\text{ is a prime ideal of }S^{-1}R\}\;\;\overset{1:1}{\longleftrightarrow}\;\;\left\{I\;\bigg{|}\;\substack{I\text{ is a prime ideal of }R\\\text{ and }I\cap S=\emptyset}\right\}$$ \tcbline
\begin{proof}
Let $\phi:R\to S^{-1}R$ be the localization map. From the above we know that $Q=S^{-1}\phi^{-1}(Q)$ for any prime ideal $Q$ of $S^{-1}R$. This implies that $S^{-1}\phi^{-1}(Q)$ is prime. By 5.4.3 this implies that $\phi^{-1}(Q)\cap S=\emptyset$. Thus the map $J\mapsto\phi^{-1}(J)$ induces a well defined map on our given sets of prime ideals. \\~\\

Conversely, by 5.4.3 we know that if $P$ is a prime ideal of $R$ such that $S\cap P=\emptyset$, then $S^{-1}P$ is a prime ideal of $S^{-1}R$. Hence the inverse map is also well defined on our domain and codomain. By the above theorem it is already a bijection, hence we are done. 
\end{proof}
\end{prp}

\begin{prp}{}{} Let $R$ be a commutative ring. Let $P$ be a prime ideal of $R$. Then the above bijection gives $$\{J\;|\;J\text{ is a prime ideal of }R_P\}\;\;\overset{1:1}{\longleftrightarrow}\;\;\left\{I\;\bigg{|}\;\substack{I\text{ is a prime ideal of }R\\\text{ and }I\subseteq P}\right\}$$ \tcbline
\begin{proof}
Notice that the condition that $I\cap S=\emptyset$ in the above proposition translates to $I\cap (R\setminus P)=\emptyset$, which is the same as saying $I\subseteq P$. 
\end{proof}
\end{prp}

\begin{prp}{}{} Let $R$ be a commutative ring and let $P$ be a prime ideal of $R$. Then $R_P$ is a local ring with unique maximal ideal given by $$PR_P=\left\{\frac{r}{s}\;|\;r\in P,s\notin P\right\}$$ \tcbline
\begin{proof}
We show that $PR_P$ is the only unique maximal ideal. Suppose that $I$ is an ideal in $R_P$ such that $I$ is not a subset of $PR_P$. Then there exists $a/s\in I$ such that $a\notin P$ and $s\notin P$. It is clear that $s/a$ is then an element of $R_P$. So $a/s$ is invertible. Hence $I=R_P$. 
\end{proof}
\end{prp}

Be wary that in general localizations does not result in a local ring. This happens only when we are localizing with respect to a prime ideal. The importance of prime ideals is not explicit in the above because only using prime ideals $P$ can $R\setminus P$ be a multiplicative set which ultimately allows localization to make sense. 

\begin{prp}{Localization of a Localization}{} Let $R$ be a commutative ring. Let $S$ be a multiplicative subset of $R$. Let $P$ be a prime ideal of $R$ such that $S^{-1}P$ is a prime ideal of $S^{-1}R$. Then $$(S^{-1}R)_{S^{-1}P}\cong R_P$$ \tcbline
\begin{proof}
Define a map $S^{-1}R\to R_P$ by the identity map. This is well defined because if $s\in S$, then we know $S^{-1}P$ is a prime ideal implies $S\cap P=\emptyset$, so $s\notin P$. Thus $r/s$ is a well defined fraction in $R_P$. Since it is just the identity map, it is a well defined ring homomorphism. Now let $r/s\in S^{-1}R\setminus S^{-1}P$. Then $r\notin P$ implies that $r$ is invertible in $R_P$. Hence $r/s\cdot s/r=1$ in $R_P$. Thus $r/s$ is invertible in $R_P$. Thus we can invoke the universal property to obtain a unique map $$(S^{-1}R)_{S^{-1}P}\to R_P$$ Conversely, define a map $R\to (S^{-1}R)_{S^{-1}P}$ by the identity map $r\mapsto (r/1)/(1/1)$. This is well defined because $1\notin P$ implies $1/1\in S^{-1}R\setminus S^{-1}P$. Clearly this is a well defined ring homomorphism. For $s\in S$, notice that $(s/1)/(1/1)$ is invertible in $(S^{-1}R)_{S^{-1}P}$ via the element $(1/s)/(1/1)$. Thus we can invoke the universal property of $S^{-1}R$ to obtain a unique map $$S^{-1}R\to(S^{-1}R)_{S^{-1}P}$$ We now have two unique maps going both directions between $S^{-1}R$ and $(S^{-1}R)_{S^{-1}P}$. This implies that they are isomorphic. 
\end{proof}
\end{prp}

\subsection{Localization of Graded Rings}
\begin{prp}{}{} Let $R=\bigoplus_{i=0}^\infty R_i$ be a commutative ring that is graded. Let $P$ be a homogeneous prime ideal of $R$. Then $R_P$ is a graded ring in which the grading structure is given as follows: $f/g\in R_P$ has degree $\deg(f)-\deg(g)$. 
\end{prp}

\begin{defn}{Localization of a Graded Ring}{} Let $R=\bigoplus_{i=0}^\infty R_i$ be a commutative ring that is graded. Let $P$ be a homogeneous prime ideal of $R$. Define the localization of $R$ with respect to $P$ to be $$(R_P)_0=\{f\in R_P\;|\;f\text{ lies in the }0\text{th graded component of }R_P\}$$
\end{defn}

\begin{prp}{}{} Let $R=\bigoplus_{i=0}^\infty R_i$ be a commutative ring that is graded. Let $P$ be a homogeneous prime ideal of $R$. Then $(R_P)_0$ is a local ring with unique maximal ideal given by $$(PR_P)\cap(R_P)_0$$
\end{prp}

\subsection{Local Properties}
\begin{defn}{Local Properties of Modules}{} Let $R$ be a commutative ring. A property of $R$-modules is local if for any $R$-modules $M$, the following are equivalent. 
\begin{itemize}
\item $M$ has the property
\item $M_P$ has the property for all primes ideals $P$
\item $M_m$ has the property for all maximal ideals $m$
\end{itemize}
\end{defn}

\begin{prp}{Injectivity and Surjectivity are Local Properties}{} Let $R$ be a commutative ring. Let $M,N$ be $R$-modules. Let $\phi:M\to N$ be an $R$-module homomorphism. Let $S$ be a multiplicative subset of $R$. Then the following are equivalent. 
\begin{itemize}
\item $\phi$ is injective (surjective)
\item For each prime ideal $P$ of $R$, the induced map $\phi_P:S^{-1}M\to S^{-1}N$ is injective (surjective)
\item For each maximal ideal $m$ of $R$, the induced map $\phi_m:S^{-1}M\to S^{-1}N$ is injective (surjective)
\end{itemize}
\end{prp}

More local properties: zero, nilpotent\\
Non-local properties: freeness, domain

\begin{prp}{Exactness is Local}{} Let $R$ be a commutative ring. Let $M_1,M_2,M_3$ be $R$-modules. Let $f:M_1\to M_2$ and $g:M_2\to M_3$ be $R$-module homomorphisms. Then the following conditions are equivalent. 
\begin{itemize}
\item The following sequence is exact: \\~\\
\adjustbox{scale=1.0,center}{\begin{tikzcd}
	0 & M_1 & M_2 & M_3 & 0
	\arrow[from=1-1, to=1-2]
	\arrow["f", from=1-2, to=1-3]
	\arrow["g", from=1-3, to=1-4]
	\arrow[from=1-4, to=1-5]
\end{tikzcd}}\\~\\
\item The following sequence is exact: \\~\\
\adjustbox{scale=1.0,center}{\begin{tikzcd}
	0 & (M_1)_P & (M_2)_P & (M_3)_P & 0
	\arrow[from=1-1, to=1-2]
	\arrow["f_P", from=1-2, to=1-3]
	\arrow["g_P", from=1-3, to=1-4]
	\arrow[from=1-4, to=1-5]
\end{tikzcd}}\\~\\
for all prime ideals $P$ of $R$. 
\item The following sequence is exact: \\~\\
\adjustbox{scale=1.0,center}{\begin{tikzcd}
	0 & (M_1)_m & (M_2)_m & (M_3)_m & 0
	\arrow[from=1-1, to=1-2]
	\arrow["f_m", from=1-2, to=1-3]
	\arrow["g_m", from=1-3, to=1-4]
	\arrow[from=1-4, to=1-5]
\end{tikzcd}}\\~\\
for all maximal ideals $m$ of $R$. 
\end{itemize}
\end{prp}

\begin{defn}{Local Properties of Elements}{}A property of an element of $M$ is local if the following is true. $m\in M$ has the property if and only if $m\in M_P$ has the property. 
\end{defn}

\pagebreak
\section{Primary Decomposition}
\subsection{The Annihilator and the Support of a Module}
Let $R$ be a commutative ring. Let $M$ be an $R$-module. Recall that we define the annihilator of a subset $S\subseteq M$ to be the ideal $$\text{Ann}_R(S)=\{r\in R\;|\;rs=0\text{ for all }s\in S\}$$ When $R$ is a commutative ring, the annihilator is a two sided ideal and consequently has some nice properties. 

\begin{prp}{}{} Let $R$ be a commutative ring. Let $M$ be an $R$-module. Let $\text{Ann}_R(x)$ for $x\in M$ be a maximal element in the set $$\{\text{Ann}_R(x)\;|\;0\neq x\in M\}$$ Then $\text{Ann}_R(x)$ is a prime ideal. \tcbline
\begin{proof}
Suppose that $ab\in\text{Ann}_R(x)$ and $b\notin\text{Ann}_R(x)$. Notice that if $rx=0$ then $r(bx)=brx=0$ so that $r$ annihilates $bx$. Hence $\text{Ann}_R(x)\subseteq\text{Ann}_R(bx)$. Since $x$ is non-zero and $b\notin I$, $bx$ is also non-zero hence $\text{Ann}_R(bx)$ lies in the given set of annihilators. Since $\text{Ann}_R(x)$ is maximal we conclude that $$\text{Ann}_R(x)=\text{Ann}_R(bx)$$ But $ab$ annihilates $x$ by definition so that $a$ annihilates $bx$. Hence $a\in\text{Ann}_R(bx)=\text{Ann}_R(x)$. Hence $\text{Ann}_R(x)$ is prime. 
\end{proof}
\end{prp}

Recall that if $S\subseteq M$ is a subset and $R$ is not a commutative ring, then in general we only have the relation $$\text{Ann}_R(\langle S\rangle)\subseteq\text{Ann}_R(S)$$

\begin{prp}{}{} Let $R$ be a commutative ring. Let $M$ be an $R$-module. Let $S\subseteq M$ be a subset. Then $$\text{Ann}_R(\langle S\rangle)=\text{Ann}_R(S)$$
\end{prp}

\begin{defn}{Support of a Module}{} Let $A$ be a commutative ring. Let $M$ be an $A$-module. The support of $M$ is the subset $$\text{Supp}(M)=\{P\text{ a prime ideal of }A\;|\;M_P\neq 0\}$$
\end{defn}

Let $R$ be a commutative ring. Let $M$ be an $R$-module. Recall that the annihilator of an element $m\in M$ is the ideal $$\text{Ann}_R(m)=\{r\in R\;|\;r\cdot m=0\}$$ Moreover, we define $$\text{Ann}_R(M)=\{r\in R\;|\;r\cdot m=0\text{ for all }m\in M\}=\bigcap_{m\in M}\text{Ann}_R(m)$$

\begin{prp}{}{} Let $R$ be a commutative ring. Let $M$ be an $R$-module. Then $$\{P\in\text{Spec}(R)\;|\;\text{Ann}_R(M)\subseteq P\}=\text{Supp}(M)$$
\end{prp}

We can write the set on the left as a vanishing set so the proposition can be read as $$\V(\text{Ann}_R(M))=\text{Supp}(M)$$

\subsection{Associated Prime}
\begin{defn}{Associated Prime}{} Let $R$ be a commutative ring. Let $M$ be an $R$-module. Let $P$ be a prime ideal of $R$. We say that $P$ is an associated prime of $M$ if $$\text{Ann}_R(m)=P$$ for some $m\in M$. 
\end{defn}

\begin{defn}{Set of Associated Prime}{} Let $R$ be a commutative ring. Let $M$ be an $R$-module. Define the set of associated primes of $M$ to be $$\text{Ass}(M)=\{P\in\text{Spec}(R)\;|\;P\text{ is an associated prime of }M\}$$
\end{defn}

\begin{prp}{}{} Let $R$ be a commutative ring. Let $M$ be an $R$-module. Then $$\text{Ass}(M)\subseteq\text{Supp}(M)$$
\end{prp}

\begin{prp}{}{} Let $R$ be a commutative ring. Let $M$ be an $R$-module. Then the following are true. 
\begin{itemize}
\item $\text{Ass}(M)$ is a finite set. 
\item For $P\in\text{Ass}(M)$, $\text{Ann}_R(M)\subseteq P$. 
\item We have $$\text{Ass}(M)=\{P\in\text{Spec}(R)\;|\;\text{ For any prime ideal }Q\subseteq P, Q\text{ does not contain }\text{Ann}_R(M)\}$$
\end{itemize} \tcbline
\begin{proof}~\\
\begin{itemize}
\item 
\item We have seen that every $P\in\text{Supp}(M)$ is such that $\text{Ann}_R(M)\subseteq P$. Since $\text{Ass}(M)\subseteq\text{Supp}(M)$, we are done. 
\end{itemize}
\end{proof}
\end{prp}

\begin{prp}{}{} Let $R$ be a commutative ring. Let $M$ be an $R$-module. Then $$\bigcup_{P\in\text{Ass}(M)}P=\{m\in M\;|\;m\text{ is a zero divisor of }M\}\cup\{0\}$$
\end{prp}

\begin{thm}{Disassembly of an R-Module}{} Let $R$ be a Noetherian commutative ring. Let $M$ be a finitely generated $R$-module. Then there exists a chain of $R$-submodules $$0=M_0\subset M_1\subset\cdots\subset M_k=M$$ such that $$\frac{M_{i+1}}{M_i}\cong\frac{R}{P_i}$$ for some prime ideal $P_i$ of $R$. 
\end{thm}

\subsection{Primary Ideals}
\begin{defn}{Primary Ideals}{} Let $R$ be a commutative ring. Let $Q$ be a proper ideal of $R$. We say that $Q$ is a primary ideal of $R$ if $fg\in Q$ implies $f\in Q$ or $g^m\in Q$ for some $m>0$. 
\end{defn}

\begin{prp}{}{} Let $R$ be a commutative ring. Let $Q$ be a proper ideal of $R$. Then $Q$ is primary if and only if every zero divisor in $R/Q$ is nilpotent. 
\end{prp}

\begin{lmm}{}{} Let $R$ be a commutative ring. Let $P$ be a prime ideal of $R$. Then $P$ is a primary ideal. 
\end{lmm}

\begin{lmm}{}{} Let $R$ be a commutative ring. Let $Q$ be a primary ideal of $R$. Then the following are true. 
\begin{itemize}
\item $\sqrt{Q}$ is a prime ideal. 
\item $\sqrt{Q}$ is minimal among primes that contain $Q$. 
\end{itemize}
\end{lmm}

\begin{defn}{P-Primary Ideals}{} Let $R$ be a commutative ring. Let $P$ be a prime ideal. Let $Q$ be an ideal. We say that $Q$ is a $P$-primary ideal of $R$ if the following are true. 
\begin{itemize}
\item $Q$ is a primary ideal. 
\item $Q=\sqrt{P}$. 
\end{itemize}
\end{defn}

\begin{prp}{}{} Let $R$ be a commutative ring. Let $I$ be an ideal of $R$. If $\sqrt{I}$ is maximal, then $I$ is an $\sqrt{I}$-primary ideal. 
\end{prp}

\begin{prp}{}{} Let $R$ be a Noetherian commutative ring. Let $P$ be a prime ideal of $R$. Let $Q$ be a proper ideal. Then the following are equivalent. 
\begin{itemize}
\item $Q$ is $P$-primary. 
\item $\text{Ann}(A/Q)=\{P\}$
\item There exists $n\in\N$ such that $P^n\subseteq Q\subseteq P$. 
\end{itemize}
\end{prp}

\subsection{Primary Decomposition}
We want to express ideal $I$ in $R$ as $I=P_1^{e_1}\cdots P_n^{e_n}$ similar to a factorization of natural numbers, for some prime ideals $P_1,\dots,P_n$. However this notion fails and thus we have the following new type of ideal. 

\begin{defn}{Primary Decompositions}{} Let $A$ be a commutative ring. Let $I$ be an ideal of $A$. A primary decomposition $I$ consists of primary ideals $Q_1,\dots,Q_r$ of $A$ such that $$I=Q_1\cap\cdots\cap Q_r$$
\end{defn}

\begin{defn}{Minimal Primary Decompositions}{} Let $A$ be a commutative ring. Let $I$ be an ideal of $A$. Let $$I=Q_1\cap\cdots\cap Q_r$$ be a primary decomposition of $I$. We say that the decomposition is minimal if the following are true. 
\begin{itemize}
\item Each $\sqrt{Q_i}$ are distinct for $1\leq i\leq r$
\item Removing a primary ideal changes the intersection. This means that for any $i$, $I\neq\bigcap_{j\neq i}Q_j$
\end{itemize}
\end{defn}

\begin{lmm}{}{} Let $\phi:R\to S$ be a ring homomorphism and $Q$ be a primary ideal in $S$. Then $\phi^{-1}(Q)$ is primary in $R$. 
\end{lmm}

\begin{defn}{Prime Divisors of an Ideal}{} Let $R$ be a commutative ring. Let $I$ be an ideal of $R$. We say that a prime ideal $P$ of $R$ is a prime divisor of $I$ if $P=\sqrt{Q}$ for some ideal $Q$ that lies in a minimal primary decomposition of $I$. 
\end{defn}

\subsection{The Noetherian Case}
\begin{thm}{}{} Let $R$ be a Noetherian commutative ring. Let $I$ be a proper ideal of $R$. Then $I$ admits a primary decomposition. 
\end{thm}

\begin{prp}{}{} Let $R$ be a Noetherian commutative ring. Let $m$ be maximal ideal of $R$. Let $I$ be an ideal of $R$. Then the following are equivalent. 
\begin{itemize}
\item $I$ is $m$-primary. 
\item $\sqrt{I}=m$. 
\item There exists $n\in\N$ such that $m^n\subseteq I\subseteq m$. 
\end{itemize}
\end{prp}

\pagebreak
\section{Integral Dependence}
\subsection{Integral Elements}
\begin{defn}{Integral Elements}{} Let $B$ be a commutative ring and let $A\subseteq B$ be a subring. Let $b\in B$. We say that $b$ is integral over $A$ if there exists a monic polynomial $p(x)=x^n+a_{n-1}x^{n-1}+\dots+a_0\in A[x]$ such that $p(b)=0$. 
\end{defn}

When $A$ and $B$ are field, this is a familiar notion in Field and Galois theory. 

\begin{lmm}{}{} Let $K$ be a field. Let $F\subseteq K$ be a subfield. Let $k\in K$. Then $k$ is integral over $F$ if and only if $k$ is algebraic over $F$. 
\end{lmm}

\begin{prp}{}{} Let $B$ be a commutative ring and let $A\subseteq B$. Let $b\in B$. Then the following are equivalent. 
\begin{itemize}
\item $b$ is integral over $A$
\item $A[b]\subseteq B$ is finitely generated $A$-submodule. 
\item There exists an $A$ sub-algebra $A'\subseteq B$ such that $A[b]\subseteq A'$ and $A'$ is finitely generated as an $A$-module. 
\end{itemize}
\end{prp}

\begin{prp}{}{} Let $A\subseteq B$ be commutative rings. Then $B$ is a finitely generated $A$-module if and only if $B=A[x_1,\dots,x_n]$ for some $x_1,\dots,x_n\in B$ that is integral over $A$. 
\end{prp}

\begin{prp}{}{} Let $B$ be a commutative ring and let $A\subseteq B$ be a subring. Let $b_1,b_2\in B$ be integral over $A$. Then $b_1+b_2$ and $b_1b_2$ are both integral over $A$. 
\end{prp}

\subsection{Integral Closure}
\begin{defn}{Integral Closure}{} Let $B$ be a commutative ring. Let $A\subseteq B$ be a subring. Define the subring $$\overline{A}=\{b\in B\;|\;b\text{ is integral over }A\}$$ to be the integral closure of $A$ in $B$. 
\end{defn}

\begin{prp}{}{} Let $B$ be a commutative ring. Let $A\subseteq B$ be a subring. Let $S$ be a multiplicatively closed subset of $A$. Then $$\overline{S^{-1}A}=S^{-1}\overline{A}$$
\end{prp}

\begin{defn}{Integral Extensions}{} Let $B$ be a commutative ring and let $A\subseteq B$ be a subring. We say that $B$ is integral over $A$ if $\overline{A}=B$. We also say that $B$ is the integral extension of $A$. 
\end{defn}

\begin{lmm}{}{} Let $A\subseteq B\subseteq C$ be commutative rings. Then $C$ is integral over $B$ and $B$ is integral over $A$ if and only if $C$ is integral over $A$. 
\end{lmm}

\begin{prp}{}{} Let $A,B$ be commutative rings such that $A\subset B$ is an integral extension. Let $J$ be an ideal of $B$. Then $\frac{B}{J}$ is integral over $\frac{A}{J\cap A}$. 
\end{prp}

\begin{prp}{}{} Let $A,B$ be commutative rings such that $A\subset B$ is an integral extension. Let $S$ be a multiplicative subset of $B$. Then $S^{-1}B$ is integral over $S^{-1}A$. 
\end{prp}

\begin{lmm}{}{} Let $A,B$ be integral domains such that $A\subset B$ is an integral extension. Then $A$ is a field if and only if $B$ is a field. 
\end{lmm}

\begin{defn}{Integrally Closed}{} Let $B$ be a commutative ring. Let $A\subseteq B$ be a subring. We say that $A$ is integrally closed in $B$ if $\overline{A}=A$. 
\end{defn}

\subsection{The Going-Up and Going-Down Theorems}
We want to compare prime ideals between integral extensions. 

\begin{prp}{}{} Let $A,B$ be rings such that $A\subset B$ is an integral extension. Let $Q$ be a prime ideal of $B$. Then $Q\cap A$ is a maximal ideal of $A$ if and only if $Q$ is a maximal ideal of $B$. 
\end{prp}

\begin{prp}{}{} Let $A,B$ be rings such that $A\subset B$ is an integral extension. Let $P$ be a prime ideal of $A$. Then the following are true. 
\begin{itemize}
\item There exists a prime ideal $Q$ of $B$ such that $P=Q\cap A$
\item If $Q_1,Q_2$ are prime ideals of $B$ such that $Q_1\cap A=P=Q_2\cap B$ and $Q_1\subseteq Q_2$, then $Q_1=Q_2$. 
\end{itemize}
\end{prp}

\begin{thm}{The Going-Up Theorem}{} Let $A,B$ be rings such that $A\subset B$ is an integral extension. Let $0\leq m<n$. Consider the following situation \\~\\
\adjustbox{scale=1.0,center}{\begin{tikzcd}
	B & {Q_1\subseteq\cdots\subseteq Q_m} && {(\text{Prime ideals of }B)} \\
	A & {P_1\subseteq\cdots\subseteq P_m} & {\subseteq P_{m+1}\subseteq\cdots\subseteq P_n} & {(\text{Prime ideals of }A)}
	\arrow[hook, from=2-1, to=1-1]
\end{tikzcd}}\\~\\
where $Q_i\cap A=P_i$ for $1\leq i\leq m$. Then there exists prime ideals $Q_{m+1},\dots,Q_n$ of $B$ such that the following are true. 
\begin{itemize}
\item $Q_{m+1}\subseteq\cdots\subseteq Q_n$
\item $Q_i\cap A=P_i$ for $m+1\leq i\leq n$
\end{itemize}
\end{thm}

\subsection{Normal Domains}
We now concern ourselves with integral domains. Let $R$ be an integral domain. A special fact about $R$ is that the canonical homomorphism $R\to R_{(0)}=\text{Frac}(R)$ is an injection. This means that we can we can think of $R$ as living inside of $\text{Frac}(R)$ while preserving all the structure of $R$. 

\begin{defn}{Normal Domains}{} Let $R$ be an integral domain. We say that $R$ is normal if $R$ is integrally closed in $\text{Frac}(R)$. 
\end{defn}

\begin{prp}{}{} Let $R$ be a normal domain. Let $S$ be a multiplicative subset of $R$. Then $S^{-1}R$ is a normal domain. \tcbline
\begin{proof}
We want to show that $S^{-1}R$ is integrally closed in $\text{Frac}(R)=\text{Frac}(S^{-1}R)$. This means that we want to show $\overline{S^{-1}R}=S^{-1}R$. It is clear that $S^{-1}R\subseteq\overline{S^{-1}R}$. So let $g\in\overline{S^{-1}R}$. Suppose that $p(x)=x^n+\sum_{k=0}^{n-1}a_kx^k\in (S^{-1}R)[x]$ such that $p(g)=0$. Choose $s\in S$ such that $sa_i\in R$ for $0\leq i\leq n-1$. Then notice that $sg\in S^{-1}R$ satisfies the monic polynomial $$q(x)=x^n+\sum_{k=0}^{n-1}s^{n-k}a_kx^k$$ since $q(sg)=s^ng^n+\sum_{k=0}^{n-1}s^na_kx^k=s^np(g)=0$. But $q$ is a polynomial in $R$ since $s^{n-k}a_k\in R$. Thus we have that $sg\in\overline{R}=R$ since $R$ is normal. This means that $g\in S^{-1}R$ and hence we conclude. 
\end{proof}
\end{prp}

\begin{prp}{}{} Let $R$ be a commutative ring. If $R$ is a UFD, then $R$ is normal. 
\end{prp}

\begin{prp}{Normal is a Local Property}{} Let $R$ be an integral domain. Then the following are equivalent. 
\begin{itemize}
\item $R$ is normal
\item $R_P$ is normal for all prime ideals $P$
\item $R_m$ is normal for all maximal ideals $m$. 
\end{itemize} \tcbline
\begin{proof}
Notice that an integral domain $R$ is normal if and only if the canonical inclusion map $R\hookrightarrow\overline{R}$ is surjective. Since surjectivity is a local property, this map is surjective if and only if for all prime ideals $P$ of $R$, $R_P\hookrightarrow\overline{R}_P$ is surjective. But $\overline{R}_P=\overline{R_P}$ by the above. Hence $R\hookrightarrow\overline{R}$ is surjective if and only if $R_P\to\overline{R_P}$ is surjective. Hence $R$ is normal if and only if $R_P$ is normal for all prime ideals $P$ of $R$. The similar holds for all maximal ideals. 
\end{proof}
\end{prp}

Atiyah-Macdonald

\begin{prp}{}{} Let $R$ be a normal domain. Then $R[x]$ is a normal domain. 
\end{prp}

\begin{prp}{}{} Let $R$ be a normal domain. Let $K/\text{Frac}(R)$ be an algebraic extension. Let $f\in K$. Then $f$ is integral over $R$ if and only if the minimal polynomial $\min(K,f)\in R[x]$. 
\end{prp}

\pagebreak
\section{Introduction to Dimension Theory for Rings}
\subsection{Krull Dimension}
\begin{defn}{Krull Dimension}{} Let $R$ be a commutative ring. Define the Krull dimension of $R$ to be $$\dim(R)=\max\{t\in\N\;|\;p_0\subset\dots\subset p_t\text{ for }p_0,\dots,p_t\text{ prime ideals}\}$$
\end{defn}

In particular, notice that a commutative ring $R$ has $\dim(R)=0$ if and only if every prime ideal is maximal. 

\begin{lmm}{}{} Let $R,S$ be commutative rings such that $R\subseteq S$ is an integral extension. Then $\dim(R)=\dim(S)$. 
\end{lmm}

\begin{prp}{}{} Let $F$ be a field. Let $n\in\N\setminus\{0\}$. Then the following are true. 
\begin{itemize}
\item $\dim(F[x_1,\dots,x_n])=n$. 
\item Every maximal chain prime ideals in $F[x_1,\dots,x_n]$ is of length $n$. 
\end{itemize}
\end{prp}

\begin{lmm}{}{} Let $R$ be a commutative ring. Then the following are true. 
\begin{itemize}
\item If $R$ is a field, then $\dim(R)=0$
\item If $R$ is Artinian, then $\dim(R)=0$
\end{itemize} \tcbline
\begin{proof}
Let $R$ be a field. Then the only proper prime ideal of $R$ is $(0)$. In particular, $(0)$ forms the only chain of prime ideals in $R$. Hence $\dim(R)=0$. \\~\\

Now let $R$ be Artinian. Let $P$ be a prime ideal of $R$. Then $R/P$ is an integral domain. Moreover, every quotient of an Artinian ring is Artinian. Hence $R/P$ is Artinian. By prp1.3.1, we conclude that $R/P$ is a field. Hence $P$ is a maximal ideal. Any chain of prime ideals of $R$ must terminate at the first prime ideal since it is maximal. Hence $\dim(R)=0$. 
\end{proof}
\end{lmm}

\begin{defn}{}{} Let $R$ be a commutative ring. Let $M$ be an $R$-module. Define the dimension of $M$ to be $$\dim(M)=\dim\left(\frac{R}{\text{Ann}_R(M)}\right)$$
\end{defn}

\subsection{Height of Prime Ideals}
\begin{defn}{Height of a Prime Ideal}{} Let $R$ be a commutative ring. Let $p$ be a prime ideal of $R$. Define the height of $p$ to be $$\text{ht}(p)=\max\{t\in\N\;|\;p_0\subset\dots\subset p_t=p\text{ for }p_0,\dots,p_t\text{ prime ideals }\}$$
\end{defn}

\begin{lmm}{}{} Let $R$ be a commutative ring. Then $$\dim(R)=\max\{\text{ht}(P)\;|\;P\in\text{Spec}(R)\}$$
\end{lmm}

\begin{lmm}{}{} Let $R$ be a commutative ring. Let $P$ be a prime ideal of $R$. Then $$\text{ht}(P)=\dim(R_P)$$ \tcbline
\begin{proof}
Let $\dim(R_P)=n$. Then there exists a strict chain of prime ideals of $R_P$ of length $n$ (and no chain of prime ideals of length $>n$). By prp5.4.6, prime ideals of $R_P$ are in bijection with prime ideals of $R$ that $P$ contains. Hence the maximal chain of prime ideals of length $n$ correspond to a chain of prime ideals in $R$ that contain $P$, of length $n$. Hence $\dim(R_p)=n\leq\text{ht}(P)$. Conversely, let $m=\text{ht}(P)$. Then there exists a strict chain of prime ideals that are subsets of $P$, that are of length $m$. By the same correspondence, the chain of prime ideals correspond to a chain of prime ideals in $R_P$ of length $m$. Hence $\text{ht}(P)=m\leq\dim(R_P)$. \\~\\

The two inequalities combine to show that $\dim(R_P)=\text{ht}(P)$. 
\end{proof}
\end{lmm}

\begin{lmm}{}{} Let $R$ be a commutative ring. Let $P$ be a prime ideal of $R$. Then $$\dim(R)\geq\dim(R/P)+\text{ht}_R(P)$$
\end{lmm}

\begin{prp}{}{} Let $k$ be a field. Let $A$ be an integral domain that is a finitely generated $k$-algebra. Then the following are true. 
\begin{itemize}
\item $\dim(A)=\text{trdeg}_k(\text{Frac}(A))$
\item For any prime ideal $P$ of $A$, we have $$\dim(A)=\dim(A/P)+\text{ht}_A(P)$$
\end{itemize}
\end{prp}

\begin{prp}{Dimension is a Local Concept}{} Let $R$ be a commutative ring. Then the following numbers are equal. 
\begin{itemize}
\item The Krull dimension $\dim(R)$
\item The supremum $\sup\{\dim(R_m)\;|\;m\text{ is a maximal ideal of }R\}$
\item The supremum $\sup\{\text{ht}_R(m)\;|\;m\text{ is a maximal ideal of }R\}$
\end{itemize}
\end{prp}

\begin{crl}{}{} Let $(R,m)$ be a local ring. Then $$\dim(R)=\dim(R_m)=\text{ht}_R(m)$$
\end{crl}

\begin{thm}{Krull's Principal Ideal Theorem}{} Let $R$ be a Noetherian ring. Let $I$ be a proper and principal ideal of $R$. Let $p$ be the smallest prime ideal containing $I$. Then $$\text{ht}_R(p)\leq 1$$
\end{thm}

\subsection{The Length of Modules over Commutative Rings}
Let $R$ be a ring. Recall that the length of an $R$-module $M$ is defined to be the supremum $$l_R(M)=\text{sup}\{n\in\N\;|\;0=M_0\subset M_1\subset\cdots\subset M_n=M\}$$

\begin{lmm}{}{} Let $(A,m)$ be a local ring and let $M$ be an $A$-module. If $mM=0$, then $$l_A(M)=\dim_{A/m}(M)$$
\end{lmm}

\begin{prp}{}{} Let $R$ be a commutative ring and let $M$ be an $R$-module. Then the following are equivalent. 
\begin{itemize}
\item $M$ is simple
\item $l_R(M)=1$
\item $M\cong R/m$ for some maximal ideal $m$ of $R$
\end{itemize}
\end{prp}

\subsection{Structure Theorem for Artinian Rings}
Let $R$ be a ring. Let $M$ be an $R$-module. Recall that a composition series for $M$ is a sequence of $R$-submodules $$0=M_0\subset M_1\subset\cdots\subset M_k=M$$ such that $\frac{M_{i+1}}{M_i}$ is a simple $R$-module for $1\leq i<k$. 

\begin{prp}{}{} Let $R\neq 0$ be a commutative ring. Then $R$ is Artinian if and only if $R$ is Noetherian and $\dim(R)=0$. \tcbline
\begin{proof}
Let $R$ be Artinian. In Rings and Modules, the Akizuki-Hopkins-Levitzki theorem proves that $R$ is Noetherian. Moreover, lmm8.1.4 shows that $\dim(R)=0$. \\~\\

Now let $R$ be Notherian and $\dim(R)=0$. This means that every prime ideal of $R$ is maximal. Let $S$ be the set of all ideals of $R$ that admit a composition series. I claim that $S$ is non-empty. Let $T=\{\text{Ann}(x)\;|\;0\neq x\in R\}$. Clearly $T$ is non-empty. Let $Y_1\subseteq Y_2\subseteq\cdots$ be a chain in $T$. Since $R$ is Noetherian, the chain terminates at finitely many sets with $Y=\text{Ann}(x)\subseteq R$ for some $x\in R$. I claim that $Y$ is a prime ideal. By definition $R=\text{Ann}(0)\notin T$ hence $R\notin T$. This means that $Y\neq R$. Let $ab\in Y=\text{Ann}(x)$. Suppose that $b\notin Y$. We know that $abx=0$ so $a\in\text{Ann}(bx)$. Since $bx\neq 0$, we have $\text{Ann}(bx)\in T$. Since $R$ is commutative, we also have that $\text{Ann}(x)\subseteq\text{Ann}(bx)$. Since $\text{Ann}(x)$ is maximal, we have that $\text{Ann}(x)=\text{Ann}(bx)$. Hence $a\in\text{Ann}(x)$. Thus $\text{Ann}(x)$ is prime. Since $\dim(R)=0$ we have $\text{Ann}(x)$ is a maximal ideal. $R/\text{Ann}(x)$ is a field (and hence a simple $R$-module). The multiplication map $r\mapsto rx$ has kernel $\text{Ann}(x)$. Hence the induced map $R/\text{Ann}(x)\to R$ is injective, and we can consider $R/\text{Ann}(x)$ as a subring of $R$. Together with the fact that it is a simple $R$-module makes it an $R$-submodule with composition series length of $1$. Hence $S$ is non-empty. \\~\\

Let $N_1\subseteq N_2\subseteq\cdots$ be a chain in $S$. Since $R$ is Noetherian, the chain terminates with some ideal $I\in S$. If $I=R$, then $R$ has a composition series. If $I\neq R$, then $R/I$ is non-zero. Choose a prime ideal $P$ of $R$ such that $I\subseteq P\neq R$ (this always exists since we can choose maximal ideals). Then we have $0\neq R/P\subseteq R/I$. Let $p:R\to R/I$ be the projection map. Let $T=p^{-1}(R/P)$. Then we have that $N\subset T\subseteq M$ and $T/N\cong R/P$. Since $\dim(R)=0$, $P$ is maximal hence $R/P$ is a field (and a simple $R$-module). This proves that $T\in S$. But this contradicts the maximality of $N$. Hence $N=R\in T$. Thus $R$ has a composition series. From Rings and Modules we know that this implies $R$ is Noetherian. Hence we conclude. 
\end{proof}
\end{prp}

Recall from Rings and Modules that we have seen that Artinian rings have finitely many maximal ideals. 

\begin{thm}{Structure Theorem for Commutative Artinian Rings}{} Let $R$ be an Artinian commutative ring. Then $R$ decomposes into a direct product of Artinian local rings $$R\cong\bigoplus_{i=1}^k R_i$$ Moreover, the decomposition is unique up to reordering of the direct product.  \tcbline
\begin{proof}
Let $m_1,\dots,m_k$ be the full list of distinct maximal ideals of $R$. Then $$\prod_{i=1}^km_i^n=0$$ for some $n\in\N\setminus\{0\}$. The ideals $m_i^n$ and $m_j^n$ are pairwise coprime for $i\neq j$. Hence by the Chinese Remainder Theorem we obtain ring isomorphisms 
\begin{align*}
R&\cong\frac{R}{0}\\
&\cong\frac{R}{\prod_{i=1}^km_i^n}\\
&\cong\frac{R}{\bigcap_{i=1}^km_i^n}\tag{$m_i^n$ and $m_j^n$ pairwise coprime}\\
&\cong\bigoplus_{i=1}^k\frac{R}{m_i^n}\tag{CRT}
\end{align*}
By the correspondence of maximal ideals, $R/m_i^n$ has a unique maximal ideal $m_i/m_i^n$. Hence it is local. Also since $R$ is Artinian, $R/m_i^n$ is Artinian. Thus we are done. 
\end{proof}
\end{thm}

\pagebreak
\section{Valuation and Valuation Rings}
\subsection{Valuation Rings}
\begin{defn}{Valuation Rings}{} Let $R$ be an integral domain. We say that $R$ is a valuation ring if for all $x\in\text{Frac}(R)$ and $x\neq 0$, then either $x$ or $x^{-1}$ is in $R$. 
\end{defn}

\begin{lmm}{}{} Let $R$ be a valuation ring. Then the following are true. 
\begin{itemize}
\item $R$ is a local ring. 
\item $R$ is normal. 
\end{itemize} \tcbline
\begin{proof}
Let $R$ be a valuation ring. The set of units of $R$ are precisely $S=\{x\in\text{Frac}(R)\;|\;x\in R\text{ and }x^{-1}\in R\}$. Let $m=R\setminus S$. Let $x\in m$ and $r\in R$. Then $rx$ is not a unit because if $arx=1$, then $ar\in R$ is an inverse of $x$, which is a contradiction since $x\in S$. Hence $rx\in R$. \\~\\

Let $x,y\in m$. If one of them are zero then their sum lies in $m$. If both are not zero, then $xy^{-1}$ is an element of $\text{Frac}(R)$. Since $R$ is a valuation ring, either $xy^{-1}$ or $yx^{-1}$ is in $R$. In either case, we have $$x+y=y(y^{-1}x+1)=x(1+x^{-1}y)\in m$$ (one factor is in $m$ and the other in $R$). Hence $m$ is an ideal. By prp2.1.3 we conclude that $R$ is a local ring with unique maximal ideal $m$. \\~\\

Let $x\in\text{Frac}(R)$ be integral over $R$. Then $$x^n+r_{n-1}x^{n-1}+\dots+r_1x+r_0=0$$ for some $r_0,\dots,r_{n-1}\in R$. If $x\in R$ then we are done. If $x\notin R$ then since $R$ is a valuation ring, $x^{-1}\in R$. Then $$x=-(r_1+r_2x^{-1}+\dots+r_nx^{1-n})\in R$$ so that $R$ is normal. 
\end{proof}
\end{lmm}

\subsection{Valuations on a Field}
\begin{defn}{Totally Ordered Group}{} Let $G$ be an abelian group. We say that $G$ is a totally ordered group if there is a total order "$\leq$" on $G$ such that $a\leq b$ implies $ca\leq cb$ for all $a,b,c\in G$. 
\end{defn}

\begin{defn}{Valuation on a Field}{} Let $K$ be a field. Let $G$ be a totally ordered abelian group. A valuation on $K$ with values in $G$ is a group homomorphism $v:K^\times\to G$ such that for all $x,y\in K^\ast$, we have 
\begin{itemize}
\item $v(xy)=v(x)+v(y)$
\item $v(x+y)\geq\min\{v(x),v(y)\}$
\end{itemize}
We use the convention that $v(0)=\infty$. 
\end{defn}

\begin{defn}{Associated Valuation Ring}{} Let $K$ be a field and $v:K\to\Z$ a discrete valuation. Define the associated valuation ring of $K$ to be the subring $$R_v=\{x\in K\;|\;v(x)\geq 0\}$$
\end{defn}

\begin{lmm}{}{} Let $K$ be a field. Let $v$ be a discrete valuation on $K$. Then $R_v$ is a valuation ring. 
\end{lmm}

\subsection{Discrete Valuations and Normalizations}
\begin{defn}{Discrete Valuations}{} Let $K$ be a field. A discrete valuation on $K$ is a valuation $v:K^\times\to\Z$. 
\end{defn}

\begin{defn}{Normalized Discrete Valuations}{} Let $(K,v)$ be a discrete valuation ring. We say that it is normalized if $v$ is surjective. 
\end{defn}

\begin{lmm}{}{} Let $K$ be a field with a discrete valuation $v$. Then $v(K^\times)=n\Z$ for some $n\in\N$. 
\end{lmm}

\begin{lmm}{Normalization of a Discrete Valuation}{} Let $K$ be a field with a discrete valuation $v$ such that $v(K^\times)=n\Z$ for some $n\in\N$. Define the normalization of $v$ to be the valuation $v_N:K^\times\to\Z$ defined by $$v_N(k)=\frac{1}{n}v(k)$$ for all $k\in K^\times$. 
\end{lmm}

Therefore we always work on normalized discrete valuation rings. 

\begin{defn}{Discrete Valuation Rings}{} Let $R$ be a commutative ring. We say that $R$ is a discrete valuation ring if there exists a field $K$ and a discrete valuation $v$ on $K$ such that $$R=R_v$$ is the associated valuation ring of $K$. 
\end{defn}

\begin{prp}{}{} Let $R$ be a discrete valuation ring with valuation $v$. Let $t\in R$ be such that $v(t)=1$. Then the following are true. 
\begin{itemize}
\item A nonzero element $u\in R$ is a unit if and only if $v(u)=0$
\item $\dim(R)=1$
\end{itemize}\tcbline
\begin{proof}~\\
\begin{itemize}
\item Let $R$ be a discrete valuation ring. Suppose that $x\in R$ is a unit. Then $v(x^{-1})=-v(x)$. Then $-v(x),v(x)\geq 0$ implies $v(x)=0$. Now if $v(y)>0$ , suppose for contradiction that $u\in R$ is an inverse of $y$, then $$0=v(1)=v(uy)=v(u)+v(y)$$ But $v(y)>0$ implies that $v(u)<0$ which implies that $u\notin R$, a contradiction. 
\end{itemize}
\end{proof}
\end{prp}

\subsection{Uniformizing Parameters}
\begin{defn}{Uniformizing Parameter}{} Let $R$ be a discrete valuation ring with valuation $v$. A uniformizing parameter for $R$ is an element $t\in R$ such that $v(t)=1$. 
\end{defn}

\begin{prp}{}{} Let $R$ be a discrete valuation ring with valuation $v$. Let $t\in R$ be a uniformizing parameter of $R$. Then the following are true. 
\begin{itemize}
\item Every non-zero ideal of $R$ is a principal ideal of the form $(t^n)$ for some $n\geq 0$
\item Every $r\in R\setminus\{0\}$ can be written in the form $r=ut^n$ for some unit $u$ and $n\geq 0$. 
\end{itemize}\tcbline
\begin{proof}~\\
\begin{itemize}
\item Let $t\in R$ such that $v(t)=1$. Let $x\in m$ where $v(x)=n>0$. Then $v(x)=nv(t)=v(t^n)$ means that every $x\in m$ is of the form $t^n$. Thus $m=(t)$. Since every ideal $I$ is a subset of this maximal ideal, any ideal is of the form $I=(t^n)$ for some $n>0$. 
\item Follows from the fact that $(t^n)$ is the unique maximal ideal. 
\end{itemize}
\end{proof}
\end{prp}

\begin{prp}{}{} Let $R$ be a discrete valuation ring. Let $t$ be a uniformizing parameter of $R$. Let $u$ be a unit of $R$. Then $$v(ut^n)=n$$ for all $n\in\N$. 
\end{prp}

\subsection{Recognizing Discrete Valuation Rings}
The rest of the section devotes efforts to recognizing discrete valuation rings. 

\begin{prp}{}{} Let $R$ be a valuation ring. Then the following are equivalent. 
\begin{itemize}
\item $R$ is a discrete valuation ring. 
\item $R$ is a principal ideal domain. 
\item $R$ is Noetherian. 
\end{itemize}
\end{prp}

\begin{prp}{Equivalent Characterizations of DVRs I}{} Let $R$ be an integral domain. Then the following are equivalent. 
\begin{itemize}
\item $R$ is a discrete valuation ring
\item $R$ is Noetherian, local, $\dim(R)=1$ and normal. 
\item $R$ is local, a PID and not a field. 
\item $R$ is a UFD with a unique irreducible element up to multiplication of a unit
\end{itemize} \tcbline
\begin{proof}~\\
\begin{itemize}
\item $(1)\implies(3)$: We have seen that the set of non-units is precisely the set $m=\{x\in K|v(x)>0\}$. We show that this is an ideal. Clearly $x,y\in m$ implies $v(x+y)=\min\{v(x),v(y)\}>0$. Let $u\in R$. Then $v(ux)=v(u)+v(x)>0$ since $v(x)>0$ and $v(u)\geq 0$. \\~\\
We have seen that every ideal is of the form $(t^n)$ for some $n>0$. Thus every ascending chains of ideal must be of the form $$(t^{n_1})\subset(t^{n_2})\subset\dots$$ for $n_1>n_2>\dots$. Since $n_1,n_2,\dots$ is strictly decreasing, the chain must eventually stabilizes. This proves that $R$ is Noetherian and has principal maximal ideal. 
\item $(1)\implies(3)$:
\end{itemize}
\end{proof}
\end{prp}

\begin{prp}{Equivalent Characterizations of DVRs II}{} Let $R$ be an integral domain that is Noetherian and local with unique maximal ideal $m$. Then the following are equivalent. 
\begin{itemize}
\item $R$ is a discrete valuation ring. 
\item $\dim(R)=1$ and $R$ is normal. 
\item $\dim(R)=1$ and $\dim_{R/m}(m/m^2)=1$ ($R$ is a regular local ring)
\item $R$ is not a field and $m$ is principal. 
\item $I=m^k$ for all non-zero ideals $I$ of $R$
\item There exists $t\in R$ and $k>0$ such that $I=(t^k)$ for all non-zero ideal $I$ of $R$
\end{itemize} \tcbline
\begin{proof}
\end{proof}
\end{prp}

\begin{prp}{}{} Let $R$ be a Noetherian integral domain and $\dim(R)=1$. Then $R$ is normal if and only if $R_m$ is a discrete valuation ring for all maximal ideals $m$. 
\end{prp}

In summary, if $R$ is a discrete valuation ring, then $R$ has the following properties. 
\begin{itemize}
\item $R$ is integrally closed and in particular is normal. 
\item $R$ is a PID and in particular is a UFD and an integral domain. 
\item $R$ is Noetherian and local
\item $R$ has Krull dimension $1$. 
\item $\dim_{R/m}(m/m^2)=1$ (these are called regular local rings as we will see in Commutative Algebra $2$)
\item Every ideal $I$ of $R$ is equal to the power $m^k$ of the maximal ideal $m$. In particular if $m$ is generated by the uniformizing parameter $t$, then $=I=(t^k)$ in this case. 
\item Such a $t$ is an irreducible element (that is unique up to multiplication by a unit), and every element of $R$ can be written as $ut^n$ for $u$ a unit and $n\in\N$. 
\end{itemize}

There is a simple diagram of relationships between DVRs and some other standard types of commutative rings. 

\adjustbox{scale=1.0,center}{\begin{tikzcd}
	{\text{DVRs}} & {\text{PIDs}} & {\text{UFDs}} & {\text{Normal Domains}} & {\text{Integral Domains}}
	\arrow[from=1-1, to=1-2, phantom, sloped, "\subset"]
	\arrow[from=1-2, to=1-3, phantom, sloped, "\subset"]
	\arrow[from=1-3, to=1-4, phantom, sloped, "\subset"]
	\arrow[from=1-4, to=1-5, phantom, sloped, "\subset"]
\end{tikzcd}}

\pagebreak

\section{Dedekind Domains}
\subsection{Fractional Ideals}
\begin{defn}{Fractional Ideal}{} Let $R$ be an integral domain. Let $I$ be a $R$-submodule of $\text{Frac}(R)$. We say that $I$ is a fractional ideal of $R$ if there exists $r\in R\setminus\{0\}$ such that $rI\subseteq R$. 
\end{defn}

While $I$ is not exactly an ideal of $R$, we can think of it as if it were an ideal because it is isomorphic to an actual ideal of $R$. 

\begin{lmm}{}{} Let $R$ be an integral domain. Let $I$ be a fractional ideal of $R$ where $rI\subseteq R$ for some $r\in R\setminus\{0\}$. Then there is an $R$-module isomorphism $$I\cong rI\subseteq R$$ given by $i\mapsto ri$. \tcbline
\begin{proof}
I claim that there is an $R$-module isomorphism $I\cong rI$ for $rI\subseteq R$ given by $i\mapsto ri$. The kernel of this $R$-module homomorphism is given by $\{i\in I\;|\;ri=0\}$. But $ri=0$ if and only if $r=0$ or $i=0$. Since $r\neq 0$ we must have $i=0$ so that the kernel is trivial. Moreover, this $R$-module homomorphism is surjective since for any $k\in rI$ it can be written as $k=ri$ for some $i$. Then $i\in I$ maps to $ri$ under the morphism. Hence $I\cong rI$ as $R$-modules. 
\end{proof}
\end{lmm}

\begin{lmm}{}{} Let $R$ be an integral domain. Let $I$ be a fractional ideal of $R$. If $R$ is Noetherian, then $I$ is finitely generated. \tcbline
\begin{proof}
Let $R$ be Noetherian. Since $I$ is isomorphic to $rI$ for some non-zero $r\in R$, and $rI$ is an ideal of $R$, $R$ being Noetherian implies that $rI$ is finitely generated and hence $I$ is finitely generated. 
\end{proof}
\end{lmm}

\subsection{Invertible Ideals}
\begin{defn}{Invertible Ideals}{} Let $R$ be an integral domain. Let $I$ be an $R$-submodule of $\text{Frac}(R)$. We say that $I$ is invertible if there exists an ideal $J$ of $R$ such that $JI=R$. 
\end{defn}

\begin{lmm}{}{} Let $R$ be an integral domain. Let $I$ be an $R$-submodule of $\text{Frac}(R)$. Then $I$ is invertible if and only if $I^{-1}I=R$ where we define $$I^{-1}=\{s\in\text{Frac}(R)\;|\;sI\subseteq R\}$$
\end{lmm}

\begin{prp}{}{} Let $R$ be an integral domain. Let $I$ be an $R$-submodule of $\text{Frac}(R)$. Then the following are true. 
\begin{itemize}
\item If $I$ is a non-zero principal ideal of $R$, then $I$ is invertible. 
\item If $I$ is invertible, then $I$ is fractional. 
\end{itemize}
\end{prp}

\begin{prp}{}{} Let $R$ be an integral domain. Let $I$ be a fractional ideal. Then $I$ is invertible if and only if $I$ is finitely generated, and for any maximal ideal $m$ of $R$, $IR_m$ is a principal ideal of $R_m$. 
\end{prp}

\begin{prp}{}{} Let $R$ be an integral domain. Let $P$ be a non-zero prime ideal of $R$. If $R$ is Noetherian and $P$ is invertible, then $R_P$ is a discrete valuation ring. \tcbline
\begin{proof}
Let $R$ be a Noetherian integral domain and $P$ a non-zero invertible prime ideal. We know that $PR_P$ is the unique maximal ideal of the local ring $R_P$. By the above prp, $PR_P$ is a principal ideal. Thus $R_P$ is now a Noetherian local ring with principal maximal ideal. By prp10.4.6 in Commutative Algebra 1, we conclude that $R_P$ is a discrete valuation ring. 
\end{proof}
\end{prp}

\subsection{Dedekind Domains}
\begin{defn}{Dedekind Domains}{} Let $R$ be an integral domain. We say that $R$ is a dedekind domain if every non-zero ideal can be expressed uniquely as a direct product of finitely many prime ideals of $R$. 
\end{defn}

Dedekind sought for an integral domain whose ideals can be factorized uniquely as a product of primes. 

\begin{prp}{}{} Let $R$ be an integral domain that is not a field. Then the following are equivalent. 
\begin{itemize}
\item $R$ is a Dedekind domain. 
\item Every non-zero fractional ideal $I$ of $R$ is invertible ($I^{-1}I=R$). 
\item $R$ is Noetherian, $\dim(R)=1$ and normal
\item $R$ is Noetherian, $\dim(R)=1$ and for any non-zero maximal ideal $m$ of $R$, $R_m$ is a discrete valuation ring. 
\item $R$ is Noetherian, $\dim(R)=1$ and every primary ideal in $R$ is a prime power. 
\end{itemize} \tcbline
\begin{proof}~\\
\begin{itemize}
\item $(2)\implies(3)$: Let $I$ be an ideal of $R$. Since $I$ is invertible, by 1.1.5 we conclude that $I$ is finitely generated. Hence $R$ is Noetherian. Let $P$ be a prime ideal of $R$. By assumption, $P$ is invertible. prp1.2.5 implies that $R_P$ is a DVR. In particular, it is integrally closed and $\dim(R_P)=1$. This means that $\text{ht}_R(P)=1$. Thus $R$ is either a field or $\dim(R)=1$. By assumption $R$ is not a field. Hence $\dim(R)=1$. We know that $R=\bigcap_{m\text{ a maximal ideal}}R_m$. Since prime ideals are maximal ideals in one dimensional rings, we can rewrite the intersection as $$R=\bigcap_{P\text{ a prime ideal}}R_P$$ But each $R_P$ is a DVR. Hence $R$ is a DVR and we conclude that $R$ is normal. 

\item $(3)\implies(2)$:  $m$ be a maximal ideal of $R$. We have seen from Commutative Algebra 1 that $R_m$ is a Noetherian local ring. By 7.4.2 in Commutative Algebra 1 we also conclude that $R_m$ is normal. By 9.3.2 of Commutative Algebra 1 we know that $\dim(R_m)=\text{ht}_R(m)=1$. By 10.4.6 of Commutative Algebra 1, $R_m$ is a DVR and in particular $m$ is a principal ideal. \\~\\

Let $I$ be a fractional ideal of $R$. We know by 1.1.3 that $I$ is finitely generated. Since $R_m$ is a normal Noetherian local ring of dimension $1$, the ideal $I_m$ of $R_m$ must be principal. By 1.1.5 we conclude that $I$ is invertible. \\~\\

\item $(3)\implies(4)$: 

\item $(4)\implies(3)$: Let $m$ be a maximal ideal of $R$. We know that $R_m$ is a DVR. In particular, it is a normal domain. 
\end{itemize}
\end{proof}
\end{prp}

By virtue of the fourth item, we can think of Dedekind domains as a patching up of local discrete valuation rings. 

\begin{prp}{}{} Let $R$ be a Dedekind domain. Let $I$ and $J$ be ideals of $R$ whose prime factorization is given by $$I=P_1^{a_1}\times\cdots\times P_n^{a_n}\;\;\;\;\text{ and }\;\;\;\;J=P_1^{b_1}\times\cdots\times P_n^{b_n}$$ for $P_1,\dots,P_n$ distinct prime ideals of $R$. Then the following are true. 
\begin{itemize}
\item $I+J=P_1^{\min\{a_1,b_1\}}\times\cdots\times P_n^{\min\{a_n,b_n\}}$
\item $I\cap J=P_1^{\max\{a_1,b_1\}}\times\cdots\times P_n^{\max\{a_n,b_n\}}$
\item $IJ=P_1^{a_1+b_1}\times\cdots\times P_n^{a_n+b_n}$
\end{itemize}
\end{prp}

\begin{prp}{}{} Let $R$ be a Dedekind domain. Let $I$ be an ideal of $R$. Then the following are true. 
\begin{itemize}
\item For any $a\in I$, there exists $b\in R$ such that $I=(a,b)$. 
\item $I$ is can be finitely generated by two elements. 
\end{itemize}
\end{prp}



We summarize the relation between Dedekind domains and other types of domains in the following diagram: 

\adjustbox{scale=1.0,center}{\begin{tikzcd}
	&& {\substack{\text{Dedekind}\\\text{Domains}}} \\
	{\text{DVRs}} & {\text{PIDs}} && {\text{Normal Domains}} & {\text{Integral Domains}} \\
	&& {\text{UFDs}}
	\arrow[from=2-1, to=2-2, phantom, sloped, "\subset"]
	\arrow[from=2-2, to=1-3, phantom, sloped, "\subset"]
	\arrow[from=2-2, to=3-3, phantom, sloped, "\subset"]
	\arrow[from=1-3, to=2-4, phantom, sloped, "\subset"]
	\arrow[from=3-3, to=2-4, phantom, sloped, "\subset"]
	\arrow[from=2-4, to=2-5, phantom, sloped, "\subset"]
\end{tikzcd}} \\~\\

In particular, DVRs, PIDs and Dedekind domains are $1$-dimensional. Moreover, notice that the only difference between DVRs and Dedekind domains is that DVRs are local rings. They both share the fact that they are Noetherian, $\dim(R)=1$ and normal. 



\end{document}