\documentclass[a4paper]{article}

%=========================================
% Packages
%=========================================
\usepackage{mathtools}
\usepackage{amsfonts}
\usepackage{amsmath}
\usepackage{amssymb}
\usepackage{amsthm}
\usepackage[a4paper, total={6in, 8in}, margin=1in]{geometry}
\usepackage[utf8]{inputenc}
\usepackage{fancyhdr}
\usepackage[utf8]{inputenc}
\usepackage{graphicx}
\usepackage{physics}
\usepackage[listings]{tcolorbox}
\usepackage{hyperref}
\usepackage{tikz-cd}
\usepackage{adjustbox}
\usepackage{enumitem}
\usepackage[font=small,labelfont=bf]{caption}
\usepackage{subcaption}
\usepackage{wrapfig}
\usepackage{makecell}



\raggedright

\usetikzlibrary{arrows.meta}

\DeclarePairedDelimiter\ceil{\lceil}{\rceil}
\DeclarePairedDelimiter\floor{\lfloor}{\rfloor}

%=========================================
% Fonts
%=========================================
\usepackage{tgpagella}
\usepackage[T1]{fontenc}


%=========================================
% Custom Math Operators
%=========================================
\DeclareMathOperator{\adj}{adj}
\DeclareMathOperator{\im}{im}
\DeclareMathOperator{\nullity}{nullity}
\DeclareMathOperator{\sign}{sign}
\DeclareMathOperator{\dom}{dom}
\DeclareMathOperator{\lcm}{lcm}
\DeclareMathOperator{\ran}{ran}
\DeclareMathOperator{\ext}{Ext}
\DeclareMathOperator{\dist}{dist}
\DeclareMathOperator{\diam}{diam}
\DeclareMathOperator{\aut}{Aut}
\DeclareMathOperator{\inn}{Inn}
\DeclareMathOperator{\syl}{Syl}
\DeclareMathOperator{\edo}{End}
\DeclareMathOperator{\cov}{Cov}
\DeclareMathOperator{\vari}{Var}
\DeclareMathOperator{\cha}{char}
\DeclareMathOperator{\Span}{span}
\DeclareMathOperator{\ord}{ord}
\DeclareMathOperator{\res}{res}
\DeclareMathOperator{\Hom}{Hom}
\DeclareMathOperator{\Mor}{Mor}
\DeclareMathOperator{\coker}{coker}
\DeclareMathOperator{\Obj}{Obj}
\DeclareMathOperator{\id}{id}
\DeclareMathOperator{\GL}{GL}
\DeclareMathOperator*{\colim}{colim}

%=========================================
% Custom Commands (Shortcuts)
%=========================================
\newcommand{\CP}{\mathbb{CP}}
\newcommand{\GG}{\mathbb{G}}
\newcommand{\F}{\mathbb{F}}
\newcommand{\N}{\mathbb{N}}
\newcommand{\Q}{\mathbb{Q}}
\newcommand{\R}{\mathbb{R}}
\newcommand{\C}{\mathbb{C}}
\newcommand{\E}{\mathbb{E}}
\newcommand{\Prj}{\mathbb{P}}
\newcommand{\RP}{\mathbb{RP}}
\newcommand{\T}{\mathbb{T}}
\newcommand{\Z}{\mathbb{Z}}
\newcommand{\A}{\mathbb{A}}
\renewcommand{\H}{\mathbb{H}}
\newcommand{\K}{\mathbb{K}}

\newcommand{\mA}{\mathcal{A}}
\newcommand{\mB}{\mathcal{B}}
\newcommand{\mC}{\mathcal{C}}
\newcommand{\mD}{\mathcal{D}}
\newcommand{\mE}{\mathcal{E}}
\newcommand{\mF}{\mathcal{F}}
\newcommand{\mG}{\mathcal{G}}
\newcommand{\mH}{\mathcal{H}}
\newcommand{\mI}{\mathcal{I}}
\newcommand{\mJ}{\mathcal{J}}
\newcommand{\mK}{\mathcal{K}}
\newcommand{\mL}{\mathcal{L}}
\newcommand{\mM}{\mathcal{M}}
\newcommand{\mO}{\mathcal{O}}
\newcommand{\mP}{\mathcal{P}}
\newcommand{\mS}{\mathcal{S}}
\newcommand{\mT}{\mathcal{T}}
\newcommand{\mV}{\mathcal{V}}
\newcommand{\mW}{\mathcal{W}}

%=========================================
% Colours!!!
%=========================================
\definecolor{LightBlue}{HTML}{2D64A6}
\definecolor{ForestGreen}{HTML}{4BA150}
\definecolor{DarkBlue}{HTML}{000080}
\definecolor{LightPurple}{HTML}{cc99ff}
\definecolor{LightOrange}{HTML}{ffc34d}
\definecolor{Buff}{HTML}{DDAE7E}
\definecolor{Sunset}{HTML}{F2C57C}
\definecolor{Wenge}{HTML}{584B53}
\definecolor{Coolgray}{HTML}{9098CB}
\definecolor{Lavender}{HTML}{D6E3F8}
\definecolor{Glaucous}{HTML}{828BC4}
\definecolor{Mauve}{HTML}{C7A8F0}
\definecolor{Darkred}{HTML}{880808}
\definecolor{Beaver}{HTML}{9A8873}
\definecolor{UltraViolet}{HTML}{52489C}



%=========================================
% Theorem Environment
%=========================================
\tcbuselibrary{listings, theorems, breakable, skins}

\newtcbtheorem[number within = subsection]{thm}{Theorem}%
{	colback=Buff!3, 
	colframe=Buff, 
	fonttitle=\bfseries, 
	breakable, 
	enhanced jigsaw, 
	halign=left
}{thm}

\newtcbtheorem[number within=subsection, use counter from=thm]{defn}{Definition}%
{  colback=cyan!1,
    colframe=cyan!50!black,
	fonttitle=\bfseries, breakable, 
	enhanced jigsaw, 
	halign=left
}{defn}

\newtcbtheorem[number within=subsection, use counter from=thm]{axm}{Axiom}%
{	colback=red!5, 
	colframe=Darkred, 
	fonttitle=\bfseries, 
	breakable, 
	enhanced jigsaw, 
	halign=left
}{axm}

\newtcbtheorem[number within=subsection, use counter from=thm]{prp}{Proposition}%
{	colback=LightBlue!3, 
	colframe=Glaucous, 
	fonttitle=\bfseries, 
	breakable, 
	enhanced jigsaw, 
	halign=left
}{prp}

\newtcbtheorem[number within=subsection, use counter from=thm]{lmm}{Lemma}%
{	colback=LightBlue!3, 
	colframe=LightBlue!60, 
	fonttitle=\bfseries, 
	breakable, 
	enhanced jigsaw, 
	halign=left
}{lmm}

\newtcbtheorem[number within=subsection, use counter from=thm]{crl}{Corollary}%
{	colback=LightBlue!3, 
	colframe=LightBlue!60, 
	fonttitle=\bfseries, 
	breakable, 
	enhanced jigsaw, 
	halign=left
}{crl}

\newtcbtheorem[number within=subsection, use counter from=thm]{eg}{Example}%
{	colback=Beaver!5, 
	colframe=Beaver, 
	fonttitle=\bfseries, 
	breakable, 
	enhanced jigsaw, 
	halign=left
}{eg}

\newtcbtheorem[number within=subsection, use counter from=thm]{ex}{Exercise}%
{	colback=Beaver!5, 
	colframe=Beaver, 
	fonttitle=\bfseries, 
	breakable, 
	enhanced jigsaw, 
	halign=left
}{ex}

\newtcbtheorem[number within=subsection, use counter from=thm]{alg}{Algorithm}%
{	colback=UltraViolet!5, 
	colframe=UltraViolet, 
	fonttitle=\bfseries, 
	breakable, 
	enhanced jigsaw, 
	halign=left
}{alg}




%=========================================
% Hyperlinks
%=========================================
\hypersetup{
    colorlinks=true, %set true if you want colored links
    linktoc=all,     %set to all if you want both sections and subsections linked
    linkcolor=DarkBlue,  %choose some color if you want links to stand out
}


\pagestyle{fancy}
\fancyhf{}
\rhead{Labix}
\lhead{Commutative Algebra 1}
\rfoot{\thepage}

\title{Commutative Algebra 1}

\author{Labix}

\date{\today}
\begin{document}
\maketitle
\begin{abstract}
\end{abstract}
\pagebreak
\tableofcontents
\pagebreak

\pagebreak
\section{Basic Notions of Rings}
\subsection{Local Rings}
\begin{defn}{Local Rings}{} A ring $R$ is said to be a local ring if it has a unique maximal ideal $m$. In this case, we say that $R/m$ is the residue field of $R$. 
\end{defn}

We will discuss more of local rings in the topic of localizations. For now, here is one example. Let $k$ be a field. It is easy to see that $(x)$ is the unique maximal ideal of $R=k[[x]]$. The residue field in this case is simply $k$. 

\begin{prp}{}{} Let $R$ be a ring and $I$ an ideal of $R$. Then $I$ is the unique maximal ideal of $R$ if and only if $I$ is the set containing all non-units of $R$. \tcbline
\begin{proof}
Let $I$ be the unique maximal ideal of $R$. Clearly $I$ does not contain any unit else $I=R$. Now suppose that $r$ is a non-unit. Suppose that $r\notin I$. Define $J=\{sr|s\in R\}$ Clearly $J$ is an ideal. It must be contained in some maximal ideal. Since $I$ is the unique maximal ideal, $J\subseteq I$. But this means that $r\in I$, a contradiction. Thus every non-unit is in $I$. \\~\\
Suppose that $I$ contains all non-units of $R$. Let $r\notin I$. Then there exists $s\notin I$ such that $rs=1$. Then $(r+I)(s+I)=1+I$ in $R/I$. This means that every element of $R/I$ has a multiplicative inverse which means that $R/I$ is a field and thus $I$ is a maximal ideal. Now let $J\neq I$ be another maximal ideal. Then $J$ contains some unit $r$. This implies that $J=R$ and thus $I$ is the unique maximal ideal. 
\end{proof}
\end{prp}

\subsection{Radical Ideals}
The radical of an ideal is a very different notion from the radical of module. 

\begin{defn}{Radical of an Ideal}{} Let $I$ be an ideal of a ring $R$. Define the radical of $I$ to be $$\sqrt{I}=\{r\in R|r^n\in I\text{ for some }n\in\N\}$$
\end{defn}

\begin{prp}{}{} Let $R$ be a commutative ring. Let $I$ be an ideal. Then $$\sqrt{I}=\bigcap_{\substack{p\text{ a prime ideal}\\I\subseteq p\subseteq R}}p$$
\end{prp}

\begin{crl}{}{} Let $R$ be a commutative ring. Let $I$ be an ideal. Then we have that $$I\subseteq\sqrt{I}$$ \tcbline
\begin{proof}
By the above proposition, $\sqrt{I}$ is the intersection of all prime ideals $P$ that contain $I$. Since each $P$ is such that $I\subseteq P$, we have that $$I=\bigcap_{\substack{p\text{ a prime ideal}\\I\subseteq p\subseteq R}}I\subseteq\bigcap_{\substack{p\text{ a prime ideal}\\I\subseteq p\subseteq R}}p=\sqrt{I}$$ and so we conclude. 
\end{proof}
\end{crl}

\begin{crl}{}{} Let $R$ be a commutative ring. Let $I$ be an ideal. Then we have that $$\sqrt{\sqrt{I}}=\sqrt{I}$$ \tcbline
\begin{proof}
By the above, we already know that $\sqrt{I}\subseteq\sqrt{\sqrt{I}}$. So let $r\in\sqrt{\sqrt{I}}$. Then there exists some $n\in\N$ such that $r^n\in\sqrt{I}$. But $r^n\in\sqrt{I}$ means that there exists some $m\in\N$ such that $(r^n)^m\in I$. But $nm\in\N$ is a natural number such that $r^{nm}\in I$. Hence $r\in\sqrt{I}$ and so we conclude. 
\end{proof}
\end{crl}

\begin{defn}{Radical Ideals}{} Let $R$ be a commutative ring. Let $I$ be an ideal of $R$. We say that $I$ is radical if $$\sqrt{I}=I$$
\end{defn}

In particular, by the above lemma it follows that the radical of an ideal is a radical ideal. 

\begin{thm}{}{} Let $R$ be a commutative ring. Let $I$ be an ideal of $R$. Denote $\varphi$ to be the inclusion preserving one-to-one bijection $$\left\{\substack{\text{Ideals of }R\\\text{containing }I}\right\}\;\;\overset{1:1}{\longleftrightarrow}\;\;\left\{\text{Ideals of }R/I\right\}$$ from the correspondence theorem for rings. In other words, $\varphi(A)=A/I$. Let $J\subseteq R$ be an ideal containing $I$. Then the following are true. 
\begin{itemize}
\item $J$ is a radical ideal if and only if $\varphi(J)=J/I$ is a radical ideal. 
\item $J$ is a prime ideal if and only if $\varphi(J)=J/I$ is a prime ideal. 
\item $J$ is a maximal ideal if and only if $\varphi(J)=J/I$ is a maximal ideal. 
\end{itemize} \tcbline
\begin{proof}~\\
\begin{itemize}
\item Let $J$ be a radical ideal. Suppose that $r+I\in\sqrt{J/I}$. This means that $(r+I)^n=r^n+I\in J/I$ for some $n\in\N$. But this means that $r^n\in J$. This implies that $r\in\sqrt{J}=J$. Thus $r+I\in J/I$ and we conclude that $\sqrt{J/I}\subseteq J/I$. Since we also have $J/I\subseteq\sqrt{J/I}$, we conclude. \\~\\

Now suppose that $J/I$ is a radical ideal. Let $r\in\sqrt{J}$. This means that $r^n\in J$ for some $n\in\N$. Now $r^n+I=(r+I)^n\in J/I$ implies that $r+I\in\sqrt{J/I}=J/I$. Hence $r\in J$ and so $\sqrt{J}\subseteq J$. Since we also have that $J\subseteq\sqrt{J}$, we conclude. 

\item Let $J$ be a prime ideal. Then $R/J$ is an integral domain. By the second isomorphism theorem, we have that $R/J\cong(R/I)/(J/I)$ and hence $(R/I)/(J/I)$ is also an integral domain. Hence $J/I$ is a prime ideal. The converse is also true. 

\item Let $J$ be a maximal ideal. Then $R/J$ is a field. By the second isomorphism theorem, we have that $R/J\cong(R/I)/(J/I)$ and hence $(R/I)/(J/I)$ is also a field. Hence $J/I$ is a maximal ideal. The converse is also true. 
\end{itemize}
\end{proof}
\end{thm}

\subsection{Nilradical and Jacobson Ideals}
Let $R$ be a ring. Recall that an element $r\in R$ is nilpotent if $r^n=0_R$ for some $n\in\N$. When $R$ is commutative, we can form an ideal out of nilpotent elements. 

\begin{defn}{Nilradicals}{} Let $R$ be a ring. Define the nilradical of $R$ to be $$N(R)=\{r\in R\;|\;r\text{ is nilpotent}\}$$
\end{defn}

Note that this is different from nilpotent ideals, as nilpotency is a property of an ideal. However the Nilradical ideal is a nil ideal and every sub-ideal of the nilradical is a nil ideal. 

\begin{prp}{}{} Let $R$ be a ring and $N(R)$ its nilradical. Then the following are true. 
\begin{itemize}
\item $N(R)$ is an ideal of $R$
\item $N(R/N(R))=0$
\end{itemize}\tcbline
\begin{proof}~\\
\begin{itemize}
\item Suppose that $r,s$ are nilpotent, meaning that $r^n=0$ and $s^m=0$. Then $(r+s)^{n+m}=0$. Moreover, if $t\in R$ then $t\cdot r$ is also nilpotent
\item Let $r\notin N(R)$. Every element $r+N(R)\in R/N(R)$ has the property that $r^n\neq 0$. Consider $(r+N(R))^n=r^n+N(R)$. If $r^n\in N(R)$ then $r^n=u$ for some nilpotent $u$, which means that $r^n$ is nilpotent and thus $r$ is nilpotent, a contradiction. This means that $r+N(R)\notin N(R/N(R))$ for all $r\notin N(R)$ and thus $N(R/N(R))=0$
\end{itemize}
\end{proof}
\end{prp}

\begin{prp}{}{} Let $R$ be a commutative ring. The nilradical of $R$ is the intersection of all prime ideals of $R$. \tcbline
\begin{proof}
We want to show that $$N(R)=\bigcap_{\substack{P\text{ a prime}\\\text{ideal of }R}}P$$
Trivially $N(R)$ is a prime ideal. Now suppose that $r\in R$ is in the intersection of all prime ideals. Then $r^n$ also lies in every prime ideal. 
\end{proof}
\end{prp}

\begin{eg}{}{} Consider the ring $$R=\frac{\C[x,y]}{(x^2-y,xy)}$$ Then its nilradical is given by $N(R)=(x,y)$. \tcbline
\begin{proof}
Notice that in the ring $R$, $x^3=x(x^2)=xy=0$ and $y^3=x^6=(x^3)^2=0$ and hence $x$ and $y$ are both nilpotent elements of $R$. By definition of the nilradical, we conclude that $(x,y)\subseteq N(R)$. Now $(x,y)$ is a maximal ideal of $\C[x,y]$ because $\C[x,y]/(x,y)\cong\C$. Also notice that $(x,y)\supseteq(x^2-y,xy)$ because for any element $f(x)(x^2-y)+g(x)(xy)\in(x^2-y,xy)$, we have that 
\begin{align*}
f(x)(x^2-y)+g(x)(xy)\in(x^2-y,xy)&=(xf(x))x-f(x)y+(g(x)x)y\\
&=(xf(x))x+(xg(x)-f(x))y\in (x,y)
\end{align*}
By the correspondence theorem, $(x,y)/(x^2-y)$ is an maximal ideal of $R$. In particular, $(x,y)$ is also a prime ideal. But the $N(R)$ is the intersection of all prime ideals and hence $N(R)\subseteq(x,y)$. We conclude that $N(R)=(x,y)$. 
\end{proof}
\end{eg}

Recall the notion of the Jacobson radical from Rings and Modules. 

\begin{prp}{}{} Let $R$ be a commutative ring. Then $$J(R)=\bigcap_{\substack{m\text{ is a}\\\text{maximal ideal}}}m$$
\end{prp}

\begin{prp}{}{} Let $R$ be a commutative ring. Then $x\in J(R)$ if and only if $1-xy\neq 0$ for all $y\in R$. \tcbline
\begin{proof}
\end{proof}
\end{prp}

\subsection{Extensions and Contractions of Ideals}
\begin{defn}{Extension of Ideals}{} Let $R,S$ be commutative rings. Let $f:R\to S$ be a ring homomorphism. Let $I$ be an ideal of $R$. Define the extension $I^e$ of $I$ to $S$ to be the ideal $$I^e\langle f(i)\;|\;i\in I\rangle$$
\end{defn}

\begin{prp}{}{} Let $R,S$ be commutative rings. Let $f:R\to S$ be a ring homomorphism. Let $I,I_1,I_2$ be an ideal of $R$. Then the following are true regarding the extension of ideals. 
\begin{itemize}
\item Closed under sum: $(I_1+I_2)^e=I_1^e+I_2^e$
\item $(I_1\cap I_2)^e\subseteq I_1^e\cap I_2^e$
\item Closed under products: $(I_1I_2)^e=I_1^eI_2^e$
\item $(I_1/I_2)^e\subseteq I_1^e/I_2^e$
\item $\text{rad}(I)^e\subseteq\text{rad}(I^e)$
\end{itemize}
\end{prp}

\begin{defn}{Contraction of Ideals}{} Let $R,S$ be commutative rings. Let $f:R\to S$ be a ring homomorphism. Let $J$ be an ideal of $S$. Define the contraction $J^c$ of $J$ to $R$ to be the ideal $$J^c=f^{-1}(J)$$
\end{defn}

\begin{prp}{}{} Let $R,S$ be commutative rings. Let $f:R\to S$ be a ring homomorphism. Let $J,J_1,J_2$ be an ideal of $S$. Then the following are true regarding the extension of ideals. 
\begin{itemize}
\item $(J_1+J_2)^e\supseteq J_1^e+J_2^e$
\item Closed under intersections: $(J_1\cap J_2)^e=J_1^e\cap J_2^e$
\item $(J_1J_2)^e\supseteq J_1^eJ_2^e$
\item $(J_1/J_2)^e\subseteq J_1^e/J_2^e$
\item Closed under taking radicals: $\text{rad}(J)^e=\text{rad}(J^e)$
\end{itemize}
\end{prp}

\begin{prp}{}{} Let $R,S$ be commutative rings. Let $f:R\to S$ be a ring homomorphism. Let $I$ be an ideal of $R$ and let $J$ be an ideal of $S$. Then the following are true. 
\begin{itemize}
\item $I\subseteq I^{ec}$
\item $J^{ce}\subseteq J$
\item $I^e=I^{ece}$
\item $J^c=J^{cec}$
\end{itemize}
\end{prp}

\pagebreak
\section{Basic Notions of Modules}
\subsection{Cayley-Hamilton Theorem}
\begin{defn}{Characteristic Polynomial}{} Let $R$ be a commutative ring. Let $A\in M_{n\times n}(R)$ be a matrix. Define the characteristic polynomial of $A$ to be the polynomial $$c_A(x)=\det(A-xI)$$
\end{defn}

\begin{thm}{Cayley-Hamilton Theorem}{} Let $R$ be a commutative ring. Let $A\in M_{n\times n}(R)$ be a matrix. Then $c_A(A)=0$. 
\end{thm}

\begin{crl}{}{} Let $R$ be a commutative ring. Let $M$ be a finitely generated $R$-module. Let $I$ be an ideal of $R$. Let $\varphi\in\text{End}_R(M)$. If $\varphi(M)\subseteq IM$, then there exists $a_1,\dots,a_n\in I$ such that $$\varphi^n+a_1\varphi^{n-1}+\dots+a_{n-1}\varphi+\text{id}_M=0:M\to M$$
\end{crl}

\subsection{Nakayama's Lemma}
\begin{lmm}{Nakayama's Lemma I}{} Let $R$ be a commutative ring. Let $M$ be a finitely generated $R$-module. Let $I$ be an ideal of $R$. If $IM=M$, then there exists $r\in R$ such that $rM=0$ and $r-1\in I$. 
\end{lmm}

\begin{lmm}{Nakayama's Lemma II}{} Let $R$ be a commutative ring. Let $M$ be a finitely generated $R$-module. Let $I$ be an ideal of $R$ such that $I\subseteq J(R)$ and $IM=M$. Then $M=0$. 
\end{lmm}

\begin{crl}{}{} Let $(R,m)$ be a local ring. Let $M$ be a finitely generated $R$-module. Then the following are true. 
\begin{itemize}
\item $M/mM$ is a finite dimensional vector space over $R/m$. 
\item $a_1,\dots,a_n\in M$ generates $M$ as an $R$-module if and only if $a_1+mM,\dots,a_n+mM$ generates $M/mM$ as a $R/m$ vector space. 
\end{itemize}
\end{crl}

\subsection{Exact Sequences}

\subsection{Change of Rings}
\begin{defn}{Extension of Scalars}{} Let $R,S$ be commutative rings. Let $\varphi:R\to S$ be a ring homomorphism. Let $M$ be an $R$-module. Define the extension of $M$ to the ring $S$ to be the $S$-module $$S\otimes_R M$$
\end{defn}

\begin{defn}{Restriction of Scalars}{} Let $R,S$ be commutative rings. Let $\varphi:R\to S$ be a ring homomorphism. Let $M$ be an $S$-module. Define the restriction of $M$ to the ring $R$ to be the $R$-module $M$ equipped with the action $$r\cdot_R m=\varphi(r)\cdot_S m$$ for all $r\in R$. 
\end{defn}

\begin{thm}{}{} Let $R,S$ be commutative rings. Let $\varphi:R\to S$ be a ring homomorphism. Then there is an isomorphism $$\Hom_S(S\otimes_R M,N)\cong\Hom_R(M,N)$$ for any $R$-module $M$ and $S$-module $N$ given as follows. 
\begin{itemize}
\item For $f\in\Hom_S(S\otimes_R M,N)$, define the map $f^+\in\Hom_R(M,N)$ by $$f^+(m)=f(1\otimes m)$$
\item For $g\in\Hom_R(M,N)$, define the map $g^-\in\Hom_S(S\otimes_RM,N)$ by $$g^-(s\otimes m)=s\cdot g(m)$$
\end{itemize}
\end{thm}

\pagebreak
\section{Localization}
\subsection{Localization of a Ring}
\begin{defn}{Multiplicative Set}{} Let $R$ be a commutative ring. $S\subseteq R$ is a multiplicative set if $1\in S$ and $S$ is closed under multiplication: $x,y\in S$ implies $xy\in S$
\end{defn}

\begin{defn}{Localization of a Ring}{} Let $R$ be a commutative ring and $S\subseteq R$ be a multiplicative set. Define the ring of fractions of $R$ with respect to $S$ by $$S^{-1}R=\left\{\frac{r}{s}|r\in R,s\in S\right\}/\sim$$ where $\sim$ is defined by $$\frac{r}{s}\sim\frac{r'}{s'}\text{ if and only if }\exists v\in S\text{ such that }v(ru'-r'u)=0$$
If $S=\{1,f,f^2,\dots\}$ then we write $S^{-1}R=R_f=R[1/f]$. 
\end{defn}

\begin{prp}{}{} Let $S^{-1}R$ be a ring of fractions. 
\begin{itemize}
\item $\sim$ as defined in the ring of fractions is an equivalence relation
\item $(S^{-1}R,+,\times)$ is a ring
\item The map $\phi:R\to S^{-1}R$ defined by $\phi(r)\to\frac{r}{1}$ is a ring homomorphism
\end{itemize}\tcbline
\begin{proof}~\\
\begin{itemize}
\item Trivial
\item Define addition by $\frac{r}{s}+\frac{r'}{s'}=\frac{rs'+r's}{ss'}$ and multiplication by $\frac{r}{s}\cdot\frac{r'}{s'}=\frac{rr'}{ss'}$. Clearly addition is abelian, and has identity $\frac{0}{1}$ and inverse $\frac{-r}{s}$ for any $\frac{r}{s}\in S^{-1}R$. Multiplication also has identity $\frac{1}{1}$. 
\item We have that $\phi(r+s)=\frac{r+s}{1}=\frac{r}{1}+\frac{s}{1}=\phi(r)+\phi(s)$ and $\phi(rs)=\frac{rs}{1}=\frac{r}{1}\cdot\frac{s}{1}=\phi(r)\cdot\phi(s)$ for any $r,s\in R$. 
\end{itemize}
\end{proof}
\end{prp}

\begin{thm}{Universal Property}{} Let $g:A\to B$ be a ring homomorphism such that $g(s)$ is a unit in $B$ for all $s\in S$. Then there exists a unique ring homomorphism $h:S^{-1}A\to B$ such that $g=h\circ\phi$. In other words, the following diagram commutes: \\~\\
\adjustbox{scale=1.1,center}{\begin{tikzcd}
A\arrow[r, "\phi"]\arrow[rd, "g"] & S^{-1}A\arrow[d, "\exists!h", dashed]\\
&B
\end{tikzcd}}
\end{thm}

\subsection{Localization at a Prime Ideal}
\begin{lmm}{}{} Let $R$ be a ring and $P$ a prime ideal of $R$. Then $R\setminus P$ is a multiplicative set. \tcbline
\begin{proof}
By definition, $xy\in P$ implies $x\in P$ or $y\in P$, since $R\setminus P$ removes all these elements, we have that $x\notin P$ and $y\notin P$ implies that $xy\notin P$. 
\end{proof}
\end{lmm}

\begin{defn}{Localization on Prime Ideals}{} Let $R$ be a commutative ring. Let $P$ be a prime ideal. Denote $$R_p=(R\setminus P)^{-1}R$$ the localization of $R$ at $P$. 
\end{defn}

\begin{lmm}{}{} Let $R$ be an integral domain. Then the localization $$(R\setminus(0))^{-1}R$$ is exactly the field of fractions of $R$. 
\end{lmm}

\begin{prp}{}{} Let $R$ be a ring and let $p$ be a prime ideal of $R$. Then $R_p$ is a local ring. \tcbline
\begin{proof}
Let $I$ be the set of all non-units of $R_p$. It is sufficient to show that $I$ is an ideal by the above lemma. Clearly if $i\in I$ then $r\cdot i$ is also not invertible. Explicitly, we have $$I=\left\{\frac{r}{s}\in R_p\bigg{|}r\in p\right\}$$ Let $\frac{r_1}{s_1},\frac{r_2}{s_2}\in I$, then $\frac{r_1}{s_1}+\frac{r_2}{s_2}=\frac{r_1s_2+r_2s_1}{s_1s_2}$ is in $I$ since $r_1,r_2\in P$ and $P$ being an ideal implies $r_1s_2+r_2s_1\in P$. 
\end{proof}
\end{prp}

Be wary that in general localizations does not result in a local ring. This happens only when we are localizing with respect to a prime ideal. The importance of prime ideals is not explicit in the above because only using prime ideals $P$ can $R\setminus P$ be a multiplicative set which ultimately allows localization to make sense. 

\subsection{Properties of Localization}
\begin{prp}{}{} Localization commutes with direct sum of modules and quotient modules. 
\end{prp}

\subsection{Localization of a Module}
\begin{defn}{Localization of a Module}{} Let $R$ be a commutative ring and $S\subseteq R$ be a multiplicative set Let $M$ be a $R$-module. Define the ring of fractions of $M$ with respect to $S$ by $$S^{-1}M=\left\{\frac{m}{s}|m\in M,s\in S\right\}/\sim$$ where $\sim$ is defined by $$\frac{m}{s}\sim\frac{m'}{s'}\text{ if and only if }\exists v\in S\text{ such that }v(mu'-m'u)=0$$
If $S=\{1,f,f^2,\dots\}$ then we write $S^{-1}M=M_f=M[1/f]$. 
\end{defn}

\begin{prp}{}{} Let $S$ be a multiplicative set of a ring $R$. Then localization at $S$ preservers exact sequences. 
\end{prp}

\begin{prp}{}{} Let $M$ be an $A$-module. Then the $S^{-1}A$ modules $S^{-1}M$ is isomorphic to $S^{-1}A\otimes_AM$. More precisely, there exists a unique isomorphism $f:S^{-1}A\otimes_AM\to S^{-1}M$ such that $$f((a/s)\otimes m)=am/s$$
\end{prp}

\pagebreak
\section{Noetherian Rings}
\subsection{Ordering on the Monomials}
Recall that a monomial in $R[x_1,\dots,x_n]$ is an element in the polynomial ring of the form $x_1^{a_1}\cdots x_n^{a_n}$. For simplicity we write this as $x^{(a_1,\dots,a_n)}$. 

\begin{defn}{Monomial Ordering}{} A monomial ordering on a polynomial ring $k[x_1,\dots,x_n]$ is a relation $>$ on $\N^n$. This means that the following are true. 
\begin{itemize}
\item $>$ is a total ordering on $\N^n$
\item If $a>b$ and $c\in\N^n$ then $a+c>b+c$
\item $>$ is a well ordering on $\N^n$ (any nonempty subset of $\N^n$ has a smallest element)
\end{itemize}
\end{defn}

\begin{defn}{Lexicographical Order}{} Let $a=(a_1,\dots,a_n)$ and $b=(b_1,\dots,b_n)$ in $\N^n$. We say that $a>_{\text{lex}}b$ if in the first nonzero entry of $a-b$ is positive. 
\end{defn}

In practise this means that the we value more powers of $x_1$

\begin{defn}{Graded Lex Order}{} Let $a=(a_1,\dots,a_n)$ and $b=(b_1,\dots,b_n)$ in $\N^n$. We say that $a>_{\text{grlex}}b$ if either of the following holds. 
\begin{itemize}
\item $\abs{a}=\sum_{k=1}^na_k>\sum_{k=1}^nb_k=\abs{b}$
\item $\abs{a}=\abs{b}$ and $a>_{\text{lex}}b$
\end{itemize}
\end{defn}

\begin{defn}{Graded Lex Order}{} Let $a=(a_1,\dots,a_n)$ and $b=(b_1,\dots,b_n)$ in $\N^n$. We say that $a>_{\text{grlex}}b$ if either of the following holds. 
\begin{itemize}
\item $\abs{a}=\sum_{k=1}^na_k>\sum_{k=1}^nb_k=\abs{b}$
\item $\abs{a}=\abs{b}$ and the last nonzero entry of $a-b$ is negative. 
\end{itemize}
\end{defn}

In practise we value lower powers of the last variable $x_n$. 

\begin{prp}{}{} The above three orders are all monomial orderings of $k[x_1,\dots,x_n]$. 
\end{prp}

\begin{defn}{Multidegree}{} Let $f\in k[x_1,\dots,x_n]$ be a polynomial in the form $f=\sum_{v\in\N^n}c_vx^v$. Define the multidegree of $f$ to be $$\text{multideg}(f)=\max_>\{v\in\N^n|a_v\neq 0\}$$ where $>$ is a monomial ordering on $k[x_1,\dots,x_n]$. 
\end{defn}

\begin{defn}{Leading Objects}{} Let $f\in k[x_1,\dots,x_n]$ be a polynomial in the form $f=\sum_{v\in\N^n}c_vx^v$. 
\begin{itemize}
\item Define the leading coefficient of $f$ to be $\text{LC}(f)=c_\text{multideg(f)}\in k$
\item Define the leading monomial of $f$ to be $\text{LM}(f)=c_\text{multideg(f)}\in k$
\item Define the leading term of $f$ to be $\text{LT}=\text{LC}(f)\cdot\text{LM}(f)$
\end{itemize}
\end{defn}

\begin{prp}{Division Algorithm in $k[x_1,\dots,x_n]$}{}
\end{prp}

\subsection{Monomial Ideals}
\begin{defn}{Monomial Ideals}{} An ideal $I\subset k[x_1,\dots,x_n]$ is said to be a monomial ideal if $I$ is generated by a set of monomials $\{x^v|v\in A\}$ for some $A\subset\N^n$. In this case we write $$I=\langle x^v|v\in A\rangle$$
\end{defn}

\begin{lmm}{}{} Let $I=\langle x^v|v\in A\rangle$ be an ideal of $k[x_1,\dots,x_n]$. Then a monomial $x^w$ lies in $I$ if and only if $x^v|x^w$ for some $v\in A$. Moreover, if $f=\sum_{w\in\N^n}c_wx^w\in k[x_1,\dots,x_n]$ lies in $I$, then each $x^w$ is divisible by $x^v$ for some $v\in A$. 
\end{lmm}

\begin{thm}{Dickson's Lemma}{} Every monomial ideal is finitely generated. In particular, every monomial ideal $I=\langle x^v|v\in A\rangle$ is of the form $$I=\langle x^{v_1},\dots,x^{v_n}\rangle$$ where $v_1,\dots,v_n\in A$. 
\end{thm}

\subsection{Groebner Bases}

\subsection{Hilbert's Basis Theorem}
\begin{prp}{}{} If $A$ is a Noetherian and $\phi$ is a homomorphism of $A$ onto a ring $B$, then $B$ is Noetherian. 
\end{prp}

\begin{thm}{Hilbert's Basis Theorem}{} If $R$ is a Noetherian ring, then $R[x_1,\dots,x_n]$ is a Noetherian ring. 
\end{thm}

\begin{prp}{}{} Let $R$ be a Noetherian ring and $I$ be an ideal in $R$. Then $R/I$ is Noetherian. 
\end{prp}

\begin{thm}{}{} Let $R=\bigoplus_{i=1}^nR_i$ be a graded ring. Then $R$ is Noetherian if and only if $R_0$ is Noetherian and $R$ is finitely generated as an $R_0$-module. 
\end{thm}

\pagebreak
\section{Primary Decomposition}
\subsection{Support of a Module}
\begin{defn}{Support of a Module}{} Let $M$ be an $A$-module. The support of $M$ is the subset $$\text{Supp}(M)=\{P\text{ a prime ideal of }A|M_P\neq 0\}$$
\end{defn}

\subsection{Associated Prime}
\begin{defn}{Associated Prime}{} Let $M$ be an $A$-module. An associated prime $P$ of $M$ is a prime ideal of $A$ such that there exists some $m\in M$ such that $P=\text{Ann}(m)$. 
\end{defn}

\subsection{Primary Ideals}
\begin{defn}{Primary Ideals}{} Let $R$ be a ring. An ideal $Q$ of $R$ is called primary if
\begin{itemize}
\item $Q\neq R$
\item $fg\in Q$ implies $f\in Q$ or $g^m\in Q$ for some $m>0$
\end{itemize}
\end{defn}

\begin{lmm}{}{} If $Q$ is primary, then $\sqrt{Q}$ is prime. 
\end{lmm}

\begin{lmm}{}{} Let $R$ be a Noetherian ring and $I$ be a proper ideal that is not primary. Then $$I=J_1\cap J_2$$ for some ideals $J_1,J_2\neq I$. 
\end{lmm}

\begin{defn}{P-Primary Ideals}{} Let $A$ be a ring and $P$ a prime ideal. An ideal $Q$ is $P$-primary if $Q$ is primary and $Q=\text{rad}(P)$
\end{defn}

\begin{thm}{}{} Let $A$ be a Noetherian ring and $Q$ an ideal of $A$. Then $Q$ is $P$-primary if and only if $\text{Ann}(A/Q)=\{P\}$. 
\end{thm}

\subsection{Primary Decomposition}
We want to express ideal $I$ in $R$ as $I=P_1^{e_1}\cdots P_n^{e_n}$ similar to a factorization of natural numbers, for some prime ideals $P_1,\dots,P_n$. However this notion fails and thus we have the following new type of ideal. 

\begin{defn}{Primary Decompositions}{} A primary decomposition of an ideal $I$ is an expression $I=Q_1\cap\dots\cap Q_r$ with each $Q_i$ primary. \\~\\
The decomposition is said to be irredundant if $I\neq\cap_{i\neq j}Q_i$ for any $j$. The decomposition is said to be minimal if $r$ is the smallest possible such decomposition for $I$. 
\end{defn}

Irredundant in this sense means that removing any one primary ideal in the intersection fails to become a decomposition of $I$. 

\begin{thm}{}{} Every proper ideal in a Noetherian ring has a primary decomposition. 
\end{thm}

\begin{lmm}{}{} Let $\phi:R\to S$ be a ring homomorphism and $Q$ be a primary ideal in $S$. Then $\phi^{-1}(Q)$ is primary in $R$. 
\end{lmm}

\pagebreak
\section{Integral Dependence}
\subsection{Integral Extensions}
\begin{defn}{Integral Elements}{} Let $B$ be a ring and let $A\subseteq B$ be a subring. Let $b\in B$. We say that $b$ is integral over $A$ if there exists a monic polynomial $p(x)=x^n+a_{n-1}x^{n-1}+\dots+a_0\in A[x]$ such that $p(b)=0$. 
\end{defn}

\begin{prp}{}{} Let $B$ be a ring and let $A\subseteq B$. Let $b\in B$. Then the following are equivalent. 
\begin{itemize}
\item $b$ is integral over $A$
\item The subring $A[b]\subseteq B$ is finite over $A$
\item There exists an $A$ sub-algebra $A'\subseteq B$ such that $A[b]\subseteq A'$ and $A'$ is finite over $A$. 
\end{itemize}
\end{prp}

\begin{prp}{}{} Let $B$ be a ring and let $A\subseteq B$ be a subring. Let $b_1,b_2\in B$ be integral over $A$. Then $b_1+b_2$ and $b_1b_2$ are both integral over $A$. 
\end{prp}

\begin{defn}{Integral Extensions}{} Let $B$ be a ring and let $A\subseteq B$ be a subring. We say that $B$ is integral over $A$ if all elements of $B$ are integral over $A$. 
\end{defn}

\begin{lmm}{}{} Let $A\subseteq B\subseteq C$ be rings. If $C$ is integral over $B$ and $B$ is integral over $A$, then $C$ is integral over $A$. 
\end{lmm}

\begin{defn}{Integral Closure}{} Let $B$ be an $A$-algebra. Define the subring $$\overline{A}=\{b\in B|b\text{ is integral over }A\}$$ to be the integral closure of $A$ in $B$. If $\overline{A}=A$, then we say that $A$ is integrally closed in $B$. 
\end{defn}

\begin{lmm}{}{} Let $B$ be a ring and let $A\subseteq B$ be a subring. Then $\overline{A}$ is an integral extension of $A$. 
\end{lmm}

\begin{defn}{Normal Domains}{} Let $R$ be a domain. We say that $R$ is normal (intergrally closed) if $A$ is integrally closed in its field of fractions. \\~\\
The integral closure of $R$ in $\text{Frac}(R)$ is called the normalization of $R$. 
\end{defn}

\subsection{The Trace and Norm}

\subsection{The Going-Up and Going-Down Theorems}

\subsection{Dedekind Domains}
\begin{defn}{Dedekind Domains}{} Let $R$ be a ring. We say that $R$ is a dedekind domain if the following are true. 
\begin{itemize}
\item $R$ is an integral domain
\item $R$ is an integrally closed
\item $R$ is Noetherian
\item Every non-zero prime ideal of $R$ is maximal
\end{itemize}
\end{defn}


\pagebreak
\section{Discrete Valuation Rings}
\subsection{Discrete Valuation Rings}
\begin{defn}{Totally Ordered Group}{} A totally ordered group is a group $G$ with a total order "$\leq$" such that it is
\begin{itemize}
\item a left ordered group: $a\leq b$ implies $ca\leq cb$ for all $a,b,c\in G$
\item a right ordered group: $a\leq b$ implies $ac\leq bc$ for all $a,b,c\in G$
\end{itemize}
\end{defn}

\begin{defn}{Valuation on a Field}{} Let $K$ be a field. Let $G$ be a totally ordered abelian group. A valuation on $K$ with values in $G$ is a map $v:K\setminus\{0\}\to G$ such that for all $x,y\in K^\ast$, we have 
\begin{itemize}
\item $v(xy)=v(x)+v(y)$
\item $v(x+y)\geq\min\{v(x),v(y)\}$
\end{itemize}
We use the convention that $v(0)=\infty$. \\~\\
$v$ is said to be a discrete valuation if $G=\Z$. 
\end{defn}

\begin{prp}{}{} Let $K$ be a field and $v:K\to\Z$ a discrete valuation. Then $$\{x\in K|v(x)\geq 0\}$$ is a subring of $K$. 
\end{prp}

\begin{defn}{Discrete Valuation Rings}{} The discrete valuation ring of a discrete valuation $v:K\to\Z$ is the subset $$A=\{x\in K|v(x)\geq 0\}$$
Alternatively, any ring isomorphic to a discrete valuation ring of some discrete valuation is also called a discrete valuation. 
\end{defn}

\begin{prp}{}{} Let $R$ be a discrete valuation ring with respect to the valuation $v$. Let $t\in R$ be such that $v(t)=1$. Then the following are true. 
\begin{itemize}
\item A nonzero element $u\in R$ is a unit if and only if $v(u)=0$
\item Every non-zero ideal of $R$ is a principal ideal of the form $(t^n)$ for some $n\geq 0$
\item Every $r\in R\setminus\{0\}$ can be written in the form $r=ut^n$ for some unit $u$ and $n\geq 0$. 
\end{itemize}\tcbline
\begin{proof}~\\
\begin{itemize}
\item Let $R$ be a discrete valuation ring. Suppose that $x\in R$ is a unit. Then $v(x^{-1})=-v(x)$. Then $-v(x),v(x)\geq 0$ implies $v(x)=0$. Now if $v(y)>0$ , suppose for contradiction that $u\in R$ is an inverse of $y$, then $$0=v(1)=v(uy)=v(u)+v(y)$$ But $v(y)>0$ implies that $v(u)<0$ which implies that $u\notin R$, a contradiction. 
\item Let $t\in R$ such that $v(t)=1$. Let $x\in m$ where $v(x)=n>0$. Then $v(x)=nv(t)=v(t^n)$ means that every $x\in m$ is of the form $t^n$. Thus $m=(t)$. Since every ideal $I$ is a subset of this maximal ideal, any ideal is of the form $I=(t^n)$ for some $n>0$. 
\item Follows from the fact that $(t^n)$ is the unique maximal ideal. 
\end{itemize}
\end{proof}
\end{prp}

\begin{prp}{}{} Let $R$ be an integral domain. Then the following are equivalent. 
\begin{itemize}
\item $R$ is a discrete valuation ring
\item $R$ is a UFD with a unique irreducible element up to multiplication of a unit
\item $R$ is a Noetherian local ring with a principal maximal ideal
\end{itemize} \tcbline
\begin{proof}~\\
\begin{itemize}
\item $(1)\implies(3)$: We have seen that the set of non-units is precisely the set $m=\{x\in K|v(x)>0\}$. We show that this is an ideal. Clearly $x,y\in m$ implies $v(x+y)=\min\{v(x),v(y)\}>0$. Let $u\in R$. Then $v(ux)=v(u)+v(x)>0$ since $v(x)>0$ and $v(u)\geq 0$. \\~\\
We have seen that every ideal is of the form $(t^n)$ for some $n>0$. Thus every ascending chains of ideal must be of the form $$(t^{n_1})\subset(t^{n_2})\subset\dots$$ for $n_1>n_2>\dots$. Since $n_1,n_2,\dots$ is strictly decreasing, the chain must eventually stabilizes. This proves that $R$ is Noetherian and has principal maximal ideal. 
\item $(1)\implies(3)$:
\end{itemize}
\end{proof}
\end{prp}


\pagebreak
\section{Dimension Theory for Rings}
\subsection{Dimension and Height}
\begin{defn}{Krull Dimension}{} Let $R$ be a commutative ring. Define the Krull dimension of $R$ to be $$\dim(R)=\sup\{t\in\N|p_0\subset\dots\subset p_t\text{ for }p_0,\dots,p_t\text{ prime ideals }\}$$
\end{defn}

\begin{defn}{Height of a Prime Ideal}{} Let $p$ be a prime ideal in a ring $R$. Define the height of $p$ to be $$\text{ht}(p)=\sup\{t\in\N|p_0\subset\dots\subset p_t=p\text{ for }p_0,\dots,p_t\text{ prime ideals }\}$$
\end{defn}

\begin{lmm}{}{} Let $p$ be a prime ideal in a ring $R$. Then $$\text{ht}(p)=\dim(R_p)$$ 
\end{lmm}

\begin{thm}{Krull's Principal Ideal Theorem}{} Let $R$ be a Noetherian ring. Let $I$ be a proper and principal ideal of $R$. Let $p$ be the smallest prime ideal containing $I$. Then $$\text{ht}_R(p)\leq 1$$
\end{thm}

\subsection{Length of a Module}
\begin{defn}{Length of a Module}{} Let $R$ be a ring and let $M$ be an $R$-module. Define the length of $M$ to be $$l_R(M)=\text{sup}\{n\in\N\;|\;0=M_0\subset M_1\subset\cdots\subset M_n=M\}$$
\end{defn}

\begin{lmm}{}{} Let $R$ be a ring. Let $0\to M'\to M\to M''\to 0$ be a short exact sequence of $R$-modules. Then $$l_R(M)=l_R(M')+l_R(M'')$$
\end{lmm}

\begin{lmm}{}{} Let $(A,m)$ be a local ring and let $M$ be an $A$-module. If $mM=0$, then $$l_A(M)=\dim_{A/m}(M)$$
\end{lmm}

\begin{prp}{}{} Let $R$ be a ring and let $M$ be an $R$-module. Then the following are equivalent. 
\begin{itemize}
\item $M$ is simple
\item $l_R(M)=1$
\item $M\cong A/m$ for some maximal ideal $m$ of $A$
\end{itemize}
\end{prp}

\subsection{The Hilbert Polynomial}
\begin{defn}{The Hilbert Polynomial}{} Let $R=\bigoplus_{k=0}^\infty R_k$ be a Noetherian graded ring. Let $M=\bigoplus_{k=0}^\infty M_k$ be a graded $R$-module. Define the Hilbert function $H_M:\N\to\N$ of $R$ to be the function defined by $$H_M(n)=l_{R_0}(M_n)$$
\end{defn}

\begin{defn}{The Hilbert Series}{} Let $R=\bigoplus_{k=0}^\infty R_k$ be a Noetherian graded ring. Let $M=\bigoplus_{k=0}^\infty M_k$ be a graded $R$-module. Define the Hilbert series $HS_M\in\Z[[t]]$ of $M$ to be the formal series $$HS_M(t)=\sum_{k=0}^\infty H_M(k)t^k=\sum_{k=0}^\infty l_{R_0}(M_k)t^k$$
\end{defn}

\begin{thm}{}{} Let $R=\bigoplus_{k=0}^\infty R_k$ be a Noetherian graded ring such that $R_0$ is Artinian. Let $M=\bigoplus_{k=0}^\infty M_k$ be a graded $R$-module. Let $\lambda:\{M_i\;|\;i\in I\}\to\Z$ be an additive function Then the function $$g(t)=\sum_{k=0}^\infty\lambda(M_k)t^k$$ is a rational function and can be written in the form $$g(t)=\frac{f(t)}{\prod_{i=1}^r(1-t^{d_i})}$$ for some $f(t)\in\Z[t]$ and $d_i\in\N$. 
\end{thm}

\begin{thm}{The Fundamental Theorem of Dimension Theory}{} Let $(R,m)$ be a local Noetherian ring. Let $I$ be an $m$-primary ideal. Then the following numbers are equal. 
\begin{itemize}
\item Let $J=\bigoplus_{k=0}^\infty\frac{I^k}{I^{k+1}}$. The order of the pole at $1$ of the rational function $HS_J$. 
\item The minimum number of elements of $R$ that can generate an $m$-primary ideal of $R$
\item The dimension $\dim_{R/m}(R)$
\end{itemize}
\end{thm}

The following is a generalization of Krull's principal ideal theorem. Both of the theorems can actually be deduced directly from the fundamental theorem. 

\begin{thm}{Krull's Height Theorem}{} Let $R$ be a Noetherian ring. Let $I$ be a proper ideal generated by $n$ elements. Let $p$ be the smallest prime ideal containing $I$. Then $$\text{ht}_R(p)\leq n$$
\end{thm}

\begin{thm}{}{} Let $(R,m)$ be a Noetherian local ring and let $k=R/m$ be the residue field. Then $$\dim(R)\leq\dim_k(m/m^2)$$
\end{thm}

\subsection{Global Dimension of a Ring}







\end{document}