\documentclass[a4paper]{article}

%=========================================
% Packages
%=========================================
\usepackage{mathtools}
\usepackage{amsfonts}
\usepackage{amsmath}
\usepackage{amssymb}
\usepackage{amsthm}
\usepackage[a4paper, total={6in, 8in}, margin=1in]{geometry}
\usepackage[utf8]{inputenc}
\usepackage{fancyhdr}
\usepackage[utf8]{inputenc}
\usepackage{graphicx}
\usepackage{physics}
\usepackage[listings]{tcolorbox}
\usepackage{hyperref}
\usepackage{tikz-cd}
\usepackage{adjustbox}
\usepackage{enumitem}


\hypersetup{
    colorlinks=true, %set true if you want colored links
    linktoc=all,     %set to all if you want both sections and subsections linked
    linkcolor=black,  %choose some color if you want links to stand out
}
\usetikzlibrary{arrows.meta}

\DeclarePairedDelimiter\ceil{\lceil}{\rceil}
\DeclarePairedDelimiter\floor{\lfloor}{\rfloor}

%=========================================
% Custom Math Operators
%=========================================
\DeclareMathOperator{\adj}{adj}
\DeclareMathOperator{\im}{im}
\DeclareMathOperator{\nullity}{nullity}
\DeclareMathOperator{\sign}{sign}
\DeclareMathOperator{\dom}{dom}
\DeclareMathOperator{\lcm}{lcm}
\DeclareMathOperator{\ran}{ran}
\DeclareMathOperator{\ext}{Ext}
\DeclareMathOperator{\dist}{dist}
\DeclareMathOperator{\diam}{diam}
\DeclareMathOperator{\aut}{Aut}
\DeclareMathOperator{\inn}{Inn}
\DeclareMathOperator{\syl}{Syl}
\DeclareMathOperator{\edo}{End}
\DeclareMathOperator{\cov}{Cov}
\DeclareMathOperator{\vari}{Var}
\DeclareMathOperator{\cha}{char}
\DeclareMathOperator{\Span}{span}
\DeclareMathOperator{\ord}{ord}
\DeclareMathOperator{\res}{res}
\DeclareMathOperator{\Hom}{Hom}
\DeclareMathOperator{\Mor}{Mor}
\DeclareMathOperator{\coker}{coker}
\DeclareMathOperator{\Obj}{Obj}
\DeclareMathOperator{\id}{id}
\DeclareMathOperator{\GL}{GL}
\DeclareMathOperator*{\colim}{colim}

%=========================================
% Custom Commands (Shortcuts)
%=========================================
\newcommand{\CP}{\mathbb{CP}}
\newcommand{\GG}{\mathbb{G}}
\newcommand{\F}{\mathbb{F}}
\newcommand{\N}{\mathbb{N}}
\newcommand{\Q}{\mathbb{Q}}
\newcommand{\R}{\mathbb{R}}
\newcommand{\C}{\mathbb{C}}
\newcommand{\E}{\mathbb{E}}
\newcommand{\Prj}{\mathbb{P}}
\newcommand{\RP}{\mathbb{RP}}
\newcommand{\T}{\mathbb{T}}
\newcommand{\Z}{\mathbb{Z}}
\newcommand{\A}{\mathbb{A}}
\renewcommand{\H}{\mathbb{H}}

\newcommand{\mA}{\mathcal{A}}
\newcommand{\mB}{\mathcal{B}}
\newcommand{\mC}{\mathcal{C}}
\newcommand{\mD}{\mathcal{D}}
\newcommand{\mE}{\mathcal{E}}
\newcommand{\mF}{\mathcal{F}}
\newcommand{\mG}{\mathcal{G}}
\newcommand{\mH}{\mathcal{H}}
\newcommand{\mJ}{\mathcal{J}}
\newcommand{\mO}{\mathcal{O}}
\newcommand{\mS}{\mathcal{S}}

%=========================================
% Theorem Environment
%=========================================
\newcommand\todoin[2][]{\todo[backgroundcolor=white!20!white, inline, caption={2do}, #1]{
\begin{minipage}{\textwidth-4pt}#2\end{minipage}}}

\tcbuselibrary{listings, theorems, breakable, skins}

\newtcbtheorem[number within=subsection]{thm}{Theorem}%
{colback=gray!5, colframe=gray!65!black, fonttitle=\bfseries, breakable, enhanced jigsaw, halign=left}{th}
\newtcbtheorem[number within=subsection, use counter from=thm]{defn}{Definition}%
{colback=gray!5, colframe=gray!65!black, fonttitle=\bfseries, breakable, enhanced jigsaw, halign=left}{th}
\newtcbtheorem[number within=subsection, use counter from=thm]{axm}{Axiom}%
{colback=gray!5, colframe=gray!65!black, fonttitle=\bfseries, breakable, enhanced jigsaw, halign=left}{th}
\newtcbtheorem[number within=subsection, use counter from=thm]{prp}{Proposition}%
{colback=gray!5, colframe=gray!65!black, fonttitle=\bfseries, breakable, enhanced jigsaw, halign=left}{th}
\newtcbtheorem[number within=subsection, use counter from=thm]{lmm}{Lemma}%
{colback=gray!5, colframe=gray!65!black, fonttitle=\bfseries, breakable, enhanced jigsaw, halign=left}{th}
\newtcbtheorem[number within=subsection, use counter from=thm]{crl}{Corollary}%
{colback=gray!5, colframe=gray!65!black, fonttitle=\bfseries, breakable, enhanced jigsaw, halign=left}{th}
\newtcbtheorem[number within=subsection, use counter from=thm]{eg}{Example}%
{colback=gray!5, colframe=gray!65!black, fonttitle=\bfseries, breakable, enhanced jigsaw, halign=left}{th}
\newtcbtheorem[number within=subsection, use counter from=thm]{ex}{Exercise}%
{colback=gray!5, colframe=gray!65!black, fonttitle=\bfseries, breakable, enhanced jigsaw, halign=left}{th}
\newtcbtheorem[number within=subsection, use counter from=thm]{alg}{Algorithm}%
{colback=gray!5, colframe=gray!65!black, fonttitle=\bfseries, breakable, enhanced jigsaw, halign=left}{th}

\newcounter{qtnc}
\newtcolorbox[use counter=qtnc]{qtn}%
{colback=gray!5, colframe=gray!65!black, fonttitle=\bfseries, breakable, enhanced jigsaw, halign=left}




\raggedright

\pagestyle{fancy}
\fancyhf{}
\rhead{Labix}
\lhead{Advanced Ring Theory}
\rfoot{\thepage}

\title{Advanced Ring Theory}

\author{Labix}

\date{\today}
\begin{document}
\maketitle
\begin{abstract}
\begin{itemize}
\item Abstract Alebra by Thomas W. Judson
\end{itemize}
\end{abstract}
\pagebreak
\tableofcontents
\pagebreak

\section{Division Rings}
\subsection{The Structure of Quaternions}
\begin{defn}{Quaternions}{} Define the quaternions as the quotient algebra $$\H=\frac{\R\langle x_1,x_2,x_3\rangle}{I}$$ where $I=(x_1^2+1,x_2^2+1,x_3^2+1,x_1x_2x_3+1)$. \\~\\
Elements of $\H$ are of the form $a+b\vb{i}+c\vb{j}+d\vb{k}$ for $a,b,c,d\in\R$ and by writing $\vb{i}=x_1+I$, $\vb{j}=x_2+I$ and $\vb{k}=x_3+I$. \\~\\
A quaternion is said to be real if $b=c=d=0$. It is said to be imaginary if $a=0$. Denote the set of all imaginary quaternions by $\H_0$. 
\end{defn}

\begin{prp}{}{} The quaternions satisfy the following multiplication table: \begin{center}
\begin{tabular}{ |c|c|c|c|c| } 
\hline
$\cdot$ & $1$ & $\vb{i}$ & $\vb{j}$ & $\vb{k}$\\\hline
$1$ & $1$ & $\vb{i}$ & $\vb{j}$ & $\vb{k}$\\\hline
$\vb{i}$ & $\vb{i}$ & $-1$& $\vb{k}$ & $-\vb{j}$ \\\hline
$\vb{j}$ & $\vb{j}$& $-\vb{k}$& $-1$& $\vb{i}$\\\hline
$\vb{k}$ & $\vb{k}$ & $\vb{j}$ & $-\vb{i}$ & $-1$\\\hline
\end{tabular}
\end{center}\tcbline
\begin{proof}
We only need to consider products that does not involve $1$. It clear for $t=1,2,3$, $x_t^2+1\in I$. This means that $x_t^2+I=-1+I$ and thus $\vb{i}^2=\vb{j}^2=\vb{k}^1=-1$. Similarly, we have that $x_1x_2x_3+I=-1+I$ and thus $\vb{ijk}=-1$. Multiplying this expression by $-\vb{i}$ on the left gives $\vb{jk}=\vb{i}$. We can also multiply the expression by $-\vb{k}$ on the right to get $\vb{ij}=\vb{k}$. Now multiply $\vb{i}$ to the left of the equation $\vb{ij}=\vb{k}$ to get $-\vb{j}=\vb{ik}$. We can also multiply $\vb{ij}=\vb{k}$ by $\vb{j}$ on the right gives $-\vb{i}=\vb{kj}$. Finally we have $\vb{j}(\vb{i}=\vb{jk})\implies\vb{ji}=-\vb{k}$ and $(\vb{ji}=-\vb{k})(-\vb{i})\implies\vb{j}=\vb{ki}$. 
\end{proof}
\end{prp}

\begin{prp}{}{} The elements $1,\vb{i},\vb{j},\vb{k}$ form a basis for the $\R$-algebra $\H$. \tcbline
\begin{proof}
It is clear that $1,x_1,x_2,x_3,x_1x_2,x_1x_3,x_2,x_3,\dots$ span $\H$. By writing $x_1,x_2,x_3$ each in terms of $1,\vb{i},\vb{j},\vb{k}$ respectively, we have can see that $1,\vb{i},\vb{j},\vb{k}$ span $\H$. It remains to show that they are linearly independent. \\~\\

Consider the $\R$-algebra homomorphism $f:\R\langle x_1,x_2,x_3\rangle\to M_{2\times 2}(\C)$ defined by $f(x_1)=\begin{pmatrix}
i & 0\\
0 & -i
\end{pmatrix}$, $f(x_2)=\begin{pmatrix}
0 & 1\\
-1 & 0
\end{pmatrix}$ and $f(x_3)=\begin{pmatrix}
0 & i\\
i & 0
\end{pmatrix}$. It is clear that $I\subseteq\ker(f)$ since $f(x_1^2+1)=f(x_2^2+1)=f(x_3^2+1)=f(x_1x_2x_3+1)=0$. By the first and third isomorphism theorem for modules, we have that $$\frac{\H}{\ker(f)/I}\cong\frac{\R\langle x_1,x_2,x_3\rangle}{\ker(f)}\cong\im(f)$$
This means that $\dim_\R(\H)\geq\dim_\R(\im(f))$. Since the matrices $f(x_1),f(x_2),f(x_3)$ and $1$ are all linearly independent over $\R$, we have that $\im(f)$ is at least $4$-dimensional. Hence the four spanning elements of $\H$ must be linearly independent. 
\end{proof}
\end{prp}

\begin{prp}{}{} The imaginary quaternions $\H_0$ form a three dimensional vector subspace of $\H$. The real quaternions form a subalgebra $\R$ of $\H$. 
\end{prp}

We treat the imaginary quaternions $\H_0$ as the standard $3$-space with dot product $$(b_1\vb{i}+c_1\vb{j}+d_1\vb{k})\cdot(b_2\vb{i}+c_2\vb{j}+d_2\vb{k})=b_1b_2+c_1c_2+d_1d_2$$ and cross product 
\begin{align*}
(b_1\vb{i}+c_1\vb{j}+d_1\vb{k})\times_c(b_2\vb{i}+c_2\vb{j}+d_2\vb{k})&=(c_1d_2-c_2d_1)\vb{i}+(d_1b_2-d_2b_1)\vb{j}+(b_1c_2-c_2b_1)\vb{k}\\
&=\begin{vmatrix}
\vb{i} & \vb{j} & \vb{k}\\
b_1 & c_1 & d_1\\
b_2 & c_2 & d_2
\end{vmatrix}
\end{align*}

\begin{prp}{}{} Let $a_1+\vb{h}_1$ and $a_2,\vb{h}_2$ be quaternions such that $a_1,a_2\in\R$ and $\vb{h}_1,\vb{h}_2\in\H_0$. Then $$(a_1+\vb{h}_1)(a_2+\vb{h}_2)=(a_1a_2-\vb{h}_1\cdot\vb{h}_2)+(a_1\vb{h}_2+a_2\vb{h}_1+\vb{h}_1\times_c\vb{h}_2)$$
\end{prp}

\begin{defn}{Conjugate and Norm}{} Let $x=a+b\vb{i}+c\vb{j}+d\vb{k}\in\H$ be a quaternion. Define the conjugate of $x$ to be $$x^\ast=a-b\vb{i}-c\vb{j}-d\vb{k}$$ Also define the norm of $x$ to be $$\|x\|=\sqrt{a^2+b^2+c^2+d^2}$$
\end{defn}

\begin{prp}{}{} Let $x,y\in\H$ be quaternions. The following are true regarding the conjugate and norm of the quaternions: 
\begin{itemize}
\item $xx^\ast=\|x\|$
\item $(xy)^\ast=y^\ast x^\ast$
\item $\|xy\|=\|x\|\|y\|$
\end{itemize} \tcbline
\begin{proof}

\end{proof}
\end{prp}

\begin{prp}{}{} $\H$ is a division ring. \tcbline
\begin{proof}
Let $x\in\H$. By the above proposition, we have that $x\frac{x^\ast}{\|x\|}=1$ which means we have found an inverse $\frac{x^\ast}{\|x\|}$ of $x$. 
\end{proof}
\end{prp}

\subsection{The Multiplicative Group of Quaternions}
\begin{defn}{The Quaternionic Unitary Group}{} Define the quaternionic unitary group to be the subgroup $$U(\H)=\{x\in\H\;|\;\|x\|=1\}$$ of $\H^\times$. 
\end{defn}

\begin{prp}{}{} The multiplicative group $\H^\times$ is isomorphic to $\R_+^\times\times U(\H)$, where $\R_+^\times$ is the multiplicative group of non-zero real numbers. \tcbline
\begin{proof}

\end{proof}
\end{prp}

\begin{prp}{Quaternionic Euler's Formula}{} Write a quaternion into the form $q=a+b\vb{x}\in\H$ where $a,b\in\R$ and $\vb{x}\in\H_0$ is purely imaginary such that $\|\vb{x}\|=1$. Then $$e^q=e^a(\cos(b)+\vb{x}\sin(b))$$
\end{prp}

\begin{prp}{Quaternionic De Moivre's Formula}{} Let $\vb{x}\in H_0$ be purely imaginary such that $\|\vb{x}\|=1$. Let $n\in\Z$. Then $$(\cos(b)+\vb{x}\sin(b))^n=\cos(nb)+\vb{x}\sin(nb)$$
\end{prp}

\subsection{3D Rotations using Quaternions}
Recall the special orthogonal group in $3$-dimensions is the group $$\text{SO}_3(\R)=\{M\in\text{GL}_3(\R)|\det(M)=1\}$$

\begin{prp}{}{} Let $M\in\text{SO}_3(\R)$ be a special orthogonal transformation. Then there exists an orthonormal basis of $\R^3$ such that the matrix decomposes into the direct sum $(1)\oplus R_\alpha$, where $R_\alpha=\begin{pmatrix}
\cos(\alpha) & -\sin(\alpha)\\
\sin(\alpha) & \cos(\alpha)
\end{pmatrix}$ is a rotation in $\R^2$. 
\end{prp}

Now we know that every special orthogonal transformation is just a rotation in the plane orthogonal to $e_1$. In generality, we write $R_{\vb{x}}^\alpha$ for the anti-clockwise rotation in angle $\alpha$ in the plane orthogonal to $\vb{x}\in\R^3$. We can use the quaternions to write out a formula for applying the special orthogonal transformation to a vector. 

\begin{lmm}{}{} Let $\vb{x}\in\H_0\cap U(\H)$ be an imaginary unit. Let $\theta\in\R$. Then $$R_{\vb{x}}^{2\theta}(\vb{w})=e^{\theta\vb{x}}\vb{w}e^{-\theta\vb{x}}$$ for all $\vb{w}\in\H_0$. 
\end{lmm}

This leads tot he fundamental fact behind the theory of spinors in Geometry and Physics. 

\begin{thm}{}{} The conjugation action map $$\phi:U(\H)\to\text{SO}(\H_0)\cong\text{SO}_3(\R)$$ defined by $\phi(x)(\vb{z})=x\vb{z}x^{-1}$ for $\vb{z}\in\H_0$ and $x\in U(\H)$ is a surjective two to one group homomorphism. 
\end{thm}

\subsection{Division Rings over Real and Complex Numbers}
\begin{prp}{}{} The only finite dimensional $\C$-division algebra is $\C$. \tcbline
\begin{proof}
Let $D$ be a finite dimensional $\C$-division algebra. Then in particular, $\C\subseteq D$. Suppose that $a\in D$. Then the minimal polynomial $\mu_a(x)$ is an irreducible element of $\C[x]$. By the fundamental theorem of algebra, $\mu_a(x)=x-\alpha$ with $\alpha\in\C$. This means that $a=\alpha\in\C$ and thus $D=\C$. 
\end{proof}
\end{prp}

\begin{prp}{}{} The only odd dimensional $\R$-division algebra is $\R$. \tcbline
\begin{proof}
Let $D$ be an $\R$-division algebra of odd dimension $n$. Then in particular, $\R\subseteq D$. Let $a\in D$. In linear algebra we know that the $\R$-linear map $L:D\to D$ defined by $L(d)=ad$ admits a real eigenvalue $\alpha\in D$ and eigenvector $v$. Then $av=\alpha v$ implies that $(a-\alpha)v=0$. Since $D$ is a division algebra, we have that $a=\alpha\in\R$. Thus $\R=D$. 
\end{proof}
\end{prp}

\begin{thm}{Frobenius Theorem}{} A finite dimensional division algebra over $\R$ is isomorphic to $\R$, $\C$ or $\H$. 
\end{thm}

\begin{thm}{}{} The only countably generated division algebra over $\R$ up to isomorphism is either $\R$, $\C$ or $\H$. 
\end{thm}

\subsection{Finite Division Rings}
\begin{lmm}{}{} Let $R$ be a division ring. Then $Z(R)$ is a field. Moreover, $R$ is a $Z(R)$-algebra. 
\end{lmm}

\begin{crl}{}{} Let $D$ be a finite division ring. Then the following statements are true regrading $D$. 
\begin{itemize}
\item $Z(D)$ is a finite field $\F_{p^n}$ for some $n\in\N\setminus\{0\}$
\item The dimension of $D$, $m=\dim_{Z(D)}D$ over $Z(D)$ is finite
\item $\abs{D}={p^n}^m$
\end{itemize}
\end{crl}

\begin{lmm}{}{} Let $R$ be a division ring and $x\in R$. Then $C_R(x)$ is a division ring and a $Z(R)$-subalgebra. 
\end{lmm}

\begin{prp}{}{} Let $D$ be a finite division ring of dimension $m$ over its center $Z(D)=\F_q$, where $q=p^n$ for some prime $p$ and $n\in\N\setminus\{0\}$. Then there exists positive integers $d_1,\dots,d_k$ such that $d_i|m$, $d_i<m$ and $$q^m=q+\sum_{i=1}^k\frac{q^m-1}{q^{d_i}-1}$$
\end{prp}

\begin{thm}{Little Wedderburn's Theorem}{} A finite division ring is a field. 
\end{thm}

\subsection{Laurent Series}
\begin{defn}{Laurent Series}{} Let $R$ be a ring. A Laurent series in variable $x$ with coefficients in a ring $R$ is an expression of the form $$\sum_{n=m}^\infty a_nx^n$$ for $m\in\Z$ and $a_n\in R$. Denote the set of Laurent series in $x$ by $$R((x))$$
\end{defn}

\begin{lmm}{}{} Let $R$ be a ring. Then $R((x))$ is also a ring. If $R$ is a field, then $R((x))$ is also a field. 
\end{lmm}

\begin{lmm}{}{} The center of $\H((x))$ is precisely $\R((x))$. 
\end{lmm}

\pagebreak
\section{Semisimplicity}
\subsection{Semisimple Modules}
\begin{defn}{Semisimple Modules}{} A left $R$-module $M$ is semisimple if $M$ is a direct sum of simple modules. 
\end{defn}

\begin{defn}{Socle of a Module}{} Denote $\mathcal{SM}(M)=\{S\leq M|S\text{ is simple}\}$. A socle of a left $R$-module $M$ is a submodule $$\text{soc}(M)=\sum_{S\in\mathcal{SM}(M)}S$$
\end{defn}

\begin{lmm}{}{} A module $M$ is semisimple if and only if $\text{soc}(M)=M$. 
\end{lmm}

\begin{crl}{}{} A quotient module of a semisimple module is semisimple. 
\end{crl}

\begin{defn}{Completely Reducible Modules}{} Let $M$ be an $R$-module. $M$ is said to be completely reducible if for every submodule $N$ of $M$, there exists a submodule $L$ of $M$ such that $M=N\oplus L$. 
\end{defn}

\begin{prp}{}{} Let $M$ be an $R$-module such that $M=N\oplus L$. Then there is an isomorphism $$L\cong\frac{M}{N}$$ of $R$-modules. 
\end{prp}

\begin{lmm}{}{} A submodule of a completely reducible module is reducible. 
\end{lmm}

\begin{lmm}{}{} A non-zero completely reducible module contains a simple submodule. 
\end{lmm}

\begin{thm}{}{} Let $M$ be an $R$-module. Then $M$ is semisimple if and only if $M$ is completely reducible. 
\end{thm}

\begin{crl}{}{} A submodule of a semisimple module is semisimple. 
\end{crl}

\subsection{Peirce Decomposition for Modules}
\begin{defn}{Idempotents}{} Let $M$ be a module. We say that $e\in M$ is an idempotent if $e^2=e$. 
\end{defn}

\begin{prp}{}{} Let $M$ be an $R$-module. Then there is a bijection between the set of all finite direct sum decompositions $M=\bigoplus_{i=1}^n M_i$ with all $M_i\neq 0$ and the set of all full orthogonal system of idempotents in $S=\text{End}_R(M)$. 
\end{prp}

Note that in particular, we can also take $M$ to just be $R$ to get a decomposition on idempotents by ideals of $R$. This means that for $\{e_1,\dots,e_n\}$ a full orthogonal system of idempotents, we have a decomposition $$R=Re_1\oplus\cdots\oplus Re_n$$

\begin{thm}{Peirce Decompositions}{} A full system of orthogonal idempotents in $R$ gives a direct sum decomposition of $R$ and $M$ into $\Z$-modules that can be written in matrix forms $$R=\bigoplus_{i,j=1}^n e_iRe_j=\begin{pmatrix}
e_1Re_1 & \cdots & e_1Re_n\\
\vdots & \ddots & \vdots\\
e_nRe_1 & \cdots & e_nRe_n
\end{pmatrix}$$ and $$M=\bigoplus_{i=1}^ne_iM=\begin{pmatrix}
e_1M\\
\vdots\\
e_nM
\end{pmatrix}$$ that satisfies the following: 
\begin{itemize}
\item If $R$ is an $\F$-algebra, all $e_iRe_j$ and $e_iM$ are $\F$-vector subspaces
\item The multiplication in $R$ defines the structure of a ring on each $e_iRe_j$. This ring is non-zero. 
\item The $R$-module action on $M$ defines a structure of $e_iRe_i$-module on $e_iM$
\item In the matrix interpretation, the multiplication in $R$ and the $R$ action on $M$ satisfies the standard matrix rules
\end{itemize}
\end{thm}

\subsection{Artin-Wedderburn Theorem}
\begin{thm}{Artin-Wedderburn Theorem}{} Let $R$ be a ring. Then the following are equivalent characterizations of semisimplicity. 
\begin{itemize}
\item Every left $R$-module is semisimple
\item The regular left $R$-module is semisimple
\item There exists $n_1,\dots,n_k\in\N$ and division rings $D_1,\dots,D_k$ such that $R$ is isomorphic to the direct product $\prod_{i=1}^kM_{n_i}(D_i)$
\end{itemize} \tcbline
\begin{proof}

\end{proof}
\end{thm}

\begin{crl}{}{} A ring is left semisimple if and only if it is right semisimple. 
\end{crl}

\begin{prp}{}{} The following are true regarding semisimple algebras over fields. 
\begin{itemize}
\item A semisimple $\C$-algebra of at most countable dimension is isomorphic to $$\prod_{i=1}^kM_{k_i}(\C)$$
\item A semisimple $\R$-algebra of at most countable dimension is isomorphic to $$\prod_{i=1}^kM_{k_i}(\R)\times\prod_{i=1}^nM_{n_i}(\C)\times\prod_{i=1}^tM_{t_i}(\H)$$
\item A finite dimensional semisimple $\F_q$ algebra is isomorphic to $$\prod_{i=1}^k M_{k_i}(\F_q^{t(i)})$$
\end{itemize}
\end{prp}

\subsection{Radical}
\begin{defn}{Cosimple}{} Let $M$ be an $R$-module. We say that a submodule $N$ of $M$ is cosimple if $\frac{M}{N}$ is simple. 
\end{defn}

\begin{lmm}{}{} Let $M$ be an $R$-module and $N$ a submodule of $M$. Then $N$ is cosimple if and only if $N$ is a maximal proper submodule of $M$. 
\end{lmm}

\begin{defn}{Radical}{} Let $M$ be an $R$-module. Define the radical of $M$ to be the intersection $$\text{rad}(M)=\bigcap_{\substack{S\leq M\\S\text{ is cosimple }}}S$$ of all cosimple submodules of $M$. 
\end{defn}

\begin{lmm}{}{} Let $M$ be an $R$-module. If $M$ is semisimple, then $\text{rad}(M)=0$. 
\end{lmm}

\begin{defn}{Artinian Modules}{} A left $R$-module $M$ is said to be Artinian if for every descending chain of submodules $$N_1\supseteq N_2\supseteq\cdots\supseteq N_n\supseteq\cdots$$ there exists $m\in\N$ such that $N_n=N_m$ for all $n>m$. 
\end{defn}

\begin{thm}{}{} Let $M$ be an Artinian left $R$-module. Then $M$ is semisimple if and only if $\text{rad}(M)=0$. 
\end{thm}

















\end{document}