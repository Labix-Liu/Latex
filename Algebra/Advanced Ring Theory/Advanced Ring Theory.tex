\documentclass[a4paper]{article}

\input{C:/Users/liula/Desktop/Latex/Headers V1.2.tex}

\pagestyle{fancy}
\fancyhf{}
\rhead{Labix}
\lhead{Advanced Ring Theory}
\rfoot{\thepage}

\title{Advanced Ring Theory}

\author{Labix}

\date{\today}
\begin{document}
\maketitle
\begin{abstract}
\begin{itemize}
\item Abstract Alebra by Thomas W. Judson
\end{itemize}
\end{abstract}
\pagebreak
\tableofcontents
\pagebreak

\section{Artinian Rings and Modules}
\subsection{Artinian Modules}
\begin{defn}{Artinian Modules}{} A left $R$-module $M$ is said to be Artinian if for every descending chain of submodules $$N_1\supseteq N_2\supseteq\cdots\supseteq N_n\supseteq\cdots$$ there exists $m\in\N$ such that $N_n=N_m$ for all $n>m$. 
\end{defn}

We have seen that if $M$ is semisimple then $\text{rad}(M)=0$. The converse holds if $M$ is Artinian. 

\begin{thm}{}{} Let $M$ be an Artinian left $R$-module. Then $M$ is semisimple if and only if $\text{rad}(M)=0$. \tcbline
\begin{proof}
Lemma 2.5.4 proves one direction. So suppose that $\text{rad}(M)=0$. Then we obtain a descending chain using intersections of cosimple submodules $$N_1\supseteq N_1\cap N_2\supseteq\cdots\text{rad}(M)=0$$ Since $M$ is Artinian, the chain stops after finitely many steps. Then this gives us finitely many cosimple modules $N_i$ such that $$N_1\cap\cdots\cap N_k=0$$ Consider the following homomorphism of $R$-modules $\psi:M\to\prod_{i=1}^k\frac{M}{N_i}$ defined by the individual projection homomorphism. It is injective since its kernel if $N_1\cap\cdots\cap N_k=0$. Since there are only finitely many submodules, together with surjectivity we have that $$M\cong\psi(M)\cong\bigoplus_{i=1}^k\frac{M}{N_i}$$ Thus $M$ is semisimple. 
\end{proof}
\end{thm}

\subsection{Artinian Rings}
\begin{defn}{Artinian Rings}{} Let $R$ be a ring. We say that $R$ is artinian if $R$ as a left $R$-module is an Artinian module. 
\end{defn}

The following is a result of theorem 1.1.2. 

\begin{crl}{}{} Let $R$ be a ring. Then $R$ is semisimple if and only if $R$ is left artinian and $J(R)=0$. \tcbline
\begin{proof}
\end{proof}
\end{crl}

\begin{thm}{}{} If $R$ is a left artinian ring, then $J(R)$ is nilpotent. 
\end{thm}

Recall that an ideal $I$ of a ring is nilpotent if $I^n=0$ for some $n\in\N$. 

\begin{crl}{}{} Let $R$ be a left artinian ring. Then the following are equivalent. 
\begin{itemize}
\item $J(R)$ is the largest nilpotent two sided ideal of $R$
\item $J(R)$ is the largest nilpotent left ideal of $R$
\item $J(R)$ is the largest nilpotent right ideal of $R$. 
\end{itemize}
\end{crl}



















\end{document}