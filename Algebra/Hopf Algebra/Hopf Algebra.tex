\documentclass[a4paper]{article}

%=========================================
% Packages
%=========================================
\usepackage{mathtools}
\usepackage{amsfonts}
\usepackage{amsmath}
\usepackage{amssymb}
\usepackage{amsthm}
\usepackage[a4paper, total={6in, 8in}, margin=1in]{geometry}
\usepackage[utf8]{inputenc}
\usepackage{fancyhdr}
\usepackage[utf8]{inputenc}
\usepackage{graphicx}
\usepackage{physics}
\usepackage[listings]{tcolorbox}
\usepackage{hyperref}
\usepackage{tikz-cd}
\usepackage{adjustbox}
\usepackage{enumitem}
\usepackage[font=small,labelfont=bf]{caption}
\usepackage{subcaption}
\usepackage{wrapfig}
\usepackage{makecell}



\raggedright

\usetikzlibrary{arrows.meta}

\DeclarePairedDelimiter\ceil{\lceil}{\rceil}
\DeclarePairedDelimiter\floor{\lfloor}{\rfloor}

%=========================================
% Fonts
%=========================================
\usepackage{tgpagella}
\usepackage[T1]{fontenc}


%=========================================
% Custom Math Operators
%=========================================
\DeclareMathOperator{\adj}{adj}
\DeclareMathOperator{\im}{im}
\DeclareMathOperator{\nullity}{nullity}
\DeclareMathOperator{\sign}{sign}
\DeclareMathOperator{\dom}{dom}
\DeclareMathOperator{\lcm}{lcm}
\DeclareMathOperator{\ran}{ran}
\DeclareMathOperator{\ext}{Ext}
\DeclareMathOperator{\dist}{dist}
\DeclareMathOperator{\diam}{diam}
\DeclareMathOperator{\aut}{Aut}
\DeclareMathOperator{\inn}{Inn}
\DeclareMathOperator{\syl}{Syl}
\DeclareMathOperator{\edo}{End}
\DeclareMathOperator{\cov}{Cov}
\DeclareMathOperator{\vari}{Var}
\DeclareMathOperator{\cha}{char}
\DeclareMathOperator{\Span}{span}
\DeclareMathOperator{\ord}{ord}
\DeclareMathOperator{\res}{res}
\DeclareMathOperator{\Hom}{Hom}
\DeclareMathOperator{\Mor}{Mor}
\DeclareMathOperator{\coker}{coker}
\DeclareMathOperator{\Obj}{Obj}
\DeclareMathOperator{\id}{id}
\DeclareMathOperator{\GL}{GL}
\DeclareMathOperator*{\colim}{colim}

%=========================================
% Custom Commands (Shortcuts)
%=========================================
\newcommand{\CP}{\mathbb{CP}}
\newcommand{\GG}{\mathbb{G}}
\newcommand{\F}{\mathbb{F}}
\newcommand{\N}{\mathbb{N}}
\newcommand{\Q}{\mathbb{Q}}
\newcommand{\R}{\mathbb{R}}
\newcommand{\C}{\mathbb{C}}
\newcommand{\E}{\mathbb{E}}
\newcommand{\Prj}{\mathbb{P}}
\newcommand{\RP}{\mathbb{RP}}
\newcommand{\T}{\mathbb{T}}
\newcommand{\Z}{\mathbb{Z}}
\newcommand{\A}{\mathbb{A}}
\renewcommand{\H}{\mathbb{H}}
\newcommand{\K}{\mathbb{K}}

\newcommand{\mA}{\mathcal{A}}
\newcommand{\mB}{\mathcal{B}}
\newcommand{\mC}{\mathcal{C}}
\newcommand{\mD}{\mathcal{D}}
\newcommand{\mE}{\mathcal{E}}
\newcommand{\mF}{\mathcal{F}}
\newcommand{\mG}{\mathcal{G}}
\newcommand{\mH}{\mathcal{H}}
\newcommand{\mI}{\mathcal{I}}
\newcommand{\mJ}{\mathcal{J}}
\newcommand{\mK}{\mathcal{K}}
\newcommand{\mL}{\mathcal{L}}
\newcommand{\mM}{\mathcal{M}}
\newcommand{\mO}{\mathcal{O}}
\newcommand{\mP}{\mathcal{P}}
\newcommand{\mS}{\mathcal{S}}
\newcommand{\mT}{\mathcal{T}}
\newcommand{\mV}{\mathcal{V}}
\newcommand{\mW}{\mathcal{W}}

%=========================================
% Colours!!!
%=========================================
\definecolor{LightBlue}{HTML}{2D64A6}
\definecolor{ForestGreen}{HTML}{4BA150}
\definecolor{DarkBlue}{HTML}{000080}
\definecolor{LightPurple}{HTML}{cc99ff}
\definecolor{LightOrange}{HTML}{ffc34d}
\definecolor{Buff}{HTML}{DDAE7E}
\definecolor{Sunset}{HTML}{F2C57C}
\definecolor{Wenge}{HTML}{584B53}
\definecolor{Coolgray}{HTML}{9098CB}
\definecolor{Lavender}{HTML}{D6E3F8}
\definecolor{Glaucous}{HTML}{828BC4}
\definecolor{Mauve}{HTML}{C7A8F0}
\definecolor{Darkred}{HTML}{880808}
\definecolor{Beaver}{HTML}{9A8873}
\definecolor{UltraViolet}{HTML}{52489C}



%=========================================
% Theorem Environment
%=========================================
\tcbuselibrary{listings, theorems, breakable, skins}

\newtcbtheorem[number within = subsection]{thm}{Theorem}%
{	colback=Buff!3, 
	colframe=Buff, 
	fonttitle=\bfseries, 
	breakable, 
	enhanced jigsaw, 
	halign=left
}{thm}

\newtcbtheorem[number within=subsection, use counter from=thm]{defn}{Definition}%
{  colback=cyan!1,
    colframe=cyan!50!black,
	fonttitle=\bfseries, breakable, 
	enhanced jigsaw, 
	halign=left
}{defn}

\newtcbtheorem[number within=subsection, use counter from=thm]{axm}{Axiom}%
{	colback=red!5, 
	colframe=Darkred, 
	fonttitle=\bfseries, 
	breakable, 
	enhanced jigsaw, 
	halign=left
}{axm}

\newtcbtheorem[number within=subsection, use counter from=thm]{prp}{Proposition}%
{	colback=LightBlue!3, 
	colframe=Glaucous, 
	fonttitle=\bfseries, 
	breakable, 
	enhanced jigsaw, 
	halign=left
}{prp}

\newtcbtheorem[number within=subsection, use counter from=thm]{lmm}{Lemma}%
{	colback=LightBlue!3, 
	colframe=LightBlue!60, 
	fonttitle=\bfseries, 
	breakable, 
	enhanced jigsaw, 
	halign=left
}{lmm}

\newtcbtheorem[number within=subsection, use counter from=thm]{crl}{Corollary}%
{	colback=LightBlue!3, 
	colframe=LightBlue!60, 
	fonttitle=\bfseries, 
	breakable, 
	enhanced jigsaw, 
	halign=left
}{crl}

\newtcbtheorem[number within=subsection, use counter from=thm]{eg}{Example}%
{	colback=Beaver!5, 
	colframe=Beaver, 
	fonttitle=\bfseries, 
	breakable, 
	enhanced jigsaw, 
	halign=left
}{eg}

\newtcbtheorem[number within=subsection, use counter from=thm]{ex}{Exercise}%
{	colback=Beaver!5, 
	colframe=Beaver, 
	fonttitle=\bfseries, 
	breakable, 
	enhanced jigsaw, 
	halign=left
}{ex}

\newtcbtheorem[number within=subsection, use counter from=thm]{alg}{Algorithm}%
{	colback=UltraViolet!5, 
	colframe=UltraViolet, 
	fonttitle=\bfseries, 
	breakable, 
	enhanced jigsaw, 
	halign=left
}{alg}




%=========================================
% Hyperlinks
%=========================================
\hypersetup{
    colorlinks=true, %set true if you want colored links
    linktoc=all,     %set to all if you want both sections and subsections linked
    linkcolor=DarkBlue,  %choose some color if you want links to stand out
}


\pagestyle{fancy}
\fancyhf{}
\rhead{Labix}
\lhead{Hopf Algebra}
\rfoot{\thepage}

\title{Hopf Algebra}

\author{Labix}

\date{\today}
\begin{document}
\maketitle
\begin{abstract}
\end{abstract}
\pagebreak
\tableofcontents

\pagebreak
\section{Algebras and Coalgebras}
\subsection{Coalgebras}
There is a need to revisit the definition of an algebra (over a field)

\begin{prp}{}{} A vector space $V$ over a field $k$ is an algebra if and only if there is a following collection of data: 
\begin{itemize}
\item A $k$-linear map $m:V\otimes V\to V$ called the multiplication map
\item An $k$-linear map $u:k\to V$ called the unital map
\end{itemize}
such that the following two diagrams are commutative: \\~\\
\adjustbox{scale=1.0,center}{\begin{tikzcd}
	{V\otimes V\otimes V} & {V\otimes V} && {k\otimes V} & {V\otimes V} \\
	{V\otimes V} & V && V & {V\otimes k}
	\arrow["{\text{id}\otimes m}", from=1-1, to=1-2]
	\arrow["{m\otimes\text{id}}"', from=1-1, to=2-1]
	\arrow["m", from=1-2, to=2-2]
	\arrow["{u\otimes\text{id}}", from=1-4, to=1-5]
	\arrow["\cong"', from=1-4, to=2-4]
	\arrow["m"{description}, from=1-5, to=2-4]
	\arrow["m"', from=2-1, to=2-2]
	\arrow["{\text{id}\otimes u}"', from=2-5, to=1-5]
	\arrow["\cong", from=2-5, to=2-4]
\end{tikzcd}}\\~\\
where the unnamed maps is the canonical isomorphisms. 
\end{prp}

Evidently, the map $\mu$ gives a multiplicative structure for $V$ and $\Delta$ gives the unitary structure of an algebra. The diagram on the left then represent associativity of multiplication. Notice that such additional structure on $V$ formally lives in the category $\bold{Vect}_k$ of vector spaces over a fixed field $k$. \\~\\

Therefore we can formally dualize all arrows to obtain a new object. 

\begin{defn}{Coalgebra}{} Let $V$ be a vector space over a field $k$. We say that $V$ is a coalgebra over $k$ if there is a collection of data: 
\begin{itemize}
\item A $k$-linear map $\Delta:V\to V\otimes V$ called the comultiplication map
\item An $k$-linear map $\varepsilon:V\to k$ called the counital map
\end{itemize} 
such that the following diagrams are commutative: \\~\\
\adjustbox{scale=1.0,center}{\begin{tikzcd}
	{V\otimes V\otimes V} & {V\otimes V} && {k\otimes V} & {V\otimes V} \\
	{V\otimes V} & V && V & {V\otimes k}
	\arrow["{\varepsilon\otimes\Delta}"', from=1-2, to=1-1]
	\arrow["{\varepsilon\otimes\text{id}}"', from=1-5, to=1-4]
	\arrow["{\text{id}\otimes\varepsilon}", from=1-5, to=2-5]
	\arrow["{\Delta\otimes\varepsilon}", from=2-1, to=1-1]
	\arrow["\Delta"', from=2-2, to=1-2]
	\arrow["\Delta", from=2-2, to=2-1]
	\arrow["\cong", from=2-4, to=1-4]
	\arrow["\Delta"{description}, from=2-4, to=1-5]
	\arrow["\cong"', from=2-4, to=2-5]
\end{tikzcd}}\\~\\
where the unnamed maps is the canonical isomorphisms. 
\end{defn}

\begin{lmm}{}{} Every vector space $V$ over a field $k$ can be given the structure of a coalgebra where 
\begin{itemize}
\item $\Delta:V\to V\otimes V$ is defined by $\Delta(v)=v\otimes v$
\item $\varepsilon:V\to k$ is defined by $\varepsilon(v)=1_k$
\end{itemize}

\end{lmm}

We would like to formally invert the definitions of algebra homomorphisms in order to define coalgebra homomorphisms. 

\subsection{}
Every coalgebra gives rise to an algebra, but not the other way. Such an assignment is moreover functorial. 

\subsection{Bialgebras}
\begin{defn}{Bialgebras}{} Let $V$ be a vector space over a field $k$. We say that $V$ is a bialgebra if there is a collection of data: 
\begin{itemize}
\item A $k$-linear map $m:V\otimes V\to V$ called the multiplication map
\item An $k$-linear map $u:k\to V$ called the unital map
\item A $k$-linear map $\Delta:V\to V\otimes V$ called the comultiplication map
\item An $k$-linear map $\varepsilon:V\to k$ called the counital map
\end{itemize}
such that $(V,m,u)$ is an algebra over $k$ and $(V,\Delta,\varepsilon)$ is a coalgebra over $k$ and that the following diagrams are commutative: \\~\\
\adjustbox{scale=1.0,center}{\begin{tikzcd}
	{V\otimes V} & V & {V\otimes V} && k && k \\
	{V\otimes V\otimes V\otimes V} && {V\otimes V\otimes V\otimes V} &&& V \\
	\\
	{V\otimes V} && V &&& {k\otimes k\cong k} \\
	& {k\otimes k\cong k} &&& {V\otimes V} && V
	\arrow["m", from=1-1, to=1-2]
	\arrow["{\Delta\otimes\Delta}"', from=1-1, to=2-1]
	\arrow["\Delta", from=1-2, to=1-3]
	\arrow["{\text{id}}", from=1-5, to=1-7]
	\arrow["u"', from=1-5, to=2-6]
	\arrow["{\text{id}\otimes\tau\otimes\text{id}}"', from=2-1, to=2-3]
	\arrow["{m\otimes m}"{description}, from=2-3, to=1-3]
	\arrow["\varepsilon"', from=2-6, to=1-7]
	\arrow["m", from=4-1, to=4-3]
	\arrow["{\varepsilon\otimes\varepsilon}"', from=4-1, to=5-2]
	\arrow["\varepsilon", from=4-3, to=5-2]
	\arrow["{u\otimes u}"', from=4-6, to=5-5]
	\arrow["u", from=4-6, to=5-7]
	\arrow["\Delta", from=5-7, to=5-5]
\end{tikzcd}}\\~\\
where $\tau:V\otimes V\to V\otimes V$ is the commutativity map defined by $\tau(x\otimes y)=y\otimes x$. 
\end{defn}

\begin{thm}{}{} Let $V$ be a vector space over $k$. Suppose that $(V,m,u)$ is an algebra and $(V,\Delta,\varepsilon)$ is a coalgebra. Then the following conditions are equivalent. 
\begin{itemize}
\item $(V,m,u,\Delta,\varepsilon)$ is a bialgebra
\item $m:V\otimes V\to V$ and $u:k\to V$ are coalgebra homomorphisms
\item $\Delta:V\to V\otimes V$ and $\varepsilon:V\to k$ are algebra homomorphisms
\end{itemize}
\end{thm}

\pagebreak
\section{Hopf Algebras}
\subsection{}
\begin{defn}{Hopf Algebra}{} Let $(H,m,u,\Delta,\varepsilon)$ be a bialgebra. We say that $H$ is a Hopf algebra if there is a $k$-linear map $S:H\to H$ called the antipode such that the following diagram commutes: \\~\\
\adjustbox{scale=1.0,center}{\begin{tikzcd}
	& {H\otimes H} && {H\otimes H} \\
	H && k && H \\
	& {H\otimes H} && {H\otimes H}
	\arrow["{S\otimes\text{id}}", from=1-2, to=1-4]
	\arrow["m", from=1-4, to=2-5]
	\arrow["\Delta", from=2-1, to=1-2]
	\arrow["\varepsilon", from=2-1, to=2-3]
	\arrow["\Delta"', from=2-1, to=3-2]
	\arrow["u", from=2-3, to=2-5]
	\arrow["{\text{id}\otimes S}"', from=3-2, to=3-4]
	\arrow["m"', from=3-4, to=2-5]
\end{tikzcd}}\\~\\
\end{defn}




















\end{document}