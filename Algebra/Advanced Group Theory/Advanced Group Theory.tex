\documentclass[a4paper]{article}

%=========================================
% Packages
%=========================================
\usepackage{mathtools}
\usepackage{amsfonts}
\usepackage{amsmath}
\usepackage{amssymb}
\usepackage{amsthm}
\usepackage[a4paper, total={6in, 8in}, margin=1in]{geometry}
\usepackage[utf8]{inputenc}
\usepackage{fancyhdr}
\usepackage[utf8]{inputenc}
\usepackage{graphicx}
\usepackage{physics}
\usepackage[listings]{tcolorbox}
\usepackage{hyperref}
\usepackage{tikz-cd}
\usepackage{adjustbox}
\usepackage{enumitem}
\usepackage[font=small,labelfont=bf]{caption}
\usepackage{subcaption}
\usepackage{wrapfig}
\usepackage{makecell}



\raggedright

\usetikzlibrary{arrows.meta}

\DeclarePairedDelimiter\ceil{\lceil}{\rceil}
\DeclarePairedDelimiter\floor{\lfloor}{\rfloor}

%=========================================
% Fonts
%=========================================
\usepackage{tgpagella}
\usepackage[T1]{fontenc}


%=========================================
% Custom Math Operators
%=========================================
\DeclareMathOperator{\adj}{adj}
\DeclareMathOperator{\im}{im}
\DeclareMathOperator{\nullity}{nullity}
\DeclareMathOperator{\sign}{sign}
\DeclareMathOperator{\dom}{dom}
\DeclareMathOperator{\lcm}{lcm}
\DeclareMathOperator{\ran}{ran}
\DeclareMathOperator{\ext}{Ext}
\DeclareMathOperator{\dist}{dist}
\DeclareMathOperator{\diam}{diam}
\DeclareMathOperator{\aut}{Aut}
\DeclareMathOperator{\inn}{Inn}
\DeclareMathOperator{\syl}{Syl}
\DeclareMathOperator{\edo}{End}
\DeclareMathOperator{\cov}{Cov}
\DeclareMathOperator{\vari}{Var}
\DeclareMathOperator{\cha}{char}
\DeclareMathOperator{\Span}{span}
\DeclareMathOperator{\ord}{ord}
\DeclareMathOperator{\res}{res}
\DeclareMathOperator{\Hom}{Hom}
\DeclareMathOperator{\Mor}{Mor}
\DeclareMathOperator{\coker}{coker}
\DeclareMathOperator{\Obj}{Obj}
\DeclareMathOperator{\id}{id}
\DeclareMathOperator{\GL}{GL}
\DeclareMathOperator*{\colim}{colim}

%=========================================
% Custom Commands (Shortcuts)
%=========================================
\newcommand{\CP}{\mathbb{CP}}
\newcommand{\GG}{\mathbb{G}}
\newcommand{\F}{\mathbb{F}}
\newcommand{\N}{\mathbb{N}}
\newcommand{\Q}{\mathbb{Q}}
\newcommand{\R}{\mathbb{R}}
\newcommand{\C}{\mathbb{C}}
\newcommand{\E}{\mathbb{E}}
\newcommand{\Prj}{\mathbb{P}}
\newcommand{\RP}{\mathbb{RP}}
\newcommand{\T}{\mathbb{T}}
\newcommand{\Z}{\mathbb{Z}}
\newcommand{\A}{\mathbb{A}}
\renewcommand{\H}{\mathbb{H}}
\newcommand{\K}{\mathbb{K}}

\newcommand{\mA}{\mathcal{A}}
\newcommand{\mB}{\mathcal{B}}
\newcommand{\mC}{\mathcal{C}}
\newcommand{\mD}{\mathcal{D}}
\newcommand{\mE}{\mathcal{E}}
\newcommand{\mF}{\mathcal{F}}
\newcommand{\mG}{\mathcal{G}}
\newcommand{\mH}{\mathcal{H}}
\newcommand{\mI}{\mathcal{I}}
\newcommand{\mJ}{\mathcal{J}}
\newcommand{\mK}{\mathcal{K}}
\newcommand{\mL}{\mathcal{L}}
\newcommand{\mM}{\mathcal{M}}
\newcommand{\mO}{\mathcal{O}}
\newcommand{\mP}{\mathcal{P}}
\newcommand{\mS}{\mathcal{S}}
\newcommand{\mT}{\mathcal{T}}
\newcommand{\mV}{\mathcal{V}}
\newcommand{\mW}{\mathcal{W}}

%=========================================
% Colours!!!
%=========================================
\definecolor{LightBlue}{HTML}{2D64A6}
\definecolor{ForestGreen}{HTML}{4BA150}
\definecolor{DarkBlue}{HTML}{000080}
\definecolor{LightPurple}{HTML}{cc99ff}
\definecolor{LightOrange}{HTML}{ffc34d}
\definecolor{Buff}{HTML}{DDAE7E}
\definecolor{Sunset}{HTML}{F2C57C}
\definecolor{Wenge}{HTML}{584B53}
\definecolor{Coolgray}{HTML}{9098CB}
\definecolor{Lavender}{HTML}{D6E3F8}
\definecolor{Glaucous}{HTML}{828BC4}
\definecolor{Mauve}{HTML}{C7A8F0}
\definecolor{Darkred}{HTML}{880808}
\definecolor{Beaver}{HTML}{9A8873}
\definecolor{UltraViolet}{HTML}{52489C}



%=========================================
% Theorem Environment
%=========================================
\tcbuselibrary{listings, theorems, breakable, skins}

\newtcbtheorem[number within = subsection]{thm}{Theorem}%
{	colback=Buff!3, 
	colframe=Buff, 
	fonttitle=\bfseries, 
	breakable, 
	enhanced jigsaw, 
	halign=left
}{thm}

\newtcbtheorem[number within=subsection, use counter from=thm]{defn}{Definition}%
{  colback=cyan!1,
    colframe=cyan!50!black,
	fonttitle=\bfseries, breakable, 
	enhanced jigsaw, 
	halign=left
}{defn}

\newtcbtheorem[number within=subsection, use counter from=thm]{axm}{Axiom}%
{	colback=red!5, 
	colframe=Darkred, 
	fonttitle=\bfseries, 
	breakable, 
	enhanced jigsaw, 
	halign=left
}{axm}

\newtcbtheorem[number within=subsection, use counter from=thm]{prp}{Proposition}%
{	colback=LightBlue!3, 
	colframe=Glaucous, 
	fonttitle=\bfseries, 
	breakable, 
	enhanced jigsaw, 
	halign=left
}{prp}

\newtcbtheorem[number within=subsection, use counter from=thm]{lmm}{Lemma}%
{	colback=LightBlue!3, 
	colframe=LightBlue!60, 
	fonttitle=\bfseries, 
	breakable, 
	enhanced jigsaw, 
	halign=left
}{lmm}

\newtcbtheorem[number within=subsection, use counter from=thm]{crl}{Corollary}%
{	colback=LightBlue!3, 
	colframe=LightBlue!60, 
	fonttitle=\bfseries, 
	breakable, 
	enhanced jigsaw, 
	halign=left
}{crl}

\newtcbtheorem[number within=subsection, use counter from=thm]{eg}{Example}%
{	colback=Beaver!5, 
	colframe=Beaver, 
	fonttitle=\bfseries, 
	breakable, 
	enhanced jigsaw, 
	halign=left
}{eg}

\newtcbtheorem[number within=subsection, use counter from=thm]{ex}{Exercise}%
{	colback=Beaver!5, 
	colframe=Beaver, 
	fonttitle=\bfseries, 
	breakable, 
	enhanced jigsaw, 
	halign=left
}{ex}

\newtcbtheorem[number within=subsection, use counter from=thm]{alg}{Algorithm}%
{	colback=UltraViolet!5, 
	colframe=UltraViolet, 
	fonttitle=\bfseries, 
	breakable, 
	enhanced jigsaw, 
	halign=left
}{alg}




%=========================================
% Hyperlinks
%=========================================
\hypersetup{
    colorlinks=true, %set true if you want colored links
    linktoc=all,     %set to all if you want both sections and subsections linked
    linkcolor=DarkBlue,  %choose some color if you want links to stand out
}


\pagestyle{fancy}
\fancyhf{}
\rhead{Labix}
\lhead{Advanced Group Theory}
\rfoot{\thepage}

\title{Advanced Group Theory}

\author{Labix}

\date{\today}
\begin{document}
\maketitle
\begin{abstract}
Leading off from Groups and Rings, these notes deal with the more advanced objects of study in the theory of groups. In particular, we will focus on finite group theory. However, in the first half we present useful result on other topics of mathematics that rely on group theory such as algebraic topology and representation theory. 
\end{abstract}
\pagebreak
\tableofcontents
\pagebreak

\section{Free Groups and Free Product of Groups}
\subsection{Free Groups}
\begin{defn}{The Words of a Set}{} Let $S$ be a set. Let $T=S\times\{0,1\}$. A word of $S$ of length $n\in\N\setminus\{0\}$ is a map of sets $\omega:\{1,\dots,n\}\to T$. Such a word is denoted by $$s_1,\dots,s_n$$ where $s_k=\omega(k)$. Define the empty word to be the unique map $\emptyset\to T$. 
\end{defn}

We think of $S\times\{0\}$ as the original copy of $S$, and $S\times\{1\}$ as formally adjoining inverse elements so that we can form a group. 

\begin{defn}{Reduced Words}{} Let $S$ be a set. Define the projection maps $p_S:S\times\{0,1\}\to S$ and $p_{\{0,1\}}:S\times\{0,1\}\to\{0,1\}$ by $p_S(s,i)=s$ and $p_{\{0,1\}}(s,i)=i$ respectively. Let $s=s_1\cdots s_n$ be a word of length $n$. We say that $s$ is reduced if for $1\leq i<n$, $p_S(s_i)=p_S(s_{i+1})$ implies that $p_{\{0,1\}}(s_i)=p_{\{0,1\}}(s_{i+1})$. By convention the empty word is reduced. 
\end{defn}

Intuitively, since $(s,1)$ for $s\in S$ represents the inverse of $(s,0)$, saying that $s_1\cdots s_n$ is reduced means that no element is next to its inverse. 

\begin{defn}{Set of Reduced Words}{} Let $S$ be a set. Define the set of reduced words by $$F_S=\{w\;|\;w\text{ is a reduced word of }S\}$$
\end{defn}

\begin{prp}{}{} Let $S$ be a set. Recursively define a product $\ast:F_S\times F_S\to F_S$ on the set of reduced words as follows $$(s_1\cdots s_m)\ast(t_1\cdots t_n)=\begin{cases}
s\cdots s_{m-1}\ast t_2\cdots t_n & \text{ if }p_S(s_m)=p_S(t_1)\text{ and }p_{\{0,1\}}(s_m)\neq p_{\{0,1\}}(t_1)\\
s_1\cdots s_mt_1\cdots t_n& \text{ otherwise }
\end{cases}$$
Then $F_S$ is a group whose binary operation is given by the above product. \tcbline
\begin{proof}
We clearly have the identity $1$ and the inverse of $g=g_1\cdots g_n$ to be $g_n^{-1}\cdots g_1^{-1}$. We just have to check associativity and closure. By definition of the free product, the free product of reduced words is again a reduced word by definition and finiteness of the length of a reduced word. \\~\\
For any $g\in G_i$, define the action of left multiplication by $L_g(h)=g\ast h$ for any $h\in\ast_iG_i$. We clearly have $L_{g_1}\circ L_{g_2}=L_{g_1g_2}$ and $L_{g^{-1}}=L_g^{-1}$ and thus $L_g\in\text{Sym}(\ast_iG_i)$. For any word $g=g_1\cdots g_m$, the map $L:\ast_iG_i\to\text{Sym}(\ast_iG_i)$ is injective since for any $g\in\ast_iG_i$, we have $L_g(1)=g$ which means that $g\neq h$ implies $L_g\neq L_h$. Thus we have embedded the set into the group $\text{Sym}(\ast_iG_i)$ so that associativity follows. 
\end{proof}
\end{prp}

\begin{defn}{The Free Group on a Set}{} Let $S$ be a set. Define the free group on $S$ to be the set of reduced words $$F_S=\{w\;|\;w\text{ is a reduced word of }S\}$$ together with the binary operation given by the above product. 
\end{defn}

\begin{prp}{Universal Property of Free Groups}{} Let $S$ be a set. Then the free group $F_S$ and the inclusion map $\iota:S\to F_S$ satisfies the following universal property. For any group $G$ and any map of sets $\psi:S\to G$, there exists a unique group homomorphism $\phi:F_S\to G$ such that the following diagram commutes: \\~\\
\adjustbox{scale=1.0,center}{\begin{tikzcd}
	S & {F_S} \\
	& G
	\arrow["\iota", hook, from=1-1, to=1-2]
	\arrow["\psi"', from=1-1, to=2-2]
	\arrow["{\exists!\phi}", dashed, from=1-2, to=2-2]
\end{tikzcd}} \\~\\
Moreover, $F_S$ is the unique group (up to unique isomorphism) that has such a property. \tcbline
\end{prp}

\begin{prp}{}{} Let $S,T$ be sets. Then $F_S\cong F_T$ as groups if and only if $S$ and $T$ are in bijection. 
\end{prp}

\begin{defn}{Free Group of Rank n}{} Let $n\in\N$. Define the free group of rank $n$ to be the free group $F_n$ on the set $\{1,\dots,n\}$. 
\end{defn}

\begin{prp}{}{} Let $G$ be a group. Then there exists a set $S$ and a normal subgroup $N$ of $F_S$ such that there is a group isomorphism $G\cong F_S/N$. 
\end{prp}

\subsection{Free Products of Groups}
\begin{defn}{The Words of a Collection of Groups}{} Let $\{G_i\;|\;i\in I\}$ be a collection of groups. A word on these groups is a finite sequence $g_1\cdots g_m$ where $g_k\in G_i$ for some $i\in I$. We say that such a word has length $m$. 
\end{defn}

\begin{defn}{Reduced Words}{} A word $g_1\cdots g_m$ is said to be reduced if 
\begin{itemize}
\item $g_k\neq 1_i$ for any $i\in I$ and for $k\in\{1,\dots,m\}$
\item For any two consecutive letters $g_i$ and $g_{i+1}$, they are not in the same group
\end{itemize}
\end{defn}

The motrivation for reduced words is to allow us to work with less amount of words since every word can be reduced by removing the identities and replacing two consecutive elements in the same group with their product in that group. 

\begin{prp}{}{} Let $\{G_i\;|\;i\in I\}$ be a collection of groups. Recursively define a product on the set of reduced words as follows. $$(g_1\cdots g_m)\ast(h_1\cdots h_n)=\begin{cases}
g_1\cdots g_mh_1\cdots h_n& \text{ if $g_m,h_1$ are not in the same group }\\
g\cdots g_{m-1}\ast g_{m+1}\ast h_2\cdots h_n & \substack{\text{ if }g_m,h_1\in G_i \text{ for some }i\in I\\\text{and }g_mh_1=g_{m+1}\neq 1}\\
g_1\cdots g_{m-1}\ast h_2,\cdots h_n & \text{ if $g_mh_1\in G_i$ for some $i\in I$ and $g_mh_1=1$}
\end{cases}$$
Then $F_S$ is a group with the above operation. \tcbline
\begin{proof}
We clearly have the identity $1$ and the inverse of $g=g_1\cdots g_n$ to be $g_n^{-1}\cdots g_1^{-1}$. We just have to check associativity and closure. By definition of the free product, the free product of reduced words is again a reduced word by definition and finiteness of the length of a reduced word. \\~\\
For any $g\in G_i$, define the action of left multiplication by $L_g(h)=g\ast h$ for any $h\in\ast_iG_i$. We clearly have $L_{g_1}\circ L_{g_2}=L_{g_1g_2}$ and $L_{g^{-1}}=L_g^{-1}$ and thus $L_g\in\text{Sym}(\ast_iG_i)$. For any word $g=g_1\cdots g_m$, the map $L:\ast_iG_i\to\text{Sym}(\ast_iG_i)$ is injective since for any $g\in\ast_iG_i$, we have $L_g(1)=g$ which means that $g\neq h$ implies $L_g\neq L_h$. Thus we have embedded the set into the group $\text{Sym}(\ast_iG_i)$ so that associativity follows. 
\end{proof}
\end{prp}

The motivation is that we want to continue reducing the product of the word so that it can be fully reduced again. 

\begin{defn}{The Free Product of Groups}{} Let $\{G_i\;|\;i\in I\}$ be a collection of groups. Define the free product of groups to be the set $$\ast_i G_i=\{w\;|\;w\text{ is a reduced word of }\{G_i\;|\;i\in I\}\}$$ together with the above product $\ast$. 
\end{defn}

Note: If $\{G_i\}$ is just a collection of two groups then we write the free product as $G_1\ast G_2$. \\

\begin{prp}{}{} Let $\{G_i\;|\;i\in I\}$ be a collection of groups. Then $G_k\leq\ast_iG_i$ for any $k\in I$. \tcbline
\begin{proof}
Clearly every element of $G_i$ is in $\ast_iG_i$ since we can just treat each element as a word. 
\end{proof}
\end{prp}

\begin{prp}{Universal Property of The Free Product}{} Let $\{G_i\;|\;i\in I\}$ be a collection of groups. Then the free product $\ast_i G_i$ and the inclusion maps $\iota_k:G_k\to\ast_i G_i$ satisfies the following universal property. For any collection of group homomorphisms $\{\phi_i:G_i\to H\;|\;i\in I\}$, there exists a unique map $$\ast_i\phi_i:\ast_i G_i\to H$$ and such that the following diagram commutes: \\~\\
\adjustbox{scale=1.0,center}{\begin{tikzcd}
G_k\arrow[r, "\lambda_k"]\arrow[rd, "\phi_k"'] & \ast_i G_i\arrow[d, "\exists!\ast_i\phi_i"]\\
& H
\end{tikzcd}} \\~\\
Moreover, $\ast_i G_i$ is the unique group (up to unique isomorphism) that has such a property. \tcbline
\begin{proof}
Define $\ast_i\phi_i:\ast_iG_i\to H$ by $\ast_i\phi_i(g_1\cdots g_m)=\phi_{i_1}(g_1)\cdots\phi_{i_m}(g_m)$ where we assume that each $g_k\in G_{i_k}$. Then clearly we have that $(\ast_i\phi_i)\circ j_i=\phi_i$. Moreover, since $\phi_i$ are group homomorphisms, we have that $\phi_i(g_k)\phi_i(g_{k+1})=\phi_i(g_kg_{k+1})$ and $\phi_i(1)=1$ and thus $\phi_i$ is compatible with reduced words. This means that $\ast_i\phi_i$ is a group homomorphism and thus proving the existence part of the statement. \\~\\

Now suppose that $A$ is such a group that satisfies the above property. Consider the following two diagrams: \\~\\
\adjustbox{scale=1.0,center}{\begin{tikzcd}
	& {\ast_i G_i} &&& {\ast_iG_i} \\
	{G_k} & A && {G_k} \\
	& {\ast_iG_i} &&& {\ast_iG_i}
	\arrow["{\exists!\ast_i\lambda_i}", dashed, from=1-2, to=2-2]
	\arrow["{\exists!=\text{id}}", dashed, from=1-5, to=3-5]
	\arrow["{\iota_k}", from=2-1, to=1-2]
	\arrow["{\lambda_k}"', from=2-1, to=2-2]
	\arrow["{\iota_k}"', from=2-1, to=3-2]
	\arrow["{\exists!\psi}", dashed, from=2-2, to=3-2]
	\arrow["{\iota_k}", from=2-4, to=1-5]
	\arrow["{\iota_k}"', from=2-4, to=3-5]
\end{tikzcd}} \\~\\
Since all the constructed maps are unique such that the diagram commutes, we conclude that $\psi\circ\ast_i\lambda_i=\text{id}$. Similarly, by considering the following diagrams: \\~\\
\adjustbox{scale=1.0,center}{\begin{tikzcd}
	& A &&& A \\
	{G_k} & {\ast_iG_i} && {G_k} \\
	& A &&& A
	\arrow["{\exists!\psi}", dashed, from=1-2, to=2-2]
	\arrow["{\exists!=\text{id}}", dashed, from=1-5, to=3-5]
	\arrow["{\lambda_k}", from=2-1, to=1-2]
	\arrow["{\iota_k}"', from=2-1, to=2-2]
	\arrow["{\lambda_k}"', from=2-1, to=3-2]
	\arrow["{\exists!\ast_i\lambda_i}", dashed, from=2-2, to=3-2]
	\arrow["{\lambda_k}", from=2-4, to=1-5]
	\arrow["{\lambda_k}"', from=2-4, to=3-5]
\end{tikzcd}} \\~\\
We conclude that $\ast_i\lambda_i\circ\psi=\text{id}$. Hence $\ast_iG_i$ and $A$ are isomorphic. Moreover, the isomorphism is unique since all maps constructed are unique such that their respective diagrams commute. 
\end{proof}
\end{prp}

\subsection{Normal Closure}
\begin{defn}{Normal Closure}{} Let $G$ be a group. Let $S\subseteq G$ be a subset of $G$. Define the normal closure of $S$ in $G$ to be the subgroup $$\langle S^G\rangle=\langle g^{-1}sg\;|\;\text{ for all }s\in S\text{ and }g\in G\rangle$$
\end{defn}

\begin{lmm}{}{} Let $G$ be a group. Let $S\subseteq G$ be a subset of $G$. Then $\langle S^G\rangle$ is a normal subgroup of $G$. 
\end{lmm}

\begin{lmm}{}{} Let $G$ be a group. Let $S\subseteq G$ be a subset of $G$. Then $$\langle S^G\rangle=\bigcap_{S\subseteq N\trianglelefteq G}N$$
\end{lmm}

\begin{lmm}{}{} Let $G$ be a group. Let $N\trianglelefteq G$ be a normal subgroup of $G$. Then $\langle N^G\rangle=N$. 
\end{lmm}

\begin{defn}{Normally Generated Groups}{} Let $G$ be a group. We say that $G$ is normally generated if there exists a subset $S\subseteq G$ such that $$G=\langle S^G\rangle$$
\end{defn}

\subsection{Presentations of a Group}
\begin{defn}{Group of Generators and Relations}{} Let $S$ be a set. Let $F_S$ be the free group on $S$. Let $R$ be a set of words on $S$. Define the group with generators $S$ and relations $R$ to be the quotient $$\langle S\;|\;R\rangle=\frac{F_S}{\langle R^{F_S}\rangle}$$ Elements of $S$ are called generators and elements of $R$ are called relations. 
\end{defn}

\begin{prp}{Universal Property of the Group of Generators and Relations}{} Let $S$ be a set. Let $R$ be a set of words on $S$. Then the group $\langle S\;|\;R\rangle$ and the inclusion map $\iota:S\hookrightarrow\langle S\;|\;R\rangle$ satisfies the following universal property. \\~\\

For any group $G$ and any map of sets $f:S\to G$ for which the induced map $\overline{f}:F_S\to G$ by the universal property of free groups is such that $f(r)=1_G$ for all $r\in R$, there exists a unique group homomorphism $\phi:\langle S\;|\;R\rangle\to G$ such that the following diagram commutes: \\~\\
\adjustbox{scale=1.0,center}{\begin{tikzcd}
	S & {\langle S\;|\;R\rangle} \\
	& G
	\arrow["\iota", from=1-1, to=1-2]
	\arrow["f"', from=1-1, to=2-2]
	\arrow["{\exists!\phi}", dashed, from=1-2, to=2-2]
\end{tikzcd}} \\~\\
Moreover, $\langle S\;|\;R\rangle$ is the unique group (up to unique isomorphism) that has such a property. 
\end{prp}

\begin{defn}{Presentations of a Group}{} Let $G$ be a group. A presentation of $G$ is a group isomorphism $$G\cong\langle S\;|\;R\rangle$$ for some set $S$ and some set of words $R$ of $S$. 
\end{defn}

\begin{prp}{}{} Every group has a presentation. \tcbline
\begin{proof}
This follows from the fact that every group is the quotient of a free group. 
\end{proof}
\end{prp}

\begin{defn}{Finitely Presented Groups}{} Let $G$ be a group. We say that $G$ is finitely presented if there exists a presentation $$G\cong\langle S\;|\;R\rangle$$ of $G$ such that $S$ is a finite set. 
\end{defn}

\subsection{Amalgamated Products}
\begin{defn}{Amalgamted Products}{} Let $G,H,K$ be groups. Let $\phi:K\to G$ and $\psi:K\to H$ be injective group homomorphisms. Define the amalgamated product of $G$ and $H$ over $K$ to be the quotient of the free product $$G\ast_KH=\frac{G\ast H}{\langle\{\phi(k)(\psi(k))^{-1}\;|\;k\in K\}^G\rangle}$$
\end{defn}

\pagebreak
\section{Groups and Generators}
\subsection{Generated Groups}
Generated Subgroups generalizes the notion of cyclic subgroups in the sense that there are now more than one generators for constructing the group. 

\begin{defn}{Subgroups Generated by a Set}{} Let $G$ be a group and $A$ a subset of $G$. Define the subgroup generated by $A$ to be $$\langle A\rangle=\{a_1^{e_1}\cdots a_n^{e_n}|n\in\N^+, a_1,\dots,a_n\in A, e_1,\dots,e_n\in\{\pm1\}\}$$ the subgroup of all elements of $G$ that can be expressed as the finite product of elements in $A$ and their inverses. 
\end{defn}

If $A=\{a_1,\dots,a_r\}$ is finite, we often write the generated subgroup as $\langle A\rangle=\langle a_1,\dots,a_r\rangle$, omitting the brackets. \\

Recall that the cyclic subgroup is the smallest subgroup containing the generator. Then generated subgroups also generalizes this particular property. In particular, we have the following identity. 

\begin{prp}{}{} Let $G$ be a group and $A$ a subset of $G$. Then we have that $$\langle A\rangle=\bigcap_{A\subseteq H\leq G}H$$
\end{prp}

This means that the generated subgroup is the smallest subgroup of $G$ containing $A$. 

\begin{defn}{Generating Set of a Group}{} Let $G$ be a group and $S$ a subset of $G$. We say that $S$ is a generating set of $G$ if $G=\langle S\rangle$. 
\end{defn}

\begin{lmm}{}{} Let $S$ be a set. Then $S$ is a generating set of the free group $F_S$ on $S$. 
\end{lmm}

\subsection{Finitely Generated Groups}
\begin{defn}{Finitely Generated Groups}{} A group $G$ is finitely generated if there is a finite subset $A=\{g_1,\dots,g_n\}$ of $G$ such that $G=\langle A\rangle$. 
\end{defn}

\begin{prp}{}{} Let $G$ be a finitely generated group. Let $N$ be a normal subgroup of $G$. Then $G/N$ is finitely generated. 
\end{prp}

\begin{prp}{}{} Let $G$ be a group. Then $G$ is finitely generated if and only if there exists $n\in\N$ such that $G$ is isomorphic to the quotient of the free group $F_n$ of rank $n$ with a normal subgroup $N$ of $F_n$, written as $$G\cong\frac{F_n}{N}$$
\end{prp}

\begin{prp}{}{} Let $G$ be a group. If $G$ is finitely presented, then $G$ is finitely generated. 
\end{prp}

The converse is in general not true. Examples: Higgs group, Lamplighter group. 

\pagebreak
\section{More Types of Groups}
\subsection{Torsion and Torsion-Free Groups}
\begin{defn}{Torsion Group}{} A group $G$ is said to be a torsion group if every element of $G$ has finite order. A group $G$ is called torsion free if every element of $G$ except the identity has infinite order. 
\end{defn}

\begin{defn}{Torsion Subgroup}{} Let $G$ be an abelian group. The torsion subgroup of $G$ is the subgroup of $G$ consisting of all elements of finite order. 
\end{defn}

\subsection{$p$-groups}
\begin{defn}{$p$-groups}{} A group of order $p^k$ for some $k>0$ is called a p-group. Subgroups of any group that is a p-group are called p-subgroups. 
\end{defn}

\subsection{Simple Groups}
\begin{defn}{Simple Groups}{} We say that a group $G$ is simple if its only normal subgroups are the trivial group and $G$. 
\end{defn}

\subsection{Abelianization and Commutators}
Commutators are useful for checking whether a given group is cyclic or not. 

\begin{defn}{Commutators}{} Let $g,h\in G$. Define the commutator of $g$ and $h$ to be $$[g,h]=ghg^{-1}h^{-1}$$
\end{defn}

\begin{defn}{Commutator Subgroup}{} Let $G$ be a group. Define the commutator subgroup of $G$ to be $$[G,G]=\langle [g,h]|g,h\in G\rangle$$
\end{defn}

Note that in general, elements of $[G,G]$ are products of several commutators because we are taking the group generated by all the commutators. In general, the set of all commutators $\{[g,h]\;|\;g\in G, h\in H\}$ is not a group. 

\begin{prp}{}{} Let $G$ be a group. Then the following hold for the commutator subgroup of $G$. 
\begin{itemize}
\item $[G,G]\trianglelefteq G$
\item $G/[G,G]$ is abelian
\item If $N\trianglelefteq G$ and $G/N$ are abelian, then $[G,G]\leq N$. 
\end{itemize} \tcbline
\begin{proof}~\\
\begin{itemize}
\item Let $g,h,k\in G$. Then we have 
\begin{align*}
k[g,h]k^{-1}&=kghg^{-1}h^{-1}k^{-1}\\
&=\left(kgk^{-1}\right)\left(khk^{-1}\right)\left(kg^{-1}k^{-1}\right)\left(kh^{-1}k^{-1}\right)\\
&=[kgk^{-1},khk^{-1}]\in[G,G]
\end{align*}
So for $[g_1,h_1],\dots,[g_n,h_n]\in[G,G]$, we have $$k[g_1,h_1]\cdots[g_n,h_n]k^{-1}=\left(k[g_1,h_1]k^{-1}\right)\cdots\left(k[g_n,h_n]k^{-1}\right)\in[G,G]$$
\item Let $g,h\in G$. Then we have 
\begin{align*}
ghg^{-1}h^{-1}\in[G,G]&\implies [G,G]ghg^{-1}h^{-1}=[G,G]\\
&\implies [G,G]gh=[G,G]hg\\
&\implies [G,G]g\cdot[G,G]h=[G,G]h\cdot[G,G]g
\end{align*}
So $G/[G,G]$ is abelian. 
\item Suppose that $G/N$ is abelian. Then $Ng\cdot Nh=Nh\cdot Ng$ for all $g,h\in G$. Thus $$N(ghg^{-1}h^{-1})=N$$ and so $[g,h]\in N$. Since $N$ is a subgroup of $G$, we must have that $[G,G]\leq N$. 
\end{itemize}
And so we conclude. 
\end{proof}
\end{prp}

\begin{defn}{Abelianization}{} Let $G$ be a group. Define the abelianization of $G$ to be the abelian group $$G^\text{ab}=\frac{G}{[G,G]}$$
\end{defn}

\begin{prp}{}{} A group $G$ is abelian if and only if $G=G^\text{ab}$. \tcbline
\begin{proof}
It suffices to show that $G$ is abelian if and only if $[G,G]=1$. We know that if $G$ is abelian, then every commutator reduces to $$[g,h]=ghg^{-1}h^{-1}=ghh^{-1}g^{-1}=1$$ so $[G,G]=1$. Now suppose that $[G,G]=1$. Then this implies that $[g,h]=ghg^{-1}h^{-1}=1$ which shows that $gh=hg$ for any $g,h\in G$. Thus $G$ is abelian. 
\end{proof}
\end{prp}

\begin{defn}{Perfect Groups}{} Let $G$ be a group. We say that $G$ is perfect if $[G,G]=G$. 
\end{defn}

\pagebreak
\section{Free Abelian Groups}
\subsection{Direct Sums of Abelian Groups}
\begin{defn}{Direct Sum of Abelian Groups}{} Let $\{G_i\;|\;i\in I\}$ be a collection of abelian group. Define the direct sum of the abelian groups to be the set $$\bigoplus_{i\in I}G_i=\{(g_i)_{i\in I}\in\prod_{i\in I}G_i\;|\;\text{Only finitely many }g_i\text{ is non-zero}\}$$ together with the product of their binary operations. 
\end{defn}

\begin{lmm}{}{} Let $\{G_i\;|\;i\in I\}$ be a collection of abelian group. Then $\bigoplus_{i\in I}G_i$ is a subgroup of $\prod_{i\in I}G_i$. 
\end{lmm}

\begin{lmm}{}{} Let $G_1,\dots,G_n$ be a finite collection of abelian groups. Then the identity map of sets give a group isomorphism $$\bigoplus_{i=1}^nG_i\cong\prod_{i=1}^nG_i$$
\end{lmm}

\subsection{Free Abelian Groups}
\begin{defn}{Free Abelian Group On a Set}{} Let $S$ be a set. Define the free abelian group on $S$ to be the direct sum $$\Z S=\Z\langle s\in S\rangle=\bigoplus_{s\in S}\Z$$ 
\end{defn}

\begin{prp}{Universal Property of Free Abelian Groups}{} Let $S$ be a set. Then the free group $\Z S$ and the inclusion map $\iota:S\to\Z S$ satisfies the following universal property. For any abelian group $A$ and any map of sets $\psi:S\to A$, there exists a unique group homomorphism $\phi:\Z S\to A$ such that the following diagram commutes: \\~\\
\adjustbox{scale=1.0,center}{\begin{tikzcd}
	S & {\Z S} \\
	& A
	\arrow["\iota", hook, from=1-1, to=1-2]
	\arrow["\psi"', from=1-1, to=2-2]
	\arrow["{\exists!\phi}", dashed, from=1-2, to=2-2]
\end{tikzcd}} \\~\\
Moreover, $\Z S$ is the unique group (up to unique isomorphism) that has such a property. \tcbline
\end{prp}

\begin{prp}{}{} Let $S$ be a set. Then there is an isomorphism $$\Z S\cong(F_S)^\text{ab}$$ given by the universal property of free abelian groups. 
\end{prp}

\begin{prp}{}{} Let $S,T$ be sets. Then $\Z S\cong\Z T$ are isomorphic if and only if $\abs{S}=\abs{T}$. 
\end{prp}

\begin{prp}{}{} Let $A$ be an abelian group and $B$ a subgroup of $A$. If $A/B$ is a free abelian group, then $$A\cong B\times A/B$$ \tcbline
\begin{proof}
Denote the quotient map $f:A\to A/B$. Since $A/B$ is free abelian, there exists $K=\{a_1+B,\dots,a_n+B,\dots\}$ such that $K$ is a basis of $A/B$. Now consider $L=\langle a_1,\dots,a_n,\dots\rangle$. I claim that this is isomorphic to $A/B$. In fact, this is given by $f|_L$. Clearly $f|_K(a_i)=a_i+B$ for each $i$. Suppose that $a\in\ker(f|_K)$. Write $a=\sum_{i=1}^nk_ia_i$. Then we have 
\begin{align*}
f|_L\left(\sum_{i=1}^nk_ia_i\right)&=B\\
\sum_{i=1}^nk_if|_L(a_i)&=B\\
\sum_{i=1}^nk_i(a_i+B)&=B\\
\end{align*}
By property of the basis, we have that $k_1=\dots=k_n=0$. This generalizes well to the countable case since every element is written in a finite sum. Thus we have that $\ker(f|_L)=0$. For surjectivity, if $\sum k_i(a_i+B)$ is a finite sum in $A/B$ then just choose $\sum k_ia_i\in L$. Thus we have shown bijectivity. \\~\\
Finally, we show that $A\cong B\times L$. Define $\phi:B\times L\to A$ by $$\phi\left(b,\sum k_ia_i\right)=\sum k_ia_i+b$$ We first show injectivity. We have that 
\begin{align*}
\sum k_ia_i+b&=0\\
\sum k_ia_i&=-b\\
\sum k_if|_L(a_i)&=0\\
\sum k_i(a_i+B)&=0
\end{align*}
This implies that $k_i=0$ and thus $b=0$. Thus $\ker(\phi)=0$. Now suppose that $a\in A$. Consider the element $a+B\in A/B$. Since $A/B$ has the basis $K$, we have that $$a+B=\sum k_i(a_i+B)=\left(\sum k_ia_i\right)+B$$ This implies that $a-\sum k_ia_i\in B$ Then choosing $b=a-\sum k_ia_i$ and $\sum k_ia_i\in L$, we are done with surjectivity. This makes sense even if $\sum k_ia_i\in B$ because in this case, $\sum k_ia_i=0$ and we can choose $b=a\in B$. 
\end{proof}
\end{prp}

Given an arbitrary abelian group, we can recognize whether it is free abelian using the concept of a basis that is analogous to that in Linear Algebra. 

\begin{defn}{Basis of an Abelian Group}{} Let $G$ be an abelian group. Let $B\subset G$. 
\begin{itemize}
\item We say that elements of $B$ are linearly independent if $\sum_{b\in B}n_b\cdot b=0$ implies $n_b=0$ for each $b\in B$
\item We say that $B$ is a basis of $G$ if it is linearly independent and generate $G$. 
\end{itemize}
\end{defn}

\begin{prp}{}{} Let $G$ be an abelian group. Then $G$ is isomorphic to a free abelian group if and only if $G$ has a basis. 
\end{prp}

\pagebreak
\section{Finite and Finitely Generated Abelian Groups}
\subsection{Finite Abelian Groups}
\begin{defn}{Finite Abelian Groups}{} A group $G$ is said to be a finite abelian group if it is finite and abelian. 
\end{defn}

\begin{lmm}{}{} Let $G$ be a finite abelian group of order $n$. If $p$ is a prime that divides $n$, then $G$ contains an element of order $p$. 
\end{lmm}

\begin{lmm}{}{} A finite abelian group is a $p$ group if and only if the order of every element is a power of $p$. 
\end{lmm}

\begin{lmm}{}{} Let $G$ be a finite abelian group. Let $H$ be a subgroup of $G$. Then there exists a complement $K$ such that $G=H\times K$. 
\end{lmm}

\begin{lmm}{}{} Let $G$ be a finite abelian $p$ group and suppose that $g\in G$ is an element in $G$ with the highest order. Then $G$ is isomorphic to $\langle g\rangle\times H$ for some subgroup $H$ of $G$. 
\end{lmm}

\begin{thm}{The Fundamental Theorem for Finite Abelian Group (Primary Decomposition)}{} Let $G$ be a finite abelian group. Then $G$ can be decomposed into a direct product of cyclic groups $$G\cong\Z/k_1\Z\times\dots\times\Z/k_n\Z$$ where $k_1,\dots,k_n\in\N$ are such that $k_{i+1}|k_i$ for $1\leq i\leq n-1$. 
\end{thm}

\begin{thm}{Invariant Factor Decomposition}{} Let $G$ be a finite abelian group. Then $G$ can be decomposed into a direct product of cyclic groups $$G\cong\Z/k_1\Z\times\dots\times\Z/k_n\Z$$ where $k_1,\dots,k_n\in\N$ are such that $k_1,\dots,k_n$ are powers of primes that are not necessarily distinct. 
\end{thm}

\subsection{Finitely Generated Abelian Groups}
\begin{defn}{Finitely Generated Abelian Groups}{} A group $G$ is said to be finitely generated abelian if it is finitely generated and abelian. 
\end{defn}

\begin{lmm}{}{} Let $G$ be an abelian group. If $G$ is finite, then $G$ is finitely generated. \tcbline
\begin{proof}
A finite choice of generators for $G$ in this case would be the set $G$. 
\end{proof}
\end{lmm}

Contrasting finitely generated groups, finitely generated abelian groups allow for a more simple description of subgroups generated by a subset because we can collect like terms by commuting elements. In particular, if $G$ is an abelian group and $A\subseteq G$ is a subset, then $$\langle A\rangle=\left\{\sum_{g\in A}k_g\cdot g|k\in\Z\right\}$$

\begin{prp}{}{} Subgroups of a finitely generated abelian group is finitely generated. \tcbline
\begin{proof}
Let $H\leq G$. We prove by induction on $n$ that $H$ can be generated by at most $n$ elements. If $n=1$, then $G$ is cyclic. Then $G=\{kx|k\in\Z\}$. Let $m$ be the smallest positive number such that $mx\in H$. If $m$ does not exists then $K=\{0\}$ and we are done. Otherwise, $\{k(mx)|k\in\Z\}\subseteq H$. We now prove the reverse inclusion. Write $tx=(qm+r)x\in G$ with $0\leq r<m$. Then $tx\in H$ if and only if $rx=(t-qm)x\in H$. But this is true if and only if $r=0$ since $m$ is already the smallest number such that $mx\in H$. Thus $H\subseteq\{k(mx)|k\in\Z\}$ and $H$ is cyclic. \\~\\
Now suppose that the induction hypothesis is true for $n-1$.  Let $K$ be the subgroup generated by $x_1,\dots,x_{n-1}$. By induction, $H\cap K$ is generated by $y_1,\dots,y_{m-1}$ with $m\leq n$. If $H\leq K$, then $H=H\cap K$ and we are done. \\~\\
Suppose that $H$ is not a subgroup of $K$. Then there exists elements of the form $k+tx_n\in H$ with $k\in K$ and $t>0$. Choose $y_m=h+tx_n$ with $t$ minimal. I claim that $H$ is generated by $y_1,\dots,y_m$. Let $h\in H$. Then $h=k'+ux_n$ with $k'\in K$ and $u\in\Z$. If $t$ does not divied $u$ then we can write $u=tq+r$ with $q,r\in\Z$ and $0,r<t$. Then $k-qy_m=(h'-qh)+rx_n\in H$ thus contradicting minimality of $t$. Thus $t|u$ and hence $u=tq$ and $k-qy_m\in H\cap K$. But $H\cap K$ is generated by $y_1,\dots,y_{m-1}$ thus we are done. 
\end{proof}
\end{prp}

\begin{prp}{}{} Let $G$ be a finitely generated abelian group generated by the elements $g_1,\dots,g_n\in G$. Then there exists a surjective group homomorphism $\phi:\Z^n\to G$ given by sending $(0,\dots,0,1,0,\dots,0)$ with non-zero entry at the $k$th column to $g_k$. 
\end{prp}

A free abelian group is in some sense freely generated by its set of basis whose only restriction is for it to be abelian. On the other hand, finitely generated abelian groups may have extra relations between basis elements preventing them from being free. Therefore every finitely generated abelian group can be thought of as a free abelian group, then quotient some non-trivial relations. 

\begin{prp}{}{} Let $G$ be a finitely generated abelian group generated by $n$ elements. Then the following are true. 
\begin{itemize}
\item There exists a finitely generated subgroup $H=\langle v_1,\dots,v_m\rangle$ of $\Z^n$ such that there is an isomorphism $$G\cong\frac{\Z^n}{H}$$
\item $G$ has the presentation of $G\cong\langle x_1,\dots,x_n\;|\;v_1,\dots,v_m\rangle$ where $x_1,\dots,x_n$ are the standard basis of $\Z^n$ (as a subgroup of $\Z^n$ so that the commutating relations are hidden)
\end{itemize} \tcbline
\begin{proof}
Let $G$ be a finitely generated abelian group. Then there exists $g_1,\dots,g_n\in G$ such that $G=\langle g_1,\dots,g_n\rangle$. Define a function $\phi:\Z^n\to G$ by sending $(0,\dots,0,1,0,\dots,0)$ where the $k$th position is non-zero, to the element $g_k$. By the universal property of free abelian groups, 
\end{proof}
\end{prp}

We can compare some of the similarities between free groups and free abelian groups. 

\begin{table}[!ht]
\centering
\begin{tabular}{p{8cm}|p{8cm}}
Free Groups                                                                               & Free Abelian Groups                                                                                       \\[1.5ex] \hline\hline
Groups with the least relations                                                           & Abelian groups with the least relations                                                                   \\[1.5ex]\hline
\multirow{2}{*}{Unique up to cardinality of underying set}                                & Unique up to cardinality of basis                                                                         \\
                                                                                          & (which is its underlying set)                                                                             \\\hline
Free product of copies of $\Z$                                                         & Direct sum of copies of $\Z$                                                                              \\[1.5ex]\hline
It is finitely generated if and only if it is isomorphic to $F_n$                         & It is finitely generated if and only if it is isomorphic to $\Z^n$                                        \\[1.5ex]\hline
Every finitely generated group is isomorphic to a quotient of a free group of finite rank & Every finitely generated abelian group is isomorphic to a quotient of a free abelian group of finite rank
\end{tabular}
\end{table}

A more surprising result would be the following, which says that if there are no elements in the finitely generated group $G$ with finite order, then it means that no extra relations are imposed on $G$ so that $G$ is a free abelian group. 

\begin{prp}{}{} Let $G$ be a finitely generated abelian group. Then $G$ is a free abelian group if and only if $G$ is torsion-free. 
\end{prp}

In particular, this representation means that $v_1=0,\dots,v_m=0$ in $\Z^n$. The question remaining is to how to transform the kernel $\ker(\phi)$ into the generators $v_1,\dots,v_m\in\Z^n$. 

\begin{prp}{}{} Let $G$ be a finitely generated abelian group. Then there exists a basis $\{b_1,\dots,b_n\}$ of $\Z^n$ such that there is an isomorphism $$G=\langle b_1,\dots,b_n\;|\;d_1b_1,\dots,d_mb_m\rangle$$ where $d_1,\dots,d_m\in\N\setminus\{0\}$ are such that $d_i|d_{i+1}$ for $1\leq i\leq m-1$. \tcbline
\begin{proof}
Suppose that $G=\langle x_1,\dots,x_n|v_1,\dots,v_m\rangle=\frac{\Z^n}{K}$. Since $K$ is a subgroup of $\Z^n$, we know that it is also a free group. 
\end{proof}
\end{prp}

\begin{thm}{Fundamental Theorem of Finitely Generated Abelian Groups (Invariant Factor Decomposition)}{} Let $G$ be a finitely generated abelian group. Then $G$ can be decomposed into $$G\cong \Z/n_1\Z\times\cdots\times\Z/n_s\Z\times\mathbb{Z}^r$$ where 
\begin{itemize}
\item (free rank / Betti number) $r\geq 0$ and $n_j\geq 2$ for all $j$
\item (invariant factors) $n_{i+1}|n_i$ for $1\leq i\leq s-1$
\item $\abs{G}=n_1\cdots n_s$
\end{itemize}
This expression is unqiue up to reordering of the external product. \tcbline
\begin{proof}
Let $G$ be a finitely generated abelian group. Denote the torsion subgroup of $G$ by $tG$. Then $G/tG$ is a torsion free abelian group and thus is free abelian. Thus we have that $G\cong tG\times G/tG$. Since $tG$ is a finite abelian group, we can invoke the the fundamental theorem for finite abelian groups to get $tG\cong\Z/k_1\Z\times\cdots\times\Z/k_s\Z$. Since $G/tG$ is free abelian, it is isomorphic to $\Z^r$ for some $r$. Combining the results gives our proof. 
\end{proof}
\end{thm}

Below is the Primary Decomposition of finite abelian groups. 

\begin{thm}{Primary Decomposition}{} Let $G$ be a finitely generated abelian group. Then $G$ can be decomposed into $$G\cong\Z/q_1\Z\times\cdots\times\Z/q_t\Z\times\Z^r$$ where 
\begin{itemize}
\item $r\geq 0$ is the free rank of $G$
\item $q_1,\dots,q_t$ are powers of primes numbers that may not be unique. 
\end{itemize}
This expression is unique up to reordering of the indices. 
\end{thm}

\begin{thm}{}{} Let $G$ be a finitely generated abelian group represented by $$\langle b_1,\dots,b_n|d_1b_1,\dots,d_mb_m\rangle$$ Further suppose that $G$ has invariant decomposition $$\left(\bigoplus_{i=1}^m\Z/d_i'\Z\right)\oplus\Z^{n'-m'}$$ Then these two representations of $G$ match up through their numbers. In particular, 
\begin{itemize}
\item Up to reordering, we have $d_i=d_i'$
\item $n=n'$ and $m=m'$
\end{itemize}
\end{thm}

\subsection{Unimodular Smith Normal Form}
\begin{defn}{Unimodular Elementary Operations}{} We define unimodular elementary operations on $A\in M_{m\times n}(\Z)$ as follows: 
\begin{itemize}
\item (UR1): Replace row $r_i$ with $r_i+tr_j$ where $j\neq i$ and $t\in\Z$
\item (UR2): Interchange two rows $r_i$ and $r_j$
\item (UR3): Replace row $r_i$ by $-r_i$
\item (UC1): Replace column $c_i$ with $c_i+tc_j$ where $j\neq i$ and $t\in\Z$
\item (UC2): Interchange two columns $c_i$ and $c_j$
\item (UC3): Replace columns $r_i$ by $-r_i$
\end{itemize}
\end{defn}

\begin{defn}{Unimodular Elementary Matrices}{} Define three elementary matrices as follows: 
\begin{itemize}
\item Recombine Matrix: The $n\times n$ recombine matrix $R_{i,j,a}$ is given the zero matrix except the diagonal is all $1$ and the $i$th row and $j$th column has the value $a\in\Z$. 
\item Scale Matrix: The $n\times n$ scale matrix $R_{i}(-1)$ is given by  the zero matrix except the diagonal is all $1$, and the $i,i$th element is $-1$. 
\item Transposition Matrix: The $n\times n$ transposition matrix $R_{i,j}$ is given by the zero matrix except the diagonal is all $1$, the $i,i$th element and $j,j$th element is $0$ and the $i,j$th and $j,i$th element is $1$
\end{itemize}
\end{defn}

\begin{prp}{}{} The unimodular elementary operations are exactly equivalent to performing left and right matrix multiplications with unimodular elementary matrices. \\~\\
In particular, left multiplication corresponds to row operations and right multiplication corresponds to column operations. 
\end{prp}

\begin{thm}{}{} Let $A\in\Z_{m\times n}$ with rank $r$. Then there exists a sequence of unimodular elementary row and column operations such that $A$ is reduced to $$\begin{pmatrix}
d_1 & 0 & \cdots & 0\\
0 & d_2 & \cdots & 0\\
\vdots & \vdots & \ddots & 0\\
0 & 0 & \cdots & 0
\end{pmatrix}$$ where the $d_r$ is the last non zero diagonal entry. The diagonal also satisfies $d_i>0$ for $1\leq i\leq r$ and $d_i|d_{i+1}$ for $1\leq i<r$. The $d_1,\dots,d_r$ are also uniquely determined by $A$. \tcbline
\begin{proof}
\end{proof}
\end{thm}

\begin{defn}{Unimodular Smith Normal Form}{} Let $A\in M_{n\times n}(\Z)$ be a matrix with coefficients in $\Z$. Define the smith normal form of $A$ to be the unique matrix above, denoted $\text{SNF}(A)$. 
\end{defn}

\begin{lmm}{}{} Let $A\in\Z_{m\times n}$ has unimodular smith normal form $S$ with rank $r$. Then the greatest common divisor of all entries of $A$ is equal to $d_1$ where we require $\gcd(r,0)=r$ for $r\geq 1$. 
\end{lmm}

\pagebreak
\section{The Sylow Theorems and its Consequences}
Lagrange's theorem leads to a natural question. Does the converse hold? That is, given a number $k$ dividing $\abs{G}$, is there a subgroup of $G$ with order $k$? This is true for cyclic groups but in general it is not true. For example, if $\abs{G}$ is a non-abelian finite simple group, then $G$ has no subgroup of order $\abs{G}/2$ because otherwise, this subgroup would be normal. \\~\\

We then want sufficient criterion for the converse to hold. This leads to the Sylow theorems and its relation to $p$-groups. 
\subsection{The Four Sylow Theorems}
We introduce a notation for subgroups of order a power of a prime. 

\begin{defn}{Sylow $p$-groups}{} Let $G$ be a group of order $p^km$ where $\gcd(p,m)=1$, then subgroups of $G$ of order $p^k$ is called a Sylow $p$-subgroup of $G$. The set of all Sylow $p$-subgroups of $G$ is denoted $$\text{Syl}_p(G)=\{H\leq G\;|\;H\text{ is a Sylow }p\text{-subgroup }\}$$ and the number of Sylow $p$-subgroups is denoted $$n_p(G)=\abs{\text{Syl}_p(G)}$$ 
\end{defn}

The maximal power of the prime $p$ that divides $\abs{G}$ is also called the $p$-part of $G$. This is denoted as $\abs{G}_p$. Before we prove the main theorems of the section, we need a lemma. 

\begin{lmm}{}{} Let $p$ be prime. Let $n,m\in\N$ such that $\gcd(m,p)=1$. Then the following are true. 
\begin{itemize}
\item $p$ divides $\binom{p}{i}$ for all $1\leq i\leq p-1$
\item $\binom{p^nm}{p^n}\equiv m\;(\bmod\;p)$
\end{itemize} \tcbline
\begin{proof}~\\
\begin{itemize}
\item By definition, we have that $$\binom{p}{i}=\frac{p(p-1)\cdots(p-(i-1))}{i(i-1)\cdots2\cdot 1}$$ Let $a=(p-1)\cdots(p-(i-1))$ and $b=i!$ so that $\binom{p}{i}=\frac{pa}{b}$. Since $b$ is the product of integers less than $p$, all prime divisors of $b$ are less than $p$. Thus $p$ does not divide $a$. We have that $b\binom{p}{i}=pa$. Since $\binom{p}{i}$ is an integer, and $p$ does not divide $b$, $p$ must divide $\binom{p}{i}$. 
\item Denote $\F_p=\Z/p\Z$. Consider the polynomial ring $\F_p[x]$ and $(1+x)^p\in\F_p[x]$. By the binomial theorem, we have that 
\begin{align*}
(1+x)^p&=\sum_{i=0}^p\binom{p}{i}x^i\\
&=\binom{p}{0}+\binom{p}{p}x^p\\
&=1+x^p
\end{align*}
A similar calculation also shows that $(1+x)^{p^n}=1+x^{p^n}$. Thus $$(1+x)^{p^nm}=(1+x^{p^n})^m$$ Using the binomial theorem on both sides, we get $$\sum_{i=0}^{p^nm}\binom{p^nm}{i}x^i=\sum_{j=0}^m\binom{m}{j}x^{p^{n^j}}$$ Comparing coefficients give $$\binom{p^nm}{p^nj}\equiv\binom{m}{j}\;(\bmod\;p)$$ Applying the result with $j=1$ proves the lemma. 
\end{itemize}
Thus we are done. 
\end{proof}
\end{lmm}

The four Sylow theorems is due to Ludwig Sylow. It is an important corner stone in finite group theory. 

\begin{thm}{The First and Second Sylow Theorems}{} Let $G$ be a finite group and $\abs{G}=p^km$ where $\gcd(p,m)=1$. Then the following are true. 
\begin{itemize}
\item There exists at least one Sylow $p$-subgroup of $G$. 
\item $n_p(G)\equiv1\;(\bmod\;p)$
\end{itemize} \tcbline
\begin{proof}
We first prove that $n_p(G)\equiv 1\;(\bmod\;p)$. Let $G$ be a finite group and $\abs{G}=p^km$ where $\gcd(p,m)=1$. Define $$X=\{S\subset G\;|\;\abs{S}=p^n\}$$ Then $\abs{X}=\binom{p^nm}{p^n}$ and thus $\abs{X}\equiv m\;(\bmod\;p)$ by the above lemma. Notice that $p$ does not divide $\abs{X}$. \\~\\

Consider the action of $G$ on $X$ by left multiplication. If $S\in X$, define $$g\cdot S=gS=\{gs\;|\;s\in S\}$$ Clearly this is a group action. We know that $\{\text{Orb}_G(S)|S\in X\}$ partitions $X$. In other words, there exists $S_1,\dots,S_r\in X$ such that $$X=\coprod_{i=1}^r\text{Orb}_G(S_i)$$ and thus $\abs{X}=\sum_{k=1}^r\abs{\text{Orb}_G(S_i)}$. \\~\\
Since we know that $p$ does not divide $\abs{X}$, there must exists at least one of the orbits with $\abs{\text{Orb}_G(S_i)}$ is not divisible by $p$. Let $t$ be the number of orbits that have order indivisible by $p$. Without loss of generality, reorder the orbits such that $\abs{\text{Orb}_G(S_i)}$ is not divisible by $p$ for $1\leq i\leq t$ and $p\;|\;\abs{\text{Orb}_G(S_i)}$ for $t+1\leq i\leq r$. \\~\\

We now prove two claims. \\
Claim 1: Fix $1\leq i\leq t$. Then there exists $x_i\in G$ such that $\text{Stab}_G(x_iS_i)=x_iS_i$. In particular, $\text{Stab}_G(x_iS_i)\in\text{Syl}_p(G)$ for each $1\leq i\leq r$. \\
Take $1\leq i\leq t$. Let $s_i\in S_i$ and take $x_i=s_i^{-1}$. Let $T_i=x_iS_i$. Then $1_G\in T_i$. By definition of orbits, we have $\text{Orb}_G(T_i)=\text{Orb}_G(S_i)$. By the orbit stabilizer theorem, we have that $p$ does not divide $\abs{O_{T_i}}=[G:\text{Stab}_G(T_i)]$. Since $\abs{G}=p^nm=[G:\text{Stab}_G(T_i)]\abs{\text{Stab}_G(T_i)}$, we have that $p^n|\abs{\text{Stab}_G(T_i)}$. On the other hand, if $g\in\text{Stab}_G(T_i)$, then $gT_i=T_i$ by definition. Since $1_G\in T_i$, we have that $g\in gT_i=T_i$ and thus $\text{Stab}_G(T_i)\subseteq T_i$. Since $\abs{T_i}=p^n\leq\abs{\text{Stab}_G(T_i)}$, we deduce that $\text{Stab}_G(T_i)=T_i$ as required. \\~\\

Claim 2: $t=\abs{\text{Syl}_p(G)}$. \\
Define a map $f:\{\text{Orb}_G(T_i)|1\leq i\leq t\}\to\text{Syl}_p(G)$ by $f(\text{Orb}_G(T_i))=T_i$. By claim 1, we know that $T_i$ is a subgroup of $G$ of order $p^n$ since $T_i=\text{Stab}_G(T_i)$, so this map is well defined. We prove that $f$ is a bijection. \\
Injectivity: Let $1\leq i,j\leq t$. Suppose that $T_i=T_j$. Then $$\text{Orb}_G(S_i)=\text{Orb}_G(T_i)=\text{Orb}_G(T_j)=\text{Orb}_G(S_j)$$ so $S_i=S_j$ since we have chosen $S_1,\leq,S_r$ so that their orbits are distinct. \\
Surjectivity: Let $p\in\text{Syl}_p(G)$. Then $P\in X$, and $\text{Stab}_G(P)=P$. It follows that $p$ does not divide $m=[G:P]=[G:\text{Stab}_G(P)]=\abs{\text{Orb}_G(P)}$, so $\text{Orb}_G(P)=\text{Orb}_G(T_i)$ for some $i$. Since $1_G\in T_i$, and $\text{Orb}_G(T_i)=\{xS_i\;|\;x\in G\}$, $T_i$ is the unique element of $\text{Orb}_G(T_i)$ containing $1_G$. Thus we must have $P=T_i=f(\text{Orb}_G(T_i))$ as required. \\~\\

We resume the proof of the second Sylow theorem. Since we know that $\abs{X}\equiv m\;(\bmod\;p)$, and that $$\abs{X}\equiv\sum_{i=1}^r\abs{\text{Orb}_G(T_i)}\;(\bmod\;p)$$ Since $\abs{\text{Orb}_G(T_i)}=[G:\text{Stab}_G(T_i)]=\abs{G}/\abs{T_i}=m$ by claim 1, and $t=\abs{\text{Syl}_p(G)}$ by claim 2. From the above sum we deduce that $$m\equiv mn_p(G)\;(\bmod\;p)$$ Since we also have that $\gcd(m,p)=1$, we can cancel $m$ on both sides to get $n_p(G)\equiv1\;(\bmod\;p)$. This also proves the first Sylow theorem. 
\end{proof}
\end{thm}

\begin{prp}{}{} Let $G$ be a group and $p$ a prime divisor of $\abs{G}$. Let $H\leq G$ and let $P$ be a Sylow $p$-subgroup of $G$. Then there exists some $g\in G$ such that $H\cap gPg^{-1}$ is a Sylow $p$-subgroup of $H$. \tcbline
\begin{proof}
Consider the set $X=\{xP|x\in G\}$ of left cosets of $P$ in $G$. Then $G$ acts on $X$ by left multiplication and thus $H$ acts on $X$. For any $xP\in X$, we have that 
\begin{align*}
\text{Stab}_H(xP)&=\{h\in H|hxP=xP\}\\
&=\{h\in H|x^{-1}hxP=xP\}\\
&=\{h\in H|h\in xPx^{-1}\}\\
&=H\cap xPx^{-1}
\end{align*}
Since $P$ is a Sylow $p$-subgroup of $G$, $\abs{X}=[G:P]$ is coprime to $p$. Let $\text{Orb}_H(x_1P),\dots,\text{Orb}_H(x_sP)$ denote the distinct orbits of $H$ in its action on $X$. We know that $\abs{X}=\sum_{i=1}^s\abs{\text{Orb}_H(x_iP)}$ If every such orbit has size divisible by $p$, then $\abs{X}$ would be divisible by $p$, which is a contradiction. Thus there exists $x_i\in G$ such that $p$ does not divide $\abs{\text{Orb}_H(x_iP)}$. \\~\\
We know that $\text{Stab}_H(x_iP)=H\cap x_iPx_i^{-1}$ is a $p$-subgroup since it is a subgroup of $x_iPx_i^{-1}$. We also have $[H:\text{Stab}_H(x_iP)]=\abs{\text{Orb}_H(x_iP)}$ is not divisible by $p$. Thus $H\cap x_iPx_i^{-1}$ is a Sylow $p$-subgroups of $H$. Thus we are done. 
\end{proof}
\end{prp}

\begin{thm}{The Third and Fourth Sylow Theorems}{} Let $G$ be a finite group and $\abs{G}=p^km$ where $\gcd(p,m)=1$. Then the following are true. 
\begin{itemize}
\item Every Sylow $p$-subgroup is conjugate to each other
\item Any $p$-subgroup of $G$ is contained in a Sylow $p$-subgroup of $G$
\end{itemize} \tcbline
\begin{proof}~\\
\begin{itemize}
\item Let $P$ and $Q$ be Sylow $p$-subgroups of $G$. By the above proposition, there exists $g\in G$ such that $Q\cap gPg^{-1}$ is a Sylow $p$-subgroup of $Q$. But $\abs{Q}$ is a power of $p$ and thus $\text{Syl}_p(Q)=\{Q\}$. Thus $Q=Q\cap gPg^{-1}$. Since also $\abs{Q}=\abs{gPg^{-1}}$ we have $Q=gPg^{-1}$. 
\item Let $H$ be a Sylow $p$-subgroup of $G$ and $P$ a Sylow $p$-subgroup of $G$. By the above proposition, there exists $g\in G$ such that $H\cap gPg^{-1}$ is a Sylow $p$-subgroup of $H$. But $\abs{H}$ is a power of $p$ and thus $\text{Syl}_p(H)=\{H\}$. Thus $H=H\cap gPg^{-1}\leq gPg^{-1}$. 
\end{itemize}
Thus we are done. 
\end{proof}
\end{thm}

\subsection{Consequences of the Sylow Theorems}
\begin{crl}{}{} Let $G$ be a finite group. Let $p$ be prime such that $\abs{G}=p^km$ with $\gcd(p,m)=1$. Let $P\in\text{Syl}_p(G)$ be a Sylow $p$-subgroup. Then the following are true with respect to $n_p(G)$. 
\begin{itemize}
\item $n_p(G)=[G:N_G(P)]$
\item $n_p(G)|m$
\item $P\trianglelefteq G$ if and only if $n_p(G)=1$
\end{itemize} \tcbline
\begin{proof}~\\
\begin{itemize}
\item By the third Sylow theorem, we know that $G$ acts on $\text{Syl}_p(G)$ by conjugation and that $\text{Orb}_G(P)=\text{Syl}_p(G)$. The stabilizer $\text{Stab}_G(P)$ is thus equal to the normalizer $N_G(P)$. Using the orbit stabilizer theorem, we have that $$\abs{\text{Syl}_p(G)}=\abs{\text{Orb}_G(P)}=[G:\text{Stab}_G(P)]=[G:N_G(P)]$$
\item We know that $P$ is a subgroup of $N_G(P)$. This means that we can apply Lagrange's theorem to get $\abs{N_G(P)}=\abs{P}\cdot[N_G(P):P]$. Thus we have that 
\begin{align*}
\abs{\text{Syl}_p(G)}&=[G:N_G(P)]\\
&=\frac{\abs{G}}{\abs{N_G(P)}}\\
&=\frac{\abs{G}}{\abs{P}\cdot[N_G(P):P]}
\end{align*}
Thus $\abs{\text{Syl}_p(G)}$ divides $\frac{\abs{G}}{\abs{P}}=m$
\item We have that $P$ is a normal subgroup of $G$ if and only if $N_G(P)=G$. And this is true if and only if $\abs{\text{Syl}_p(G)}=1$ by the first item. 
\end{itemize}
Thus we are done. 
\end{proof}
\end{crl}

\begin{prp}{}{} There are no simple groups of order $20$. \tcbline
\begin{proof}
Let $G$ be a group of order $20$. Then $\abs{G}=2^2\cdot 5$. By Sylow theorem 2, we know that $n_5(G)\equiv 1\;(\bmod\;5)$. By corollary 5.2.1, we have that $n_5(G)$ divides $\frac{\abs{G}}{\abs{P}}=\frac{20}{5}=4$ and thus $n_5(G)=1$. By the same corollary, we know that the unique Sylow $5$-subgroup is normal. Thus $G$ cannot be simple. 
\end{proof}
\end{prp}

\begin{prp}{}{} There are no simple groups of order $48$. \tcbline
\begin{proof}
Let $G$ be a group of order $48$. Then $\abs{G}=2^4\cdot 3$. Consider $n_2(G)$. By the second Sylow theorem, we have that $n_2(G)\cong 1\;(\;\bmod 2)$. By corollary 5.2.1, we have that $n_2(G)|3$ and thus $n_2(G)=1$ or $3$. \\~\\
Suppose that $n_2(G)=1$. Then by the same corollary, we have a unique Sylow $2$-subgroup that is normal to $G$. Thus $G$ is not simple. \\~\\
Suppose that $n_2(G)=3$. Then $G$ acts on $\text{Syl}_2(G)$ non-trivially. Then we know that there is a homomorphism $\phi:G\to S_3$. The first isomorphism theorem and Lagrange's theorem tells us that $\frac{\abs{G}}{\abs{\ker(\phi)}}=\abs{\im(\phi)}$. $\im(\phi)$ is a subgroup of $S_3$ whose order is greater than $1$ since $\phi$ is non-trivial. Thus $1\leq\abs{\im(\phi)}\leq 6$ and thus $48/6\leq\abs{\ker(\phi)}<48$. Thus $\ker(\phi)$ is a non-trivial normal subgroup of $G$ and thus $G$ is not simple. 
\end{proof}
\end{prp}

\subsection{Simplicity of $A_n$ for $n\geq 5$}
Using the Sylow theorems, we can also analyzes the number of elements that possess a power of prime order. For a finite group $G$, define $$F_p(G)=\{x\in G\;|\;x\neq 1\text{ and }\abs{x}=p^n\text{ for some }n\}$$ to be the number of elements in $G$ that has its order the power of a prime $p$. We can deduce a number of information from  the Sylow theorems, so that we can eventually some sort of contradiction in our proof of simplicity. 

\begin{crl}{}{} Let $G$ be a finite group such that $\abs{G}=p^km$ where $p$ is a prime and $\gcd(p,m)=1$. Let $$F_p(G)=\{x\in G\;|\;x\neq 1\text{ and }\abs{x}=p^n\text{ for some }n\}$$ Then the following are true. 
\begin{itemize}
\item $F_p(G)=\bigcup_{P\in\text{Syl}_p(G)}(P\setminus\{1\})$
\item $\abs{F_p(G)}\geq p^k-1$ with equality if and only if $\abs{\text{Syl}_p(G)}=1$
\item If $k=1$ then $\abs{F_p(G)}=\abs{\text{Syl}_p(G)}\cdot(p-1)$
\end{itemize} \tcbline
\begin{proof}~\\
\begin{itemize}
\item First let $x\in F_p(G)$. Then $\langle x\rangle$ has a prime power order and so is a $p$-subgroup of $G$. Any $p$-subgroup is contained in a Sylow $p$-subgroup and so $x$ lies in $\bigcup_{P\in\text{Syl}_p(G)}(P\setminus\{1\})$. Now if $y$ lies in the union of the Sylow $p$-subgroups, then $\langle y\rangle$ is a subgroup of some Sylow $p$-subgroup. By Lagrange's theorem, the order of $y$ must be a power of $p$ since any Sylow $p$-subgroup has prime power order. Thus $y\in F_p(G)$. 

\item From the above characterization of $F_p(G)$, we have that $F_p(G)\supseteq(P\setminus\{1\})$ for any Sylow $p$-subgroup. Thus $\abs{F_p(G)}\geq p^k-1$. Now suppose that equality holds. Suppose for a contradiction that there are two distinct Sylow $p$-subgroups $P$ and $Q$. This means that there exists an element in $Q$ not contained in $P$. Hence $\abs{F_p(G)}\geq\abs{P\setminus\{1\}}+1=p^k$, this contradicts the assumption of equality. It is also clear that if $n_p(G)=1$ then $\abs{F_p(G)}=p^k-1$. 

\item Firstly notice that when $k=1$, we have that the first property in this corollary is a disjoint union. Indeed, if $P_i$ and $P_j$ are Sylow $p$-subgroups that intersect with element $x\in P_i\cap P_j$, then $\langle x\rangle$ has order $p$ so that $P_i=\langle x\rangle=P_j$. \\~\\

We will now show that $$F_p(G)=\coprod_{P\in\text{Syl}_p(G)}(P\setminus\{1\})$$ If $x\in F_p(G)$, then $x$ has order $p$ so that $\langle x\rangle$ is a Sylow $p$-subgroup and $x\in\langle x\rangle\subseteq\coprod_{P\in\text{Syl}_p(G)}(P\setminus\{1\})$. Now suppose that $y$ is an element that lies in the coproduct. Then $y$ lies in some Sylow $p$-subgroup. Any Sylow $p$-subgroup is cyclic and isomorphic to $C_p$ since $k=1$. Thus $y$ has order $p$ and $y\in F_p(G)$. Now $\abs{F_p(G)}=\abs{\text{Syl}_p(G)}(p-1)$ and so we conclude. 
\end{itemize}
This completes the proof. 
\end{proof}
\end{crl}

\begin{lmm}{}{} Let $G$ be a finite group such that $\abs{G}=p^km$ where $p$ is a prime and $\gcd(p,m)=1$. Let $$F_p(G)=\{x\in G|x\neq 1\text{ and }\abs{x}=p^n\text{ for some }n\}$$ Let $N$ be a normal subgroup of $G$. Then the following are true. 
\begin{itemize}
\item If $x\in N$, then $\{gxg^{-1}\;|\;g\in G\}\subseteq N$
\item If $p$ does not divide $n$, then $\text{Syl}_p(G)=\text{Syl}_p(N)$ and $F_p(G)=F_p(N)$. 
\end{itemize}
\end{lmm}

\begin{prp}{}{} The alternating group $A_5$ is simple. \tcbline
\begin{proof}
Suppose that there exists $N$ a non-trivial normal subgroup of $A_5$. By Lagrange's theorem, $\abs{N}$ divides $\abs{A_5}=60$ and thus at least one of the prime divisors are either $2,3$ or $5$. \\~\\
The $5$-cycles of $A_5$ are precisely elements of $A_5$ with order $5$. Since each $5$-cycle is of the form $\begin{pmatrix} a_1 & \cdots & a_5\end{pmatrix}$, there are $5$ possible choices for $a_1$, $4$ for $a_2$ and so on. Since also that moving elements transitively along the cycle gives another presentation of the same cycle, we divide by $5$ to obtain that there are precisely $24$ elements of order $5$ of $A_5$. A similar argument show that $A_5$ has $20$ elements of order $3$. \\~\\

The only elements of order $2$ are products of two disjoint transpositions. A similar argument shows that there are $15$ of them. We know that they are part of the same conjugacy class in $A_5$. \\~\\

Case 1: Either $3$ divides $\abs{N}$ or $5$ divides $\abs{N}$. \\
Let $p=3$ or $5$ such that $p$ divides $\abs{N}$. Then we know that $p$ does not divide $[G:N]$ so by the above theorem we deduce that $F_p(G)=F_p(N)$. If $p=5$ then $F_p(N)=24$ and thus $\abs{N}\geq 25$. Since $\abs{N}\;|\;60$ this implies that $\abs{N}=30$. If $p=3$ then $F_p(N)=20$ and thus $\abs{N}\geq 21$. Since $\abs{N}\;|\;60$ we thus have $\abs{N}=30$. If either $3$ divides $\abs{N}$ or $5$ divides $\abs{N}$ then both $3$ and $5$ divides $\abs{N}$. But $F_3(N)+F_5(N)>\abs{N}$, a contradiction. \\~\\

Case 2: Neither $3$ or $5$ divides $\abs{N}$. \\
Then $\abs{N}$ divides $4$ by Lagrange's theorem. By Cauchy's theorem, there exists $x\in N$ of order $2$. Thus using the above corollary, we have that $4=\abs{N}\geq\text{Cl}(x)=15$ which is a contradiction. 
\end{proof}
\end{prp}

\begin{lmm}{}{} Let $n\geq 3$. Let $X$ be the set of $3$-cycles in $S_n$. Then $A_n=\langle X\rangle$. \tcbline
\begin{proof}
By definition, we know that $A_n$ is consists of elements of an even number of transpositions. We just have to show that every product of a pair of transpositions can be written as a product of $3$-cycles. Let $\begin{pmatrix}a&b\end{pmatrix}$ and $\begin{pmatrix}c&d\end{pmatrix}$ be a pair of transpositions for $a,b,c,d\in\{1,\dots,n\}$ with $a\neq b$ and $c\neq d$. There are three cases. \\~\\

Case 1: $\{a,b\}\cap\{c,d\}=\emptyset$. \\
Then the two cycles are disjoint. By direct calculation, we have that $$\begin{pmatrix}a&b\end{pmatrix}\begin{pmatrix}c&d\end{pmatrix}=\begin{pmatrix}a&b&c\end{pmatrix}\begin{pmatrix}b&c&d\end{pmatrix}$$ and so we are done. \\~\\

Case 2: $\abs{\{a,b\}\cap\{c,d\}}=1$. \\
Without loss of generality assume that $a=c$. Then $$\begin{pmatrix}a&b\end{pmatrix}\begin{pmatrix}a&d\end{pmatrix}=\begin{pmatrix}a&d&b\end{pmatrix}$$ and so we are done. \\~\\

Case 3: $\abs{\{a,b\}\cap\{c,d\}}=2$. \\
Then $\begin{pmatrix}a&b\end{pmatrix}^2=()=(a,b,e)^3$ for any $e\in\{1,\dots,n\}\setminus\{a,b,c,d\}$ and so we conclude. 
\end{proof}
\end{lmm}

\begin{lmm}{}{} Let $n\geq 5$. Then any two $3$-cycles are conjugate in $A_n$. \tcbline
\begin{proof}
We know that $gfg^{-1}$ has the same cycle type as $f$ for $f,g\in S_n$. Let $X$ be the set of $3$-cycles in $S_n$. Then $A_n$ acts on $X$ by conjugation. By the above lemma, the group action is transitive so that it has only one orbit. 
\end{proof}
\end{lmm}

\begin{lmm}{}{} Let $n\geq 5$. Let $\sigma\in A_n$ be a non-trivial element. Then for any conjugate $\tau$ of $\sigma$, there exists $si\in\{1,\dots,n\}$ such that $\tau(i)=\sigma(i)$. \tcbline
\begin{proof}
Let $r$ be the longest length of a disjoint cycle in $\sigma$. Relabelling if necessary, we have that $$\sigma=\begin{pmatrix}1 & 2 & \cdots & r\end{pmatrix}\pi$$ where $\pi$ and $\begin{pmatrix}1 & 2 & \cdots & r\end{pmatrix}$ are disjoint permutations. If $r\geq 3$. Let $\gamma=\begin{pmatrix}3  & 4 & 5\end{pmatrix}$ and $\tau=\gamma\sigma\gamma^{-1}$. Clearly $\tau\neq\sigma$ since $\sigma(3)=4$ and $\tau(3)=6$. But $\sigma(1)=2=\tau(1)$ and so we are done. \\~\\

If $r=2$, then $\sigma$ is a product of two disjoint transpositions. If there are at least $3$ disjoint transpositions, then $n\geq 6$ and after relabelling we can write $\sigma=\begin{pmatrix}1 & 2 \end{pmatrix}\begin{pmatrix}3 & 4 \end{pmatrix}\begin{pmatrix}5 & 6 \end{pmatrix}\cdots$. Let $\gamma=\begin{pmatrix}1 & 3 & 2 \end{pmatrix}$ and set $\tau=\gamma\sigma\gamma^{-1}$. Clearly $\tau\neq\sigma$. We also have $\sigma(5)=5=\tau(5)$ and so we conclude. 
\end{proof}
\end{lmm}

\begin{thm}{}{} The alternating group $A_n$ for $n\geq 5$ is simple. \tcbline
\begin{proof}
We know that $A_5$ is simple. So suppose that $n\geq 6$. Recall that $A_n$ acts on $X_n=\{1,\dots,n\}$, inherited by the action of $S_n$. For each $i\in X_n$, we have that $\text{Stab}_{A_n}(i)\cong A_{n-1}$. We proceed by induction. The case $n=5$ is clear. So suppose that $A_{n-1}$ is simply. By induction, $\text{Stab}_{A_n}(i)$ is a simple group. Notice that it also contains a $3$-cycle. \\~\\

Assume there exists a non-trivial proper normal group $N$ of $A_n$. Let $\sigma\in N$ be a non-trivial element of $N$. By the above lemma, there exists a conjugate $\tau\in A_n$ of $\sigma$ such that $\tau\neq\sigma$ but $\sigma(i)=\tau(i)$ for some $i\in X_n$. Since $N\trianglelefteq A_n$, $\tau\in N$. Hence $\sigma^{-1}\tau\in N$ and $\sigma^{-1}\tau\neq 1_{A_n}$ and $\sigma^{-1}\tau(i)=i$. Thus $\sigma^{-1}\tau\in\text{Stab}_{A_n}(i)$ and so $N\cap\text{Stab}_{A_n}(i)\neq\{1_{A_n}\}$. $N$ being normal to $A_n$ implies that $$N\cap\text{Stab}_{A_n}(i)\trianglelefteq\text{Stab}_{A_n}(i)$$ Since $\text{Stab}_{A_n}(i)$ is simple, we have $N\cap\text{Stab}_{A_n}(i)=\text{Stab}_{A_n}(i)$ so that $\text{Stab}_{A_n}(i)\leq N$. But $\text{Stab}_{A_n}(i)$ contains a $3$-cycle and thus so does $N$. By lemma 5.3.4, $N$ contains all $3$-cycles of $A_n$. By lemma 5.3.3, $N=A_n$ which is a contradiction. Thus we conclude. 
\end{proof}
\end{thm}

In fact, we can show that $A_5$ is the unique simple group of order $60$. We will prove this once we classified small groups. 

\pagebreak
\section{Classification of Groups of Order up to 16}
\subsection{Collection of Useful Results}
\begin{lmm}{}{} Let $G$ be a group such that $g^2=1$ for all $g\in G$. Then $G$ is abelian. \tcbline
\begin{proof}
For $g,h\in G$, we have that $$gh=g^{-1}h^{-1}=(hg)^{-1}=hg$$ and so we conclude. 
\end{proof}
\end{lmm}

Recall the following result from groups and rings. 

\begin{prp}{}{} Let $G$ be a group and $a,b\in G$ such that $a,b$ commutes. If furthermore $\langle a\rangle\cap\langle b\rangle=\{1\}$, then $$\abs{ab}=\lcm(\abs{a},\abs{b})$$ \tcbline
\begin{proof}
Let $n=\abs{ab}$. Since $a$ and $b$ commute, we have that $$1=(ab)^n=a^nb^n$$ so that $a^n=b^{-n}$. Thus $a^n\in\langle b\rangle$. But $a^n\in\langle a\rangle\cap\langle b\rangle=\{1\}$ implies that $a^n=1$. Similarly, we have that $b^n=1$. This means that we have $\abs{a},\abs{b}$ divides $n$. and thus $\lcm(\abs{a},\abs{b})$. We know from groups and rings that $\abs{ab}$ divides $\lcm(\abs{a},\abs{b})$ and so we conclude. 
\end{proof}
\end{prp}

\begin{defn}{Inversion Homomorphism}{} The inversion homomorphism between two subgroups $H$ and $K$ of $G$ is the homomorphism $$\phi:H\to\text{Aut}(K)$$ defined by $\phi_1(k)=k$ and $\phi_h(k)=k^{-1}$ for $h\neq 1$. 
\end{defn}

\begin{lmm}{Fitting's Lemma}{} Let $G$ be a finite group. Suppose that $K\trianglelefteq G$ is abelian with odd order $\frac{\abs{G}}{2}$. Let $H=\langle x\rangle\in\text{Syl}_2(G)$. Define $$[K,x]=\langle[v,x]\;|\;v\in K\rangle$$ Then the following are true. 
\begin{itemize}
\item $xax^{-1}=a^{-1}$ for all $a\in[K,x]$
\item $K=C_K(x)\times[K,x]$
\item $G\cong(H\ltimes_\phi[K,x])\times C_K(x)$ where $\phi:H\to\text{Aut}(K)$ is the inversion homomorphism. 
\end{itemize} \tcbline
\begin{proof}~\\
\begin{itemize}
\item Since $K$ is normal, $[K,x]$ is contained in $K$. Moreover, since $K$ is abelian, we have that $xax^{-1}=a^{-1}$ for $a\in[K,x]$ if and only if $xax^{-1}=a^{-1}$ for all $a$ in the generating set of $[K,x]$. So it suffices to prove that $x[v,x]x^{-1}=[v,x]^{-1}$ for all $v\in K$. But we have $$x(vxv^{-1}x^{-1})x=xvxv^{-1}=[v,x]^{-1}$$ since $x=x^{-1}$ and so we are done. 
\item Define $f:K\to[K,x]$ by $f(k)=[k,x]$ for $k\in K$. Then 
\begin{align*}
f(k_1k_2)&=k_1k_2xk_2^{-1}k_1^{-1}x^{-1}\\
&=k_1k_2xk_2^{-1}x^{-1}xk_1^{-1}x^{-1}\\
&=k_1k_2(xk_2^{-1}x^{-1})(xk_1^{-1}x^{-1})\\
&=k_1(xk_2^{-1}x^{-1})k_2(xk_1^{-1}x^{-1})\tag{$K$ is abelian}\\
&=[k_1,x][k_2,x]\\
&=f(k_1)f(k_2)
\end{align*}
Thus $f$ is a homomorphism. Clearly $\ker(f)=C_K(x)$. Also since $\im(f)$ is a subgroup of $[K,x]$ and contains a generating set for $[K,x]$, we have $\im(f)=[K,x]$. Thus by the first isomorphism theorem, we have that $$\abs{K}=\abs{\ker(f)}\abs{\im(f)}=\abs{C_K(x)}\abs{[K,x]}$$ Thus $K=C_K(x)[K,x]$. Since $K$ is abelian, we have that $K=C_K(x)\times[K,x]$ and so we conclude. 
\item Step 1: $[K,x]\trianglelefteq G$. \\
Notice that $G=\langle x\cup K\rangle$. We just have to show that every element in the generating set normalizes $[K,x]$. By the first part of this lemma, it is easy to see that $x$ normalizes $[K,x]$. Since $K$ is abelian, $K$ also normalizes $[K,x]$. Thus we are done. 
\end{itemize}
\end{proof}
\end{lmm}

\subsection{The Dihedral Groups}
\begin{defn}{The Dihedral Group}{} Let $X=\{1,\dots,n\}$ for $n\geq 3$. Define $$\sigma=\begin{pmatrix}1 & \cdots &  n\end{pmatrix}\in\text{Sym}(X)\text{\;\;\;\;and\;\;\;\;}\tau=\prod_{i=1}^{\floor{n/2}}\begin{pmatrix}i & n-i+1\end{pmatrix}\in\text{Sym}(X)$$ Define the dihedral group to be $$D_{2n}\cong\langle\sigma,\tau\rangle$$ In particular, $D_{2n}$ is a subgroup of $\text{Sym}(X)$. 
\end{defn}

\begin{prp}{}{} Let $n\geq 3$. Then $D_{2n}\cong\langle a,b\;|\;a^n, b^2, ab=ba^{-1}\rangle$
\end{prp}

\begin{prp}{}{} Let $n\geq 3$. Then $\abs{D_{2n}}=2n$. 
\end{prp}

\begin{prp}{}{} Let $n\geq 3$. Write $D_{2n}=\langle\sigma,\tau\rangle$. Then $\langle\sigma\rangle$ is a normal subgroup of $D_{2n}$. 
\end{prp}

With the inversion homomorphism, the following proposition shows that $D_{2n}$ can be decomposed into a semidirect product. 

\begin{prp}{}{} Let $\sigma=\begin{pmatrix}1 & \cdots &  n\end{pmatrix}\in S_n$ and $\tau=\prod_{i=1}^{\floor{n/2}}\begin{pmatrix}i & n-i+1\end{pmatrix}\in S_n$. Write $K=\langle\sigma\rangle$ and $H=\langle\tau\rangle$. Then $$D_{2n}\cong H\ltimes_\phi K$$ where $\phi:H\to\text{Aut}(K)$ is the inversion homomorphism. \tcbline
\begin{proof}
It is clear that $\abs{\tau}=2$ and $\abs{\sigma}=n$ and $D_{2n}=K\amalg\tau K$. This implies that $G=HK$ and $H\cap K=\{1\}$. Hence $G\cong H\ltimes_\phi K$ where $\phi:\{1,\tau\}\to\text{Aut}(K)$ is given by $\phi_1(k)=k$ and $\phi_{\tau}(k)=k^{-1}$. 
\end{proof}
\end{prp}

\begin{thm}{}{} Let $G$ be a nonabelian finite group. Further suppose that 
\begin{itemize}
\item $G$ has a cyclic subgroup of $K$ of order $n=\frac{\abs{G}}{2}$
\item $G\setminus K$ contains an element of $G$ of order $2$
\item If $i\in\{0,\dots,n-1\}$ satisfies $i^2\equiv 1\;(\bmod\; n)$ implies $i\equiv \pm 1\;(\bmod\; n)$
\end{itemize}
Then $G\cong D_{2n}$. \tcbline
\begin{proof}
Let $G$ be a group satisfying the above requirements. Then $K$ is a normal subgroup of $G$. Let $x$ be an element of $G\setminus K$ of order $2$. Let $H=\langle x\rangle$. Then $H\cap K=\{1_G\}$. This implies that $\abs{HK}=\abs{H}\abs{K}=2p=\abs{G}$. Thus have that $G=HK$. Let $\phi:H\to\text{Aut}(K)$ be the homomorphism defined by $\phi_h(k)=hkh^{-1}$ for each $h\in H$. Then from groups and rings theorem 5.3.5 we have that $$G\cong H\ltimes_\phi K$$~\\

The next step is to show that $\phi_h(k)=k^{-1}$ for all $k\in K$. Write $K=\langle y\rangle$ where $\abs{y}=n$. Then $xyx^{-1}=y^i$ for some $0\leq i<n$. This means that 
\begin{align*}
y&=x^2yx^{-2}\tag{$x^2=1$}\\
&=xy^ix^{-1}\\
&=(xyx^{-1})^i\\
&=y^{i^2}
\end{align*}
It follows that $y^{i^2-1}=1_G$. Since $n=\abs{y}$ we have that $i\equiv\pm1\;(\bmod\;n)$. Since $0\leq i<n$, we have that $i=1$ or $n-1$. If $i=1$, then $xyx^{-1}=y$ means that $G$ is abelian, which is a contradiction. So we must have $xyx^{-1}=y^{-1}$. Thus the homomorphism $\phi_h$ is given by $\phi_1(k)=k$ and $\phi_h(k)=k^{-1}$ for all $k\in K$. \\~\\

By the above proposition, we must have $G\cong H\ltimes_\phi K\cong D_{2n}$. 
\end{proof}
\end{thm}

Some particular integers satisfying the third condition above include
\begin{itemize}
\item If $n=6$, then by inspection it is easy to see that $i^2\equiv 1\;(\bmod 6)$ if and only if $i=1$ or $5$
\item If $n=p$ a prime, then $\Z/p\Z$ is a field and has no zero divisor. We can then factorize the expression into $(i-1)(i+1)\equiv 0\;(\bmod p)$ so that $i=1$ or $p-1$. 
\item If $n=p^2$ a square of prime, the case $p=2$ is obvious by inspection. Namely $i=1$ or $3$. If $p$ is odd, $p^2$ divides $(i-1)(i+1)$. Notice that $p$ cannot divide both $i+1$ and $i-1$ else $p$ divides $i+1+i-1=2i$, which is a contradiction. So $p$ divides either $i-1$ or $i+1$. Thus $i=1$ or $p^2-1$. 
\end{itemize}

These are examples of quadratic residues in number theory. Notice that the above examples all have two solutions. This is similar to the case where a quadratic can have at most two solutions. 

\subsection{Groups of Order $p$, $p^2$, $2p$, $2p^2$}
In this section we classify all groups of order $p$, $p^2$, $2p$, $2p^2$, $pq$ for $p$ a prime. We also give some result on the case $pq$ where $p$ and $q$ are distinct. 

\begin{prp}{}{} If $\abs{G}=p$ is prime, then $G\cong C_p$. \tcbline
\begin{proof}
Let $1\neq g\in G$. By Lagrange's theorem, $\abs{g}|p$ and thus $\abs{g}=p$ and $G=\langle g\rangle$. 
\end{proof}
\end{prp}

\begin{prp}{}{} If $\abs{G}=p^2$ where $p$ is prime, then $G\cong C_{p^2}$ or $G\cong C_p\times C_p$. \tcbline
\begin{proof}
We know by proposition 6.1.1 that all groups of order $p^2$ are abelian. The results then follows by the fundamental theorem of finite abelian groups. 
\end{proof}
\end{prp}

\begin{prp}{}{} If $\abs{G}=2p$ with $p$ an odd prime, then either $G\cong C_{2p}$ or $G\cong D_{2p}$. \tcbline
\begin{proof}
Let $G$ be a group of order $2p$. If $G$ is abelian, then we must have $G\cong C_{2p}$ by the fundamental theorem of finite abelian groups. Now assume that $G$ is non-abelian. Let $P\in\text{Syl}_p(G)$. Since $\abs{\text{Syl}_p(G)}$ divides $\frac{\abs{G}}{\abs{P}}=2$ by corollary 5.2.1, together with $\abs{\text{Syl}_p(G)}\equiv 1\;(\bmod\;p)$, we must have that $\abs{\text{Syl}_p(G)}=1$. This means that $P$ is a normal subgroup of $G$. Since $p$ is odd, all elements of $G$ of order $2$ lies in $G\setminus P$. Now since $\Z/p\Z$ is a field, the only solutions to the equation $x^2-1$ in $\Z/p\Z$ are congruent to $1$ modulo $p$. It then follows from theorem 6.2.3 that $G\cong D_{2p}$ as required. 
\end{proof}
\end{prp}

\begin{defn}{Generalized Dihedral Group}{} Let $p$ be an odd prime. Let $H=C_2=\langle x\rangle$ and let $K=C_p\times C_p$. Let $\phi:H\to\text{Aut}(K)$ be the inversion homomorphism. Define the generalized dihedral group of order $2p^2$ to be the semidirect product $$\text{GD}_{2p^2}=H\ltimes_\phi K$$
\end{defn}

\begin{prp}{}{} Let $G$ be a group of order $2p^2$ where $p$ is odd. Then $G$ is isomorphic to one of the following groups: 
\begin{itemize}
\item The cyclic group $C_{2p^2}$
\item $C_p\times C_{2p}$
\item $C_p\times D_{2p}$
\item The dihedral group $D_{2p^2}$
\item The generalized dihedral group $\text{GD}_{2p^2}$
\end{itemize} \tcbline
\begin{proof}
Suppose that $G$ is abelian. Then by the fundamental theorem of finite abelian groups, $G$ is isomorphic to either $C_{2p^2}$ or $C_{2p}\times C_p$. So suppose that $G$ is not abelian. Let $K\in\text{Syl}_p(G)$ and $H=\langle x\rangle\in\text{Syl}_2(H)$ with $\abs{x}=2$. Then $\abs{K}=p^2$, so $[G:K]=2$ and that $K\trianglelefteq G$. By by proposition 6.2.2, we have that $K$ is isomorphic to either $C_{p^2}$ or $C_p\times C_p$. \\~\\

Case 1: $K\cong C_{p^2}$. \\
It is clear that $G$ now satisfies the first two conditions of theorem 6.1.3. It remains to show the last. Suppose that $i\in\{0,\dots,p^2-1\}$ is such that $i^2\equiv 1\;(\bmod\;p^2)$. Then $p^2$ divides $(i-1)(i+1)$. Since $p$ is odd, either $p$ does not divide $i-1$ or $p$ does not divide $i+1$. This means that either $p^2$ divides $i-1$ or $p^2$ divides $i+1$. Since $0\leq i<p^2$, the only possibilities are either $i=1$ or $i=p^2-1$, and so we conclude that $$G\cong D_{2p^2}$$ in this case. \\~\\

Case 2: $K\cong C_p\times C_p$. \\
In this case, $G$, satisfies all the conditions of Fitting's lemma so that $$G\cong(H\ltimes_\phi K)\times C_K(x)$$ where $\phi$ is the inversion homomorphism. By Lagrange's theorem, we must have $\abs{C_K(x)}=1$ or $p$ or $p^2$. However, since $G=\langle K\cup\{x\}\rangle$, we see that $C_K(x)$ cannot have order $p^2$. Otherwise, $G$ would be abelian since $K$ is abelian.  Note that $\abs{[K,x]}=\frac{\abs{K}}{\abs{C_K(x)}}$ by Fitting's lemma. There are now two cases. \\~\\

Case 2(a): $\abs{C_k(x)}=1$. \\
We then have that $G\cong H\ltimes_\phi K$ since $[K,x]=K$ and $C_K(x)=\{1\}$. Combining the fact that $K\cong C_p\times C_p$, we have that $$G\cong\text{GD}_{2p^2}$$~\\

Case 2(b): $\abs{C_K(x)}=p$. \\
In this case, we have $\abs{C_K(x)}=p=[K,x]$. Thus $H\ltimes_\phi[K,x]$ is a non-abelian group of order $2p$. Hence $H\ltimes_\phi[K,x]\cong D_{2p}$ by proposition 6.2.3. Since $C_K(x)\cong C_p$ by proposition 6.2.1, we deduce that $$G\cong D_{2p}\times C_p$$ which completes the proof. 
\end{proof}
\end{prp}

\begin{prp}{}{} Let $p$ and $q$ be distinct primes with $p<q$ and $p$ does not divide $q-1$. Let $G$ be a group of order $pq$. Then $G\cong C_{pq}$. \tcbline
\begin{proof}
By corollary 5.2.1, we know that $n_p(G)|q$. By Sylow's theorem, we have that $n_p(G)\equiv1\;(\bmod\;p)$. If $ n_p(G)=q$, then $p$ divides $q-1$ which is a contradiction. So $n_p(G)=1$. By a similar argument and the fact that $q>p$, we have that $n_q(G)=1$. Therefore we have that $G$ has a normal $p$-subgroup, say $H$ and a normal $q$-subgroup, say $K$. By Lagrange's theorem, we must have that $H\cap K=\{1\}$. This means that $\abs{HK}=\abs{H}\abs{K}$. By proposition 5.2.5 in Groups and Rings, we have that $G=HK$. Hence we have that $G\cong H\times K$ by proposition 5.2.5 in Groups and Rings. Let $x$ and $y$ be generators of $H$ and $K$ respectively. Then $x$ and $y$ commute since $H\cap K=\{1\}$. So $\abs{xy}=\abs{x}\abs{y}=pq$ by proposition 1.1.9 in Groups and Rings. Thus $G=\langle x,y\rangle$. 
\end{proof}
\end{prp}

\subsection{Groups of Order 4}
The Klein Four Group is the only other group of order $4$ that is not isomorphic to the cyclic group. 
\begin{defn}{The Klein Four Group $K_4$}{} The Klein four group of order $4$ is defined to be the group $$K_4=\{1,a,b,c\}$$ where multiplication is defined by $a^2=b^2=c^2=1$, $ab=ba=c$, $ac=ca=b$ and $bc=cb=a$. 
\end{defn}

Treating $c$ in the definition as $ab$ makes it easy to see that $K_4$ is in fact $D_4$ in disguise. 

\begin{lmm}{}{} The Klein four group $K_4$ is isomorphic to $D_4$ but not isomorphic to $C_4$. 
\end{lmm}

\subsection{Groups of Order 8}
\begin{defn}{Quaternion Group}{} Let $i,j,k$ be indeterminates and consider the set $$Q_8=\{\pm1,\pm i,\pm j,\pm k\}$$ together with a binary operation $\cdot:Q_8\times Q_8\to Q_8$ defined by 
\begin{itemize}
\item $1\cdot g=g\cdot 1=1$ and $(-1)\cdot g=g\cdot(-1)=-g$ for all $g\in Q_8$
\item $i\cdot j=k$, $j\cdot k=i$ and $k\cdot i =j$
\item $j\cdot i=-k$, $k\cdot j=-i$ and $i\cdot k=-j$
\item $(\pm1)^2=1$, $(\pm i)^2=(\pm j)^2=(\pm k)^2=-1$
\end{itemize}
Then $(Q_8,\cdot)$ is called the Quaternion group. 
\end{defn}

\begin{prp}{}{} The following are properties of the quaternion group $Q_8$. 
\begin{itemize}
\item $Z(Q_8)=\{\pm 1\}$
\item $Q_8$ has precisely $1$ element of order $2$, which is $-1$
\item $Q_8$ has precisely $6$ elements of order $4$, which is $\pm i$, $\pm j$ and $\pm k$
\item $Q_8=\langle i,j\rangle=\langle j, k\rangle=\langle k, i\rangle$
\end{itemize}
\end{prp}

\begin{thm}{}{} Any group of order $8$ is isomorphic to one of the following groups: 
\begin{itemize}
\item The Dihedral group $D_8$
\item The Quaternion group $Q_8$
\item The Cyclic group $C_8$
\item $C_4\times C_2$
\item $C_2\times C_2\times C_2$
\end{itemize} \tcbline
\begin{proof}
Let $G$ be a group of order $8$. Suppose that $G$ is abelian. Then $G$ is isomorphic to one of $C_2\times C_2\times C_2$ or $C_4\times C_2$ or $C_8$ by the fundamental theorem of finitely generated abelian groups. \\~\\
So suppose that $G$ is nonabelian. Then $G$ has no element of order $8$ else $G\cong C_8$. Moreover $G$ must have at least one nontrivial element of order not equal to $2$ else $G$ is abelian. So say $u\in G$ has order larger than $2$. By Lagrange's theorem we must have $\abs{u}=4$. Let $H=\langle u\rangle$. Let $v\in G\setminus H$ with $\abs{v}$ minimal. Then either $\abs{v}=2$ or $4$. There are two cases. \\~\\

Case 1: $\abs{v}=2$\\
Suppose that $\abs{v}=2$. Then $G$ satisfies the three hypotheses for the Dihedral group by proposition 4.1.2 and thus $G\cong D_8$. \\~\\

Case 2: $\abs{v}=4$\\
Suppose that $\abs{v}=4$. Then since $v\notin H$ and $[G:H]=2$, we must have $G/H=\{H,vH\}$. Since $v\in H$ has minimal order, this means that all elements in $vH$ has order $4$. Since furthermore $u,u^{-1}\in H$ both has order $4$, there are precisely $6$ elements of order $4$ and one element of order $2$ ($u^2$ has order $2$). It follows that for $x$ an element of order $6$, $x^2=u^2$ since $u^2$ has order $2$. Some algebra shows that $u^2x=xu^2$ which means that $u^2\in Z(G)$. We also have $u^2x=x^3=x^{-1}$ for all $x$ with order $6$. \\~\\
Side claim: $vuv^{-1}=u^{-1}$\\
Clearly $G=\langle u,v\rangle $ since $v\notin H=\langle u\rangle$, and $\abs{H}=4$. Moreover, $H\trianglelefteq G$ since $[G:H]=2$. Thus $vuv^{-1}\in H$. If $vuv^{-1}=u$, the $v$ and $u$ commute, thus $G$ is abelian which is a contradiction. $u^3$ is then the only element of order $4$ in $H$ and we know that $\abs{vuv^{-1}}=\abs{u}$. This means that $vuv^{-1}=u^{-1}$. \\~\\
Continuing the proof, let $w=uv$. We have that 
\begin{align*}
w&=uv\\
&=vvuv^{-1}\\
&=vu^{-1}\in vH\tag{$vuv^{-1}=u^{-1}$}\\
\end{align*}
We already know that $u,u^{-1},v,v^{-1}$ are distinct elements since they lie in different cosets. But also now $w,w^{-1}$ is not any of the four. Indeed $w\in vH$ thus $w,w^{-1}$ will not be equal to $u,u^{-1}$. But if $w=v$, then this implies $u=1$, a contradiction. Similarly, if $w=v^{-1}$, then $v^2=u$ which is also a contradiction since $\abs{v^2}=2$ while $\abs{u}=4$. \\~\\
Thus we must have that $$G=\{1,u^2,u^{\pm 1}, v^{\pm 1}, w^{\pm 1}\}=\{1,u,u^2,u^3,v,u^2v,w,u^2w\}$$ Notice that 
\begin{align*}
uv&=vv^{-1}uv\\
&=vu^{-1}\\
&=vu^3\\
&=(vu)(u^2)\\
&=(u^2)(vu)
\end{align*}
A similar argument show that $gh=u^2hg$ for all $g,h\in\{u,v,w\}$. Using the fact that $v^2=w^2=u^2\in Z(G)$, we see that 
\begin{itemize}
\item $u^2g=gu^2$ for all $g\in G$
\item $uv=w$, $vw=vuv=u^2v^2u=u$ and $wu=uvu=u^2vu^2=v$
\item $vu=u^2w$, $wv=u^3$, $uw=u^2v$
\item $(\pm 1)^2=1$ and $(\pm g)^2=u^2$ for all $g\in\{u,u^3,v,u^2v,w,u^2w\}$
\end{itemize}
It is easy to see that the Cayley diagram can be formed from these relations which shows that this group is isomorphic to $Q_8$. 
\end{proof}
\end{thm}

\subsection{Groups of Order 12}
\begin{defn}{Dicyclic Group of Order $12$}{} Let $C_4=\langle h\rangle$. Define $\phi:C_4\to\text{Aut}(C_3)$ to be the inversion homomorphism for. Define the dicyclic group of order $12$ to be $$\text{Dic}_{12}=C_4\ltimes_\phi C_3$$ 
\end{defn}

\begin{thm}{}{} Let $G$ be a group of order $12$. Then $G$ is isomorphic to one of the following groups: 
\begin{itemize}
\item The cyclic group $C_{12}$
\item $C_6\times C_2$
\item $\text{Dic}_{12}$
\item The dihedral group $D_{12}$
\item The alternating group $A_4$
\end{itemize} \tcbline
\begin{proof}
When $G$ is abelian, the fundamental theorem of finite abelian group tells us that $G$ is isomorphic to either $C_{12}$ or $C_6\times C_2$. So suppose that $G$ is non-abelian. We split into two cases. \\~\\

Case 1: $G$ contains an element of order $6$. \\
Let $\abs{a}=6$ and $K=\langle a\rangle$. \\~\\

Case 1(a): $G\setminus K\rangle$ has an element of order $2$. \\
Then $G$ satisfies the first two hypothesis of theorem 6.1.3. It remains to show that the final condition is satisfied. But it is clear that $2^2,3^2,4^2$ is congruent to $4,3,4$ respectively modulo $6$. So $i^2$ modulo $6$ is $1$ if and only if $i=1$ or $5$. So we can conclude that in this case, $G\cong D_{12}$. \\~\\

Case 1(b): No element of $G\setminus K$ is of order $2$. \\
Let $H\in\text{Syl}_2(G)$. Then $H$ is not a subgroup of $K$ since $K$ has two elements of order $2$ and $\abs{H}=4$. If every non-identity element of $H$ has order $2$, then $H\setminus K$ would consist of elements of order $2$, contrary of our assumption that $G\setminus K$ has no elements of order $2$. Thus $H$ must have an element of order $4$ and thus $H\cong C_4$. Now let $K_1=\langle a^2\rangle$. Then $\abs{K_1}=3$. Thus $K_1\in\text{Syl}_3(G)$. \\~\\

Claim: $K_1$ is normal. We want to show that $ga^2g^{-1}\in K_1$ and $ga^4g^{-1}\in K_1$. Since $K$ is normal, we have that $ga^2g^{-1}=a^i$ for $i\in\{0,\dots,5\}$. But $(ga^2g^{-1})^3=1$ implies that $a^{3i}=1$. This means that $3i$ is a multiple of $6$. This means that $i$ is a multiple of $2$. Thus $ga^2g^{-1}\in K_1$. A similar method shows that $ga^4g^{-1}\in K_1$. Thus $K_1$ is normal. \\~\\

We now have that $H\leq G$, $K_1\trianglelefteq G$, $H\cap K_1=\{1\}$ by Lagrange's theorem and $\abs{HK_1}=\abs{H}=\abs{K_1}=\abs{G}$ implies that $HK_1=G$ by proposition 5.2.5 in Groups and Rings. Using proposition 5.3.5 in Groups and Rings, we have that $G\cong H\ltimes_\phi K$ where $\phi:H\text{Aut}(K)$ is given by $\phi_h(k)=hkh^{-1}$. Suppose that $H\cong C_4=\langle h\rangle$ and $k\in K\setminus\{1_G\}$. Then $G=\langle h,k\rangle$ so $hkh^{-1}\neq k$ since $G$ is non-abelian. Since $hkh^{-1}\in K_1$ and $\abs{hkh^{-1}}=\abs{k}$, we must have that $hkh^{-1}=k^{-1}$. This means $\phi$ is the inversion homomorphism and that $G$ is just the dicyclic group of order $12$ by definition. \\~\\

Case 2: $G$ has no element of order $6$. \\
By Cauchy's theorem, we may choose an element of order $3$. Since $x\in C_G(x)$ and $G$ has no element of order $6$, we see that $2$ does not divide $\abs{C_G(x)}$. Indeed if it did, then we could choose an element of order $2$ in $C_G(x)$ by Cauchy's theorem. But then $\abs{xy}=6$ by proposition 1.3.6 in Groups and Rings. Thus we must have that $\abs{C_G(x)}=3$ by Lagrange's theorem. By the orbit stabilizer theorem, we have that $\abs{\text{Cl}(x)}=[G:C_G(x)]=4$. Since $G$ has precisely $n_3(G)\times(3-1)$ elements of order $3$ by corollary 5.3.1, we deduce that $n_3(G)>1$. In particular, $H$ is not a normal subgroup of $G$. \\~\\

Finally, let $H\in\text{Syl}_3(G)$ and consider the action of $G$ on $\frac{G}{H}$ by left multiplication. Let $K$ be the kernel of this action. Then $K\leq H$ by ???. But $H$ is not normal in $G$. So we must have $\abs{K}=1$ since $\abs{H}=3$. Thus $G\cong\frac{G}{K}$ is isomorphic to a subgroup of $S_4$. But the only subgroup of $S_4$ of order $12$ is $A_4$. Thus $G\cong A_4$. 
\end{proof}
\end{thm}

\subsection{Unique Simple Group of Order 60}
As an application of the classification of small groups, together with the Sylow theorems, we prove that the unique group of order $60$ is $A_5$. 

\begin{thm}{}{} Let $G$ be a simple group of order $60$. Then $G\cong A_5$. \tcbline
\begin{proof}~\\
Step 1: $G$ does not contain a proper subgroup of index $n\leq 4$. 
Assume the contrary. Let $H\leq G$ with $1\neq[G:H]\leq 4$. Then $G$ acts non-trivially by left multiplication on the set of left cosets $G/H$ of $H$ in $G$, and $\abs{G/H}=n$ for some $2\leq n\leq 4$. Since $G$ is simple, the kernel of this action is trivial. Thus $G$ is isomorphic to a subgroup of $S_n$. Since $2\leq n\leq 4$, we have that $\abs{S_n}\leq 4!=24$. But $\abs{G}=60$ so that this contradicts Lagrange's theorem. \\~\\

Step 2: $G$ contains a subgroup of index $5$. \\
Assume the contrary. Let $P\in\text{Syl}_2(G)$. By Sylow's theorem, we have that $n_2(G)$ is odd and $n_2(G)$ divides 15. Moreover $n_2(G)\neq 1$ since $G$ is simple. We also know that $n_2(G)=\frac{\abs{G}}{\abs{N_G(P)}}$ so by step 1, $n_2(G)=5$ or $15$. If $n_2(G)=5$ then $\frac{\abs{G}}{\abs{N_G(P)}}=5$, a contradiction. \\~\\

So suppose $n_2(G)=15$. Let $g\in G\setminus N_G(P)$. Let $H=\langle P\cup gPg^{-1}\rangle$. Then $H$ contains at least two Sylow $2$-subgroups namely $P$ and $gPg^{-1}$. So $n_2(H)$ being an odd number, is at least $3$. Since $4$ divides $\abs{H}$, we deduce that $\abs{H}$ is divisible by $4n_2(H)$. By step 1, we have either $H=G$ or $\abs{H}<\frac{\abs{G}}{4}=15$. Thus we must have either $H\neq G$ and $n_2(H)=3$ or $H=G$. If $H\neq G$ and $n_2(H)=3$, then 12 divides $\abs{H}$ so we must have $\abs{H}=12$ by step $1$. Thus $[G:H]=5$ as needed. \\~\\

Now assume that for all $g\in G\setminus N_G(P)$, we then have $G=\langle P\cup gPg^{-1}\rangle$. Assume that $x\in P\cap gPg^{-1}$, then $x$ is centralized by both $P$ and $gPg^{-1}$, since groups of order $4$ are abelian. Thus $x$ is centralized in $G$ since $G=\langle P\cup gPg^{-1}\rangle$. Then $x\in Z(G)$. Since $G$ is simple, we have $Z(G)=\{1\}$ and so $x=1$ and we conclude that $P\cap gPg^{-1}=\{1\}$ for all $g\in G$. Now $$F_2(G)=\bigcup_{P\in\text{Syl}_2(G)}P\setminus\{1_G\}$$ by consequences of the Sylow theorems. By the above work, this is a disjoint union. Thus $\abs{F_2(G)}=n_2(G)\times 3=45$. By Sylow's theorem, we have that $n_5(G)\equiv 1\;(\bmod\;5)$ and $n_5(G)$ divides $12$. Since $G$ is simple, we have $n_5(G)>1$ so that $n_5(G)=6$. But then $\abs{F_5(G)}=6\times 4=24$. This means that $\abs{G}>\abs{F_2(G)}+\abs{F_5(G)}=69$, a contradiction. This completes the proof of step 2. \\~\\

Step 3: Final result. \\
Let $H\leq G$ be a subgroup of index $5$. Then $G$ acts non-trivially by left multiplication on the set of left cosets $G/H$ of $H$ in $G$. Since $G$ is simple, the kernel of this action is trivial, and thus $G$ is isomorphic to a subgroup of $S_5$. But the only subgroup of $S_5$ of order $60$ is $A_5$, which completes the proof. 
\end{proof}
\end{thm}


\pagebreak
\section{Series of Subgroups}
\subsection{Series of Subgroups}
\begin{defn}{Subnormal Series}{} Let $G$ be a group. A subnormal series  of a group $G$ is a sequence of subgroups, each a normal subgroup of the next one. This is written as $$\{1\}=G_0\triangleleft G_1\triangleleft\cdots\triangleleft G_n=G$$ In this case we say that the series has length $n$. 
\end{defn}

\begin{defn}{Types of Subnormal Series}{} Let $G$ be a group and $$\{1\}=G_0\triangleleft G_1\triangleleft\cdots\triangleleft G_n=G$$ a subnormal series of $G$. Then the subnormal series can have additional properties. 
\begin{itemize}
\item The series is normal if each $G_i\triangleleft G$ for $0\leq i\leq r$
\item The series is a composition series if $G_{i+1}/G_i$ is simple for  $0\leq i\leq r-1$
\item The series is a central series if it is a normal series in which $G_{i+1}/G_i\leq Z(G/G_i)$ for  $0\leq i\leq r-1$
\end{itemize}
\end{defn}

\subsection{Composition Series}
\begin{lmm}{}{} Let $G$ be a finite group. Let $N$ be a normal subgroup of $G$. Suppose that $$\{1_G\}=N_0\triangleleft N_1\triangleleft\cdots\triangleleft N_r=N$$ and $$\{1_{G/N}\}=\frac{X_0}{N}\triangleleft\frac{X_1}{N}\triangleleft\cdots\triangleleft\frac{X_s}{N}=\frac{G}{N}$$ are composition series for $N$ and $\frac{G}{N}$ respectively. Set $G_i=N_i$ for $0\leq i\leq r$ and $G_i=X_i$ for $r+1\leq i\leq r+s$. Then $$G_0\triangleleft\cdots\triangleleft G_s$$ is a composition series of $G$. \tcbline
\begin{proof}
It is clear that we can write the composition series of $\frac{G}{N}$ as as quotient groups by the correspondence theorem: Suppose we write $\overline{G}_i$ for the composition series of $\frac{G}{N}$. Fix $0\leq i\leq s$. By the correspondence theorem, there exists a subgroup $X_i$ of $G$ containing $N$ such that $\overline{G}_i=\frac{X_i}{N}$. \\~\\

Since $\overline{G}_i$ is a normal subgroup of $\overline{G}_{i+1}$, it follows that $X_i$ is a normal subgroup of $X_{i+1}$. By the third isomorphism theorem, we have $$\frac{\overline{G}_{i+1}}{\overline{G}_i}=\frac{X_{i+1}/N}{X_i/N}\cong\frac{X_{i+1}}{X_i}$$ so that $\frac{X_{i+1}}{X_i}$ is simple. Now notice that $X_s=G$ and $N_r=N=X_0$. Set $G_i=N_i$ for $0\leq i\leq r$ and set $G_i=X_{i-r}$ for $r+1\leq i\leq r+s$. Then $G_0\triangleleft\cdots\triangleleft G_{r+s}$ is a composition series for $G$.
\end{proof}
\end{lmm}


\begin{prp}{}{} Every finite group has a composition series. \tcbline
\begin{proof}
We induct on $\abs{G}$. If $\abs{G}=1$, there is a trivial composition series. So assume that $\abs{G}>1$, and that every group of order less than $\abs{G}$ has a composition series. If $G$ is simple, we are done. So assume $G$ is not simple. Then $G$ has a normal subgroup $N$ with $\{1_G\}\neq N\neq G$. Then $\abs{N}<\abs{G}$ and $\frac{G}{N}<\abs{G}$, so our inductive hypothesis implies that $N$ has a composition series $N_0\triangleleft\cdots\triangleleft N_r$ and $\frac{G}{N}$ has a composition series $\overline{G}_0\triangleleft\cdots\triangleleft\overline{G}_{r+s}$. The results follow by the above lemma. 
\end{proof}
\end{prp}

\begin{defn}{Equivalent Composition Series}{} Let $G$ be a group. Two composition series of $G$, namely $A_0\triangleleft\cdots\triangleleft A_r$ and $B_0\triangleleft\cdots\triangleleft B_s$ is said to be equivalent if $r=s$ and there exists a bijection $$f:\{A_i/A_{i-1}|1\leq i\leq r\}\to\{B_j/B_{j-1}|1\leq j\leq s\}$$ such that $A_i/A_{i-1}\cong f(A_i/A_{i-1})$ for $1\leq i\leq r$. 
\end{defn}

\begin{thm}{The Jordan–Hölder Theorem}{} Let $A_0\triangleleft\cdots\triangleleft A_r$ and $B_0\triangleleft\cdots\triangleleft B_s$ be two compositions series of a finite group $G$. Then they are equivalent. \tcbline
\begin{proof}
Without loss of generality, let $r\leq s$. We induct on $r$. If $r=0$, then $G=\{1_G\}$ and the result is clear. Assume that $r>0$. There are two cases. \\~\\

Case 1: $A_{r-1}=B_{s-1}$. \\
Then $A_0\triangleleft\cdots\triangleleft A_{r-1}$ and $B_0\triangleleft\cdots\triangleleft B_{s-1}$ are two composition series of $A_{r-1}=B_{s-1}$ of lengths $r-1$ and $s-1$. By induction hypothesis, they are equivalent and hence so are the two composition series of $G$. \\~\\

Case 2: $A_{r-1}\neq B_{s-1}$. \\
Consider $A_{r-1}B_{s-1}$. Since $A_r$ and $B_s$ are proper normal subgroups of $G$, we have $B_{s-1}<A_{r-1}B_{s-1}\trianglelefteq G$. It follows from the third isomorphism theorem that $A_{r-1}B_{s-1}/B_{s-1}$ is a normal subgroup of $G/B_{s-1}$. Since $B_{s-1}<A_{r-1}B_{s-1}$, it is in fact a non-trivial normal subgroup by the correspondence theorem. Since $G/B_{s-1}$ is simple, we have $A_{r-1}B_{s-1}/B_{s-1}=G/B_{s-1}$ so that $A_{r-1}B_{s-1}=G$ by the correspondence theorem. Let $D=A_{r-1}\cap B_{s-1}$. By the second isomorphism theorem, we have $$\frac{G}{A_{r-1}}=\frac{A_{r-1}B_{s-1}}{A_{r-1}}\cong\frac{B_{s-1}}{A_{r-1}\cap B_{s-1}}=\frac{B_{s-1}}{D}$$ and as $G/A_{r-1}$ is simple, so is $B_{s-1}/D$. Similarly, by applying the second isomorphism theorem again, we have $$\frac{G}{B_{s-1}}=\frac{A_{r-1}B_{s-1}}{B_{s-1}}\cong\frac{A_{r-1}}{A_{r-1}\cap B_{s-1}}=\frac{A_{r-1}}{D}$$ and as $G/B_{s-1}$ is simple, so is $A_{r-1}/D$. Now let $$\{1_G\}=D_0\triangleleft\cdots\triangleleft D_t=D$$ be a composition series for $D$. Then $$\{1_G\}=D_0\triangleleft\cdots\triangleleft D_t=D\triangleleft A_{r-1}\triangleleft G$$ and $$\{1_G\}=D_0\triangleleft\cdots\triangleleft D_t=D\triangleleft B_{s-1}\triangleleft G$$ are both composition series of $G$. \\~\\

Since $r\leq s$, we can apply the first case to the get that $$A_0\triangleleft\cdots\triangleleft A_r\text{\;\;\;\; and \;\;\;\;}D_0\triangleleft\cdots\triangleleft D_t=D\triangleleft A_{r-1}\triangleleft G$$ are equivalent. In particular, we have that $r=t+2$ and hence $D_0\triangleleft\cdots\triangleleft D_t=D\triangleleft B_{s-1}\triangleleft G$ has length $t+2=r$. Thus we can apply case $1$ again to conclude that $$B_0\triangleleft\cdots\triangleleft B_s\text{\;\;\;\; and \;\;\;\;}D_0\triangleleft\cdots\triangleleft D_t=D\triangleleft B_{s-1}\triangleleft G$$ are equivalent. Finally, we have seen that $G/A_{r-1}\cong B_{s-1}/D$ and $A_{r-1}/D\cong G/B_{s-1}$. Thus $$D_0\triangleleft\cdots\triangleleft D_t=D\triangleleft A_{r-1}\triangleleft G\text{\;\;\;\; and \;\;\;\;}D_0\triangleleft\cdots\triangleleft D_t=D\triangleleft B_{s-1}\triangleleft G$$ are equivalent so that we conclude. 
\end{proof}
\end{thm}

We have thus proved that the following definition of is an invariant of a group. 

\begin{defn}{Composition Factors}{} Let $G$ be a group and $$\{1_G\}=G_0\triangleleft\cdots\triangleleft G_r=G$$ be a composition series of $G$. We say that the simple groups $\frac{G_{i+1}}{G_i}$ for $0\leq i<r$ are called the composition factors of $G$. 
\end{defn}

\subsection{Soluble Groups and Derived Series}
Soluble group, while interesting in its own right, has two main applications: To the study of Galois theory and to the study of composition series. We will focus on its relations with composition series and leave the relation to Galois theory in Fields and Galois Theory. 

\begin{defn}{Soluble Groups}{} A group $G$ is said to be soluble if it is either trivial, or its composition factors are all cyclic groups of prime order. 
\end{defn}

Notice that all abelian groups are soluble. Indeed every quotient group of $G$ is abelian and so every composition factors of $G$ are abelian simple groups, which must be cyclic groups of prime order. 

\begin{lmm}{}{} Let $G$ be a finite group. Let $N$ be a normal subgroup of $G$. Then $G$ is soluble if and only if both $N$ and $\frac{G}{N}$ are soluble. \tcbline
\begin{proof}
Write $\text{CF}(G)$ the composition factors of $G$. By the Jordan–Hölder theorem and lemma 7.2.1, we have $$\text{CF}(G)=\text{CF}(N)\cup\text{CF}(G/N)$$ Thus $G$ is soluble if and only if both $N$ and $G/N$ are soluble. 
\end{proof}
\end{lmm}

\begin{defn}{Derived Series}{} Let $G$ be a group. Define $G^{(1)}=[G,G]$ and inductively, $G^{(n+1)}=[G^{(n)},G^{(n)}]$ for integers $n>0$. We call the descending series $$G=G^{(0)}\geq G^{(1)}\geq \cdots\geq G^{(n)}\geq \cdots$$ the derived series of $G$. 
\end{defn}

\begin{lmm}{}{} Let $G$ be a group and $G=G^{(0)}\geq G^{(1)}\geq \cdots\geq G^{(n)}\geq \cdots$ its derived series. Then the following are true: 
\begin{itemize}
\item $(G^{(n)})^{(m)}=G^{(n+m)}$
\item $H^{(n)}\leq G^{(n)}$ if $H\leq G$
\end{itemize} \tcbline
\begin{proof}~\\
\begin{itemize}
\item We proceed by induction on $m$. When $m=0$ the results are clear. If $(G^{(n)})^{(m-1)}=G^{(n+m-1)}$, then we have 
\begin{align*}
(G^{(n)})^{(m)}&=[(G^{(n)})^{m-1},(G^{(n)})^{m-1}]\\
&=[G^{(n+m-1)},G^{(n+m-1)}]\\
&=G^{(n+m)}
\end{align*} and so we are done. 
\item Similarly by induction, we have 
\begin{align*}
H^{(n)}&=[H^{(n-1)},H^{(n-1)}]\\
&\leq[G^{(n-1)},G^{(n-1)}]\\
&=G^{(n)}
\end{align*}
\end{itemize}
And so we conclude. 
\end{proof}
\end{lmm}

We focus the rest of the chapter to the case where $G$ is finite. 

\begin{thm}{}{} Let $G$ be a finite group. The following are equivalent characterizations for solubility. 
\begin{itemize}
\item $G$ is soluble. 
\item $G^{(n)}=1$ for some $n\geq 0$ in the derived series of $G$
\item $G$ has a normal series with abelian factors. 
\end{itemize} \tcbline
\begin{proof}~\\
\begin{itemize}
\item $(1)\implies(2)$: Suppose that $G$ is soluble. We proceed by induction on the order of $G$. When $\abs{G}=1$ the results are clear. Suppose that $\abs{G}>1$. Let $N=[G,G]$. Since $N$ is normal and $G$ is soluble, $N$ is soluble by lemma 7.2.1. By definition of solubility, $G$ has all composition factors cyclic groups of prime order. In particular, $G_{r-1}$ is a normal subgroup of $G$ with $$G_r/G_{r-1}=G/G_{r-1}$$ being cyclic of prime order. By proposition 3.4.3, we have that $N\leq G_{r-1}$. In particular, $\abs{N}<\abs{G}$. So $N^{(m)}=1$ by induction hypothesis. Since $G^{(n)}=[G^{(n-1)},G^{(n-1)}]$ by definition, it follows that $G^{(m+1)}=1$ and so we are done. 
\item $(2)\implies(1)$: Suppose that $G^{(n)}=1$ for some $n\in\N$. We will again induct on the order of $G$. This is clearly true if $\abs{G}=1$. So suppose that $\abs{G}>1$. Let $N=[G,G]$. If $N=G$ then we have that $$G^{(n)}=[G,G]^{(n-1)}=G^{(n-1)}=\cdots=G$$ which is a contradiction. So we must have $N<G$. Since $N^{(n-1)}=G^{(n)}=1$, we have that $N$ is soluble by induction hypothesis. Also $G/N=G/[G,G]$ is abelian so that both $N$ and $G/N$ is soluble. Hence $G$ is soluble by lemma 7.4.2. Thus we are done. 
\end{itemize}
\end{proof}
\end{thm}

\begin{crl}{}{} Let $G$ be a finite soluble group and $H\leq G$. Then $H$ is soluble. \tcbline
\begin{proof}
Suppose that $G$ is soluble. Then $G^{(n)}=1$ for some $n\in\N$. Then since $H^{(n)}\leq G^{(n)}$, we have $H^{(n)}=1$ and so $H$ is soluble. 
\end{proof}
\end{crl}

\begin{thm}{}{} Let $p$ be prime and $n\in\N$. Let $G$ be a finite group of order $p^n$. Then $G$ is soluble, and all composition factors of $G$ are isomorphic to $C_p$. \tcbline
\begin{proof}
We prove the theorem by induction on $\abs{G}$. If $\abs{G}=p$ then $G$ is cyclic of prime order, so $G$ is soluble of composition length $1$, and its composition factors are $C_p$. \\~\\

Assume that $\abs{G}>p$. The normal subgroup $Z(G)$ is non-trivial in this case. Since $Z(G)$ is abelian, $Z(G)$ is soluble. By induction hypothesis, $G/Z(G)$ is soluble. Thus $G$ is soluble by lemma 7.4.2. By the same lemma, we have that the composition factors of $G$ is the union of the composition factors of $Z(G)$ and $G/Z(G)$. By induction hypothesis all their composition factors are isomorphic to $C_p$ and so we conclude. 
\end{proof}
\end{thm}

\begin{thm}{}{} Let $G_1,G_2$ be soluble groups. Then $G_1\times G_2$ is soluble. \tcbline
\begin{proof}
Recall that $G_1\times G_2\cong G_1\rtimes_\phi G_2$ where $\phi:G_2\to\text{Aut}(G_1)$ is the trivial homomorphism. Also since $G_1\trianglelefteq G_1\times G_2$, and $G_1\times G_2/G_1\cong G_2$, together with $G_1,G_2$ being soluble implies that $G_1\times G_2$ is soluble. 
\end{proof}
\end{thm}

As usual, this theorem can be applied inductively to see that $G_1\times\cdots\times G_n$ is soluble given that $G_1,\dots,G_n$ are soluble. 


\subsection{Nilpotent Groups}
\begin{defn}{Nilpotent Groups}{} Let $G$ be a group. We say that $G$ is nilpotent if $G$ has a central series of finite length. 
\end{defn}

\begin{defn}{Lower Central Series}{} Let $G$ be a group. Define the lower central series of $G$ to be the descending series of subgroups $$G=G_1\trianglerighteq G_2\triangleright\cdots\triangleright G_n\triangleright$$ where $G_{n+1}=[G_n,G]$. We often write $\gamma_n(G)=G_n=[G_{n-1},G]$ for the $n$th term in the lower central series of $G$. 
\end{defn}

\begin{prp}{}{} Let $G$ be a group. Then $G$ is nilpotent if and only if the lower central series terminates at the trivial subgroup after finitely many steps. 
\end{prp}

\begin{defn}{Nilpotency Class}{} Let $G$ be a nilpotent group. Define the nilpotentcy class of $G$ to be the smallest $n\in\N$ such that $G$ has a central series of length $n$. Equivalently, the nilpotency class is the maximal of $n\in\N$ such that $\gamma_n(G)\neq 1$. 
\end{defn}

\begin{prp}{}{} Let $G$ be a nilpotent group. Let $H\leq G$ and $N\trianglelefteq G$. Then the following are true. 
\begin{itemize}
\item $H$ is nilpotent
\item $\gamma\left(\frac{G}{N}\right)=\frac{\gamma_n(G)N}{N}$ for all $n\in\N$. In particular, $G/N$ is nilpotent. 
\end{itemize}
\end{prp}

\begin{prp}{}{} Let $G$ be a finite nilpotent group. Then $G$ is soluble. 
\end{prp}

\begin{prp}{}{} Let $G$ be a finite group of prime power order. Then $G$ is nilpotent. 
\end{prp}








\end{document}