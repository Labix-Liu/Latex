\documentclass[a4paper]{article}

\input{C:/Users/liula/Desktop/Latex/Headers V1.2.tex}

\pagestyle{fancy}
\fancyhf{}
\rhead{Labix}
\lhead{Hochschild Homology}
\rfoot{\thepage}

\title{Hochschild Homology}

\author{Labix}

\date{\today}
\begin{document}
\maketitle
\begin{abstract}
\end{abstract}
\pagebreak
\tableofcontents
\pagebreak

\section{Differential Graded Algebras}
Recall that a graded ring is a ring that can be decomposed into a direct sum $$R=\bigoplus_{i\in\N}R_i$$ of abelian groups, indexed by $\N$, such that $R_iR_j\subseteq R_{i+j}$. A graded algebra is then an algebra that is graded as a ring. If $x\in R_i$ then we say that $x$ has degree $\deg(x)=i$. 

\begin{defn}{Degree of a Graded Morphism}{} Let $M,N$ be graded $R$-modules and let $f:M\to N$ be an $R$-module homomorphism. We say that $f$ has degree $i$ if $f(M_k)\subseteq N_{k+i}$. 
\end{defn}

\begin{defn}{Differential}{} Let $M$ be a graded $R$-module. We say that an $R$-module homomorphism $d:M\to M$ is a differential if the following are true. 
\begin{itemize}
\item $d$ has degree $1$
\item $d\circ d=0$
\end{itemize}
\end{defn}

It is clear the grades of $M$ form a chain complex with a differential. Depending on whether one wants a chain or a cochain complex, we can define the differential to go upwards in degree, meaning that $d(M_i)\subseteq M_{i+1}$. 

\begin{defn}{Derivation of Degree $k$}{} Let $A$ be a graded algebra. We say that a degree $k$, $A$-algebra homomorphism $d:A\to A$ a derivation of degree $k$ if $$d(xy)=(dx)y+(-1)^{k\deg(x)}x(dy)$$ for all $x,y\in A$. 
\end{defn}

If $k=1$ and $A$ is not graded (so that $A=A_0$), then one recovers the notion of a derivation in chapter 1. 

\begin{defn}{Differential Graded Algebra}{} A differential graded algebra is a graded algebra $A$ together with a differential $d:A\to A$ that is a derivation. 
\end{defn}

\pagebreak
\section{Hochschild Homology}
\subsection{Presimplicial Modules}
\begin{defn}{Presimplicial Modules}{} Let $M$ be a graded $R$-module. We say that $M$ is a presimplicial module if there are $R$-module homomorphisms $d_i:M_n\to M_{n-1}$ for $0\leq i\leq n$ such that $$d_i\circ d_j=d_{j-1}d_i$$ for $0\leq i<j\leq n$. 
\end{defn}

\begin{prp}{}{} Let $M=\bigoplus_{n\in\N}M_n$ be a presimplicial module. Then the components $M_n$ of $M$ together with $d=\sum_{i=0}^n(-1)^id_i$ forms a chain complex. 
\end{prp}

\begin{prp}{}{} Let $M,N$ be a presimplicial $R$-module. A morphism of presimplicial modules is a collection of maps $f_n:M_n\to N_n$ such that $f_{n-1}\circ d_i=d_i\circ f_n$. 
\end{prp}

\subsection{Hoschild Homology}
\begin{defn}{Hochschild Complex}{} Let $M$ be an $R$-module. Define the Hoschild complex to be the chain complex $C(R,M)$ associate to the presimplicial module $\bigoplus_{n\in\N} M\otimes R^{\otimes n}$. This means that \\~\\
\adjustbox{scale=0.95,center}{\begin{tikzcd}
	\cdots & {M\otimes R^{\otimes n+1}} & {M\otimes R^{\otimes n}} & {M\otimes R^{\otimes n-1}} & \cdots & {M\otimes R} & M & 0
	\arrow[from=1-1, to=1-2]
	\arrow["d", from=1-2, to=1-3]
	\arrow["d", from=1-3, to=1-4]
	\arrow[from=1-4, to=1-5]
	\arrow[from=1-5, to=1-6]
	\arrow[from=1-6, to=1-7]
	\arrow[from=1-7, to=1-8]
\end{tikzcd}}\\~\\
and $d$ is defined as follows. Define the map $d_i:M\otimes R^{\otimes n}\to M\otimes R^{\otimes n-1}$ as follows. 
\begin{itemize}
\item If $i=0$, then $d_0(m\otimes r_1\otimes\cdots\otimes r_n)=mr_1\otimes r_2\otimes\cdots\otimes r_n$
\item If $i=n$, then $d_n(m\otimes r_1\otimes\cdots\otimes r_n)=r_nm\otimes r_1\otimes\cdots\otimes r_{n-1}$
\item Otherwise, then $d_i(m\otimes r_1\otimes\cdots\otimes r_n)=m\otimes r_1\otimes\cdots\otimes r_ir_{i+1}\otimes \cdots\otimes r_{n-1}$
\end{itemize}
Now define $d=\sum_{i=0}^n(-1)^id_i$. 
\end{defn}

\begin{defn}{Hochschild Homology}{} Let $M$ be an $R$-module. Define the Hochschild homology of $M$ to be the homology groups of the Hochschild complex $C(R,M)$: $$H_n(R,M)=\frac{\ker(d:M\otimes R^{\otimes n}\to M\otimes R^{\otimes n-1})}{\im(d:M\otimes R^{\otimes n+1}\to M\otimes R^{\otimes n})}=H_n(C(R,M))$$ If $M=R$ then we simply write $$HH_n(R)=H_n(R,R)=H_n(C(R,R))$$
\end{defn}

TBA: Functoriality. 

\begin{prp}{}{} Let $A$ be an $R$-algebra. Then $HH_n(A)$ is a $Z(A)$-module. 
\end{prp}

\begin{prp}{}{} Let $A$ be an $R$-algebra. Then the following are true regarding the $0$th Hochschild homology. 
\begin{itemize}
\item Let $M$ be an $A$-module. Then $H_0(A,M)=\frac{M}{\{am-ma\;|\;a\in A, m\in M\}}$
\item The $0$th Hochschild homology of $A$ is given by $HH_0(A)=\frac{A}{[A,A]}$
\item If $A$ is commutative, then the $0$th Hochschild homology is given by $HH_0(A)=A$. 
\end{itemize}
\end{prp}

\begin{thm}{}{} Let $A$ be a commutative $R$-algebra. Then there is a canonical isomorphism $$HH_1(A)\cong\Omega_{A/R}^1$$
\end{thm}

\subsection{Bar Complex}
\begin{defn}{Enveloping Algebra}{} Let $A$ be an $R$-algebra. Define the enveloping algebra of $A$ to be $$A^e=A\otimes A^\text{op}$$
\end{defn}

\begin{prp}{}{} Let $A$ be an $R$-algebra. Then any $A,A$-bimodule $M$ equal to a left (right) $A^e$-module. 
\end{prp}

\begin{defn}{Bar Complex}{}
\end{defn}

\begin{prp}{}{} Let $A$ be an $R$-algebra. The bar complex of $A$ is a resolution of the $A$ viewed as an $A^e$-module. 
\end{prp}

\begin{thm}{}{} Let $A$ be an $R$-algebra that is projective as an $R$-module. If $M$ is an $A$-bimodule, then there is an isomorphism $$H_n(A,M)=\text{Tor}_n^{A^e}(M,A)$$
\end{thm}


\end{document}