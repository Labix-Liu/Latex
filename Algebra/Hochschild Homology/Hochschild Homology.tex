\documentclass[a4paper]{article}

%=========================================
% Packages
%=========================================
\usepackage{mathtools}
\usepackage{amsfonts}
\usepackage{amsmath}
\usepackage{amssymb}
\usepackage{amsthm}
\usepackage[a4paper, total={6in, 8in}, margin=1in]{geometry}
\usepackage[utf8]{inputenc}
\usepackage{fancyhdr}
\usepackage[utf8]{inputenc}
\usepackage{graphicx}
\usepackage{physics}
\usepackage[listings]{tcolorbox}
\usepackage{hyperref}
\usepackage{tikz-cd}
\usepackage{adjustbox}
\usepackage{enumitem}
\usepackage[font=small,labelfont=bf]{caption}
\usepackage{subcaption}
\usepackage{wrapfig}
\usepackage{makecell}



\raggedright

\usetikzlibrary{arrows.meta}

\DeclarePairedDelimiter\ceil{\lceil}{\rceil}
\DeclarePairedDelimiter\floor{\lfloor}{\rfloor}

%=========================================
% Fonts
%=========================================
\usepackage{tgpagella}
\usepackage[T1]{fontenc}


%=========================================
% Custom Math Operators
%=========================================
\DeclareMathOperator{\adj}{adj}
\DeclareMathOperator{\im}{im}
\DeclareMathOperator{\nullity}{nullity}
\DeclareMathOperator{\sign}{sign}
\DeclareMathOperator{\dom}{dom}
\DeclareMathOperator{\lcm}{lcm}
\DeclareMathOperator{\ran}{ran}
\DeclareMathOperator{\ext}{Ext}
\DeclareMathOperator{\dist}{dist}
\DeclareMathOperator{\diam}{diam}
\DeclareMathOperator{\aut}{Aut}
\DeclareMathOperator{\inn}{Inn}
\DeclareMathOperator{\syl}{Syl}
\DeclareMathOperator{\edo}{End}
\DeclareMathOperator{\cov}{Cov}
\DeclareMathOperator{\vari}{Var}
\DeclareMathOperator{\cha}{char}
\DeclareMathOperator{\Span}{span}
\DeclareMathOperator{\ord}{ord}
\DeclareMathOperator{\res}{res}
\DeclareMathOperator{\Hom}{Hom}
\DeclareMathOperator{\Mor}{Mor}
\DeclareMathOperator{\coker}{coker}
\DeclareMathOperator{\Obj}{Obj}
\DeclareMathOperator{\id}{id}
\DeclareMathOperator{\GL}{GL}
\DeclareMathOperator*{\colim}{colim}

%=========================================
% Custom Commands (Shortcuts)
%=========================================
\newcommand{\CP}{\mathbb{CP}}
\newcommand{\GG}{\mathbb{G}}
\newcommand{\F}{\mathbb{F}}
\newcommand{\N}{\mathbb{N}}
\newcommand{\Q}{\mathbb{Q}}
\newcommand{\R}{\mathbb{R}}
\newcommand{\C}{\mathbb{C}}
\newcommand{\E}{\mathbb{E}}
\newcommand{\Prj}{\mathbb{P}}
\newcommand{\RP}{\mathbb{RP}}
\newcommand{\T}{\mathbb{T}}
\newcommand{\Z}{\mathbb{Z}}
\newcommand{\A}{\mathbb{A}}
\renewcommand{\H}{\mathbb{H}}
\newcommand{\K}{\mathbb{K}}

\newcommand{\mA}{\mathcal{A}}
\newcommand{\mB}{\mathcal{B}}
\newcommand{\mC}{\mathcal{C}}
\newcommand{\mD}{\mathcal{D}}
\newcommand{\mE}{\mathcal{E}}
\newcommand{\mF}{\mathcal{F}}
\newcommand{\mG}{\mathcal{G}}
\newcommand{\mH}{\mathcal{H}}
\newcommand{\mI}{\mathcal{I}}
\newcommand{\mJ}{\mathcal{J}}
\newcommand{\mK}{\mathcal{K}}
\newcommand{\mL}{\mathcal{L}}
\newcommand{\mM}{\mathcal{M}}
\newcommand{\mO}{\mathcal{O}}
\newcommand{\mP}{\mathcal{P}}
\newcommand{\mS}{\mathcal{S}}
\newcommand{\mT}{\mathcal{T}}
\newcommand{\mV}{\mathcal{V}}
\newcommand{\mW}{\mathcal{W}}

%=========================================
% Colours!!!
%=========================================
\definecolor{LightBlue}{HTML}{2D64A6}
\definecolor{ForestGreen}{HTML}{4BA150}
\definecolor{DarkBlue}{HTML}{000080}
\definecolor{LightPurple}{HTML}{cc99ff}
\definecolor{LightOrange}{HTML}{ffc34d}
\definecolor{Buff}{HTML}{DDAE7E}
\definecolor{Sunset}{HTML}{F2C57C}
\definecolor{Wenge}{HTML}{584B53}
\definecolor{Coolgray}{HTML}{9098CB}
\definecolor{Lavender}{HTML}{D6E3F8}
\definecolor{Glaucous}{HTML}{828BC4}
\definecolor{Mauve}{HTML}{C7A8F0}
\definecolor{Darkred}{HTML}{880808}
\definecolor{Beaver}{HTML}{9A8873}
\definecolor{UltraViolet}{HTML}{52489C}



%=========================================
% Theorem Environment
%=========================================
\tcbuselibrary{listings, theorems, breakable, skins}

\newtcbtheorem[number within = subsection]{thm}{Theorem}%
{	colback=Buff!3, 
	colframe=Buff, 
	fonttitle=\bfseries, 
	breakable, 
	enhanced jigsaw, 
	halign=left
}{thm}

\newtcbtheorem[number within=subsection, use counter from=thm]{defn}{Definition}%
{  colback=cyan!1,
    colframe=cyan!50!black,
	fonttitle=\bfseries, breakable, 
	enhanced jigsaw, 
	halign=left
}{defn}

\newtcbtheorem[number within=subsection, use counter from=thm]{axm}{Axiom}%
{	colback=red!5, 
	colframe=Darkred, 
	fonttitle=\bfseries, 
	breakable, 
	enhanced jigsaw, 
	halign=left
}{axm}

\newtcbtheorem[number within=subsection, use counter from=thm]{prp}{Proposition}%
{	colback=LightBlue!3, 
	colframe=Glaucous, 
	fonttitle=\bfseries, 
	breakable, 
	enhanced jigsaw, 
	halign=left
}{prp}

\newtcbtheorem[number within=subsection, use counter from=thm]{lmm}{Lemma}%
{	colback=LightBlue!3, 
	colframe=LightBlue!60, 
	fonttitle=\bfseries, 
	breakable, 
	enhanced jigsaw, 
	halign=left
}{lmm}

\newtcbtheorem[number within=subsection, use counter from=thm]{crl}{Corollary}%
{	colback=LightBlue!3, 
	colframe=LightBlue!60, 
	fonttitle=\bfseries, 
	breakable, 
	enhanced jigsaw, 
	halign=left
}{crl}

\newtcbtheorem[number within=subsection, use counter from=thm]{eg}{Example}%
{	colback=Beaver!5, 
	colframe=Beaver, 
	fonttitle=\bfseries, 
	breakable, 
	enhanced jigsaw, 
	halign=left
}{eg}

\newtcbtheorem[number within=subsection, use counter from=thm]{ex}{Exercise}%
{	colback=Beaver!5, 
	colframe=Beaver, 
	fonttitle=\bfseries, 
	breakable, 
	enhanced jigsaw, 
	halign=left
}{ex}

\newtcbtheorem[number within=subsection, use counter from=thm]{alg}{Algorithm}%
{	colback=UltraViolet!5, 
	colframe=UltraViolet, 
	fonttitle=\bfseries, 
	breakable, 
	enhanced jigsaw, 
	halign=left
}{alg}




%=========================================
% Hyperlinks
%=========================================
\hypersetup{
    colorlinks=true, %set true if you want colored links
    linktoc=all,     %set to all if you want both sections and subsections linked
    linkcolor=DarkBlue,  %choose some color if you want links to stand out
}


\pagestyle{fancy}
\fancyhf{}
\rhead{Labix}
\lhead{Hochschild Homology}
\rfoot{\thepage}

\title{Hochschild Homology}

\author{Labix}

\date{\today}
\begin{document}
\maketitle
\begin{abstract}
\end{abstract}
\pagebreak
\tableofcontents
\pagebreak

\section{Hochschild Homology}
\subsection{Hochschild Homology}
\begin{defn}{Hochschild Complex}{} Let $M$ be an $R$-module. Define the Hoschild complex to be the chain complex $C(R,M)$ given as follows. \\~\\
\adjustbox{scale=0.95,center}{\begin{tikzcd}
	\cdots & {M\otimes R^{\otimes n+1}} & {M\otimes R^{\otimes n}} & {M\otimes R^{\otimes n-1}} & \cdots & {M\otimes R} & M & 0
	\arrow[from=1-1, to=1-2]
	\arrow["d", from=1-2, to=1-3]
	\arrow["d", from=1-3, to=1-4]
	\arrow[from=1-4, to=1-5]
	\arrow[from=1-5, to=1-6]
	\arrow[from=1-6, to=1-7]
	\arrow[from=1-7, to=1-8]
\end{tikzcd}}\\~\\
The map $d$ is defined by $d=\sum_{i=0}^n(-1)^id_i$ where $d_i:M\otimes R^{\otimes n}\to M\otimes R^{\otimes n-1}$ is given by the following formula. 
\begin{itemize}
\item If $i=0$, then $d_0(m\otimes r_1\otimes\cdots\otimes r_n)=mr_1\otimes r_2\otimes\cdots\otimes r_n$
\item If $i=n$, then $d_n(m\otimes r_1\otimes\cdots\otimes r_n)=r_nm\otimes r_1\otimes\cdots\otimes r_{n-1}$
\item Otherwise, then $d_i(m\otimes r_1\otimes\cdots\otimes r_n)=m\otimes r_1\otimes\cdots\otimes r_ir_{i+1}\otimes \cdots\otimes r_{n-1}$
\end{itemize}
\end{defn}

\begin{defn}{Hochschild Homology}{} Let $M$ be an $R$-module. Define the Hochschild homology of $M$ to be the homology groups of the Hochschild complex $C(R,M)$: $$H_n(R,M)=\frac{\ker(d:M\otimes R^{\otimes n}\to M\otimes R^{\otimes n-1})}{\im(d:M\otimes R^{\otimes n+1}\to M\otimes R^{\otimes n})}=H_n(C(R,M))$$ If $M=R$ then we simply write $$HH_n(R)=H_n(R,R)=H_n(C(R,R))$$
\end{defn}

TBA: Functoriality. 

\begin{prp}{}{} Let $A$ be an $R$-algebra. Then $HH_n(A)$ is a $Z(A)$-module. 
\end{prp}

\begin{prp}{}{} Let $A$ be an $R$-algebra. Then the following are true regarding the $0$th Hochschild homology. 
\begin{itemize}
\item Let $M$ be an $A$-module. Then $H_0(A,M)=\frac{M}{\{am-ma\;|\;a\in A, m\in M\}}$
\item The $0$th Hochschild homology of $A$ is given by $HH_0(A)=\frac{A}{[A,A]}$
\item If $A$ is commutative, then the $0$th Hochschild homology is given by $HH_0(A)=A$. 
\end{itemize}
\end{prp}

\begin{thm}{}{} Let $A$ be a commutative $R$-algebra. Then there is a canonical isomorphism $$HH_1(A)\cong\Omega_{A/R}^1$$
\end{thm}

\subsection{Bar Complex}
\begin{defn}{Enveloping Algebra}{} Let $A$ be an $R$-algebra. Define the enveloping algebra of $A$ to be $$A^e=A\otimes A^\text{op}$$
\end{defn}

\begin{prp}{}{} Let $A$ be an $R$-algebra. Then any $A,A$-bimodule $M$ equal to a left (right) $A^e$-module. 
\end{prp}

\begin{defn}{Bar Complex}{}
\end{defn}

\begin{prp}{}{} Let $A$ be an $R$-algebra. The bar complex of $A$ is a resolution of the $A$ viewed as an $A^e$-module. 
\end{prp}

\begin{thm}{}{} Let $A$ be an $R$-algebra that is projective as an $R$-module. If $M$ is an $A$-bimodule, then there is an isomorphism $$H_n(A,M)=\text{Tor}_n^{A^e}(M,A)$$
\end{thm}

\subsection{Relative Hochschild Homology}

\subsection{The Trace Map}
\begin{defn}{The Generalized Trace Map}{} Let $R$ be a ring and let $M$ be an $R$-module. Define the generalized trace map $$\text{tr}:M_r(M)\otimes M_r(A)^{\oplus n}\to M\otimes A^{\otimes n}$$ by the formula $$\text{tr}((m_{i,j})\otimes (a_{i,j})_1\otimes\cdots\otimes(a_{i,j})_n)=\sum_{0\leq i_0,\dots,i_n\leq r}m_{i_0,i_1}\otimes (a_{i_1,i_2})_1\otimes\cdots\otimes (a_{i_n,i_0})_n$$
\end{defn}

\begin{thm}{}{} The trace map defines a morphism of chain complex $$\text{tr}:C_\bullet(M_r(A),M_r(M))\to C_\bullet(A,M)$$
\end{thm}

\subsection{Morita Equivalence and Morita Invariance}
\begin{defn}{}{} Let $R$ and $S$ be rings. We say that $R$ and $S$ are Morita equivalent if there is an equivalence of categories $$\bold{Mod}_R\cong\bold{Mod}_S$$
\end{defn}

\begin{thm}{Morita Invariance for Matrices}{}
\end{thm}


\end{document}