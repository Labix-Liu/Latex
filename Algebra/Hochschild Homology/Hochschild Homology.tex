\documentclass[a4paper]{article}

\input{C:/Users/liula/Desktop/Latex/Headers V1.2.tex}

\pagestyle{fancy}
\fancyhf{}
\rhead{Labix}
\lhead{Hochschild Homology}
\rfoot{\thepage}

\title{Hochschild Homology}

\author{Labix}

\date{\today}
\begin{document}
\maketitle
\begin{abstract}
\end{abstract}
\pagebreak
\tableofcontents
\pagebreak

\section{Hochschild Homology}
\subsection{Hochschild Homology}
\begin{defn}{Hochschild Complex}{} Let $M$ be an $R$-module. Define the Hoschild complex to be the chain complex $C(R,M)$ given as follows. \\~\\
\adjustbox{scale=0.95,center}{\begin{tikzcd}
	\cdots & {M\otimes R^{\otimes n+1}} & {M\otimes R^{\otimes n}} & {M\otimes R^{\otimes n-1}} & \cdots & {M\otimes R} & M & 0
	\arrow[from=1-1, to=1-2]
	\arrow["d", from=1-2, to=1-3]
	\arrow["d", from=1-3, to=1-4]
	\arrow[from=1-4, to=1-5]
	\arrow[from=1-5, to=1-6]
	\arrow[from=1-6, to=1-7]
	\arrow[from=1-7, to=1-8]
\end{tikzcd}}\\~\\
The map $d$ is defined by $d=\sum_{i=0}^n(-1)^id_i$ where $d_i:M\otimes R^{\otimes n}\to M\otimes R^{\otimes n-1}$ is given by the following formula. 
\begin{itemize}
\item If $i=0$, then $d_0(m\otimes r_1\otimes\cdots\otimes r_n)=mr_1\otimes r_2\otimes\cdots\otimes r_n$
\item If $i=n$, then $d_n(m\otimes r_1\otimes\cdots\otimes r_n)=r_nm\otimes r_1\otimes\cdots\otimes r_{n-1}$
\item Otherwise, then $d_i(m\otimes r_1\otimes\cdots\otimes r_n)=m\otimes r_1\otimes\cdots\otimes r_ir_{i+1}\otimes \cdots\otimes r_{n-1}$
\end{itemize}
\end{defn}

\begin{defn}{Hochschild Homology}{} Let $M$ be an $R$-module. Define the Hochschild homology of $M$ to be the homology groups of the Hochschild complex $C(R,M)$: $$H_n(R,M)=\frac{\ker(d:M\otimes R^{\otimes n}\to M\otimes R^{\otimes n-1})}{\im(d:M\otimes R^{\otimes n+1}\to M\otimes R^{\otimes n})}=H_n(C(R,M))$$ If $M=R$ then we simply write $$HH_n(R)=H_n(R,R)=H_n(C(R,R))$$
\end{defn}

TBA: Functoriality. 

\begin{prp}{}{} Let $A$ be an $R$-algebra. Then $HH_n(A)$ is a $Z(A)$-module. 
\end{prp}

\begin{prp}{}{} Let $A$ be an $R$-algebra. Then the following are true regarding the $0$th Hochschild homology. 
\begin{itemize}
\item Let $M$ be an $A$-module. Then $H_0(A,M)=\frac{M}{\{am-ma\;|\;a\in A, m\in M\}}$
\item The $0$th Hochschild homology of $A$ is given by $HH_0(A)=\frac{A}{[A,A]}$
\item If $A$ is commutative, then the $0$th Hochschild homology is given by $HH_0(A)=A$. 
\end{itemize}
\end{prp}

\begin{thm}{}{} Let $A$ be a commutative $R$-algebra. Then there is a canonical isomorphism $$HH_1(A)\cong\Omega_{A/R}^1$$
\end{thm}

\subsection{Bar Complex}
\begin{defn}{Enveloping Algebra}{} Let $A$ be an $R$-algebra. Define the enveloping algebra of $A$ to be $$A^e=A\otimes A^\text{op}$$
\end{defn}

\begin{prp}{}{} Let $A$ be an $R$-algebra. Then any $A,A$-bimodule $M$ equal to a left (right) $A^e$-module. 
\end{prp}

\begin{defn}{Bar Complex}{}
\end{defn}

\begin{prp}{}{} Let $A$ be an $R$-algebra. The bar complex of $A$ is a resolution of the $A$ viewed as an $A^e$-module. 
\end{prp}

\begin{thm}{}{} Let $A$ be an $R$-algebra that is projective as an $R$-module. If $M$ is an $A$-bimodule, then there is an isomorphism $$H_n(A,M)=\text{Tor}_n^{A^e}(M,A)$$
\end{thm}

\subsection{Relative Hochschild Homology}

\subsection{The Trace Map}
\begin{defn}{The Generalized Trace Map}{} Let $R$ be a ring and let $M$ be an $R$-module. Define the generalized trace map $$\text{tr}:M_r(M)\otimes M_r(A)^{\oplus n}\to M\otimes A^{\otimes n}$$ by the formula $$\text{tr}((m_{i,j})\otimes (a_{i,j})_1\otimes\cdots\otimes(a_{i,j})_n)=\sum_{0\leq i_0,\dots,i_n\leq r}m_{i_0,i_1}\otimes (a_{i_1,i_2})_1\otimes\cdots\otimes (a_{i_n,i_0})_n$$
\end{defn}

\begin{thm}{}{} The trace map defines a morphism of chain complex $$\text{tr}:C_\bullet(M_r(A),M_r(M))\to C_\bullet(A,M)$$
\end{thm}

\subsection{Morita Equivalence and Morita Invariance}
\begin{defn}{}{} Let $R$ and $S$ be rings. We say that $R$ and $S$ are Morita equivalent if there is an equivalence of categories $$\bold{Mod}_R\cong\bold{Mod}_S$$
\end{defn}

\begin{thm}{Morita Invariance for Matrices}{}
\end{thm}


\end{document}