\documentclass[a4paper]{article}

\input{C:/Users/liula/Desktop/Latex/Headers V1.2.tex}

\pagestyle{fancy}
\fancyhf{}
\rhead{Labix}
\lhead{Advanced Linear Algebra}
\rfoot{\thepage}

\title{Advanced Linear Algebra}

\author{Labix}

\date{\today}
\begin{document}
\maketitle
\begin{abstract}
Linear algebra is at the heart of mathematics. Almost all areas of mathematics make some use of the notion of vector spaces and its properties. It began with the study of systems of linear equations. \\~\\

Nowadays as we understand vector spaces independently from systems of linear equations, we will also treat the material differently. The first three chapter begins with the basis definitions: vector spaces and the maps between them called linear maps. Eigenvalues and eigenspaces will be an important invariant for vector spaces. Together with the aid of matrices, we will have a good grasp of how to write a given vector space in a simpler form. Chapter 4 will then improve on the further simplifying a given matrix so that we can read information from each easily. \\~\\

The rest of the chapters will focus on particular properties of vector spaces and linear maps. They each correspond to an important class of matrices. For examples, quadratic form corresponds to matrices equivalent up to congruency while orthogonality of basis vectors give orthogonal matrices. 
\end{abstract}
\pagebreak
\tableofcontents


\end{document}