\documentclass[a4paper]{article}

\input{C:/Users/liula/Desktop/Latex/Headers V1.2.tex}

\pagestyle{fancy}
\fancyhf{}
\rhead{Labix}
\lhead{Commutative Algebra 2}
\rfoot{\thepage}

\title{Commutative Algebra 2}

\author{Labix}

\date{\today}
\begin{document}
\maketitle
\begin{abstract}
\end{abstract}
\pagebreak
\tableofcontents
\pagebreak

\pagebreak
\section{Dimension Theory}
\subsection{Dimension and Height}
\begin{defn}{Krull Dimension}{} Let $R$ be a commutative ring. Define the Krull dimension of $R$ to be $$\dim(R)=\sup\{t\in\N|p_0\subset\dots\subset p_t\text{ for }p_0,\dots,p_t\text{ prime ideals }\}$$
\end{defn}

\begin{defn}{Height of a Prime Ideal}{} Let $p$ be a prime ideal in a ring $R$. Define the height of $p$ to be $$\text{ht}(p)=\sup\{t\in\N|p_0\subset\dots\subset p_t=p\text{ for }p_0,\dots,p_t\text{ prime ideals }\}$$
\end{defn}

\begin{lmm}{}{} Let $p$ be a prime ideal in a ring $R$. Then $\text{ht}(p)=\dim(R_p)$. 
\end{lmm}

\subsection{The Hilbert Polynomial}

\subsection{Fundamental Theorem on Local Rings}
\begin{thm}{}{}
\end{thm}

\begin{prp}{}{} Let $(R,m)$ be a Noetherian local ring and let $k=R/m$ be the residue field. Then $$\dim(R)\leq\dim_k(m/m^2)$$
\end{prp}

\begin{thm}{Krull's Principal Ideal Theorem}{}
\end{thm}

\pagebreak
\section{Regular Local Rings}
\subsection{Regular Local Rings}
Regularity is an important concept in algebraic geometry to detecting singularities. We motivate the definition by the following proposition. 

\begin{defn}{Regular Local Rings}{} A local ring $R$ is said to be regular if $\dim_k(m/m^2)=\dim(R)$ for $k$ the residue field of $R$. 
\end{defn}

\begin{thm}{}{} Let $A$ be a Noetherian local ring of dimension $1$ with maximal ideal $m$. Then the following are equivalent: 
\begin{itemize}
\item $A$ is regular
\item $m$ is principal
\item $A$ is an integral domain, and all ideals are of the form $m^n$ for $n\geq 0$ or $(0)$
\item $A$ is a principal ideal domain
\end{itemize}
\end{thm}






\end{document}