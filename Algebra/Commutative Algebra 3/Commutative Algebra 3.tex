\documentclass[a4paper]{article}

\input{C:/Users/liula/Desktop/Latex/Headers V1.2.tex}

\pagestyle{fancy}
\fancyhf{}
\rhead{Labix}
\lhead{Commutative Algebra 3}
\rfoot{\thepage}

\title{Commutative Algebra 3}

\author{Labix}

\date{\today}
\begin{document}
\maketitle
\begin{abstract}
\end{abstract}
\pagebreak
\tableofcontents
\pagebreak

\section{Kähler Differentials}
The goal of this section is to define the derivations and the module of Kähler differentials, as well as seeing some first consequences such as the two exact sequences. To show existence of the module of Kähler differentials, we will see two different constructions of the module. 

\subsection{Derivations}
We begin with the definition of derivations. It will serve as the base of our discussions not only for the module of Kähler differentials, but also for manifolds. \\~\\

By a ring, we mean that it is a commutative ring with identity $1\neq 0$. 

\begin{defn}{Derivations}{2.1.1} Let $A$ be a ring and $B$ an $A$-algebra. Let $M$ be a $B$-module. An $A$-derivation of $B$ into $M$ is an $A$-module homomorphism $d:B\to M$ such that the Leibniz rule holds: $$d(b_1b_2)=b_1d(b_2)+d(b_1)b_2$$ for $b_1,b_2\in B$. Denote the set of all $A$-derivations from $B$ to $M$ by $$\text{Der}_A(B,M)=\{d:B\to M\;|\;d\text{ is an }A\text{ derivation }\}$$
\end{defn}

This is reminiscent of properties of a derivative. Indeed, from the above definition, take $A=\R$ and $B=M=\R[x_1,\dots,x_n]$. Then the formal partial derivatives $\frac{\partial}{\partial x_i}:\R[x_1,\dots,x_n]\to\R[x_1,\dots,x_n]$ defined by $$\left(f(x)=\sum_{k_1,\dots,k_n}a_{k_1,\dots,k_n}x_1^{k_1}\cdots x_i^{k_i}\cdots x_n^{k_n}\right)\mapsto\left(\frac{\partial f}{\partial x_i}=\sum_{k_1,\dots,k_n}a_{k_1,\dots,k_n}k_ix_1^{k_1}\cdots x_i^{k_i-1}\cdots x_n^{k_n}\right)$$ (provided $k_i\geq 1$, otherwise the derivative is constant on that term) is $\R$-linear and satisfies the Leibniz rule. These are the two fundamental properties that a derivative should possess. \\~\\

Derivatives in analysis also satisfy the quotient rule and the fact that constant maps have $0$ derivative. The following lemma shows that instead of defining derivatives for it so that constant maps have $0$ derivative, it is in fact a consequence of linearity and Leibniz rule. 

\begin{lmm}{}{2.1.2} Let $A$ be a ring and $B$ an $A$-algebra Let $M$ be a $B$-module. Let $d:B\to M$ be an $A$-derivation. Then $d(a)=0$ for all $a\in A$. \tcbline
\begin{proof}
Since $d:B\to M$ is an $A$-module homomorphism, $d(a\cdot 1)=a\cdot d(1)$. We also have, by the Leibniz rule that $d(1)=1\cdot d(1)+d(1)\cdot 1=2d(1)$ which implies $d(1)=0$. Thus $d(a\cdot 1)=a\cdot d(1)=0$. 
\end{proof}
\end{lmm}

As mentioned above, derivatives in analysis also satisfy the quotient rule. However, one must be careful in the question of existence of the quotient rule given the Leibniz rule because first of all $B$ and $M$ may not formally have quotients since they are not fields. Instead, what one can do is to pass on the derivative to the fraction field so that quotients are well defined. Interested readers are referred to \cite{Zar-Sam}. \\~\\

The set of all derivations itself also has an extra structure of being a $B$-module in its own right. 

\begin{lmm}{}{2.1.3} Let $A$ be a ring and $B$ an $A$-algebra. Let $M$ be a $B$-module. Then $\text{Der}_A(B,M)$ is a $B$-module with the following operations: 
\begin{itemize}
\item Addition is defined by sending $d_1,d_2:B\to M$ to $(d_1+d_2):B\to M$ that maps $b$ to $d_1(b)+d_2(b)$. 
\item Left action is defined by $\cdot:B\times\text{Der}_A(B,M)\to\text{Der}_A(B,M)$ that sends $b\in B$ and $d:B\to M$ to $(bd):B\to M$ defined by $u\mapsto b\cdot d(u)$.
\end{itemize} \tcbline
\begin{proof}
Firstly, $\text{Der}_A(B,M)$ is an abelian group. We check the group axioms. 
\begin{itemize}
\item Closure: Let $a\in A$ and $b_1,b_2\in B$. $d_1+d_2:B\to M$ is an $A$-module homomorphism because 
\begin{align*}
(d_1+d_2)(ab_1+b_2)&=d_1(ab_1+b_2)+d_2(ab_1+b_2)\\
&=ad_1(b_1)+d_1(b_2)+ad_2(b_1)+d_2(b_2)\\
&=a(d_1+d_2)(b_1)+(d_1+d_2)(b_2)
\end{align*}
Finally, the Leibniz rule is satisfied because 
\begin{align*}
(d_1+d_2)(b_1b_2)&=d_1(b_1b_2)+d_2(b_1b_2)\\
&=b_1d_1(b_2)+d_1(b_1)b_2+b_1d_2(b_2)+d_2(b_1)b_2\\
&=b_1(d_1+d_2)(b_2)+(d_1+d_2)(b_1)b_2
\end{align*}
\item Associativity: Follows from the fact that $M$ is a group
\item Identity: The zero map is the identity since for any $d:B\to M$, $d+0:B\to M$ sends $b$ to $d(b)$ and thus $d+0=d$. 
\item Inverse: For each $d:B\to M$ the maps sending $b$ to $-d(b)$ is an inverse
\item Abelian: Follows from the fact that $M$ is abelian. 
\end{itemize}
Finally, left action is defined by $\cdot:B\times\text{Der}_A(B,M)\to\text{Der}_A(B,M)$ that sends $b\in B$ and $d:B\to M$ to $(bd):B\to M$ defined by $u\mapsto b\cdot d(u)$. Associativity and identity is clear. 
\end{proof}
\end{lmm}

We can see that $\text{Der}_\R(\R[x_1,\dots,x_n],\R[x_1,\dots,x_n])$ has more than just the standard partial derivatives from the module structure. For examples, the sum of partial derivatives $$\frac{\partial}{\partial x_i}+\frac{\partial}{\partial x_j}:\R[x_1,\dots,x_n]\to\R[x_1,\dots,x_n]$$ defined by $f\mapsto\frac{\partial f}{\partial x_i}+\frac{\partial f}{\partial x_j}$. \\~\\

However, second order derivatives (which are compositions of the first order partial derivatives) are not derivations! Indeed they satisfy not the Leibniz property but instead, we have that $$\frac{\partial (fg)}{\partial x_ix_j}=\frac{\partial}{\partial x_i}\left(\frac{\partial f}{\partial x_j}g+f\frac{\partial g}{\partial x_j}\right)=\frac{\partial^2 f}{\partial x_ix_j}+\frac{\partial f}{\partial x_j}\frac{\partial g}{\partial x_i}+\frac{\partial f}{\partial x_i}\frac{\partial g}{\partial x_j}+\frac{\partial^2g}{\partial x_ix_j}$$ which is way more complicated! \\~\\

Finally, there is one more example of derivations. While we have done a completely general treatment of partial derivatives above, we can in fact evaluate the derivative at a chosen point and it will again be an $\R$-derivation. Writing $f(p)=\text{ev}_p(f)$ where $\text{ev}$ is the evaluation homomorphism, the $\R$-derivation $\R[x_1,\dots,x_n]\to\R[x_1,\dots,x_n]$ defined by $$f\mapsto \frac{\partial f}{\partial x_i}g(p)+f(p)\frac{\partial g}{\partial x_i}$$ is also a derivation! 

\begin{prp}{}{2.1.4} Let $B$ be an $A$-algebra. Let $S$ be a multiplicative set of $B$. Let $M$ be an $S^{-1}(B)$-module. Then for any $A$-derivation $d:B\to M$, there exists one unique way of extending the derivation to $d:S^{-1}B\to M$, defined by the formula: $$d\left(\frac{b}{s}\right)=\frac{sd(b)-bd(s)}{s^2}$$ \tcbline
\begin{proof}
Temporarily denote a derivation from $S^{-1}B$ to $M$ by $D$. Suppose that $b\in B$ and $s\in S$. Notice that $D$ has to satisfy the following: $$d(b)=D(b)=D\left(s\frac{b}{s}\right)=\frac{b}{s}D(s)+sD\left(\frac{b}{s}\right)$$ Now multiply both sides by $s^{-1}$ to obtain $$D\left(\frac{b}{s}\right)=\frac{sD(b)-bD(s)}{s^2}$$ Thus any $A$-derivation $S^{-1}B$ to $M$ must satisfy the above formula. This shows that there can only be one unique way of extending it. \\~\\

For existence, we just have to show that it is a well defined map. Suppose that $\frac{a}{r}=\frac{b}{s}$. This means that there exists $q\in S$ such that $q(sa-rb)=0$. The goal is to show that $$\frac{rd(a)-ad(r)}{r^2}=\frac{sd(b)-bd(s)}{s^2}$$ or in other words, there exists $p\in S$ such that $p\left(s^2(rd(a)-ad(r))-r^2sd(b)-bd(s)\right)=0$. I claim that $p=q^2$ does the job. Indeed we have that
\begin{align*}
q^2\left(s^2(rd(a)-ad(r))-r^2sd(b)-bd(s)\right)&=q^2(sad(rs)-rsd(as)-rbd(rs)+rsd(br))\\
&=q^2((sa-rb)d(rs)+rs(d(br-as)))\\
&=rsq^2d(br-as)
\end{align*}
Now in fact, $q^2d(br-as)=0$ because 
\begin{align*}
q^2d(br-as)&=q(qd(br-as))\\
&=q(d(q(br-as))-(br-as)d(q))\\
&=0
\end{align*}
Thus we conclude. 
\end{proof}
\end{prp}

\subsection{Kähler Differentials}
We now define the module of Kähler Differentials which is the main object of study. For each $A$-derivation $d$ from an $A$-algebra $B$ to a $B$-module $M$, $d$ factors through a universal object no matter what $d$ we choose. This is the content of the following definition. 

\begin{defn}{Kähler Differentials}{} A $B$-module $\Omega_{B/A}^1$ together with an $A$-derivation $d:B\to\Omega_{B/A}^1$ is said to be a module Kähler Differentials of $B$ over $A$ if it satisfies the following universal property: \\~\\
For any $B$-module $M$, and for any $A$-derivation $d':B\to M$, there exists a unique $B$-module homomorphism $f:\Omega_{B/A}^1\to M$ such that $d'=f\circ d$. In other words, the following diagram commutes: \\~\\
\adjustbox{scale=1.1,center}{\begin{tikzcd}
B\arrow[r, "d"]\arrow[rd, "d'"'] & \Omega_{B/A}^1\arrow[d, "\exists!f", dashed]\\
& M
\end{tikzcd}}
\end{defn}

The above definition merely shows what properties we would like a module of Kähler differentials to satisfy. Notice that we have yet to show its existence. The above construction is also universal in the following sense. 

\begin{lmm}{}{2.2.2} Let $A$ be a ring and $B$ an $A$-algebra. Let $M$ be a $B$-module. Then there is a canonical $B$-module isomorphism $$\text{Hom}_B(\Omega_{B/A}^1,M)\cong\text{Der}_A(B,M)$$ \tcbline
\begin{proof}
Fix $M$ a $B$-module. Let $d'\in\text{Der}_A(B,M)$. By the universal property of $\Omega_{B/A}^1(M)$, there exists a unique $B$-module homomorphism $f:\Omega_{B/A}^1\to M$ such that $d'=f\circ d$. This gives a map $\phi:\text{Der}_A(B,M)\to\Hom_B(\Omega_{B/A}^1,M)$ defined by $\phi(d')=f$. \\~\\
Conversely, given a map $g\in\Hom_B(\Omega_{B/A}^1,M)$, pre-composition with $d$ gives a pull back map $d^\ast:\Hom_B(\Omega_{B/A}^1,M)\to\text{Der}_A(B,M)$ defined by $d^\ast(g)=g\circ d$. These map are inverses of each other: 
\begin{align*}
(d^\ast\circ\phi)(d')&=d^\ast(f)\\
&=f\circ d\\
&=d'\tag{By universal property}
\end{align*} and 
$(\phi\circ d^\ast)(g)=\phi(g\circ d)=g$. 
Thus these map is a bijective map of sets. \\~\\

It remains to show that $d^\ast$ is a $B$-module homomorphism. Let $f,g\in\Hom_B(\Omega_{B/A}^1,M)$. 
\begin{itemize}
\item $d^\ast(f+g)=(f+g)\circ d$ is a map $$b\overset{d}{\mapsto}d(b)\overset{f+g}{\mapsto}f(d(b))+g(d(b))$$ for $b\in B$. $d^\ast(f)+d^\ast(g)=f\circ d+g\circ d$ is a map $$b\mapsto f(d(b))+g(d(b))$$ thus addition is preserved by $d^\ast$. 
\item Let $u\in B$. We want to show that $d^\ast(u\cdot f)=u\cdot d^\ast(f)$. The left hand side sends an element $b\in B$ by $$b\overset{d}{\mapsto}d(b)\overset{u\cdot f}{\mapsto}u\cdot f(d(b))$$ The right hand side sends $b\mapsto u\cdot f(d(b))$. Thus proving they are the same. 
\end{itemize}
And so we have reached the conclusion. 
\end{proof}
\end{lmm}

The definition of the module and the above lemma shows the following: The functor $M\mapsto\text{Der}_A(B,M)$ between the category of $B$-modules is representable. Indeed, one may recall that a functor is said to be representable if it is naturally isomorphic to the $\Hom$ functor together with a fixed object, which is precisely the content of the above lemma. \\~\\

Let us now see an explicit construction of the module to prove the existence of the module of Kähler Differentials. 

\begin{prp}{}{} Let $A$ be a ring and $B$ be an $A$-algebra. Let $F$ be the free $B$-module generated by the symbols $\{d(b)\;|\;b\in B\}$. Let $R$ be the submodule of $F$ generated by the following relations: 
\begin{itemize}
\item $d(a_1b_1+a_2b_2)-a_1d(b_1)-a_2d(b_2)$ for all $b_1,b_2\in B$ and $a_1,a_2\in A$
\item $d(b_1b_2)-b_1d(b_2)-b_2d(b_1)$ for all $b_1,b_2\in B$
\end{itemize}
Then $F/R$ is a module of Kähler Differentials for $B$ over $A$. \tcbline
\begin{proof}
Clearly $F/R$ is a $B$-module. Moreover, define $d:B\to F/R$ by $b\mapsto d(b)+R$. This map is an $A$-derivation since the following are satisfied: 
\begin{itemize}
\item $d$ is an $A$-module homomorphism: Let $b_1,b_2\in B$ and $a_1,a_2\in A$. Then $a_1b_1+a_2b_2$ is mapped to $d(a_1b_1+a_2b_2)+R$. We know from the relations that $d(a_1b_1+a_2b_2)+R=a_1d(b_1)+a_2d(b_2)+R$. Thus $d$ is $A$-linear. 
\item $d$ satisfies the Leibniz rule: Let $b_1,b_2\in B$. Then $b_1b_2$ is mapped to $d(b_1b_2)+R$. Since $d(b_1b_2)+R=b_1d(b_2)+d(b_1)b_2$, we have that $b_1b_2$ is mapped to $b_1d(b_2)+d(b_1)b_2+R$. 
\end{itemize}
This shows that $d:B\to F/R$ is an $A$ derivation. \\~\\

It remains to show that $(F/R,d)$ has the universal property. Let $M$ be a $B$-module and $d':B\to M$ an $A$-derivation. Define a map $f:F\to M$ on generators by $d(b)\mapsto d'(b)$ and extending from generators to the entire module. This is a $B$-module homomorphism by definition. Clearly $f\circ d=d'$. It also unique since $f$ is defined on the generators of $F$. \\~\\

Finally we want to show that $f$ projects to a map $\bar{f}:F/R\to M$. This requires us to check that $f(d(a_1b_1+a_2b_2))=f(a_1d(b_1)+a_2d(b_2))$ and $f(d(b_1b_2))=f(b_1d(b_2)+d(b_1)b_2)$. But this is clear. Since $f:F\to R$ is a $B$-module homomorphism, we have $$f(d(a_1b_1+a_2b_2))-f(a_1d(b_1)+a_2d(b_2))=0$$ and $$f(d(b_1b_2))-f(b_1d(b_2)+d(b_1)b_2)=0$$ implying $f$ sends $d(a_1b_1+a_2b_2)-a_1d(b_1)-a_2d(b_2)$ and $d(b_1b_2)-b_1d(b_2)-d(b_1)b_2$ to $0$. Since we checked them on generators of $R$ this result extends to all of $R$. Thus we are done. 
\end{proof}
\end{prp}

Aside from the construction through quotients, we can also express the module explicitly via the kernel of a diagonal morphism. Using the universal property, we see that all these constructions are the same. 

\begin{prp}{}{} Let $A$ be a ring and $B$ be an $A$-algebra. Let $f:B\otimes_A B\to B$ be a function defined to be $f(b_1\otimes_A b_2)=b_1b_2$. Let $I$ be the kernel of $f$. Then $(I/I^2,d)$ is a module of Kähler Differentials of $B$ over $A$, where the derivation is the homomorphism $d:B\to I/I^2$ defined by $db=1\otimes b-b\otimes1\;(\bmod\;I^2)$. \tcbline
\begin{proof}
We break down the proof in 3 main steps. \\~\\
Step 1: Show that $\ker(f)=\langle 1\otimes b-b\otimes 1\;|\;b\in B\rangle$. \\
Write $I=\langle 1\otimes b-b\otimes 1\;|\;b\in B\rangle$. For any generator $1\otimes b-b\otimes 1$ of $I$, we see that $$f(1\otimes b-b\otimes 1)=0$$ Thus $I\subseteq\ker(f)$. Now suppose that $\sum_{i,j} b_i\otimes b_j\in\ker(f)$. Then using the identity $$b_i\otimes b_j=b_ib_j\otimes 1+(b_i\otimes 1)(1\otimes b_j-b_j\otimes 1)$$ and the fact that $b_ib_j=0$ (because $0=f(b_i\otimes b_j)=b_ib_j$) we see that $$\sum_{i,j} b_i\otimes b_j=\sum_{i,j}(b_i\otimes 1)(1\otimes b_j-b_j\otimes 1)$$ Since each $1\otimes b_j-b_j\otimes 1$ lies in $\ker(f)$, we conclude that $\sum_{i,j}b_i\otimes b_j$ so that $I=\ker(f)$. \\~\\
Step 2: Check that $d:B\to I/I^2$ is an $A$-derivation. \\
\begin{itemize}
\item $d:B\to I/I^2$ is an $A$-module homomorphism: Let $a_1a_2\in A$ and $b_1,b_2\in B$. Then we have 
\begin{align*}
d(a_1b_1+a_2b_2)&=1\otimes (a_1b_2+a_2b_2)-(a_1b_2+a_2b_2)\otimes 1+I^2\\
&=a_1(1\otimes b_1)+a_2(1\otimes b_2)-a_1(b_1\otimes 1)-a_2(b_2\otimes 1)+I^2\\
&=a_1d(b_1b_2)+a_2d(b_1b_2)+I^2
\end{align*}
Thus we are done. (Notice that we did not use the fact that all the expressions are taken modulo $I^2$)
\item $d$ satisfies the Leibniz rule: Let $b_1,b_2\in B$. Then we have $d(b_1b_2)=1\otimes b_1b_2-b_1b_2\otimes 1+I^2$ on one hand. On the other hand we have $$b_1d(b_2)+b_2d(b_1)=b_1(1\otimes b_2-b_2\otimes 1)+b_2(1\otimes b_1-b_1\otimes 1)+I^2$$ Subtracting them gives 
\begin{align*}
d(b_1b_2)-b_1d(b_2)-b_2d(b_1)&=1\otimes b_1b_2-b_1\otimes b_2-b_2\otimes b_1+b_2b_1\otimes 1\\
&=(1\otimes b_1-b_1\otimes 1)(1\otimes b_2-b_2\otimes 1)+I^2
\end{align*}
But $(1\otimes b_1-b_1\otimes 1)(1\otimes b_2-b_2\otimes 1)$ lies in $I^2$ thus subtraction gives $0$. 
\end{itemize}
Thus $d$ is an $A$-derivation. \\~\\

Step 3: Show that the universal property is satisfied. \\
Let $M$ be a $B$-module and $d':B\to M$ an $A$-derivation. We want to find a unique $\tilde{\phi}:B\to M$ such that $d'=\tilde{\phi}\circ d$. \\~\\

Step 3.1: Construct a homomorphism of $A$-algebra from $B\otimes B$ to $B\ltimes M$ \\
Define $\phi:B\otimes B\to B\ltimes M$ (Refer to \ref{defn:7.1.7} for definition of $B\ltimes M$) by $$\phi(b_1\otimes b_2)=(b_1b_2,b_1d'(b_2))$$ and extend it linearly so that $\phi(b_1\otimes b_2+b_3\otimes b_4)=\phi(b_1\otimes b_2)+\phi(b_3\otimes b_4)$. This is a homomorphism of $A$-algebra since 
\begin{itemize}
\item Addition is preserved: This is by definition. 
\item $\phi(ab_1\otimes b_2)=\phi(b_1\otimes ab_2)=a\phi(b_1\otimes b_2$: Let $a\in A$ and $b_1\otimes b_2\in B\otimes_A B$. Then 
\begin{align*}
\phi(ab_1\otimes b_2)&=(ab_1b_2,ab_1d'(b_2))\\
&=a\cdot\phi(b_1\otimes b_2)\\
\phi(b_1\otimes ab_2)&=(ab_1b_2,b_1d'(ab_2))\\
&=(ab_1b_2,ab_1d'(b_2))
\end{align*}
Thus we are done. 
\item Product is preserved: For $u_1,u_2,v_1,v_2\in B$, we have
\begin{align*}
\phi((u_1\otimes u_2)\cdot\phi(v_1\otimes v_2))&=(u_1u_2,u_1d'(u_2))\cdot(v_1v_2,v_1d'(v_2))\\
&=(u_1u_2v_1v_2,u_1u_2v_1d'(v_2)+v_1v_2u_1d'(u_2))\\
&=(u_1v_1u_2v_2,u_1v_1d'(u_2v_2)\\
&=\phi(u_1v_1\otimes u_2v_2)
\end{align*}
\end{itemize}
Thus $\phi$ is a homomorphism of $A$-algebra. \\~\\

Step 3.2: Construct $\tilde{\phi}$ from $\phi$. \\
Since $\phi$ is a map $B\otimes B$ to $B\ltimes M$, we can restrict this map to $I$ a result in a new map $\bar{\phi}:I\to B\ltimes M$. Notice that for $1\otimes b-b\otimes 1$ a generator of $I$, we have 
\begin{align*}
\bar{\phi}(1\otimes b-b\otimes 1)&=\bar{\phi}(1\otimes b)-\bar{\phi}(b\otimes 1)\\
&=(b,d'(b))-(b,d'(1))\\
&=(b,d'(b))-(b,0)\\
&=(0,d'(b))
\end{align*}
Thus we actually have a map $\bar{\phi}:I\to M$. Finally, notice that for $(1\otimes u-u\otimes 1)(1\otimes v-v\otimes 1)$ a generator of $I^2$, we have 
\begin{align*}
\bar{\phi}(x)&=\phi(1\otimes u-u\otimes 1)\phi(1\otimes v-v\otimes 1)\\
&=\sum(0,d'(u))(0,d'(v))\\
&=\sum(0,0)\tag{Mult. in Trivial Extension}\\
&=(0,0)
\end{align*}
which shows $\bar{\phi}$ kills of $I^2$ and thus $\bar{\phi}$ factors through $I/I^2$ so that we get a map $\tilde{\phi}:I/I^2\to M$. \\~\\

Step 3.3: Show that $\tilde{\phi}$ satisfies all the required properties. \\
For $b\in B$, we have that $$\tilde{\phi}(d(b))=\tilde{\phi}(1\otimes b-b\otimes 1+I^2)=d'(b)$$ and thus $d'=\tilde{\phi}\circ d$. Moreover, this map is unique since it is defined on the generators of $I$, namely the $d(b)$ for $b\in B$. \\~\\

This concludes the proof. \\
Materials referenced: \cite{Rav}, \cite{Ern}, \cite{Mat}
\end{proof}
\end{prp}

This version of the module of Kähler Differentials generalizes well to the theory of schemes. Interested readers are referred to \cite{Har}. \\~\\

Our first step towards computing the module of Kähler Differentials for coordinate rings comes from a computation of the polynomial ring. 

\begin{lmm}{}{} Let $A$ be a ring and $B=A[x_1,\dots,x_n]$ so that $B$ is an $A$-algebra. Then $$\Omega_{B/A}^1=\bigoplus_{i=1}^nBd(x_i)$$ 
is a finitely generated $B$-module. \tcbline
\begin{proof}
I claim that  $\Omega_{B/A}^1$ has basis $d(x_1),\dots,d(x_n)$. 
We proceed by induction. \\~\\

When $n=1$, a general polynomial in $A[x]$ is of the form $$f(x)=\sum_{i=0}^nc_ix^i$$ for $c_i\in A$. Applying $d$ subject to the conditions of quotienting gives $$d(f)=\sum_{i=0}^nc_id(x^i)$$ But $d(x^i)=xd(x^{i-1})+x^{i-1}d(x)$. Repeating this allows us to reduce $d(x^i)=g_i(x)d(x)$. Doing this for each $x^i$ in the sum in fact gives us $f(x)=\frac{df}{dx}d(x)$. Thus we see that $\Omega_{A[x]/A}^1$ is a $A[x]$ module with basis $d(x)$. \\~\\

Now suppose that $\Omega_{A[x_1,\dots,x_{n-1}]/A}^1=\bigoplus_{i=1}^{n-1}Bd(x_i)$. Then for every $f\in A[x_1,\dots,x_n]$, we can write the function as $$f(x_1,\dots,x_n)=\sum_{i=0}^sg_i(x_1,\dots,x_{n-1})x_n^i$$ and then we can apply the same process again: $$d(f)=\sum_{i=0}^s(x_n^id(g_i)+g_id(x_n^i))$$ except that now $d(g_i)$ by induction hypothesis can be written in terms of the basis $d(x_1),\dots,d(x_{n-1})$. As a side note: by doing some multiplication, one can easily see that $$d(f)=\sum_{i=0}^s\frac{\partial f}{\partial x_i}d(x_i)$$~\\

By \ref{prp:7.1.6}, since $\Omega_{B/A}^1$ is a $B$-module, there exists a free $B$ module $\bigoplus_{i=1}^mB$ such that the map $\psi:\bigoplus_{i=1}^mB$ is surjective. In fact, by choosing $m=n$ and mapping each basis $e_i$ of $\bigoplus_{i=1}^nB$ to $d(x_i)$, we obtain a surjective map. \\~\\

Now consider the map $\partial:B\to\bigoplus_{i=1}^nB$ (No calculus involved, just notation!) defined by $$f\mapsto\left(\frac{\partial f}{\partial x_1},\dots,\frac{\partial f}{\partial x_n}\right)$$ It is clear that this map is an $A$-derivation. By the universal property of $\Omega_{B/A}^1$, the derivation factors through $d:A\to\Omega_{B/A}^1$. This leaves us with a $B$-module homomorphism $\phi:\Omega_{B/A}^1\to\bigoplus_{i=1}^nB$ defined by $$d(f)\mapsto\left(\frac{\partial f}{\partial x_1},\dots,\frac{\partial f}{\partial x_n}\right)$$ This map is surjective. Notice that for each monomial $x_i$ in $B$, we have $\partial(x_i)=e_i$. Since $\partial=\phi\circ d$, $d(x_i)\in\Omega_{A/k}^1$ maps to $e_i$ and thus $\phi$ is surjective. \\~\\

It is clear that $\phi$ and $\psi$ are inverses of each other since the basis elements that they map to and from are the same. 
\end{proof}
\end{lmm}

\subsection{Transfering the System of Differentials}
This section aims to develop the necessary machinery in order to compute the module of Kähler Differentials for coordinate rings. We will see explicit calculation of the cuspidal cubic, an ellipse and the double cone to demonstrate how the two exact sequences can be used along with the Jacobian of the defining equations of the variety to compute the module of Kähler Differentials. 

\begin{thm}{First Exact Sequence}{2.3.1} Let $B,C$ be $A$-algebras and let $\phi:B\to C$ be an $A$-algebra homomorphism. Then the following sequence is an exact sequence of $C$-modules: \\~\\
\adjustbox{scale=1.0,center}{\begin{tikzcd}
\Omega_{B/A}^1\otimes_BC\arrow[r, "f"] & \Omega_{C/A}^1\arrow[r, "g"] & \Omega_{C/B}^1\arrow[r] & 0
\end{tikzcd}} \\~\\
where $f$ and $g$ is defined respectively as $$f(d_{B/A}(b)\otimes c)=c\cdot d_{C/A}(\phi(b))$$ and $$g(d_{C/A}(c))=d_{C/B}(c)$$ and extended linearly. \tcbline
\begin{proof}
Denote $d_{B/A},d_{C/A},d_{C/B}$ the derivations for $\Omega_{B/A}^1,\Omega_{C/A}^1,\Omega_{C/B}^1$ respectively. Clearly $g$ is surjective since for any $c_1d_{C/B}(c_2)\in\Omega_{C/B}^1$, just choose $c_1d_{C/A}(c_2)\in\Omega_{C/A}^1$. We just have to show that $\ker(g)=\im(f)$. It is enough to show that \\~\\
\adjustbox{scale=1.0,center}{\begin{tikzcd}
0\arrow[r] & \Hom_C(\Omega_{C/B}^1,N)\arrow[r] & \Hom_C(\Omega_{C/A}^1,N)\arrow[r] & \Hom_C(\Omega_{B/A}^1\otimes_BC,N)
\end{tikzcd}}\\~\\
is exact by \ref{thm:7.1.2}. Using the fact that $\Hom_C(\Omega_{B/A}^1\otimes_BC,N)=\Hom_B(\Omega_{B/A}^1,N)$ (\ref{thm:7.1.3}) and the fact that $\Hom(\Omega_{B/A}^1,N)\cong\text{Der}_A(B,N)$, we can transform the sequence into \\~\\
\adjustbox{scale=1.0,center}{\begin{tikzcd}
0\arrow[r] & \text{Der}_B(C,N)\arrow[r, "u"] & \text{Der}_A(C,N)\arrow[r, "v"] & \text{Der}_A(B,N)
\end{tikzcd}}\\~\\
Notice that $u$ is just the inclusion map and $v$ is just the restriction map. In particular, an $A$-derivation is a $B$-derivation if and only if its restriction to $B$ is trivial. Hence we conclude that $\im(u)=\ker(v)$. 
Materials Referenced: \cite{Liu}, \cite{CRing}
\end{proof}
\end{thm}

\begin{thm}{Second Exact Sequence}{2.3.2} Let $A$ be a ring and $B$ an $A$-algebra. Let $I$ be an ideal of $B$ and $C=B/I$. Then the following sequence is an exact sequence of $C$-modules: \\~\\
\adjustbox{scale=1.0,center}{\begin{tikzcd}
I/I^2\arrow[r] & \Omega_{B/A}^1\otimes_B C\arrow[r, "\delta"] & \Omega_{C/A}^1\arrow[r, "f"] & 0
\end{tikzcd}}\\~\\
where $\delta$ and $f$ is defined respectively as $$\delta(i+I^2)=d(i)\otimes 1$$ and $$f(d(b)\otimes c)=c\cdot d(\phi(b))$$ and then extended linearly. \tcbline
\begin{proof}
Notice that $\delta$ is well defined. Indeed, if $i+I^2=j+I^2$, then there exists $h_1,h_2\in I$ such that $i-j=h_1h_2$. Now we have that 
\begin{align*}
\delta(i-j)&=d(h_1h_2)\otimes 1\\
&=h_1d(h_2)\otimes 1+h_2d(h_1)\otimes 1\\
&=d(h_2)\otimes h_1+I+d(h_1)\otimes h_2+I\\
&=d(h_2)\otimes 0+d(h_1)\otimes 0\\
&=0
\end{align*}
We can see that $f$ is surjective. Indeed for any $d(b+I)\in\Omega_{C/A}^1$, just choose $d(b)\otimes 1\in\Omega_{B/A}^1\otimes_BC$. Then $f(d(b)\otimes 1)=d(b+I)$. \\~\\

It remains to show that $\im(\delta)=\ker(f)$. Notice that to prove the exactness of the sequence in question, we just have to show the exactness of the following sequence (by \ref{thm:7.1.2}): \\~\\
\adjustbox{scale=1.0,center}{\begin{tikzcd}
0\arrow[r] & \Hom_C(\Omega_{C/A}^1,N)\arrow[r] & \Hom_C(\Omega_{B/A}^1\otimes_B\frac{B}{I})\arrow[r] & \Hom_C(I/I^2,N)
\end{tikzcd}}\\~\\
Using the fact that $I/I^2\cong I\otimes_B\frac{B}{I}$ (by \ref{prp:7.1.4}) and $\Hom_C(\Omega_{B/A}^1\otimes_BB/I,N)=\Hom_B(\Omega_{B/A}^1,N)$ (by \ref{thm:7.1.3}) we can transform this sequence into \\~\\
\adjustbox{scale=1.0,center}{\begin{tikzcd}
0\arrow[r] & \Hom_C(\Omega_{C/A}^1,N)\arrow[r] & \Hom_B(\Omega_{B/A}^1,N)\arrow[r] & \Hom_B(I,N)
\end{tikzcd}}\\~\\
and further using $\text{Der}_A(B,N)\cong\Hom_B(\Omega_{B/A}^1,N)$ (by \ref{lmm:2.2.2}), transform into \\~\\
\adjustbox{scale=1.0,center}{\begin{tikzcd}
0\arrow[r] & \text{Der}_A(B/I,N)\arrow[r, "f_\ast"] & \text{Der}_A(B,N)\arrow[r, "\delta_\ast"] & \Hom_B(I,N)
\end{tikzcd}}\\~\\
There is no need to prove the second arrow to be injective. We need to show exactness between the second and third arrow. \\~\\

Notice that any $\phi\in\text{Der}_A(B/I,N)$ can be extended naturally to an $A$-linear derivation from $B$ to $N$: just pre-compose it with the projection map $p:B\to B/I$. This map is $A$-linear hence $\phi\circ p$ is $A$-linear. Moreover, $p$ is $B$-linear and $\phi$ is a derivation so that it satisfies the Leibniz rule. Also, a natural map from $\text{Der}_A(B,N)$ to $\Hom_B(I,N)$ is given just by restricting $\psi\in\text{Der}_A(B,N)$ to $I$. The new map under restriction will naturally become a homomorphism from $I$ to $N$. The kernel of the third arrow is just any derivation in $\text{Der}_A(B,N)$ that is identically $0$ on $I$. \\~\\
But these derivations are precisely those of $\text{Der}_A(B/I,N)$! 
\end{proof}
\end{thm}

A very nice application towards computing the module of differential forms is given by the second exact sequence. For $B=A[x_1,\dots,x_n]$ and $C=\frac{B}{I=(f_1,\dots,f_r)}$, we can use \ref{prp:7.1.5} to see that $\Omega_{B/A}^1\otimes C\cong\bigoplus_{i=1}^nCdx_i$. By the second exact sequence \ref{thm:2.3.2}, we see that $$\Omega_{C/A}^1\cong\text{coker}\left(\frac{I}{I^2}\to\bigoplus_{i=1}^nCdx_i\right)$$ Since $I/I^2$ is a $C$-module, by \ref{prp:7.1.6} there exists a surjective map $\bigoplus_{i=1}^mCde_i\twoheadrightarrow I/I^2$. In fact $m=r$ since $I$ is finitely generated by $f_1,\dots,f_r$ and thus the map sends $e_i$ to $f_i$ for $1\leq i\leq r$. \\~\\

Now consider the map $$J:\bigoplus_{i=1}^rCde_i\twoheadrightarrow\frac{I}{I^2}\rightarrow\bigoplus_{i=1}^nCdx_i$$ This is a map from a free module of rank $r$ to a free module of rank $n$. So we can write this in an $n\times r$ matrix. Since the map $I/I^2\to\bigoplus_{i=1}^nCdx_i$ sends $f_i$ to $d(f_i)=\sum_{k=1}^n\frac{\partial f_i}{\partial x_k}dx_k$ (by second exact sequence \ref{thm:2.3.2}) and $e_i$ is sent $f_i$, we have that $J$ is the matrix $$\begin{pmatrix}
\frac{\partial f_1}{\partial x_1} & \cdots & \frac{\partial f_r}{\partial x_1}\\
\vdots & \ddots & \vdots\\
\frac{\partial f_1}{\partial x_n} & \cdots & \frac{\partial f_r}{\partial x_n}
\end{pmatrix}$$~\\

Finally, since $\im(A\twoheadrightarrow B\to C)=\im(B\to C)$, we thus have $$\text{coker}(J)\cong\Omega_{C/A}^1$$ which means that $\Omega_{C/A}^1$ is just the cokernel of the matrix. This exposition can be found in \cite{Dav}. 

\subsection{Characterization for Separability}
The module of Kähler differentials give a necessary and sufficient condition for a finite extension to be separable. Before the main proposition, we will need a lemma. 

\begin{lmm}{}{3.1.1} Let $L/K$ be a finite field extension and $\Omega_{L/K}^1$ the module of Kähler Differentials. Let $f(b)=c_0+c_1b+\dots+c_nb^n\in L$ for $c_0,\dots,c_n\in K$ and $b\in L$. Then $d(f(b))=f'(b)d(b)$ where $f'(b)$ is the derivative of $f(b)$ with respect to $b$ in the sense of calculus. \tcbline
\begin{proof}
Since $f(b)$ is a finite sum, we apply linearity and Leibniz rule of $d$ to get $$f'(b)=d(c_0)+bd(c_1)+c_1d(b)+\dots+b^nd(c_n)+c_nd(b^n)$$ Since each $c_0,\dots,c_n\in K$, we obtain $f'(b)=c_1d(b)+\dots+c_n\cdot nb^{n-1}d(b)$. Thus factoring out $d(b)$ in the sum, we obtain precisely the standard derivative in calculus, and that $d(f(b))=f'(b)d(b)$
\end{proof}
\end{lmm}

\begin{prp}{}{3.1.2} Let $K$ be a field and $L/K$ a finite field extension. Then $L/K$ is separable if and only if $\Omega_{L/K}^1=0$. \tcbline
\begin{proof}
Suppose that $L/K$ is separable. Suppose that $b\in L$ has minimal polynomial $f\in K[x]$. $f$ is separable since $L/K$ is separable. By \ref{lmm:3.1.1}, we have that $d(f(b))=f'(b)d(b)$. But the fact that $f$ is separable implies that $f'(b)\neq 0$. At the same time we have $f(b)=0$ since $f$ is the minimal polynomial of $b$. This implies that $d(f(b))=0$ in $\Omega_{L/K}^1=0$. Since $L$ is a field, and $f'(b)\neq 0$, we must have $d(b)=0$ for all $b\in L$. This means that $\Omega_{L/K}^1=0$. \\~\\

If $L/K$ is inseparable, then there exists an intermediate field $E$ such that $L/E$ is a simple inseparable extension. Since $L/K$ is finite, $L/E$ is finite and thus is algebraic which means that there exists some polynomial $p\in E[t]$ for which $L=\frac{E[t]}{(p(t))}$. In this case, we have already seen that $$\Omega_{L/E}^1\cong\frac{Ldt}{(p'(t)dt)}\cong\frac{L}{(p'(t))}$$ Since $p'(t)=0$, we have that $\Omega_{L/E}^1\cong L\neq 0$. By the first exact sequence \ref{thm:2.3.1}, we have that $\Omega_{L/K}^1$ maps surjectively onto $\Omega_{L/E}^1\neq 0$ which proves that $\Omega_{L/K}^1$ is non-zero. \\
Materials referenced: \cite{Per}, \cite{Liu}
\end{proof}
\end{prp}

This gives a very nice characterization of separability. Readers can find more in \cite{Har} and \cite{Liu}. To extend this equivalence under the assumption that $L/K$ is algebraic instead of finite, one can show that $\Omega^1$ preserves colimits in the sense in $\cite{Dav}$. Namely that the functor $F:\text{Algebra}_R\to\text{Mod}_T$ from the category of $R$-algebra to the category of $T$-modules where $T$ is a colimit of a diagram in the category of $R$-algerba preserves colimits. Then observe that an algebraic extension is the colimit of the finite subextensions. \\~\\

Analogous to the above result, there is a similar proposition for $\text{Der}_K(L)$ for when $L/K$ is algebraic and separable. This is given by \cite{Pat}. 

\begin{prp}{}{3.1.3} Let $L/K$ be an algebraic field extension that is separable. Then $\text{Der}_K(L)=0$. \tcbline
\begin{proof}
Suppose that $D\in\text{Der}_K(L)$. If $a\in L$, let $p$ be the minimal polynomial of $a$. Then $$0=D(p(a))=p'(a)D(a)$$ by \ref{lmm:3.1.1}. Since $p$ is separable over $K$, $p'(a)\neq 0$. Thus $D(a)=0$ and so we are done. \\
Materials referenced: \cite{Pat}
\end{proof}
\end{prp}

This proposition will be of use at \ref{eg:4.1.7}. 

\subsection{Detecting Smoothness in Varieties}
We can recover the cotangent space from the cotangent bundle. Recall that by defining $m_p=\{f\in\C[V]|f(p)=0\}$ for a variety $V$, we have that $m_p/m_p^2$ is the cotangent space of $V$ from \cite{Sha}. \\~\\
Combined with the following theorem, we see that by localization, we can see that we recover the cotangent space, at least in the affine, non-scheme theoretic sense: 

\begin{thm}{}{3.2.1} Let $B$ be a local ring which contains a field $K$ that is isomorphic to $B/m$ the residue field. Then the map $$\delta:\frac{m}{m^2}\to\Omega_{B/K}^1\otimes_B K$$ is an isomorphism. \tcbline
\begin{proof}
Using the second exact sequence \ref{thm:2.3.2}, we have that \\~\\
\adjustbox{scale=1.0,center}{\begin{tikzcd}
m/m^2\arrow[r, "\delta"] & \Omega_{B/K}^1\otimes_B \frac{B}{m}\arrow[r] & \Omega_{(B/m)/K}^1\arrow[r] & 0
\end{tikzcd}}\\~\\
But the third term is just $\Omega_{K/K}^1$ which is clearly just $0$. Thus $\delta$ is surjective. Now $m/m^2$ and $\Omega_{B/K}^1\otimes_B K$ are both modules over $B/m\cong K$ and thus are vector spaces over $K$. To show injectivity of $\delta$ is the same as to show surjectivity of the dual map $$\delta^\ast:\Hom_K(\Omega_{B/K}^1\otimes K,N)\to\Hom_K\left(\frac{m}{m^2},N\right)$$ for arbitrary $B$-module $N$. By \ref{thm:7.1.3}, we have that $\Hom_K(\Omega_{B/K}^1\otimes K,N)\cong\Hom_B(\Omega_{B/K}^1,N)$. By \ref{lmm:2.2.2}, this is isomorphic to $\text{Der}_K(B,N)$. Our new map becomes $\delta^\ast:\text{Der}_K(B,N)\to\Hom_B(m/m^2,N)$. \\~\\

Let $b\in B$. I claim that $b$ is a unique sum of an element in $m$ and an element in $B/m$. Suppose that $b=c_1+m_1=c_2+m_2$ for $c_1,c_2\in K$ and $m_1,m_2\in m$. Then this implies that $c_1-c_2\in m$ is a non-unit. But $c_1-c_2\in K$ does not have an inverse if and only if $c_1-c_2=0$ thus $c_1=c_2$. This leaves $m_1=m_2$. \\~\\

I claim that the map is surjective as follows. For $h\in\Hom_B(m/m^2,N)$, define $d\in\text{Der}_K(B,N)$ by sending $d(b)=d(c+n)=h(n)$ where $c+n$ is the unique representation of $b$ using $c\in R/m$ and $n\in m$. \\~\\

For $b_1,b_2\in B$, we have that 
\begin{align*}
d(b_1b_2)&=h((c_1m_2+c_2m_1+m_1m_2)+m^2)\tag{Write $b_i=c_i+m_i$ where $c_i\in B/m$ and $k_i\in m$}\\
&=h((c_1m_2+c_2m_1)+m^2)\\
&=c_1h(m_2+m^2)+c_2h(m_1+m^2)\\
&=c_1h(b_2)+c_2h(b_1)
\end{align*} and 
\begin{align*}
b_1d(b_2)+b_2d(b_1)&=(c_1+m_1)h(m_2+m^2)+(c_2+m_2)h(m_1+m^2)\\
&=c_1h(b_2)+c_2h(b_1)
\end{align*}
where the second equality follows from the fact that $h$ is a $B/m$ linear map ($c_i+m_i+m=c_i+m$ in $B/m$). Thus $d$ is a derivation. \\~\\

We can conclude that $\delta^\ast$ is surjective so that we are done. 
Materials Referenced: \cite{Har}, \cite{CRing}
\end{proof}
\end{thm}

We are almost ready in recovering the cotangent space. By considering the localization of a coordinate ring $\C[V]$ with a maximal ideal $m_p$ corresponding to points on the variety, we obtain a local ring $\C[V]_{m_p}$ with maximal ideal again $m_p$. Then the cotangent space $\frac{m_p}{m_p^2}$ as seen in \cite{Sha}, is isomorphic to $\Omega_{\C[V]_{m_p}/\C}^1\otimes_{\C[V]_{m_p}} \C$ by the above theorem. Therefore what remains is to compute the module of Kähler differentials for the localization of a coordinate ring. \\~\\

Fortunately localization commutes with the construction of the module of Kähler differentials: 

\begin{prp}{}{3.2.2} Let $B$ be an algebra over $A$. Let $S$ be a multiplicative subset of $B$. Then $$S^{-1}\Omega_{B/A}^1\cong\Omega_{S^{-1}B/A}^1$$ \tcbline
\begin{proof}
This is done in two steps. \\~\\
Step 1: $\Omega_{S^{-1}B/B}^1=0$. \\
We have that for any $u\in S^{-1}B$, there exists some $s\in S$ such that $su\in B$. Applying the canonical derivation gives 
\begin{align*}
sd(u)&=d(su)\tag{$s\in S\subset B$}\\
&=0\tag{$su\in B$}
\end{align*}
Since $s\in S$ is invertible, we must have $d(u)=0$. Thus $\Omega_{S^{-1}B/B}^1=0$. \\~\\

Step 2: Apply the first exact sequence. \\
By the first exact sequence \ref{thm:2.3.1} and apply it to $C=S^{-1}B$, we obtain a surjective map $$\Omega_{B/A}^1\otimes_BS^{-1}B\to\Omega_{S^{-1}B/A}^1$$  which by definition of localization of modules, is equal to $$S^{-1}\Omega_{B/A}^1\to\Omega_{S^{-1}B/A}^1$$ In order to show injectivity of this map, we show that $$\Hom_{S^{-1}B}(\Omega_{S^{-1}B/A}^1,N)\to\Hom_{S^{-1}B}(S^{-1}\Omega_{B/A}^1,N)$$ is surjective for any $S^{-1}B$-module $N$. Now the latter module is isomorphic to $\Hom_B(\Omega_{B/A}^1,N)$ by \ref{thm:7.1.3} Using \ref{lmm:2.2.2}, this is equivalent to showing surjectivity of the map $$\text{Der}_A(S^{-1}B,N)\to\text{Der}_A(B,N)$$ But this is precisely the content of \ref{prp:2.1.4}. So we are done. \\
Materials referenced: \cite{Liu}
\end{proof}
\end{prp}

\pagebreak
\section{Second Course on Algebra}
\subsection{Smooth Algebras}
\begin{defn}{Smooth Algebra}{} Let $A$ be a commutative algebra over a field $k$. We say that $A$ is a smooth algebra over $k$ if $\Omega_{A/k}^1$ is a projective $A$-module. 
\end{defn}

\begin{defn}{Formal Smoothness}{} Let $A$ be a commutative algebra over a field $k$. We say that $A$ is a formally smooth if for every $k$-algebra $C$ together with a $k$-algebra homomorphism $u:A\to C/N$ where $N^2=0$, there exists a $k$-algebra map $v:A\to C$ such that the following diagram is commutative:  \\~\\
\adjustbox{scale=1.0,center}{\begin{tikzcd}
	A & C \\
	& {\frac{C}{N}}
	\arrow["{\exists v}", dashed, from=1-1, to=1-2]
	\arrow["u"', from=1-1, to=2-2]
	\arrow["p", from=1-2, to=2-2]
\end{tikzcd}}\\~\\
We say that $A$ is etale over $k$ if such a map $v:A\to C$ is unique. 
\end{defn}


\end{document}