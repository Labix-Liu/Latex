\documentclass[a4paper]{article}

\input{C:/Users/liula/Desktop/Latex/Headers V1.2.tex}

\pagestyle{fancy}
\fancyhf{}
\rhead{Labix}
\lhead{Differential Graded Algebra}
\rfoot{\thepage}

\title{Differential Graded Algebra}

\author{Labix}

\date{\today}
\begin{document}
\maketitle
\begin{abstract}
\end{abstract}
\pagebreak
\tableofcontents

\pagebreak

\section{Differential Graded Algebra}
\subsection{Basic Definitions}
Similar to how chain complexes and cochain complexes are two names of the same object, we can define differential graded algebra using either the chain complex notation or cochain complex notation. For our purposes, we will use the cochain version. This means that differentials will go up in index. \\~\\

A differential graded algebra equips a graded algebra with a differential so that the algebra in the grading form a cochain complex. 

\begin{defn}{Differential Graded Algebra}{} A differential graded algebra is a graded algebra $A_\bullet$ together with a map $d:A\to A$ that has degree $1$ such that the following are true. 
\begin{itemize}
\item $d\circ d=0$
\item For $a\in A_n$ and $b\in A_m$, we have $d(ab)=(da)b+(-1)^na(db)$
\end{itemize}
\end{defn}

\begin{lmm}{}{} Let $(A,d)$ be a differential graded algebra. Then $(A,d)$ is also a cochain complex. 
\end{lmm}

Recall that a graded commutative algebra $A$ is a collection of algebra over some ring $A_0$, graded in $\N$ together with a multiplication $A_n\times A_m\to A_{m+n}$ such that $$a\cdot b=(-1)^{nm}b\cdot a$$ Such a multiplication rule is said to be graded commutative. 

\begin{defn}{Commutative Differential Graded Algebra}{} A differential graded algebra $A$ is said to be a commutative differential graded algebra (CDGA) if $A$ is also graded commutative. 
\end{defn}

We will often be concerned of differential graded algebra over a field $\Q$, $\R$ or $\C$. In particular this means that the algebra has the structure of a vector space. 







\end{document}