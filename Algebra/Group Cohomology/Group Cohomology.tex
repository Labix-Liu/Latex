\documentclass[a4paper]{article}

%=========================================
% Packages
%=========================================
\usepackage{mathtools}
\usepackage{amsfonts}
\usepackage{amsmath}
\usepackage{amssymb}
\usepackage{amsthm}
\usepackage[a4paper, total={6in, 8in}, margin=1in]{geometry}
\usepackage[utf8]{inputenc}
\usepackage{fancyhdr}
\usepackage[utf8]{inputenc}
\usepackage{graphicx}
\usepackage{physics}
\usepackage[listings]{tcolorbox}
\usepackage{hyperref}
\usepackage{tikz-cd}
\usepackage{adjustbox}
\usepackage{enumitem}
\usepackage[font=small,labelfont=bf]{caption}
\usepackage{subcaption}
\usepackage{wrapfig}
\usepackage{makecell}



\raggedright

\usetikzlibrary{arrows.meta}

\DeclarePairedDelimiter\ceil{\lceil}{\rceil}
\DeclarePairedDelimiter\floor{\lfloor}{\rfloor}

%=========================================
% Fonts
%=========================================
\usepackage{tgpagella}
\usepackage[T1]{fontenc}


%=========================================
% Custom Math Operators
%=========================================
\DeclareMathOperator{\adj}{adj}
\DeclareMathOperator{\im}{im}
\DeclareMathOperator{\nullity}{nullity}
\DeclareMathOperator{\sign}{sign}
\DeclareMathOperator{\dom}{dom}
\DeclareMathOperator{\lcm}{lcm}
\DeclareMathOperator{\ran}{ran}
\DeclareMathOperator{\ext}{Ext}
\DeclareMathOperator{\dist}{dist}
\DeclareMathOperator{\diam}{diam}
\DeclareMathOperator{\aut}{Aut}
\DeclareMathOperator{\inn}{Inn}
\DeclareMathOperator{\syl}{Syl}
\DeclareMathOperator{\edo}{End}
\DeclareMathOperator{\cov}{Cov}
\DeclareMathOperator{\vari}{Var}
\DeclareMathOperator{\cha}{char}
\DeclareMathOperator{\Span}{span}
\DeclareMathOperator{\ord}{ord}
\DeclareMathOperator{\res}{res}
\DeclareMathOperator{\Hom}{Hom}
\DeclareMathOperator{\Mor}{Mor}
\DeclareMathOperator{\coker}{coker}
\DeclareMathOperator{\Obj}{Obj}
\DeclareMathOperator{\id}{id}
\DeclareMathOperator{\GL}{GL}
\DeclareMathOperator*{\colim}{colim}

%=========================================
% Custom Commands (Shortcuts)
%=========================================
\newcommand{\CP}{\mathbb{CP}}
\newcommand{\GG}{\mathbb{G}}
\newcommand{\F}{\mathbb{F}}
\newcommand{\N}{\mathbb{N}}
\newcommand{\Q}{\mathbb{Q}}
\newcommand{\R}{\mathbb{R}}
\newcommand{\C}{\mathbb{C}}
\newcommand{\E}{\mathbb{E}}
\newcommand{\Prj}{\mathbb{P}}
\newcommand{\RP}{\mathbb{RP}}
\newcommand{\T}{\mathbb{T}}
\newcommand{\Z}{\mathbb{Z}}
\newcommand{\A}{\mathbb{A}}
\renewcommand{\H}{\mathbb{H}}
\newcommand{\K}{\mathbb{K}}

\newcommand{\mA}{\mathcal{A}}
\newcommand{\mB}{\mathcal{B}}
\newcommand{\mC}{\mathcal{C}}
\newcommand{\mD}{\mathcal{D}}
\newcommand{\mE}{\mathcal{E}}
\newcommand{\mF}{\mathcal{F}}
\newcommand{\mG}{\mathcal{G}}
\newcommand{\mH}{\mathcal{H}}
\newcommand{\mI}{\mathcal{I}}
\newcommand{\mJ}{\mathcal{J}}
\newcommand{\mK}{\mathcal{K}}
\newcommand{\mL}{\mathcal{L}}
\newcommand{\mM}{\mathcal{M}}
\newcommand{\mO}{\mathcal{O}}
\newcommand{\mP}{\mathcal{P}}
\newcommand{\mS}{\mathcal{S}}
\newcommand{\mT}{\mathcal{T}}
\newcommand{\mV}{\mathcal{V}}
\newcommand{\mW}{\mathcal{W}}

%=========================================
% Colours!!!
%=========================================
\definecolor{LightBlue}{HTML}{2D64A6}
\definecolor{ForestGreen}{HTML}{4BA150}
\definecolor{DarkBlue}{HTML}{000080}
\definecolor{LightPurple}{HTML}{cc99ff}
\definecolor{LightOrange}{HTML}{ffc34d}
\definecolor{Buff}{HTML}{DDAE7E}
\definecolor{Sunset}{HTML}{F2C57C}
\definecolor{Wenge}{HTML}{584B53}
\definecolor{Coolgray}{HTML}{9098CB}
\definecolor{Lavender}{HTML}{D6E3F8}
\definecolor{Glaucous}{HTML}{828BC4}
\definecolor{Mauve}{HTML}{C7A8F0}
\definecolor{Darkred}{HTML}{880808}
\definecolor{Beaver}{HTML}{9A8873}
\definecolor{UltraViolet}{HTML}{52489C}



%=========================================
% Theorem Environment
%=========================================
\tcbuselibrary{listings, theorems, breakable, skins}

\newtcbtheorem[number within = subsection]{thm}{Theorem}%
{	colback=Buff!3, 
	colframe=Buff, 
	fonttitle=\bfseries, 
	breakable, 
	enhanced jigsaw, 
	halign=left
}{thm}

\newtcbtheorem[number within=subsection, use counter from=thm]{defn}{Definition}%
{  colback=cyan!1,
    colframe=cyan!50!black,
	fonttitle=\bfseries, breakable, 
	enhanced jigsaw, 
	halign=left
}{defn}

\newtcbtheorem[number within=subsection, use counter from=thm]{axm}{Axiom}%
{	colback=red!5, 
	colframe=Darkred, 
	fonttitle=\bfseries, 
	breakable, 
	enhanced jigsaw, 
	halign=left
}{axm}

\newtcbtheorem[number within=subsection, use counter from=thm]{prp}{Proposition}%
{	colback=LightBlue!3, 
	colframe=Glaucous, 
	fonttitle=\bfseries, 
	breakable, 
	enhanced jigsaw, 
	halign=left
}{prp}

\newtcbtheorem[number within=subsection, use counter from=thm]{lmm}{Lemma}%
{	colback=LightBlue!3, 
	colframe=LightBlue!60, 
	fonttitle=\bfseries, 
	breakable, 
	enhanced jigsaw, 
	halign=left
}{lmm}

\newtcbtheorem[number within=subsection, use counter from=thm]{crl}{Corollary}%
{	colback=LightBlue!3, 
	colframe=LightBlue!60, 
	fonttitle=\bfseries, 
	breakable, 
	enhanced jigsaw, 
	halign=left
}{crl}

\newtcbtheorem[number within=subsection, use counter from=thm]{eg}{Example}%
{	colback=Beaver!5, 
	colframe=Beaver, 
	fonttitle=\bfseries, 
	breakable, 
	enhanced jigsaw, 
	halign=left
}{eg}

\newtcbtheorem[number within=subsection, use counter from=thm]{ex}{Exercise}%
{	colback=Beaver!5, 
	colframe=Beaver, 
	fonttitle=\bfseries, 
	breakable, 
	enhanced jigsaw, 
	halign=left
}{ex}

\newtcbtheorem[number within=subsection, use counter from=thm]{alg}{Algorithm}%
{	colback=UltraViolet!5, 
	colframe=UltraViolet, 
	fonttitle=\bfseries, 
	breakable, 
	enhanced jigsaw, 
	halign=left
}{alg}




%=========================================
% Hyperlinks
%=========================================
\hypersetup{
    colorlinks=true, %set true if you want colored links
    linktoc=all,     %set to all if you want both sections and subsections linked
    linkcolor=DarkBlue,  %choose some color if you want links to stand out
}


\pagestyle{fancy}
\fancyhf{}
\rhead{Labix}
\lhead{Group Cohomology}
\rfoot{\thepage}

\title{Group Cohomology}

\author{Labix}

\date{\today}
\begin{document}
\maketitle
\begin{abstract}
\end{abstract}
\pagebreak
\tableofcontents

\pagebreak
\section{Introduction to Group Homology and Cohomology}
\subsection{G-Modules}
\begin{defn}{G-Modules}{} Let $G$ be a group. A $G$-module is an abelian group $A$ together with a group action of $G$ on $A$. 
\end{defn}

\begin{defn}{Morphisms of G-Modules}{} Let $G$ be a group. Let $M$ and $N$ be $G$-modules. A function $f:M\to N$ is said to be a $G$-module homomorphism if it is an equivariant group homomorphism. This means that $$f(g\cdot m)=g\cdot f(m)$$ for all $m\in M$ and $g\in G$. 
\end{defn}

\subsection{Invariants and Coinvariants}
\begin{defn}{The Group of Invariants}{} Let $G$ be a group and let $M$ be a $G$-module. Define the group of invariants of $G$ in $M$ to be the subgroup $$M^G=\{m\in M\;|\;gm=m\text{ for all }g\in G\}$$
\end{defn}

This is the largest subgroup of $M$ for which $G$ acts trivially. 

\begin{defn}{Functor of Invariants}{} Let $G$ be a group. Define the functor of invariants by $$(-)^G:{_G}\bold{Mod}\to\bold{Ab}$$ as follows. 
\begin{itemize}
\item For each $G$-module $M$, $M^G$ is the group of invariants
\item For each morphism $f:M\to N$ of $G$-modules, $f^G:M^G\to N^G$ is the restriction of $f$ to $M^G$. 
\end{itemize}
\end{defn}

\begin{thm}{}{} Let $G$ be a group. The functor of invariants $(-)^G:{_G}\bold{Mod}\to\bold{Ab}$ is left exact. 
\end{thm}

\begin{defn}{The Group of Coinvariants}{} Let $G$ be a group and let $M$ be a $G$-module. Define the group of coinvariants of $G$ in $M$ to be the quotient group $$M_G=\frac{M}{\langle gm-m\;|\;g\in G, m\in M\rangle}$$
\end{defn}

This is the largest quotient of $M$ for which $G$ acts trivially. 

\subsection{Group Cohomology and its Equivalent Forms}
\begin{defn}{The nth Cohomology Group}{} Let $G$ be a group. Define the $n$th cohomology group of $G$ with coefficients in a $G$-module $M$ to be $$H_n(G;M)=(L_n(-)_G)(M)$$ the $n$th left derived functor of $(-)_G:{_G\bold{Mod}}\to\bold{Ab}$. 
\end{defn}

\begin{thm}{}{} Let $G$ be a group and let $M$ be a $G$-module. Then there is an isomorphism $$H^n(G;M)\cong\text{Ext}^n_{\Z[G]}(\Z,M)$$ that is natural in $M$. 
\end{thm}

Recall that there are two descriptions of $\text{Ext}$ by considering it as a functor of the first or second variable. Since the above theorem exhibits an isomorphism that is natural in the second variable, let us consider $\text{Ext}$ as the right derived functor of the functor $\Hom_{\Z[G]}(-,M)$ applied to $\Z$ as a $\Z[G]$-module. 

\begin{prp}{}{} Let $G$ be a group and let $M$ be a $G$-module. Let $P_\bullet\to\Z$ be a projective resolution of $\Z$ with $\Z[G]$-modules. Then there is an isomorphism $$H^n(G;M)\cong H^n(\Hom_{\Z[G]}(P_\bullet,M))$$ that is natural in $M$. 
\end{prp}

For any group $G$, there is always the trivial choice of projective resolution. 
In the following lemma, we use the notation $(g_0,\dots,\hat{g_i},\dots,g_n)$ as a shorthand for writing the element in $G^n$ but with the $i$th term omitted. 

\begin{lmm}{}{} Let $G$ be a group. Then the cochain complex \\~\\
\adjustbox{scale=1,center}{\begin{tikzcd}
	\cdots & {\Z[G^{n+1}]} & {\Z[G^n]} & {\Z[G^{n-1}]} & \cdots & {\Z[G]} & \Z & 0
	\arrow[from=1-1, to=1-2]
	\arrow["{f_n}", from=1-2, to=1-3]
	\arrow["{f_{n-1}}", from=1-3, to=1-4]
	\arrow[from=1-4, to=1-5]
	\arrow[from=1-5, to=1-6]
	\arrow[from=1-6, to=1-7]
	\arrow[from=1-7, to=1-8]
\end{tikzcd}}\\~\\ where $f_n:\Z[G^{n+1}]\to\Z[G^n]$ is defined by $$(g_0,\dots,g_n)\mapsto\sum_{i=0}^n(-1)^i(g_0,\dots,\hat{g_i},\dots,g_n)$$ is a projective resolution of $\Z$ lying in ${_{\Z[G]}}\bold{Mod}$. 
\end{lmm}

Let $A$ be an $R$-algebra and let $M$ be an $A$-module. Recall that the bar resolution is defined to be the chain complex consisting of $M\otimes A^{\otimes n}$ for each $n\in\N$ together with the boundary maps defined by multiplying the $i$the element to the $i+1$th element. Now let $G$ be a group. By considering $\Z[G]$ as a $\Z$-algebra and that and ring is a module over itself, it makes sense to talk about the bar resolution of $\Z[G]$. 

\begin{thm}{}{} Let $G$ be a group. Consider the bar resolution \\~\\
\adjustbox{scale=1.0,center}{\begin{tikzcd}
	\cdots & {\Z[G^{n+1}]} & {\Z[G^n]} & {\Z[G^{n-1}]} & \cdots & \Z[G] & \Z & 0
	\arrow[from=1-1, to=1-2]
	\arrow[from=1-2, to=1-3]
	\arrow[from=1-3, to=1-4]
	\arrow[from=1-4, to=1-5]
	\arrow[from=1-5, to=1-6]
	\arrow[from=1-6, to=1-7]
	\arrow[from=1-7, to=1-8]
\end{tikzcd}}\\~\\
of $\Z[G]$. Then it is a free resolution, and hence a projective resolution of $\Z$ with $\Z[G]$-modules. 
\end{thm}

Thus, given a group $G$ and a $G$-module $M$, the group cohomology of $G$ with coefficients in $M$ can be thought of in the following way: 
\begin{itemize}
\item It is the right derived functor of the functor of invariants $(-)^G:{_G\bold{Mod}}\to\bold{Ab}$
\item It is the extension group $\text{Ext}_{\Z[G]}^n(\Z,M)$ (which is computable by the obvious projective resolution $\Z[G^\bullet]$, or the bar resolution)
\end{itemize}

\subsection{Group Homology and its Equivalent Forms}
\begin{defn}{The nth Cohomology Group}{} Let $G$ be a group. Define the $n$th cohomology group of $G$ with coefficients in a $G$-module $M$ to be $$H^n(G;M)=(R^n(-)^G)(M)$$ the $n$th right derived functor of $(-)^G:{_G\bold{Mod}}\to\bold{Ab}$. 
\end{defn}

\begin{thm}{}{} Let $G$ be a group and let $M$ be a $G$-module. Then there is an isomorphism $$H_n(G;M)\cong\text{Tor}_n^{\Z[G]}(\Z,M)$$ that is natural in $M$. 
\end{thm}

\subsection{Low Degree Interpretations}
\begin{thm}{}{} Let $G$ be a group and let $M$ be a $G$-module. Then there are natural isomorphisms $$H^0(G,M)=M^G\;\;\;\;\text{ and }\;\;\;H_0(G;M)=M_G$$
\end{thm}

\begin{thm}{}{} Let $G$ be a group and let $M$ be a $G$-module. Then there is an isomorphism $$H_1(G,M)\cong\frac{G}{[G,G]}=G_\text{ab}$$
\end{thm}

\begin{thm}{}{} Let $G$ be a group and let $M$ be a trivial $G$-module. Then there is a natural isomorphism $$H^1(G;M)=\frac{(\{f:G\to M\;|\;f(ab)=f(a)+af(b)\},+)}{\langle f:G\to M\;|\;f(g)=gm-m\text{ for some fixed }m\rangle}$$
\end{thm}

\begin{crl}{}{} Let $G$ be a group and let $M$ be a trivial $G$-module. Then there is a natural isomorphism $$H^1(G;M)\cong\Hom_\bold{Grp}(G,M)$$
\end{crl}









\end{document}