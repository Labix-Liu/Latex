\documentclass[a4paper]{article}

\input{C:/Users/liula/Desktop/Latex/Headers V1.2.tex}

\pagestyle{fancy}
\fancyhf{}
\rhead{Labix}
\lhead{Group Cohomology}
\rfoot{\thepage}

\title{Group Cohomology}

\author{Labix}

\date{\today}
\begin{document}
\maketitle
\begin{abstract}
\end{abstract}
\pagebreak
\tableofcontents

\pagebreak
\section{Group Cohomology}
\subsection{G-Modules}
\begin{defn}{G-Modules}{} Let $G$ be a group. A $G$-module is an abelian group $A$ together with a group action of $G$ on $A$. 
\end{defn}

\begin{defn}{Morphisms of G-Modules}{} Let $G$ be a group. Let $M$ and $N$ be $G$-modules. A function $f:M\to N$ is said to be a $G$-module homomorphism if it is an equivariant group homomorphism. This means that $$f(g\cdot m)=g\cdot f(m)$$ for all $m\in M$ and $g\in G$. 
\end{defn}

\subsection{The Group of Invariants}
\begin{defn}{The Group of Invariants}{} Let $G$ be a group and let $M$ be a $G$-module. Define the group of invariants of $G$ in $M$ to be the subgroup $$M^G=\{m\in M\;|\;gm=m\text{ for all }g\in G\}$$
\end{defn}

This is the largest subgroup of $M$ for which $G$ acts trivially. 

\begin{thm}{}{} Let $G$ be a group and let $M$ be a $G$-module. Then there are canonical isomorphisms $$M^G\cong\Z\otimes_{\Z[G]}M\cong\Hom_{\Z[G]}(\Z,M)$$
\end{thm}

\begin{defn}{Functor of Invariants}{} Let $G$ be a group. Define the functor of invariants by $$(-)^G:{_G}\bold{Mod}\to\bold{Ab}$$ as follows. 
\begin{itemize}
\item For each $G$-module $M$, $M^G$ is the group of invariants
\item For each morphism $f:M\to N$ of $G$-modules, $f^G:M^G\to N^G$ is the restriction of $f$ to $M^G$. 
\end{itemize}
\end{defn}

\begin{thm}{}{} Let $G$ be a group. The functor of invariants $(-)^G:{_G}\bold{Mod}\to\bold{Ab}$ is left exact. 
\end{thm}

\subsection{The Different Forms of Cohomology of Groups}
\begin{defn}{The nth Cohomology Group}{} Let $G$ be a group. Define the $n$th cohomology group of $G$ with coefficients in a $G$-module $M$ to be $$H^n(G;M)=(R^n(-)^G)(M)$$ the $n$th right derived functor of $(-)^G$. 
\end{defn}

\begin{thm}{}{} Let $G$ be a group and let $M$ be a $G$-module. Then there are natural isomorphisms $$H^n(G;M)\cong\text{Ext}_{\Z[G]}^n(\Z,M)$$
\end{thm}

In the following theorem, we use the notation $(g_0,\dots,\hat{g_i},\dots,g_n)$ as a shorthand for writing the element in $G^n$ but with the $i$th term omitted. 

\begin{thm}{}{} Let $G$ be a group. Then the cochain complex \\~\\
\adjustbox{scale=1,center}{\begin{tikzcd}
	\cdots & {\Z[G^{n+1}]} & {\Z[G^n]} & {\Z[G^{n-1}]} & \cdots & {\Z[G]} & \Z & 0
	\arrow[from=1-1, to=1-2]
	\arrow["{f_n}", from=1-2, to=1-3]
	\arrow["{f_{n-1}}", from=1-3, to=1-4]
	\arrow[from=1-4, to=1-5]
	\arrow[from=1-5, to=1-6]
	\arrow[from=1-6, to=1-7]
\end{tikzcd}}\\~\\ where $f_n:\Z[G^{n+1}]\to\Z[G^n]$ is defined by $$(g_0,\dots,g_n)\mapsto\sum_{i=0}^n(-1)^i(g_0,\dots,\hat{g_i},\dots,g_n)$$ is a projective resolution of $\Z$ lying in ${_\Z[G]}\bold{Mod}$. 
\end{thm}

\begin{crl}{}{} Let $G$ be a group and let $M$ be a $G$-module. Then there is a natural isomorphism $$H^n(G;M)\cong H^n(\Hom_G(\Z[G^\bullet],M))$$
\end{crl}

\pagebreak
\section{Group Homology}
\begin{defn}{The Group of Coinvariants}{} Let $G$ be a group and let $M$ be a $G$-module. Define the group of coinvariants of $G$ in $M$ to be the quotient group $$M_G=\frac{M}{\langle gm-m\;|\;g\in G, m\in M\rangle}$$
\end{defn}

This is the largest quotient of $M$ for which $G$ acts trivially. 








\end{document}