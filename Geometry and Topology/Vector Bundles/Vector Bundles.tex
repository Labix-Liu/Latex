\documentclass[a4paper]{article}

%=========================================
% Packages
%=========================================
\usepackage{mathtools}
\usepackage{amsfonts}
\usepackage{amsmath}
\usepackage{amssymb}
\usepackage{amsthm}
\usepackage[a4paper, total={6in, 8in}, margin=1in]{geometry}
\usepackage[utf8]{inputenc}
\usepackage{fancyhdr}
\usepackage[utf8]{inputenc}
\usepackage{graphicx}
\usepackage{physics}
\usepackage[listings]{tcolorbox}
\usepackage{hyperref}
\usepackage{tikz-cd}
\usepackage{adjustbox}
\usepackage{enumitem}


\hypersetup{
    colorlinks=true, %set true if you want colored links
    linktoc=all,     %set to all if you want both sections and subsections linked
    linkcolor=black,  %choose some color if you want links to stand out
}
\usetikzlibrary{arrows.meta}

\DeclarePairedDelimiter\ceil{\lceil}{\rceil}
\DeclarePairedDelimiter\floor{\lfloor}{\rfloor}

%=========================================
% Custom Math Operators
%=========================================
\DeclareMathOperator{\adj}{adj}
\DeclareMathOperator{\im}{im}
\DeclareMathOperator{\nullity}{nullity}
\DeclareMathOperator{\sign}{sign}
\DeclareMathOperator{\dom}{dom}
\DeclareMathOperator{\lcm}{lcm}
\DeclareMathOperator{\ran}{ran}
\DeclareMathOperator{\ext}{Ext}
\DeclareMathOperator{\dist}{dist}
\DeclareMathOperator{\diam}{diam}
\DeclareMathOperator{\aut}{Aut}
\DeclareMathOperator{\inn}{Inn}
\DeclareMathOperator{\syl}{Syl}
\DeclareMathOperator{\edo}{End}
\DeclareMathOperator{\cov}{Cov}
\DeclareMathOperator{\vari}{Var}
\DeclareMathOperator{\cha}{char}
\DeclareMathOperator{\Span}{span}
\DeclareMathOperator{\ord}{ord}
\DeclareMathOperator{\res}{res}
\DeclareMathOperator{\Hom}{Hom}
\DeclareMathOperator{\Mor}{Mor}
\DeclareMathOperator{\coker}{coker}
\DeclareMathOperator{\Obj}{Obj}
\DeclareMathOperator{\id}{id}
\DeclareMathOperator{\GL}{GL}
\DeclareMathOperator*{\colim}{colim}

%=========================================
% Custom Commands (Shortcuts)
%=========================================
\newcommand{\CP}{\mathbb{CP}}
\newcommand{\GG}{\mathbb{G}}
\newcommand{\F}{\mathbb{F}}
\newcommand{\N}{\mathbb{N}}
\newcommand{\Q}{\mathbb{Q}}
\newcommand{\R}{\mathbb{R}}
\newcommand{\C}{\mathbb{C}}
\newcommand{\E}{\mathbb{E}}
\newcommand{\Prj}{\mathbb{P}}
\newcommand{\RP}{\mathbb{RP}}
\newcommand{\T}{\mathbb{T}}
\newcommand{\Z}{\mathbb{Z}}
\newcommand{\A}{\mathbb{A}}
\renewcommand{\H}{\mathbb{H}}

\newcommand{\mA}{\mathcal{A}}
\newcommand{\mB}{\mathcal{B}}
\newcommand{\mC}{\mathcal{C}}
\newcommand{\mD}{\mathcal{D}}
\newcommand{\mE}{\mathcal{E}}
\newcommand{\mF}{\mathcal{F}}
\newcommand{\mG}{\mathcal{G}}
\newcommand{\mH}{\mathcal{H}}
\newcommand{\mJ}{\mathcal{J}}
\newcommand{\mO}{\mathcal{O}}
\newcommand{\mS}{\mathcal{S}}

%=========================================
% Theorem Environment
%=========================================
\newcommand\todoin[2][]{\todo[backgroundcolor=white!20!white, inline, caption={2do}, #1]{
\begin{minipage}{\textwidth-4pt}#2\end{minipage}}}

\tcbuselibrary{listings, theorems, breakable, skins}

\newtcbtheorem[number within=subsection]{thm}{Theorem}%
{colback=gray!5, colframe=gray!65!black, fonttitle=\bfseries, breakable, enhanced jigsaw, halign=left}{th}
\newtcbtheorem[number within=subsection, use counter from=thm]{defn}{Definition}%
{colback=gray!5, colframe=gray!65!black, fonttitle=\bfseries, breakable, enhanced jigsaw, halign=left}{th}
\newtcbtheorem[number within=subsection, use counter from=thm]{axm}{Axiom}%
{colback=gray!5, colframe=gray!65!black, fonttitle=\bfseries, breakable, enhanced jigsaw, halign=left}{th}
\newtcbtheorem[number within=subsection, use counter from=thm]{prp}{Proposition}%
{colback=gray!5, colframe=gray!65!black, fonttitle=\bfseries, breakable, enhanced jigsaw, halign=left}{th}
\newtcbtheorem[number within=subsection, use counter from=thm]{lmm}{Lemma}%
{colback=gray!5, colframe=gray!65!black, fonttitle=\bfseries, breakable, enhanced jigsaw, halign=left}{th}
\newtcbtheorem[number within=subsection, use counter from=thm]{crl}{Corollary}%
{colback=gray!5, colframe=gray!65!black, fonttitle=\bfseries, breakable, enhanced jigsaw, halign=left}{th}
\newtcbtheorem[number within=subsection, use counter from=thm]{eg}{Example}%
{colback=gray!5, colframe=gray!65!black, fonttitle=\bfseries, breakable, enhanced jigsaw, halign=left}{th}
\newtcbtheorem[number within=subsection, use counter from=thm]{ex}{Exercise}%
{colback=gray!5, colframe=gray!65!black, fonttitle=\bfseries, breakable, enhanced jigsaw, halign=left}{th}
\newtcbtheorem[number within=subsection, use counter from=thm]{alg}{Algorithm}%
{colback=gray!5, colframe=gray!65!black, fonttitle=\bfseries, breakable, enhanced jigsaw, halign=left}{th}

\newcounter{qtnc}
\newtcolorbox[use counter=qtnc]{qtn}%
{colback=gray!5, colframe=gray!65!black, fonttitle=\bfseries, breakable, enhanced jigsaw, halign=left}




\raggedright

\pagestyle{fancy}
\fancyhf{}
\rhead{Labix}
\lhead{Vector Bundles}
\rfoot{\thepage}

\title{Vector Bundles}

\author{Labix}

\date{\today}
\begin{document}
\maketitle
\begin{abstract}

\end{abstract}
\pagebreak
\tableofcontents
\pagebreak
\section{Vector Bundles}
\subsection{Basic Definitions}
\begin{defn}{Vector Bundles}{} Let $B,E$ be topological spaces and $p:E\to B$ a map. A $K$-vector bundle of rank $r$ is a triple $(B,E,p)$ such that $p$ is a continuous surjection and that 
\begin{itemize}
\item For every $b\in B$, the fibre $E_b=p^{-1}(b)$ is a $K$-vector space of dimension $r$. 
\item For every $b\in B$, there exists an open neighbourhood $U\subseteq B$ of $p$ and a homeomorphism ($(U,\phi)$ is called local trivialization) $\phi:p^{-1}(U)\to U\times K^r$ such that the map 
\begin{itemize}
\item The following diagram commutes \\~\\
\adjustbox{scale=1.1,center}{\begin{tikzcd}
p^{-1}(U)\arrow[rr, "\phi"]\arrow[rdd, "p"'] & & U\times K^r\arrow[ldd, "\pi"]\\
&&\\
& U &
\end{tikzcd}} \\
where $\pi$ is the projection. 
\item The map $$E_p\overset{\phi|_{E_b}}{\longrightarrow}\{b\}\times K^r\overset{\pi}{\longrightarrow}K^r$$ is a vector space isomorphism. 
\end{itemize}
\end{itemize}
\end{defn}

\begin{defn}{Transition Functions}{} Let $p:E\to B$ be a $K$-vector bundle of rank $r$. Let $(U_\alpha,\phi_\alpha)$ and $(U_\beta,\phi_\beta)$ be local trivialization. Define the induced map from $\phi_\alpha\circ\phi_\beta^{-1}$ to be the transition function $g_{\alpha\beta}:U_\alpha\cap U_\beta\to GL(r,K)$ where $$g_{\alpha\beta}(p):K^r\to K^r$$
\end{defn}

\begin{prp}{}{} Let $p:E\to B$ be a $K$-vector bundle of rank $r$. The transition functions of the vector bundle satisfies the following. 
\begin{itemize}
\item $g_{\alpha\beta}\circ g_{\beta\gamma}\circ g_{\gamma\alpha}=I_r$ on $U_\alpha\cap U_\beta\cap U_\gamma$
\item $g_{\alpha\alpha}=I_r$ on $U_\alpha$
\end{itemize}
\end{prp}

\begin{defn}{Sections}{} A section of a vector bundle $p:E\to B$ is a map $s:B\to E$ assigning to each $b\in B$ a vector space $s(b)$ in the fiber $p^{-1}(b)$. 
\end{defn}

\begin{prp}{}{} Let $p:E\to B$ be a vector bundle. Let $s,s_1,s_2$ be sections of $E$. Then $s_1+s_2$ and $\lambda s$ are also vector bundles for any $\lambda\in\R$. Moreover, the set of all sections $s(E)$ is a vector space. 
\end{prp}

\begin{defn}{Morphism of Vector Bundles}{} Let $p_1:E_1\to B_1$ and $p_2:E_2\to B_2$ be vector bundles. A morphism of these vector bundles is given by is a pair of continuous maps $f:E_1\to E_2$ and $g:B_1\to B_2$ such that the following diagram commutes \\~\\
\adjustbox{scale=1.1,center}{\begin{tikzcd}
E_1\arrow[r, "f"]\arrow[d, "p_1"] & E_2\arrow[d, "p_2"]\\
B_1\arrow[r, "g"] & B_2
\end{tikzcd}} \\
If $B=B_1=B_2$  then the diagram collapses: \\~\\
\adjustbox{scale=1.1,center}{\begin{tikzcd}
E_1\arrow[rr, "f"]\arrow[rd, "p_1"] && E_2\arrow[ld, "p_2"]\\
&B&
\end{tikzcd}}
\end{defn}

\begin{defn}{Isomorphism of Vector Bundles}{} A bundle homomorphism from $E_1$ to $E_2$ is an isomorphism if there exists an inverse bundle homomorphism from $E_2$ to $E_1$. In this case, we say that $E_1$ and $E_2$ are isomorphic. 
\end{defn}

\subsection{Operations on Vector Bundles}
\begin{thm}{Whitney Sum}{} Let $p_1:E_1\to B$ and $p_2:E_2\to B$ be two vector bundles. Define the direct sum of the vector bundles to be $$E_1\oplus E_2=\{(v_1,v_2)\in E_1\times E_2|p_1(v_1)=p_2(v_2)\}$$ together with the projection $p:E_1\oplus E_2\to B$ defined by $(v_1,v_2)\mapsto p_1(v)=p_2(v)$. \\~\\
The construction $E_1\oplus E_2$ is a vector bundle over $B$. 
\end{thm}

\begin{prp}{Tensor Product Bundle}{} Let $p_1:E_1\to B$ and $p_2:E_2\to B$ be vector bundles. Define the tensor product bundle of it to be $$E_1\otimes E_2=\{p_1^{-1}(x)\otimes p_2^{-1}(x)|x\in B\}$$ The construction $E_1\otimes E_2$ is a vector bundle over $B$. 
\end{prp}

\begin{thm}{Pullback Bundle}{} Let $p:E\to Y$ be a vector bundle. Let $f:X\to Y$ be a continuous map. Then there exists $E'$ and $p'$ such that $p':E'\to X$ is a vector bundle. 
\end{thm}

\begin{thm}{Dual Bundle}{} Let $p:E\to B$ be a $K$-vector bundle. Then the dual bundle $p^\ast:E^\ast\to B$ defined by $$E_b^\ast=(E_b)^\ast=\Hom(E_b,K)$$ is a vector bundle over $B$. 
\end{thm}

\subsection{Trivial Bundles}
\begin{defn}{The Trivial Bundle}{} Let $B$ be a base space. Define the trivial rank $n$ bundle to be vector bundle with total space $E=B\times\R^n$. We say that a vector bundle is trivial if it is isomorphic to $B\times\R^n$. 
\end{defn}

\begin{prp}{}{} Let $p:E\to B$ be a vector bundle over a paracompact base $B$ and $E_0\subset E$ is a vector subbundle, then there exists a vector subbundle $E_0^\perp$ such that $E_0\otimes E_0^\perp=E$. 
\end{prp}

\begin{prp}{}{} For each vector bundle $E\to B$ over a compact Hausdorff space $B$, there exists a vector bundle $E'\to B$ such that $E\otimes E'$ is the trivial bundle
\end{prp}


\end{document}