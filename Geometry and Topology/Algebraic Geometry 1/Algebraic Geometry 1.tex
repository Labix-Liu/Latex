\documentclass[a4paper]{article}

%=========================================
% Packages
%=========================================
\usepackage{mathtools}
\usepackage{amsfonts}
\usepackage{amsmath}
\usepackage{amssymb}
\usepackage{amsthm}
\usepackage[a4paper, total={6in, 8in}, margin=1in]{geometry}
\usepackage[utf8]{inputenc}
\usepackage{fancyhdr}
\usepackage[utf8]{inputenc}
\usepackage{graphicx}
\usepackage{physics}
\usepackage[listings]{tcolorbox}
\usepackage{hyperref}
\usepackage{tikz-cd}
\usepackage{adjustbox}
\usepackage{enumitem}
\usepackage[font=small,labelfont=bf]{caption}
\usepackage{subcaption}
\usepackage{wrapfig}
\usepackage{makecell}



\raggedright

\usetikzlibrary{arrows.meta}

\DeclarePairedDelimiter\ceil{\lceil}{\rceil}
\DeclarePairedDelimiter\floor{\lfloor}{\rfloor}

%=========================================
% Fonts
%=========================================
\usepackage{tgpagella}
\usepackage[T1]{fontenc}


%=========================================
% Custom Math Operators
%=========================================
\DeclareMathOperator{\adj}{adj}
\DeclareMathOperator{\im}{im}
\DeclareMathOperator{\nullity}{nullity}
\DeclareMathOperator{\sign}{sign}
\DeclareMathOperator{\dom}{dom}
\DeclareMathOperator{\lcm}{lcm}
\DeclareMathOperator{\ran}{ran}
\DeclareMathOperator{\ext}{Ext}
\DeclareMathOperator{\dist}{dist}
\DeclareMathOperator{\diam}{diam}
\DeclareMathOperator{\aut}{Aut}
\DeclareMathOperator{\inn}{Inn}
\DeclareMathOperator{\syl}{Syl}
\DeclareMathOperator{\edo}{End}
\DeclareMathOperator{\cov}{Cov}
\DeclareMathOperator{\vari}{Var}
\DeclareMathOperator{\cha}{char}
\DeclareMathOperator{\Span}{span}
\DeclareMathOperator{\ord}{ord}
\DeclareMathOperator{\res}{res}
\DeclareMathOperator{\Hom}{Hom}
\DeclareMathOperator{\Mor}{Mor}
\DeclareMathOperator{\coker}{coker}
\DeclareMathOperator{\Obj}{Obj}
\DeclareMathOperator{\id}{id}
\DeclareMathOperator{\GL}{GL}
\DeclareMathOperator*{\colim}{colim}

%=========================================
% Custom Commands (Shortcuts)
%=========================================
\newcommand{\CP}{\mathbb{CP}}
\newcommand{\GG}{\mathbb{G}}
\newcommand{\F}{\mathbb{F}}
\newcommand{\N}{\mathbb{N}}
\newcommand{\Q}{\mathbb{Q}}
\newcommand{\R}{\mathbb{R}}
\newcommand{\C}{\mathbb{C}}
\newcommand{\E}{\mathbb{E}}
\newcommand{\Prj}{\mathbb{P}}
\newcommand{\RP}{\mathbb{RP}}
\newcommand{\T}{\mathbb{T}}
\newcommand{\Z}{\mathbb{Z}}
\newcommand{\A}{\mathbb{A}}
\renewcommand{\H}{\mathbb{H}}
\newcommand{\K}{\mathbb{K}}

\newcommand{\mA}{\mathcal{A}}
\newcommand{\mB}{\mathcal{B}}
\newcommand{\mC}{\mathcal{C}}
\newcommand{\mD}{\mathcal{D}}
\newcommand{\mE}{\mathcal{E}}
\newcommand{\mF}{\mathcal{F}}
\newcommand{\mG}{\mathcal{G}}
\newcommand{\mH}{\mathcal{H}}
\newcommand{\mI}{\mathcal{I}}
\newcommand{\mJ}{\mathcal{J}}
\newcommand{\mK}{\mathcal{K}}
\newcommand{\mL}{\mathcal{L}}
\newcommand{\mM}{\mathcal{M}}
\newcommand{\mO}{\mathcal{O}}
\newcommand{\mP}{\mathcal{P}}
\newcommand{\mS}{\mathcal{S}}
\newcommand{\mT}{\mathcal{T}}
\newcommand{\mV}{\mathcal{V}}
\newcommand{\mW}{\mathcal{W}}

%=========================================
% Colours!!!
%=========================================
\definecolor{LightBlue}{HTML}{2D64A6}
\definecolor{ForestGreen}{HTML}{4BA150}
\definecolor{DarkBlue}{HTML}{000080}
\definecolor{LightPurple}{HTML}{cc99ff}
\definecolor{LightOrange}{HTML}{ffc34d}
\definecolor{Buff}{HTML}{DDAE7E}
\definecolor{Sunset}{HTML}{F2C57C}
\definecolor{Wenge}{HTML}{584B53}
\definecolor{Coolgray}{HTML}{9098CB}
\definecolor{Lavender}{HTML}{D6E3F8}
\definecolor{Glaucous}{HTML}{828BC4}
\definecolor{Mauve}{HTML}{C7A8F0}
\definecolor{Darkred}{HTML}{880808}
\definecolor{Beaver}{HTML}{9A8873}
\definecolor{UltraViolet}{HTML}{52489C}



%=========================================
% Theorem Environment
%=========================================
\tcbuselibrary{listings, theorems, breakable, skins}

\newtcbtheorem[number within = subsection]{thm}{Theorem}%
{	colback=Buff!3, 
	colframe=Buff, 
	fonttitle=\bfseries, 
	breakable, 
	enhanced jigsaw, 
	halign=left
}{thm}

\newtcbtheorem[number within=subsection, use counter from=thm]{defn}{Definition}%
{  colback=cyan!1,
    colframe=cyan!50!black,
	fonttitle=\bfseries, breakable, 
	enhanced jigsaw, 
	halign=left
}{defn}

\newtcbtheorem[number within=subsection, use counter from=thm]{axm}{Axiom}%
{	colback=red!5, 
	colframe=Darkred, 
	fonttitle=\bfseries, 
	breakable, 
	enhanced jigsaw, 
	halign=left
}{axm}

\newtcbtheorem[number within=subsection, use counter from=thm]{prp}{Proposition}%
{	colback=LightBlue!3, 
	colframe=Glaucous, 
	fonttitle=\bfseries, 
	breakable, 
	enhanced jigsaw, 
	halign=left
}{prp}

\newtcbtheorem[number within=subsection, use counter from=thm]{lmm}{Lemma}%
{	colback=LightBlue!3, 
	colframe=LightBlue!60, 
	fonttitle=\bfseries, 
	breakable, 
	enhanced jigsaw, 
	halign=left
}{lmm}

\newtcbtheorem[number within=subsection, use counter from=thm]{crl}{Corollary}%
{	colback=LightBlue!3, 
	colframe=LightBlue!60, 
	fonttitle=\bfseries, 
	breakable, 
	enhanced jigsaw, 
	halign=left
}{crl}

\newtcbtheorem[number within=subsection, use counter from=thm]{eg}{Example}%
{	colback=Beaver!5, 
	colframe=Beaver, 
	fonttitle=\bfseries, 
	breakable, 
	enhanced jigsaw, 
	halign=left
}{eg}

\newtcbtheorem[number within=subsection, use counter from=thm]{ex}{Exercise}%
{	colback=Beaver!5, 
	colframe=Beaver, 
	fonttitle=\bfseries, 
	breakable, 
	enhanced jigsaw, 
	halign=left
}{ex}

\newtcbtheorem[number within=subsection, use counter from=thm]{alg}{Algorithm}%
{	colback=UltraViolet!5, 
	colframe=UltraViolet, 
	fonttitle=\bfseries, 
	breakable, 
	enhanced jigsaw, 
	halign=left
}{alg}




%=========================================
% Hyperlinks
%=========================================
\hypersetup{
    colorlinks=true, %set true if you want colored links
    linktoc=all,     %set to all if you want both sections and subsections linked
    linkcolor=DarkBlue,  %choose some color if you want links to stand out
}


\pagestyle{fancy}
\fancyhf{}
\rhead{Labix}
\lhead{Algebraic Geometry 1}
\rfoot{\thepage}

\title{Algebraic Geometry 1}

\author{Labix}

\date{\today}
\begin{document}
\maketitle
\begin{abstract}
Algebraic Geometry is such a messy subject in a sense that a different books and lecture notes introduce different materials in a different orders, as well as having different prerequisites. After understanding a bit more in the subject, I believe that there is the need to give a clear distinction between traditional algebraic geometry and contemporary algebraic geometry. Although there are undoubtedly many overlappings between the two, I attempt to separate them to make clear their motivations as well as their results. Ultimately, it is simply a crucial tool in understanding different polynomials including planar curves, and surfaces and more. \\~\\

This book will mainly cover traditional algebraic geometry in the sense that the construction of affine and projective varieties will be covered, as well as the Hilbert Nullstellensatz theorems, a tad bit of morphisms, and perhaps tangent maps and smoothness as well as classical constructions of morphisms. Affine schemes and sheaf theory are left for another time where they attempt to reinvent the fundamentals of algebraic geometry. \\~\\

Knowledge on commutative algebra is required as a prerequisite. These set of notes make use of
\begin{itemize}
\item Algebraic Geometry I by I. R. Shafarevich and V. I. Danilov
\item Algebraic Geometry by R. Hartshorne
\item An Invitation to Algebraic Geometry by Karen. S, Pekka. K, Lauri .K, William .T
\end{itemize}
\end{abstract}
\pagebreak
\tableofcontents
\pagebreak

\section{Introduction to Affine Varieties}
\subsection{Affine Varieties}
\begin{defn}{Affine Space}{} For a field $k$, define the affine space over $k$ to be the set $$\A^n(k)=\{(a_1,\dots,a_n)|a_i\in k\text{ for }i=1,\dots,n\}$$
\end{defn}

In particular, there is no additional sturcture on $\A^n$ compared to $k^n$ being a vector space equipped with addition and scalar multiplication. 

\begin{defn}{Affine Algebraic Sets}{} Let $F=\{f_i\}$ be a collection of polynomials in $k[x_1,\dots,x_n]$. The zero locus of $F$ is defined to be $$V(F)=\{x\in\A^n|f(x)=0\text{ for all }f\in F\}\subseteq\A^n$$ Subsets of $\A^n$ of this form is called affine algebraic sets. 
\end{defn}

Some authors use the name algebraic sets, reserving the name algebraic variety to refer to irreducible algebraic sets. 

\begin{prp}{}{} If $I$ is the ideal of $F=\{f_i\}$ in $k[x_1,\dots,x_n]$ then $V(F)=V(I)$. \tcbline
\begin{proof}
Basic properties of ideals proves the theorem. This is because every $f\in F$ is a finite sum and product of polynomials in the generating set. 
\end{proof}
\end{prp}

Thus from now on we need not consider ourselves with the affine variety of a countable collection of polynomials since we know that the ring of polynomials of $n$ variables is finitely generated. 

\begin{prp}{}{} Let $\{F_i\;|\;i\in I\}$ be a collection of subsets of $k[x_1,\dots,x_n]$. Then the following are true regarding the zero loci. 
\begin{itemize}
\item Closed under countable intersections: $\bigcap_{i\in I}V(F_i)=V\left(\bigcup_{i\in I}F_i\right)$
\item Closed under finite unions: $\bigcup_{i=1}^nV(F_i)=V\left(\bigcap_{i=1}^nF_i\right)$
\end{itemize}\tcbline
\begin{proof}
\end{proof}
\end{prp}

\begin{prp}{Zariski Topology}{} The complements of the set of all affine algebraic subsets of $X\subseteq\A^n$ forms a topology over $X$ called the Zariski Topology
\end{prp}

\begin{defn}{Affine Algebraic Varieties}{} An affine set is said to be irreducible if $$V=V_1\cup V_2$$ implies $V_1=V$ or $V_2=V$. In this case $V$ is also said to be an affine algebraic variety. 
\end{defn}

\begin{prp}{}{} Every affine algebraic set is a finite union of affine algebraic variety. This decomposition is also unique up to reordering. 
\end{prp}

\subsection{Hilbert's Nullstellensatz}
\begin{defn}{Ideals of an Affine Variety}{} Let $V$ be an affine algebraic set. Define the ideal of $V$ to be $$I(V)=\{f\in k[x_1,\dots,x_n]\;|\;f(x)=0\text{ for all }x\in V\}$$ 
\end{defn}

\begin{lmm}{}{} Let $k$ be a field. Let $V\subseteq\A_k^n$ be an affine algebraic set. Then $I(V)$ is an ideal of $k[x_1,\dots,x_n]$. 
\end{lmm}

Similarly as above, $I(V)$ is finitely generated since ideals in a polynomial ring is finitely generated. 

\begin{prp}{}{} Let $V_1,V_2$ be affine algebraic varieties. The following are true. 
\begin{itemize}
\item If $V_1\subseteq V_2$, then $I(V_1)\supseteq I(V_2)$
\item $I(V_1\cup V_2)=I(V_1)\cap I(V_2)$
\end{itemize}
\end{prp}

Recall that the radical ideal is defined as $$\sqrt{I}=\{f\in k[x_1,\dots,x_n]\;|\;f^r\in I\text{ for some }r>0\}$$

\begin{thm}{Hilbert's Nullstellensatz}{} Let $k$ be an algebraically closed field. Let $I$ be an ideal of $k[x_1,\dots,x_n]$. Then $$I(V(I))=\sqrt{I}$$
\end{thm}

\begin{crl}{}{} Let $k$ be an algebraically closed field. Then there is an inclusion reversing bijection $$\left\{\substack{\text{Radical ideals of}\\ k[x_1,\dots,x_n]}\right\}\;\;\overset{\text{1:1}}{\longleftrightarrow}\;\;\left\{\substack{\text{Affine algebraic}\\\text{sets of }\A_k^n}\right\}$$ between the radical ideals of $k[x_1,\dots,x_n]$ and affine algebraic sets of $\A_k^n$ given by $V(-)$ and $I(-)$. 
\end{crl}

Note that this bijection is compatible with subset inclusion in the sense of proposition 1.2.3. Bijections of this form that induce a relation on subsets are called Galois connections or Galois correspondence, mimicking his work in Galois theory. 

\begin{crl}{}{} Let $k$ be an algebraically closed field. Then there is an inclusion reversing bijection $$\left\{\substack{\text{Prime ideals of}\\ k[x_1,\dots,x_n]}\right\}\;\;\overset{\text{1:1}}{\longleftrightarrow}\;\;\left\{\substack{\text{Affine algebraic}\\\text{varieties of }\A_k^n}\right\}$$ between the prime ideals of $k[x_1,\dots,x_n]$ and affine varieties of $\A_k^n$ given by $V(-)$ and $I(-)$. 
\end{crl}

\begin{crl}{}{} Let $k$ be an algebraically closed field. Then there is an inclusion reversing bijection $$\left\{\substack{\text{Maximal ideals of}\\ k[x_1,\dots,x_n]}\right\}\;\;\overset{\text{1:1}}{\longleftrightarrow}\;\;\left\{\substack{\text{Points in}\\\A_k^n}\right\}$$ between the maximal ideals of $k[x_1,\dots,x_n]$ and points in $\A_k^n$ given by $V(-)$ and $I(-)$. 
\end{crl}

\begin{prp}{}{} Every radical ideal $J$ in $\F[x_1,\dots,x_n]$ is a finite intersection of prime ideals. \tcbline
\begin{proof}
Given a radical ideal $I$, translate it over to its corresponding affine variety $V(I)$. Then the affine variety can be decomposed into a finite union of algebraic varieties $V(I)=\bigcup_{k=1}^nV_k$. These algebraic varities are able to be matched with a prime ideal by the above proposition. This bijection conjugates the union to the intersection and we are done. 
\end{proof}
\end{prp}

\subsection{Polynomial Functions on a Variety}
\begin{defn}{Coordinate Ring}{} Let $k$ be a field and let $V\subseteq\A_k^n$ be an affine variety. Define the coordinate ring of $V$ to be $$k[V]=\frac{k[x_1,\dots,x_n]}{I(V)}$$ to be the ring of polynomial functions on $V$. 
\end{defn}

An example does better than its definition. Let us make an example out of $\R^2$. Let $f(x,y)=xy-1$. Then $V(f)=\{(x,y)\in\R^2|xy=1\}$. Then $\R[V]$ can be described simply where if you see any polynomial with a factor of $xy$ in it, treat it as $1$. For example, if $g(x,y)=(x+y)^2\in\R[x,y]$, then $g(x,y)=x^2+2xy+y^2=x^2+y^2+2\in\R[V]$. This example makes the next theorem quite obvious. 

\begin{prp}{}{} Let $V$ be an affine variety over an algebraically closed field $k$. Then the following are equivalent. 
\begin{itemize}
\item $V$ is irreducible
\item $I(V)$ is a prime ideal
\item $k[V]$ is an integral domain. 
\end{itemize}\tcbline
\begin{proof}
It is clear from ring theory that $I(V)$ is a prime ideal if and only if $\C[V]$ is an integral domain. Suppose now that $V$ is irreducible. Suppose for a contradiction that $\C[V]$ is not an integral domain. Then there exists nonzero $f_1,f_2\in\C[V]$ such that $f_1f_2=0$. Since they are nonzero, $V(f_1)$ and $V(f_2)$ are not $V$. But $V(f_1f_2)=V$. This means that $V(f_1)\cup V(f_2)=V(f_1f_2)=V$. Which means that $V$ is reducible, a contradiction. \\~\\
Suppose now that $\C[V]$ is an integral domain but $V$ is reducible. Then there are some $V_1,V_2$ nonempty and closed such that $V_1\cup V_2=V$. By nullstellensatz, $I(V_1)$ and $I(V_2)$ are non empty since they are not the entire $V$. Choose nonzero $f_1\in I(V_1)$ and $f_2\in I(V_2)$. Then $f_1f_2$ vanishes on $V$. Thus $f_1f_2=0$ which contradicts the fact that $\C[V]$ is an integral domain. 
\end{proof}
\end{prp}

\subsection{Morphisms of Affine Varieties}
\begin{defn}{Regular Maps}{} Let $V\subseteq\A^n$ and $W\subseteq\A^m$ be affine varieties. A regular map from $V$ to $W$ is a map $$\phi:V\to W$$ such that for every $p\in V$, $\phi(p)=(f_1(p),\dots,f_m(p))$ for some polynomial in $f_1,\dots,f_m\in k[V]$. 
\end{defn}

The name regular maps has exactly the same meaning as morphisms of affine varieties. 

\begin{defn}{Isomorphic Varieties}{} A regular map $\phi:X\to Y$ between two varieties is an isomorphism if it has an inverse that is a regular map. $X$ and $Y$ are said to be isomorphic in this case. 
\end{defn}

\begin{prp}{}{} Let $V,W,U$ be affine varieties. If $f:V\to W$ and $g:W\to U$ are regular maps, then $g\circ f:V\to U$ is also a regular map. 
\end{prp}

\begin{defn}{Pullback of a Regular Map}{} Let $\phi:V\to W$ be a morphism of varieties. Then define the pull back of $\phi$ by $$\phi^\ast:k[W]\to k[V]$$ where $\phi^\ast(p)=p\circ\phi$ for each $p\in k[W]$. 
\end{defn}

\begin{lmm}{}{} Let $V,W,U$ be affine varieties. If $f:V\to W$ and $g:W\to U$ are regular maps, then $$(g\circ f)^\ast=f^\ast\circ g^\ast$$
\end{lmm}

\begin{prp}{}{} Let $k$ be an algebraically closed field. Let $V$ and $W$ be affine varieties over $k$. Then there is a bijection $$\left\{\substack{\text{Regular maps}\\ V\to W}\right\}\;\;\overset{\text{1:1}}{\longleftrightarrow}\;\;\left\{\substack{\text{Algebra Homomorphisms}\\ k[W]\to k[V]}\right\}$$ between the regular maps and algebra homomorphisms. 
\end{prp}

\begin{crl}{}{} Let $V$ and $W$ be affine varieties over a field $k$. Then $V$ and $W$ are isomorphic if and only if $k[V]\cong k[W]$. 
\end{crl}

\subsection{Rational Functions on a Variety}
While polynomial rings are more than sufficient to classify varieties up to isomorphism, we would still like to enlarge the set of functions on a variety. We will do this by introducing rational functions on a variety. 

\begin{defn}{Function Field}{} Let $V$ be an affine variety over a field $k$. Define the function field of $V$ to be $$k(V)=\text{Frac}(k[V])$$ Elements of $k(V)$ are said to be rational functions on $V$. 
\end{defn}

Notice that functions in $k(V)$ are not well defined on all of $V$. Some functions may have poles on $V$. But by restricting to a certain open set in $V$ (namely by removing the poles from the domain), we obtain a well defined rational function. \\~\\

The remaining section constructs a sheaf of rational functions on varieties. Sheaves will not be formally in these notes. 

\begin{defn}{Regular Functions}{} Let $V$ be an affine algebraic variety over a field $k$. Then $f\in k(V)$ is said to be a regular function at $p\in V$ if $$f(x)=\frac{g(x)}{h(x)}$$ for $g,h\in k[V]$ and $h(p)\neq 0$. Let $U$ be an open subset of $V$. We say that $f\in k(V)$ is regular at $U$ if $f$ is regular at all $p\in U$. 
\end{defn}

\begin{defn}{Set of Regular Functions}{} Let $V$ be an affine algebraic variety over a field $k$. Denote $$\mO_V(U)=\{f\in k(V)\;|\;f\text{ is regular at }U\}$$ the set of all regular functions on $U$. 
\end{defn}

\begin{defn}{Local Ring}{} Let $V$ be an affine variety over a field $k$. Let $p\in V$. Define the local ring of $V$ at $p$ to be $$\mO_{V,p}=\{f\in k(V)\;|\;f\text{ is regular at }p\}$$ the set of all regular functions at $p$. 
\end{defn}

It is natural to ask that given a regular map between two varieties. How does all these regular functions and local rings transfer from one to another. In the case of polynomial functions (coordinate rings), the transferral comes from the opposite direction. The case for regular functions are similar. However we will not continue the exposition here since the language of sheaves will make definitions easier. 

\pagebreak
\section{Introduction to Projective Varieties}
\subsection{Projective Space}
\begin{lmm}{}{} Let $\A$ be a field. The relation $\sim$ in $\A^{n+1}$ where $(x_0,\dots,x_n)\sim(y_0,\dots,y_n)$ if and only if $y_i=\lambda x_i$ for all $i\{1,\dots,n\}$ with $\lambda\in\A$ is an equivalence relation. 
\end{lmm}

\begin{defn}{Projective Space}{} The equivalence relation $\sim$ on $\A^{n+1}$ induces the projective space with elements in it being $1$ dimensional subspaces of $\A^n$, written as $$\Prj^n(\A)=\frac{\A^{n+1}\setminus\{0\}}{\sim}$$
\end{defn}

\begin{prp}{}{} There is a bijection between $\mathbb{P}^n$ and the set of lines through the origin in $\A^{n+1}$. 
\end{prp}

\begin{prp}{}{} We have the following identification of projective space: $$\Prj^n\cong\A^n\cup\Prj^{n-1}$$ where the union is disjoint and  $$[x_0:\dots:x_n]\to\begin{cases}
\left(\frac{x_1}{x_0},\dots,\frac{x_n}{x_0}\right)\in\A^n & \text{ for } x_0\neq0\\
[x_1,\dots,x_n]\in\Prj^{n-1} & \text{ for } x_0=0
\end{cases}$$
\end{prp}

This is an identification of points in $\Prj^n$. They are either affine coordinates or coordinates at infinity. Recursively, we can deduce that $$\Prj^n\cong\A^n\cup\dots\cup\A\cup\{\infty\}$$ All of the above unions are disjoint. 

\begin{prp}{}{} Let $U_k=\{[x_0:\dots x_n]\in\Prj^n|x_k\neq0\}=\{[x_0:\dots x_n]\in\Prj^n|x_k=1\}$. Then $$\Prj^n=\bigcup_{k=0}^nU_k$$
\end{prp}

\begin{prp}{}{} The projective space is an $n$ dimensional manifold. 
\end{prp}

\subsection{Homogenous Functions}
\begin{defn}{Homogenous Functions}{} A polynomial $f$ is said to be homogenous of degree $d$ if all its terms has total degree $d$. Denote $\C_d[x_1,\dots,x_n]$ the set of all homogenous polynomials of degree $d$ over $\C$. 
\end{defn}

\begin{lmm}{}{} If $f$ is homogenous of degree $d$ then $f(\lambda x_0,\dots,\lambda x_n)=\lambda^df(x_0,\dots,x_n)$
\end{lmm}

\begin{lmm}{}{} For all $f\in\A[x_0,\dots,x_n]$, $$f=f_0+f_1+\dots+f_d$$ where $f_i$ is homogenous of degree $i$. 
\end{lmm}

\subsection{Projective Varities}
Recall that an ideal is homogenous if it is generated by homogenous elements. 

\begin{defn}{Projective Variety}{} Let $I$ be an homogenous ideal. A projective variety in $\Prj^n$ is the common vanishing locus $$V(I)=\{x\in\Prj^n|F(x)=0\text{ for all }F\in I\}$$ 
\end{defn}

\begin{defn}{Ideals of Projective Varieties}{} Let $V$ be a projective variety. Define $$I(V)=\{F\in k[x_0,\dots,x_n]|F(V)=0\}$$ It is obvious that $I$ would be a homogenous ideal. 
\end{defn}

\begin{prp}{}{} The following are true for projective varieties. 
\begin{itemize}
\item Projective varieties are closed under countable intersections. $$\bigcap_{i\in I}V(F_i)=V\left(\bigcup_{i\in I}F_i\right)$$
\item Projective varieties are closed under finite unions. If $F=\{f_1,\dots,f_n\}$, then $$\bigcup_{i=1}^nV(F_i)=V(F)$$
\end{itemize}
\end{prp}

\begin{crl}{Zariski Topology}{} The complements of the set of all projective variety forms a topology over $\Prj^n$ called the Zariski Topology
\end{crl}

\subsection{The Projective Nullstellensatz}
We have the projective version of the Nullstellensatz. It works exactly the same as that of the affine version. 

\begin{thm}{The Projective Nullstellensatz}{} Let $J\subseteq\C[x_1,\dots,x_n]$ be an homogenous ideal. Then $$I^H(V(J))=\sqrt{J}$$
\end{thm}

\begin{crl}{}{} There is an inclusion reversing bijective correspondence between projective varities in $\Prj^n$ and homogenous radical ideals in $\C[x_0,\dots,x_n]$. $$\left\{\substack{\text{Homogenous Radical Ideals of}\\ k[x_1,\dots,x_n]\\\text{such that }J\nsupseteq(x_1,\dots,x_n)}\right\}\;\;\overset{\text{1:1}}{\longleftrightarrow}\;\;\left\{\substack{\text{Projective sets of}\\ \Prj^n}\right\}$$ given by $V(-)$ and $I^H(-)$. 
\end{crl}

\begin{crl}{}{} There is an inclusion reversing bijective correspondence between projective varities in $\Prj^n$ and homogenous radical ideals in $\C[x_0,\dots,x_n]$. $$\left\{\substack{\text{Homogenous Prime Ideals of}\\ k[x_1,\dots,x_n]\\\text{such that }J\nsupseteq(x_1,\dots,x_n)}\right\}\;\;\overset{\text{1:1}}{\longleftrightarrow}\;\;\left\{\substack{\text{Projective varieties of}\\ \Prj^n}\right\}$$ given by $V(-)$ and $I^H(-)$. 
\end{crl}

\begin{prp}{}{} Every projective variety is a finite union of irreducible projective varieties. 
\end{prp}

\subsection{The Relation Between Affine and Projective Varieties}
\begin{defn}{The Dehomogenization Map}{} The map $\phi_i:\C_d[z_0,\dots,z_n]\to\C[x_1,\dots,x_n]$ defined by $$f(z_0,\dots,z_n)\mapsto f\left(1,\frac{z_1}{z_0},\dots,\frac{z_n}{z_0}\right)=f(1,x_1,\dots,x_n)$$ is called dehomogenization with respect to $z_i$. 
\end{defn}

\begin{lmm}{}{} Let $V\subseteq\Prj^n$ be a projective variety. Let $U_i$ be a chart of $\Prj^n$ where the $i$th coordinate is $1$. Then $$(V\cap U_i)\subseteq U_i\cong\A^n$$ is an affine variety. 
\end{lmm}

\begin{thm}{}{} Let $(U_i,\psi_i)$ be the chart of $\Prj^n$ where the $i$th coordinate is $1$. Then the map $$\psi_i:V(f)\cap U_i\subseteq\Prj^n\to V(\phi_i(f))\subseteq\A^n$$ where $\phi_i$ is the dehomogenization with respect to $z_i$. 
\end{thm}

\begin{defn}{Homogenization}{} Let $f\in k[x_1,\dots,x_n]$ where $f=g_0+\dots+g_d$ and $g_k$ are the terms of degree $k$ in $f$. Define the homogenization of $f$ to be the new function $$F(x_0,\dots,x_n)=x_0^dg_0+x_0^{d-1}g_1+\dots+g_d$$ The new function $F$ is homogenous of degree $d$, not divisble by $x_0$. This map from $\C[x_1,\dots,x_n]$ to $\C[x_0,\dots,x_n]$ denoted by $\varphi$ is called homogenization. 
\end{defn}

\begin{thm}{}{} Let $f\in\C[x_1,\dots,x_n]$. Let $F\in\C[x_0,\dots,x_n]$ and $F=x_0^kG$ such that $z_0$ does not divide $G$. Then the following are true. 
\begin{itemize}
\item $\phi_0(\varphi(f))=f$
\item $\varphi(\phi_0(F))=G$
\end{itemize}
\end{thm}

\begin{lmm}{}{} Let $\{f_i|i\in I\}\subset\C[x_1,\dots,x_n]$. Let $U_0$ be the chart of $\Prj^n$ for which $x_0=1$. Then $$V(\{\varphi(f_i)|i\in I\})\cap U_0=V(\{f_i|i\in I\})$$
\end{lmm}

\begin{thm}{}{} Suppose that $U_i$ inherits the Zariski topology from $\Prj^n$. Then $\psi_i:U_i\to\A^n$ is a homeomorphism. 
\end{thm}

In general, for $W$ an affine variety of $\C^n$, considering $\C^n$ in the open cover $U_0$, $W$ may not be closed and so may not be a projective variety. However the closure certainly is. 

\begin{defn}{Projective Closure}{} Let $V$ be an affine variety of $\C^n$. Then define the projective closure of $V$ to be the closure of $V$ when considered inside $\Prj^n$. 
\end{defn}

\begin{prp}{}{} Let $I$ be a radical ideal of $\C[x_1,\dots,x_n]$. Let $W=V(I)\subseteq\A^n$. Denote $\overline{W}$ the projective closure of $W$. Then $$(\{\varphi(f)|f\in I\})=I(\overline{W})$$
\end{prp}

In other words, the homogenization of the radical ideal is precisely the generating set of the projective closure. 


\begin{prp}{}{} Let $V\subset U_0\cong\A^n$ be an affine variety. Then the closure of $V$ in the Zariski topology of $\A^n$ and are exactly the projective closures of $V$. They coincide. 
\end{prp}

\subsection{Morphisms of Projective Varieties}
\begin{defn}{Morphisms of Projective Varieties}{} Let $V\subseteq\Prj^n$ and $W\subseteq\Prj^m$ be projective varieties. Let $F:V\to W$ be a map (of sets) from $V$ to $W$. We say that $F$ is a morphism of projective varieties if for each $p\in V$, there exists homogeneous polynomials $F_0,\dots,F_m\in\C[x_0,\dots,x_n]$ of the same degree and an open neighbourhood $U$ of $p$ such that the following holds. 
\begin{itemize}
\item $V(F_0,\dots,F_m)\cap U=\emptyset$ (They cannot all vanish at the same time)
\item $F|_U:U\to W$ agrees with the map $U\to\Prj^m$ defined by $$[z_0:\cdots:z_n]\mapsto[F_0(z_0,\dots,z_n):\cdots:F_m(z_0,\dots,z_n)]$$
\end{itemize}
\end{defn}

This defintion ensures that when a morphism of projective varities is restricted to its open covers $\A^n$, it defines a morphism of affine varieties. They are locally affine varieties. 

\begin{defn}{Isomorphism of Projective Varieties}{} A morphism of projective varieties $F:V\to W$ is an isomorphism if there exists a morphism $G:W\to V$ such that $G$ is the inverse of $F$. In this case we say that $V$ and $W$ are isomorphic. \\~\\
An automophism of a projective variety $V$ is an isomorphism from $V$ to itself. 
\end{defn}

\begin{defn}{Projectively Equivalent}{} Two projective varieties are said to be projectively equivalent if there exists a change of coordinates of $\Prj^n$ that defines an isomorphism between them. 
\end{defn}

\pagebreak
\section{Quasi-Projective Varieties}
\subsection{Quasi-Projective Varieties}
In this section we attempt to unify the two types of varieties, affine and projective into one unified theory. 

\begin{defn}{Locally Closed Subsets}{} A locally closed subset of a topological space $X$ is a subset of the form $U\cap V$ where $U$ is open in $X$ and $V$ is closed in $X$. 
\end{defn}

\begin{defn}{Quasi-Projective Varieties}{} A quasiprojective variety is a locally closed subset of $\Prj^n$. 
\end{defn}

\begin{lmm}{}{} Affine varieties and projective varieties are both quasiprojective varieties. \tcbline
\begin{proof}
Let $W\subseteq\A^n$ be an affine variety. Then $W$ is closed and thus $W=\overline{W}\cap U_0$ where $\overline{W}$ is the closure of $W$ in $\Prj^n$ by $\A^n\cong U_0\subseteq\Prj^n$. \\~\\
Let $V\subseteq\Prj^n$ be a projective variety. Then $V$ being closed implies $V=V\cap\Prj^n$ trivially. 
\end{proof}
\end{lmm}

\begin{lmm}{}{} Any open subset of $\Prj^n$ or $\A^n$ is a quasiprojective variety. 
\end{lmm}

\subsection{Morphisms of Quasi-Projective Varieties}
\begin{defn}{Morphisms of Quasi-Projective Varities}{} Let $X\subseteq\Prj^n$ and $Y\subseteq\Prj^m$ be quasiprojective varieties. A morphism from $X$ to $Y$  is a map $F:X\to Y$ such that for all $p\in X$, there exists an open neighbourhood $U_p$ together with homogenous polynomials $F_0,\dots,F_m\in k[x_0,\dots,x_n]$ of the same degree such that 
\begin{itemize}
\item $V(F_0,\dots,F_m)\cap U=\emptyset$
\item $F|_U$ agrees with $[x_0,\dots,x_n]\to[F_0(x_0,\dots,x_n),\dots,F_m(x_0,\dots,x_n)]$
\end{itemize}
\end{defn}

\begin{lmm}{}{} Every morphism of affine or projective varieties is a morphism of quasiprojective varieties. 
\end{lmm}

\subsection{Redefining Varieties}
\begin{defn}{Extended Definition of Affine Varities}{} A quasiprojective variety is said to be affine if it is isomorphic to a closed subset of affine space. 
\end{defn}

\begin{defn}{Extended Definition of Projective Varities}{} A quasiprojective variety is said to be projective if it is isomorphic to a closed subvariety of projective space. 
\end{defn}

\begin{defn}{Basic Open Sets}{} Let $V$ be Zariski closed on $\A^n$. Let $f\in k[V]$. Then $$D(f)=V\setminus V(f)$$ is said to be a basic open set. 
\end{defn}

In other words, $D(f)$ is exactly the points of $V$ where $f$ is not zero. Some literature like to use $V_f$ for notation. 

\begin{prp}{}{} Let $V$ be Zariski closed on $\A^n$. Let $f\in k[V]$. Then 
\begin{itemize}
\item $D(f)$ is an affine algebraic variety
\item Every open subset of $V$ is a union of basic open sets
\item The set of all basic open sets of $X$ forms a basis for the Zariski Topology
\end{itemize}
\end{prp}

\begin{prp}{}{} Every quasiprojective variety is locally affine. 
\end{prp}

\subsection{Regular Functions}
The following definition is simply a special case of regular maps, as in regular functions are simply regular maps with codomain $\A^1$. 

\begin{defn}{Regular Functions on Affine Varieties}{} Let $U$ be an open subset of an affine variety $V$. A complex valued function $f:U\to\C$ is regular at a point $p\in U$ if there exists functions $g,h\in\C[V]$ such that $h(p)\neq 0$ and that $$f(x)=\frac{g(x)}{h(x)}$$ in some neighbourhood of $p$. Moreover, $f$ is said to be regular on $U$ if it is regular at every point of $U$. The set of all regular functions is denoted by $$\mathcal{O}_V(U)=\{f:U\to\C\;|\;f\text{ is regular on }U\}$$
\end{defn}

\begin{defn}{Regular Functions on Quasi-Projective Varities}{} Let $U$ be an open subset of a quasiprojective variety $V$. A complex valued function $f:U\to\C$ is regular at a point $p\in U$ if there some affine open set containing $p$ on which $f$ is regular at $p$. 
$f$ is said to be regular on $U$ if it is regular at every point of $U$. \\~\\
The set of all regular functions is denoted by $\mathcal{O}_V(U)$. 
\end{defn}

\begin{lmm}{}{} For any $U\subset V$ open and $V$ a quasiprojective variety, $\mathcal{O}_V(U)$ is a $\C$-algebra. 
\end{lmm}

\begin{defn}{Ring of Germs of Regular Functions}{} Let $p$ be a point of a variety $X$. Define the local ring of $p$ on $X$ to be $$\mathcal{O}_{V,p}=\{(U,f)\;|\;U\subseteq X\text{ is open}, p\in U, f\text{ is regular on }U\}/\sim$$ where $(U,f)\sim(V,g)$ if and only if $f=g$ on $U\cap V$. 
\end{defn}

\begin{prp}{}{} Let $X$ be a variety and $p\in X$. Then the ring of germs $\mathcal{O}_{X,p}$ is a local ring. 
\end{prp}

\begin{defn}{Function Field}{} Let $X$ be a variety. Define the function field of $X$ to be $$K(X)=\{(U,f)|U\subseteq X\text{ is open}, f\text{ is regular on }U\}/\sim$$ where $(U,f)\sim(V,g)$ if and only if $f=g$ on $U\cap V$. Elements of $K(X)$ are called rational functions on $X$. 
\end{defn}

\begin{lmm}{}{} Let $X$ be a variety. Then the function field $K(X)$ of $X$ is a field. 
\end{lmm}

\begin{lmm}{}{} Let $X$ be a variety. For any point $p$, there are natural injective maps $\mathcal{O}_X(X)\to\mathcal{O}_{X,p}\to K(X)$. 
\end{lmm}

\begin{prp}{}{} Let $X$ and $Y$ be isomorphic varieties. Then $\mathcal{O}_X(X)\cong\mathcal{O}_Y(Y)$, $K(X)\cong K(Y)$ and if $p\in X$ maps to $q\in Y$, then $\mathcal{O}_{X,p}\cong\mathcal{O}_{Y,q}$. 
\end{prp}

\begin{thm}{}{} Let $V$ be an affine variety. Let $U$ be an open subset of $V$. 
\begin{itemize}
\item There is a $\C$-algebra homomorphism $\C[V]\to\mathcal{O}_V(U)$ given by restriction of functions
\item The above map is injective if $U$ is dense in $V$
\item If $V$ is an irreducible affine variety, then the map is surjective and an isomorphsim. 
\end{itemize}
\end{thm}

\begin{thm}{}{} Let $V\subseteq\A^n$ be an affine variety. Let $p\in V$ be a point. Then the following are true. 
\begin{itemize}
\item $\mathcal{O}_V(V)\cong\C[V]$
\item Let $m_p=\{f\in\C[V]|f(p)=0\}$ be the ideal of functions that vanish at $p$. Then the map $p\mapsto m_p$ gives a one to one correspondence between points of $V$ and the maximal ideals of $\C[V]$
\item For each $p\in V$, $\mathcal{O}_{V,p}\cong\C[V]_{m_p}$ and $\dim(\mathcal{O}_{V,p})=\dim(X)$
\item $K(X)\cong\text{Frac}(\C[V])$ and $K(X)$ is a finitely generated extension field of $\C$. 
\end{itemize}
\end{thm}

\begin{thm}{}{} Let $X\subseteq\Prj^n$ be a projective variety. Let $p\in V$. Then the following are true. 
\begin{itemize}
\item $\mathcal{O}_V(V)\cong\C$
\item Let $m_p=\{f\in\C[V]|f\text{ is homogenous and }f(p)=0\}$ be the ideal of functions that vanish at $p$. Then $\mathcal{O}_{V,p}=\C[V]_{m_p}$
\item $K(X)\cong\C[V]_{(0)}$
\end{itemize}
\end{thm}

\pagebreak
\section{Classical Constructions}
\subsection{Veronese Maps}
\begin{defn}{Veronese Maps}{} The $d$th veronese map of $\Prj^n$ is the morphism $\nu_d:\Prj^n\to\Prj^m$ defined by $$\nu_d([x_0:\cdots:x_n])=[x_0^d:x_0^{d-1}x_1:\cdots:x_n^d]$$ where $m=\binom{d+n}{n}-1$. 
\end{defn}

\begin{prp}{}{} The veronese mapping $\nu_d$ defines an isomorphism of $\Prj^n$ onto its image. 
\end{prp}

\subsection{Segre Maps}
\begin{defn}{General Segre Maps}{} Define the general Segre map $\sum_{m,n}:\Prj^m\times\Prj^n\to\Prj^{(m+1)(n+1)-1}$ by $$\sum_{m,n}([x_0:\cdots:x_m],[y_0:\cdots:y_n])=[x_0y_0:x_0y_1:\cdots:x_iy_j:\cdots:x_my_n]$$
\end{defn}

\subsection{Grassmannians}
\begin{defn}{Grassmannians}{} Let $n\in\N^+$. Let $k\in\N$ with $0\leq k\leq n$. Denote $G(k,n)$ the set of all $k$-dimensional linear subspaces of $\C^n$. 
\end{defn}

Similar to how $G(1,n)=\Prj^{n-1}$, $G(k,n)$ is essentially the $k-1$ dimensional projective subspaces of $\Prj^{n-1}$. 

\begin{lmm}{}{} $G(k,n)$ can be identified with the set $$G=\frac{\{M\in M_{k\times n}(\C)|M\text{ has full rank }\}}{\text{ Orbits of }GL(k)}$$ \tcbline
\begin{proof}
Let $V$ be a $k$-dimensional vector subspace of $\C^n$. Choose basis vectors $(a_{j1},\dots,a_{jn})$ where $j=1,\dots,k$ for $V$. Form the row matrix of basis vectors $$A=\begin{pmatrix}a_{11} & \cdots & a_{1n}\\\vdots & \ddots & \vdots\\ a_{n1} & \cdots & a_{nn}\end{pmatrix}$$
This matrix is formed by a basis thus the rows must be linearly independent, which means it achieves full rank. Two matrices span the same subspace if and only if there exists an invertible matrix of dimension $k$ such that $(a_{ij})=g(b_{ij})$. So we can quotient out extra elements in $M_{k\times n}(\C)$ that represent the same vector subspace to get an identification of $G(k,n)$: $$G=\frac{\{M\in M_{k\times n}(\C)|M\text{ has full rank }\}}{\text{ Orbits of }GL(k)}$$ 
\end{proof}
\end{lmm}

\begin{defn}{Plucker Embedding}{} Denote $\Delta_{i_1,\dots,i_k}$ the $k\times k$ subdeterminant of $A\in M_{k\times n}(\C)$. formed by the columns $i_1,\dots,i_k$ in $A$. The Plucker embedding is the map $\phi:G\to\Prj^{\binom{n}{k}-1}$ given by $$\phi\left(\begin{pmatrix}a_{11} & \cdots & a_{1n}\\\vdots & \ddots & \vdots\\ a_{n1} & \cdots & a_{nn}\end{pmatrix}\right)=[\Delta_{1,\dots,k}:\dots:\Delta_{i_1,\dots,i_k}:\dots:\Delta_{n-k+1,\dots,n}]$$ 
\end{defn}

\begin{prp}{}{} The Plucker embedding is well defined and is injective. \tcbline
\begin{proof}
This map is well defined since for any two matrices of rank $k$ that span the same subspace that differ by multiplication of $G\in GL(k)$, they give the same point since multiplying $G$ changes the subdeterminants by a factor of $\det(G)$, and in projective space they mean the same point. Moreover, since matrices in $G$ has full rank, there must be at least one subdeterminant is nonzero. 
\end{proof}
\end{prp}

\begin{thm}{}{} The Grassmannians $G(k,n)$ can be embedded as a complex submanifold of $\Prj^{\binom{n}{k}-1}$. \tcbline
\begin{proof}
Using the Plucker embedding, we see that $G(k,n)$ can be identified with a subset of $\Prj^{\binom{n}{k}-1}$. We now need to give an atlas to it. An open cover of $G(k,n)$ in the projective space is given by $$U_{(i_1,\dots,i_k)}=\{V\in Gr(k,n)|\Delta_{i_1,\dots,i_k}\neq 0\}$$. Since the submatrix formed by the columns $i_1,\dots,i_k$ is nonzero, we can find a representation of the subspace where each columns $i_1,\dots,i_k$ is the unit vector $e_1,\dots,e_k$. The rest of the $k(n-k)$ coordinates can be used to as an identification in the atlas. This means that we have a map $U_{i_1,\dots,i_k}\to\C^{k(n-k)}$. \\~\\
The transition maps between two open cover is given by the rational functions $\frac{\Delta_{i_1,\dots,i_k}}{\Delta_{j_1,\dots,j_k}}$, which is clearly analytic. 
\end{proof}
\end{thm}

\begin{thm}{}{} The Grassmannians $G(k,n)$ is a projective algebraic variety. 
\end{thm}

















\end{document}