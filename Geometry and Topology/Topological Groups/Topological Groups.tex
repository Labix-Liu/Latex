\documentclass[a4paper]{article}

%=========================================
% Packages
%=========================================
\usepackage{mathtools}
\usepackage{amsfonts}
\usepackage{amsmath}
\usepackage{amssymb}
\usepackage{amsthm}
\usepackage[a4paper, total={6in, 8in}, margin=1in]{geometry}
\usepackage[utf8]{inputenc}
\usepackage{fancyhdr}
\usepackage[utf8]{inputenc}
\usepackage{graphicx}
\usepackage{physics}
\usepackage[listings]{tcolorbox}
\usepackage{hyperref}
\usepackage{tikz-cd}
\usepackage{adjustbox}
\usepackage{enumitem}
\usepackage[font=small,labelfont=bf]{caption}
\usepackage{subcaption}
\usepackage{wrapfig}

\raggedright

\usetikzlibrary{arrows.meta}

\DeclarePairedDelimiter\ceil{\lceil}{\rceil}
\DeclarePairedDelimiter\floor{\lfloor}{\rfloor}

%=========================================
% Custom Math Operators
%=========================================
\DeclareMathOperator{\adj}{adj}
\DeclareMathOperator{\im}{im}
\DeclareMathOperator{\nullity}{nullity}
\DeclareMathOperator{\sign}{sign}
\DeclareMathOperator{\dom}{dom}
\DeclareMathOperator{\lcm}{lcm}
\DeclareMathOperator{\ran}{ran}
\DeclareMathOperator{\ext}{Ext}
\DeclareMathOperator{\dist}{dist}
\DeclareMathOperator{\diam}{diam}
\DeclareMathOperator{\aut}{Aut}
\DeclareMathOperator{\inn}{Inn}
\DeclareMathOperator{\syl}{Syl}
\DeclareMathOperator{\edo}{End}
\DeclareMathOperator{\cov}{Cov}
\DeclareMathOperator{\vari}{Var}
\DeclareMathOperator{\cha}{char}
\DeclareMathOperator{\Span}{span}
\DeclareMathOperator{\ord}{ord}
\DeclareMathOperator{\res}{res}
\DeclareMathOperator{\Hom}{Hom}
\DeclareMathOperator{\Mor}{Mor}
\DeclareMathOperator{\coker}{coker}
\DeclareMathOperator{\Obj}{Obj}
\DeclareMathOperator{\id}{id}
\DeclareMathOperator{\GL}{GL}
\DeclareMathOperator*{\colim}{colim}

%=========================================
% Custom Commands (Shortcuts)
%=========================================
\newcommand{\CP}{\mathbb{CP}}
\newcommand{\GG}{\mathbb{G}}
\newcommand{\F}{\mathbb{F}}
\newcommand{\N}{\mathbb{N}}
\newcommand{\Q}{\mathbb{Q}}
\newcommand{\R}{\mathbb{R}}
\newcommand{\C}{\mathbb{C}}
\newcommand{\E}{\mathbb{E}}
\newcommand{\Prj}{\mathbb{P}}
\newcommand{\RP}{\mathbb{RP}}
\newcommand{\T}{\mathbb{T}}
\newcommand{\Z}{\mathbb{Z}}
\newcommand{\A}{\mathbb{A}}
\renewcommand{\H}{\mathbb{H}}

\newcommand{\mA}{\mathcal{A}}
\newcommand{\mB}{\mathcal{B}}
\newcommand{\mC}{\mathcal{C}}
\newcommand{\mD}{\mathcal{D}}
\newcommand{\mE}{\mathcal{E}}
\newcommand{\mF}{\mathcal{F}}
\newcommand{\mG}{\mathcal{G}}
\newcommand{\mH}{\mathcal{H}}
\newcommand{\mJ}{\mathcal{J}}
\newcommand{\mO}{\mathcal{O}}
\newcommand{\mS}{\mathcal{S}}


%=========================================
% Colours!!!
%=========================================
\definecolor{LightBlue}{HTML}{2D64A6}
\definecolor{ForestGreen}{HTML}{4BA150}
\definecolor{DarkBlue}{HTML}{000080}


%=========================================
% Theorem Environment
%=========================================
\tcbuselibrary{listings, theorems, breakable, skins}

\newtcbtheorem[number within = subsection]{thm}{Theorem}%
{	colback=gray!5, 
	colframe=gray!65!black, 
	fonttitle=\bfseries, 
	breakable, 
	enhanced jigsaw, 
	halign=left
}{thm}

\newtcbtheorem[number within=subsection, use counter from=thm]{defn}{Definition}%
{	colback=gray!5, 
	colframe=gray!65!black, 
	fonttitle=\bfseries, breakable, 
	enhanced jigsaw, 
	halign=left
}{defn}

\newtcbtheorem[number within=subsection, use counter from=thm]{axm}{Axiom}%
{	colback=gray!5, 
	colframe=gray!65!black, 
	fonttitle=\bfseries, 
	breakable, 
	enhanced jigsaw, 
	halign=left
}{axm}

\newtcbtheorem[number within=subsection, use counter from=thm]{prp}{Proposition}%
{	colback=gray!5, 
	colframe=gray!65!black, 
	fonttitle=\bfseries, 
	breakable, 
	enhanced jigsaw, 
	halign=left
}{prp}

\newtcbtheorem[number within=subsection, use counter from=thm]{lmm}{Lemma}%
{	colback=gray!5, 
	colframe=gray!65!black, 
	fonttitle=\bfseries, 
	breakable, 
	enhanced jigsaw, 
	halign=left
}{lmm}

\newtcbtheorem[number within=subsection, use counter from=thm]{crl}{Corollary}%
{	colback=gray!5, 
	colframe=gray!65!black, 
	fonttitle=\bfseries, 
	breakable, 
	enhanced jigsaw, 
	halign=left
}{crl}

\newtcbtheorem[number within=subsection, use counter from=thm]{eg}{Example}%
{	colback=gray!5, 
	colframe=gray!65!black, 
	fonttitle=\bfseries, 
	breakable, 
	enhanced jigsaw, 
	halign=left
}{eg}

\newtcbtheorem[number within=subsection, use counter from=thm]{ex}{Exercise}%
{	colback=gray!5, 
	colframe=gray!65!black, 
	fonttitle=\bfseries, 
	breakable, 
	enhanced jigsaw, 
	halign=left
}{ex}

\newtcbtheorem[number within=subsection, use counter from=thm]{alg}{Algorithm}%
{	colback=gray!5, 
	colframe=gray!65!black, 
	fonttitle=\bfseries, 
	breakable, 
	enhanced jigsaw, 
	halign=left
}{alg}




%=========================================
% Hyperlinks
%=========================================
\hypersetup{
    colorlinks=true, %set true if you want colored links
    linktoc=all,     %set to all if you want both sections and subsections linked
    linkcolor=DarkBlue,  %choose some color if you want links to stand out
    citecolor=blue,  %choose some color if you want links to stand out
}


\pagestyle{fancy}
\fancyhf{}
\rhead{Labix}
\lhead{Algebraic Topology 1}
\rfoot{\thepage}

\title{Algebraic Topology 1}

\author{Labix}

\date{\today}
\begin{document}
\maketitle
\begin{abstract}
Algebraic topology aims to classify topological spaces by associating them with algebraic invariants. In these set of notes, we will discuss the notion of homotopic paths and the fundamental group as an algebraic invariant of spaces and discuss its properties, applications and limitations. \\~\\

We will then discuss specific maps that preserve homotopies. One such type is covering spaces and covering maps. The theory of covering spaces and the fundamental group is strongly related, with a major result establishing a correspondence between subgroups of the fundamental group and covering spaces. The deck group as automorphisms of a covering space is also helpful in identifying normal subgroups in the fundamental group. These groups represents covering spaces which attain maximal symmetry in terms of base point switching. \\~\\

Finally, we will end the discussion with CW-complexes which allows handy calculation of the fundamental group by constructing CW-complexes of a given space and apply Seifert-van Kampen theorem so that we find ourselves deformation retracting subspaces into known spaces such as circles and the Möbius band. \\~\\

\textbf{References}
\begin{itemize}
\item Notes on Algebraic Topology by Oscar Randal-Williams
\item Algebraic Topology by Allen Hatcher
\item MA3F1 Lecture Notes in University of Warwick
\item Algebraic Topology: An Introduction by W. S. Massey
\end{itemize}
\end{abstract}
\pagebreak
\tableofcontents
\pagebreak

\section{Topological Groups}
\subsection{Basic Definitions}
\begin{defn}{Topological Groups}{} Let $G$ be a group. We say that $G$ is a topological group if $G$ is also a topological space and that the following are true. 
\begin{itemize}
\item The map $l_h:G\to G$ defined by $g\mapsto hg$ is continuous for all $h\in G$
\item The map $i:G\to G$ defined by $g\mapsto g^{-1}$ is continuous
\end{itemize}
\end{defn}

\pagebreak
\section{Topological Group Actions}
\subsection{Continuous Group Actions}
In algebraic topology, we have the results of considering groups acting on spaces. We can in fact consider topological groups acting on spaces. 

\begin{defn}{Continuous Group Actions}{} Let $G$ be a topological group and $X$ a space. We say that $G$ is a continuous group action if $G$ is a group acting on $X$ such that the group action map $$\cdot:G\times X\to X$$ is continuous. 
\end{defn}

\begin{prp}{}{} Let $G$ be a continuous group action of $X$. Then for each $g\in G$, the map $\text{A}_g:X\to X$ defined by $x\mapsto g\cdot x$ is a homeomorphism. \tcbline
\begin{proof}
Every element of $g$ has an inverse $g^{-1}$ which are both continuous and are bijections on $X$. 
\end{proof}
\end{prp}

\begin{prp}{}{} Let $G$ be a topological group and $(X,\mathcal{T})$ a topological space. Then $G$ is a continuous group action on $X$ if and only if $G$ acts on $\mathcal{T}$. \tcbline
\begin{proof}
Suppose that $G$ is a continuous group action on $X$. Then for each $g\in G$, $g\cdot U=\{g\cdot x\;|\; x\in U\}$ for $U\in\mathcal{T}$ is open since $A_g$ as above is a homeomorphism. Now suppose that $G$ acts on $\mathcal{T}$. Then for each open set $U$ of $X$, $g^{-1}\cdot U$ is open. Thus $G$ is a continuous group action. 
\end{proof}
\end{prp}

In particular, some authors would assume one knows this fact, so it is always nice to see it spelled out. It is also standard to denote this action just by the element $g$ instead of $A_g$. Notice that in particular, if $G$ is a continuous group action, then there is a homomorphism $G\to\text{Homeo}(X)$. If this homomorphism is injective, then $G$ includes into $\text{Homeo}(X)$ so that $G$ is a subgroup of homeomophisms. 

\begin{defn}{Proper Group Actions}{} Let $G$ be a topological group acting continuously on a topological space $X$. The action is said to be proper if the map $G\times X\to X\times X$ defined by $$(g,x)\mapsto(x,g\cdot x)$$ is a proper map. 
\end{defn}

\subsection{Equivariant Maps}
\begin{defn}{Equivariant Maps}{} Let $G$ be a topological group and let $X,Y$ be $G$-spaces. A map $f:X\to Y$ is said to be $G$-equivariant if $$f(g\cdot x)=g\cdot f(x)$$ for all $g\in G$ and $x\in X$. 
\end{defn}

\subsection{Properly Discontinuous Group Actions}
\begin{defn}{Properly Discontinuous Group Actions}{} Let $G$ be a group acting on a space $X$. Then we say that $G$ is a properly discontinuous group action if for every compact set $K\subseteq X$, we have $$(g\cdot K)\cap K\neq\emptyset$$ for finitely many $g\in G$. 
\end{defn}

\begin{prp}{}{} Every properly discontinuous group action is a wandering action. 
\end{prp}

\begin{prp}{}{} If $G$ is a proper group action on a space $X$, then the action is properly discontinuous. 
\end{prp}

The converse is not true in general, unless we assume that $X$ is locally compact. \\~\\

Recall the notion of a covering space action. $G$ is a covering space action on $X$ if $g\cdot U\cap U\neq\emptyset$ implies $g=1$. This is also related to properly discontinuous group actions. In fact, properly discontinuous group actions are in general stronger than covering space actions. 

\begin{prp}{}{} Let $G$ be a covering space action on $X$. If $X$ is locally compact and Hausdorff, then $G$ is a properly discontinuous group action on $X$. 
\end{prp}

\pagebreak
\section{The Coset Space}
\subsection{The Coset Space}
\begin{defn}{Coset Space}{} Let $B$ be a topological group and $G$ a closed subgroup of $B$. The coset space of $B$ by $G$ is the set $$B/G=\{bG\;|\;b\in B\}$$ together with the topology in which $U\subseteq B/G$ is open $p^{-1}(U)$ is open, where $p:B\to B/G$ is the quotient homomorphism. 
\end{defn}

Note that there is also a definition of the coset space by right cosets instead of left. However it is easy to show that they are homeomorphic through the inverse map $b\mapsto b^{-1}$ for each $b\in B$. 

\begin{prp}{}{} Let $B$ be a topological group and $G$ a closed subgroup of $B$. Then the quotient map $p:B\to B/G$ is an open map. 
\end{prp}

\begin{prp}{}{} Let $B$ be a topological group and $G$ a closed subgroup of $B$. Then $B/G$ is a Hausdorff space. 
\end{prp}

\subsection{The Translation Map}
\begin{defn}{The Translation Map}{} Let $B$ be a topological group and $G$ a closed subgroup of $B$. Let $b\in B$. Define the translation map $B\times B/G\to B/G$ defined by $$b\cdot x\mapsto p(bp^{-1}(x))$$
\end{defn}

\begin{prp}{}{} $B$ is a group of transformations of $B/G$ under the above operation. Moreover, $B$ is a group of homeomorphisms of $B/G$. 
\end{prp}

\begin{prp}{}{} Let $B$ be a topological group and $G$ a closed subgroup of $B$. Let $$G_0=\bigcap_{b\in B}bGb^{-1}$$ Then $B/G_0$ acts faithfully on $B/G$. Moreover, $B/G_0$ is a group acting continuously on $B/G$. 
\end{prp}














\end{document}
