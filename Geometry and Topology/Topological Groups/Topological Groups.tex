\documentclass[a4paper]{article}

%=========================================
% Packages
%=========================================
\usepackage{mathtools}
\usepackage{amsfonts}
\usepackage{amsmath}
\usepackage{amssymb}
\usepackage{amsthm}
\usepackage[a4paper, total={6in, 8in}, margin=1in]{geometry}
\usepackage[utf8]{inputenc}
\usepackage{fancyhdr}
\usepackage[utf8]{inputenc}
\usepackage{graphicx}
\usepackage{physics}
\usepackage[listings]{tcolorbox}
\usepackage{hyperref}
\usepackage{tikz-cd}
\usepackage{adjustbox}
\usepackage{enumitem}
\usepackage[font=small,labelfont=bf]{caption}
\usepackage{subcaption}
\usepackage{wrapfig}
\usepackage{makecell}



\raggedright

\usetikzlibrary{arrows.meta}

\DeclarePairedDelimiter\ceil{\lceil}{\rceil}
\DeclarePairedDelimiter\floor{\lfloor}{\rfloor}

%=========================================
% Fonts
%=========================================
\usepackage{tgpagella}
\usepackage[T1]{fontenc}


%=========================================
% Custom Math Operators
%=========================================
\DeclareMathOperator{\adj}{adj}
\DeclareMathOperator{\im}{im}
\DeclareMathOperator{\nullity}{nullity}
\DeclareMathOperator{\sign}{sign}
\DeclareMathOperator{\dom}{dom}
\DeclareMathOperator{\lcm}{lcm}
\DeclareMathOperator{\ran}{ran}
\DeclareMathOperator{\ext}{Ext}
\DeclareMathOperator{\dist}{dist}
\DeclareMathOperator{\diam}{diam}
\DeclareMathOperator{\aut}{Aut}
\DeclareMathOperator{\inn}{Inn}
\DeclareMathOperator{\syl}{Syl}
\DeclareMathOperator{\edo}{End}
\DeclareMathOperator{\cov}{Cov}
\DeclareMathOperator{\vari}{Var}
\DeclareMathOperator{\cha}{char}
\DeclareMathOperator{\Span}{span}
\DeclareMathOperator{\ord}{ord}
\DeclareMathOperator{\res}{res}
\DeclareMathOperator{\Hom}{Hom}
\DeclareMathOperator{\Mor}{Mor}
\DeclareMathOperator{\coker}{coker}
\DeclareMathOperator{\Obj}{Obj}
\DeclareMathOperator{\id}{id}
\DeclareMathOperator{\GL}{GL}
\DeclareMathOperator*{\colim}{colim}

%=========================================
% Custom Commands (Shortcuts)
%=========================================
\newcommand{\CP}{\mathbb{CP}}
\newcommand{\GG}{\mathbb{G}}
\newcommand{\F}{\mathbb{F}}
\newcommand{\N}{\mathbb{N}}
\newcommand{\Q}{\mathbb{Q}}
\newcommand{\R}{\mathbb{R}}
\newcommand{\C}{\mathbb{C}}
\newcommand{\E}{\mathbb{E}}
\newcommand{\Prj}{\mathbb{P}}
\newcommand{\RP}{\mathbb{RP}}
\newcommand{\T}{\mathbb{T}}
\newcommand{\Z}{\mathbb{Z}}
\newcommand{\A}{\mathbb{A}}
\renewcommand{\H}{\mathbb{H}}
\newcommand{\K}{\mathbb{K}}

\newcommand{\mA}{\mathcal{A}}
\newcommand{\mB}{\mathcal{B}}
\newcommand{\mC}{\mathcal{C}}
\newcommand{\mD}{\mathcal{D}}
\newcommand{\mE}{\mathcal{E}}
\newcommand{\mF}{\mathcal{F}}
\newcommand{\mG}{\mathcal{G}}
\newcommand{\mH}{\mathcal{H}}
\newcommand{\mI}{\mathcal{I}}
\newcommand{\mJ}{\mathcal{J}}
\newcommand{\mK}{\mathcal{K}}
\newcommand{\mL}{\mathcal{L}}
\newcommand{\mM}{\mathcal{M}}
\newcommand{\mO}{\mathcal{O}}
\newcommand{\mP}{\mathcal{P}}
\newcommand{\mS}{\mathcal{S}}
\newcommand{\mT}{\mathcal{T}}
\newcommand{\mV}{\mathcal{V}}
\newcommand{\mW}{\mathcal{W}}

%=========================================
% Colours!!!
%=========================================
\definecolor{LightBlue}{HTML}{2D64A6}
\definecolor{ForestGreen}{HTML}{4BA150}
\definecolor{DarkBlue}{HTML}{000080}
\definecolor{LightPurple}{HTML}{cc99ff}
\definecolor{LightOrange}{HTML}{ffc34d}
\definecolor{Buff}{HTML}{DDAE7E}
\definecolor{Sunset}{HTML}{F2C57C}
\definecolor{Wenge}{HTML}{584B53}
\definecolor{Coolgray}{HTML}{9098CB}
\definecolor{Lavender}{HTML}{D6E3F8}
\definecolor{Glaucous}{HTML}{828BC4}
\definecolor{Mauve}{HTML}{C7A8F0}
\definecolor{Darkred}{HTML}{880808}
\definecolor{Beaver}{HTML}{9A8873}
\definecolor{UltraViolet}{HTML}{52489C}



%=========================================
% Theorem Environment
%=========================================
\tcbuselibrary{listings, theorems, breakable, skins}

\newtcbtheorem[number within = subsection]{thm}{Theorem}%
{	colback=Buff!3, 
	colframe=Buff, 
	fonttitle=\bfseries, 
	breakable, 
	enhanced jigsaw, 
	halign=left
}{thm}

\newtcbtheorem[number within=subsection, use counter from=thm]{defn}{Definition}%
{  colback=cyan!1,
    colframe=cyan!50!black,
	fonttitle=\bfseries, breakable, 
	enhanced jigsaw, 
	halign=left
}{defn}

\newtcbtheorem[number within=subsection, use counter from=thm]{axm}{Axiom}%
{	colback=red!5, 
	colframe=Darkred, 
	fonttitle=\bfseries, 
	breakable, 
	enhanced jigsaw, 
	halign=left
}{axm}

\newtcbtheorem[number within=subsection, use counter from=thm]{prp}{Proposition}%
{	colback=LightBlue!3, 
	colframe=Glaucous, 
	fonttitle=\bfseries, 
	breakable, 
	enhanced jigsaw, 
	halign=left
}{prp}

\newtcbtheorem[number within=subsection, use counter from=thm]{lmm}{Lemma}%
{	colback=LightBlue!3, 
	colframe=LightBlue!60, 
	fonttitle=\bfseries, 
	breakable, 
	enhanced jigsaw, 
	halign=left
}{lmm}

\newtcbtheorem[number within=subsection, use counter from=thm]{crl}{Corollary}%
{	colback=LightBlue!3, 
	colframe=LightBlue!60, 
	fonttitle=\bfseries, 
	breakable, 
	enhanced jigsaw, 
	halign=left
}{crl}

\newtcbtheorem[number within=subsection, use counter from=thm]{eg}{Example}%
{	colback=Beaver!5, 
	colframe=Beaver, 
	fonttitle=\bfseries, 
	breakable, 
	enhanced jigsaw, 
	halign=left
}{eg}

\newtcbtheorem[number within=subsection, use counter from=thm]{ex}{Exercise}%
{	colback=Beaver!5, 
	colframe=Beaver, 
	fonttitle=\bfseries, 
	breakable, 
	enhanced jigsaw, 
	halign=left
}{ex}

\newtcbtheorem[number within=subsection, use counter from=thm]{alg}{Algorithm}%
{	colback=UltraViolet!5, 
	colframe=UltraViolet, 
	fonttitle=\bfseries, 
	breakable, 
	enhanced jigsaw, 
	halign=left
}{alg}




%=========================================
% Hyperlinks
%=========================================
\hypersetup{
    colorlinks=true, %set true if you want colored links
    linktoc=all,     %set to all if you want both sections and subsections linked
    linkcolor=DarkBlue,  %choose some color if you want links to stand out
}


\pagestyle{fancy}
\fancyhf{}
\rhead{Labix}
\lhead{Topological Groups}
\rfoot{\thepage}

\title{Topological Groups}

\author{Labix}

\date{\today}
\begin{document}
\maketitle
\begin{abstract}
\end{abstract}
\pagebreak
\tableofcontents
\pagebreak

\section{Topological Groups and Actions}
\subsection{Basic Definitions}
\begin{defn}{Topological Groups}{} Let $G$ be a group. We say that $G$ is a topological group if $G$ is also a topological space and that the following are true. 
\begin{itemize}
\item The map $l_h:G\to G$ defined by $g\mapsto hg$ is continuous for all $h\in G$
\item The map $i:G\to G$ defined by $g\mapsto g^{-1}$ is continuous
\end{itemize}
\end{defn}

\subsection{Continuous Group Actions}
In algebraic topology, we have the results of considering groups acting on spaces. We can in fact consider topological groups acting on spaces. 

\begin{defn}{Continuous Group Actions}{} Let $G$ be a topological group and $X$ a space. We say that $G$ is a continuous group action if $G$ is a group acting on $X$ such that the group action map $$\cdot:G\times X\to X$$ is continuous. 
\end{defn}

Frequently a continuous group action is also called a (topological) transformation group, for example in Milnor's Topology of Fiber Bundles. 

\begin{prp}{}{} Let $G$ be a continuous group action of $X$. Then for each $g\in G$, the map $\text{A}_g:X\to X$ defined by $x\mapsto g\cdot x$ is a homeomorphism. \tcbline
\begin{proof}
Every element of $g$ has an inverse $g^{-1}$ which are both continuous and are bijections on $X$. 
\end{proof}
\end{prp}

\begin{prp}{}{} Let $G$ be a topological group and $(X,\mathcal{T})$ a topological space. Then $G$ is a continuous group action on $X$ if and only if $G$ acts on $\mathcal{T}$. \tcbline
\begin{proof}
Suppose that $G$ is a continuous group action on $X$. Then for each $g\in G$, $g\cdot U=\{g\cdot x\;|\; x\in U\}$ for $U\in\mathcal{T}$ is open since $A_g$ as above is a homeomorphism. Now suppose that $G$ acts on $\mathcal{T}$. Then for each open set $U$ of $X$, $g^{-1}\cdot U$ is open. Thus $G$ is a continuous group action. 
\end{proof}
\end{prp}

In particular, some authors would assume one knows this fact, so it is always nice to see it spelled out. It is also standard to denote this action just by the element $g$ instead of $A_g$. Notice that in particular, if $G$ is a continuous group action, then there is a homomorphism $G\to\text{Homeo}(X)$. If this homomorphism is injective, then $G$ includes into $\text{Homeo}(X)$ so that $G$ is a subgroup of homeomophisms. 

\begin{defn}{Group of Diffeomorphisms}{} Let $G$ be a continuous group action on a space $X$. We say that $G$ is a group of homeomorphisms of $X$ if for every $g\in G$, the map $$x\mapsto g\cdot x$$ is a homeomorphism. 
\end{defn}

\subsection{Point Set Topology of G-Spaces}
\begin{defn}{G-Spaces}{} Let $G$ be a topological group. A $G$-space is a topological space $X$ together with a continuous group action given by $G$. 
\end{defn}

\begin{defn}{G-Equivariant Maps}{} Let $G$ be a topological group and let $X,Y$ be $G$-spaces. A $G$-equivariant map is a continuous map $f:X\to Y$ such that $f$ is equivariant. In other words, we require that $$f(g\cdot x)=g\cdot f(x)$$ for all $x\in X$ and all $g\in G$. 
\end{defn}

\begin{defn}{G-Homotopy}{} Let $G$ be a topological group and let $X,Y$ be $G$-spaces. Let $f,g:X\to Y$ be $G$-equivariant maps. A $G$-homotopy from $f$ to $g$ is a homotopy $H:X\time I\to Y$ from $f$ to $g$ such that for each $t\in I$, the map $$H(-,t):X\to Y$$ is $G$-equivariant map. 
\end{defn}

\subsection{Properly Discontinuous Group Actions}
\begin{defn}{Proper Group Actions}{} Let $G$ be a topological group acting continuously on a topological space $X$. The action is said to be proper if the map $G\times X\to X\times X$ defined by $$(g,x)\mapsto(x,g\cdot x)$$ is a proper map. 
\end{defn}

\begin{defn}{Properly Discontinuous Group Actions}{} Let $G$ be a group acting on a space $X$. Then we say that $G$ is a properly discontinuous group action if for every compact set $K\subseteq X$, we have $$(g\cdot K)\cap K\neq\emptyset$$ for finitely many $g\in G$. 
\end{defn}

\begin{prp}{}{} Every properly discontinuous group action is a wandering action. 
\end{prp}

\begin{prp}{}{} If $G$ is a proper group action on a space $X$, then the action is properly discontinuous. 
\end{prp}

The converse is not true in general, unless we assume that $X$ is locally compact. \\~\\

Recall the notion of a covering space action. $G$ is a covering space action on $X$ if $g\cdot U\cap U\neq\emptyset$ implies $g=1$. This is also related to properly discontinuous group actions. In fact, properly discontinuous group actions are in general stronger than covering space actions. 

\begin{prp}{}{} Let $G$ be a covering space action on $X$. If $X$ is locally compact and Hausdorff, then $G$ is a properly discontinuous group action on $X$. 
\end{prp}

\pagebreak
\section{The Coset Space}
\subsection{The Topology of Coset Spaces}
\begin{defn}{Coset Space}{} Let $B$ be a topological group and $G$ a closed subgroup of $B$. The coset space of $B$ by $G$ is the set $$B/G=\{bG\;|\;b\in B\}$$ together with the topology in which $U\subseteq B/G$ is open $p^{-1}(U)$ is open, where $p:B\to B/G$ is the quotient homomorphism. 
\end{defn}

Note that there is also a definition of the coset space by right cosets instead of left. However it is easy to show that they are homeomorphic through the inverse map $b\mapsto b^{-1}$ for each $b\in B$. 

\begin{prp}{}{} Let $B$ be a topological group and $G$ a closed subgroup of $B$. Then the quotient map $p:B\to B/G$ is an open map. 
\end{prp}

\begin{prp}{}{} Let $B$ be a topological group and $G$ a closed subgroup of $B$. Then $B/G$ is a Hausdorff space. 
\end{prp}

\subsection{Coset Spaces of Transitive Actions}
\begin{thm}{}{} Let $G$ be a topological group acting continuously on a space $X$.  Let $x_0\in X$. Then the map $$p:\frac{G}{\text{Stab}_G(x_0)}\to Gx_0\subseteq X$$ given by $g\mapsto g\cdot x_0$ is well defined, and moreover, a homeomorphism. 
\end{thm}

\begin{crl}{}{} Let $G$ be a topological group and let $X$ be a $G$-space. If $G$ is free and transitive, then $G$ is homeomorphic to $X$. 
\end{crl}

\begin{crl}{}{} Let $G$ be a topological group acting continuously on a space $X$.  Then there exists a normal subgroup $H$ of $G$ such that $$\frac{G}{H}\cong X$$
\end{crl}

\subsection{Coset Spaces with the Group Action of the Base Space}
\begin{defn}{The Translation Map}{} Let $B$ be a topological group and $G$ a closed subgroup of $B$. Let $b\in B$ and suppose that $p:B\to B/G$ is the quotient map. Define the translation map $B\times B/G\to B/G$ defined by $$(b,x)\mapsto p(bp^{-1}(x))$$
\end{defn}

\begin{prp}{}{} Let $B$ be a topological group and $G$ a closed subgroup of $B$. Then the translation map is a continuous group action of $B$ on $B/G$. Moreover, $B$ is a group of homeomorphisms of $B/G$. 
\end{prp}

\begin{prp}{}{} Let $B$ be a topological group and $G$ a closed subgroup of $B$. Let $$G_0=\bigcap_{b\in B}bGb^{-1}$$ Then $B/G_0$ acts faithfully on $B/G$. Moreover, $B/G_0$ is a group acting continuously on $B/G$. 
\end{prp}












\end{document}
