\documentclass[a4paper]{article}

\input{C:/Users/liula/Desktop/Latex/Headers.tex}

\pagestyle{fancy}
\fancyhf{}
\rhead{Labix}
\lhead{Topological Manifolds}
\rfoot{\thepage}

\title{Topological Manifolds}

\author{Labix}

\date{\today}
\begin{document}
\maketitle
\begin{abstract}
\end{abstract}
\pagebreak
\tableofcontents
\pagebreak

\section{Topological Manifolds and Singular Homology}
\subsection{Orientability}
Recall the notion of orientation in finite dimensional vector bases. We say that two bases of a vector space have the same orientation if the change of basis matrix has determinant greater than $0$. Since topological manifolds locally look like finite-dimensional vector spaces, we expect that orientations can be generalized to manifolds. \\~\\

The key observation in defining orientation through homology is the following proposition. 

\begin{prp}{}{} Let $M$ be a $k$-dimensional topological manifold and $x\in M$ a point. Then $$H_n(M,M\setminus\{x\})\cong H_n(\R^k,\R^k\setminus\{\ast\})\cong\begin{cases}
\Z & \text{ if } n=k\\
0 & \text{ if } n\neq k
\end{cases}$$
\end{prp}

\begin{defn}{Local Orientation}{} A local orientation of $M$ at $x$ is a choice of generator of $H_k(M,M\setminus\{x\})$. 
\end{defn}

Let $U$ be a chart on a topological manifold $M$ and that $B\subseteq M$ is such that on the chart $U$, $B$ is an open / closed ball $B_r(z)$. For convention, we give a name to subsets of these type. 

\begin{defn}{Open and Closed Ball in Manifolds}{} Let $M$ be a $k$-dimensional topological manifold and $U$ a chart of $M$. We say that $B$ is an open / closed ball if under the homeomorphism of the chart $U\cong\R^k$, the image of $B$ is a ball $B_r(x)\subseteq\R^k$ for some $r\in\R^+$ and $x\in\R^k$. 
\end{defn}

Notice that the inclusion $(M,M\setminus B)\hookrightarrow(M,M\setminus\{y\})$ induces a map in homology: $$H_k(M,M\setminus B)\overset{\cong}{\rightarrow}H_k(M,M\setminus\{y\})$$ It is an isomorphism since $B$ is homeomorphic to a ball in $\R^k$ which is contractible. This leads to the following definition. 

\begin{defn}{Consistent Local Orientations}{} Let $(\omega_y)_{y\in B}$ be a family of local orientations. We say that it is consistent if there is a generator $\omega_B\in H_k(M,M\setminus B)$ such that $\omega_B\mapsto\omega_y$ for each $y\in B$ under the isomorphism $$H_k(M,M\setminus B)\cong H_k(M,M\setminus\{y\})$$
\end{defn}

With this, we can now formally define orientations in a manifold. 

\begin{defn}{Orientation of a Manifold}{} Let $M$ be a $k$-dimensional topological manifold. An orientation of $M$ is a function $x\mapsto\omega_x\in H_k(M,M\setminus\{x\})$ assigning every point to a local orientation such that for every $x\in M$, there exists $x\in B$ a subset of a chart $U$ for $B$ homeomorphic to an open / closed ball in $\R^k$, for $(\omega_x)_{x\in B}$ a consistent local orientation. 
\end{defn}

Since $H_k(M,M\setminus\{x\})$ is isomorphic to $\Z$, this means that there are only two possible choices of distinct orientation classes for each point $x\in M$. 

\begin{defn}{Orientation Bundle}{} Let $M$ be a topological manifold. Define the orientation bundle $\widetilde{M}$ to be the set of pairs $$\widetilde{M}=\left\{(x,\omega_x)|\; x\in M, \omega_x\in H_k(M,M\setminus\{x\})\right\}$$ together the projection map $\pi:\widetilde{M}\to M$ defined by $\pi(x,\omega_x)=x$ and with the topology defined as follows. \\~\\

Let $B$ be an open ball in $M$. Since there are exactly two distinct orientation classes on $B$, $\pi^{-1}=B_+\amalg B_-$. Define the topology of $\widetilde{M}$ to be generated by sets of the form $B_+$ and $B_-$. 
\end{defn}

\begin{lmm}{}{} For any topological manifold $M$, $\widetilde{M}$ is a manifold. Moreover, it is orientable with a canonical orientation. 
\end{lmm}

\begin{lmm}{}{} Giving an orientation of $M$ is equivalent to giving a continuous map $s:M\to\widetilde{M}$ such that $s\circ\pi=\text{id}$ (section of the orientation bundle). \tcbline
\begin{proof}
Let $s:M\to\widetilde{M}$ be continuous and that $s\circ\pi=\text{id}$. Then $s$ assigns a orientation $\omega_x$ to each $x\in M$. The map is continuous if and only if for each open ball in $M$ and $\pi^{-1}(B)=B_+\amalg B_-$, the preimages $s^{-1}(B_+)$ and $s^{-1}(B_-)$ are both open in $B$. Since these two preimages are disjoint and jointly cover $B$, this condition is equivalent $s(B)=B_+$ or $s(B)=B_-$. This precisely means that the local orientations are consistent. 
\end{proof}
\end{lmm}

\begin{crl}{}{} Let $M$ be a connected topological manifold. Then exactly one of the following holds: 
\begin{itemize}
\item $\widetilde{M}\to M$ is a non-trivial $2$-sheeted cover and $M$ is non-orientable
\item $\widetilde{M}\cong M\amalg M$ and $M$ admits precisely two orientations
\end{itemize}
\end{crl}

\begin{crl}{}{} Any simply connected manifold is orientable. 
\end{crl}

\subsection{Fundamental Class}
\begin{prp}{}{} Let $M$ be a connected compact smooth manifold of dimension $n$. If $M$ is orientable then $H_n(M)\cong\Z$. Otherwise $H_n(M)=0$. 
\end{prp}

\begin{defn}{Fundamental Class}{} Let $M$ be a connected compact orientable smooth manifold of dimension $n$. A fundamental class for $M$ is a generator for the top homology $$H_n(M)\cong\Z$$
\end{defn}

Recall that $S^k$ and $\partial\Delta^{k+1}$ are homeomorphic. 

\begin{prp}{}{} The cycle $\partial\Delta^{k+1}\in C_k(\partial\Delta^{k+1})$ represents a generator in for the top homology of $S^k$. 
\end{prp}

\begin{crl}{}{} Let $S_+^k$ and $S_-^k$ be the northern and southern hemisphere of $S^k$ respectively. Choose homomorphisms $$\sigma_+:\Delta^k\overset{\cong}{\longrightarrow} S_+^k\;\;\text{ and }\;\;\sigma_-:\Delta^k\overset{\cong}{\longrightarrow} S_-^k$$ such that both $\sigma_+,\sigma_-$ map the boundary $\partial\Delta^k$ homeomorphically onto the equator $S_+^k\cap S_-^k$ and the composition $$\partial\Delta^k\overset{\sigma_+}{\longrightarrow}S_+^k\cap S_-^k\overset{(\sigma_-)^{-1}}{\longrightarrow}S_-^k$$ is the identity. Then the cycle $\sigma_+-\sigma_-\in C_k(S^k)$ represents a fundamental class for $S^k$. 
\end{crl}

\pagebreak
\section{The Theory of Surfaces}
\subsection{The Homology of Surfaces}
Recall that a compact surface is a connected topological manifold of dimension $2$ that is compact. Moreover, every compact surface is homeomorphic to either $\Sigma_g=\T\#\cdots\#\T$ for $g\geq 0$ or $N_h=\R\Prj^2\#\cdots\#\R\Prj^2$ for $h\geq 1$. We can now compute the homology groups of these surfaces and moreover, show that $\sum_g$ is orientable while $N_h$ is not. 

\begin{prp}{}{} Let $g\geq 0$. The homology of the $g$-holed torus $\Sigma_g$ is given by $$H_n(\Sigma_g)=\begin{cases}
\Z & \text{ if } n=0,2\\
\Z^{2g} & \text{ if } n=1\\
0 & \text{otherwise}
\end{cases}$$
\end{prp}

\begin{crl}{}{} The surfaces $\Sigma_g$ for $g\geq 0$ is orientable. 
\end{crl}

\begin{prp}{}{} Let $h\geq 1$. The homology of $N_h$ is given by $$H_n(N_h)=\begin{cases}
\Z & \text{ if } n=0\\
\Z^{h-1}\oplus\Z/2\Z & \text{ if } n=1\\
0 & \text{otherwise}
\end{cases}$$
\end{prp}

\begin{crl}{}{} The surfaces $N_h$ for $h\geq 1$ is non-orientable. 
\end{crl}

\subsection{The Euler Characteristic}


\pagebreak
\section{Homology and Cohomology on Manifolds}
\subsection{de Rham Cohomology}
\begin{prp}{}{} Let $M$ be a smooth manifold. Then differential forms of $M$, $\Omega^0(M),\dots,\Omega^n(M),\dots$ together with the exterior derivative $d:\Omega^n(M)\to\Omega^{n+1}(M)$ form a cochain complex. 
\end{prp}

\begin{defn}{}{} Let $M$ be a smooth manifold. Define the de Rham cohomology groups of $M$ to be the cohomology of the chain of differential forms: $$H_{\text{dR}}^n(M;\R)=H^n(\Omega^\bullet(M);\R)$$
\end{defn}

\begin{prp}{}{} Let $M$ be a smooth manifold of dimension $n$. Then the following are true for the de Rham cohomology of $M$. 
\begin{itemize}
\item $H_{\text{dR}}^k(M;\R)$ is a vector space over $\R$ for all $k\in\N$. 
\item For $r>n$ we have $H_{\text{dR}}^r(M;\R)=0$
\item If $M$ has $m$ connected components then $H_{\text{dR}}^0(M;\R)=\R^k$
\end{itemize}
\end{prp}

\begin{thm}{}{} Let $M$ be a smooth manifold of dimension $n$. Then the direct sum $$H^\ast(M)=\bigoplus_{k=1}^nH_{\text{dR}}^k(M;\R)$$ is an $\R$-algebra where multiplication defined by $a\wedge b\in H_{\text{dR}}^{s+l}(M;\R)$ for $a\in H_{\text{dR}}^s(M;\R)$ and $b\in H_{\text{dR}}^l(M;\R)$. Moreover, this multiplication is anti-commutative, namely for $a\in H_{\text{dR}}^s(M;\R)$ and $b\in H_{\text{dR}}^l(M;\R)$, we have $$a\wedge b=(-1)^{sl}b\wedge a$$
\end{thm}

\begin{prp}{}{} Let $M,N$ be smooth manifolds and $f:M\to N$ a smooth map. Then $f$ induces an $\R$-linear map $$f^\ast:H^\ast(N)\to H^\ast(M)$$ such that $f^\ast(a\wedge b)=f^\ast(a)\wedge f^\ast(b)$. Moreover, it is functorial: 
\begin{itemize}
\item If $g:N\to K$ is another smooth map of manifolds, then $(g\circ f)^\ast=f^\ast\circ g^\ast$
\item If $\text{id}:M\to M$ is the identity map on the manifold, then $\text{id}^\ast:H^\ast(M)\to H^\ast(M)$ is the trivial map on $\R$-algebras. 
\end{itemize}
\end{prp}

\begin{thm}{Homotopy Invariance of de Rham Cohomology}{} Let $f:M\times I\to N$ be a smooth map of manifolds varying for each $t\in I=[0,1]$. Write $f_t(x)=f(x,t)$. Then the pull back maps $f_0^\ast,f_1^\ast:H^\ast(N)\to H^\ast(M)$ are equal: $$f_0^\ast=f_1^\ast$$
\end{thm}

\subsection{de Rham Cohomology of Common Manifolds}
\begin{prp}{}{} The real space $\R^n$ has the de Rham cohomology $$H_{\text{dR}}^k(\R^n)=\begin{cases}
\R & \text{ if }k=0\\
0 & \text{ otherwise }
\end{cases}$$
\end{prp}

\begin{prp}{}{} The $n$-sphere $S^n$ has the de Rham cohomology $$H_{\text{dR}}^k(S^n)=\begin{cases}
\R & \text{ if }k=0,n\\
0 & \text{ otherwise }
\end{cases}$$
\end{prp}

\begin{thm}{}{} Let $p,q\geq 1$, the sphere $S^{p+q}$ is not diffeomorphic to any $M\times N$ manifolds where $\dim(M)=p$ and $\dim(N)=q$. 
\end{thm}

\begin{prp}{}{} Every smooth vector fields on $S^{2n}$ vanishes at some point of the sphere. 
\end{prp}

\begin{prp}{}{} The real projective space $\R\Prj^n$ has the de Rham cohomology $$H_{\text{dR}}^k(\R\Prj^n)=\begin{cases}
\R & \text{ if }k=0\text{ or }k=n\text{ where }n\text{ odd }\\
0 & \text{ otherwise }
\end{cases}$$
\end{prp}

\pagebreak
\section{Poincare Duality}
\subsection{The Cap Product}
\begin{defn}{The Cap Product}{} Let $\sigma=[v_0,\dots,v_k]\in C_k(X)$ and $\phi\in C^l(X)$ where $k\geq l$ with coefficients in a ring $R$. Define the cap product to be $$\sigma\frown\phi=\phi(\sigma|_{[v_0,\dots,v_l]})\sigma|_{[v_l,\dots,v_k]}\in C_{k-l}(X)$$
\end{defn}

\begin{lmm}{}{} The cap product $\frown: C_k(X)\times C^l(X)\to C_{k-l}(X)$ with coefficients in a ring $R$ induces a cap product in homology $\frown: H_k(X)\times H^l(X,R)\to H_{k-l}(X)$ for $k\geq l$. 
\end{lmm}

\subsection{Cohomology with Compact Support}

\subsection{The Duality Theorem}
\begin{thm}{Poincare Duality}{} Let $M$ be a compact and oriented topological $n$-manifold. Then the homomorphism $$D:H^p(M)\to H_{n-p}(M)$$ is an isomorphism. 
\end{thm}
















\end{document}
