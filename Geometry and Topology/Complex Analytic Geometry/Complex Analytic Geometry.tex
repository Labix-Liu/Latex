\documentclass[a4paper]{article}

%=========================================
% Packages
%=========================================
\usepackage{mathtools}
\usepackage{amsfonts}
\usepackage{amsmath}
\usepackage{amssymb}
\usepackage{amsthm}
\usepackage[a4paper, total={6in, 8in}, margin=1in]{geometry}
\usepackage[utf8]{inputenc}
\usepackage{fancyhdr}
\usepackage[utf8]{inputenc}
\usepackage{graphicx}
\usepackage{physics}
\usepackage[listings]{tcolorbox}
\usepackage{hyperref}
\usepackage{tikz-cd}
\usepackage{adjustbox}
\usepackage{enumitem}
\usepackage[font=small,labelfont=bf]{caption}
\usepackage{subcaption}
\usepackage{wrapfig}
\usepackage{makecell}



\raggedright

\usetikzlibrary{arrows.meta}

\DeclarePairedDelimiter\ceil{\lceil}{\rceil}
\DeclarePairedDelimiter\floor{\lfloor}{\rfloor}

%=========================================
% Fonts
%=========================================
\usepackage{tgpagella}
\usepackage[T1]{fontenc}


%=========================================
% Custom Math Operators
%=========================================
\DeclareMathOperator{\adj}{adj}
\DeclareMathOperator{\im}{im}
\DeclareMathOperator{\nullity}{nullity}
\DeclareMathOperator{\sign}{sign}
\DeclareMathOperator{\dom}{dom}
\DeclareMathOperator{\lcm}{lcm}
\DeclareMathOperator{\ran}{ran}
\DeclareMathOperator{\ext}{Ext}
\DeclareMathOperator{\dist}{dist}
\DeclareMathOperator{\diam}{diam}
\DeclareMathOperator{\aut}{Aut}
\DeclareMathOperator{\inn}{Inn}
\DeclareMathOperator{\syl}{Syl}
\DeclareMathOperator{\edo}{End}
\DeclareMathOperator{\cov}{Cov}
\DeclareMathOperator{\vari}{Var}
\DeclareMathOperator{\cha}{char}
\DeclareMathOperator{\Span}{span}
\DeclareMathOperator{\ord}{ord}
\DeclareMathOperator{\res}{res}
\DeclareMathOperator{\Hom}{Hom}
\DeclareMathOperator{\Mor}{Mor}
\DeclareMathOperator{\coker}{coker}
\DeclareMathOperator{\Obj}{Obj}
\DeclareMathOperator{\id}{id}
\DeclareMathOperator{\GL}{GL}
\DeclareMathOperator*{\colim}{colim}

%=========================================
% Custom Commands (Shortcuts)
%=========================================
\newcommand{\CP}{\mathbb{CP}}
\newcommand{\GG}{\mathbb{G}}
\newcommand{\F}{\mathbb{F}}
\newcommand{\N}{\mathbb{N}}
\newcommand{\Q}{\mathbb{Q}}
\newcommand{\R}{\mathbb{R}}
\newcommand{\C}{\mathbb{C}}
\newcommand{\E}{\mathbb{E}}
\newcommand{\Prj}{\mathbb{P}}
\newcommand{\RP}{\mathbb{RP}}
\newcommand{\T}{\mathbb{T}}
\newcommand{\Z}{\mathbb{Z}}
\newcommand{\A}{\mathbb{A}}
\renewcommand{\H}{\mathbb{H}}
\newcommand{\K}{\mathbb{K}}

\newcommand{\mA}{\mathcal{A}}
\newcommand{\mB}{\mathcal{B}}
\newcommand{\mC}{\mathcal{C}}
\newcommand{\mD}{\mathcal{D}}
\newcommand{\mE}{\mathcal{E}}
\newcommand{\mF}{\mathcal{F}}
\newcommand{\mG}{\mathcal{G}}
\newcommand{\mH}{\mathcal{H}}
\newcommand{\mI}{\mathcal{I}}
\newcommand{\mJ}{\mathcal{J}}
\newcommand{\mK}{\mathcal{K}}
\newcommand{\mL}{\mathcal{L}}
\newcommand{\mM}{\mathcal{M}}
\newcommand{\mO}{\mathcal{O}}
\newcommand{\mP}{\mathcal{P}}
\newcommand{\mS}{\mathcal{S}}
\newcommand{\mT}{\mathcal{T}}
\newcommand{\mV}{\mathcal{V}}
\newcommand{\mW}{\mathcal{W}}

%=========================================
% Colours!!!
%=========================================
\definecolor{LightBlue}{HTML}{2D64A6}
\definecolor{ForestGreen}{HTML}{4BA150}
\definecolor{DarkBlue}{HTML}{000080}
\definecolor{LightPurple}{HTML}{cc99ff}
\definecolor{LightOrange}{HTML}{ffc34d}
\definecolor{Buff}{HTML}{DDAE7E}
\definecolor{Sunset}{HTML}{F2C57C}
\definecolor{Wenge}{HTML}{584B53}
\definecolor{Coolgray}{HTML}{9098CB}
\definecolor{Lavender}{HTML}{D6E3F8}
\definecolor{Glaucous}{HTML}{828BC4}
\definecolor{Mauve}{HTML}{C7A8F0}
\definecolor{Darkred}{HTML}{880808}
\definecolor{Beaver}{HTML}{9A8873}
\definecolor{UltraViolet}{HTML}{52489C}



%=========================================
% Theorem Environment
%=========================================
\tcbuselibrary{listings, theorems, breakable, skins}

\newtcbtheorem[number within = subsection]{thm}{Theorem}%
{	colback=Buff!3, 
	colframe=Buff, 
	fonttitle=\bfseries, 
	breakable, 
	enhanced jigsaw, 
	halign=left
}{thm}

\newtcbtheorem[number within=subsection, use counter from=thm]{defn}{Definition}%
{  colback=cyan!1,
    colframe=cyan!50!black,
	fonttitle=\bfseries, breakable, 
	enhanced jigsaw, 
	halign=left
}{defn}

\newtcbtheorem[number within=subsection, use counter from=thm]{axm}{Axiom}%
{	colback=red!5, 
	colframe=Darkred, 
	fonttitle=\bfseries, 
	breakable, 
	enhanced jigsaw, 
	halign=left
}{axm}

\newtcbtheorem[number within=subsection, use counter from=thm]{prp}{Proposition}%
{	colback=LightBlue!3, 
	colframe=Glaucous, 
	fonttitle=\bfseries, 
	breakable, 
	enhanced jigsaw, 
	halign=left
}{prp}

\newtcbtheorem[number within=subsection, use counter from=thm]{lmm}{Lemma}%
{	colback=LightBlue!3, 
	colframe=LightBlue!60, 
	fonttitle=\bfseries, 
	breakable, 
	enhanced jigsaw, 
	halign=left
}{lmm}

\newtcbtheorem[number within=subsection, use counter from=thm]{crl}{Corollary}%
{	colback=LightBlue!3, 
	colframe=LightBlue!60, 
	fonttitle=\bfseries, 
	breakable, 
	enhanced jigsaw, 
	halign=left
}{crl}

\newtcbtheorem[number within=subsection, use counter from=thm]{eg}{Example}%
{	colback=Beaver!5, 
	colframe=Beaver, 
	fonttitle=\bfseries, 
	breakable, 
	enhanced jigsaw, 
	halign=left
}{eg}

\newtcbtheorem[number within=subsection, use counter from=thm]{ex}{Exercise}%
{	colback=Beaver!5, 
	colframe=Beaver, 
	fonttitle=\bfseries, 
	breakable, 
	enhanced jigsaw, 
	halign=left
}{ex}

\newtcbtheorem[number within=subsection, use counter from=thm]{alg}{Algorithm}%
{	colback=UltraViolet!5, 
	colframe=UltraViolet, 
	fonttitle=\bfseries, 
	breakable, 
	enhanced jigsaw, 
	halign=left
}{alg}




%=========================================
% Hyperlinks
%=========================================
\hypersetup{
    colorlinks=true, %set true if you want colored links
    linktoc=all,     %set to all if you want both sections and subsections linked
    linkcolor=DarkBlue,  %choose some color if you want links to stand out
}


\pagestyle{fancy}
\fancyhf{}
\rhead{Labix}
\lhead{Complex Analytic Geometry}
\rfoot{\thepage}

\title{Complex Analytic Geometry}

\author{Labix}

\date{\today}
\begin{document}
\maketitle
\begin{abstract}
\end{abstract}
\pagebreak
\tableofcontents

\pagebreak
\section{Complex Analytic Space}
\subsection{Analytic Manifolds}
To be moved: every $C^k$ manifold for $k\geq 1$ has a compatible smooth structure (non-unique) for $k\geq 4$. Every smooth manifold admits a compatible analytic structure. 

\begin{defn}{Analytic Manifolds}{} An analytic manifold is a topological manifold with analytic transition maps. In other words, the pesudogroup of transformations are analytic. 
\end{defn}

In the real case, every analytic manifold is thus a differentiable manifold because every analytic function is necessarily infinitely differentiable. However, since not every differentiable function is analytic, not every differentiable manifold is analytic. However, in the complex case, holomorphic functions are precisely the analytic ones hence there are no actually difference between analytic manifolds that are complex and complex manifolds. 

\begin{defn}{Sheaf of Analytic Functions}{} Let $M$ be an analytic manifold. Define the sheaf of analytic functions $$\mA_M:\bold{Open}\to\bold{Ring}$$ as follows. 
\begin{itemize}
\item For each open set $U\subseteq M$, $\mA_M(U)$ consists of the ring of analytic functions on $U$
\item For each inclusion $V\subseteq U$, there is a unique morphism $\mA_M(U)\to\mA_M(V)$ given by restricting the functions from $U$ to $V$
\end{itemize}
\end{defn}

Again in the complex case, $\mA_M=\mO_M$ for any complex manifold $M$. Recall that a domain $U$ of $\C^n$ is an open and connected subset of $\C^n$. 

\begin{thm}{Oka's (Coherence) Theorem}{} Let $M$ be a complex manifold. Then the sheaf $\mO_M$ of holomorphic functions on $M$ is a coherent sheaf. 
\end{thm}

\subsection{Analytic Subsets}
\begin{defn}{Analytic Subset}{} Let $M$ be an (real / complex) analytic manifold. A subset $A\subseteq M$ is said to be (real / complex) analytic subset of $M$ if for all $x\in A$, there exists a neighbourhood $U$ of $x$ and (real / complex) analytic functions $f_1,\dots,f_r\in\mA_M(U)$ such that $$A\cap U=V(f_1,\dots,f_r)$$
\end{defn}

\begin{defn}{Sheaf of Ideals of Analytic Subsets}{} Let $M$ be a (real / complex) analytic manifold and let $A$ be an analytic subset of $M$. Define the sheaf of ideals of $A$ to be the sheaf $$\mI_A:\bold{Open}(A)\to\bold{Ring}$$ with the following data. 
\begin{itemize}
\item Each open set $U$ is sent to $\mI_A(U)=I(U)$
\item For each inclusion $V\subseteq U$, there is a unique ring homomorphism $I(U)\to I(V)$ defined by the restriction of functions. 
\end{itemize}
\end{defn}

\begin{prp}{}{} Let $M$ be a complex manifold and let $A$ be an analytic subset of $M$. Then $\mI_A$ is a subsheaf of $\mO_M$. 
\end{prp}

\begin{thm}{Cartan's (Coherence) Theorem}{} Let $M$ be a complex manifold and let $A$ be an analytic subset of $M$. Then $\mI_A$ is a coherent sheaf of $\mO_M$-modules. 
\end{thm}

\subsection{Regular and Singular Points}
\begin{defn}{Regular and Singular Points}{} Let $M$ be a complex manifold and let $A$ be an analytic subset of $M$. We say that $x\in A$ is a regular point if there exists some open neighbourhood $U$ of $x$ such that $A\cap U$ is a complex submanifold of $M$. Otherwise, $x$ is said to be singular. Denote $$A_\text{reg}=\{x\in A\;|\;x\text{ is a regular point }\}\;\;\;\;\text{ and }\;\;\;\; A_\text{sing}=\{x\in A\;|\;x\text{ is a singular point }\}$$
\end{defn}

\begin{thm}{}{} Let $M$ be a complex manifold and let $A$ be an analytic subset of $M$. Then $A_\text{sing}$ is an analytic subset of $A$. 
\end{thm}

\subsection{Complex Analytic Space}
\begin{defn}{Local Model}{} A local model of $\C^n$ is a ringed space of the form $$(X,\mO_X)$$ where $X\subseteq\C^n$ and $\mO_X:\bold{Open}(X)\to\bold{Ring}$ are obtained as follows. 
\begin{itemize}
\item There is some $U\subseteq\C^n$ and $f_1,\dots,f_r\in\mO_U(U)$ for which $X=V(f_1,\dots,f_r)$
\item If $j:X\to U$ is the inclusion, then $$\mO_X=\mO_X=j^{-1}\left(\left(\frac{\mO_U}{\mI}\right)^+\right)$$ where $\mI$ is the subsheaf of $\mO_U$ generated by $f_1,\dots,f_r)$. This means that $\mI:\bold{Open}(U)\to\bold{Ring}$ is defined by $W\mapsto(f_1|_W,\dots,f_r|_W)\subseteq\mO_U(W)$
\end{itemize}
\end{defn}

\begin{defn}{Complex Analytic Space}{} A complex analytic space is a ringed space $(X,\mO_X)$ such that for all $x\in X$, there exists a neighbourhood $U$ of $x$ such that $(U,\mO_X|_U)$ is isomorphic to some $(A,\mI_A)$ where $A$ is an analytic set and $\mI_A$ is the sheaf of ideals of $A$. 
\end{defn}

\begin{thm}{}{} Let $X$ be a complex space. Then $X_\text{reg}$ is dense and open in $X$. Moreover, $X_\text{reg}$ consists of a disjoin union of complex manifolds. 
\end{thm}

\begin{thm}{}{} Let $X$ be an irreducible complex space. Then every non-constant holomorphic function $f:X\to\C$ is an open map. 
\end{thm}

\begin{crl}{}{} Let $X$ be an irreducible compact complex space. Then every holomorphic function $f:X\to\C$ on $X$ is constant. 
\end{crl}






\end{document}