\documentclass[a4paper]{article}

%=========================================
% Packages
%=========================================
\usepackage{mathtools}
\usepackage{amsfonts}
\usepackage{amsmath}
\usepackage{amssymb}
\usepackage{amsthm}
\usepackage[a4paper, total={6in, 8in}, margin=1in]{geometry}
\usepackage[utf8]{inputenc}
\usepackage{fancyhdr}
\usepackage[utf8]{inputenc}
\usepackage{graphicx}
\usepackage{physics}
\usepackage[listings]{tcolorbox}
\usepackage{hyperref}
\usepackage{tikz-cd}
\usepackage{adjustbox}
\usepackage{enumitem}
\usepackage[font=small,labelfont=bf]{caption}
\usepackage{subcaption}
\usepackage{wrapfig}
\usepackage{makecell}



\raggedright

\usetikzlibrary{arrows.meta}

\DeclarePairedDelimiter\ceil{\lceil}{\rceil}
\DeclarePairedDelimiter\floor{\lfloor}{\rfloor}

%=========================================
% Fonts
%=========================================
\usepackage{tgpagella}
\usepackage[T1]{fontenc}


%=========================================
% Custom Math Operators
%=========================================
\DeclareMathOperator{\adj}{adj}
\DeclareMathOperator{\im}{im}
\DeclareMathOperator{\nullity}{nullity}
\DeclareMathOperator{\sign}{sign}
\DeclareMathOperator{\dom}{dom}
\DeclareMathOperator{\lcm}{lcm}
\DeclareMathOperator{\ran}{ran}
\DeclareMathOperator{\ext}{Ext}
\DeclareMathOperator{\dist}{dist}
\DeclareMathOperator{\diam}{diam}
\DeclareMathOperator{\aut}{Aut}
\DeclareMathOperator{\inn}{Inn}
\DeclareMathOperator{\syl}{Syl}
\DeclareMathOperator{\edo}{End}
\DeclareMathOperator{\cov}{Cov}
\DeclareMathOperator{\vari}{Var}
\DeclareMathOperator{\cha}{char}
\DeclareMathOperator{\Span}{span}
\DeclareMathOperator{\ord}{ord}
\DeclareMathOperator{\res}{res}
\DeclareMathOperator{\Hom}{Hom}
\DeclareMathOperator{\Mor}{Mor}
\DeclareMathOperator{\coker}{coker}
\DeclareMathOperator{\Obj}{Obj}
\DeclareMathOperator{\id}{id}
\DeclareMathOperator{\GL}{GL}
\DeclareMathOperator*{\colim}{colim}

%=========================================
% Custom Commands (Shortcuts)
%=========================================
\newcommand{\CP}{\mathbb{CP}}
\newcommand{\GG}{\mathbb{G}}
\newcommand{\F}{\mathbb{F}}
\newcommand{\N}{\mathbb{N}}
\newcommand{\Q}{\mathbb{Q}}
\newcommand{\R}{\mathbb{R}}
\newcommand{\C}{\mathbb{C}}
\newcommand{\E}{\mathbb{E}}
\newcommand{\Prj}{\mathbb{P}}
\newcommand{\RP}{\mathbb{RP}}
\newcommand{\T}{\mathbb{T}}
\newcommand{\Z}{\mathbb{Z}}
\newcommand{\A}{\mathbb{A}}
\renewcommand{\H}{\mathbb{H}}
\newcommand{\K}{\mathbb{K}}

\newcommand{\mA}{\mathcal{A}}
\newcommand{\mB}{\mathcal{B}}
\newcommand{\mC}{\mathcal{C}}
\newcommand{\mD}{\mathcal{D}}
\newcommand{\mE}{\mathcal{E}}
\newcommand{\mF}{\mathcal{F}}
\newcommand{\mG}{\mathcal{G}}
\newcommand{\mH}{\mathcal{H}}
\newcommand{\mI}{\mathcal{I}}
\newcommand{\mJ}{\mathcal{J}}
\newcommand{\mK}{\mathcal{K}}
\newcommand{\mL}{\mathcal{L}}
\newcommand{\mM}{\mathcal{M}}
\newcommand{\mO}{\mathcal{O}}
\newcommand{\mP}{\mathcal{P}}
\newcommand{\mS}{\mathcal{S}}
\newcommand{\mT}{\mathcal{T}}
\newcommand{\mV}{\mathcal{V}}
\newcommand{\mW}{\mathcal{W}}

%=========================================
% Colours!!!
%=========================================
\definecolor{LightBlue}{HTML}{2D64A6}
\definecolor{ForestGreen}{HTML}{4BA150}
\definecolor{DarkBlue}{HTML}{000080}
\definecolor{LightPurple}{HTML}{cc99ff}
\definecolor{LightOrange}{HTML}{ffc34d}
\definecolor{Buff}{HTML}{DDAE7E}
\definecolor{Sunset}{HTML}{F2C57C}
\definecolor{Wenge}{HTML}{584B53}
\definecolor{Coolgray}{HTML}{9098CB}
\definecolor{Lavender}{HTML}{D6E3F8}
\definecolor{Glaucous}{HTML}{828BC4}
\definecolor{Mauve}{HTML}{C7A8F0}
\definecolor{Darkred}{HTML}{880808}
\definecolor{Beaver}{HTML}{9A8873}
\definecolor{UltraViolet}{HTML}{52489C}



%=========================================
% Theorem Environment
%=========================================
\tcbuselibrary{listings, theorems, breakable, skins}

\newtcbtheorem[number within = subsection]{thm}{Theorem}%
{	colback=Buff!3, 
	colframe=Buff, 
	fonttitle=\bfseries, 
	breakable, 
	enhanced jigsaw, 
	halign=left
}{thm}

\newtcbtheorem[number within=subsection, use counter from=thm]{defn}{Definition}%
{  colback=cyan!1,
    colframe=cyan!50!black,
	fonttitle=\bfseries, breakable, 
	enhanced jigsaw, 
	halign=left
}{defn}

\newtcbtheorem[number within=subsection, use counter from=thm]{axm}{Axiom}%
{	colback=red!5, 
	colframe=Darkred, 
	fonttitle=\bfseries, 
	breakable, 
	enhanced jigsaw, 
	halign=left
}{axm}

\newtcbtheorem[number within=subsection, use counter from=thm]{prp}{Proposition}%
{	colback=LightBlue!3, 
	colframe=Glaucous, 
	fonttitle=\bfseries, 
	breakable, 
	enhanced jigsaw, 
	halign=left
}{prp}

\newtcbtheorem[number within=subsection, use counter from=thm]{lmm}{Lemma}%
{	colback=LightBlue!3, 
	colframe=LightBlue!60, 
	fonttitle=\bfseries, 
	breakable, 
	enhanced jigsaw, 
	halign=left
}{lmm}

\newtcbtheorem[number within=subsection, use counter from=thm]{crl}{Corollary}%
{	colback=LightBlue!3, 
	colframe=LightBlue!60, 
	fonttitle=\bfseries, 
	breakable, 
	enhanced jigsaw, 
	halign=left
}{crl}

\newtcbtheorem[number within=subsection, use counter from=thm]{eg}{Example}%
{	colback=Beaver!5, 
	colframe=Beaver, 
	fonttitle=\bfseries, 
	breakable, 
	enhanced jigsaw, 
	halign=left
}{eg}

\newtcbtheorem[number within=subsection, use counter from=thm]{ex}{Exercise}%
{	colback=Beaver!5, 
	colframe=Beaver, 
	fonttitle=\bfseries, 
	breakable, 
	enhanced jigsaw, 
	halign=left
}{ex}

\newtcbtheorem[number within=subsection, use counter from=thm]{alg}{Algorithm}%
{	colback=UltraViolet!5, 
	colframe=UltraViolet, 
	fonttitle=\bfseries, 
	breakable, 
	enhanced jigsaw, 
	halign=left
}{alg}




%=========================================
% Hyperlinks
%=========================================
\hypersetup{
    colorlinks=true, %set true if you want colored links
    linktoc=all,     %set to all if you want both sections and subsections linked
    linkcolor=DarkBlue,  %choose some color if you want links to stand out
}


\pagestyle{fancy}
\fancyhf{}
\rhead{Labix}
\lhead{Transcendental Algebraic Geometry}
\rfoot{\thepage}

\title{Transcendental Algebraic Geometry}

\author{Labix}

\date{\today}
\begin{document}
\maketitle
\begin{abstract}
\end{abstract}
\pagebreak
\tableofcontents

\pagebreak

\section{Analytification of a Variety}
\subsection{The Set of Closed Points of a Scheme}
Recall that a point $x\in X$ of a space is said to be closed $\{x\}$ is a closed set. 

\begin{defn}{Closed Points of a Variety}{} Let $X$ be a variety over $\C$. Denote its set of closed points by $$X(\C)=\{x\in X\;|\;x\text{ is a closed point}\}$$
\end{defn}

\begin{defn}{Subspace Topology on Closed Points}{} Let $X$ be a variety over $\C$. Denote $$\text{Max}(X)$$ the set $X(\C)$ together with the subspace topology inherited from $X$. If $X=\text{Spec}(R)$ for some ring $R$, then we simply write $\text{maxSpec}(R)=\text{Max}(\text{Spec}(R))$. 
\end{defn}

Note: For a ring $R$, $X=\text{Spec}(R)$, then $\text{Max}(X)=\text{maxSpec}(R)$ because the closed points are precisely the maximal ideals. Moreover, the Zariski topology of $\text{maxSpec}(R)$ coincides with the subspace topology of $\text{Max}(X)$. \\~\\

We will first investigate for when $X$ is affine, before moving on to the general theory of schemes. Therefore much of the following section, we will be working with $X=\text{Spec}(R)$ for some $R$ a finitely generated $\C$-algebra. 

\begin{thm}{}{} Let $R$ be a finitely generated $\C$-algebra. Then there is a natural bijection $$\text{maxSpec}(R)=\left\{\substack{\text{Closed points}\\\text{in Spec}(R)}\right\}\;\;\;\;\overset{1:1}{\longleftrightarrow}\;\;\;\;\left\{\substack{\C\text{-algebra homomorphisms}\\\varphi:R\to\C}\right\}$$ The forward map sends $x\in\text{Spec}(R)$ to the unique $\varphi$ whose kernel is $(x)$. The backward map sends $\varphi:R\to\C$ to the image of $\varphi^\ast:\text{Spec}(\C)\to\text{Spec}(R)$. 
\end{thm}

Now we pair it up with the natural bijection between $\C$-algebra homomorphisms $\varphi:R\to\C$ and morphisms of locally ringed spaces $$(\varphi^\ast,\varphi^\#):(\text{Spec}(\C),\mO_{\text{Spec}(\C)})\to(\text{Spec}(R),\mO_{\text{Spec}(R)})$$ In fact, we can do one step further by starting with an arbitrary scheme $(X,\mO_X)$ locally of finite type over $\C$. 

\begin{prp}{}{} 
\end{prp}

\begin{prp}{}{} Let $(\Psi,\Psi^\#):(X,\mO_X)\to(Y,\mO_Y)$ be a morphisms of schemes that is locally of finite type over $\C$. Then the continuous map $\Psi:X\to Y$ takes the subspace $\text{Max}(X)$ to the subspace $\text{Max}(Y)$. 
\end{prp}

\begin{defn}{Max Map}{} Let $(\Psi,\Psi^\#):(X,\mO_X)\to(Y,\mO_Y)$ be a morphisms of schemes that is locally of finite type over $\C$. Define the induced map of closed points by $$\text{Max}(\Psi):\text{Max}(X)\to\text{Max}(Y)$$
\end{defn}

\begin{prp}{}{} Let $\theta:R\to S$ be a surjective map of finitely generated $\C$-algerbas. Then the map $$\text{maxSpec}(\theta):\text{maxSpec}(S)\to\text{maxSpec}(R)$$ embeds $\text{maxSpec}(S)$ homeomorphically into a subspace of $\text{maxSpec}(R)$. The image is identified with the set of all $\varphi:R\to\C$ such that $\varphi(\ker(\theta))=0$
\end{prp}

\subsection{Complex Topology on Spec}
\begin{lmm}{}{} There is a bijective correspondence $$\C^n\;\;\;\;\overset{1:1}{\longleftrightarrow}\;\;\;\;\left\{\substack{\C\text{-algebra homomorphisms}\\\varphi:\C[x_1,\dots,x_n]\to\C}\right\}$$ The forward map sends $a=(a_1,\dots,a_n)$ to the map $\varphi_a:\C[x_1,\dots,x_n]\to\C$ defined by $f\mapsto f(a_1,\dots,a_n)$. The backward map sends $\varphi:\C[x_1,\dots,x_n]\to\C$ to $(\varphi(x_1),\dots,\varphi(x_n))$. 
\end{lmm}

For the finitely generated $\C$-algebra $\C[x_1,\dots,x_n]$, we now have a series of correspondences $$\text{maxSpec}(\C[x_1,\dots,x_n])=\left\{\substack{\text{Closed points}\\\text{in Spec}(\C[x_1,\dots,x_n])}\right\}\;\;\;\;\overset{1:1}{\longleftrightarrow}\;\;\;\;\left\{\substack{\C\text{-algebra homomorphisms}\\\varphi:\C[x_1,\dots,x_n]\to\C}\right\}\;\;\;\;\overset{1:1}{\longleftrightarrow}\;\;\;\;\C^n$$

\begin{defn}{Complex Topology on Spec}{} Let $S$ be a finitely generated $\C$-algebra. Let $a_1,\dots,a_n$ be generators of $S$. Consider the surjection $$\theta:\C[x_1,\dots,x_n]\to S$$ defined by $x_i\mapsto a_i$. Define the complex topology of $X=\text{Spec}(S)$ to be the subspace topology of $\C^n$ via the injective map $$\text{maxSpec}(\theta):\text{maxSpec}(S)\to\text{maxSpec}(\C^n)\cong\C^n$$ Denote $X^\text{an}$ to be the set $X=\text{maxSpec}(S)$ together with the complex topology. 
\end{defn}

\begin{lmm}{}{} Let $S$ be a finitely generated $\C$-algebra. Then the complex topology on $\text{maxSpec}(S)$ is independent of the choice of generators of $S$. 
\end{lmm}

\begin{prp}{}{} Let $S$ be a finitely generated $\C$-algebra. Then the natural inclusion $$\text{maxSpec}(S)\hookrightarrow\text{Spec}(S)$$ is continuous if we give $\text{maxSpec}(S)$ the complex topology and $\text{Spec}(S)$ the Zariski topology. 
\end{prp}

\begin{prp}{}{} Let $\varphi:R\to S$ be a homomorphism of finitely generated $\C$-algebras. Then the natural map $$\text{maxSpec}(\varphi):(\text{Spec}(S))^\text{an}\to(\text{Spec}(R))^\text{an}$$ is continuous.
\end{prp}

This marks the fact that the passage from affine varieties to topological spaces defined by sending $X=\text{Spec}(R)$ to $X^\text{an}$ is functorial. The following corollary should be of no surprise. 

\begin{crl}{}{} Let $\varphi:R\to S$ be an isomorphism of finitely generated $\C$-algebras. Then the natural map $$\text{maxSpec}(\varphi):(\text{Spec}(S))^\text{an}\to(\text{Spec}(R))^\text{an}$$ is a homeomorphism. 
\end{crl}

\begin{lmm}{}{} Let $\varphi:R\to S$ be a surjective homomorphism of finitely generated $\C$-algebras. Then the natural map $$\text{maxSpec}(\varphi):(\text{Spec}(S))^\text{an}\to(\text{Spec}(R))^\text{an}$$ an embedding. 
\end{lmm}

\subsection{Complex Topology for Schemes Locally of Finite Type}
Recall that a scheme is locally of finite type over $\C$ if it has an open cover $X=\bigcup_{i\in I}U_i$ for which $U_i\cong\text{Spec}(R_i)$ for some $R_i$ a finitely generated $\C$-algebra. Every scheme of finite type is necessarily a scheme that is locally of finite type. And it follows that when we discuss schemes that is locally of finite type, this includes the general theory of varieties. 

\begin{lmm}{}{} Let $(Y,\mO_Y)$ be a scheme locally of finite type over $\C$. Let $X\subseteq Y$ be an open set. Then the inclusion map $$\Psi:X\to Y$$ embeds $\text{Max}(X)$ homeomorphically onto the open subset $\Psi(X)\cap\text{Max}(Y)$. 
\end{lmm}

\begin{crl}{}{} Let $X$ be a scheme locally of finite type over $\C$. If $X=\bigcup_{i\in I}U_i$ is an open cover, then $\text{Max}(U_i)$ is an open cover for $\text{Max}(X)$. 
\end{crl}

This does not help much with respect to the complex topology unfortunately. Therefore we need a technical lemma. 

\begin{lmm}{}{} Let $(Z,\mO_Z)$ be a scheme locally of finite type over $\C$. Let $U$ and $V$ be open subsets of $Z$. Suppose that $(U,\mO_X|_U)\cong(\text{Spec}(R),\mO_{\text{Spec}(R)})$ and $(V,\mO_X|_V)\cong(\text{Spec}(S),\mO_{\text{Spec}(S)})$. Then $\text{Max}(U)\cap\text{Max}(V)$ is open in both $(\text{Spec}(R))^\text{an}$ and $(\text{Spec}(S))^\text{an}$. Moreover, the subspaces topologies induced with respect to both embeddings agree with each other. 
\end{lmm}

The final ingredient would be the weak topology. Let $X$ be a set. Let $\varphi_i:U_i\to X$ for $i\in I$ be functions from a topological space $U_i$ to $X$. Then the weak topology of $X$ with respect to $\varphi_i$ is the finest topology such that all $\varphi_i$ are continuous. This means that a subset $V\subseteq X$ is open if and only if $\varphi_i^{-1}(V)$ is open in $U_i$ for all $i\in I$. 

\begin{defn}{Complex Topology}{} Let $X$ be a scheme locally of finite type over $\C$. Let $V$ be the set of all open immersions $$(\Psi_i,\Psi_i^\#):(\text{Spec}(R_i),\mO_{\text{Spec}(R_i)})$$ of ringed spaces over $\C$ with each $R_i$ a finitely generated $\C$-algebra. Define the complex topology on $\text{Max}(X)$ to be the weak topology with respect to the maps $$\text{Max}(\Psi_i):(\text{Spec}(R_i))^\text{an}\to\text{Max}(X)$$ In this case we denote $\text{Max}(X)$ together with the complex topology by $X^\text{an}$. 
\end{defn}

\begin{lmm}{}{} Let $X$ be a scheme locally of finite type over $\C$. Suppose that there is an open immersion $$(\Psi,\Psi^\#):(\text{Spec}(R),\mO_{\text{Spec}(R)})$$ Then the map $\text{Max}(\Psi):(\text{Spec}(R))^\text{an}\to X^\text{an}$ is a homeomorphism onto its image, and the image is open in $X$. 
\end{lmm}

\begin{lmm}{}{} Let $X$ be a scheme locally of finite type over $\C$. Then the inclusion $$X^\text{an}\hookrightarrow X$$ is a continuous map where $X$ has the Zariski topology and $X^\text{an}$ has the complex topology. 
\end{lmm}

\begin{crl}{}{} Let $X,Y,Z$ be schemes locally of finite type over $\C$. Suppose that there are morphisms of schemes $\Phi:X\to Y$ and $\Psi:Y\to Z$. Then $$\Psi^\text{an}\circ\Phi^\text{an}=(\Psi\circ\Phi)^\text{an}$$
\end{crl}

\begin{crl}{}{} Let $(\Psi,\Psi^\#):(X,\mO_X)\to(Y,\mO_Y)$ be a morphism of schemes locally of finite type over $\C$. Then the following square commutes: \\~\\
\adjustbox{scale=1.0,center}{\begin{tikzcd}
	{X^\text{an}} & {Y^\text{an}} \\
	X & Y
	\arrow["{\Psi^\text{an}}", from=1-1, to=1-2]
	\arrow["{\lambda_X}"', from=1-1, to=2-1]
	\arrow["{\lambda_Y}", from=1-2, to=2-2]
	\arrow["\Psi"', from=2-1, to=2-2]
\end{tikzcd}}\\~\\
where $\lambda_X:X^\text{an}\to X$ is the inclusion. 
\end{crl}

We are almost done with complex analytification. We even showed that analytification is functorial, and more over there is a natural transformation from the analytification functor to the forgetful functor. Given a scheme locally of finite type, we constructed a topological space that is a subspace of $\C^n$. We also want to produce a sheaf on the subspace so that the resulting construct is an analytic space. 

\subsection{The Analytic Sheaf}
Once again, we first work with the affine case. 

\subsection{The Functorial Conclusion}
\begin{defn}{Complex Analytification Functor}{} Define the complex analytification functor $(\;\cdot\;)^\text{an}:\text{Var}_\C\to\text{ASpace}$ as follows. 
\begin{itemize}
\item For each variety $(X,\mO_X)$ over $\C$, it is sent to $(X^\text{an},\mO_X^\text{an})$
\item For each morphism $(\Psi,\Psi^\#):(X,\mO_X)\to(Y,\mO_Y)$, it is sent to the morphism $$(\Psi^\text{an},(\Psi^\#)^\text{an}):(X^\text{an},\mO_X^\text{an})\to(Y^\text{an},\mO_Y^\text{an})$$
\end{itemize}
\end{defn}

\begin{prp}{}{} Let $I_V:\text{Var}_\C\to\text{RSpace}$ and $I_A:\text{ASpace}\to\text{RSpace}$ be inclusion functors. Then there is a natural transformation $\lambda:I_A\circ(\;\cdot\;)^\text{an}\to I_V$
\end{prp}

\begin{thm}{GAGA I}{} Let $X$ be a projective complex algebraic variety. The restricted complex analytification functor from the category of coherent sheaves on $X$ to the category of coherent analytic sheaves on $X$ defines an equivalence of categories. 
\end{thm}









\end{document}
