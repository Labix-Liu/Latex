\documentclass[a4paper]{article}

\input{C:/Users/liula/Desktop/Latex/Headers V1.2.tex}

\pagestyle{fancy}
\fancyhf{}
\rhead{Labix}
\lhead{Lie Groups and Lie Algebra}
\rfoot{\thepage}

\title{Lie Groups and Lie Algebra}

\author{Labix}

\date{\today}
\begin{document}
\maketitle
\begin{abstract}
\end{abstract}
\pagebreak
\tableofcontents
\pagebreak

\section{Introduction to Lie Algebras}
\subsection{Lie Algebras}
\begin{defn}{Lie Algebras}{} A Lie algebra is a vector space $V$ over a field $K$ together with a bilinear map $[-,-]:V\times V\to V$ such that for all $X,Y,Z\in V$, we have the following. 
\begin{itemize}
\item Anti-commutativity: $[X,Y]=-[Y,X]$
\item Jacobi identity: $[[X,Y],Z]+[[Y,Z],X]+[[Z,X],Y]=0$
\end{itemize}
The bilinear map $[-,-]$ is called a Lie bracket. 
\end{defn}

Lie Algebras are not algebras because the Lie bracket fails associativity. Therefore we have to redefine all the standard notions one has in algebra. 

\begin{defn}{Homomorphism of Lie algebra}{} Let $V$ and $W$ be Lie algebras over $K$. A homomorphism from $V$ to $W$ is an $K$-linear map $F:V\to W$ such that $$[F(a),F(b)]=[a,b]$$ for all $a,b\in V$. 
\end{defn}

\begin{defn}{Lie Subalgebra}{} Let $V$ be a Lie algebra over $K$. A lie subalgebra of $V$ is a subset $W\subseteq V$ such that 
\begin{itemize}
\item $W$ is a vector subspace of $V$
\item $[w_1,w_2]\in W$ for all $w_1,w_2\in W$
\end{itemize}
\end{defn}

It is clear that a Lie subalgebra is also a Lie algebra in its own right. Moreover, the inclusion $W\to V$ is a Lie algebra homomorphism. 

\begin{defn}{Ideal}{} Let $V$ be a Lie algebra over $K$. Let $I$ be a subset of $V$. Then $I$ is an ideal of $V$ if the following are true. 
\begin{itemize}
\item $I$ is a vector subspace of $V$
\item $[v,i]\in I$ for all $v\in V$ and $i\in I$. 
\end{itemize}
\end{defn}

\begin{prp}{}{} Let $V$ be a Lie algebra and $I,J$ ideals of $V$. Then the following are also ideals of $V$. 
\begin{itemize}
\item The intersection $I\cap J$
\item The sum $I+J=\{i+j\;|\;i\in I\text{ and }j\in J\}$
\item The Lie bracket $[I,J]=\langle[i,j]\;|\;i\in I\text{ and }j\in J\rangle$
\end{itemize}
\end{prp}

\begin{prp}{}{} Let $V$ be a Lie algebra over $K$ and $U$ an ideal of $V$. Then $V/U$ has a unique Lie algebra structure such that the quotient map $V\to V/U$ is a Lie algebra homomorphism. 
\end{prp}

\begin{defn}{Center}{} Let $L$ be a Lie algebra. Define the center of $L$ by $$Z(L)=\{z\in L\;|\;[z,x]=0\text{ for all }x\in L\}$$
\end{defn}

\begin{lmm}{}{} Let $L$ be a Lie algebra. Then $Z(L)$ is an ideal of $L$. 
\end{lmm}

\begin{defn}{Direct Sum of Lie Algebras}{} Let $L_1$ and $L_2$ be Lie algebras. Define the direct sum of $L_1$ and $L_2$ by $$L_1\oplus L_2=\{(a_1,a_2)\;|\;a_1\in L_1,a_2\in L_2\}$$ together with component wise addition and scalar multiplication and Lie bracket operation $$[(a_1,a_2),(b_1,b_2)]=([a_1,b_1],[a_2,b_2])$$ which is component wise application of the Lie bracket for $(a_1,a_2),(b_1,b_2)\in L_1\oplus L_2$. 
\end{defn}

\begin{prp}{}{} Let $L_1$ and $L_2$ be Lie algebras. Then the following are true. 
\begin{itemize}
\item $[L_1\oplus L_2,L_1\oplus L_2]=[L_1,L_1]\oplus[L_2,L_2]$
\item $Z(L_1\oplus L_2)=Z(L_1)\oplus Z(L_2)$
\item $\{(x,0)\;|\;x\in L_1\}\cong L_1$ is an ideal of $L_1\oplus L_2$
\item $\{(0,y)\;|\;y\in L_2\}\cong L_2$ is an ideal of $L_1\oplus L_2$
\end{itemize}
\end{prp}

\subsection{The Isomorphism Theorems}
\begin{thm}{First Isomorphism Theorem}{} Let $\phi:L_1\to L_2$ be a homomorphism of Lie algebras. Then the following are true. 
\begin{itemize}
\item $\ker(\phi)$ is an ideal of $L_1$
\item $\im(\phi)$ is a Lie subalgebra of $L_2$
\end{itemize}
Moreover, we have an isomorphism $$\frac{L_1}{\ker(\phi)}\cong\im(\phi)$$
\end{thm}

\begin{thm}{Second Isomorphism Theorem}{} Let $L$ be a Lie algebra. Let $I$ and $J$ be ideals of $L$. Then the following are true. 
\begin{itemize}
\item $I$ and $J$ are ideals of $I+J$
\item $I\cap J$ is an ideal of $I$ and $J$
\end{itemize}
Moreover, we have an isomorphism $$\frac{I+J}{J}\cong\frac{I}{I\cap J}$$
\end{thm}

\begin{thm}{Third Isomorphism Theorem}{} Let $L$ be a Lie algebra. Let $I$ and $J$ be ideals of $L$ such that $I\subseteq J$. Then $J/I$ is an ideal of $L/I$. Moreover, there is an isomorphism $$\frac{L/I}{J/I}\cong\frac{L}{J}$$
\end{thm}

\begin{thm}{Correspondence Theorem}{} Let $L$ be a Lie algebra with ideal $I$. Then there exists a bijective correspondence $$\{J\;|\;J\text{ is an ideal of }L\text{ and }I\subseteq J\}\;\;\;\;\overset{1:1}{\longleftrightarrow}\;\;\;\;\{K\;|\;K\text{ is an ideal of }L/I\}$$
\end{thm}

\pagebreak
\section{Introduction to Lie Groups}
\subsection{Lie Groups}
\begin{defn}{Lie Groups}{} A Lie group $G$ is a smooth manifold which is also a group such that the multiplication map $G\times G\to G$ given by $(g,h)\mapsto gh$ and the inverse map $i:G\to G$ given by $g\mapsto g^{-1}$ are smooth maps. 
\end{defn}

\begin{prp}{}{} Let $G$ be a Lie group. A subgroup $H$ of $G$ has the unique structure of a Lie subgroup if $H$ is closed in $G$. 
\end{prp}

\subsection{Relation between Lie Groups and Lie Algebras}
For a group $G$, denote the left multiplication map of $h\in G$ by $l_h$. If $G$ is a Lie group, we have seen that $l_h$ is a smooth map, and so it induces a differential $(l_h)_\ast$. 

\begin{defn}{Left Invariant Vector Field}{} Let $G$ be a Lie group and $X$ a vector field on $G$. We say that $X$ is left invariant if $$(l_h)_\ast(X_g)=X_{hg}$$ for all $X_g\in T_g(G)$. 
\end{defn}

\begin{prp}{}{} Let $G$ be a Lie group. The vector space of left invariant vector fields of $G$ is a Lie algebra of dimension $\dim(G)$. Moreover, if $X_e\in T_e(G)$ is a tangent vector at $e$ the identity, then there is a unique left invariant vector field $X$ on $G$ such that its identity is $X_e$. 
\end{prp}

\begin{defn}{Lie Algebra of a Lie Group}{} Let $G$ be a Lie group. Define the Lie algebra $V$ of $G$ to be the vector space $T_e(G)$. 
\end{defn}

Recall that given a homomorphism of Lie groups $\phi:G\to H$, it induces a differential $\phi_\ast:T_g(G)\to T_{\phi(g)}(H)$. 

\begin{prp}{}{} Let $\phi:G\to H$ be a homomorphism of Lie groups with Lie algebras $V$ and $W$ respectively. Then the induced map from the differential $\phi_\ast:V\to W$ is a Lie algebra homomorphism. 
\end{prp}












\end{document}
