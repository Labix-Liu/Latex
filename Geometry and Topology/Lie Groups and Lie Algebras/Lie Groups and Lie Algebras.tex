\documentclass[a4paper]{article}

%=========================================
% Packages
%=========================================
\usepackage{mathtools}
\usepackage{amsfonts}
\usepackage{amsmath}
\usepackage{amssymb}
\usepackage{amsthm}
\usepackage[a4paper, total={6in, 8in}, margin=1in]{geometry}
\usepackage[utf8]{inputenc}
\usepackage{fancyhdr}
\usepackage[utf8]{inputenc}
\usepackage{graphicx}
\usepackage{physics}
\usepackage[listings]{tcolorbox}
\usepackage{hyperref}
\usepackage{tikz-cd}
\usepackage{adjustbox}
\usepackage{enumitem}
\usepackage[font=small,labelfont=bf]{caption}
\usepackage{subcaption}
\usepackage{wrapfig}
\usepackage{makecell}



\raggedright

\usetikzlibrary{arrows.meta}

\DeclarePairedDelimiter\ceil{\lceil}{\rceil}
\DeclarePairedDelimiter\floor{\lfloor}{\rfloor}

%=========================================
% Fonts
%=========================================
\usepackage{tgpagella}
\usepackage[T1]{fontenc}


%=========================================
% Custom Math Operators
%=========================================
\DeclareMathOperator{\adj}{adj}
\DeclareMathOperator{\im}{im}
\DeclareMathOperator{\nullity}{nullity}
\DeclareMathOperator{\sign}{sign}
\DeclareMathOperator{\dom}{dom}
\DeclareMathOperator{\lcm}{lcm}
\DeclareMathOperator{\ran}{ran}
\DeclareMathOperator{\ext}{Ext}
\DeclareMathOperator{\dist}{dist}
\DeclareMathOperator{\diam}{diam}
\DeclareMathOperator{\aut}{Aut}
\DeclareMathOperator{\inn}{Inn}
\DeclareMathOperator{\syl}{Syl}
\DeclareMathOperator{\edo}{End}
\DeclareMathOperator{\cov}{Cov}
\DeclareMathOperator{\vari}{Var}
\DeclareMathOperator{\cha}{char}
\DeclareMathOperator{\Span}{span}
\DeclareMathOperator{\ord}{ord}
\DeclareMathOperator{\res}{res}
\DeclareMathOperator{\Hom}{Hom}
\DeclareMathOperator{\Mor}{Mor}
\DeclareMathOperator{\coker}{coker}
\DeclareMathOperator{\Obj}{Obj}
\DeclareMathOperator{\id}{id}
\DeclareMathOperator{\GL}{GL}
\DeclareMathOperator*{\colim}{colim}

%=========================================
% Custom Commands (Shortcuts)
%=========================================
\newcommand{\CP}{\mathbb{CP}}
\newcommand{\GG}{\mathbb{G}}
\newcommand{\F}{\mathbb{F}}
\newcommand{\N}{\mathbb{N}}
\newcommand{\Q}{\mathbb{Q}}
\newcommand{\R}{\mathbb{R}}
\newcommand{\C}{\mathbb{C}}
\newcommand{\E}{\mathbb{E}}
\newcommand{\Prj}{\mathbb{P}}
\newcommand{\RP}{\mathbb{RP}}
\newcommand{\T}{\mathbb{T}}
\newcommand{\Z}{\mathbb{Z}}
\newcommand{\A}{\mathbb{A}}
\renewcommand{\H}{\mathbb{H}}
\newcommand{\K}{\mathbb{K}}

\newcommand{\mA}{\mathcal{A}}
\newcommand{\mB}{\mathcal{B}}
\newcommand{\mC}{\mathcal{C}}
\newcommand{\mD}{\mathcal{D}}
\newcommand{\mE}{\mathcal{E}}
\newcommand{\mF}{\mathcal{F}}
\newcommand{\mG}{\mathcal{G}}
\newcommand{\mH}{\mathcal{H}}
\newcommand{\mI}{\mathcal{I}}
\newcommand{\mJ}{\mathcal{J}}
\newcommand{\mK}{\mathcal{K}}
\newcommand{\mL}{\mathcal{L}}
\newcommand{\mM}{\mathcal{M}}
\newcommand{\mO}{\mathcal{O}}
\newcommand{\mP}{\mathcal{P}}
\newcommand{\mS}{\mathcal{S}}
\newcommand{\mT}{\mathcal{T}}
\newcommand{\mV}{\mathcal{V}}
\newcommand{\mW}{\mathcal{W}}

%=========================================
% Colours!!!
%=========================================
\definecolor{LightBlue}{HTML}{2D64A6}
\definecolor{ForestGreen}{HTML}{4BA150}
\definecolor{DarkBlue}{HTML}{000080}
\definecolor{LightPurple}{HTML}{cc99ff}
\definecolor{LightOrange}{HTML}{ffc34d}
\definecolor{Buff}{HTML}{DDAE7E}
\definecolor{Sunset}{HTML}{F2C57C}
\definecolor{Wenge}{HTML}{584B53}
\definecolor{Coolgray}{HTML}{9098CB}
\definecolor{Lavender}{HTML}{D6E3F8}
\definecolor{Glaucous}{HTML}{828BC4}
\definecolor{Mauve}{HTML}{C7A8F0}
\definecolor{Darkred}{HTML}{880808}
\definecolor{Beaver}{HTML}{9A8873}
\definecolor{UltraViolet}{HTML}{52489C}



%=========================================
% Theorem Environment
%=========================================
\tcbuselibrary{listings, theorems, breakable, skins}

\newtcbtheorem[number within = subsection]{thm}{Theorem}%
{	colback=Buff!3, 
	colframe=Buff, 
	fonttitle=\bfseries, 
	breakable, 
	enhanced jigsaw, 
	halign=left
}{thm}

\newtcbtheorem[number within=subsection, use counter from=thm]{defn}{Definition}%
{  colback=cyan!1,
    colframe=cyan!50!black,
	fonttitle=\bfseries, breakable, 
	enhanced jigsaw, 
	halign=left
}{defn}

\newtcbtheorem[number within=subsection, use counter from=thm]{axm}{Axiom}%
{	colback=red!5, 
	colframe=Darkred, 
	fonttitle=\bfseries, 
	breakable, 
	enhanced jigsaw, 
	halign=left
}{axm}

\newtcbtheorem[number within=subsection, use counter from=thm]{prp}{Proposition}%
{	colback=LightBlue!3, 
	colframe=Glaucous, 
	fonttitle=\bfseries, 
	breakable, 
	enhanced jigsaw, 
	halign=left
}{prp}

\newtcbtheorem[number within=subsection, use counter from=thm]{lmm}{Lemma}%
{	colback=LightBlue!3, 
	colframe=LightBlue!60, 
	fonttitle=\bfseries, 
	breakable, 
	enhanced jigsaw, 
	halign=left
}{lmm}

\newtcbtheorem[number within=subsection, use counter from=thm]{crl}{Corollary}%
{	colback=LightBlue!3, 
	colframe=LightBlue!60, 
	fonttitle=\bfseries, 
	breakable, 
	enhanced jigsaw, 
	halign=left
}{crl}

\newtcbtheorem[number within=subsection, use counter from=thm]{eg}{Example}%
{	colback=Beaver!5, 
	colframe=Beaver, 
	fonttitle=\bfseries, 
	breakable, 
	enhanced jigsaw, 
	halign=left
}{eg}

\newtcbtheorem[number within=subsection, use counter from=thm]{ex}{Exercise}%
{	colback=Beaver!5, 
	colframe=Beaver, 
	fonttitle=\bfseries, 
	breakable, 
	enhanced jigsaw, 
	halign=left
}{ex}

\newtcbtheorem[number within=subsection, use counter from=thm]{alg}{Algorithm}%
{	colback=UltraViolet!5, 
	colframe=UltraViolet, 
	fonttitle=\bfseries, 
	breakable, 
	enhanced jigsaw, 
	halign=left
}{alg}




%=========================================
% Hyperlinks
%=========================================
\hypersetup{
    colorlinks=true, %set true if you want colored links
    linktoc=all,     %set to all if you want both sections and subsections linked
    linkcolor=DarkBlue,  %choose some color if you want links to stand out
}


\pagestyle{fancy}
\fancyhf{}
\rhead{Labix}
\lhead{Lie Groups and Lie Algebra}
\rfoot{\thepage}

\title{Lie Groups and Lie Algebra}

\author{Labix}

\date{\today}
\begin{document}
\maketitle
\begin{abstract}
Potentially good books: Humphreys, Erdmann and Wildson
\end{abstract}
\pagebreak
\tableofcontents
\pagebreak

\section{Introduction to Lie Algebras}
\subsection{Lie Algebras}
\begin{defn}{Lie Brackets}{} Let $V$ be a vector space over a field $k$. Let $[-,-]:V\times V\to V$ be a bilinear map. We say that $[-,-]$ is a Lie bracket if the following are true. 
\begin{itemize}
\item The Alternating Property: $[X,X]=0$
\item Jacobi identity: $[[X,Y],Z]+[[Y,Z],X]+[[Z,X],Y]=0$
\end{itemize}
\end{defn}

Consider the cross product $\times:\R^3\times\R^3\to\R^3$ in $\R^3$. It is easy to see that it is a Lie bracket. 

\begin{defn}{Lie Algebras}{} A Lie algebra is a vector space $V$ over a field $K$ together with a Lie bracket $[-,-]:V\times V\to V$. 
\end{defn}

For $k$ a field, $M_n(k)$ for any $n\geq 1$ is a Lie algebra with Lie bracket defined as $[A,B]=AB-BA$ for $A,B\in M_n(k)$. 

\begin{lmm}{}{} Let $L$ be a Lie Algebra. Then for all $x,y\in L$, we have that $$[x,y]=-[y,x]$$ In other words, the Lie bracket is anti-commutative. 
\begin{proof}
We have that 
\begin{align*}
[x,y]+[y,x]&=[x,x]+[x,y-x]+[y,y]+[y,x-y]\tag{Bilinearity}\\
&=[x,x]+[y,y]-[x-y,x-y]\tag{Bilinearity}\\
&=0\tag{Alternating}
\end{align*}
and so we conclude. 
\end{proof}
\end{lmm}

Lie Algebras are not algebras (in the sense of Rings and Modules) because the Lie bracket fails associativity. Therefore we have to redefine all the standard notions one has in algebra. 

While Lie Algebras are not in general algebras, every associative algebra can be equipped with a Lie algebra. For $A$ an associative algebra over a field, we can define a bilinear map on $A$ by $$[a,b]=ab-ba$$ for all $a,b\in A$. There may also be more than one way to equip an algebra with a Lie algebra structure. One should not think that Lie Algebras encompasses associative algebras because of the different Lie algebras one can equip. Instead, we think of the Lie bracket as an extra structure on associative algebras such that they become Lie algebras. 

\begin{defn}{Structure Constants}{} Let $L$ be a Lie algebra such that its underlying vector space has basis $e_1,\dots,e_n$. Define the structure constants of $L$ to be the elements $c_{ij}^k\in\F$ such that $$[e_i,e_j]=\sum_{k=1}^nc_{ij}^ke_k$$ for all $1\leq i,j\leq n$. 
\end{defn}

The structure constants are useful in the following sense. Let $L$ be a Lie algebra and let $a=\sum_{k=1}^na_ke_k$ and $b=\sum_{k=1}^nb_ke_k$ be elements of $L$. Then there Lie bracket can be written as $$[a,b]=\sum_{1\leq i<j\leq n}(a_ib_j-a_jb_i)[e_i,e_j]$$ by bilinearity. Plugging in the structure constants, we obtain $$[a,b]=\sum_{1\leq i<j\leq n}(a_ib_j-a_jb_i)\sum_{k=1}^nc_{ij}^ke_k$$ Thus we can write $[a,b]$ in terms of the basis $e_1,\dots,e_n$ using structure constants. 

\subsection{Homomorphisms and Ideals}
\begin{defn}{Homomorphism of Lie algebra}{} Let $V$ and $W$ be Lie algebras over $K$. A homomorphism from $V$ to $W$ is an $K$-linear map $F:V\to W$ such that $$[F(a),F(b)]=[a,b]$$ for all $a,b\in V$. 
\end{defn}

\begin{defn}{Lie Subalgebra}{} Let $V$ be a Lie algebra over $K$. A lie subalgebra of $V$ is a subset $W\subseteq V$ such that 
\begin{itemize}
\item $W$ is a vector subspace of $V$
\item $[w_1,w_2]\in W$ for all $w_1,w_2\in W$
\end{itemize}
\end{defn}

It is clear that a Lie subalgebra is also a Lie algebra in its own right. Moreover, the inclusion $map W\to V$ is a Lie algebra homomorphism. 

\begin{defn}{Ideal}{} Let $V$ be a Lie algebra over $K$. Let $I$ be a subset of $V$. Then $I$ is an ideal of $V$ if the following are true. 
\begin{itemize}
\item $I$ is a vector subspace of $V$
\item $[v,i]\in I$ for all $v\in V$ and $i\in I$. 
\end{itemize}
\end{defn}

It is clear from definitions that every ideal of a Lie algebra is a Lie subalgebra. However, the converse is not always true. 

\begin{prp}{}{} Let $V$ be a Lie algebra and $I,J$ ideals of $V$. Then the following are also ideals of $V$. 
\begin{itemize}
\item The intersection $I\cap J$
\item The sum $I+J=\{i+j\;|\;i\in I\text{ and }j\in J\}$
\end{itemize}
\end{prp}

\begin{defn}{The Lie Bracket}{} Let $V$ be a Lie algebra. Let $I,J$ be ideals of $V$. Define the Lie bracket of $I$ and $J$ to be $$[I,J]=\langle[i,j]\;|\;i\in I\text{ and }j\in J\rangle$$
\end{defn}

\begin{lmm}{}{} Let $V$ be a Lie algebra. Let $I,J$ be ideals of $V$. Then the Lie bracket $[I,J]$ is an ideal of $V$. 
\end{lmm}

\subsection{Products and Quotients of Lie Algebras}
\begin{defn}{Direct Sum of Lie Algebras}{} Let $L_1$ and $L_2$ be Lie algebras. Define the direct sum of $L_1$ and $L_2$ by $$L_1\oplus L_2=\{(a_1,a_2)\;|\;a_1\in L_1,a_2\in L_2\}$$ together with component wise addition and scalar multiplication and Lie bracket operation $$[(a_1,a_2),(b_1,b_2)]=([a_1,b_1],[a_2,b_2])$$ which is component wise application of the Lie bracket for $(a_1,a_2),(b_1,b_2)\in L_1\oplus L_2$. 
\end{defn}

\begin{prp}{}{} Let $L_1$ and $L_2$ be Lie algebras. Then the following are true. 
\begin{itemize}
\item $[L_1\oplus L_2,L_1\oplus L_2]=[L_1,L_1]\oplus[L_2,L_2]$
\item $\{(x,0)\;|\;x\in L_1\}\cong L_1$ is an ideal of $L_1\oplus L_2$
\item $\{(0,y)\;|\;y\in L_2\}\cong L_2$ is an ideal of $L_1\oplus L_2$
\end{itemize}
\end{prp}

\begin{prp}{}{} Let $V$ be a Lie algebra over $K$ and $U$ an ideal of $V$. Then $V/U$ has a unique Lie algebra structure such that the quotient map $V\to V/U$ is a Lie algebra homomorphism. 
\end{prp}

\subsection{The Center of a Lie Algebra}
\begin{defn}{Center of a Lie Algebra}{} Let $L$ be a Lie algebra. Define the center of $L$ by $$Z(L)=\{z\in L\;|\;[z,x]=0\text{ for all }x\in L\}$$
\end{defn}

\begin{lmm}{}{} Let $L$ be a Lie algebra. Then $Z(L)$ is an ideal of $L$. 
\end{lmm}

\begin{prp}{}{} Let $L_1,L_2$ be Lie algebras over the same field $K$. Then $$Z(L_1\oplus L_2)=Z(L_1)\oplus Z(L_2)$$
\end{prp}

\begin{defn}{The Adjoint Homomorphism}{} Let $V$ be a Lie algebra. Define the adjoint homomorphism $\text{ad}:V\to GL(V)$ to be the map given by $$\text{ad}(x)(y)=[x,y]$$
\end{defn}

\begin{lmm}{}{} Let $V$ be a Lie algebra. Then then the adjoint homomorphism $\text{ad}:V\to GL(V)$ is a Lie algebra homomorphism. 
\end{lmm}

\begin{lmm}{}{} Let $V$ be a Lie algebra. Then the kernel of the adjoint homomorphism is equal to $$\ker(\text{ad})=Z(V)$$ the center of $V$. 
\end{lmm}

\subsection{The Isomorphism Theorems}
\begin{thm}{First Isomorphism Theorem}{} Let $\phi:L_1\to L_2$ be a homomorphism of Lie algebras. Then the following are true. 
\begin{itemize}
\item $\ker(\phi)$ is an ideal of $L_1$
\item $\im(\phi)$ is a Lie subalgebra of $L_2$
\end{itemize}
Moreover, we have an isomorphism $$\frac{L_1}{\ker(\phi)}\cong\im(\phi)$$
\end{thm}

\begin{thm}{Second Isomorphism Theorem}{} Let $L$ be a Lie algebra. Let $I$ and $J$ be ideals of $L$. Then the following are true. 
\begin{itemize}
\item $I$ and $J$ are ideals of $I+J$
\item $I\cap J$ is an ideal of $I$ and $J$
\end{itemize}
Moreover, we have an isomorphism $$\frac{I+J}{J}\cong\frac{I}{I\cap J}$$
\end{thm}

\begin{thm}{Third Isomorphism Theorem}{} Let $L$ be a Lie algebra. Let $I$ and $J$ be ideals of $L$ such that $I\subseteq J$. Then $J/I$ is an ideal of $L/I$. Moreover, there is an isomorphism $$\frac{L/I}{J/I}\cong\frac{L}{J}$$
\end{thm}

\begin{thm}{Correspondence Theorem}{} Let $L$ be a Lie algebra with ideal $I$. Then there exists a bijective correspondence $$\{J\;|\;J\text{ is an ideal of }L\text{ and }I\subseteq J\}\;\;\;\;\overset{1:1}{\longleftrightarrow}\;\;\;\;\{K\;|\;K\text{ is an ideal of }L/I\}$$
\end{thm}

\pagebreak
\section{Types of Lie Algebras}
\subsection{Abelian Lie Algebras}
Lie algebras that are Abelian are the simplest Lie algebra there is to study. 

\begin{defn}{Abelian Lie Algebras}{} Let $(L,[-,-])$ be a Lie algebra. We say that $L$ is abelian if $$[x,y]=0$$ for all $x,y\in L$. 
\end{defn}

\begin{lmm}{}{} Let $L$ be a Lie algebra. Let $I$ be an ideal of $L$. Then $L/I$ is abelian if and only if $[L,L]\subseteq I$
\end{lmm}

\begin{crl}{}{} Let $L$ be a Lie algebra. Then the smallest ideal $I$ such that $L/I$ is abelian is given by $$I=[L,L]$$
\end{crl}

\subsection{Soluble Lie Algebras}
Let $L$ be a Lie algebra. We have seen that $\text{rad}(L)$ is soluble and $L/\text{rad}(L)$ is semisimple. Therefore to study a general Lie algebra, we need to understand soluble Lie algebras and semisimple Lie algebras. If we restrict the case to Lie algebras over $\C$, Lie's theorem will solve the first part of the problem, while the study of semisimple Lie algebras is postponed until section 5. 

\begin{defn}{Derived Series}{} Let $L$ be a Lie algebra. Define the derived series $L^{(n)}$ of $L$ to be the sequence recursively defined as follows. 
\begin{itemize}
\item For $n=0$, define $L^{(0)}=L$
\item When $n\in\N\setminus\{0\}$, define $$L^{(n)}=[L^{(n-1)},L^{(n-1)}]$$
\end{itemize}
\end{defn}

\begin{lmm}{}{} Let $L_1,L_2$ be Lie algebras. Let $\phi:L_1\to L_2$ be a Lie algebra homomorphism. Then $$\phi(L_1^{(k)})=\phi(L_1)^{(k)}$$
\end{lmm}

\begin{defn}{Soluble Lie Algebras}{} Let $L$ be a Lie algebra. We say that $L$ is soluble if there exists $n\in\N$ such that $$L^{(n)}=0$$
\end{defn}

\begin{lmm}{}{} Let $L$ be a Lie algebra. If $L$ is abelian, then $L$ is soluble. 
\end{lmm}

\begin{eg}{}{} Consider the following Lie lalgebras. 
\begin{itemize}
\item $b_n(\C)$ the set of all upper triangular $n\times n$ matrices is soluble
\item $SL(2,\C)$ is not soluble
\item $GL(2,\C)$ is not soluble
\end{itemize}
\end{eg}

\begin{prp}{}{} Let $L$ be a Lie algebra. Let $I$ and $J$ be ideals of $L$. Then the following are true. 
\begin{itemize}
\item Let $\phi:L\to K$ be a Lie algebra homomorphism. If $L$ is soluble then $\phi(L)$ is soluble. 
\item Let $M$ be a Lie subalgebra of $L$. If $L$ is soluble, then $M$ is soluble. 
\item If $I$ and $L/I$ are soluble, then $L$ is soluble. 
\item If $I$ and $J$ are soluble, then $I+J$ is soluble. 
\end{itemize}
\end{prp}

\begin{thm}{Lie's Theorem}{} Let $V$ be a vector space over $\C$. Let $L$ be a soluble Lie subalgebra of $GL(V)$. Then there exists a basis $B$ of $V$ such that for all $M\in L$, $M$ is upper triangular. 
\end{thm}

\subsection{Nilpotent Lie Algebras}
\begin{defn}{Lower Central Series}{} Let $L$ be a Lie algebra. Define the lower central series $L^0,L^1,\dots,L^n,\dots$ as follows. 
\begin{itemize}
\item For $n=0$, define $L^0=L$
\item For $n\in\N\setminus\{0\}$, define $$L^n=[L,L^{n-1}]$$
\end{itemize}
\end{defn}

\begin{lmm}{}{} Let $L$ be a Lie algebra. Then there is an isomorphism $$[L,L^n]=[L^n,L]$$ for all $n\in\N$ given by the opposite map $x\mapsto -x$. 
\end{lmm}

\begin{lmm}{}{} Let $L$ be a Lie algebra. Then the following are true. 
\begin{itemize}
\item For all $n\in\N$, $L^n$ is an ideal of $L$. 
\item $L^{n+1}\subseteq L^n$. 
\end{itemize}
\end{lmm}

\begin{defn}{Nilpotent Lie Algebras}{} Let $L$ be a Lie algebra. We say that $L$ is nilpotent if there exists $n\in\N$ such that $$L^n=0$$
\end{defn}

\begin{lmm}{}{} Let $L$ be a Lie algebra. If $L$ is abelian, then $L$ is nilpotent. 
\end{lmm}

\begin{eg}{}{} Consider the following Lie algebras. 
\begin{itemize}
\item $SL(2,\C)$ is nilpotent. 
\item $b_n(\C)$ is not nilpotent for all $n\geq 2$. 
\item $U_3(\C)$ the Heisenberg Lie algebra is nilpotent ($3\times 3$ strictly upper triangular matrices)
\item $U_n(\C)$ is nilpotent for all $n\geq 3$. 
\end{itemize}
\end{eg}

\begin{lmm}{}{} Let $L_1,L_2$ be Lie algebras. Let $\phi:L_1\to L_2$ be a Lie algebra homomorphism. Then $$\phi(L^k)=(\phi(L))^k$$ for all $k\in\N$. \tcbline
\begin{proof}
We prove by induction. The base case $k=0$ is clear. Suppose that $\phi(L^k)=(\phi(L))^k$. Then we have that 
\begin{align*}
\phi(L^{k+1})&=\phi([L,L^k])\\
&=[\phi(L),\phi(L^k)]\\
&=[\phi(L),\phi(L)^k]\\
&=(\phi(L))^{k+1}
\end{align*}
By induction, we conclude. 
\end{proof}
\end{lmm}

\begin{lmm}{}{} Let $L$ be a Lie algebra. Then the following are true. 
\begin{itemize}
\item Let $M$ be a Lie subalgebra of $L$. If $L$ is nilpotent, then $M$ is nilpotent. 
\item If $L\neq 0$ is nilpotent, then $Z(L)\neq 0$
\item If $L/Z(L)$ is nilpotent, then $L$ is nilpotent. 
\end{itemize} \tcbline
\begin{proof}~\\
\begin{itemize}
\item Let $M$ be a Lie subalgebra of $L$. I claim that $M^k\subseteq L^k$ for all $k\in\N$. The base case $k=0$ is clearly true. Suppose that $M^k\subseteq L^k$. Let $x\in[M,M^k]=M^{k+1}$. Then $x=[m,t]$ for some $m\in M$ and $t\in M^k$. Then $t\in L^k$. Also $m\in L$ implies that $x=[m,t]\in[L,L^k]=L^{k+1}$. Thus $M^{k+1}\subseteq L^{k+1}$. Now since $L$ is nilpotent, there exists $n\in\N$ such that $L^n=0$. Then $M^n\subseteq L^n=0$ so that $M$ is also nilpotent. 
\item Suppose that $n\in\N$ is the smallest natural number such that $L^n=0$. Then $[L,L^{n-1}]=0$. Let $x\in L$. Then for all $y\in L^{n-1}$, we have that $[x,y]=0$. Thus $x\in Z(L)$. 
\item Since $L/Z(L)$ is nilpotent, there exists $n\in\N$ such that $(L/Z(L))^n=0$. Let $\pi:L\to L/Z(L)$ be the quotient homomorphism. Since $\pi$ is surjective, we use the above lemma to find that $$\pi(L^n)=\pi(L)^n=\left(\frac{L}{Z(L)}\right)^n=\frac{L^n+Z(L)}{Z(L)}=0$$ This means that $L^n\subseteq Z(L)$. It follows that $L^{n+1}=[L,L^n]\subseteq[L,Z(L)]=0$ and we conclude. 
\end{itemize}
\end{proof}
\end{lmm}

\subsection{Engel's Theorem}
\begin{defn}{Ad-Nilpotency}{} Let $L$ be a Lie algebra. Let $x\in L$. We say that $x$ is ad-nilpotent if there exists $n\in\N$ such that $$\text{ad}(x)^n=0\in GL(L)$$
\end{defn}

\begin{lmm}{}{} Let $L$ be Lie algebra. If $L$ is nilpotent, then all elements $x\in L$ are ad-nilpotent. 
\end{lmm}

\begin{thm}{Engel's Theorem}{} Let $V$ be a vector space over a field $k$. Let $L$ be a Lie subalgebra of $GL(V)$. Suppose that for all $x\in L$, $x$ is ad-nilpotent. Then the following are true. 
\begin{itemize}
\item There exists a basis $B$ of $V$ such that every $T\in L$ is upper triangular. 
\item $L$ is nilpotent. 
\end{itemize}
\end{thm}

\subsection{Lie's Theorem}

\subsection{Simple and Semisimple Lie Algebras}
\begin{prp}{}{} Let $L$ be a Lie algebra. Then there exists a unique soluble ideal $I$ of $L$ such that for any soluble ideal $J\subseteq L$, we have $J\subseteq I$. 
\end{prp}

\begin{defn}{}{} Let $L$ be a Lie algebra. Define the radical ideal $\text{rad}(L)\subseteq L$ of $L$ to be the unique soluble ideal of $L$ that contains all other soluble ideals. 
\end{defn}

\begin{defn}{Semisimple Lie Algebras}{}{} Let $L$ be a Lie algebra. We say that $L$ is semisimple if $$\frac{L}{\text{rad}(L)}=\{0\}$$
\end{defn}

\begin{eg}{}{} Consider the following Lie algebras. 
\begin{itemize}
\item $\{0\}$ is semisimple. 
\item $SL(2,\C)$ is semisimple. 
\item $GL(2,\C)$ is not semisimple. 
\end{itemize}
\end{eg}

\begin{lmm}{}{} Let $L$ be a Lie algebra. Then $L/\text{rad}(L)$ is semisimple. \tcbline
\begin{proof}
Let $K$ be a soluble ideal of $L/\text{rad}(L)$. By the correspondence theorem, there exists an ideal $I$ of $L$ such that $\text{rad}(L)\subseteq I$ and $K=I/\text{rad}(L)$. Since $\text{rad}(L)$ and $K$ are soluble, we conclude that $I$ is soluble. Hence $I\subseteq\text{rad}(L)$. We conclude that $I=\text{rad}(L)$. Hence $K=\{0\}$. Thus $L/\text{rad}(L)$ is semisimple. 
\end{proof}
\end{lmm}

\pagebreak
\section{The Killing Form and Cartan's Criteria}

\pagebreak
\section{Introduction to Lie Groups}
\subsection{Lie Groups}
\begin{defn}{Lie Groups}{} A Lie group $G$ is a smooth manifold which is also a group such that the multiplication map $G\times G\to G$ given by $(g,h)\mapsto gh$ and the inverse map $i:G\to G$ given by $g\mapsto g^{-1}$ are smooth maps. 
\end{defn}

\begin{prp}{}{} Let $G$ be a Lie group. A subgroup $H$ of $G$ has the unique structure of a Lie subgroup if $H$ is closed in $G$. 
\end{prp}

\subsection{Relation between Lie Groups and Lie Algebras}
For a group $G$, denote the left multiplication map of $h\in G$ by $l_h$. If $G$ is a Lie group, we have seen that $l_h$ is a smooth map, and so it induces a differential $(l_h)_\ast$. 

\begin{defn}{Left Invariant Vector Field}{} Let $G$ be a Lie group and $X$ a vector field on $G$. We say that $X$ is left invariant if $$(l_h)_\ast(X_g)=X_{hg}$$ for all $X_g\in T_g(G)$. 
\end{defn}

\begin{prp}{}{} Let $G$ be a Lie group. The vector space of left invariant vector fields of $G$ is a Lie algebra of dimension $\dim(G)$. Moreover, if $X_e\in T_e(G)$ is a tangent vector at $e$ the identity, then there is a unique left invariant vector field $X$ on $G$ such that its identity is $X_e$. 
\end{prp}

\begin{defn}{Lie Algebra of a Lie Group}{} Let $G$ be a Lie group. Define the Lie algebra $V$ of $G$ to be the vector space $T_e(G)$. 
\end{defn}

Recall that given a homomorphism of Lie groups $\phi:G\to H$, it induces a differential $\phi_\ast:T_g(G)\to T_{\phi(g)}(H)$. 

\begin{prp}{}{} Let $\phi:G\to H$ be a homomorphism of Lie groups with Lie algebras $V$ and $W$ respectively. Then the induced map from the differential $\phi_\ast:V\to W$ is a Lie algebra homomorphism. 
\end{prp}

\pagebreak
\section{Root Systems}

\pagebreak
\section{Representation Theory of Lie Algebras}
\begin{defn}{Representations of a Lie Algebra}{} Let $L$ be a Lie algebra. Let $V$ be a vector space. A representation of $L$ is a Lie algebra homomorphism $$\rho:L\to GL(V)$$
\end{defn}

\begin{lmm}{}{} Let $L$ be a Lie algebra. Then the adjoint homomorphism $$\text{ad}:L\to GL(L)$$ is a representation of $L$. 
\end{lmm}

\subsection{Weights}
\begin{defn}{Eigenvectors of GL(V)}{} Let $V$ be a vector space over a field $k$. Let $M$ be a Lie subalgebra of $GL(V)$. We say that $v\in V$ is an eigenvector of $M$ if for all $T\in M$, $v$ is an eigenvector of $T$ in the sense of Linear Algebra. 
\end{defn}

This notion of eigenvectors for Lie algebras is different to the standard notion of eigenvectors in linear algebra. Notice that an eigenvector of a Lie algebra $M$ is a vector $v\in V$ that is simultaneously an eigenvector of all linear maps $T\in M\leq GL(V)$. Now we can rephrase this in another way. 

\begin{lmm}{}{} Let $V$ be a vector space over a field $k$. Let $M$ be a Lie subalgebra of $GL(V)$. Then $v\in V$ is an eigenvector of $M$ if and only if there exists a linear map $\lambda:M\to k$ such that $$T(v)=\lambda(T)v$$ for all $T\in M$. 
\end{lmm}

We use the existence of a linear map $\lambda:M\to k$, conditional on $v\in V$, to determine whether $v$ is an eigenvector of $M$. 

\begin{defn}{Subspace of Eigenvectors of GL(V)}{} Let $V$ be a vector space over a field $k$. Let $M$ be a Lie subalgebra of $GL(V)$. Define the subspace of eigenvectors of $M$ by $$V_\lambda=\{v\in V\;|\;T(v)=\lambda(T)v\text{ for all }T\in M\}$$
\end{defn}

\begin{defn}{Weights}{} Let $V$ be a vector space over a field $k$. Let $M$ be a Lie subalgebra of $GL(V)$. We say that a linear map $\lambda:M\to k$ is a weight of $M$ if $V_\lambda\neq 0$. 
\end{defn}

\begin{lmm}{}{} Let $V$ be a vector space over a field $k$. Let $M$ be a Lie subalgebra of $GL(V)$. Let $I$ be an ideal of $M$. Let $W=\{w\in V\;|\;T(w)=0\text{ for all }T\in I\}$. Then $W$ is an $M$-invariant subspace of $V$. 
\end{lmm}

\pagebreak
\section{Classification of Semisimple Lie Algebras over $\C$}












\end{document}
