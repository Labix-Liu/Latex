\documentclass[a4paper]{article}

%=========================================
% Packages
%=========================================
\usepackage{mathtools}
\usepackage{amsfonts}
\usepackage{amsmath}
\usepackage{amssymb}
\usepackage{amsthm}
\usepackage[a4paper, total={6in, 8in}, margin=1in]{geometry}
\usepackage[utf8]{inputenc}
\usepackage{fancyhdr}
\usepackage[utf8]{inputenc}
\usepackage{graphicx}
\usepackage{physics}
\usepackage[listings]{tcolorbox}
\usepackage{hyperref}
\usepackage{tikz-cd}
\usepackage{adjustbox}
\usepackage{enumitem}


\hypersetup{
    colorlinks=true, %set true if you want colored links
    linktoc=all,     %set to all if you want both sections and subsections linked
    linkcolor=black,  %choose some color if you want links to stand out
}
\usetikzlibrary{arrows.meta}

\DeclarePairedDelimiter\ceil{\lceil}{\rceil}
\DeclarePairedDelimiter\floor{\lfloor}{\rfloor}

%=========================================
% Custom Math Operators
%=========================================
\DeclareMathOperator{\adj}{adj}
\DeclareMathOperator{\im}{im}
\DeclareMathOperator{\nullity}{nullity}
\DeclareMathOperator{\sign}{sign}
\DeclareMathOperator{\dom}{dom}
\DeclareMathOperator{\lcm}{lcm}
\DeclareMathOperator{\ran}{ran}
\DeclareMathOperator{\ext}{Ext}
\DeclareMathOperator{\dist}{dist}
\DeclareMathOperator{\diam}{diam}
\DeclareMathOperator{\aut}{Aut}
\DeclareMathOperator{\inn}{Inn}
\DeclareMathOperator{\syl}{Syl}
\DeclareMathOperator{\edo}{End}
\DeclareMathOperator{\cov}{Cov}
\DeclareMathOperator{\vari}{Var}
\DeclareMathOperator{\cha}{char}
\DeclareMathOperator{\Span}{span}
\DeclareMathOperator{\ord}{ord}
\DeclareMathOperator{\res}{res}
\DeclareMathOperator{\Hom}{Hom}
\DeclareMathOperator{\Mor}{Mor}
\DeclareMathOperator{\coker}{coker}
\DeclareMathOperator{\Obj}{Obj}
\DeclareMathOperator{\id}{id}
\DeclareMathOperator{\GL}{GL}
\DeclareMathOperator*{\colim}{colim}

%=========================================
% Custom Commands (Shortcuts)
%=========================================
\newcommand{\CP}{\mathbb{CP}}
\newcommand{\GG}{\mathbb{G}}
\newcommand{\F}{\mathbb{F}}
\newcommand{\N}{\mathbb{N}}
\newcommand{\Q}{\mathbb{Q}}
\newcommand{\R}{\mathbb{R}}
\newcommand{\C}{\mathbb{C}}
\newcommand{\E}{\mathbb{E}}
\newcommand{\Prj}{\mathbb{P}}
\newcommand{\RP}{\mathbb{RP}}
\newcommand{\T}{\mathbb{T}}
\newcommand{\Z}{\mathbb{Z}}
\newcommand{\A}{\mathbb{A}}
\renewcommand{\H}{\mathbb{H}}

\newcommand{\mA}{\mathcal{A}}
\newcommand{\mB}{\mathcal{B}}
\newcommand{\mC}{\mathcal{C}}
\newcommand{\mD}{\mathcal{D}}
\newcommand{\mE}{\mathcal{E}}
\newcommand{\mF}{\mathcal{F}}
\newcommand{\mG}{\mathcal{G}}
\newcommand{\mH}{\mathcal{H}}
\newcommand{\mJ}{\mathcal{J}}
\newcommand{\mO}{\mathcal{O}}
\newcommand{\mS}{\mathcal{S}}

%=========================================
% Theorem Environment
%=========================================
\newcommand\todoin[2][]{\todo[backgroundcolor=white!20!white, inline, caption={2do}, #1]{
\begin{minipage}{\textwidth-4pt}#2\end{minipage}}}

\tcbuselibrary{listings, theorems, breakable, skins}

\newtcbtheorem[number within=subsection]{thm}{Theorem}%
{colback=gray!5, colframe=gray!65!black, fonttitle=\bfseries, breakable, enhanced jigsaw, halign=left}{th}
\newtcbtheorem[number within=subsection, use counter from=thm]{defn}{Definition}%
{colback=gray!5, colframe=gray!65!black, fonttitle=\bfseries, breakable, enhanced jigsaw, halign=left}{th}
\newtcbtheorem[number within=subsection, use counter from=thm]{axm}{Axiom}%
{colback=gray!5, colframe=gray!65!black, fonttitle=\bfseries, breakable, enhanced jigsaw, halign=left}{th}
\newtcbtheorem[number within=subsection, use counter from=thm]{prp}{Proposition}%
{colback=gray!5, colframe=gray!65!black, fonttitle=\bfseries, breakable, enhanced jigsaw, halign=left}{th}
\newtcbtheorem[number within=subsection, use counter from=thm]{lmm}{Lemma}%
{colback=gray!5, colframe=gray!65!black, fonttitle=\bfseries, breakable, enhanced jigsaw, halign=left}{th}
\newtcbtheorem[number within=subsection, use counter from=thm]{crl}{Corollary}%
{colback=gray!5, colframe=gray!65!black, fonttitle=\bfseries, breakable, enhanced jigsaw, halign=left}{th}
\newtcbtheorem[number within=subsection, use counter from=thm]{eg}{Example}%
{colback=gray!5, colframe=gray!65!black, fonttitle=\bfseries, breakable, enhanced jigsaw, halign=left}{th}
\newtcbtheorem[number within=subsection, use counter from=thm]{ex}{Exercise}%
{colback=gray!5, colframe=gray!65!black, fonttitle=\bfseries, breakable, enhanced jigsaw, halign=left}{th}
\newtcbtheorem[number within=subsection, use counter from=thm]{alg}{Algorithm}%
{colback=gray!5, colframe=gray!65!black, fonttitle=\bfseries, breakable, enhanced jigsaw, halign=left}{th}

\newcounter{qtnc}
\newtcolorbox[use counter=qtnc]{qtn}%
{colback=gray!5, colframe=gray!65!black, fonttitle=\bfseries, breakable, enhanced jigsaw, halign=left}




\raggedright

\pagestyle{fancy}
\fancyhf{}
\rhead{Labix}
\lhead{Lie Groups and Lie Algebra}
\rfoot{\thepage}

\title{Lie Groups and Lie Algebra}

\author{Labix}

\date{\today}
\begin{document}
\maketitle
\begin{abstract}
\end{abstract}
\pagebreak
\tableofcontents
\pagebreak

\section{Lie Groups}
\subsection{Lie Groups}
\begin{defn}{Lie Groups}{} A Lie group $G$ is a smooth manifold which is also a group such that the multiplication map $G\times G\to G$ given by $(g,h)\mapsto gh$ and the inverse map $i:G\to G$ given by $g\mapsto g^{-1}$ are smooth maps. 
\end{defn}

\begin{prp}{}{} Let $G$ be a Lie group. A subgroup $H$ of $G$ has the unique structure of a Lie subgroup if $H$ is closed in $G$. 
\end{prp}

\subsection{Lie Algebras}
\begin{defn}{Lie Algebras}{} A Lie algebra is a vector space $V$ over a field $K$ together with a bilinear map $[-,-]:V\times V\to V$ such that for all $X,Y,Z\in V$, we have the following. 
\begin{itemize}
\item Anti-commutativity: $[X,Y]=-[Y,X]$
\item Jacobi identity: $[[X,Y],Z]+[[Y,Z],X]+[[Z,X],Y]=0$
\end{itemize}
\end{defn}

\begin{defn}{Homomorphism of Lie algebra}{} Let $V$ and $W$ be Lie algebras over $K$. A homomorphism from $V$ to $W$ is an $K$-linear map $F:V\to W$ such that $$[F(a),F(b)]=[a,b]$$ for all $a,b\in V$. 
\end{defn}

\begin{defn}{Lie Subalgebra}{} Let $V$ be a Lie algebra over $K$. A lie subalgebra of $V$ is a vector subspace $W$ of $V$ such that $$[w_1,w_2]\in W$$ for all $w_1,w_2\in W$. 
\end{defn}

It is clear that a Lie subalgebra is also a Lie algebra in its own right. Moreover, the inclusion $W\to V$ is a Lie algebra homomorphism. 

\begin{defn}{Ideal}{} Let $V$ be a Lie algebra over $K$. An ideal in $V$ is a subspace $U$ of $V$ such that $$[v,u]\in U$$ for all $v\in V$ and $u\in U$. 
\end{defn}

\begin{prp}{}{} Let $V$ be a Lie algebra over $K$ and $U$ an ideal of $V$. Then $V/U$ has a unique Lie algebra structure such that the quotient map $V\to V/U$ is a Lie algebra homomorphism. 
\end{prp}

\subsection{Relation between Lie Groups and Lie Algebras}
For a group $G$, denote the left multiplication map of $h\in G$ by $l_h$. If $G$ is a Lie group, we have seen that $l_h$ is a smooth map, and so it induces a differential $(l_h)_\ast$. 

\begin{defn}{Left Invariant Vector Field}{} Let $G$ be a Lie group and $X$ a vector field on $G$. We say that $X$ is left invariant if $$(l_h)_\ast(X_g)=X_{hg}$$ for all $X_g\in T_g(G)$. 
\end{defn}

\begin{prp}{}{} Let $G$ be a Lie group. The vector space of left invariant vector fields of $G$ is a Lie algebra of dimension $\dim(G)$. Moreover, if $X_e\in T_e(G)$ is a tangent vector at $e$ the identity, then there is a unique left invariant vector field $X$ on $G$ such that its identity is $X_e$. 
\end{prp}

\begin{defn}{Lie Algebra of a Lie Group}{} Let $G$ be a Lie group. Define the Lie algebra $V$ of $G$ to be the vector space $T_e(G)$. 
\end{defn}

Recall that given a homomorphism of Lie groups $\phi:G\to H$, it induces a differential $\phi_\ast:T_g(G)\to T_{\phi(g)}(H)$. 

\begin{prp}{}{} Let $\phi:G\to H$ be a homomorphism of Lie groups with Lie algebras $V$ and $W$ respectively. Then the induced map from the differential $\phi_\ast:V\to W$ is a Lie algebra homomorphism. 
\end{prp}












\end{document}
