\documentclass[a4paper]{article}

%=========================================
% Packages
%=========================================
\usepackage{mathtools}
\usepackage{amsfonts}
\usepackage{amsmath}
\usepackage{amssymb}
\usepackage{amsthm}
\usepackage[a4paper, total={6in, 8in}, margin=1in]{geometry}
\usepackage[utf8]{inputenc}
\usepackage{fancyhdr}
\usepackage[utf8]{inputenc}
\usepackage{graphicx}
\usepackage{physics}
\usepackage[listings]{tcolorbox}
\usepackage{hyperref}
\usepackage{tikz-cd}
\usepackage{adjustbox}
\usepackage{enumitem}
\usepackage[font=small,labelfont=bf]{caption}
\usepackage{subcaption}
\usepackage{wrapfig}
\usepackage{makecell}



\raggedright

\usetikzlibrary{arrows.meta}

\DeclarePairedDelimiter\ceil{\lceil}{\rceil}
\DeclarePairedDelimiter\floor{\lfloor}{\rfloor}

%=========================================
% Fonts
%=========================================
\usepackage{tgpagella}
\usepackage[T1]{fontenc}


%=========================================
% Custom Math Operators
%=========================================
\DeclareMathOperator{\adj}{adj}
\DeclareMathOperator{\im}{im}
\DeclareMathOperator{\nullity}{nullity}
\DeclareMathOperator{\sign}{sign}
\DeclareMathOperator{\dom}{dom}
\DeclareMathOperator{\lcm}{lcm}
\DeclareMathOperator{\ran}{ran}
\DeclareMathOperator{\ext}{Ext}
\DeclareMathOperator{\dist}{dist}
\DeclareMathOperator{\diam}{diam}
\DeclareMathOperator{\aut}{Aut}
\DeclareMathOperator{\inn}{Inn}
\DeclareMathOperator{\syl}{Syl}
\DeclareMathOperator{\edo}{End}
\DeclareMathOperator{\cov}{Cov}
\DeclareMathOperator{\vari}{Var}
\DeclareMathOperator{\cha}{char}
\DeclareMathOperator{\Span}{span}
\DeclareMathOperator{\ord}{ord}
\DeclareMathOperator{\res}{res}
\DeclareMathOperator{\Hom}{Hom}
\DeclareMathOperator{\Mor}{Mor}
\DeclareMathOperator{\coker}{coker}
\DeclareMathOperator{\Obj}{Obj}
\DeclareMathOperator{\id}{id}
\DeclareMathOperator{\GL}{GL}
\DeclareMathOperator*{\colim}{colim}

%=========================================
% Custom Commands (Shortcuts)
%=========================================
\newcommand{\CP}{\mathbb{CP}}
\newcommand{\GG}{\mathbb{G}}
\newcommand{\F}{\mathbb{F}}
\newcommand{\N}{\mathbb{N}}
\newcommand{\Q}{\mathbb{Q}}
\newcommand{\R}{\mathbb{R}}
\newcommand{\C}{\mathbb{C}}
\newcommand{\E}{\mathbb{E}}
\newcommand{\Prj}{\mathbb{P}}
\newcommand{\RP}{\mathbb{RP}}
\newcommand{\T}{\mathbb{T}}
\newcommand{\Z}{\mathbb{Z}}
\newcommand{\A}{\mathbb{A}}
\renewcommand{\H}{\mathbb{H}}
\newcommand{\K}{\mathbb{K}}

\newcommand{\mA}{\mathcal{A}}
\newcommand{\mB}{\mathcal{B}}
\newcommand{\mC}{\mathcal{C}}
\newcommand{\mD}{\mathcal{D}}
\newcommand{\mE}{\mathcal{E}}
\newcommand{\mF}{\mathcal{F}}
\newcommand{\mG}{\mathcal{G}}
\newcommand{\mH}{\mathcal{H}}
\newcommand{\mI}{\mathcal{I}}
\newcommand{\mJ}{\mathcal{J}}
\newcommand{\mK}{\mathcal{K}}
\newcommand{\mL}{\mathcal{L}}
\newcommand{\mM}{\mathcal{M}}
\newcommand{\mO}{\mathcal{O}}
\newcommand{\mP}{\mathcal{P}}
\newcommand{\mS}{\mathcal{S}}
\newcommand{\mT}{\mathcal{T}}
\newcommand{\mV}{\mathcal{V}}
\newcommand{\mW}{\mathcal{W}}

%=========================================
% Colours!!!
%=========================================
\definecolor{LightBlue}{HTML}{2D64A6}
\definecolor{ForestGreen}{HTML}{4BA150}
\definecolor{DarkBlue}{HTML}{000080}
\definecolor{LightPurple}{HTML}{cc99ff}
\definecolor{LightOrange}{HTML}{ffc34d}
\definecolor{Buff}{HTML}{DDAE7E}
\definecolor{Sunset}{HTML}{F2C57C}
\definecolor{Wenge}{HTML}{584B53}
\definecolor{Coolgray}{HTML}{9098CB}
\definecolor{Lavender}{HTML}{D6E3F8}
\definecolor{Glaucous}{HTML}{828BC4}
\definecolor{Mauve}{HTML}{C7A8F0}
\definecolor{Darkred}{HTML}{880808}
\definecolor{Beaver}{HTML}{9A8873}
\definecolor{UltraViolet}{HTML}{52489C}



%=========================================
% Theorem Environment
%=========================================
\tcbuselibrary{listings, theorems, breakable, skins}

\newtcbtheorem[number within = subsection]{thm}{Theorem}%
{	colback=Buff!3, 
	colframe=Buff, 
	fonttitle=\bfseries, 
	breakable, 
	enhanced jigsaw, 
	halign=left
}{thm}

\newtcbtheorem[number within=subsection, use counter from=thm]{defn}{Definition}%
{  colback=cyan!1,
    colframe=cyan!50!black,
	fonttitle=\bfseries, breakable, 
	enhanced jigsaw, 
	halign=left
}{defn}

\newtcbtheorem[number within=subsection, use counter from=thm]{axm}{Axiom}%
{	colback=red!5, 
	colframe=Darkred, 
	fonttitle=\bfseries, 
	breakable, 
	enhanced jigsaw, 
	halign=left
}{axm}

\newtcbtheorem[number within=subsection, use counter from=thm]{prp}{Proposition}%
{	colback=LightBlue!3, 
	colframe=Glaucous, 
	fonttitle=\bfseries, 
	breakable, 
	enhanced jigsaw, 
	halign=left
}{prp}

\newtcbtheorem[number within=subsection, use counter from=thm]{lmm}{Lemma}%
{	colback=LightBlue!3, 
	colframe=LightBlue!60, 
	fonttitle=\bfseries, 
	breakable, 
	enhanced jigsaw, 
	halign=left
}{lmm}

\newtcbtheorem[number within=subsection, use counter from=thm]{crl}{Corollary}%
{	colback=LightBlue!3, 
	colframe=LightBlue!60, 
	fonttitle=\bfseries, 
	breakable, 
	enhanced jigsaw, 
	halign=left
}{crl}

\newtcbtheorem[number within=subsection, use counter from=thm]{eg}{Example}%
{	colback=Beaver!5, 
	colframe=Beaver, 
	fonttitle=\bfseries, 
	breakable, 
	enhanced jigsaw, 
	halign=left
}{eg}

\newtcbtheorem[number within=subsection, use counter from=thm]{ex}{Exercise}%
{	colback=Beaver!5, 
	colframe=Beaver, 
	fonttitle=\bfseries, 
	breakable, 
	enhanced jigsaw, 
	halign=left
}{ex}

\newtcbtheorem[number within=subsection, use counter from=thm]{alg}{Algorithm}%
{	colback=UltraViolet!5, 
	colframe=UltraViolet, 
	fonttitle=\bfseries, 
	breakable, 
	enhanced jigsaw, 
	halign=left
}{alg}




%=========================================
% Hyperlinks
%=========================================
\hypersetup{
    colorlinks=true, %set true if you want colored links
    linktoc=all,     %set to all if you want both sections and subsections linked
    linkcolor=DarkBlue,  %choose some color if you want links to stand out
}


\pagestyle{fancy}
\fancyhf{}
\rhead{Labix}
\lhead{Simplicial Methods in Topology}
\rfoot{\thepage}

\title{Simplicial Methods in Topology}

\author{Labix}

\date{\today}
\begin{document}
\maketitle
\begin{abstract}

\end{abstract}
References: 

\pagebreak
\tableofcontents

\pagebreak

\section{Simplicial Sets and Simplicial Objects}
\subsection{The Simplex Category}
Recall the simplex category in Category Theory 1. 

\begin{defn}{Simplex Category}{} The simplex category $\Delta$ consists of the following data. 
\begin{itemize}
\item The objects are $[n]=\{0,\dots,n\}$ for $n\in\N$. 
\item The morphisms are the non-strictly order preserving functions. This means that a morphism $f:[n]\to[m]$ must satisfy $f(i)\leq f(j)$ for all $i\leq j$. 
\item Composition is the usual composition of functions. 
\end{itemize}
\end{defn}

\begin{defn}{Maps in the Simplex Category}{} Consider the simplex category $\Delta$. Define the face maps and the degeneracy maps as follows. 
\begin{itemize}
\item A face map in $\Delta$ is the unique morphism $d^i:[n-1]\to[n]$ that is injective and whose image does not contain $i$. Explicitly, we have $$d^i(k)=\begin{cases}
k & \text{ if } 0\leq k <i\\
k+1 & \text{ if } i\leq k\leq n-1
\end{cases}$$
\item A degeneracy map in $\Delta$ is the unique morphism $s^i:[n+1]\to[n]$ that is surjective and hits $i$ twice. Explicitly, we have $$s^i(k)=\begin{cases}
k & \text{ if } 0\leq k\leq i\\
k-1 & \text{ if } i+1\leq k\leq n+1
\end{cases}$$
\end{itemize}
\end{defn}

\begin{prp}{}{} The face maps and the degeneracy maps in the simplex category $\Delta$ satisfy the following simplicial identities: 
\begin{itemize}
\item $d^j\circ d^i=d^i\circ d^{j-1}$ if $i<j$
\item $s^j\circ s^i=s^i\circ s^{j+1}$ if $i\leq j$
\item $s^j\circ d^i=\begin{cases}
\text{id} & \text{ if } i=j\text{ or }j+1\\
d^i\circ s^{j-1} & \text{ if } i<j\\
d^{i-1}\circ s^j & \text{ if }i>j+1
\end{cases}$
\end{itemize} \tcbline
\begin{proof}~\\
\begin{itemize}
\item Consider the object $[n-1]$. On the left, we have that 
\begin{align*}
d^j(d^i[n-1])&=d^j([0,\dots,i-1,i+1,\dots,n])\\
&=[0,\dots,i-1,i+1,\dots,j-1,j+1,\dots,n+1]
\end{align*}
On the right, we have that
\begin{align*}
d^i(d^{j-1}[n-1])&=d^i([0,\dots,j-2,j,\dots,n])\\
&=[0,\dots,i-1,i+1,\dots,j-1,j+1,\dots,n+1]
\end{align*} and so the relation is indeed true. 
\end{itemize}
\end{proof}
\end{prp}

\begin{prp}{}{} Every morphism in the simplex category $\Delta$ is a composition of the face maps and the degeneracy maps. 
\end{prp}

\subsection{Simplicial Sets}
\begin{defn}{Simplicial Sets}{} A simplicial set is a presheaf $$S:\Delta^\text{op}\to\bold{Set}$$ Define the $n$-simplicies of $S$ to be the set $$S_n=S([n])$$
\end{defn}

\begin{eg}{}{} Let $X$ be a topological space. The set of singular simplices of $X$ is given by the presheaf $$S:\Delta\to\bold{Set}$$ defined by $[n]\mapsto\Hom_\bold{Top}(\abs{\Delta^n},X)$. In other words, an $n$-simplex of $X$ is simply a continuous function $\sigma:\Delta^n\to X$. This is exactly the same as how we defined singular $n$-simplexes in Algebraic Topology 2. 
\end{eg}

\begin{defn}{Category of Simplicial Sets}{} The category of simplicial sets $\text{sSet}$ is defined as follows. 
\begin{itemize}
\item The objects are simplicial sets $S:\Delta\to\text{Sets}$
\item The morphisms are just morphisms of presheaves. This means that if $S,T:\Delta\to\text{Sets}$ are simplicial sets, then a morphism $\lambda:S\to T$ consists of morphisms $\lambda_n:S([n])\to T([n])$ for $n\in\N$ such that the following diagram commutes: \\~\\
\adjustbox{scale=1.0,center}{\begin{tikzcd}
	{S([n])} & {S([m])} \\
	{T([n])} & {T([m])}
	\arrow["{S(f)}", from=1-1, to=1-2]
	\arrow["{\lambda_n}"', from=1-1, to=2-1]
	\arrow["{\lambda_m}", from=1-2, to=2-2]
	\arrow["{T(f)}"', from=2-1, to=2-2]
\end{tikzcd}}
\item Composition is defined as the usual composition of functors. 
\end{itemize}
\end{defn}

The Yoneda lemma in this context implies that there is a bijection $$\Hom_{\text{sSet}}(\Hom_\Delta(-,[n]),S)\cong S([n])=S_n$$ that is natural in the variable $[n]$. 

\begin{defn}{Standard $n$-Simplicies}{} Let $n\in\N$. Define the standard $n$-simplex to be the simplicial set $$\Delta^n=\Hom_\Delta(-,[n]):\Delta\to\bold{Set}$$
\end{defn}

Notice that $\Delta^n$ is a simplicial set $$\Delta^n:\Delta\to\text{Set}$$ defined by $[m]\mapsto\Hom_\Delta([m],[n])$. Notice that if $m>n$, then it is impossible to have an order preserving function $[m]\to[n]$. Hence when $m>n$, $\Hom_\Delta([m],[n])$ is empty. \\

All such simplicial sets $\Delta^n$ are useful in determining the contents of an arbitrary simplicial set. As for any presheaf, instead of focusing between the passage of data from $\Delta$ to $\text{Set}$, we should instead think of what kind of structure the presheaf brings to $\text{Set}$. Let $C$ be a simplicial set. Then this means the following. For each $n$, there is a set $C_n=\Hom_\text{sSet}(\Delta^n,C)$. For each morphism in $\Delta$, there is a corresponding morphism in $\text{Set}$, which we shall discuss now. 

\begin{prp}{}{} Let $S:\Delta\to\text{Set}$ be a simplicial set. Then every morphism in $S(\Delta)$ is the composite of two kinds of maps: 
\begin{itemize}
\item The face maps: $d_i:S_n\to S_{n-1}$ for $0\leq i\leq n$ defined by $$d_i=S(d^i:[n-1]\to[n])$$
\item The degeneracy maps: $s_i:S_n\to S_{n+1}$ for $0\leq i\leq n$ defined by $$s_i=S(s^i:[n+1]\to[n])$$
\end{itemize}
Moreover, these maps satisfy the following simplicial identities: 
\begin{itemize}
\item $d_i\circ d_j=d_{j-1}\circ d_i$ if $i<j$
\item $s_i\circ s_j=s_{j+1}\circ s_i$ if $i\leq j$
\item $d_i\circ s_j=\begin{cases}
\text{id} & \text{ if } i=j\text{ or }j+1\\
s_{j-1}\circ d_i & \text{ if } i<j\\
s_j\circ d_{i-1} & \text{ if }i>j+1
\end{cases}$
\end{itemize} \tcbline
\begin{proof}
Results are immediate using prp 1.1.3 and the fact that $S$ is contravariant. 
\end{proof}
\end{prp}

We can now explicitly determine a simplicial set using the above proposition. We can alternatively define a simplicial set to be the following data. 
\begin{itemize}
\item For each $n\in\N$, a set $S_n$. 
\item For $n\in\N$ and $0\leq i\leq n$, a face map $$d_i^n:S_n\to S_{n-1}$$ determining the $i$th face of the $n$-simplex
\item For $n\in\N$ and $0\leq i\leq n$, a degeneracy map $$s_i^n:S_n\to S_{n+1}$$
\end{itemize}

The above data is sufficient to determine a unique simplicial set. 

\begin{prp}{}{} The category $\bold{sSet}$ is a symmetric monoidal category with level-wise cartesian product. 
\end{prp}

Recall the notion of a $\Delta$-set from Algebraic Topology 2 and one might realize they look suspiciously similar to that of a simplicial set. Let us recall. A $\Delta$-set is a collection of sets $S_n$ for $n\in\N$ together with maps $d_i^n:S_n\to S_{n-1}$ for $0\leq i\leq n$ such that $$d_i^{n-1}\circ d_j^n=d_{j-1}^{n-1}\circ d_i^n$$ for $i<j$. One can easily convince themselves that every simplicial set is a $\Delta$-set. Indeed, a simplicial set satisfies five more relations than a $\Delta$-set. Therefore we have that $$\bold{sSet}\subset\Delta\text{ Complexes}$$

\begin{prp}{}{} Every simplicial set is a $\Delta$-set. 
\end{prp}

Combining with the previously learnt combinatorial objects in algebraic topology, we now have the following tower:  $$\text{Simplicial Complexes}\subset\bold{sSet}\subset\Delta\text{ Complexes}\subset\bold{CW}$$

\begin{defn}{Faces of a Simplex}{} Let $n\in\N$ and consider the standard $n$-simplex $\Delta^n$. 
\begin{itemize}
\item Denote $\partial_i\Delta^n\subset\Delta^n$ the simplicial subset generated by the $i$th face $$d_i(\text{id}:[n]\to[n])=d^i:[n-1]\to[n]$$
\item Denote $\partial\Delta^n$ the simplicial subset generated by the faces $\partial_i\Delta^n$ for $0\leq i\leq n$. Define $\partial\Delta^0=\emptyset$. 
\end{itemize}
\end{defn}

\begin{defn}{Degenerate Simplices}{} Let $S:\Delta\to\bold{Set}$ be a simplicial set. Let $x\in S_n$ be an $n$-simplex. We say that $x$ is degenerate if if $x\in\im(s_k)$ for some degenerate map $s_k$. 
\end{defn}

Intuitively, degenerate simplices refer to simplicies that consecutively have the same face. Indeed, if $x=s_k(y)$ for some $k$, then the $k$th and $k+1$th position of the simplex $x$ is given by the same element. Collapsing the two vertices should return the simplex $y$. 

\subsection{Geometric Realization of Simplicial Sets}
\begin{defn}{Geometric Realization of Standard n-Simplexes}{} Let $n\in\N$. Consider the standard $n$-simplex $\Delta^n$. Define the geometric realization of $\Delta^n$ to be $$\abs{\Delta^n}=\left\{\sum_{k=0}^nt_kv_k\bigg{|}\sum_{k=0}^nt_k=1\text{ and }t_k\geq 0\text{ for all }k=0,\dots,n\right\}$$
\end{defn}

This definition is exactly the same as the definition of an $n$-simplex in Algebraic Topology 2. Now we proceed to the general case. 

\begin{defn}{Geometric Realization of Simplicial Sets}{} Let $C$ be a simplicial set. Define the geometric realization of $C$ to be $$\abs{C}=\left(\coprod_{n\geq 0}C_n\times\abs{\Delta^n}\right)/\sim$$ where the equivalence relation is generated by the following. 
\begin{itemize}
\item The $i$th face of $\{x\}\times\abs{\Delta^n}$ is identified with $\{d_ix\}\times\abs{\Delta^{n-1}}$ by the linear homeomorphism preserving the order of the vertices. 
\item $\{s_ix\}\times\abs{\Delta^n}$ is collapsed onto $\{x\}\times\abs{\Delta^{n-1}}$ via the linear projection parallel to the line connecting the $i$th and the $(i+1)$st vertiex. 
\end{itemize}
\end{defn}

This construction of geometric realization is moreover functorial. Once again, we first define a map of geometric realization of simplicial sets. 

\begin{defn}{Induced Map of Geometric Realization of Standard Simplicial Sets}{} Let $f:\Delta^n\to\Delta^m$ be a map of standard simplexes. Define $f_\ast:\abs{\Delta^n}\to\abs{\Delta^m}$ by $$(t_0,\dots,t_n)\mapsto(s_0,\dots,s_m)$$ where $$s_i=\begin{cases}
0 & \text{ if } f^{-1}(i)=0\\
\sum_{j\in f^{-1}(i)}t_j & \text{ otherwise }
\end{cases}$$
\end{defn}

\begin{thm}{}{} The geometric realization of a simplicial set is functorial $\abs{\;\cdot\;}:\text{sSet}\to\text{Top}$ in the following way. 
\begin{itemize}
\item On objects, it sends any simplicial set $C$ to its geometric realization $\abs{C}$. 
\item On morphisms, it sends any morphism $C\to D$ of simplicial sets to a continuous map defined by 
\end{itemize}
\end{thm}

We thus have that $$\substack{\text{Geometric Relizations}\\\text{ of simplicial sets}}\subset\substack{\text{Geometric Relizations}\\\text{ of }\Delta\text{-sets}}\subset\text{CW-Complexes}$$

\subsection{Simplicial and Semisimplicial Objects}
\begin{defn}{Simplicial Objects}{} Let $\mC$ be a category. A simplicial object in $\mC$ is a presheaf $S:\Delta^\text{op}\to\mC$. 
\end{defn}

Hence a simplicial object in $\bold{Set}$ is just simplical sets. 

\begin{defn}{Category of Simplicial Objects}{} Let $\mC$ be a category. Define the category of simplicial objects $\text{s}\mC$ of $\mC$ as follows. 
\begin{itemize}
\item The objects are simplicial objects $S:\Delta^\text{op}\to\mC$ of $\mC$ which are presheaves
\item The morphism of simplcial objects are just morphisms of presheaves, which are natural transformations
\item Composition is given by composition of natural transformations
\end{itemize}
\end{defn}

\begin{defn}{The Semisimplex Category}{} The Semisimplex category $\bold{SS}$ is the subcategory of $\Delta$ consisting of strict order preserving functions. 
\end{defn}

\begin{defn}{Semisimplicial Objects}{} Let $\mC$ be a category. A semisimplicial object in $\mC$ is a presheaf $$\bold{SS}\to\mC$$
\end{defn}

\begin{lmm}{}{} Let $S$ be a set. Then $S$ is a semisimplicial set if and only if $S$ is a $\Delta$-set (in the sense of Algebraic Topology 2). 
\end{lmm}

\pagebreak
\section{The Category of Simplicial Sets}
\subsection{The Singular Functor}
\begin{defn}{Singular Functor}{} The singular functor $S:\text{Top}\to\text{sSet}$ is defined as follows. 
\begin{itemize}
\item On objects, it sends a space $X$ to the simplicial set $S(X):\Delta\to\text{Set}$ called the singular set, whose $n$ vertices is given by $$S(X)[n]=\Hom_{\text{Top}}(\abs{\Delta^n},X)$$ For each non-decreasing map $\alpha:[m]\to[n]$ the pre-composition map $$\Hom_{\bold{sSet}}(\abs{\Delta^n},X)\overset{-\circ\abs{\alpha}}{\rightarrow}\Hom_{\bold{sSet}}(\abs{\Delta^m},X)$$
\item On morphisms, it sends a continuous map $f:X\to Y$ to the morphism of simplicial sets $\lambda:S(X)\to S(Y)$ defined as follows. For each $n\in\N$, $\lambda_n:S(X)[n]\to S(Y)[n]$ is defined by $$\left(h:\abs{\Delta^n}\to X\right)\mapsto\left(f\circ h:\abs{\Delta^n}\to Y\right)$$ such that the following diagram commutes: \\~\\
\adjustbox{scale=1.0,center}{\begin{tikzcd}
	{S(X)[n]} & {S(X)[m]} \\
	{S(Y)[n]} & {S(Y)[m]}
	\arrow["{S(X)(f)}", from=1-1, to=1-2]
	\arrow["{\lambda_n}"', from=1-1, to=2-1]
	\arrow["{\lambda_m}", from=1-2, to=2-2]
	\arrow["{S(Y)(f)}"', from=2-1, to=2-2]
\end{tikzcd}}\\~\\
\end{itemize}
\end{defn}

Notice that this is reminiscent of the definitions in Algebraic Topology 2. Indeed $S(X)[n]$  for each $n\in\N$ is in fact the basis of the abelian group $C_n(X)$. It represents all the possible ways that an $n$-simplex could fit into $X$. Then the passage $$\bold{Top}\overset{S}{\longrightarrow}\bold{sSet}\overset{H_\bullet^\Delta(-;R)}{\longrightarrow}{_R\bold{Mod}}$$ recovers the singular homology of a space $X$ with coefficients in a ring $R$. This is formulated slightly differently in Algebraic Topology 2. \\~\\

In particular, it is easy to see that $S(X)[0]$ recovers the set of points of $X$. 

\begin{thm}{}{} The singular functor $S:\text{Top}\to\text{sSet}$ is right adjoint to the geometric realization functor $\abs{\;\cdot\;}:\text{sSet}\to\text{Top}$. This means that there is a natural bijection $$\Hom_\text{Top}(\abs{X},Y)\cong\Hom_\text{sSet}(X,S(Y))$$ for any space $Y$ and any simplicial set $X$. 
\end{thm}

\subsection{Completeness and Cocompleteness}
\begin{defn}{Product of Simplicial Sets}{} Let $S,T$ be simplicial sets. Define the product of $S$ and $T$ to be the simplicial set $$S\times T:\Delta\to\bold{Set}$$ by the formula $$[n]\mapsto S_n\times T_n$$
\end{defn}

\begin{prp}{}{} Let $S,T$ be simplicial sets. Then the categorical product of $S$ and $T$ is precisely $S\times T$. 
\end{prp}

\begin{lmm}{}{} Let $S,T$ be simplicial sets. Then there is a canonical homeomorphism $$\abs{S\times T}\cong\abs{S}\times\abs{T}$$ given by geometric realization. 
\end{lmm}

\subsection{The Closed Monoidal Structure of $\bold{sSet}$}
\begin{defn}{Internal Hom of Simplicial Sets}{} Let $S,T$ be simplicial sets. Define the simplicial set $$[S,T]:\Delta\to\bold{Set}$$ by the formula $$[n]\mapsto\Hom_\bold{sSet}(\Delta^n\times S\to T)$$ By the Yoneda embedding this is well defined. 
\end{defn}

We can easily identify morphisms of simplicial sets as the vertices of the internal hom. 

\begin{defn}{Evaluation Morphism}{} Let $S,T$ be simplicial sets. Define the evaluation morphism $$\text{Ev}:[S,T]\times S\to T$$ as follows. For $f\in[S,T]$ and $n$-simplex and $\sigma\in S$ an $n$-simplex, define an $n$-simplex in $T$ by $$\Delta^n\overset{\delta}{\rightarrow}\Delta^n\times\Delta^n\overset{\text{id}\times\sigma}{\rightarrow}\Delta^n\times S\overset{f}{\rightarrow}T$$
\end{defn}

\begin{thm}{The Closed Monoidal Adjunction}{} Let $S,T,U$ be simpplicial sets. Then there is a natural isomorphism $$\Hom_\bold{sSet}(S\times T,U)\cong\Hom_\bold{sSet}(S,[T,U])$$ induced by the post composition of the evaluation momorphism (in the opposite way). Moreover $\bold{sSet}$ is a closed monoidal category. 
\end{thm}

\pagebreak
\section{Introduction to Kan Complexes}
\subsection{Horns and Fillers}
\begin{defn}{Inner and Outer Horns}{} Let $n\in\N$ and $0<i<n$. Define the $i$th horn $\Lambda_i^n$ of $\Delta^n$ to be the the simplicial subset generated by all the faces $\partial_k\Delta^n$ except the $i$th one. It is called inner if $0<i<n$. It is called outer otherwise. 
\end{defn}

\begin{defn}{Fillers for a Horn}{} Let $n\in\N$ and $0<i<n$. Let $\Lambda_i^n$ be a horn of $\Delta^n$. We say that $\Lambda$ admits a filler if for all maps $F:\Lambda_i^n\to C$ there exists a map $U:\Delta^n\to C$ such that the following diagram commutes: \\~\\
\adjustbox{scale=1.0,center}{\begin{tikzcd}
	{\Lambda_i^n} & C \\
	{\Delta^n}
	\arrow["F", from=1-1, to=1-2]
	\arrow[hook, from=1-1, to=2-1]
	\arrow["\exists U"', dashed, from=2-1, to=1-2]
\end{tikzcd}}
\end{defn}

\subsection{Kan Complexes}
\begin{defn}{Kan Complexes}{} Let $S$ be a simplicial set. We say that $S$ is a Kan complex if every inner horn admits a filler. 
\end{defn}

\begin{defn}{The Category of Kan Complexes}{} Define the category of Kan complexes $$\bold{Kan}$$ to be the full subcategory of $\bold{sSet}$ consisting of Kan complexes. 
\end{defn}

\begin{lmm}{}{} Let $X$ be a topological space. Then $S(X)$ is a Kan complex. 
\end{lmm}

In particular, the adjunction between the singular functor and geometric realization descends to an adjunction on Kan complexes. We would like to think of Kan complexes as some sort of generalization of $\bold{Top}$. 

\begin{defn}{Contractible Kan Complex}{} Let $X$ be a simplicial set. We say that $X$ is a contractible Kan complex if for every morphism $\sigma_0:\partial\Delta^n\to X$, there exists a morphism $\sigma:\Delta^n\to X$ such that the following diagram commutes: \\~\\
\adjustbox{scale=1.0,center}{\begin{tikzcd}
	{\partial\Delta^n} & X \\
	{\Delta^n}
	\arrow["{\sigma_0}", from=1-1, to=1-2]
	\arrow[hook, from=1-1, to=2-1]
	\arrow["\sigma"', dashed, from=2-1, to=1-2]
\end{tikzcd}}
\end{defn}

\begin{lmm}{}{} Let $X$ be a simplicial set. If $X$ is a contractible Kan complex, then $X$ is a Kan complex. 
\end{lmm}

\subsection{Lifting Properties}
\begin{defn}{The Left Lifting Property}{} Let $A,B,X,Y\in\bold{sSet}$ be simplicial sets. Let $p:X\to Y$ and $i:A\to B$ be morphisms. We say that $i$ has the has the left lifting property with respect to $p$ if the following is true. For all morphisms $f:A\to X$ and $g:B\to Y$ such that $p\circ f=g\circ i$, there exists $h:B\to Y$ such that the following diagram commutes: \\~\\
\adjustbox{scale=1.0,center}{\begin{tikzcd}
	A & X \\
	B & Y
	\arrow["f", from=1-1, to=1-2]
	\arrow["i"', from=1-1, to=2-1]
	\arrow["p", from=1-2, to=2-2]
	\arrow["{\exists h}", dashed, from=2-1, to=1-2]
	\arrow["g"', from=2-1, to=2-2]
\end{tikzcd}}\\~\\
\end{defn}

\begin{defn}{The Right Lifting Property}{} Let $A,B,X,Y\in\bold{sSet}$ be simplicial sets. Let $p:X\to Y$ and $i:A\to B$ be morphisms. We say that $p$ has the has the right lifting property with respect to $i$ if the following is true. For all morphisms $f:A\to X$ and $g:B\to Y$ such that $p\circ f=g\circ i$, there exists $h:B\to Y$ such that the following diagram commutes: \\~\\
\adjustbox{scale=1.0,center}{\begin{tikzcd}
	A & X \\
	B & Y
	\arrow["f", from=1-1, to=1-2]
	\arrow["i"', from=1-1, to=2-1]
	\arrow["p", from=1-2, to=2-2]
	\arrow["{\exists h}", dashed, from=2-1, to=1-2]
	\arrow["g"', from=2-1, to=2-2]
\end{tikzcd}}\\~\\
\end{defn}

\subsection{Kan Fibrations}
\begin{defn}{Kan Fibrations}{} Let $f:X\to Y$ be a morphism of simplicial sets. We say that $f$ is a Kan fibration if the following condition is satisfied: For every commutative diagram: \\~\\
\adjustbox{scale=1.0,center}{\begin{tikzcd}
	{\Lambda_k^n} & X \\
	{\Delta^n} & Y
	\arrow[from=1-1, to=1-2]
	\arrow[hook, from=1-1, to=2-1]
	\arrow["f", from=1-2, to=2-2]
	\arrow[from=2-1, to=2-2]
\end{tikzcd}}\\~\\
where $n\geq 1$ and $0\leq k\leq n$, there exists a lift $\Delta^n\to Y$ such that the following diagram commutes: \\~\\
\adjustbox{scale=1.0,center}{\begin{tikzcd}
	{\Lambda_k^n} & X \\
	{\Delta^n} & Y
	\arrow[from=1-1, to=1-2]
	\arrow[hook, from=1-1, to=2-1]
	\arrow["f", from=1-2, to=2-2]
	\arrow[dashed, from=2-1, to=1-2]
	\arrow[from=2-1, to=2-2]
\end{tikzcd}}\\~\\
\end{defn}

\begin{lmm}{}{} Let $X$ be a simplicial set. Then $X$ is a Kan complex if and only if the unique map $X\to\ast$ is a Kan fibration. 
\end{lmm}

Closed under retracts, pullbacks, filtered colimits, products with standard simplicies, composition

\pagebreak
\section{Simplicial Homotopy Theory}
Algebraic topology is a new subject which received a name change from combinatorial topology. At that time, combinatorics was highly involved in deriving topological invariants because the combinatorial structure of any invariants or spaces of study makes computation easy. In particular, a great deal of work has been put into the homotopy theory of simplicial spaces. \\

Nowadays, the study of combinatorial objects in topology is less prominent, but the category of simplicial sets still play a distinguished role in algebraic topology. Indeed the category of simplicial sets is the prototypical example of a model category (we have not seen it), as well as exhibiting a Quillen equivalence between the model of a simplicial set and some model category structure of $\bold{Top}$. We will not explore the concept of model categories but will instead display foundational knowledge of it in order to develop the category of simplicial sets as a workable category when studying Model Categories. This means that we will develop the notion of homotopy, fibrations and cofibrations in this category. 

\subsection{Homotopy of Simplicial Sets}
\begin{defn}{Homotopy between Two Morphisms}{} Let $X,Y$ be simplicial sets. Let $f,g:X\to Y$ be two morphisms. A simplicial homotopy from $f$ to $g$ is a morphism $\eta:X\times\Delta^1\to Y$ such that the following diagram commutes: \\~\\
\adjustbox{scale=1.0,center}{\begin{tikzcd}
	{X\simeq X\times\Delta[0]} & {X\times\Delta[1]} & {X\simeq X\times\Delta[0]} \\
	\\
	& Y
	\arrow["{\text{id}_X\times\delta_1}", from=1-1, to=1-2]
	\arrow["f"', from=1-1, to=3-2]
	\arrow["{\exists\eta}"{description}, dashed, from=1-2, to=3-2]
	\arrow["{\text{id}_X\times\delta_1}"', from=1-3, to=1-2]
	\arrow["g", from=1-3, to=3-2]
\end{tikzcd}}\\~\\
We say that $f$ and $g$ are homotopic either if there is a homotopy from $f$ to $g$, or there is a homotopy from $g$ to $f$. 
\end{defn}

\begin{defn}{Homotopy Relative to Subsets}{} Let $X,Y$ be simplicial sets. Let $K\subseteq X$ be a simplicial subset. Let $\iota:K\hookrightarrow X$ denote the inclusion. Let $f,g:X\to Y$ be two morphisms. Let $\eta:X\times\Delta^1\to Y$ be a homotopy from $f$ to $g$. We say that $\iota$ is a homotopy relative to $K$ if the following diagram commutes: \\~\\
\adjustbox{scale=1.0,center}{\begin{tikzcd}
	{K\times\Delta^1} & K \\
	{X\times\Delta^1} & Y
	\arrow["{\text{proj}_K}", from=1-1, to=1-2]
	\arrow["{\iota\times\text{id}_{\Delta^1}}"', hook, from=1-1, to=2-1]
	\arrow["\alpha", from=1-2, to=2-2]
	\arrow["\eta"', from=2-1, to=2-2]
\end{tikzcd}}\\~\\
where $\alpha=f|_K=g|_K$. 
\end{defn}

\begin{prp}{}{} Let $X,Y$ be simplicial sets. Let $f,g:X\to Y$ be morphisms. Then there exists a homotopy from $g$ to $f$ if and only if there exists a family of morphisms $h_i^n:X_n\to Y_{n+1}$ such that the following are true. 
\begin{itemize}
\item $d_0^n\circ h_0=f_n$
\item $d_0^{n+1}\circ h_1=g_n$
\item The composition $$d_i\circ h_j=\begin{cases}
h_{j-1}\circ d_i^{n+1} & \text{ if } 0\leq i<j\leq n\\
d_i\circ h_{i-1} & \text{ if } i=j\neq 0\\
h_j\circ d_{i-1}& \text{ if } i>j+1
\end{cases}$$
\item The composition $$s_i\circ h_j=\begin{cases}
h_{j+1}\circ s_i & \text{ if } i\leq j\\
h_j\circ s_{i-1} & \text{ if } i>j
\end{cases}$$
\end{itemize}
\end{prp}

The homotopy relation is in general not an equivalence relation. However, when $Y$ is a Kan complex, this is true. 

\begin{prp}{}{} Let $X$ be a simplcial set. Let $Y$ be a Kan complex. Then the relation of homotopic morphisms from $X$ to $Y$ is an equivalence relation. 
\end{prp}

\begin{defn}{Homotopy Inverses}{} Let $X,Y$ be simplicial sets. Let $f:X\to Y$ be a morphism. A homotopy inverse of $f$ is a morphism $g:Y\to X$ such that there exist a homotopy from $g\circ f$ to $\text{id}_X$ and a homotopy from $f\circ g$ to $\text{id}_Y$. 
\end{defn}

\begin{defn}{Homotopy Equivalences}{} Let $X,Y$ be simplicial sets. Let $f:X\to Y$ be a morphism. We say that $f$ is a homotopy equivalence between $X$ and $Y$ if $f$ admits a homotopy inverse. 
\end{defn}

\begin{lmm}{}{} Let $X,Y$ be simplicial sets. Let $f:X\to Y$ be a homotopy equivalence. Then the homotopy inverse of $f$ is unique up to homotopy. 
\end{lmm}

\subsection{Homotopy between Vertices}
We can also talk about homotopies between vertices of a simplicial set. 

\begin{defn}{Homotopy between Vertices}{} Let $X$ be a simplicial set. Let $v_0,v_1$ be vertices of $X$. A homotopy from $v_0$ to $v_1$ is a homotopy from the inclusion $\Delta^0\cong\{v_0\}\hookrightarrow X$ to the inclusion $\Delta^0\cong\{v_1\}\hookrightarrow X$. 
\end{defn}

\begin{lmm}{}{} Let $X$ be a simplicial set. Let $v_0,v_1$ be vertices of $X$. Then there exists a homotoyp from $v_0$ to $v_1$ if and only there exists and edge of $X$ from $v_0$ to $v_1$. 
\end{lmm}

We can generalize this to $n$-simplicies. 

\subsection{The Simplicial Homotopy Groups}
Because homotopies from $X$ to $Y$ is an equivalence relation only when $Y$ is a Kan complex, we can only define simplicial homotopy groups for Kan complexes. This notion can easily be extended to arbitrary simplicial sets by replacing the simplicial set with a fibrant replacement (and hence becomes a Kan complex). 

\begin{defn}{Simplicial Homotopy Groups}{} Let $X$ be a Kan complex. Let $v\in X$ be a vertex of $X$. Define the simplicial homotopy groups of $(X,v)$ as follows: 
\begin{itemize}
\item For $n\geq 1$, define the $n$th simplicial homotopy group $\pi_n^\text{sSet}(X,v)$ of $X$ at $v$ to be the set of homotopy classes of maps $[\alpha:\Delta^n\to X]$ relative to boundary $\partial\Delta^n$. In other words, the following diagram commutes: \\~\\
\adjustbox{scale=1.0,center}{\begin{tikzcd}
	{\partial\Delta^n} & {\Delta^0} \\
	{\Delta^n} & X
	\arrow["{!}", from=1-1, to=1-2]
	\arrow[hook, from=1-1, to=2-1]
	\arrow["v", from=1-2, to=2-2]
	\arrow["\alpha"', from=2-1, to=2-2]
\end{tikzcd}}\\~\\
\item Define the $0$th simplicial homotopy group $\pi_0^\text{sSet}(X)$ of $X$ to be the set of homotopy classes of vertices of $X$. 
\end{itemize}
If $X$ is a general simplicial set, define the simplicial homotopy group of $X$ to be $$\pi_n^\text{sSet}(X,v)=\pi_n^\text{sSet}(PX,Pv)$$ where $PX$ is a fibrant replacement of $X$. 
\end{defn}

\begin{thm}{}{} Let $X$ be a Kan complex. Let $f,g:\Delta^n\to X$ be two representatives of two elements in $\pi_n(X,v)$ for $n\geq 1$. Then the following data $$v_i=\begin{cases}
s_0^n(v) & \text{ if } 0\leq i\leq n-2\\
f & \text{ if } i=n-1\\
g & \text{ if } i=n+1
\end{cases}$$
defines a horn $\Lambda_i^{n+1}\to X$. Such a map extends to a map $\theta:\Delta^{n+1}\to X$. Define $$[f]\cdot[g]=d_n\theta$$ Then such an operation is well defined on the equivalence class. Moreover, it defines a group operation on $\pi_n^\text{sSet}(X,v)$. 
\end{thm}

\begin{thm}{}{} Let $X$ be a Kan complex. Then for $n\geq 2$, the above group structure on $\pi_n^\text{sSet}(X,v)$ is abelian. 
\end{thm}

\begin{thm}{}{} Let $X$ be a simplicial set. Let $v\in X_0$. Then there is an isomorphism $$\pi_n^\text{sSet}(X,v)\cong\pi_n(\abs{X},v)$$ for all $n\in\N$. 
\end{thm}

\pagebreak
\section{Constructing a Category from Simplicial Sets}
\subsection{The Nerve of a Category}
Let $P$ be a partially ordered set with partial ordering $\leq$. Recall that we defined a category associated to $P$ to consist of the following. 
\begin{itemize}
\item The objects are the elements of $P$
\item For any two elements $x,y\in P$, we define $$\Hom_P(x,y)=\begin{cases}
\ast & \text{ if }x\leq y\\
\emptyset & \text{ otherwise }
\end{cases}$$
\end{itemize}

This also applies to the totally ordered set $[n]=\{0,\dots,n\}$. There is a morphism $i\to j$ if and only if $i\leq j$. In this case. 

\begin{defn}{Nerve of a Category}{} Let $\mC$ be a category. Define the nerve of the category $N(C)\in\bold{sSet}$ as follows. 
\begin{itemize}
\item For each $n\in\N$, the set of $n$-simplicies is given by the Yoneda embedding as $$N(C)[n]=\Hom_\bold{Cat}([n],\mC)$$ where we view $[n]$ as a category. 
\item For each non-decreasing morphism $\alpha:[m]\to[n]$ there is a pre-composition map $$-\circ\alpha:\Hom_\bold{Cat}([n],\mC)\to\Hom_\bold{Cat}([m],\mC)$$ where we consider $\alpha:[m]\to[n]$ to be the associated functor between the categories $[m]$ and $[n]$. 
\end{itemize}
\end{defn}

Let $\mC$ be a category. We can explicitly write out the $n$-simplicies and face / degeneracy maps in the following way. 
\begin{itemize}
\item For $n\in\N$, $N(C)[n]$ consists of paths of morphisms with $n$ compositions: \\~\\
\adjustbox{scale=1.0,center}{\begin{tikzcd}
	{c_0} & {c_1} & {c_2} & \cdots & {c_n}
	\arrow["{f_1}", from=1-1, to=1-2]
	\arrow["{f_2}", from=1-2, to=1-3]
	\arrow[from=1-3, to=1-4]
	\arrow[from=1-4, to=1-5]
\end{tikzcd}}\\
\item The face map $d_i:N(C)[n]\to N(C)[n-1]$ sends the above element to \\~\\
\adjustbox{scale=1.0,center}{\begin{tikzcd}
	{c_0} & {c_1} & \cdots & {c_i} & {c_i} & \cdots & {c_n}
	\arrow["{f_1}", from=1-1, to=1-2]
	\arrow[from=1-2, to=1-3]
	\arrow[from=1-3, to=1-4]
	\arrow["{\text{id}_{c_i}}", from=1-4, to=1-5]
	\arrow[from=1-5, to=1-6]
	\arrow[from=1-6, to=1-7]
\end{tikzcd}}\\
\item The degeneracy map $s^i:N(C)[n]\to N(C)[n+1]$ sends the above element to \\~\\
\adjustbox{scale=1.0,center}{\begin{tikzcd}
	{c_0} & {c_1} & \cdots & {c_{i-1}} & {c_{i+1}} & \cdots & {c_n}
	\arrow["{f_0}", from=1-1, to=1-2]
	\arrow[from=1-2, to=1-3]
	\arrow[from=1-3, to=1-4]
	\arrow["{f_i\circ f_{i-1}}", from=1-4, to=1-5]
	\arrow[from=1-5, to=1-6]
	\arrow[from=1-6, to=1-7]
\end{tikzcd}}\\
\end{itemize}

\begin{defn}{Nerve Functor}{} The nerve functor $N:\text{Cat}\to\text{sSet}$ is defined as follows. 
\begin{itemize}
\item Each $\mC\in\text{Cat}$ is sent to the nerve $N(C)$
\item Every functor $F:\mC\to\mD$ in $\text{Cat}$ is sent to the morphism of presheaves $N(F):N(C)\to N(D)$ defined by $N(F)_n:N(C)([n])\to N(D)([n])$, of which is defined as the map \\~\\
\adjustbox{scale=1.0,center}{\begin{tikzcd}
	{c_0} & {c_1} & {c_2} & \cdots & {c_n} \\
	{F(c_0)} & {F(c_1)} & {F(c_2)} & \cdots & {F(c_n)}
	\arrow["{f_1}", from=1-1, to=1-2]
	\arrow["{f_2}", from=1-2, to=1-3]
	\arrow[from=1-3, to=1-4]
	\arrow[from=1-4, to=1-5]
	\arrow["{F(f_1)}", from=2-1, to=2-2]
	\arrow["{F(f_2)}", from=2-2, to=2-3]
	\arrow[from=2-3, to=2-4]
	\arrow[from=2-4, to=2-5]
\end{tikzcd}}\\~\\
from the upper path in $\mC$ to the lower path in $\mD$, such that the following diagram commutes: \\~\\
\adjustbox{scale=1.0,center}{\begin{tikzcd}
	{N(C)[n]} & {N(C)[m]} \\
	{N(D)[n]} & {N(D)[m]}
	\arrow["{N(C)(f)}", from=1-1, to=1-2]
	\arrow["{N(F)_n}"', from=1-1, to=2-1]
	\arrow["{N(F)_m}", from=1-2, to=2-2]
	\arrow["{N(D)(f)}"', from=2-1, to=2-2]
\end{tikzcd}}\\~\\
where $f:[m]\to[n]$ is a morphism in $\Delta$. 
\end{itemize}
\end{defn}

\begin{prp}{}{} The nerve functor $N:\bold{Cat}\to\bold{sSet}$ is fully faithful. 
\end{prp}

In particular, the nerve of a category is a complete invariant of categories. This means that given a simplicial set which we know is the nerve of some category, we can reconstruct a category whose nerve is the simplicial set, up to isomorphism. \\~\\

While $N:\bold{Cat}\to\bold{sSet}$ is fully faithful, it is not essentially surjective. We can find the image of the functor through the following characterization. 

\begin{prp}{}{} Let $S$ be a simplicial set. Then $S$ is isomorphic to the nerve of some category $\mC$ if and only if every inner horn admits a filler. 
\end{prp}

In particular, every Kan complex lies in the essential image of the nerve functor $N:\bold{Cat}\to\bold{sSet}$. 

\begin{lmm}{}{} Let $\mC$ be a category. Then $\mC$ is a groupoid if and only if $N(\mC)$ is a Kan complex. 
\end{lmm}

\subsection{The Homotopy Category of a Simplicial Set}
\begin{defn}{Homotopy Category of Simplicial Sets}{} Let $S\in\bold{sSet}$ be a simplicial set. Define the homotopy category $$h(S)$$ of $S$ to consist of the following data. 
\begin{itemize}
\item The objects consists of the vertices of $S$
\item Each edge $e$ of $S$ determines a unique morphism $[e]:d_1^1(e)\to d_0^1(e)$
\item The collection of morphisms are generated by all morphisms of the form $[e]$ for $e$ an edge, quotient by the relations $$[s_0^0(x)]=\text{id}_x\;\;\;\;\text{ and }\;\;\;\;[d_1^2(\sigma)]=[d_0^2(\sigma)]\circ[d_2^2(\sigma)]$$ for any $x\in S$ a vertex. 
\item Composition is defined through the generator and relations above. 
\end{itemize}
\end{defn}

The quotient relation means that given a $2$-simplex, going through any two directed edges gives the same map as going through the third edge. 

\begin{prp}{}{} Let $S$ be a simplicial set. Then $h(S)$ satisfies the following universal property. For any category $\mC$, (what is $k_S$?) and any morphism of simplicial sets $F:S\to N(\mC)$, there exists a unique functor $u:h(S)\to\mC$ such that the following diagram commutes: \\~\\
\adjustbox{scale=1.0,center}{\begin{tikzcd}
	S & {N(h(S))} \\
	& {N(\mC)}
	\arrow["k_S", from=1-1, to=1-2]
	\arrow["F"', from=1-1, to=2-2]
	\arrow["{\exists!N(u)}", dashed, from=1-2, to=2-2]
\end{tikzcd}}\\~\\
Moreover, if $\mD$ is any other category satisfying the above, then there exists a unique isomorphism $h(S)\cong\mD$. 
\end{prp}

Recall that the nerve functor is fully faithful. We can rewrite the universal property to mean the following: For any category $\mC$, precomposition with $-\circ k_S$ gives a bijection of sets $$\Hom_\bold{Cat}(h(S),\mC)\overset{\cong}{\rightarrow}\Hom_\bold{sSet}(N(h(S),N(\mC))\overset{-\circ k_S}{\rightarrow}\Hom_\bold{sSet}(S,N(\mC))$$

\begin{defn}{The Homotopy Functor}{} Define the homotopy functor $h:\text{sSet}\to\text{Cat}$ as follows. 
\begin{itemize}
\item On objects, $h$ sends a simplicial set $S:\Delta\to\text{Set}$ to $h(S)$. 
\item For $f:S\to T$ a morphism of simplicial sets, $h(f):h(S)\to h(T)$ is the unique morphism of simpicial sets corresponding to the functor $k_T\circ F:S\to N(h(T))$ under the universal property of $h(S)$. 
\end{itemize}
\end{defn}

\begin{thm}{}{} The homotopy functor $h$ is left adjoint $$h:\bold{sSet}\rightleftarrows\bold{Cat}:N$$ to the nerve functor $N$. This means that there is a natural bijection $$\Hom_\text{Cat}(h(C),D)\cong\Hom_\text{sSet}(C,N(D))$$
\end{thm}

We have now constructed two pairs of adjoint functors: \\~\\
\adjustbox{scale=1.0,center}{\begin{tikzcd}
	& {\bold{Cat}} \\
	{\bold{sSet}} \\
	& {\bold{Top}}
	\arrow["{N_\bullet}", shift left, from=1-2, to=2-1]
	\arrow["h", shift left, from=2-1, to=1-2]
	\arrow["{\abs{\;\cdot\;}}", shift left, from=2-1, to=3-2]
	\arrow["{S_\bullet}", shift left, from=3-2, to=2-1]
\end{tikzcd}}\\~\\

\pagebreak
\section{Homotopy Limits and Colimits}
\subsection{The Standard Homotopy Pullback and Pushout}
\begin{defn}{The Standard Homotopy Pullbacks}{} Let the following be a diagram of simplicial sets and morphisms. \\~\\
\adjustbox{scale=1.0,center}{\begin{tikzcd}
	X & Z & Y
	\arrow["f", from=1-1, to=1-2]
	\arrow["g"', from=1-3, to=1-2]
\end{tikzcd}}\\~\\
Define the standard homotopy pullback to be the simplicial set $$\text{Holim}_Q(X\rightarrow Z\leftarrow Y)=X\times_Z^hY=X\times_{\Hom_\bold{sSet}(\{0\},Z)}\Hom_\bold{sSet}(\Delta^1,Z)\times_{\Hom_\bold{sSet}(\{1\},Z)}Y$$ Here we identify the vertices of $\Delta^1$ as $0$ and $1$, and we use the isomorphism $\mC\cong\Hom_\bold{sSet}(\Delta^0,Z)$. 
\end{defn}

Kerodon 3.4.0.3

\begin{lmm}{}{} Let the following be a diagram of simplicial sets and morphisms, where $Z$ is a Kan complex. \\~\\
\adjustbox{scale=1.0,center}{\begin{tikzcd}
	X & Z & Y
	\arrow["f", from=1-1, to=1-2]
	\arrow["g"', from=1-3, to=1-2]
\end{tikzcd}}\\~\\
Then the following diagram commutes up to homotopy: \\~\\
\adjustbox{scale=1.0,center}{\begin{tikzcd}
	{\text{Holim}_Q(X\rightarrow Z\leftarrow Y)} & X \\
	Y & Z
	\arrow["{\text{proj}}", from=1-1, to=1-2]
	\arrow["{\text{proj}}"', from=1-1, to=2-1]
	\arrow["f", from=1-2, to=2-2]
	\arrow["g"', from=2-1, to=2-2]
\end{tikzcd}}\\~\\
\end{lmm}

\subsection{Homotopy Pullback and Pushout Squares}

\subsection{Recognizing Homotopy Pullbacks and Pushouts}
\begin{prp}{Kerodon 3.4.0.7}{} Let the following be a diagram in simplicial sets. \\~\\
\adjustbox{scale=1.0,center}{\begin{tikzcd}
	X & Z & Y
	\arrow["f", from=1-1, to=1-2]
	\arrow["g"', from=1-3, to=1-2]
\end{tikzcd}}\\~\\
If $Z$ is a Kan complex and one of $f$ or $g$ is a Kan fibration, then the following are true. 
\begin{itemize}
\item The following diagram \\~\\
\adjustbox{scale=1.0,center}{\begin{tikzcd}
	{X\times_ZY} & X \\
	Y & Z
	\arrow["{\text{pr}}", from=1-1, to=1-2]
	\arrow["{\text{pr}}"', from=1-1, to=2-1]
	\arrow["f", from=1-2, to=2-2]
	\arrow["g"', from=2-1, to=2-2]
\end{tikzcd}}\\~\\
is a homotopy pullback square. 
\item The inclusion map $$X\times_ZY\hookrightarrow X\times_Z^hY$$ is a weak homotopy equivalence. 
\item $X\times_ZY$ is a homotopy pullback of the diagram. 
\end{itemize}
\end{prp}

Secret: if the diagram is fibrant in the diagram category, then the homotopy pullback is the same as the standard pullback up to weak equivalence. 

\pagebreak
\section{Topics in Kan Complexes}
\subsection{Anodyne Morphisms}
They are the acyclic cofibrations of sSet with standard model structure. 

\begin{defn}{Anodyne Morphisms}{} Let $X,Y\in\bold{sSet}$ be simplicial sets. Let $f:X\to Y$ be a morphism. We say that $f$ is anodyne if $f$ has the left lifting property with respect to all Kan fibrations: \\~\\
\adjustbox{scale=1.0,center}{\begin{tikzcd}
	A & X \\
	B & Y
	\arrow[from=1-1, to=1-2]
	\arrow["{\text{Kan Fib.}}"', from=1-1, to=2-1]
	\arrow["f", from=1-2, to=2-2]
	\arrow["{\exists h}", dashed, from=2-1, to=1-2]
	\arrow[from=2-1, to=2-2]
\end{tikzcd}}\\~\\
\end{defn}

\subsection{Trivial Kan Fibrations}
They are the acyclic fibrations of sSet with standard model structure. 

\begin{defn}{Trivial Kan Fibrations}{} Let $X,Y\in\bold{sSet}$ be simplicial sets. Let $f:X\to Y$ be a morphism. We say that $f$ is a trivial Kan fibration if $f$ has the right lifting property with respect to all inclusions $\partial\Delta^n\hookrightarrow\Delta^n$: \\~\\
\adjustbox{scale=1.0,center}{\begin{tikzcd}
	{\partial\Delta^n} & X \\
	{\Delta^n} & Y
	\arrow[from=1-1, to=1-2]
	\arrow[hook, from=1-1, to=2-1]
	\arrow["f", from=1-2, to=2-2]
	\arrow["{\exists k}", dashed, from=2-1, to=1-2]
	\arrow[from=2-1, to=2-2]
\end{tikzcd}}\\~\\
\end{defn}

\begin{prp}{}{} Let $X,Y\in\bold{sSet}$ be simplicial sets. Let $f:X\to Y$ be a morphism. Then $f$ is a trivial Kan fibration if and only if $f$ has the right lifting property with respect to all monomorphisms in $\bold{sSet}$. 
\end{prp}

\begin{crl}{}{} Let $X,Y\in\bold{sSet}$ be simplicial sets. Let $f:X\to Y$ be a morphism. If $f$ is a trivial Kan fibration, then $f$ admits a section. This means that there exists a morphism $s:Y\to X$ such that $$f\circ s=\text{id}_Y$$ Moreover, $s\circ f:X\to X$ is homotopic to the identity map $\text{id}_X$. 
\end{crl}

\begin{prp}{}{} Let $X,Y\in\bold{sSet}$ be simplicial sets. Let $f:X\to Y$ be a trivial Kan fibration. If $X$ is a Kan complex, then $Y$ is a Kan complex. 
\end{prp}

\begin{prp}{}{} Let $X,Y\in\bold{sSet}$ be simplicial sets. Let $f:X\to Y$ be a morphism. If $f$ is a trivial Kan fibration, then the induced morphism $$f\circ -:\Hom_\bold{sSet}(A,X)\to\Hom_\bold{sSet}(A,Y)$$ is a trivial Kan fibration for all $A\in\bold{sSet}$. 
\end{prp}

\subsection{Trivial Kan Fibrations and Contractible Kan Complexes}
\begin{prp}{}{} Let $X$ be a simplicial set. Then the following are equivalent. 
\begin{itemize}
\item $X$ is a contractible Kan complex (every morphism $\partial\Delta^n\to X$ has an extension $\Delta^n\to X$)
\item The unique map $X\to\Delta^0$ is a trivial Kan fibration. 
\item For every monomorphism $i:A\to B$ of simplicial sets and every morphism of simplicial sets $f:A\to X$, there exists a map $g:B\to X$ such that the following diagram commutes: \\~\\
\adjustbox{scale=1.0,center}{\begin{tikzcd}
	A & X \\
	B
	\arrow["f", from=1-1, to=1-2]
	\arrow[hook, from=1-1, to=2-1]
	\arrow["g"', dashed, from=2-1, to=1-2]
\end{tikzcd}}\\~\\
\end{itemize}
\end{prp}

\begin{prp}{}{} Let $X,Y\in\bold{sSet}$ be simplicial sets. Let $f:X\to Y$ be a trivial Kan fibration. If $X$ is a contractible Kan complex, then $Y$ is a contractible Kan complex. 
\end{prp}

\subsection{$\text{Ex}^\infty$ and the Subdivision Functor}

\subsection{Weak Factorization}
\begin{prp}{}{} Let $X,Y$ be simplicial sets. Let $f:X\to Y$ be a morphism. Then there exists a simplicial set $Q(f)$ and maps $f':X\to Q(f)$ and $f'':Q(f)\to Y$ such that $f'$ is anodyne, $f''$ is a Kan fibration and the following diagram commutes: \\~\\
\adjustbox{scale=1.0,center}{\begin{tikzcd}
	X && Y \\
	& {Q(f)}
	\arrow["f", from=1-1, to=1-3]
	\arrow["{\exists f'}"', dashed, from=1-1, to=2-2]
	\arrow["{\exists f''}"', dashed, from=2-2, to=1-3]
\end{tikzcd}}\\~\\
\end{prp}

\begin{prp}{}{} Let $X,Y$ be simplicial sets. Let $f:X\to Y$ be a morphism. Then there exists a choice of $Q(f)$ such that $Q$ defines a functor $$Q:\Hom_\bold{Cat}(\Delta^1,\bold{sSet})\to\bold{sSet}$$ that commutes with filtered colimits. 
\end{prp}

\subsection{(Co)Fibrant Replacements}
\begin{defn}{Fibrant Replacements}{} Let $X$ be a simplicial set. A fibrant replacement of $X$ is a Kan complex $QX$ such that the unique map $X\to\ast$ factorizes into $X\to QX\to\ast$ where $X\to QX$ is anodyne. 
\end{defn}

Secret: these are just the fibrant replacements in model category. 

\begin{lmm}{}{} Let $X$ be a simplicial set. Then the following are all fibrant replacements of $X$. 
\begin{itemize}
\item $Q_X=Q(X\to\ast)$
\item $\text{Ex}^\infty(X)$
\end{itemize}
\end{lmm}




\end{document}
