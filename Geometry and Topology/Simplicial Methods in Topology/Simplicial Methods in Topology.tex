\documentclass[a4paper]{article}

%=========================================
% Packages
%=========================================
\usepackage{mathtools}
\usepackage{amsfonts}
\usepackage{amsmath}
\usepackage{amssymb}
\usepackage{amsthm}
\usepackage[a4paper, total={6in, 8in}, margin=1in]{geometry}
\usepackage[utf8]{inputenc}
\usepackage{fancyhdr}
\usepackage[utf8]{inputenc}
\usepackage{graphicx}
\usepackage{physics}
\usepackage[listings]{tcolorbox}
\usepackage{hyperref}
\usepackage{tikz-cd}
\usepackage{adjustbox}
\usepackage{enumitem}
\usepackage[font=small,labelfont=bf]{caption}
\usepackage{subcaption}
\usepackage{wrapfig}
\usepackage{makecell}



\raggedright

\usetikzlibrary{arrows.meta}

\DeclarePairedDelimiter\ceil{\lceil}{\rceil}
\DeclarePairedDelimiter\floor{\lfloor}{\rfloor}

%=========================================
% Fonts
%=========================================
\usepackage{tgpagella}
\usepackage[T1]{fontenc}


%=========================================
% Custom Math Operators
%=========================================
\DeclareMathOperator{\adj}{adj}
\DeclareMathOperator{\im}{im}
\DeclareMathOperator{\nullity}{nullity}
\DeclareMathOperator{\sign}{sign}
\DeclareMathOperator{\dom}{dom}
\DeclareMathOperator{\lcm}{lcm}
\DeclareMathOperator{\ran}{ran}
\DeclareMathOperator{\ext}{Ext}
\DeclareMathOperator{\dist}{dist}
\DeclareMathOperator{\diam}{diam}
\DeclareMathOperator{\aut}{Aut}
\DeclareMathOperator{\inn}{Inn}
\DeclareMathOperator{\syl}{Syl}
\DeclareMathOperator{\edo}{End}
\DeclareMathOperator{\cov}{Cov}
\DeclareMathOperator{\vari}{Var}
\DeclareMathOperator{\cha}{char}
\DeclareMathOperator{\Span}{span}
\DeclareMathOperator{\ord}{ord}
\DeclareMathOperator{\res}{res}
\DeclareMathOperator{\Hom}{Hom}
\DeclareMathOperator{\Mor}{Mor}
\DeclareMathOperator{\coker}{coker}
\DeclareMathOperator{\Obj}{Obj}
\DeclareMathOperator{\id}{id}
\DeclareMathOperator{\GL}{GL}
\DeclareMathOperator*{\colim}{colim}

%=========================================
% Custom Commands (Shortcuts)
%=========================================
\newcommand{\CP}{\mathbb{CP}}
\newcommand{\GG}{\mathbb{G}}
\newcommand{\F}{\mathbb{F}}
\newcommand{\N}{\mathbb{N}}
\newcommand{\Q}{\mathbb{Q}}
\newcommand{\R}{\mathbb{R}}
\newcommand{\C}{\mathbb{C}}
\newcommand{\E}{\mathbb{E}}
\newcommand{\Prj}{\mathbb{P}}
\newcommand{\RP}{\mathbb{RP}}
\newcommand{\T}{\mathbb{T}}
\newcommand{\Z}{\mathbb{Z}}
\newcommand{\A}{\mathbb{A}}
\renewcommand{\H}{\mathbb{H}}
\newcommand{\K}{\mathbb{K}}

\newcommand{\mA}{\mathcal{A}}
\newcommand{\mB}{\mathcal{B}}
\newcommand{\mC}{\mathcal{C}}
\newcommand{\mD}{\mathcal{D}}
\newcommand{\mE}{\mathcal{E}}
\newcommand{\mF}{\mathcal{F}}
\newcommand{\mG}{\mathcal{G}}
\newcommand{\mH}{\mathcal{H}}
\newcommand{\mI}{\mathcal{I}}
\newcommand{\mJ}{\mathcal{J}}
\newcommand{\mK}{\mathcal{K}}
\newcommand{\mL}{\mathcal{L}}
\newcommand{\mM}{\mathcal{M}}
\newcommand{\mO}{\mathcal{O}}
\newcommand{\mP}{\mathcal{P}}
\newcommand{\mS}{\mathcal{S}}
\newcommand{\mT}{\mathcal{T}}
\newcommand{\mV}{\mathcal{V}}
\newcommand{\mW}{\mathcal{W}}

%=========================================
% Colours!!!
%=========================================
\definecolor{LightBlue}{HTML}{2D64A6}
\definecolor{ForestGreen}{HTML}{4BA150}
\definecolor{DarkBlue}{HTML}{000080}
\definecolor{LightPurple}{HTML}{cc99ff}
\definecolor{LightOrange}{HTML}{ffc34d}
\definecolor{Buff}{HTML}{DDAE7E}
\definecolor{Sunset}{HTML}{F2C57C}
\definecolor{Wenge}{HTML}{584B53}
\definecolor{Coolgray}{HTML}{9098CB}
\definecolor{Lavender}{HTML}{D6E3F8}
\definecolor{Glaucous}{HTML}{828BC4}
\definecolor{Mauve}{HTML}{C7A8F0}
\definecolor{Darkred}{HTML}{880808}
\definecolor{Beaver}{HTML}{9A8873}
\definecolor{UltraViolet}{HTML}{52489C}



%=========================================
% Theorem Environment
%=========================================
\tcbuselibrary{listings, theorems, breakable, skins}

\newtcbtheorem[number within = subsection]{thm}{Theorem}%
{	colback=Buff!3, 
	colframe=Buff, 
	fonttitle=\bfseries, 
	breakable, 
	enhanced jigsaw, 
	halign=left
}{thm}

\newtcbtheorem[number within=subsection, use counter from=thm]{defn}{Definition}%
{  colback=cyan!1,
    colframe=cyan!50!black,
	fonttitle=\bfseries, breakable, 
	enhanced jigsaw, 
	halign=left
}{defn}

\newtcbtheorem[number within=subsection, use counter from=thm]{axm}{Axiom}%
{	colback=red!5, 
	colframe=Darkred, 
	fonttitle=\bfseries, 
	breakable, 
	enhanced jigsaw, 
	halign=left
}{axm}

\newtcbtheorem[number within=subsection, use counter from=thm]{prp}{Proposition}%
{	colback=LightBlue!3, 
	colframe=Glaucous, 
	fonttitle=\bfseries, 
	breakable, 
	enhanced jigsaw, 
	halign=left
}{prp}

\newtcbtheorem[number within=subsection, use counter from=thm]{lmm}{Lemma}%
{	colback=LightBlue!3, 
	colframe=LightBlue!60, 
	fonttitle=\bfseries, 
	breakable, 
	enhanced jigsaw, 
	halign=left
}{lmm}

\newtcbtheorem[number within=subsection, use counter from=thm]{crl}{Corollary}%
{	colback=LightBlue!3, 
	colframe=LightBlue!60, 
	fonttitle=\bfseries, 
	breakable, 
	enhanced jigsaw, 
	halign=left
}{crl}

\newtcbtheorem[number within=subsection, use counter from=thm]{eg}{Example}%
{	colback=Beaver!5, 
	colframe=Beaver, 
	fonttitle=\bfseries, 
	breakable, 
	enhanced jigsaw, 
	halign=left
}{eg}

\newtcbtheorem[number within=subsection, use counter from=thm]{ex}{Exercise}%
{	colback=Beaver!5, 
	colframe=Beaver, 
	fonttitle=\bfseries, 
	breakable, 
	enhanced jigsaw, 
	halign=left
}{ex}

\newtcbtheorem[number within=subsection, use counter from=thm]{alg}{Algorithm}%
{	colback=UltraViolet!5, 
	colframe=UltraViolet, 
	fonttitle=\bfseries, 
	breakable, 
	enhanced jigsaw, 
	halign=left
}{alg}




%=========================================
% Hyperlinks
%=========================================
\hypersetup{
    colorlinks=true, %set true if you want colored links
    linktoc=all,     %set to all if you want both sections and subsections linked
    linkcolor=DarkBlue,  %choose some color if you want links to stand out
}


\pagestyle{fancy}
\fancyhf{}
\rhead{Labix}
\lhead{Simplicial Methods in Topology}
\rfoot{\thepage}

\title{Simplicial Methods in Topology}

\author{Labix}

\date{\today}
\begin{document}
\maketitle
\begin{abstract}

\end{abstract}
References: 

\pagebreak
\tableofcontents

\pagebreak

\section{The Category of Simplicial Sets}
\subsection{The Simplex Category}
Recall the simplex category in Category Theory 1. 

\begin{defn}{Simplex Category}{} The simplex category $\Delta$ consists of the following data. 
\begin{itemize}
\item The objects are $[n]=\{0,\dots,n\}$ for $n\in\N$. 
\item The morphisms are the non-strictly order preserving functions. This means that a morphism $f:[n]\to[m]$ must satisfy $f(i)\leq f(j)$ for all $i\leq j$. 
\item Composition is the usual composition of functions. 
\end{itemize}
\end{defn}

\begin{defn}{Maps in the Simplex Category}{} Consider the simplex category $\Delta$. Define the face maps and the degeneracy maps as follows. 
\begin{itemize}
\item A face map in $\Delta$ is the unique morphism $d^i:[n-1]\to[n]$ that is injective and whose image does not contain $i$. Explicitly, we have $$d^i(k)=\begin{cases}
k & \text{ if } 0\leq k <i\\
k+1 & \text{ if } i\leq k\leq n-1
\end{cases}$$
\item A degeneracy map in $\Delta$ is the unique morphism $s^i:[n+1]\to[n]$ that is surjective and hits $i$ twice. Explicitly, we have $$s^i(k)=\begin{cases}
k & \text{ if } 0\leq k\leq i\\
k-1 & \text{ if } i+1\leq k\leq n+1
\end{cases}$$
\end{itemize}
\end{defn}

\begin{prp}{}{} The face maps and the degeneracy maps in the simplex category $\Delta$ satisfy the following simplicial identities: 
\begin{itemize}
\item $d^j\circ d^i=d^i\circ d^{j-1}$ if $i<j$
\item $s^j\circ s^i=s^i\circ s^{j+1}$ if $i\leq j$
\item $s^j\circ d^i=\begin{cases}
\text{id} & \text{ if } i=j\text{ or }j+1\\
d^i\circ s^{j-1} & \text{ if } i<j\\
d^{i-1}\circ s^j & \text{ if }i>j+1
\end{cases}$
\end{itemize} \tcbline
\begin{proof}~\\
\begin{itemize}
\item Consider the object $[n-1]$. On the left, we have that 
\begin{align*}
d^j(d^i[n-1])&=d^j([0,\dots,i-1,i+1,\dots,n])\\
&=[0,\dots,i-1,i+1,\dots,j-1,j+1,\dots,n+1]
\end{align*}
On the right, we have that
\begin{align*}
d^i(d^{j-1}[n-1])&=d^i([0,\dots,j-2,j,\dots,n])\\
&=[0,\dots,i-1,i+1,\dots,j-1,j+1,\dots,n+1]
\end{align*} and so the relation is indeed true. 
\end{itemize}
\end{proof}
\end{prp}

\begin{prp}{}{} Every morphism in the simplex category $\Delta$ is a composition of the face maps and the degeneracy maps. 
\end{prp}

\subsection{Simplicial Sets}
\begin{defn}{Simplicial Sets}{} A simplicial set is a presheaf $$S:\Delta\to\text{Sets}$$ 
\end{defn}

\begin{eg}{}{} Let $X$ be a topological space. The set of singular simplices of $X$ is given by the presheaf $$S:\Delta\to\bold{Set}$$ defined by $[n]\mapsto\Hom_\bold{Top}(\abs{\Delta^n},X)$. In other words, an $n$-simplex of $X$ is simply a continuous function $\sigma:\Delta^n\to X$. This is exactly the same as how we defined singular $n$-simplexes in Algebraic Topology 2. 
\end{eg}

\begin{defn}{Category of Simplicial Sets}{} The category of simplicial sets $\text{sSet}$ is defined as follows. 
\begin{itemize}
\item The objects are simplicial sets $S:\Delta\to\text{Sets}$
\item The morphisms are just morphisms of presheaves. This means that if $S,T:\Delta\to\text{Sets}$ are simplicial sets, then a morphism $\lambda:S\to T$ consists of morphisms $\lambda_n:S([n])\to T([n])$ for $n\in\N$ such that the following diagram commutes: \\~\\
\adjustbox{scale=1.0,center}{\begin{tikzcd}
	{S([n])} & {S([m])} \\
	{T([n])} & {T([m])}
	\arrow["{S(f)}", from=1-1, to=1-2]
	\arrow["{\lambda_n}"', from=1-1, to=2-1]
	\arrow["{\lambda_m}", from=1-2, to=2-2]
	\arrow["{T(f)}"', from=2-1, to=2-2]
\end{tikzcd}}
\item Composition is defined as the usual composition of functors. 
\end{itemize}
\end{defn}

The Yoneda lemma in this context implies that there is a bijection $$\Hom_{\text{sSet}}(\Hom_\Delta([n],-),S)\cong S([n])$$ that is natural in the variable $[n]$. We will denote $$\Delta^n=\Hom_\Delta([n],-)$$ which is the image of $[n]$ under the yoneda embedding $y:\Delta\to\text{sSet}$ defined by $[n]\mapsto\Hom_\Delta([n],-)$. 

\begin{defn}{n-Simplices}{} Let $S:\Delta\to\text{Set}$ be a simplicial set. For $n\in\N$, define the $n$-simplices of $S$ to be $$S_n=S([n])=\Hom_\text{sSet}(\Delta^n,S)$$
\end{defn}

Notice that $\Delta^n$ is a simplicial set $$\Delta^n:\Delta\to\text{Set}$$ defined by $[m]\mapsto\Hom_\Delta([n],[m])$. Notice that if $n>m$, then it is impossible to have an order preserving function $[n]\to[m]$. Hence when $n>m$, $\Hom_\Delta([n],[m])$ is empty. It is also clear that the $m$-simplices of $\Delta^n$ are precisely the order preserving maps $[m]\to[n]$. 

\begin{defn}{Standard n-Simplex}{} Let $n\in\N$. The standard $n$-simplex is the simplicial set $\Delta^n:\Delta\to\text{Set}$ defined by $$\Delta^n=\Hom_\Delta([n],-)$$
\end{defn}

All such simplicial sets $\Delta^n$ are useful in determining the contents of an arbitrary simplicial set. As for any presheaf, instead of focusing between the passage of data from $\Delta$ to $\text{Set}$, we should instead think of what kind of structure the presheaf brings to $\text{Set}$. Let $C$ be a simplicial set. Then this means the following. For each $n$, there is a set $C_n=\Hom_\text{sSet}(\Delta^n,C)$. For each morphism in $\Delta$, there is a corresponding morphism in $\text{Set}$, which we shall discuss now. 

\begin{prp}{}{} Let $S:\Delta\to\text{Set}$ be a simplicial set. Then every morphism in $S(\Delta)$ is the composite of two kinds of maps: 
\begin{itemize}
\item The face maps: $d_i:S_n\to S_{n-1}$ for $0\leq i\leq n$ defined by $$d_i=S(d^i:[n-1]\to[n])$$
\item The degeneracy maps: $s_i:S_{n+1}\to S_n$ for $0\leq i\leq n$ defined by $$s_i=S(s^i:[n+1]\to[n])$$
\end{itemize}
Moreover, these maps satisfy the following simplicial identities: 
\begin{itemize}
\item $d_i\circ d_j=d_{j-1}\circ d_i$ if $i<j$
\item $s_i\circ s_j=s_{j+1}\circ s_i$ if $i\leq j$
\item $d_i\circ s_j=\begin{cases}
\text{id} & \text{ if } i=j\text{ or }j+1\\
s_{j-1}\circ d_i & \text{ if } i<j\\
s_j\circ d_{i-1} & \text{ if }i>j+1
\end{cases}$
\end{itemize} \tcbline
\begin{proof}
Results are immediate using prp 1.1.3 and the fact that $S$ is contravariant. 
\end{proof}
\end{prp}

\begin{prp}{}{} The category $\bold{sSet}$ is a symmetric monoidal category with level-wise cartesian product. 
\end{prp}

Recall the notion of a $\Delta$-set from Algebraic Topology 2 and one might realize they look suspiciously similar to that of a simplicial set. Let us recall. A $\Delta$-set is a collection of sets $S_n$ for $n\in\N$ together with maps $d_i^n:S_n\to S_{n-1}$ for $0\leq i\leq n$ such that $$d_i^{n-1}\circ d_j^n=d_{j-1}^{n-1}\circ d_i^n$$ for $i<j$. One can easily convince themselves that every simplicial set is a $\Delta$-set. Indeed, a simplicial set satisfies five more relations than a $\Delta$-set. Therefore we have that $$\bold{sSet}\subset\Delta\text{ Complexes}$$

\begin{prp}{}{} Every simplicial set is a $\Delta$-set. 
\end{prp}

Combining with the previously learnt combinatorial objects in algebraic topology, we now have the following tower:  $$\text{Simplicial Complexes}\subset\bold{sSet}\subset\Delta\text{ Complexes}\subset\bold{CW}$$

\begin{defn}{Faces of a Simplex}{} Let $n\in\N$ and consider the standard $n$-simplex $\Delta^n$. 
\begin{itemize}
\item Denote $\partial_i\Delta^n\subset\Delta^n$ the simplicial subset generated by the $i$th face $$d_i(\text{id}:[n]\to[n])=d^i:[n-1]\to[n]$$
\item Denote $\partial\Delta^n$ the simplicial subset generated by the faces $\partial_i\Delta^n$ for $0\leq i\leq n$. Define $\partial\Delta^0=\emptyset$. 
\end{itemize}
\end{defn}

\begin{defn}{Degenerate Simplices}{} Let $S:\Delta\to\bold{Set}$ be a simplicial set. Let $x\in S_n$ be an $n$-simplex. We say that $x$ is degenerate if if $x\in\im(s_k)$ for some degenerate map $s_k$. 
\end{defn}

Intuitively, degenerate simplices refer to simplicies that consecutively have the same face. Indeed, if $x=s_k(y)$ for some $k$, then the $k$th and $k+1$th position of the simplex $x$ is given by the same element. Collapsing the two vertices should return the simplex $y$. 

\subsection{Geometric Realization of Simplicial Sets}
\begin{defn}{Geometric Realization of Standard n-Simplexes}{} Let $n\in\N$. Consider the standard $n$-simplex $\Delta^n$. Define the geometric realization of $\Delta^n$ to be $$\abs{\Delta^n}=\left\{\sum_{k=0}^nt_kv_k\bigg{|}\sum_{k=0}^nt_k=1\text{ and }t_k\geq 0\text{ for all }k=0,\dots,n\right\}$$
\end{defn}

This definition is exactly the same as the definition of an $n$-simplex in Algebraic Topology 2. Now we proceed to the general case. 

\begin{defn}{Geometric Realization of Simplicial Sets}{} Let $C$ be a simplicial set. Define the geometric realization of $C$ to be $$\abs{C}=\left(\coprod_{n\geq 0}C_n\times\abs{\Delta^n}\right)/\sim$$ where the equivalence relation is generated by the following. 
\begin{itemize}
\item The $i$th face of $\{x\}\times\abs{\Delta^n}$ is identified with $\{d_ix\}\times\abs{\Delta^{n-1}}$ by the linear homeomorphism preserving the order of the vertices. 
\item $\{s_ix\}\times\abs{\Delta^n}$ is collapsed onto $\{x\}\times\abs{\Delta^{n-1}}$ via the linear projection parallel to the line connecting the $i$th and the $(i+1)$st vertiex. 
\end{itemize}
\end{defn}

This construction of geometric realization is moreover functorial. Once again, we first define a map of geometric realization of simplicial sets. 

\begin{defn}{Induced Map of Geometric Realization of Standard Simplicial Sets}{} Let $f:\Delta^n\to\Delta^m$ be a map of standard simplexes. Define $f_\ast:\abs{\Delta^n}\to\abs{\Delta^m}$ by $$(t_0,\dots,t_n)\mapsto(s_0,\dots,s_m)$$ where $$s_i=\begin{cases}
0 & \text{ if } f^{-1}(i)=0\\
\sum_{j\in f^{-1}(i)}t_j & \text{ otherwise }
\end{cases}$$
\end{defn}

\begin{thm}{}{} The geometric realization of a simplicial set is functorial $\abs{\;\cdot\;}:\text{sSet}\to\text{Top}$ in the following way. 
\begin{itemize}
\item On objects, it sends any simplicial set $C$ to its geometric realization $\abs{C}$. 
\item On morphisms, it sends any morphism $C\to D$ of simplicial sets to a continuous map defined by 
\end{itemize}
\end{thm}

We thus have that $$\substack{\text{Geometric Relizations}\\\text{ of simplicial sets}}\subset\substack{\text{Geometric Relizations}\\\text{ of }\Delta\text{-sets}}\subset\text{CW-Complexes}$$

\subsection{Simplicial and Semisimplicial Objects}
\begin{defn}{Simplicial Objects}{} Let $\mC$ be a category. A simplicial object in $\mC$ is a presheaf $S:\Delta^\text{op}\to\mC$. 
\end{defn}

Hence a simplicial object in $\bold{Set}$ is just simplical sets. 

\begin{defn}{Category of Simplicial Objects}{} Let $\mC$ be a category. Define the category of simplicial objects $\text{s}\mC$ of $\mC$ as follows. 
\begin{itemize}
\item The objects are simplicial objects $S:\Delta^\text{op}\to\mC$ of $\mC$ which are presheaves
\item The morphism of simplcial objects are just morphisms of presheaves, which are natural transformations
\item Composition is given by composition of natural transformations
\end{itemize}
\end{defn}

\begin{defn}{The Semisimplex Category}{} The Semisimplex category $\bold{SS}$ is the subcategory of $\Delta$ consisting of strict order preserving functions. 
\end{defn}

\begin{defn}{Semisimplicial Objects}{} Let $\mC$ be a category. A semisimplicial object in $\mC$ is a presheaf $$\bold{SS}\to\mC$$
\end{defn}

\begin{lmm}{}{} Let $S$ be a set. Then $S$ is a semisimplicial set if and only if $S$ is a $\Delta$-set (in the sense of Algebraic Topology 2). 
\end{lmm}

\pagebreak
\section{Simplicial Homological Algebra}
\subsection{Chain Complexes of Simplicial Objects}
\begin{defn}{Associated Chain Complex}{} Let $\mA$ be an abelian category. Let $A$ be a (semi)-simplicial object in $\mA$. Define the associated chain complex of $A$ to be \\~\\
\adjustbox{scale=1.0,center}{\begin{tikzcd}
	\cdots & {C_{n+1}(A)} & {C_n(A)} & {C_{n-1}(A)} & \cdots & {C_1(A)}
	\arrow[from=1-1, to=1-2]
	\arrow["{\partial_{n+1}}", from=1-2, to=1-3]
	\arrow["{\partial_n}", from=1-3, to=1-4]
	\arrow[from=1-4, to=1-5]
	\arrow[from=1-5, to=1-6]
\end{tikzcd}}\\~\\
where $C_n(A)=A_n$ and the boundary operator given by $\partial_n=\sum_{i=0}^n(-1)^id_i^n:A_n\to A_{n-1}$. 
\end{defn}

TBA: Functoriality of associated chain complex

\begin{defn}{Simplicial Homology}{} Let $R$ be a ring. Let $X$ be a (semi)-simplicial set. Define the simplicial homology of $X$ with coefficients in $R$ to be the homology groups $$H_n^\Delta(X;R)=H_n(C_\bullet(R[X]))$$
\end{defn}

Notice that this definition coincides with that in Algebraic Topology 2. Recall that in AT2 we defined the simplicial homology of a $\Delta$-set, but in $\Z$ coefficients. 

\subsection{The Singular Functor}
\begin{defn}{Singular Functor}{} The singular functor $S:\text{Top}\to\text{sSet}$ is defined as follows. 
\begin{itemize}
\item On objects, it sends a space $X$ to the simplicial set $S(X):\Delta\to\text{Set}$ called the singular set, defined by $$S(X)[n]=\Hom_{\text{Top}}(\abs{\Delta^n},X)$$
\item On morphisms, it sends a continuous map $f:X\to Y$ to the morphism of simplicial sets $\lambda:S(X)\to S(Y)$ defined as follows. For each $n\in\N$, $\lambda_n:S(X)[n]\to S(Y)[n]$ is defined by $$\left(h:\abs{\Delta^n}\to X\right)\mapsto\left(f\circ h:\abs{\Delta^n}\to Y\right)$$ such that the following diagram commutes: \\~\\
\adjustbox{scale=1.0,center}{\begin{tikzcd}
	{S(X)[n]} & {S(X)[m]} \\
	{S(Y)[n]} & {S(Y)[m]}
	\arrow["{S(X)(f)}", from=1-1, to=1-2]
	\arrow["{\lambda_n}"', from=1-1, to=2-1]
	\arrow["{\lambda_m}", from=1-2, to=2-2]
	\arrow["{S(Y)(f)}"', from=2-1, to=2-2]
\end{tikzcd}}\\~\\
\end{itemize}
\end{defn}

Notice that this is reminiscent of the definitions in Algebraic Topology 2. Indeed $S(X)[n]$  for each $n\in\N$ is in fact the basis of the abelian group $C_n(X)$. It represents all the possible ways that an $n$-simplex could fit into $X$. Then the passage $$\bold{Top}\overset{S}{\longrightarrow}\bold{sSet}\overset{H_\bullet^\Delta(-;R)}{\longrightarrow}{_R\bold{Mod}}$$ recovers the singular homology of a space $X$ with coefficients in a ring $R$. This is formulated slightly differently in Algebraic Topology 2. 

\begin{thm}{}{} The singular functor $S:\text{Top}\to\text{sSet}$ is right adjoint to the geometric realization functor $\abs{\;\cdot\;}:\text{sSet}\to\text{Top}$. This means that there is a natural bijection $$\Hom_\text{Top}(\abs{X},Y)\cong\Hom_\text{sSet}(X,S(Y))$$ for any space $Y$ and any simplicial set $X$. 
\end{thm}

\subsection{Normalized Chain Complexes}
\begin{defn}{Normalized Chain Complexes}{} Let $\mA$ be an abelian category or the category $\bold{Grp}$. Let $A$ be a simplicial object in $\mA$. Define the normalized chain complex of $A$ to be the chain complex: \\~\\
\adjustbox{scale=1.0,center}{\begin{tikzcd}
	\cdots & {N_{k+1}(A)} & {N_k(A)} & {N_{k-1}(A)} & \cdots & {N_1(A)}
	\arrow[from=1-1, to=1-2]
	\arrow["{\partial_{k+1}}", from=1-2, to=1-3]
	\arrow["{\partial_k}", from=1-3, to=1-4]
	\arrow[from=1-4, to=1-5]
	\arrow[from=1-5, to=1-6]
\end{tikzcd}}\\~\\
where $$N_k(A)=\bigcap_{i=1}^k\ker(d_i^k:A_k\to A_{k-1})$$ and the differential given by $\partial_k=d_0^K|_{N_k(A)}$. We denote the normalized chain complex by $(N_\bullet(G),\partial_\bullet)$
\end{defn}

nLab: We may think of the elements of the complex in degree $k$ as being $k$-dimensional disks in $G$ all of whose boundary is captured by a single face. 

\begin{lmm}{}{} Let $G$ be a simplicial group. Consider the normalized chain complex $(N_\bullet(G),\partial_\bullet)$. Then $\partial_n N_n(G)$ is a normal subgroup of $N-{n-1}(G)$. 
\end{lmm}

Because of this lemma, it now makes sense to take the homology group of the normalized chain complex even if we take a simplicial object in $\bold{Grp}$. 

\begin{defn}{Normalized Chain Complex Functor}{} Let $\mA$ be an abelian category. Define the normalized chain complex functor $N$
\end{defn}

\begin{defn}{Degenerate Chain Complex}{} Let $\mA$ be an abelian category. Let $A$ be a simplicial object in $\mA$. Define the degenerate chain complex $D_\bullet(A)$ of $A$ to be the subcomplex of the associated chain complex $C_\bullet(A)$ defined by $$D_n(A)=\langle s_i^n:A_n\to A_{n+1}\;|\;s_i\text{ is the degenerate maps}\rangle$$
\end{defn}

\begin{prp}{}{} Let $\mA$ be an abelian category. Let $A$ be a simplicial object in $\mA$. Then there is a splitting $$C_\bullet(A)\cong N_\bullet(A)\oplus D_\bullet(A)$$ in the abelian category of chain complexes of $\mA$. 
\end{prp}

\begin{thm}{Eilenberg-Maclane}{} Let $\mA$ be an abelian category. Let $A$ be a simplicial object in $A$. Then the inclusion $$N_\bullet(A)\hookrightarrow C_\bullet(A)$$ is a natural chain homotopy equivalence. In other words, $D_\bullet(A)$ is null homotopic. 
\end{thm}

\begin{thm}{The Dold-Kan Correspondence}{} Consider the abelian category $\bold{Ab}$ of abelian groups. The normalized chain complex functor $$N:\text{s}\bold{Ab}\overset{\cong}{\longrightarrow}\text{Ch}_{\geq 0}(\bold{Ab})$$ gives an equivalence of categories, with inverse as the simplicialization functor $$\Gamma:\text{Ch}_{\geq 0}(\bold{Ab})\to\text{s}\bold{Ab}$$
\end{thm}

\subsection{Bar Resolutions}
\begin{defn}{Bar Construction}{} Let $A$ be an algebra over a ring $R$. Let $M$ be an $A$-algebra. Define the maps $d_i^n:M\otimes A^{\otimes n}\to M\otimes A^{\otimes n-1}$ by the following formulas: 
\begin{itemize}
\item If $i=0$, then $$d_i^n(m\otimes a_1\otimes\cdots\otimes a_n)=ma_1\otimes a_2\otimes\cdots\otimes a_n$$
\item If $0<i<n$, then $$d_i^n(m\otimes a_1\otimes\cdots\otimes a_n)=m\otimes a_1\otimes\cdots a_{i-1}\otimes a_ia_{i+1}\otimes a_{i+2}\otimes\cdots\otimes a_n$$
\item If $i=n$, then $$d_i^n(m\otimes a_1\otimes\cdots\otimes a_n)=ma_n\otimes a_1\otimes a_2\otimes\cdots\otimes a_{n-1}$$
\end{itemize}
\end{defn}

Highly related: Cotriple homology / cotriple constructions

\begin{prp}{}{} Let $A$ be an algebra over a ring $R$. Let $M$ be an $A$-algebra. Then the collection $$\{M\otimes A^{\otimes n},d_i^n\;|\;n\in\N\}$$ is a simplicial object. 
\end{prp}

\begin{defn}{Bar Resolutions}{} Let $A$ be an algebra over a ring $R$. Let $M$ be an $A$-algebra. Define the bar resolution of $M$ to be the associated chain complex of the simplicial object $$\{M\otimes A^{\otimes n},d_i^n\;|\;n\in\N\}$$ Explicitly, it is the chain complex \\~\\
\adjustbox{scale=1.0,center}{\begin{tikzcd}
	\cdots & {A^{\otimes n+1}\otimes M} & {A^{\otimes n}\otimes M} & {A^{\otimes n-1}\otimes M} & \cdots & A\otimes M &M & 0
	\arrow[from=1-1, to=1-2]
	\arrow[from=1-2, to=1-3]
	\arrow[from=1-3, to=1-4]
	\arrow[from=1-4, to=1-5]
	\arrow[from=1-5, to=1-6]
	\arrow[from=1-6, to=1-7]
	\arrow[from=1-7, to=1-8]
\end{tikzcd}}\\~\\
with the boundary map $\partial:A^{\otimes n}\otimes M\to A^{\otimes n-1}\otimes M$ given by $$\partial=\sum_{i=0}^n(-1)^id_i^n$$
\end{defn}

\pagebreak
\section{Simplicial Homotopy Theory}
Algebraic topology is a new subject which received a name change from combinatorial topology. At that time, combinatorics was highly involved in deriving topological invariants because the combinatorial structure of any invariants or spaces of study makes computation easy. In particular, a great deal of work has been put into the homotopy theory of simplicial spaces. \\~\\

Nowadays, the study of combinatorial objects in topology is less prominent, but the category of simplicial sets still play a distinguished role in algebraic topology. Indeed the category of simplicial sets is the prototypical example of a model category (we have not seen it), as well as exhibiting a Quillen equivalence between the model of a simplicial set and some model category structure of $\bold{Top}$. We will not explore the concept of model categories but will instead display foundational knowledge of it in order to develop the category of simplicial sets as a workable category when studying Model Categories. This means that we will develop the notion of homotopy, fibrations and cofibrations in this category. 

\subsection{Homotopies of Simplicial Objects}
\begin{defn}{Homotopy between Simplicial Maps}{} Let $\mC$ be a category. Let $f,g:X\to Y$ be two morphisms of simplicial objects. We say that $f$ and $g$ are homotopic if there is a family of morphisms $h_i^n:X_n\to Y_{n+1}$ such that the following are true. 
\begin{itemize}
\item $d_0^n\circ h_0=f_n$
\item $d_0^{n+1}\circ h_1=g_n$
\item The composition $$d_i\circ h_j=\begin{cases}
h_{j-1}\circ d_i^{n+1} & \text{ if } 0\leq i<j\leq n\\
d_i\circ h_{i-1} & \text{ if } i=j\neq 0\\
h_j\circ d_{i-1}& \text{ if } i>j+1
\end{cases}$$
\item The composition $$s_i\circ h_j=\begin{cases}
h_{j+1}\circ s_i & \text{ if } i\leq j\\
h_j\circ s_{i-1} & \text{ if } i>j
\end{cases}$$
\end{itemize}
\end{defn}

When $\mC$ is either the abelian category or $\bold{Set}$, we can provide an equivalence characterization that is reminiscent to the classical notion of homotopy in $\bold{Top}$. 

\begin{thm}{}{} Let $\mA$ be either an abelian category or $\bold{Set}$. Let $f,g:X\to Y$ be two morphisms of simplicial objects. Then $f$ and $g$ are homotopic if and only if there exists a morphism $\eta:X\times\Delta[1]\to Y$ such that the following diagram commutes: \\~\\
\adjustbox{scale=1.0,center}{\begin{tikzcd}
	{X\simeq X\times\Delta[0]} & {X\times\Delta[1]} & {X\simeq X\times\Delta[0]} \\
	\\
	& Y
	\arrow["{\text{id}_X\times\delta_1}", from=1-1, to=1-2]
	\arrow["f"', from=1-1, to=3-2]
	\arrow["{\exists\eta}"{description}, dashed, from=1-2, to=3-2]
	\arrow["{\text{id}_X\times\delta_1}"', from=1-3, to=1-2]
	\arrow["g", from=1-3, to=3-2]
\end{tikzcd}}\\~\\
\end{thm}

Here, we think of $\Delta[1]$ as playing the role of the unit interval $I=[0,1]$ in $\bold{Top}$.

\begin{prp}{}{} Let $\mA$ be an abelian category. Let $f,g:X\to Y$ be two morphisms of simplicial objects. Then the induced map $$f_\ast,g_\ast:N_\bullet(X)\to N_\bullet(Y)$$ between normalized chain complexes are chain homotopic $f\simeq g$. 
\end{prp}

\subsection{Horns and Fillers}
\begin{defn}{Inner and Outer Horns}{} Let $n\in\N$ and consider the standard $n$-simplex $\Delta^n$. Define the $i$th horn $\Lambda_i^n$ of $\Delta^n$ to be the the simplicial subset generated by all the faces $\partial_k\Delta^n$ except the $i$th one. It is called inner if $0<i<n$. It is called outer otherwise. 
\end{defn}

\begin{defn}{Fillers for a Horn}{} Let $n\in\N$ and consider the standard $n$-simplex $\Delta^n$. Let $\Lambda_i^n$ be a horn. We say that $\Lambda$ admits a filler if for all maps $F:\Lambda_i^n\to C$ there exists a map $U:\Delta^n\to C$ such that the following diagram commutes: \\~\\
\adjustbox{scale=1.0,center}{\begin{tikzcd}
	{\Lambda_i^n} & C \\
	{\Delta^n}
	\arrow["F", from=1-1, to=1-2]
	\arrow[hook, from=1-1, to=2-1]
	\arrow["\exists U"', dashed, from=2-1, to=1-2]
\end{tikzcd}}
\end{defn}

\subsection{Fibrations and Cofibrations}
\begin{defn}{Kan Fibrations}{} Let $f:X\to Y$ be a morphism of simplicial sets. We say that $f$ is a Kan fibration if the following condition is satisfied: For every commutative diagram: \\~\\
\adjustbox{scale=1.0,center}{\begin{tikzcd}
	{\Lambda_k^n} & X \\
	{\Delta^n} & Y
	\arrow[from=1-1, to=1-2]
	\arrow[hook, from=1-1, to=2-1]
	\arrow["f", from=1-2, to=2-2]
	\arrow[from=2-1, to=2-2]
\end{tikzcd}}\\~\\
where $n\geq 1$ and $0\leq k\leq n$, there exists a lift $\Delta^n\to Y$ such that the following diagram commutes: \\~\\
\adjustbox{scale=1.0,center}{\begin{tikzcd}
	{\Lambda_k^n} & X \\
	{\Delta^n} & Y
	\arrow[from=1-1, to=1-2]
	\arrow[hook, from=1-1, to=2-1]
	\arrow["f", from=1-2, to=2-2]
	\arrow[dashed, from=2-1, to=1-2]
	\arrow[from=2-1, to=2-2]
\end{tikzcd}}\\~\\
\end{defn}

Kan fibrations are the combinatorial analogue of Serre fibrations. Indeed notice that Kan fibrations satisfies a homotopy lift property similar to that of Serre fibrations. 

\begin{defn}{Kan Complexes}{} Let $X$ be a simplicial set. We say that $X$ is a Kan complex if the unique map $X\to\ast$ is a Kan fibration. 
\end{defn}

\begin{lmm}{}{} Let $X$ be a space. Then $S(X)$ is a Kan complex. 
\end{lmm}

\begin{thm}{}{} Let $X$ be a simplicial set. Then $X$ is a Kan complex if and only if every horn of $X$ admits a filler. 
\end{thm}

\begin{defn}{Weak Equivalences}{} Let $f:X\to Y$ be a map of simplicial sets. We say that $f$ is a weak equivalence if $f$ induces isomorphisms $$f_\ast:\pi_n(\abs{X},v)\overset{\cong}{\longrightarrow}\pi_n(\abs{Y},v)$$ for all $n\in\N$. We say that $X$ and $Y$ are weakly equivalent if there exists a weak equivalence $f:X\to Y$. 
\end{defn}

\begin{defn}{Fibrant Replacement}{}
\end{defn}

\begin{thm}{}{} Every simplicial set admits a fibrant replacement. 
\end{thm}

\subsection{The Simplicial Homotopy Groups}
\begin{defn}{Simplicial Homotopy Groups}{} Let $X$ be a Kan complex. Let $v\in X$ be a vertex of $X$. Define the simplicial homotopy groups of $(X,v)$ as follows: 
\begin{itemize}
\item For $n\geq 1$, define the $n$th simplicial homotopy group $\pi_n^\text{sSet}(X,v)$ of $X$ at $v$ to be the set of homotopy classes of maps $[\alpha:\Delta^n\to X]$ relative to boundary $\partial\Delta^n$ such that the following diagram commutes: \\~\\
\adjustbox{scale=1.0,center}{\begin{tikzcd}
	{\partial\Delta^n} & {\Delta^0} \\
	{\Delta^n} & X
	\arrow["{!}", from=1-1, to=1-2]
	\arrow[hook, from=1-1, to=2-1]
	\arrow["v", from=1-2, to=2-2]
	\arrow["\alpha"', from=2-1, to=2-2]
\end{tikzcd}}\\~\\
\item Define the $0$th simplicial homotopy group $\pi_0^\text{sSet}(X)$ of $X$ to be the set of homotopy classes of vertices of $X$. 
\end{itemize}
If $X$ is a general simplicial set, define the simplicial homotopy group of $X$ to be $$\pi_n^\text{sSet}(X,v)=\pi_n^\text{sSet}(PX,Pv)$$ where $PX$ is a fibrant replacement of $X$. 
\end{defn}

\begin{thm}{}{} Let $X$ be a Kan complex. Let $f,g:\Delta^n\to X$ be two representatives of two elements in $\pi_n(X,v)$ for $n\geq 1$. Then the following data $$v_i=\begin{cases}
s_0^n(v) & \text{ if } 0\leq i\leq n-2\\
f & \text{ if } i=n-1\\
g & \text{ if } i=n+1
\end{cases}$$
defines a horn $\Lambda_i^{n+1}\to X$. Such a map extends to a map $\theta:\Delta^{n+1}\to X$. Define $$[f]\cdot[g]=d_n\theta$$ Then such an operation is well defined on the equivalence class. Moreover, it defines a group operation on $\pi_n^\text{sSet}(X,v)$. 
\end{thm}

\begin{thm}{}{} Let $X$ be a Kan complex. Then for $n\geq 2$, the above group structure on $\pi_n^\text{sSet}(X,v)$ is abelian. 
\end{thm}

\begin{thm}{}{} Let $X$ be a simplicial set. Let $v\in X_0$. Then there is an isomorphism $$\pi_n^\text{sSet}(X,v)\cong\pi_n(\abs{X},v)$$ for all $n\in\N$. 
\end{thm}

\pagebreak
\section{The Nerve-Homotopy Adjunction}
\subsection{The Nerve of a Category}
\begin{defn}{Nerve of a Category}{} Let $\mC$ be a category. Define the nerve of the category $N(C):\Delta\to\text{sSet}$ as follows. 
\begin{itemize}
\item For $n\in\N$, $N(C)_n$ consists of paths of morphisms with $n$ compositions: \\~\\
\adjustbox{scale=1.0,center}{\begin{tikzcd}
	{c_0} & {c_1} & {c_2} & \cdots & {c_n}
	\arrow["{f_1}", from=1-1, to=1-2]
	\arrow["{f_2}", from=1-2, to=1-3]
	\arrow[from=1-3, to=1-4]
	\arrow[from=1-4, to=1-5]
\end{tikzcd}}\\~\\
\item The face map $d_i:C_n\to C_{n-1}$ sends the above element to \\~\\
\adjustbox{scale=1.0,center}{\begin{tikzcd}
	{c_0} & {c_1} & \cdots & {c_i} & {c_i} & \cdots & {c_n}
	\arrow["{f_1}", from=1-1, to=1-2]
	\arrow[from=1-2, to=1-3]
	\arrow[from=1-3, to=1-4]
	\arrow["{\text{id}_{c_i}}", from=1-4, to=1-5]
	\arrow[from=1-5, to=1-6]
	\arrow[from=1-6, to=1-7]
\end{tikzcd}}\\~\\
\item The degeneracy map $s^i:C_n\to C_{n+1}$ sends the above element to 
\end{itemize}
\end{defn}

\begin{defn}{Nerve Functor}{} The nerve functor $N:\text{Cat}\to\text{sSet}$ is defined as follows. 
\begin{itemize}
\item Each $\mC\in\text{Cat}$ is sent to the nerve $N(C)$
\item Every functor $\mC\to\mD$ in $\text{Cat}$ is sent to the morphism of presheaves $\lambda:N(C)\to N(D)$ defined by $\lambda_n:N(C)([n])\to N(D)([n])$, of which is defined as the map \\~\\
\adjustbox{scale=1.0,center}{\begin{tikzcd}
	{c_0} & {c_1} & {c_2} & \cdots & {c_n} \\
	{F(c_0)} & {F(c_1)} & {F(c_2)} & \cdots & {F(c_n)}
	\arrow["{f_1}", from=1-1, to=1-2]
	\arrow["{f_2}", from=1-2, to=1-3]
	\arrow[from=1-3, to=1-4]
	\arrow[from=1-4, to=1-5]
	\arrow["{F(f_1)}", from=2-1, to=2-2]
	\arrow["{F(f_2)}", from=2-2, to=2-3]
	\arrow[from=2-3, to=2-4]
	\arrow[from=2-4, to=2-5]
\end{tikzcd}}\\~\\
from the upper path in $\mC$ to the lower path in $\mD$, such that the following diagram commutes: \\~\\
\adjustbox{scale=1.0,center}{\begin{tikzcd}
	{N(C)[n]} & {N(C)[m]} \\
	{N(D)[n]} & {N(D)[m]}
	\arrow["{N(C)(f)}", from=1-1, to=1-2]
	\arrow["{\lambda_n}"', from=1-1, to=2-1]
	\arrow["{\lambda_m}", from=1-2, to=2-2]
	\arrow["{N(D)(f)}"', from=2-1, to=2-2]
\end{tikzcd}}\\~\\
where $f:[m]\to[n]$ is a morphism in $\Delta$. 
\end{itemize}
\end{defn}

\begin{thm}{}{} The nerve functor $N:\text{Cat}\to\text{sSet}$ is fully faithful. Moreover, the nerve of a category is a complete invariant for categories. 
\end{thm}

\subsection{The Homotopy Category of a Simplicial Set}
\begin{defn}{Homotopy Category of Simplicial Sets}{} Let $X\in\bold{sSet}$ be a simplicial set. Define the homotopy category $h(X)$ of $X$ as follows: 
\begin{itemize}
\item The objects are the zero set $h(X)=X_0$
\item Every $f\in X_1=\Hom_\bold{sSet}(\Delta^1,X)$ gives a map $d_1(f)\to d_0(f)$.
\item Composition is generated arbitrarily: for every $f$ and $g$ morphisms such that $d_0(f)=d_1(g)$, define a new morphism $h=g\ast f$. Then further modulo the morphisms by defining relations as follows: For every two simplex $\sigma:\Delta^2\to X$ such that $d_0(\sigma)=g$, $d_1(\sigma)=h$ and $d_2(\sigma)=f$, define $h\sim g\ast f$. 
\end{itemize}
\end{defn}

\begin{defn}{The Homotopy Functor}{} Define the homotopy functor $h:\text{sSet}\to\text{Cat}$ as follows. 
\begin{itemize}
\item On objects, $h$ sends a simplicial set $S:\Delta\to\text{Set}$ to 
\end{itemize}
\end{defn}

\begin{thm}{}{} The homotopy functor $h:\text{sSet}\to\text{Cat}$ is left adjoint to the nerve functor $N:\text{Cat}\to\text{sSet}$. This means that there is a natural bijection $$\Hom_\text{Cat}(h(C),D)\cong\Hom_\text{sSet}(C,N(D))$$
\end{thm}

\subsection{The Classifying Space}
\begin{defn}{Classifying Space of a Category}{} Let $\mC$ be a category. Define the classifying space of $\mC$ to be $$B\mC=|N(\mC)|$$ the geometric realization of the nerve $N(\mC)$. 
\end{defn}

\begin{defn}{Classifying Space of a Group}{} Let $G$ be a group. Define the classifying space of $G$ to be $$BG=|N(G)|$$ where $G$ here is considered as a groupoid with one element. 
\end{defn}





\end{document}
