\documentclass[a4paper]{article}

\input{C:/Users/liula/Desktop/Latex/Headers V1.2.tex}

\pagestyle{fancy}
\fancyhf{}
\rhead{Labix}
\lhead{Algebraic Curves}
\rfoot{\thepage}

\title{Algebraic Curves}

\author{Labix}

\date{\today}
\begin{document}
\maketitle
\begin{abstract}
\end{abstract}
\pagebreak
\tableofcontents
\pagebreak

\section{Algebraic Curves in Classical Algebraic Geometry}
\subsection{Basic Properties of Curves}
\begin{defn}{Curves}{} Let $k$ be a field. Let $X$ be a variety over $k$. We say that $X$ is a curve if $\dim(X)=1$. 
\end{defn}

\begin{prp}{}{} Let $k$ be an algebraically closed field. Let $C$ be an irreducible curve over $k$. Let $p\in C$ be a non-singular point. Then $\mO_{C,p}$ is a DVR. Moreover, the valuation is given by the degree of the regular function. \tcbline
\begin{proof}
Since $p$ is non-singular, by definition $\mO_{C,p}$ is a regular local ring. Moreover, we know that $1=\dim(C)=\dim(\mO_{C,p})$ so that $\mO_{C,p}$ has Krull dimension $1$. By the equivalent characterization of DVR we conclude. 
\end{proof}
\end{prp}

We denote the valuation map by $v_p:\text{Frac}(\mO_{C,p})\to\Z$. 

\begin{eg}{}{} Consider the projective curve $C=\V(x^2+y^2-z^2)\subset\Prj_\C^2$. Let $p=[p_0:p_1:p_2]$ be a point on the curve. \\~\\

If $p_2\neq 0$, then $p\in U_2$. Under the affine chart $(U_2,\varphi_2)$, we find that $C_2=\varphi_2(C\cap U_2)=\V(x^2+y^2-1)$. The corresponding coordinate ring is given by $\frac{\C[x,y]}{(x^2+y^2-1)}$. The formula for the local ring in the affine case gives $$\mO_{C,p}\cong\left(\frac{\C[x,y]}{(x^2+y^2-1)}\right)_{m_{(p_0/p_2,p_1/p_2)}}$$ Recall that the unique maximal ideal of the local ring is given as the $\mO_{X,p}$-module $m_p=\{f\in\C[C_2]\;|\;f(p_0/p_2,p_1/p_2)=0\}$, which under the nullstellensatz is the maximal ideal corresponding to the point $(p_0/p_2,p_1/p_2)$ and is given by $m_p=(x-r,y-s)$ where $r=p_0/p_1$ and $s=p_0/p_2$. By Nakayama's lemma, since $x-r,y-s$ generate $m_p$ we know that $x-r+m_p^2,y-s+m_p^2$ span the vector space $m_p/m_p^2$ over $\mO_{X,p}/m_p$. I claim that they are linearly dependent. This mean that I want to find $f+m_p^2$ and $g+m_p^2$ in $\mO_{X,p}/m_p$ that are non-trivial, and that $(x-r)f+(y-s)g+m_p^2=m_p^2$. This means that we want to find $f,g\in\mO_{X,p}\setminus m_p$ such that $(x-r)f+(y-s)g\in m_p^2$. Choose $f=x+r$ and $g=y+s$ to get $$(x-r)(x+r)+(y-s)(y+s)=x^2-r^2+y^2-s^2=1-1=0$$ since $(r,s)$ lie on the curve. Moreover, $x+r,y+s\mO_{X,p}\setminus m_p$ since evaluating at $(r,s)$ at the functions are non-zero. This verifies that $\mO_{X,p}$ is a regular local ring of dimension $1$, hence is a DVR. \\~\\

We can even find its uniformizer and valuation. Since $x-r$ and $y+s$ are linearly dependent and spans $m_p/m_p^2$, any one of the two is a basis for the vector space. WLOG take $x-r$ to be a basis. Nakayama's lemma implies that $x-r$ generates $m_p$. Being a DVR means that for all $f\in\mO_{X,p}$, $f=u(x-r)^n$ where $u$ is invertible. Then the valuation of $f$ is $n$. 
\end{eg}

\begin{prp}{}{} Let $C$ be an affine irreducible curve over $\C$. Then $C$ is smooth if and only if $C$ is a normal variety. 
\end{prp}

\subsection{Regular Maps between Curves}
\begin{prp}{}{} Let $k$ be a field. Let $C$ be a smooth curve over $k$. Then for any projective variety $X\subseteq\Prj^n$ and rational map $\phi:C\to X$, there exists a regular map $$\overline{\phi}:C\to X$$ such that $\overline{\phi}|_U=\phi|_U$ for some dense subset $U\subseteq C$. 
\end{prp}

\begin{prp}{}{} Let $k$ be an algebraically closed field. Let $X,Y$ be smooth irreducible projective curves over $k$. Let $\phi:X\to Y$ be a non-constant regular map. Then $\phi$ is a finite morphism. 
\end{prp}

\subsection{Blowing Up Curves and Normalization}
Recall that by taking the integral closure of the coordinate ring $k[C]$ of an irreducible affine curve $C\subseteq\A^n$, we obtain a corresponding variety $\widetilde{C}$ called the normalization of $C$. 

\begin{prp}{}{} Let $k$ be an algebraically closed field. Let $C\subseteq\A_k^n$ be an irreducible affine curve over $k$. Then the normalization $\widetilde{C}$ is smooth. 
\end{prp}

\begin{thm}{}{} Let $k$ be an algebraically closed field. Let $C$ be an irreducible curve over $k$. Then $C$ is birational to a unique non-singular projective irreducible curve. 
\end{thm}

\subsection{Ramification Index}
\begin{defn}{Ramification Index}{} Let $k$ be an algebraically closed field. Let $X,Y$ be smooth irreducible projective curves over $k$. Let $\phi:X\to Y$ be a non-constant regular map. Let $p\in X$. Define the ramification index of $\phi$ at $p$ to be $$e_\phi(p)=v_p(\phi^\ast(\pi))$$ where $\pi$ is a uniformizing parameter of $\mO_{Y,\phi(p)}$. 
\end{defn}

\begin{lmm}{}{} Let $k$ be an algebraically closed field. Let $X,Y$ be smooth irreducible projective curves over $k$. Let $\phi:X\to Y$ be a non-constant regular map. Let $p\in X$. Then $$e_\phi(p)=\dim_k\left(\frac{\mO_{X,p}}{(\phi^\ast(\pi)}\right)$$ where $\pi$ is a uniformizing parameter of $\mO_{Y,\phi(p)}$. 
\end{lmm}

Let $\phi:X\to Y$ be a non-constant regular map between smooth irreducible and projective curves. Since $\phi$ is finite, the notion of degree makes sense. Recall that the degree is defined to be $$\deg(\phi)=\dim_{K(Y)}K(X)$$

\begin{prp}{}{} Let $k$ be an algebraically closed field. Let $X,Y$ be smooth irreducible projective curves over $k$. Let $\phi:X\to Y$ be a non-constant regular map. Let $q\in Y$. Then we have $$\sum_{p\in\phi^{-1}(q)}e_\phi(p)=\deg(\phi)$$
\end{prp}

\pagebreak
\section{Classical Divisors on Curves}
\subsection{The Pullback Map of Divisors}
\begin{defn}{Pullback Map of Divisors}{} Let $k$ be an algebraically closed field. Let $X,Y$ be smooth irreducible projective curves over $k$. Let $\phi:X\to Y$ be a non-constant regular map. Define the induced pullback map $\phi^\ast:\text{Div}(Y)\to\text{Div}(X)$ by $$\phi^\ast\left(\sum_{q\in Y} n_q\cdot q\right)=\sum_{q\in Y}n_q\cdot\left(\sum_{p\in\phi^{-1}(q)}e_\phi(p)\cdot p\right)=\sum_{p\in X}n_{\phi(p)}e_\phi(p)\cdot p$$
\end{defn}

\begin{prp}{}{} Let $k$ be an algebraically closed field. Let $X,Y$ be smooth irreducible projective curves over $k$. Let $\phi:X\to Y$ be a non-constant regular map. Then we have $$\deg(\phi^\ast(D))=\deg(\phi)\deg(D)$$ for any $D\in\text{Div}(Y)$. 
\end{prp}

\begin{prp}{}{} Let $k$ be an algebraically closed field. Let $X$ be a smooth irreducible projective curve over $k$. Let $D\in\text{Div}(X)$ be a principal divisor of $X$. Then $\deg(D)=0$. 
\end{prp}

\begin{prp}{}{} Let $k$ be an algebraically closed field. Let $X,Y$ be smooth irreducible projective curves over $k$. Let $\phi:X\to Y$ be a non-constant regular map. Then $\phi(\text{Prin}(Y))\subseteq\text{Prin}(X)$. 
\end{prp}

\begin{defn}{Induced Map of Divisor Class Groups}{} Let $k$ be an algebraically closed field. Let $X,Y$ be smooth irreducible projective curves over $k$. Let $\phi:X\to Y$ be a non-constant regular map. Define the induced map of divisor class groups $\phi^\ast:\text{Cl}(Y)\to\text{Cl}(X)$ by $$\phi^\ast([D])=[\phi^\ast(D)]$$
\end{defn}

\subsection{The Linear System of Divisors}
\begin{defn}{The Linear System of Divisors}{} Let $k$ be an algebraically closed field. Let $X$ be a smooth irreducible projective curve over $k$. Let $D\in\text{Div}(X)$ be a divisor. Define the linear system of $D$ to be $$\mL(D)=\{0\}\cup\{f\in K(X)\;|\;\deg(D+\text{div}(f))\geq 0\}\subseteq K(X)$$
\end{defn}

\begin{lmm}{}{} Let $k$ be an algebraically closed field. Let $X$ be a smooth irreducible projective curve over $k$. Let $D\in\text{Div}(X)$ be a divisor. Then $\mL(D)$ is a vector space over $k$. 
\end{lmm}

\begin{prp}{}{} Let $k$ be an algebraically closed field. Let $X$ be a smooth irreducible projective curve over $k$. Let $D,D'\in\text{Div}(X)$ be divisors. If $D\sim D'$ are linearly equivalent, then we have $$\dim_k(\mL(D))=\dim_k(\mL(D'))$$
\end{prp}

\begin{prp}{}{} Let $k$ be an algebraically closed field. Let $X$ be a smooth irreducible projective curve over $k$. Let $D\in\text{Div}(X)$ be a divisor. Then the following are true. 
\begin{itemize}
\item If $\deg(D)<0$, then we have $$\dim_k(\mL(D))=0$$
\item If $\deg(D)=0$, then we have $$\dim_k(\mL(D))=\begin{cases}
0 & \text{ if }D\not\sim0\\
1 & \text{ if }D\sim 0
\end{cases}$$
\end{itemize}
\end{prp}

\begin{prp}{}{} Let $k$ be an algebraically closed field. Let $X$ be a smooth irreducible projective curve over $k$. Let $D\in\text{Div}(X)$ be a divisor. Then we have $$\dim_k(\mL(D))\leq\deg(D)-1$$
\end{prp}

\subsection{The Riemann-Roch Theorem}
\begin{thm}{Riemann-Roch Theorem}{} Let $k$ be an algebraically closed field. Let $X$ be a smooth irreducible projective curve over $k$. Let $D\in\text{Div}(X)$ be a divisor on $X$ and let $K$ be the canonical divisor of $X$. Let $\mL(D)$ be the associated sheaf of the divisor $D$. Then $$\dim_k(\mL(D))+\dim_k(\mL(K-D))=\deg(D)+1-p_g(X)$$
\end{thm}


\pagebreak
\section{Algebraic Curves in the Context of Schemes}
\begin{defn}{Algebraic Curves}{} Let $k$ be an algebraically closed field. A curve over $k$ is an integral separated scheme $X$ of finite type over $k$ that has dimension $1$. 
\end{defn}

\begin{prp}{}{} Let $X$ be an algebraic curve. Then the arithmetic and geometric genus coincide. In particular, $$p_a(X)=p_g(X)=\dim_kH^1(X,\mO_X)$$
\end{prp}

We will simply call the genus of a curve $g$ from now on since the arithmetic genus is the same as the geometric genus. 

\subsection{Riemann-Roch Theorem}
\begin{defn}{Canonical Divisor}{} Let $X$ be an algebraic curve. The canonical divisor $K$ of $X$ is a divisor in the linear equivalence class of $$\Omega_{X/k}^1=\omega_X$$
\end{defn}

\begin{thm}{Riemann-Roch Theorem}{} Let $X$ be an algebraic curve. Let $D$ be a divisor on $X$ and let $K$ be the canonical divisor of $X$. Let $\mL(D)$ be the associated sheaf of the divisor $D$. Then $$\dim_k(H^0(X,\mL(D)))+\dim_k(H^0(X,\mL(K-D)))=\deg(D)+1-p_g(X)$$
\end{thm}

\subsection{Classification of Curves in $\Prj^3$}




















\end{document}