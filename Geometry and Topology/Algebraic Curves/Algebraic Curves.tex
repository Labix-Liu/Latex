\documentclass[a4paper]{article}

\input{C:/Users/liula/Desktop/Latex/Headers V1.2.tex}

\pagestyle{fancy}
\fancyhf{}
\rhead{Labix}
\lhead{Algebraic Curves}
\rfoot{\thepage}

\title{Algebraic Curves}

\author{Labix}

\date{\today}
\begin{document}
\maketitle
\begin{abstract}
\end{abstract}
\pagebreak
\tableofcontents
\pagebreak

\section{Algebraic Curves in Classical Algebraic Geometry}
\subsection{Basic Properties of Curves}
\begin{defn}{Curves}{} Let $k$ be a field. Let $X$ be a variety over $k$. We say that $X$ is a curve if $\dim(X)=1$. 
\end{defn}

\begin{prp}{}{} Let $k$ be an algebraically closed field. Let $C$ be an irreducible curve over $k$. Let $p\in C$ be a non-singular point. Then $\mO_{C,p}$ is a DVR. Moreover, the valuation is given by the degree of the regular function. 
\end{prp}

\begin{prp}{}{} Let $k$ be a field. Let $C$ be a smooth curve over $k$. Then for any projective variety $X\subseteq\Prj^n$ and rational map $\phi:C\to X$, there exists a regular map $$\overline{\phi}:C\to X$$ such that $\overline{\phi}|_U=\phi|_U$ for some dense subset $U\subseteq C$. 
\end{prp}

\subsection{Blowing Up Curves and Normalization}
Recall that by taking the integral closure of the coordinate ring $k[C]$ of an irreducible affine curve $C\subseteq\A^n$, we obtain a corresponding variety $\widetilde{C}$ called the normalization of $C$. 

\begin{prp}{}{} Let $k$ be an algebraically closed field. Let $C\subseteq\A_k^n$ be an irreducible affine curve over $k$. Then the normalization $\widetilde{C}$ is smooth. 
\end{prp}

\begin{thm}{}{} Let $k$ be an algebraically closed field. Let $C$ be an irreducible curve over $k$. Then $C$ is birational to a unique non-singular projective irreducible curve. 
\end{thm}

\pagebreak
\section{Algebraic Curves in the Context of Schemes}
\begin{defn}{Algebraic Curves}{} Let $k$ be an algebraically closed field. A curve over $k$ is an integral separated scheme $X$ of finite type over $k$ that has dimension $1$. 
\end{defn}

\begin{prp}{}{} Let $X$ be an algebraic curve. Then the arithmetic and geometric genus coincide. In particular, $$p_a(X)=p_g(X)=\dim_kH^1(X,\mO_X)$$
\end{prp}

We will simply call the genus of a curve $g$ from now on since the arithmetic genus is the same as the geometric genus. 

\subsection{Riemann-Roch Theorem}
\begin{defn}{Canonical Divisor}{} Let $X$ be an algebraic curve. The canonical divisor $K$ of $X$ is a divisor in the linear equivalence class of $$\Omega_{X/k}^1=\omega_X$$
\end{defn}

\begin{thm}{Riemann-Roch Theorem}{} Let $X$ be an algebraic curve. Let $D$ be a divisor on $X$ and let $K$ be the canonical divisor of $X$. Let $\mL(D)$ be the associated sheaf of the divisor $D$. Then $$\dim_k(H^0(X,\mL(D)))+\dim_k(H^0(X,\mL(K-D)))=\deg(D)+1-p_g(X)$$
\end{thm}

\subsection{Classification of Curves in $\Prj^3$}




















\end{document}