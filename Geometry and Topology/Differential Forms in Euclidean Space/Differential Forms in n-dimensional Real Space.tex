\documentclass[a4paper]{article}

%=========================================
% Packages
%=========================================
\usepackage{mathtools}
\usepackage{amsfonts}
\usepackage{amsmath}
\usepackage{amssymb}
\usepackage{amsthm}
\usepackage[a4paper, total={6in, 8in}, margin=1in]{geometry}
\usepackage[utf8]{inputenc}
\usepackage{fancyhdr}
\usepackage[utf8]{inputenc}
\usepackage{graphicx}
\usepackage{physics}
\usepackage[listings]{tcolorbox}
\usepackage{hyperref}
\usepackage{tikz-cd}
\usepackage{adjustbox}
\usepackage{enumitem}


\hypersetup{
    colorlinks=true, %set true if you want colored links
    linktoc=all,     %set to all if you want both sections and subsections linked
    linkcolor=black,  %choose some color if you want links to stand out
}
\usetikzlibrary{arrows.meta}

\DeclarePairedDelimiter\ceil{\lceil}{\rceil}
\DeclarePairedDelimiter\floor{\lfloor}{\rfloor}

%=========================================
% Custom Math Operators
%=========================================
\DeclareMathOperator{\adj}{adj}
\DeclareMathOperator{\im}{im}
\DeclareMathOperator{\nullity}{nullity}
\DeclareMathOperator{\sign}{sign}
\DeclareMathOperator{\dom}{dom}
\DeclareMathOperator{\lcm}{lcm}
\DeclareMathOperator{\ran}{ran}
\DeclareMathOperator{\ext}{Ext}
\DeclareMathOperator{\dist}{dist}
\DeclareMathOperator{\diam}{diam}
\DeclareMathOperator{\aut}{Aut}
\DeclareMathOperator{\inn}{Inn}
\DeclareMathOperator{\syl}{Syl}
\DeclareMathOperator{\edo}{End}
\DeclareMathOperator{\cov}{Cov}
\DeclareMathOperator{\vari}{Var}
\DeclareMathOperator{\cha}{char}
\DeclareMathOperator{\Span}{span}
\DeclareMathOperator{\ord}{ord}
\DeclareMathOperator{\res}{res}
\DeclareMathOperator{\Hom}{Hom}
\DeclareMathOperator{\Mor}{Mor}
\DeclareMathOperator{\coker}{coker}
\DeclareMathOperator{\Obj}{Obj}
\DeclareMathOperator{\id}{id}
\DeclareMathOperator{\GL}{GL}
\DeclareMathOperator*{\colim}{colim}

%=========================================
% Custom Commands (Shortcuts)
%=========================================
\newcommand{\CP}{\mathbb{CP}}
\newcommand{\GG}{\mathbb{G}}
\newcommand{\F}{\mathbb{F}}
\newcommand{\N}{\mathbb{N}}
\newcommand{\Q}{\mathbb{Q}}
\newcommand{\R}{\mathbb{R}}
\newcommand{\C}{\mathbb{C}}
\newcommand{\E}{\mathbb{E}}
\newcommand{\Prj}{\mathbb{P}}
\newcommand{\RP}{\mathbb{RP}}
\newcommand{\T}{\mathbb{T}}
\newcommand{\Z}{\mathbb{Z}}
\newcommand{\A}{\mathbb{A}}
\renewcommand{\H}{\mathbb{H}}

\newcommand{\mA}{\mathcal{A}}
\newcommand{\mB}{\mathcal{B}}
\newcommand{\mC}{\mathcal{C}}
\newcommand{\mD}{\mathcal{D}}
\newcommand{\mE}{\mathcal{E}}
\newcommand{\mF}{\mathcal{F}}
\newcommand{\mG}{\mathcal{G}}
\newcommand{\mH}{\mathcal{H}}
\newcommand{\mJ}{\mathcal{J}}
\newcommand{\mO}{\mathcal{O}}
\newcommand{\mS}{\mathcal{S}}

%=========================================
% Theorem Environment
%=========================================
\newcommand\todoin[2][]{\todo[backgroundcolor=white!20!white, inline, caption={2do}, #1]{
\begin{minipage}{\textwidth-4pt}#2\end{minipage}}}

\tcbuselibrary{listings, theorems, breakable, skins}

\newtcbtheorem[number within=subsection]{thm}{Theorem}%
{colback=gray!5, colframe=gray!65!black, fonttitle=\bfseries, breakable, enhanced jigsaw, halign=left}{th}
\newtcbtheorem[number within=subsection, use counter from=thm]{defn}{Definition}%
{colback=gray!5, colframe=gray!65!black, fonttitle=\bfseries, breakable, enhanced jigsaw, halign=left}{th}
\newtcbtheorem[number within=subsection, use counter from=thm]{axm}{Axiom}%
{colback=gray!5, colframe=gray!65!black, fonttitle=\bfseries, breakable, enhanced jigsaw, halign=left}{th}
\newtcbtheorem[number within=subsection, use counter from=thm]{prp}{Proposition}%
{colback=gray!5, colframe=gray!65!black, fonttitle=\bfseries, breakable, enhanced jigsaw, halign=left}{th}
\newtcbtheorem[number within=subsection, use counter from=thm]{lmm}{Lemma}%
{colback=gray!5, colframe=gray!65!black, fonttitle=\bfseries, breakable, enhanced jigsaw, halign=left}{th}
\newtcbtheorem[number within=subsection, use counter from=thm]{crl}{Corollary}%
{colback=gray!5, colframe=gray!65!black, fonttitle=\bfseries, breakable, enhanced jigsaw, halign=left}{th}
\newtcbtheorem[number within=subsection, use counter from=thm]{eg}{Example}%
{colback=gray!5, colframe=gray!65!black, fonttitle=\bfseries, breakable, enhanced jigsaw, halign=left}{th}
\newtcbtheorem[number within=subsection, use counter from=thm]{ex}{Exercise}%
{colback=gray!5, colframe=gray!65!black, fonttitle=\bfseries, breakable, enhanced jigsaw, halign=left}{th}
\newtcbtheorem[number within=subsection, use counter from=thm]{alg}{Algorithm}%
{colback=gray!5, colframe=gray!65!black, fonttitle=\bfseries, breakable, enhanced jigsaw, halign=left}{th}

\newcounter{qtnc}
\newtcolorbox[use counter=qtnc]{qtn}%
{colback=gray!5, colframe=gray!65!black, fonttitle=\bfseries, breakable, enhanced jigsaw, halign=left}




\raggedright

\pagestyle{fancy}
\fancyhf{}
\rhead{Labix}
\lhead{Differential Forms in n-dimensional Real Space}
\rfoot{\thepage}

\title{Differential Forms in n-dimensional Real Space}

\author{Labix}

\date{\today}
\begin{document}
\maketitle
\begin{abstract} Before diving deep into the abstract Differential Geometry, iit is crucial to understand those methods we apply to non-euclidean surfaces by first studying it on a more familiar setting. 
\end{abstract}
\pagebreak
\tableofcontents
\pagebreak

\section{Introduction to Multilinear Algebra}
\subsection{Basic Definitions}
\begin{defn}[Multilinear Function] Let $V$ be a vector space over $\R$. A function $f:V^k\to\R$ is $k$-linear if it is linear in each of its $k$ arguments $$f(v_1,\dots,av_i+bw_i,\dots,v_k)=af(v_1,\dots,v_i,\dots,v_k)+bf(v_1,\dots,w_i,\dots,v_k)$$ for $i\in\{1,\dots,k\}$ nd $a,b\in\R$. It is also called a $k$-tensor on $V$. Denote the set of all $k$-tensors on $V$ by $L_k(V)$
\end{defn}

\begin{defn}[Symmetric] Let $V$ be a vector space over $\R$. $f:V^k\to\R$ is symmetric if $$f(v_{\sigma(1)},\dots,v_{\sigma(k)})=f(v_1,\dots,v_k)$$ for all $\sigma\in S_k$
\end{defn}

\begin{defn}[Alternating] Let $V$ be a vector space over $\R$. $f:V^k\to\R$ is alternating if $$f(v_{\sigma(1)},\dots,v_{\sigma(k)})=\sign(\sigma)f(v_1,\dots,v_k)$$ for all $\sigma\in S_k$. Alternating $k$-tensors are also called $k$-covectors. Denote the set of all $k$-covectors $\Lambda_k(V)$. Thus we have $\Lambda_k(V)\subseteq L_k(V)$
\end{defn}

\begin{defn} Let $f:V^k\to\R$ be a $k$-linear function. Define $$(Sf)(v_1,\dots,v_k)=\sum_{\sigma\in S_k}\sigma(f)$$ Define $$(Af)(v_1,\dots,v_k)=\sum_{\sigma\in S_k}\sign(\sigma)\sigma(f)$$
\end{defn}

\begin{prp} Let $f:V^k\to\R$ be a $k$-linear function. Then $Sf$ is symmetric and $Af$ is alternating. 
\end{prp}
\begin{proof} We have 
\begin{align*}
\tau(Sf)&=\sum_{\sigma\in S_k}(\tau\sigma)f\\
&=Sf
\end{align*} and
\begin{align*}\tau(Af)&=\sum_{\sigma\in S_k}\sign(\sigma)(\tau\sigma)f\\
&=\sign(\tau)\sum_{\sigma\in S_k}\sign(\tau\sigma)(\tau\sigma)f\\
&=\sign(\tau)(Af)
\end{align*}
\end{proof}

\begin{lmm} If $f$ is an alternating $k$-linear function on a vector space $V$, then $Af=(k!)f$. 
\end{lmm}
\begin{proof} We have
\begin{align*}
Af&=\sum_{\sigma\in S_k}\sign(\sigma)(\sigma f)\\
&=\sum_{\sigma\in S_k}\sign(\sigma)\sign(\sigma)f\\
&=\sum_{\sigma\in S_k}f\\
&=(k!)f
\end{align*}
\end{proof}

\subsection{Tensor Product and Wedge Product}
\begin{defn}[Tensor Product] Let $f$ be $k$-linear on $V$ and $g$ be $l$ linear on $V$. Their tensor product is defined to be the $k+l$ linear function $$(f\otimes g)(v_1,\dots,v_{k+l})=f(v_1,\dots,v_k)g(v_{k+1},\dots,v_{k+l})$$
\end{defn}

\begin{prp} Let $f,g,h$ be multilinear functions on $V$. Then $$f\otimes(g\otimes h)=(f\otimes g)\otimes h$$
\end{prp}

\begin{defn}[Wedge Product] Let $f\in\Lambda_k(V)$ and $g\in\Lambda_l(V)$. Their wedge product is defined to be the $k+l$ linear function $$f\wedge g=\frac{1}{k!l!}A(f\otimes g)$$
\end{defn}

\begin{prp} Let $f\in\Lambda_k(V)$ and $g\in\Lambda_l(V)$. Then $$f\wedge g=(-1)^{kl}g\wedge f$$
\end{prp}

\begin{crl} Let $f\in\Lambda_k(V)$ and $k$ is odd. Then $f\wedge f=0$
\end{crl}

\begin{prp} Let $f,g,h$ be multilinear functions on $V$. Then $$f\wedge(g\wedge h)=(f\wedge g)\wedge h$$
\end{prp}

\begin{prp} Let $f_k\in\Lambda_{d_k}(V)$ for $k\in\{1,\dots,n\}$. Then $$f_1\wedge\cdots\wedge f_n=\frac{1}{(d_1)!\cdots(d_n)!}A(f_1\otimes\cdots\otimes f_n)$$
\end{prp}

\begin{defn}[Multi-index Notation] Suppose that $V$ is a vector space and $\alpha^1,\dots,\alpha^n$ the dual basis of $V$. Define $I=(i_1,\dots,i_k)$ and write $\alpha^I$ for $\alpha^{i_1}\wedge\cdots\wedge\alpha^{i_k}$. We usually want $i_1<\dots<i_k$. 
\end{defn}

\begin{lmm} Let $e_1,\dots,e_n$ be a basis for $V$ and $\alpha^1,\dots,\alpha^n$ be the dual basis of $V$. Then $$\alpha^I(e_J)=\delta_J^I\begin{cases}
1 & \text{ if }I=J\\
0 & \text{ if }I\neq J
\end{cases}$$
\end{lmm}

\begin{prp} The set of all $\alpha^I$ where $I=(i_1<\dots<i_k)$ form a basis for the space $\Lambda_k(V)$. The dimension of $\Lambda_k(V)$ is $\binom{n}{k}$
\end{prp}

\begin{crl} If $k>\dim(V)$, then $\Lambda_k(V)=0$
\end{crl}

\pagebreak
\section{Tangent Vectors in $\R^n$}
\subsection{Tangent Space}
\begin{defn}[Tangent Space] The set of all vectors with tail at $p\in\R^n$ is denoted $T_p(\R^n)$. We write a point in $\R^n$ as $p=(p_1,\dots,p_n)$ and a vector $v$ in $T_p(\R^n)$ as $\langle v_1,\dots,v_n\rangle$
\end{defn}

\begin{defn}[Line Through a Point] The line through a point $p\in\R^n$ with direction $v$ has parametrization $$c(t)=(p_1+tv_1,\dots,p_n+tv_n)$$ with its $i$-component $c_i(t)=p_i+tv_i$
\end{defn}

\begin{defn}[Directional Derivative] Let $f:U\subseteq\R^n\to\R$ be $\mathcal{C}^\infty$. Let $v\in T_p(\R^n)$. The directional derivative of $f$ in the direction $v$ at $p$ is defined to be $$D_v(f)=\lim_{t\to 0}\frac{f(c(t))-f(p)}{t}=\frac{d}{dt}\bigg\vert_{t=0}f(c(t))$$
\end{defn}

\begin{prp} Let $f:U\subseteq\R^n\to\R$ be $\mathcal{C}^\infty$. Then $$D_v(f)=\sum_{k=1}^nv_k\frac{\partial f}{\partial x_k}\bigg\vert_p$$ and $D_v$ is a map from $\mathcal{C}_p^\infty(\R^n)\to\R$
\end{prp}

\begin{prp} The map $\phi:T_p(\R^n)\to\mathcal{D}_p(\R^n)$ given by $\phi(v)=D_v$ is an isomorphism of vector spaces. 
\end{prp}

\begin{prp} The standard basis of $T_p(\R^n)$ corresponds to $$\left\{\frac{\partial}{\partial x_1},\dots,\frac{\partial}{\partial x_n}\right\}$$
\end{prp}

\begin{defn}[Vector Fields] A vector field $X$ on an open subset $U$ of $\R^n$ is a function that assigns to each point $p$ in $U$ a tangent vector denoted $X_p\in T_p(R^n)$. This means that $X:\R^n\to T_p(\R^n)$
\end{defn}

\begin{prp} For every vector field $X$, $$X_p=\sum_{k=1}^na_k(p)\frac{\partial}{\partial x_k}\bigg\vert_p$$ where $a_k(p)\in\R$
\end{prp}

\pagebreak
\section{Differential Forms on $\R^n$}
\subsection{Differential $1$-forms}
\begin{defn}[Cotangent Space] Define the cotangent space to $\R^n$ at $p$ to be $T_p^\ast(\R^n)$, the dual space of $T_p(\R^n)$. 
\end{defn}

\begin{defn}[Differential $1$-form] A differential $1$-form is a function $\omega:U\subseteq\R^n\to\bigcup_{p\in U}T_p^\ast(\R^n)$ from $p\in\R^n$ to $\omega_p\in T_p^\ast(\R^n)$
\end{defn}

\begin{prp} Fix $f\in\mathcal{C}^\infty(\R^n)$. Define $df_p:T_p(\R^n)\to\R$ by $$(df)_p(X_p)=X_p(f)$$ Then the mapping $(df)(p)=(df)_p$ from $p$ to $(df)_p$ is a differential $1$-form. 
\end{prp}

\begin{prp} Suppose that $x_1,\dots,x_n$ are the standard coordinate for $\R^n$. Then for each point $p\in\R^n$, $$\left\{(dx_1)_p,\dots,(dx_n)_p\right\}$$ is the basis for $T_p^\ast(\R^n)$ dual to $$\left\{\frac{\partial}{\partial x_1},\dots,\frac{\partial}{\partial x_n}\right\}$$ in $T_p(\R^n)$
\end{prp}

\begin{prp} If $f:U\subseteq\R^n\to\R$ is $\mathcal{C}^\infty$, then $$df=\sum_{k=1}^n\frac{\partial f}{\partial x_k}dx_k$$
\end{prp}

\subsection{Differential $k$-forms}
\begin{defn}[Differential $k$-forms] A differential $k$-form $\omega$ on $U\subseteq\R^n$ is a function that assigns to each point $p\in U$ an alternating $k$-linear function. This means $\omega:\R^n\to\Lambda_k(T_p(\R^n))$ Denote $\Omega^k(U)$ the vector space of $\mathcal{C}^\infty$ $k$-forms on $U$. 
\end{defn}

\begin{prp} A differential $k$-form $\omega$ is of the form $$\omega=\sum_I\alpha_Idx^I$$ with $a_I:U\subseteq\R^n\to\R$
\end{prp}

\subsection{Exterior Derivative}
\begin{defn}[Exterior Derivative of $0$-forms] Let $f\in\mathcal{C}^\infty(U)$. Then $f$ is a $0$-form. Define its exterior derivative to be its differential $df\in\Omega^1(U)$. 
\end{defn}

\begin{defn}[Exterior Derivative of $k$-forms] Let $\omega=\sum_I\alpha_Idx^I\in\Omega^k(U)$. Define $$d\omega=\sum_Id\alpha_I\wedge dx^I=\sum_I\left(\sum_j\frac{\partial\alpha_I}{\partial x_j}dx_j\right)\wedge dx^I\in\Omega^{k+1}(U)$$
\end{defn}

\begin{prp} Let $\omega\in\Omega^k(\R^n)$. Then $d^2\omega=0$
\end{prp}

\begin{defn}[Closed Forms] A $k$-form $\omega$ on $U$ is closed if $d\omega=0$
\end{defn}

\begin{defn}[Exact Forms] A $k$-form $\omega$ on $U$ is exact if there exists a $k-1$ form $\tau$ such that $\omega=d\tau$. 
\end{defn}











\end{document}