\documentclass[a4paper]{article}

\input{C:/Users/liula/Desktop/Latex/Headers V1.2.tex}

\pagestyle{fancy}
\fancyhf{}
\rhead{Labix}
\lhead{Geometric Group Theory}
\rfoot{\thepage}

\title{Geometric Group Theory}

\author{Labix}

\date{\today}
\begin{document}
\maketitle
\begin{abstract}
Potentially good books: Humphreys, Erdmann and Wildson
\end{abstract}
\pagebreak
\tableofcontents

\pagebreak
\section{The Geometry of Presentations}
\subsection{The Cayley Graph of a Group}
\begin{defn}{The Cayley Graph of a Group}{} Let $G$ be a group. Let $S$ be a generating set of $G$. Define the Cayley graph $\text{Cay}(G,S)$ of $G$ with respect to $S$ to consist of the following data. 
\begin{itemize}
\item The vertices are given by $V(\text{Cay}(G,S))=G$
\item The edges are given by $E(\text{Cay}(G,S))=\{(g,gs)\;|\;g\in G,s\in S\}$
\end{itemize}
\end{defn}

Let $(V,E)$ be a graph. Recall that a graph automorphism consists of a bijective map of vertices and a bjective map of edges such that $$\{\phi(v),\phi(w)\}\in E$$ for all $\{v,w\}\in E$. They form a group by composition. 

\begin{lmm}{The Action Lemma}{} Let $G$ be a group. Let $S$ be a generating set of $G$. Then $G$ acts on the Cayley graph $\text{Cay}(G,S)$ of $G$ with respect to $S$ via tha map $$\cdot:G\times\text{Cay}(G,S)\to\text{Cay}(G,S)$$ defined by $h\cdot g=hg$ and $h\cdot (g,gs)=(hg,hgs)$. Moreover, the action is faithful. 
\end{lmm}

\begin{prp}{}{} Let $G$ be a group. Let $S$ be a generating set of $G$. Then the following are true regarding $\text{Cay}(G,S)$. 
\begin{itemize}
\item $\text{Cay}(G,S)$ has no embedded cycles. 
\item $\text{Cay}(G,S)$ is connected. 
\end{itemize}
\end{prp}

\begin{prp}{}{} Let $S$ be a set. Then $\text{Cay}(F_S,S)$ is a tree. 
\end{prp}

\begin{prp}{}{} Let $G$ be a group. Let $S$ be a generating set of $G$. Then $\text{Cay}(F_S,S)$ is a universal cover of $\text{Cay}(G,S)$. 
\end{prp}

\subsection{Giving the Cayley Graph a Metric}
Given a graph $\Gamma$, there are two ways to specify a path in $\Gamma$. 
\begin{itemize}
\item We can define a path by a sequence $\gamma_V:[n]\to V(\Gamma)$ of adjacent vertices. 
\item We can also define a path by a sequence $\gamma_E:[n-1]\to E(\Gamma)$ of edges. 
\end{itemize}

The above notation also indicates that any path is determined by either $n$ vertices or $n-1$ edges. 

\begin{defn}{The Word Metric}{} Let $G$ be a group. Let $S$ be a generating set of $G$. Define the word metric on $\text{Cay}(G,S)$ to be the map $$d_S:V(\text{Cay}(G,S))\times V(\text{Cay}(G,S))\to\N$$ given by $$d_S(g,h)=\min\{n\in\N\;|\;\gamma_V:[n]\to V(\text{Cay}(G,S))\text{ is a path from }g\text{ to }h\}$$
\end{defn}

\begin{lmm}{}{} Let $G$ be a group. Let $S$ be a generating set of $G$. Then $d_S$ is a metric on $\text{Cay}(G,S)$. 
\end{lmm}

\begin{prp}{}{} Let $G$ be a group. Let $S$ be a generating set of $G$. Let $g\in G$ be fixed. Then the map $$(h,k)\mapsto(gh,gk)$$ given by the action lemma is an isometry. In other words, $$d_S(h,k)=d_S(gh,gk)$$
\end{prp}

Let $X$ be a metric space with two metrics $d_1$ and $d_2$. Recall that $d_1$ and $d_2$ are bilipschitz equivalent if there exists two constants $0<c_1\leq c_2<\infty$ such that $$c_1d_1(x,y)\leq d_2(x,y)\leq c_2d_1(x,y)$$ for all $x,y\in X$. 

\begin{lmm}{}{} Let $G$ be a group. Let $S,T$ be generating sets of $G$. Then $d_S$ and $d_T$ are bilipschitz equivalent. 
\end{lmm}

\begin{defn}{The Word Norm}{} Let $G$ be a group. Let $S$ be a generating set of $G$. Let $\text{Cay}(G,S)$ be the Cayley complex of $G$ and $S$. Define the word norm of $g\in G$ to be $$\|g\|_S=d_S(1_G,g)$$
\end{defn}

\begin{lmm}{}{} Let $G$ be a group. Let $S$ be a generating set of $G$. Then the following are true. 
\begin{itemize}
\item $d_S(g,h)=\|g^{-1}h\|_S$ for all $g,h\in G$. 
\item $\|g^{-1}\|_S=\|g\|_S$ for all $g\in G$. 
\item $\|gh\|_S\leq\|g\|_S+\|h\|_S$ for all $g,h\in G$. 
\end{itemize}
\end{lmm}

\subsection{Realizing the Cayley Graph as a Connected Space}
We have proved that Cayley graphs are connected as graphs, in the sense that any two vertices are connected by a path. But a priori the graph is not connected as a topological space, whose topology is generated by the metric. 

\begin{defn}{Geometric Realization of Cayley Graphs}{} Let $G$ be a group. Let $S$ be a generating set of $G$. Define the geometric realization $\abs{\text{Cay}(G,S)}$ of the Cayley graph to be the space $$\abs{\text{Cay}(G,S)}=\frac{E(\text{Cay}(G,S))\times I}{\sim}$$ where $((g_1,g_1s_1),t_1)\sim((g_2,g_2s_2),t_2)$ if one of the following are true. 
\begin{itemize}
\item They describe the same vertex (with different representations of elements in $G$): $((g_1,g_1s_1),t_1)=((g_2,g_2s_2),t_2)$. 
\item They describe the same vertex but they lie on different edges: Either one of the following
\begin{itemize}
\item $g_1=g_2$ and $t_1=t_2=0$
\item $g_1=g_2s_2$ and $t_1=0$, $t_2=1$
\item $g_1s_1=g_2s_2$ and $t_1=t_2=1$
\item $g_1s_1=g_2$ and $t_1=1$, $t_2=0$
\end{itemize}
\item They describe the same point on an edge but different orientations: $(g_1,g_1s_1)=(g_2,g_2s_2^{-1})$ and $t_1=1-t_2$. 
\end{itemize}
\end{defn}

In particular, this gives a $1$-dimensional CW complex. \\

We can also give a metric on the realization so that its restriction to the actual Cayley graph recovers the word metric. 

\begin{defn}{Metric on realization}{} Let $G$ be a group. Let $S$ be a generating set of $G$. Define a metric $d:\abs{\text{Cay}(G,S)}\times\abs{\text{Cay}(G,S)}\to\R$ as follows. $$d([((g_1,g_1s_1)t_1)],[((g_2,g_2s_2),t_2)])=\begin{cases}
\abs{t_1-t_2} & \text{ if } (g_1,g_1s_1)=(g_2,g_2s_2)\\
\abs{t_1-(1-t_2)} & \text{ if }(g_1,g_1s_1)=(g_2s_2,g_2)\\
\min\left\{\substack{t_1+d_S(g_1,g_2)+t_2\\t_1+d_S(g_1,g_2s_2)+1-t_2\\1-t_1+d_S(g_1s_1,g_2)+t_2\\1-t_1+d_S(g_1s_1,g_2s_2)+1-t_2}\right\} & \text{otherwise}
\end{cases}$$
\end{defn}

We abuse notation sometimes and freely interchange the use of the Cayley graph and its geometric realization when the context is clear. 

\subsection{Geodesics on Cayley Graphs}
\begin{defn}{Geodesic Words}{} Let $G$ be a group. Let $S$ be a generating set. Let $\gamma_V:[n]\to V(\text{Cay}(G,S))$ be a path in $\text{Cay}(G,S)$. We say that $\gamma_V$ is a geodesic word if $$d_S(\gamma_V(0),\gamma_V(n))=n$$
\end{defn}

This is not the same definition as geodesics in metric spaces. (It doesn't make sense to talk about paths in $\text{Cay}(G,S)$ because it is a discrete topological space when we consider the topology generated by the metric). 

\begin{prp}{}{} Let $G$ be a group. Let $S$ be a generating set of $G$. Then $\gamma_V:[n]\to V(\text{Cay}(G,S))$ is a geodesic word if and only if $\abs{\gamma_V}:[0,n]\to\abs{\text{Cay}(G,S)}$ is a geodesic in the sense of metric spaces. 
\end{prp}

\begin{lmm}{}{} Let $G$ be a group. Let $S$ be a generating set. If $\gamma_V:[n]\to\text{Cay}(G,S)$ is a geodesic, then $\gamma_V(0)\ast\cdots\ast\gamma_V(n)$ is a reduced word. 
\end{lmm}

Note: The converse is not true. Consider $G=\langle a,b\rangle a^3=b^2$. Both $a^3$ and $b^2$ are reduced words but they have different lengths. \\

Note: geodesics are not the unique distance minimizing curve between two elements. Therefore we want to find a representative. 

\begin{defn}{Short Lex Ordering}{} Let $G$ be a group. Let $S$ be a finite generating set of $G$. Let $u,v\in F(S)$. We say that $$u<_{sl}v$$ if one of the following are true. 
\begin{itemize}
\item $\abs{u}<\abs{v}$
\item $\abs{u}=\abs{v}$ and there exists $w$ such that $u=w\ast u'$, $v=w\ast v'$ and $u'<_{sl}v'$. 
\end{itemize}
We call $<_{sl}$ the short lex ordering on $F(S)$. 
\end{defn}

\begin{lmm}{}{} Let $G$ be a group. Let $S$ be a generating set. Then $<_{sl}$ is a total order on $F(S)$. 
\end{lmm}

\begin{defn}{Short Lex Representative}{} Let $G$ be a group. Let $S$ be a generating set of $G$. Let $g\in G$. Define the short lex representative of $g$ with respect to $S$ to be $$\min_{<_{sl}}\left\{s\in F(S)\;|\;s=g\text{ in G}\right\}$$
\end{defn}

\begin{lmm}{}{} Let $G$ be a group. Let $S$ be a generating set of $G$. Any subword of a short lex representative with respect to $S$ is a short lex representative. 
\end{lmm}

\begin{crl}{}{} Let $G$ be a group. Let $S$ be a generating set of $G$. Then the set of paths in $\text{Cay}(G,S)$ consisting of short lex representatives form a spanning tree for $\text{Cay}(G,S)$. 
\end{crl}

\subsection{Growth Function}
\begin{defn}{Ball Around an Element}{} Let $G$ be a group. Let $S$ be a finite generating set of $G$. Let $R>0$. Define the ball around $g\in G$ with radius $n$ to be $$B_n^{G,S}(g)=\{h\in G\;|\;d_S(g,h)\leq n\}$$
\end{defn}

\begin{prp}{}{} Let $G$ be a group. Let $S$ be a finite generating set. Let $g,h\in G$. Then $$\abs{B_n^G(g)}=\abs{B_m^G(h)}$$ for any $n\in\N$. 
\end{prp}

\begin{defn}{Growth Function}{} Let $G$ be a group. Let $S$ be a finite generating set of $G$. Let $R>0$. Define the growth function $\Gamma_{G,S}:\N\to\N$ of $G$ with respect to $S$ to be $$\Gamma_{G,S}(n)=\abs{B_n^{G,S}(1_G)}$$ for $n\in\N$. 
\end{defn}

\begin{prp}{}{} Let $G$ be a group Let $S$ be a finite generating set of $G$. Then the following are true. 
\begin{itemize}
\item $\Gamma_{G,S}(m+n)\leq\Gamma_{G,S}(m)\Gamma_{G,S}(n)$ for all $m,n\in\N$
\item $\Gamma_{G,S}(n)\leq(2\abs{S}+1)^n$ for all $n\in\N$. 
\end{itemize} \tcbline
\begin{proof}
For any pair $(h,k)$ of elements of $G$ such that $d_S(1,h)=m$ and $d_S(1,k)=n$, we have that $$d_S(1_G,hk)\leq d_S(1_G,h)+d_S(h,hk)=d_S(1_G,h)+d_S(1_G,k)=m+n$$ This means that for any unique pair of elements $(h,k)$ with $h\in B_m^{G,S}(1_G)$ and $k\in B_n^{G,S}(1_G)$, there exists a possibly non-unique element $hk\in B_{m+n}^{G,S}(1_G)$. Hence $$\abs{B_{m+n}^{G,S}(1_G)}\leq\abs{B_m^{G,S}(1_G)}\cdot\abs{B_n^{G,S}(1_G)}$$ and so $\Gamma_{G,S}(m+n)\leq\Gamma_{G,S}(m)\Gamma_{G,S}(n)$. \\~\\

Notice that $\Gamma_{G,S}(1)=(2\abs{S}+1)$ since the paths of the Cayley graph is given by $S$ and their inverses. Together with the identity element which has zero norm gives the formula. We can then recursively apply the above inequality to get $$\Gamma_{G,S}(n)\leq\left(\Gamma_{G,S}(1)\right)^n=(2\abs{S}+1)^n$$
\end{proof}
\end{prp}

\begin{lmm}{}{} Let $G$ be a group Let $S$ be a finite generating set of $G$. Then the following are true. 
\begin{itemize}
\item $\Gamma_{G,S}(n)\leq\Gamma_{F(S),S}(n)$ for all $n\in\N$. 
\item $\Gamma_{G,S}(n)=\Gamma_{F(S),S}(n)$ for all $n\in\N$ if and only if $G\cong F(S)$. 
\end{itemize} \tcbline
\begin{proof}
The induced homomorphism $\phi:F(S)\to G$ sends $B_n^{F(S),S}(1_{F(S)})$ surjectively to $B_n^{F(S),S}(1_{F(S)})$. Indeed if $\gamma_V:[n]\to F(S)$ is a geodesic, then $\phi\circ\gamma_V$ may not be a geodesic so that $d_S(1_G,\phi\circ\gamma_V(n))\leq n$. This means that $\phi\circ\gamma_V(n)\in B_n^{G,S}(1_G)$. Conversely, if $g\in B_n^{G,S}(1_G)$ then $g=w_1\cdots w_n$ is a reduced word in $G$ for $w_1,\dots,w_n\in S$. Then $w_1\cdots w_n$ is also a reduced word in $F(S)$ and hence lie in $B_n^{F(S),S}(1_{F(S)})$. Moreover, $\phi(w_1\cdots w_n)=g$. Hence $\phi$ is surjective on the two balls. Then we have $$\Gamma_{G,S}(n)=\abs{B_n^{G,S}(1_G)}=\abs{\phi\left(B_n^{F(S),S}(1_{F(S)})\right)}\leq\abs{B_n^{F(S),S}(1_{F(S)})}=\Gamma_{F(S),S}(n)$$
\end{proof}
\end{lmm}

\begin{lmm}{}{} Let $S$ be a finite set. Then $$\Gamma_{F(S),S}(n)=\frac{1-\abs{S}(2\abs{S}-1)^n}{1-\abs{S}}$$ \tcbline
\begin{proof}
I claim that the number of reduced words of length $n$ is $2\abs{S}(2\abs{S}-1)^{n-1}$ when $n\geq 1$. We induct on $n$. When $n=1$, then any reduced word is just the choice of a letter. Hence there are $2\abs{S}$ number of reduced words of length $1$. Now suppose that the number of reduced words of length $k$ is given by $2\abs{S}(2\abs{S}-1)^{k-1}$. Any reduced word of length $k+1$ is given by the concatenation of a reduced word of length $k$ and a choice of letter that is not the inverse of the last element of the given word. Thus there are $2\abs{S}(2\abs{S}-1)^{k-1}\cdot(2\abs{S}-1=2\abs{S}(2\abs{S}-1)^k)$ number of reduced words of length $k+1$. This completely the induction step. \\~\\

Then we have 
\begin{align*}
\Gamma_{F(S),S}(n)&=1+\sum_{i=1}^n2\abs{S}(2\abs{S}-1)^{n-1}\\
&=1+2\abs{S}\sum_{i=0}^{n-1}(2\abs{S}-1)^i\\
&=1+2\abs{S}\frac{1-(2\abs{S}-1)^n}{1-2\abs{S}+1}\\
&=1+\abs{S}\frac{1-(2\abs{S}-1)^n}{1-\abs{S}}\\
&=\frac{1-\abs{S}+\abs{S}\left(1-(2\abs{S}-1)^n\right)}{1-\abs{S}}\\
&=\frac{1-\abs{S}(2\abs{S}-1)^n}{1-\abs{S}}
\end{align*}
\end{proof}
\end{lmm}

\begin{prp}{}{} Let $G$ be a group Let $S$ be a finite generating set of $G$. Then the following are equivalent. 
\begin{itemize}
\item $G$ is a finite group. 
\item $\Gamma_{G,S}$ is bounded. 
\item $\Gamma_{G,S}(n)=\Gamma_{G,S}(n+1)$ for some $n\in\N$. 
\end{itemize}
\end{prp}

\begin{lmm}{}{} Let $G$ be a group. Let $S,T$ be finite generating sets of $G$. Then there exists $C,D>0$ such that $$\Gamma_{G,S}(n)\leq C\Gamma_{G,T}(n)\;\;\;\;\text{ and }\;\;\;\;\Gamma_{G,T}(n)\leq D\Gamma_{G,S}(n)$$ for all $n\in\N$. 
\end{lmm}

\begin{thm}{}{} There exists a finitely generated group $G$ with finite generators $S$ such that $\Gamma_{G,S}$ has superpolynomial growth but subexponential growth. 
\end{thm}

\begin{thm}{[Hirsch 1958]}{} Let $G$ be a finitely generated nilpotent group. Let $H\leq G$ be a subgroup of $G$. Then $[G:H]$ is finite and $H$ is torsion-free. 
\end{thm}

\begin{thm}{[Jennings 1955]}{} Let $H$ be a finitely generated torsion-free and nilpotent group. Then $H$ is isomorphic to a subgroup of $H_d(\Z)$ for some $d\geq 1$. 
\end{thm}

Note: $H_d(\Z)$ is the upper triangular matrices of $SL_d(\Z)$. 

\begin{thm}{[Gromov 1981]}{} Let $G$ be a finitely generated group such that $\Gamma_{G,S}$ has at most polynomial growth. Then there exists some subgroup $H\leq G$ such that $[G:H]$ is finite and $H$ is nilpotent. 
\end{thm}

\begin{thm}{[Bass 1972, Guivarch 1973]}{} Let $G$ be a finitely generated nilpotent group. Then there exists $C,D,d\in\N$ such that $$Cn^d\leq\Gamma_{G,S}(n)\leq Dp^d$$ ($\Gamma_{G,S}$ has polynomial growth rate). 
\end{thm}

\subsection{Distortion}
\begin{defn}{Undistorted Subgroups}{} Let $G$ be a group. Let $S,T$ be generating sets of $G$. Let $H\leq G$ be a subgroup. We say that $H$ is undistorted in $G$ if there exists $C>0$ such that $$d_T(g,h)\leq C d_S(g,h)$$ for all $g,h\in H$. 
\end{defn}

Intuitively, this means that when we restrict the metric to the subgroup, the shortest path when we had in $H$ for two elements is still the shortest when we consider the two elements in $G$. 

\pagebreak
\section{The Geometry of Boundaries}
\subsection{Ends of a Group via Geodesic Rays}
\begin{defn}{Geodesic Ray}{} Let $G$ be a group. Let $S$ be a finite generating set of $G$. A geodesic ray in $\text{Cay}(G,S)$ is a continuous map $$\phi:[0,\infty)\to\text{Cay}(G,S)$$ such that if $B\subseteq\text{Cay}(G,S)$ is bounded then $\phi^{-1}(B)$ is bounded. 
\end{defn}

This is called proper rays in Loh. 

\begin{defn}{Ends of a Group}{} Let $G$ be a group. Let $S$ be a finite generating set of $G$. Define $$\text{Ends}(G,S)=\{\phi:[0,\infty)\to\text{Cay}(G,S)\;|\;\phi\text{ is a geodesic ray }\}/\sim$$ where $\phi_1\sim\phi_2$ if for all $n\in\N$, there exists $t\in\R$ such that $\im(\phi_1)\setminus B_n^{G,S}(1)$ and $\im(\phi_2)\setminus B_n^{G,S}(1)$ lie in the same path component of $\text{Cay}(G,S)\setminus B_n^{G,S}(1)$. 
\end{defn}

\subsection{Infinite Connected Components at Infinity}
\begin{defn}{Infinite Connected Components}{} Let $G$ be a group. Let $S$ be a finite generating set of $G$. Define the set of infinite connected components of $G$ with respect to $S$ by $$E_S(n)=\{[X]\in\pi_0(\text{Cay}(G,S)\setminus B_n^{G,S}(1))\;|\;\abs{X}=\infty\}$$
\end{defn}

\begin{defn}{}{} Let $G$ be a group. Let $S$ be a finite generating set of $G$. Define $$C_S(n)=\text{Cay}(G,S)\setminus\bigcup_{A\in E_S(n)}A$$
\end{defn}





\end{document}
