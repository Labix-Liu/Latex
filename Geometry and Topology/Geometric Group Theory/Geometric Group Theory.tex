\documentclass[a4paper]{article}

%=========================================
% Packages
%=========================================
\usepackage{mathtools}
\usepackage{amsfonts}
\usepackage{amsmath}
\usepackage{amssymb}
\usepackage{amsthm}
\usepackage[a4paper, total={6in, 8in}, margin=1in]{geometry}
\usepackage[utf8]{inputenc}
\usepackage{fancyhdr}
\usepackage[utf8]{inputenc}
\usepackage{graphicx}
\usepackage{physics}
\usepackage[listings]{tcolorbox}
\usepackage{hyperref}
\usepackage{tikz-cd}
\usepackage{adjustbox}
\usepackage{enumitem}
\usepackage[font=small,labelfont=bf]{caption}
\usepackage{subcaption}
\usepackage{wrapfig}
\usepackage{makecell}



\raggedright

\usetikzlibrary{arrows.meta}

\DeclarePairedDelimiter\ceil{\lceil}{\rceil}
\DeclarePairedDelimiter\floor{\lfloor}{\rfloor}

%=========================================
% Fonts
%=========================================
\usepackage{tgpagella}
\usepackage[T1]{fontenc}


%=========================================
% Custom Math Operators
%=========================================
\DeclareMathOperator{\adj}{adj}
\DeclareMathOperator{\im}{im}
\DeclareMathOperator{\nullity}{nullity}
\DeclareMathOperator{\sign}{sign}
\DeclareMathOperator{\dom}{dom}
\DeclareMathOperator{\lcm}{lcm}
\DeclareMathOperator{\ran}{ran}
\DeclareMathOperator{\ext}{Ext}
\DeclareMathOperator{\dist}{dist}
\DeclareMathOperator{\diam}{diam}
\DeclareMathOperator{\aut}{Aut}
\DeclareMathOperator{\inn}{Inn}
\DeclareMathOperator{\syl}{Syl}
\DeclareMathOperator{\edo}{End}
\DeclareMathOperator{\cov}{Cov}
\DeclareMathOperator{\vari}{Var}
\DeclareMathOperator{\cha}{char}
\DeclareMathOperator{\Span}{span}
\DeclareMathOperator{\ord}{ord}
\DeclareMathOperator{\res}{res}
\DeclareMathOperator{\Hom}{Hom}
\DeclareMathOperator{\Mor}{Mor}
\DeclareMathOperator{\coker}{coker}
\DeclareMathOperator{\Obj}{Obj}
\DeclareMathOperator{\id}{id}
\DeclareMathOperator{\GL}{GL}
\DeclareMathOperator*{\colim}{colim}

%=========================================
% Custom Commands (Shortcuts)
%=========================================
\newcommand{\CP}{\mathbb{CP}}
\newcommand{\GG}{\mathbb{G}}
\newcommand{\F}{\mathbb{F}}
\newcommand{\N}{\mathbb{N}}
\newcommand{\Q}{\mathbb{Q}}
\newcommand{\R}{\mathbb{R}}
\newcommand{\C}{\mathbb{C}}
\newcommand{\E}{\mathbb{E}}
\newcommand{\Prj}{\mathbb{P}}
\newcommand{\RP}{\mathbb{RP}}
\newcommand{\T}{\mathbb{T}}
\newcommand{\Z}{\mathbb{Z}}
\newcommand{\A}{\mathbb{A}}
\renewcommand{\H}{\mathbb{H}}
\newcommand{\K}{\mathbb{K}}

\newcommand{\mA}{\mathcal{A}}
\newcommand{\mB}{\mathcal{B}}
\newcommand{\mC}{\mathcal{C}}
\newcommand{\mD}{\mathcal{D}}
\newcommand{\mE}{\mathcal{E}}
\newcommand{\mF}{\mathcal{F}}
\newcommand{\mG}{\mathcal{G}}
\newcommand{\mH}{\mathcal{H}}
\newcommand{\mI}{\mathcal{I}}
\newcommand{\mJ}{\mathcal{J}}
\newcommand{\mK}{\mathcal{K}}
\newcommand{\mL}{\mathcal{L}}
\newcommand{\mM}{\mathcal{M}}
\newcommand{\mO}{\mathcal{O}}
\newcommand{\mP}{\mathcal{P}}
\newcommand{\mS}{\mathcal{S}}
\newcommand{\mT}{\mathcal{T}}
\newcommand{\mV}{\mathcal{V}}
\newcommand{\mW}{\mathcal{W}}

%=========================================
% Colours!!!
%=========================================
\definecolor{LightBlue}{HTML}{2D64A6}
\definecolor{ForestGreen}{HTML}{4BA150}
\definecolor{DarkBlue}{HTML}{000080}
\definecolor{LightPurple}{HTML}{cc99ff}
\definecolor{LightOrange}{HTML}{ffc34d}
\definecolor{Buff}{HTML}{DDAE7E}
\definecolor{Sunset}{HTML}{F2C57C}
\definecolor{Wenge}{HTML}{584B53}
\definecolor{Coolgray}{HTML}{9098CB}
\definecolor{Lavender}{HTML}{D6E3F8}
\definecolor{Glaucous}{HTML}{828BC4}
\definecolor{Mauve}{HTML}{C7A8F0}
\definecolor{Darkred}{HTML}{880808}
\definecolor{Beaver}{HTML}{9A8873}
\definecolor{UltraViolet}{HTML}{52489C}



%=========================================
% Theorem Environment
%=========================================
\tcbuselibrary{listings, theorems, breakable, skins}

\newtcbtheorem[number within = subsection]{thm}{Theorem}%
{	colback=Buff!3, 
	colframe=Buff, 
	fonttitle=\bfseries, 
	breakable, 
	enhanced jigsaw, 
	halign=left
}{thm}

\newtcbtheorem[number within=subsection, use counter from=thm]{defn}{Definition}%
{  colback=cyan!1,
    colframe=cyan!50!black,
	fonttitle=\bfseries, breakable, 
	enhanced jigsaw, 
	halign=left
}{defn}

\newtcbtheorem[number within=subsection, use counter from=thm]{axm}{Axiom}%
{	colback=red!5, 
	colframe=Darkred, 
	fonttitle=\bfseries, 
	breakable, 
	enhanced jigsaw, 
	halign=left
}{axm}

\newtcbtheorem[number within=subsection, use counter from=thm]{prp}{Proposition}%
{	colback=LightBlue!3, 
	colframe=Glaucous, 
	fonttitle=\bfseries, 
	breakable, 
	enhanced jigsaw, 
	halign=left
}{prp}

\newtcbtheorem[number within=subsection, use counter from=thm]{lmm}{Lemma}%
{	colback=LightBlue!3, 
	colframe=LightBlue!60, 
	fonttitle=\bfseries, 
	breakable, 
	enhanced jigsaw, 
	halign=left
}{lmm}

\newtcbtheorem[number within=subsection, use counter from=thm]{crl}{Corollary}%
{	colback=LightBlue!3, 
	colframe=LightBlue!60, 
	fonttitle=\bfseries, 
	breakable, 
	enhanced jigsaw, 
	halign=left
}{crl}

\newtcbtheorem[number within=subsection, use counter from=thm]{eg}{Example}%
{	colback=Beaver!5, 
	colframe=Beaver, 
	fonttitle=\bfseries, 
	breakable, 
	enhanced jigsaw, 
	halign=left
}{eg}

\newtcbtheorem[number within=subsection, use counter from=thm]{ex}{Exercise}%
{	colback=Beaver!5, 
	colframe=Beaver, 
	fonttitle=\bfseries, 
	breakable, 
	enhanced jigsaw, 
	halign=left
}{ex}

\newtcbtheorem[number within=subsection, use counter from=thm]{alg}{Algorithm}%
{	colback=UltraViolet!5, 
	colframe=UltraViolet, 
	fonttitle=\bfseries, 
	breakable, 
	enhanced jigsaw, 
	halign=left
}{alg}




%=========================================
% Hyperlinks
%=========================================
\hypersetup{
    colorlinks=true, %set true if you want colored links
    linktoc=all,     %set to all if you want both sections and subsections linked
    linkcolor=DarkBlue,  %choose some color if you want links to stand out
}


\pagestyle{fancy}
\fancyhf{}
\rhead{Labix}
\lhead{Geometric Group Theory}
\rfoot{\thepage}

\title{Geometric Group Theory}

\author{Labix}

\date{\today}
\begin{document}
\maketitle
\begin{abstract}
Potentially good books: Humphreys, Erdmann and Wildson
\end{abstract}
\pagebreak
\tableofcontents

\pagebreak
\section{The Geometry of Presentations}
\subsection{Cayley Graphs}
\begin{defn}{The Cayley Graph of a Group}{} Let $G$ be a group. Let $S$ be a generating set of $G$. Define the Cayley graph $\Gamma=\Gamma(G,S)$ of $G$ with respect to $S$ to consist of the following data. 
\begin{itemize}
\item The vertices are given by $V(\Gamma)=G$
\item The edges are given by $E(\Gamma)=\{(g,gs)\;|\;g\in G,s\in S\}$
\end{itemize}
\end{defn}

Let $(V,E)$ be a graph. Recall that a graph automorphism consists of a bijective map of vertices and a bjective map of edges such that $$\{\phi(v),\phi(w)\}\in E$$ for all $\{v,w\}\in E$. They form a group by composition. 

\begin{lmm}{The Action Lemma}{} Let $G$ be a group. Let $S$ be a generating set of $G$. Then $G$ acts on the Cayley graph $\Gamma$ of $G$ with respect to $S$ via tha map $$\cdot:G\times\Gamma\to\Gamma$$ defined by $h\cdot g=hg$ and $h\cdot (g,gs)=(hg,hgs)$. 
\end{lmm}

\begin{lmm}{}{} Let $G$ be a group. Let $S$ be a generating set of $G$. Then $G$ acts on the Cayley graph faithfully. 
\end{lmm}

\begin{prp}{}{} Let $G$ be a group. Let $S$ be a generating set of $G$. Then the following are true regarding the Cayley graph $\Gamma(G,S)$ of $G$ with respect to $S$. 
\begin{itemize}
\item $\Gamma(G,S)$ has no embedded cycles. 
\item $\Gamma(G,S)$ is connected. 
\end{itemize}
\end{prp}

\begin{prp}{}{} Let $S$ be a set. Then $\Gamma(F_S,S)$ is a tree. 
\end{prp}

\begin{prp}{}{} Let $G$ be a group. Let $S$ be a generating set of $G$. Consider $\Gamma(G,S)$ as a CW complex. Then $\Gamma(F_S,S)$ is a universal cover of $\Gamma(G,S)$. 
\end{prp}

\subsection{Giving the Cayley Graph a Metric}
\begin{defn}{The Word Metric}{} Let $G$ be a group. Let $S$ be a generating set of $G$. Let $\Gamma$ be the Cayley complex of $G$ and $S$. Define the word metric to be the map $d_S:V(\Gamma)\times V(\Gamma)\to\N$ by $$d_S(g,h)=\min\{n\in\N\;|\;\gamma:[0,n]\to\Gamma\text{ is a path from }g\text{ to }h\}$$
\end{defn}

\begin{lmm}{}{} Let $G$ be a group. Let $S$ be a generating set of $G$. Let $\Gamma$ be the Cayley complex of $G$ and $S$. Then $d_\Gamma$ is a metric on $V(\Gamma)$. 
\end{lmm}

\begin{prp}{}{} Let $G$ be a group. Let $S$ be a generating set of $G$. Let $\Gamma$ be the Cayley complex of $G$ and $S$. Then for each $k\in G$, the induced map $\Gamma\to\Gamma$ from the action of $G$ on $\Gamma$ is an isometry. In other words, $$d_S(k\cdot g,k\cdot h)=d_S(g,h)$$ for all $g,h,k\in G$. 
\end{prp}

\begin{defn}{The Word Norm}{} Let $G$ be a group. Let $S$ be a generating set of $G$. Let $\Gamma$ be the Cayley complex of $G$ and $S$. Define the word norm of $g\in G$ to be $$\|g\|_S=d_S(1_G,g)$$
\end{defn}

\begin{lmm}{}{} Let $G$ be a group. Let $S$ be a generating set of $G$. Let $\Gamma$ be the Cayley complex of $G$ and $S$. Then the following are true. 
\begin{itemize}
\item $d_S(g,h)=\|g^{-1}h\|_S$ for all $g,h\in G$. 
\item $\|g^{-1}\|_S=\|g\|_S$ for all $g\in G$. 
\item $\|gh\|_S\leq\|g\|_S+\|h\|_S$ for all $g,h\in G$. 
\end{itemize}
\end{lmm}





\end{document}
