\documentclass[a4paper]{article}

\input{C:/Users/liula/Desktop/Latex/Headers V1.2.tex}

\pagestyle{fancy}
\fancyhf{}
\rhead{Labix}
\lhead{Geometric Group Theory}
\rfoot{\thepage}

\title{Geometric Group Theory}

\author{Labix}

\date{\today}
\begin{document}
\maketitle
\begin{abstract}
Potentially good books: Humphreys, Erdmann and Wildson
\end{abstract}
\pagebreak
\tableofcontents

\pagebreak
\section{The Geometry of Presentations}
\subsection{Cayley Graphs}
\begin{defn}{The Cayley Graph of a Group}{} Let $G$ be a group. Let $S$ be a generating set of $G$. Define the Cayley graph $\Gamma=\Gamma(G,S)$ of $G$ with respect to $S$ to consist of the following data. 
\begin{itemize}
\item The vertices are given by $V(\Gamma)=G$
\item The edges are given by $E(\Gamma)=\{(g,gs)\;|\;g\in G,s\in S\}$
\end{itemize}
\end{defn}

Let $(V,E)$ be a graph. Recall that a graph automorphism consists of a bijective map of vertices and a bjective map of edges such that $$\{\phi(v),\phi(w)\}\in E$$ for all $\{v,w\}\in E$. They form a group by composition. 

\begin{lmm}{The Action Lemma}{} Let $G$ be a group. Let $S$ be a generating set of $G$. Then $G$ acts on the Cayley graph $\Gamma$ of $G$ with respect to $S$ via tha map $$\cdot:G\times\Gamma\to\Gamma$$ defined by $h\cdot g=hg$ and $h\cdot (g,gs)=(hg,hgs)$. 
\end{lmm}

\begin{lmm}{}{} Let $G$ be a group. Let $S$ be a generating set of $G$. Then $G$ acts on the Cayley graph faithfully. 
\end{lmm}

\begin{prp}{}{} Let $G$ be a group. Let $S$ be a generating set of $G$. Then the following are true regarding the Cayley graph $\Gamma(G,S)$ of $G$ with respect to $S$. 
\begin{itemize}
\item $\Gamma(G,S)$ has no embedded cycles. 
\item $\Gamma(G,S)$ is connected. 
\end{itemize}
\end{prp}

\begin{prp}{}{} Let $S$ be a set. Then $\Gamma(F_S,S)$ is a tree. 
\end{prp}

\begin{prp}{}{} Let $G$ be a group. Let $S$ be a generating set of $G$. Consider $\Gamma(G,S)$ as a CW complex. Then $\Gamma(F_S,S)$ is a universal cover of $\Gamma(G,S)$. 
\end{prp}

\subsection{Giving the Cayley Graph a Metric}
\begin{defn}{The Word Metric}{} Let $G$ be a group. Let $S$ be a generating set of $G$. Let $\Gamma$ be the Cayley complex of $G$ and $S$. Define the word metric to be the map $d_S:V(\Gamma)\times V(\Gamma)\to\N$ by $$d_S(g,h)=\min\{n\in\N\;|\;\gamma:[0,n]\to\Gamma\text{ is a path from }g\text{ to }h\}$$
\end{defn}

\begin{lmm}{}{} Let $G$ be a group. Let $S$ be a generating set of $G$. Let $\Gamma$ be the Cayley complex of $G$ and $S$. Then $d_\Gamma$ is a metric on $V(\Gamma)$. 
\end{lmm}

\begin{prp}{}{} Let $G$ be a group. Let $S$ be a generating set of $G$. Let $\Gamma$ be the Cayley complex of $G$ and $S$. Then for each $k\in G$, the induced map $\Gamma\to\Gamma$ from the action of $G$ on $\Gamma$ is an isometry. In other words, $$d_S(k\cdot g,k\cdot h)=d_S(g,h)$$ for all $g,h,k\in G$. 
\end{prp}

\begin{defn}{The Word Norm}{} Let $G$ be a group. Let $S$ be a generating set of $G$. Let $\Gamma$ be the Cayley complex of $G$ and $S$. Define the word norm of $g\in G$ to be $$\|g\|_S=d_S(1_G,g)$$
\end{defn}

\begin{lmm}{}{} Let $G$ be a group. Let $S$ be a generating set of $G$. Let $\Gamma$ be the Cayley complex of $G$ and $S$. Then the following are true. 
\begin{itemize}
\item $d_S(g,h)=\|g^{-1}h\|_S$ for all $g,h\in G$. 
\item $\|g^{-1}\|_S=\|g\|_S$ for all $g\in G$. 
\item $\|gh\|_S\leq\|g\|_S+\|h\|_S$ for all $g,h\in G$. 
\end{itemize}
\end{lmm}





\end{document}
