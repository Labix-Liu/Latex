\documentclass[a4paper]{article}

%=========================================
% Packages
%=========================================
\usepackage{mathtools}
\usepackage{amsfonts}
\usepackage{amsmath}
\usepackage{amssymb}
\usepackage{amsthm}
\usepackage[a4paper, total={6in, 8in}, margin=1in]{geometry}
\usepackage[utf8]{inputenc}
\usepackage{fancyhdr}
\usepackage[utf8]{inputenc}
\usepackage{graphicx}
\usepackage{physics}
\usepackage[listings]{tcolorbox}
\usepackage{hyperref}
\usepackage{tikz-cd}
\usepackage{adjustbox}
\usepackage{enumitem}
\usepackage[font=small,labelfont=bf]{caption}
\usepackage{subcaption}
\usepackage{wrapfig}
\usepackage{makecell}



\raggedright

\usetikzlibrary{arrows.meta}

\DeclarePairedDelimiter\ceil{\lceil}{\rceil}
\DeclarePairedDelimiter\floor{\lfloor}{\rfloor}

%=========================================
% Fonts
%=========================================
\usepackage{tgpagella}
\usepackage[T1]{fontenc}


%=========================================
% Custom Math Operators
%=========================================
\DeclareMathOperator{\adj}{adj}
\DeclareMathOperator{\im}{im}
\DeclareMathOperator{\nullity}{nullity}
\DeclareMathOperator{\sign}{sign}
\DeclareMathOperator{\dom}{dom}
\DeclareMathOperator{\lcm}{lcm}
\DeclareMathOperator{\ran}{ran}
\DeclareMathOperator{\ext}{Ext}
\DeclareMathOperator{\dist}{dist}
\DeclareMathOperator{\diam}{diam}
\DeclareMathOperator{\aut}{Aut}
\DeclareMathOperator{\inn}{Inn}
\DeclareMathOperator{\syl}{Syl}
\DeclareMathOperator{\edo}{End}
\DeclareMathOperator{\cov}{Cov}
\DeclareMathOperator{\vari}{Var}
\DeclareMathOperator{\cha}{char}
\DeclareMathOperator{\Span}{span}
\DeclareMathOperator{\ord}{ord}
\DeclareMathOperator{\res}{res}
\DeclareMathOperator{\Hom}{Hom}
\DeclareMathOperator{\Mor}{Mor}
\DeclareMathOperator{\coker}{coker}
\DeclareMathOperator{\Obj}{Obj}
\DeclareMathOperator{\id}{id}
\DeclareMathOperator{\GL}{GL}
\DeclareMathOperator*{\colim}{colim}

%=========================================
% Custom Commands (Shortcuts)
%=========================================
\newcommand{\CP}{\mathbb{CP}}
\newcommand{\GG}{\mathbb{G}}
\newcommand{\F}{\mathbb{F}}
\newcommand{\N}{\mathbb{N}}
\newcommand{\Q}{\mathbb{Q}}
\newcommand{\R}{\mathbb{R}}
\newcommand{\C}{\mathbb{C}}
\newcommand{\E}{\mathbb{E}}
\newcommand{\Prj}{\mathbb{P}}
\newcommand{\RP}{\mathbb{RP}}
\newcommand{\T}{\mathbb{T}}
\newcommand{\Z}{\mathbb{Z}}
\newcommand{\A}{\mathbb{A}}
\renewcommand{\H}{\mathbb{H}}
\newcommand{\K}{\mathbb{K}}

\newcommand{\mA}{\mathcal{A}}
\newcommand{\mB}{\mathcal{B}}
\newcommand{\mC}{\mathcal{C}}
\newcommand{\mD}{\mathcal{D}}
\newcommand{\mE}{\mathcal{E}}
\newcommand{\mF}{\mathcal{F}}
\newcommand{\mG}{\mathcal{G}}
\newcommand{\mH}{\mathcal{H}}
\newcommand{\mI}{\mathcal{I}}
\newcommand{\mJ}{\mathcal{J}}
\newcommand{\mK}{\mathcal{K}}
\newcommand{\mL}{\mathcal{L}}
\newcommand{\mM}{\mathcal{M}}
\newcommand{\mO}{\mathcal{O}}
\newcommand{\mP}{\mathcal{P}}
\newcommand{\mS}{\mathcal{S}}
\newcommand{\mT}{\mathcal{T}}
\newcommand{\mV}{\mathcal{V}}
\newcommand{\mW}{\mathcal{W}}

%=========================================
% Colours!!!
%=========================================
\definecolor{LightBlue}{HTML}{2D64A6}
\definecolor{ForestGreen}{HTML}{4BA150}
\definecolor{DarkBlue}{HTML}{000080}
\definecolor{LightPurple}{HTML}{cc99ff}
\definecolor{LightOrange}{HTML}{ffc34d}
\definecolor{Buff}{HTML}{DDAE7E}
\definecolor{Sunset}{HTML}{F2C57C}
\definecolor{Wenge}{HTML}{584B53}
\definecolor{Coolgray}{HTML}{9098CB}
\definecolor{Lavender}{HTML}{D6E3F8}
\definecolor{Glaucous}{HTML}{828BC4}
\definecolor{Mauve}{HTML}{C7A8F0}
\definecolor{Darkred}{HTML}{880808}
\definecolor{Beaver}{HTML}{9A8873}
\definecolor{UltraViolet}{HTML}{52489C}



%=========================================
% Theorem Environment
%=========================================
\tcbuselibrary{listings, theorems, breakable, skins}

\newtcbtheorem[number within = subsection]{thm}{Theorem}%
{	colback=Buff!3, 
	colframe=Buff, 
	fonttitle=\bfseries, 
	breakable, 
	enhanced jigsaw, 
	halign=left
}{thm}

\newtcbtheorem[number within=subsection, use counter from=thm]{defn}{Definition}%
{  colback=cyan!1,
    colframe=cyan!50!black,
	fonttitle=\bfseries, breakable, 
	enhanced jigsaw, 
	halign=left
}{defn}

\newtcbtheorem[number within=subsection, use counter from=thm]{axm}{Axiom}%
{	colback=red!5, 
	colframe=Darkred, 
	fonttitle=\bfseries, 
	breakable, 
	enhanced jigsaw, 
	halign=left
}{axm}

\newtcbtheorem[number within=subsection, use counter from=thm]{prp}{Proposition}%
{	colback=LightBlue!3, 
	colframe=Glaucous, 
	fonttitle=\bfseries, 
	breakable, 
	enhanced jigsaw, 
	halign=left
}{prp}

\newtcbtheorem[number within=subsection, use counter from=thm]{lmm}{Lemma}%
{	colback=LightBlue!3, 
	colframe=LightBlue!60, 
	fonttitle=\bfseries, 
	breakable, 
	enhanced jigsaw, 
	halign=left
}{lmm}

\newtcbtheorem[number within=subsection, use counter from=thm]{crl}{Corollary}%
{	colback=LightBlue!3, 
	colframe=LightBlue!60, 
	fonttitle=\bfseries, 
	breakable, 
	enhanced jigsaw, 
	halign=left
}{crl}

\newtcbtheorem[number within=subsection, use counter from=thm]{eg}{Example}%
{	colback=Beaver!5, 
	colframe=Beaver, 
	fonttitle=\bfseries, 
	breakable, 
	enhanced jigsaw, 
	halign=left
}{eg}

\newtcbtheorem[number within=subsection, use counter from=thm]{ex}{Exercise}%
{	colback=Beaver!5, 
	colframe=Beaver, 
	fonttitle=\bfseries, 
	breakable, 
	enhanced jigsaw, 
	halign=left
}{ex}

\newtcbtheorem[number within=subsection, use counter from=thm]{alg}{Algorithm}%
{	colback=UltraViolet!5, 
	colframe=UltraViolet, 
	fonttitle=\bfseries, 
	breakable, 
	enhanced jigsaw, 
	halign=left
}{alg}




%=========================================
% Hyperlinks
%=========================================
\hypersetup{
    colorlinks=true, %set true if you want colored links
    linktoc=all,     %set to all if you want both sections and subsections linked
    linkcolor=DarkBlue,  %choose some color if you want links to stand out
}


\pagestyle{fancy}
\fancyhf{}
\rhead{Labix}
\lhead{Classical Homotopy Theory}
\rfoot{\thepage}

\title{Classical Homotopy Theory}

\author{Labix}

\date{\today}
\begin{document}
\maketitle
\begin{abstract}
\begin{itemize}
\item Notes on Algebraic Topology by Oscar Randal-Williams
\end{itemize}
\end{abstract}
\pagebreak
\tableofcontents

\pagebreak
\section{A Convenient Category of Spaces}
Reason: 
\begin{itemize}
\item Want $-\wedge-$ associative and unital and commutative (so that the category is symmetric monoidal)
\item Want adjunction $-\times X$ and $\Hom_{\mC}(X,-)$ (non pointed)
\item Want adjunction $-\wedge X$ and $\text{Map}_\ast(X,-)$ (pointed) (Intuitively, $X\wedge Y$ represents maps from $X\times Y$ that are base point preserving separately in each variable)
\end{itemize}


\subsection{Compactly Generated Spaces}
\begin{defn}{Compactly Generated Spaces}{} Let $X$ be a space. We say that $X$ is compactly generated ($k$-space) if for every set $A\subseteq X$, $A$ is open if and only if $A\cap K$ is open in $K$ for every compact subspace $K\subseteq X$. 
\end{defn}

\begin{defn}{Category of Compactly Generated Spaces}{} Define the category of compactly generated spaces $\bold{CG}$ to be the full subcategory of $\bold{Top}$ consisting of spaces that are compactly generated. In other words, $\bold{CG}$ consists of the following data: 
\begin{itemize}
\item $\text{Obj}(\bold{CG})$ consists of all spaces that are compactly generated. 
\item For $X,Y\in\text{Obj}(\bold{CG})$, the morphisms are $$\Hom_{\bold{CG}}(X,Y)=\Hom_{\bold{Top}}(X,Y)$$
\item Association is given by composition of functions. 
\end{itemize}
Define similarly the category of pointed compactly generated spaces $\bold{CG}_\ast$. 
\end{defn}

\begin{defn}{New $k$-space from Old}{} Let $X$ be a space. Define $k(X)$ to be the set $X$ together with the topology defined as follows: $A\subseteq X$ is open if and only if $A\cap K$ is open in $K$ for every compact subspace $K\subseteq X$. 
\end{defn}

\begin{lmm}{}{} Let $X$ be a space. Then $k(X)$ is a compactly generated space. 
\end{lmm}

Unfortunately $X\times Y$ may not be compactly generated even when $X$ and $Y$ are. But as it turns out, products do exists in $\bold{CG}$ and are given by $X\times_{\bold{CG}}Y=k(X\times_{\bold{Top}} Y)$. 

\begin{prp}{}{} Let $X,Y$ be compactly generated spaces. Then the categorical product of $X$ and $Y$ in the category of compactly generated spaces is given by $$X\times_{\bold{CG}}Y=k(X\times_{\bold{Top}} Y)$$
\end{prp}

\begin{prp}{}{} Every CW complex is compactly generated. 
\end{prp}

\begin{defn}{Category of Compactly Generated and Weakly Hausdorff Spaces}{} Define the category of compactly generated and weakly Hausdorff spaces $\bold{CGWH}$ to be the full subcategory of $\bold{Top}$ consisting of spaces that are compactly generated and weakly Hausdorff. In other words, $\bold{CGWH}$ consists of the following data: 
\begin{itemize}
\item $\text{Obj}(\bold{CGWH})$ consists of all spaces that are compactly generated and weakly Hausdorff. 
\item For $X,Y\in\text{Obj}(\bold{CGWH})$, the morphisms are $$\Hom_{\bold{CGWH}}(X,Y)=\Hom_{\bold{Top}}(X,Y)$$
\item Association is given by composition of functions. 
\end{itemize}
Define similarly the category of pointed compactly generated spaces $\bold{CGWH}_\ast$. 
\end{defn}

\begin{prp}{}{} A compactly generated space $X$ is weakly Hausdorff if and only if the diagonal subspace $\Delta=\{(x,x)\;|\;x\in X\}$ is closed in $X\times X$. 
\end{prp}

\begin{prp}{}{} Product of CGWH is CGWH
\end{prp}

CGWH is complete and cocomplete

\subsection{The Cartesian Product and the Mapping Space}
\begin{defn}{The Mapping Space}{} Let $X,Y\in\bold{CG}$. Define the mapping space of $X$ and $Y$ by $$\text{Map}(X,Y)=k(\Hom_{\bold{CG}}(X,Y))$$ where $\Hom_{\bold{CG}}(X,Y)$ is equipped with the compact open topology. If $(X,x_0)$ and $(Y,y_0)$ are pointed spaces, define the mapping space to be $$\text{Map}_\ast((X,x_0),(Y,y_0))=k(\Hom_{\bold{CG}}((X,x_0),(Y,y_0)))$$
\end{defn}

By restricting to also weakly Hausdorff spaces, we obtain an adjunction. 

\begin{thm}{}{} Let $X,Y,Z\in\bold{CGWH}$. Then the functors $-\times_{\bold{CGWH}}Y:\bold{CGWH}\to\bold{CGWH}$ and $\text{Map}(Y,-):\bold{CGWH}\to\bold{CGWH}$ are adjoint functors with the adjunction formula $$\Hom_{\bold{CGWH}}(X\times_{\bold{CGWH}}Y,Z)\cong\Hom_{\bold{CGWH}}(X,\text{Map}(Y,Z))$$ Moreover, by giving the Hom set the compact open topology and applying $k$, we obtain an isomorphism $$\text{Map}(X\times_{\bold{CGWH}}Y,Z)\cong\text{Map}(X,\text{Map}(Y,Z))$$
\end{thm}

\subsection{The Smash Product and the Pointed Mapping Space}
Aside from the adjunction between the product space and the mapping space, another major reason one considers compactly generated spaces is that the smash product gives another adjunction. 

\begin{defn}{The Smash Product}{} Let $(X,x_0)$ and $(Y,y_0)$ be pointed topological spaces. Define the smash product of the two pointed spaces to be the pointed space $$X\wedge Y=\frac{X\times Y}{X\vee Y}$$ together with the point $(x_0,y_0)$. 
\end{defn}

\begin{prp}{}{} Let $X,Y,Z$ be compactly generated spaces with a chosen base point. Then the following are true. 
\begin{itemize}
\item $(X\wedge Y)\wedge Z\cong X\wedge(Y\wedge Z)$
\item $X\wedge Y\cong Y\wedge X$
\end{itemize}
\end{prp}

\begin{thm}{}{} The category $\bold{CG}$ of compactly generated spaces is a symmetric monoidal category with operator the smash product $\wedge:\bold{CG}\times\bold{CG}\to\bold{CG}$ and the unit $S^0$. 
\end{thm}

Note that this is not true if we do not restrict the spaces to the category of compactly generated spaces. 

\begin{lmm}{}{} Let $X$ be a pointed space. Then the reduced suspension and the smash product with the circle $$\Sigma X\cong X\wedge S^1$$ are homeomorphic spaces. 
\end{lmm}

\begin{thm}{}{} Let $X,Y,Z$ be compactly generated with a chosen basepoint. Then the functors $-\wedge Y:\mK_\ast\to\mK_\ast$ and $\text{Map}_\ast(Y,-):\mK_\ast\to\mK_\ast$ are adjoint functors with the adjunction formula $$\Hom_{\mK_\ast}(X\wedge Y,Z)\cong\Hom_{\mK_\ast}(X,\text{Map}_\ast(Y,Z))$$ Moreover, by giving the Hom set the compact open topology and applying $k$, we obtain an isomorphism $$\text{Map}_\ast(X\wedge Y,Z)\cong\text{Map}_\ast(X,\text{Map}_\ast(Y,Z))$$
\end{thm}

By choosing $Y=I$ in the adjunction, we recover the usual suspension-loopspace adjunction in $\bold{Top}_\ast$. 

\begin{crl}{}{} Let $X$ be a compactly generated space with a chosen basepoint. Then there is a natural homeomorphism $$\text{Map}_\ast(\Sigma X,Y)\cong\text{Map}_\ast(X,\Omega Y)$$ given by adjunction of the functors $-\wedge S^1:\mK_\ast\to\mK_\ast$ and $\text{Map}_\ast(S^1,-):\mK_\ast\to\mK_\ast$. 
\end{crl}

\subsection{The Mapping Cylinder and the Mapping Path Space}
Equipped with the Cartesian closed structure in $\bold{CG}$ together with a canonical topology on the mapping space $Y^X$, we can now talk about the duality between the mapping cylinder and the mapping path space. 

\begin{defn}{Mapping Cylinder}{} Let $X,Y$ be spaces and let $f:X\to Y$ a map. Define the mapping cylinder of $f$ to be $$M_f=\frac{(X\times I)\amalg Y}{(x,0)\sim f(x)}=(X\times I)\amalg_fY$$ for $f:X\times\{1\}\cong X\to Y$ together with the quotient topology. It is the push forward of $f$ and the inclusion map $i_0:X\cong X\times\{0\}\hookrightarrow X\times I$. 
\end{defn}

\begin{lmm}{}{} Let $X,Y$ be spaces and let $f:X\to Y$ be a map. Then $Y$ is a deformation retract of $M_f$. 
\end{lmm}

\begin{defn}{The Mapping Path Space}{} Let $X,Y$ be spaces and let $f:X\to Y$ be a map. Define the mapping path space of $f$ to be $$P_f=\{(x,\gamma)\in X\times\text{Map}(I,Y)\;|\;\gamma(0)=f(x)\}$$ It is the pull back of $f$ and $\pi_0:\text{Map}(I,Y)\to Y$ given by $\pi_0(\gamma)=\gamma(0)$ in $\bold{CG}$. 
\end{defn}

\pagebreak
\section{(Co)Fibers of (Co)Fibrations}
\subsection{The Relative Point of View}
\begin{defn}{Fibers of a Map}{} Let $X,Y$ be spaces. Let $f:X\to Y$ be a map. Define the fiber of $f$ at $y\in Y$ to be $$\text{Fib}_y(f)=f^{-1}(y)$$
\end{defn}

\begin{defn}{Cofibers of a Map}{} Let $X,Y$ be spaces. Let $f:X\to Y$ be a map. Define the cofiber of $f$ to be $$\text{Cofib}(f)=\frac{Y}{f(X)}$$
\end{defn}

Up until this point, in algebraic topology we have asked questions relating to two spaces and tried to answer them. For instance, we can ask whether two spaces are homeomorphic, homotopy equivalent or weakly equivalent. We can also ask these questions in a relative setting, this involves considering maps of spaces as objects themselves, instead of just the spaces. 

\begin{defn}{Maps of Maps}{} Let $X,Y,A,B$ be spaces. Let $f:X\to Y$ and $g:A\to B$ be maps. A map from $f$ to $g$ is a pair of maps $(\alpha:X\to A,\beta:Y\to B)$ such that the following diagram commutes: \\~\\
\adjustbox{scale=1.0,center}{\begin{tikzcd}
	X & Y \\
	A & B
	\arrow["f", from=1-1, to=1-2]
	\arrow["\alpha"', from=1-1, to=2-1]
	\arrow["\beta", from=1-2, to=2-2]
	\arrow["g"', from=2-1, to=2-2]
\end{tikzcd}}\\~\\
\end{defn}

\begin{defn}{Homotopy from Maps to Maps}{} Let $X,Y,A,B$ be spaces. Let $f:X\to Y$ and $g:A\to B$ be maps. Let $(a,b)$ and $(c,d)$ be two maps from $f$ to $g$. We say that $(a,b)$ and $(c,d)$ are homotopic if there exists maps $H:X\times I\to Y$ and $K:A\times I\to B$ such that the following diagram \\~\\
\adjustbox{scale=1.0,center}{\begin{tikzcd}
	{X\times I} & {Y\times I} \\
	A & B
	\arrow["{f\times\text{id}_I}", from=1-1, to=1-2]
	\arrow["H"', from=1-1, to=2-1]
	\arrow["K", from=1-2, to=2-2]
	\arrow["g"', from=2-1, to=2-2]
\end{tikzcd}}\\~\\
commutes and the following are true. 
\begin{itemize}
\item $H$ is a homotopy from $a$ to $c$
\item $K$ is a homotopy from $b$ to $d$
\end{itemize}
\end{defn}

We now restrict to the point of view where maps are over a fixed space $Z$ (In other words we are considering objects of $\bold{Top}_Z$). A map of maps becomes the following data: Let $f:X\to Z$ and $g:Y\to Z$ be maps. A map from $f$ to $g$ is a map $h:X\to Y$ such that $g\circ h=f$. In other words the following diagram commutes: \\~\\
\adjustbox{scale=1.0,center}{\begin{tikzcd}
	X && Y \\
	& Z
	\arrow["h", from=1-1, to=1-3]
	\arrow["f"', from=1-1, to=2-2]
	\arrow["g", from=1-3, to=2-2]
\end{tikzcd}}\\~\\

A homotopy then becomes the following data: Let $f:X\to Z$ and $g:Y\to Z$ be two maps. Let $h,k:X\to Y$ be two maps from $f$ to $g$. We say that $h$ and $k$ are homotopic if there exists a homotopy $H:X\times I\to Y$ from $h$ to $k$ such that $H(-,t)$ for each $t$ is a map over $Z$. This means that the following diagram commutes: \\~\\
\adjustbox{scale=1.0,center}{\begin{tikzcd}
	{X\times I} && Y \\
	& Z
	\arrow["H", from=1-1, to=1-3]
	\arrow["{f\circ p_X}"', from=1-1, to=2-2]
	\arrow["g", from=1-3, to=2-2]
\end{tikzcd}}\\~\\

We can now discuss homotopy equivalences on these maps. 

\begin{defn}{Fiber Homotopy Equivalence}{} Let $X,Y,Z$ be spaces. Let $f:X\to Z$ and $g:Y\to Z$ be maps. We say that $f$ and $g$ are homotopy equivalent if there exists two maps $h:X\to Y$ and $k:Y\to X$ such that $k\circ h$ and $h\circ k$ are both homotopic to the identity over $Z$. 
\end{defn}

The reason that it is called a fiber homotopy equivalence is because it gives homotopy equivalences on fibers. 

\begin{prp}{}{} Let $X,Y,Z$ be spaces. Let $f:X\to B$ and $g:Y\to B$ be maps. If $f$ and $g$ are homotopy equivalent, then for any $b\in B$, the fibers $$\text{Fib}_b(f)\simeq\text{Fib}_b(g)$$ are homotopy equivalent. \tcbline
\begin{proof}
Suppose that $f$ and $g$ are homotopy equivalent via two maps $h:X\to Y$ and $k:Y\to X$. This means that there exists a homotopy $H:X\times I\to X$ such that $H(-,0)=h\circ k$ and $H(-,1)=\text{id}_X$. Similarly, there exists a homotopy $K:Y\times I\to Y$ such that $K(-,0)=k\circ h$ and $K(-,1)=\text{id}_Y$. Consider the map $h|_{f^{-1}(b)}:f^{-1}(b)\to g^{-1}(b)$ and similarly for $k|_{g^{-1}(b)}$. Define two maps $\overline{H}:H|_{f^{-1}(b)\times I}$ and $\overline{K}=K|_{g^{-1}(b)\times I}$. To show that they are homotopies, we just need to show that $\overline{H}\subseteq f^{-1}(b)$ and similarly for $\overline{K}$. Now by definition, each $H(-,t):X\to Y$ is such that $g(H(-,t))=f$. Choose $x\in f^{-1}(b)$. Then $$g(H(x,t))=f(x)=b$$ Hence the entire homotopy $\overline{H}$ stays in the fiber $f^{-1}(b)$. Hence $\overline{H}$ is a well defined homotopy on $f^{-1}(b)$. Similarly for $\overline{K}$. Hence the two fibers are homotopy equivalent. 
\end{proof}
\end{prp}

Unfortunately for most maps $f:X\to Y$, the fibers themselves are not homeomorphic, and not even homotopy equivalent. 

\begin{eg}{}{} The fibers of the projection map $S^1\to\R$ to the $x$-axis are not homotopy equivalent. \tcbline
\begin{proof}
It is clear that the fiber of $S^1\to\R$ is either empty, consist of one point, or of two points. Neither two of the three are homotopy equivalent. 
\end{proof}
\end{eg}

There are two ways to proceed from here. We first try to find a set of maps in which all fibers are homotopy equivalent. This is the content of this section. Otherwise, we try and define a new notion of fiber so that we obtain homotopy equivalence. This is the content of the next section. 

\subsection{(Co)Fibration Replacements}
\begin{prp}{}{} Let $X,Y\in\bold{CGWH}$ be spaces. Let $f:X\to Y$ be a map. Then the map $$q:P_f\to Y$$ given by $q(x,\gamma)=\gamma(1)$ is a fibration. \tcbline
\begin{proof}
Suppose that we are given a homotopy lifting problem: \\~\\
\adjustbox{scale=1.0,center}{\begin{tikzcd}
	{A\times\{0\}} && P_f \\
	\\
	{A\times I} && Y
	\arrow["H"', from=3-1, to=3-3]
	\arrow["{\widetilde{H}}"{description}, dashed, from=3-1, to=1-3]
	\arrow["q", from=1-3, to=3-3]
	\arrow["\iota"', hook, from=1-1, to=3-1]
	\arrow["g", from=1-1, to=1-3]
\end{tikzcd}}\\~\\
We write $g(a)=(g_1(a),g_2(a))$ for the components of $g$. Now recall that the definition of the mapping path space implies that $f(g_1(a))=g_2(a)(0)$. By commutativity of the diagram and definition of $q$ we also have $g_2(a)(1)=H(a,0)$. Define $\tilde{H}:A\times I\to P_f$ by the fomula $$\tilde{H}(a,t)=(g_1(a),h_2(a,t))$$ where $$h_2(a,t)(s)=\begin{cases}
g_2(a)(1+t)(s) & \text{ if }0\leq s\leq\frac{1}{1+t}\\
H(a,(1+t)s-1) & \text{ if }\frac{1}{1+t}\leq s\leq 1
\end{cases}$$ The definition of $h_2$ makes sense because $g_2(a)(1)=H(a,0)$. By the gluing lemma $h_2(a,t)$ is continuous. $h_2$ is also continuous in $a$ and $t$ because $g_2(a)$ is a path and $g_2$ is continuous and $H$ is continuous in both variables and the composite of continuous functions are continuous. Hence $\tilde{H}$ is continuous. Now $\tilde{H}(-,0)=(g_1(-),g_2(-))=g(-)$. Thus $\tilde{H}(-,0)$ is a lift of $g$. It remains to show that $\tilde{H}$ is a lift of $H$. We have that 
\begin{align*}
q(\tilde{H}(a,t))&=q(g_1(a),h_2(a,t))\\
&=h_2(a,t)(1)\\
&=H(a,t)
\end{align*}
and so we conclude. 
\end{proof}
\end{prp}

We can factorize any continuous map into a fibration and a homotopy equivalence through the mapping path space. Because we are working with the mapping path space here, we need to restrict our attention to compactly generated space. 

\begin{thm}{}{} Let $X,Y\in\bold{CGWH}$. Let $f:X\to Y$ be a map. Then there exists a homotopy equivalence $h:X\to P_f$ such that the following diagram commutes: \\~\\
\adjustbox{scale=1,center}{\begin{tikzcd}
	X && Y \\
	& {P_f}
	\arrow["f", from=1-1, to=1-3]
	\arrow["{\exists h}"', dashed, from=1-1, to=2-2]
	\arrow["q"', from=2-2, to=1-3]
\end{tikzcd}} \\~\\
where $q:P_f\to Y$ is the above defined map. \tcbline
\begin{proof}
Define the map $h:X\to P_f$ by $h(x)=(x,e_{f(x)})$. It is easy to see that $q\circ h=f$. I claim that the projection map $p_X:P_f\to X$ gives the homotopy inverse of $h$. Define a map $H:P_f\times I\to P_f$ by $$H(x,\gamma,t)=(x,\gamma_t)$$ where $\gamma_s$ is the path $s\mapsto\gamma(st)$. It is continuous since the composition of continuous functions are continuous and each component of $H$ is continuous. Also, we have that $h(p_X(x,\gamma))=h(x)=(x,e_{f(x)})$ and $H(x,\gamma,0)=(x,\gamma_0)=(x,e_{f(x)})$ so that $H(-,0)=h\circ p_X$. When $t=1$ we also have $$H(-,1)=(x,\gamma,1)=(x,\gamma_1)=(x,\gamma)=\text{id}_{P_f}(x,\gamma)$$ so that $H$ is a homotopy. 
\end{proof}
\end{thm}

\begin{prp}{}{} Let $X,Y\in\bold{CGWH}$ be spaces. Let $f:X\to Y$ be a map. Let $h:X\to P_f$ be the map that gives a factorization $q\circ h=f$. If $f$ is a fibration, then $h$ is a fiber homotopy equivalence. 
\end{prp}

Cofibrations and fibrations are dual in the following sense. Recall from section 1 that if $X$ and $Y$ are in $\bold{CGWH}$, then there is a bijection $$\Hom_{\bold{CGWH}}(X\times I,Y)\cong\Hom_{\bold{CGWH}}(X,\text{Map}(I,Y))$$ Now under this bijection, we can rewrite the diagram in the homotopy lifting property: \\~\\
\adjustbox{scale=1.0,center}{\begin{tikzcd}
	X & {E^I} \\
	X & {B^I}
	\arrow[from=1-1, to=1-2]
	\arrow["{\text{id}_X}"', from=1-1, to=2-1]
	\arrow["{p_\ast}", from=1-2, to=2-2]
	\arrow["{\exists\tilde{H}}", dashed, from=2-1, to=1-2]
	\arrow["H"', from=2-1, to=2-2]
\end{tikzcd}}\\~\\

\begin{prp}{}{} Let $A,X\in\bold{CGWH}$ be spaces. Let $f:A\to X$ be a map. Then the map $$q:A\to M_f$$ given by $q(a)=[a,0]$ is a cofibration. \tcbline
\begin{proof}
Suppose that we are given a homotopy lifting problem: \\~\\
\adjustbox{scale=1.0,center}{\begin{tikzcd}
	{A\cong A\times\{0\}} & {A\times I} \\
	{X\cong X\times\{0\}} & {X\times I} \\
	&& Y
	\arrow["{\iota_0}", from=1-1, to=1-2]
	\arrow["i"', from=1-1, to=2-1]
	\arrow["{i\times\text{id}_I}", from=1-2, to=2-2]
	\arrow["H", bend left = 30, from=1-2, to=3-3]
	\arrow["{\iota_0}"', from=2-1, to=2-2]
	\arrow["f", bend right = 20, from=2-1, to=3-3]
	\arrow["{\tilde{H}}", dashed, from=2-2, to=3-3]
\end{tikzcd}}\\~\\
\end{proof}
\end{prp}

Dual to the factorization through the mapping path space, we can factorize a map into a homotopy equivalence and a cofibration through the mapping cylinder $$M_f=\frac{(X\times I)\amalg Y}{(x,0)\sim f(x)}=(X\times I)\amalg_fY$$

\begin{thm}{}{} Let $f:A\to X$ be a map. Then the inclusion map $i:A\to M_f$ defined by $i(a)=[a,0]$ is a cofibration. Moreover, there exists a homotopy equivalence $h:M_f\to X$ such that the following diagram commutes: \\~\\
\adjustbox{scale=1,center}{\begin{tikzcd}
	A && X \\
	& {M_f}
	\arrow["f", from=1-1, to=1-3]
	\arrow["i"', from=1-1, to=2-2]
	\arrow["{\exists h}"', dashed, from=2-2, to=1-3]
\end{tikzcd}}
\end{thm}

\subsection{Long Exact Sequences from (Co)Fibrations}
\begin{thm}{Homotopy Long Exact Sequence in Fibration}{} Let $p:E\to B$ be a fibration over a path connected space $B$ with fiber $F$. Let $\iota:F\hookrightarrow E$ be the inclusion of the fiber. Then there is a long exact sequence in homotopy groups: \\~\\
\adjustbox{scale=0.75,center}{\begin{tikzcd}
	\cdots & {\pi_{n+1}(B,b_0)} & {\pi_n(F,e_0)} & {\pi_n(E,e_0)} & {\pi_n(B,b_0)} & {\pi_{n-1}(F,e_0)} & \cdots & {\pi_1(E,e_0)} & {\pi_1(B,b_0)}
	\arrow[from=1-1, to=1-2]
	\arrow["\partial", from=1-2, to=1-3]
	\arrow["{\iota_\ast}", from=1-3, to=1-4]
	\arrow["{p_\ast}", from=1-4, to=1-5]
	\arrow["\partial", from=1-5, to=1-6]
	\arrow[from=1-6, to=1-7]
	\arrow[from=1-7, to=1-8]
	\arrow["{p_\ast}", from=1-8, to=1-9]
\end{tikzcd}}\\~\\
for $e_0\in E$ and $b_0=p(e_0)$. Moreover, $p_\ast$ is an isomorphism. 
\end{thm}

\begin{thm}{Homology Long Exact Sequence in Cofibration}{} Let $p:X\to Y$ be a cofibration with cofiber $C=\frac{Y}{p(X)}$. Let $\text{proj}:Y\to C$ be the projection map. Then there is a long exact sequence in homology groups: \\~\\
\adjustbox{scale=0.85,center}{\begin{tikzcd}
	\cdots & {\widetilde{H}_{n+1}(C)} & {\widetilde{H}_n(X)} & {\widetilde{H}_n(Y)} & {\widetilde{H}_n(C)} & {\widetilde{H}_{n-1}(X} & \cdots & {\widetilde{H}_0(Y)} & {\widetilde{H}_0(B,b_0)}
	\arrow[from=1-1, to=1-2]
	\arrow["\partial", from=1-2, to=1-3]
	\arrow["{f_\ast}", from=1-3, to=1-4]
	\arrow["{\text{proj}_\ast}", from=1-4, to=1-5]
	\arrow["\partial", from=1-5, to=1-6]
	\arrow[from=1-6, to=1-7]
	\arrow[from=1-7, to=1-8]
	\arrow["{\text{proj}_\ast}", from=1-8, to=1-9]
\end{tikzcd}}\\~\\
\end{thm}

\subsection{(Co)Fibers of a (Co)Fibration are Homotopic}
The following definition is a supporting notion for our proof that fibers of a fibration are homotopy equivalent. 

\begin{defn}{Induced Map of Fibers}{} Let $p:E\to B$. Let $\gamma:I\to B$ be a path from $b_1$ to $b_2$. Define the induced map of fibers of $\gamma$ as follows: The map $H:E_{b_1}\times I\to B$ defined by $H(x,t)=\gamma(t)$ is a homotopy. Using the HLP of $p$, we obtain a lift: \\~\\
\adjustbox{scale=1,center}{\begin{tikzcd}
	{E_{b_1}\times\{0\}} & E \\
	{E_{b_1}\times I} & B
	\arrow["{\widetilde{H(-,0)}}", hook, from=1-1, to=1-2]
	\arrow[hook, from=1-1, to=2-1]
	\arrow["p", from=1-2, to=2-2]
	\arrow["{\widetilde{H}}"{description}, dashed, from=2-1, to=1-2]
	\arrow["H"', from=2-1, to=2-2]
\end{tikzcd}} \\~\\
Since $p\circ\widetilde{H}(x,t)=\gamma(t)$, we have that $\widetilde{H}(x,1)\in E_{b_2}$. The induced map of fibers is then the map $$L_\gamma:E_{b_1}\to E_{b_2}$$ defined by $L_\gamma=\widetilde{H(-,1)}$
\end{defn}

\begin{lmm}{}{} Let $p:E\to B$ be a fibration. Let $\gamma:I\to B$ be a path from $b_1$ to $b_2$. Then the following are true regarding $L_\gamma$. 
\begin{itemize}
\item If $\gamma\simeq\gamma'$ relative to boundary, then $L_\gamma\simeq L_{\gamma'}$.
\item If $\gamma:I\to B$ and $\gamma':I\to B$ are two composable paths, there is a homotopy equivalence $L_{\gamma\cdot\gamma'}\simeq L_{\gamma'}\circ L_\gamma$
\end{itemize} \tcbline
\begin{proof}
\begin{itemize}
\item Let $F:I\times I\to B$ be a homotopy equivalence from $\gamma$ to $\gamma'$. Now consider the map $G:E_{b_1}\times I\times I\to B$ defined by $G(x,s,t)=F(s,t)$. Notice that $G(x,s,0)=F(s,0)=\gamma(s)$ and $G(x,s,1)=F(s,1)=\gamma'(s)$. Thus, we proceed as above by lifting $G(x,s,0)$ and $G(x,s,1)$ to obtain respectively $\widetilde{G(x,s,0)}$ and $\widetilde{G(x,s,1)}$ for which $\widetilde{G(x,1,0)}=L_\gamma$ and $\widetilde{G(x,1,1)}=L_{\gamma'}$. Now define $K:E_{b_1}\times I\times\partial I\to E$ by $$K(x,s,t)=\begin{cases}
\widetilde{G(x,s,1)} & \text{ if } t=0\\
G(x,s,1) & \text{ if } t=1
\end{cases}$$ We now obtain a homotopy called $\widetilde{G}:E_{b_1}\times I\times I\to E$ by the homotopy lifting property: \\~\\
\adjustbox{scale=1,center}{\begin{tikzcd}
	{X\times I\times\partial I} & E \\
	{X\times I\times I} & B
	\arrow["K", from=1-1, to=1-2]
	\arrow[hook, from=1-1, to=2-1]
	\arrow["p", from=1-2, to=2-2]
	\arrow["\widetilde{G}"{description}, dashed, from=2-1, to=1-2]
	\arrow["G"', from=2-1, to=2-2]
\end{tikzcd}} \\~\\
Now $\tilde{G}(-,1,-):E_b\times I\to E$ is then a homotopy equivalence from $\widetilde{G}(x,1,0)=L_\gamma$ to $\widetilde{G}(x,1,1)=L_{\gamma'}$. 
\item We can repeat the above construction for $\gamma$ and $\gamma'$ to obtain homotopies $G:E_{b_1}\times I\to E$ and $G':E_{b_1}\times I\to E$ such that when $t=1$ we recover $\tilde{\gamma}$, $\tilde{\gamma'}$ and $\tilde{\gamma\cdot\gamma'}$ respectively. Now the composition of $G$ and $G'$ by traversing along $t\in I$ with twice the speed gives precisely a lift of $\gamma\cdot\gamma'$ (one can check the boundary conditions). Thus $L_{\gamma\cdot\gamma'}$ obtained in this manner coincides up to homotopy equivalence to $L_{\gamma'}\circ L_\gamma$ by invoking part a). 
\end{itemize}
\end{proof}
\end{lmm}

\begin{thm}{}{} Let $p:E\to B$ be a fibration. Let $b_1$ and $b_2$ lie in the same path component of $B$. Then there is a homotopy equivalence $$E_{b_1}\simeq E_{b_2}$$ given by the lift of any path $\gamma:I\to B$ from $b_1$ to $b_2$. \tcbline
\begin{proof}
Let $\gamma:I\to B$ be a path from $b_1$ to $b_2$. From the above, it follows that $L_{\overline{\gamma}}\circ L_\gamma\simeq\text{id}_{E_b}$ for any loop $\gamma:I\to B$ with basepoint $b$. We conclude that $L_\gamma$ is a homotopy equivalence and so the fibers of $p:E\to B$ are homotopy equivalent. 
\end{proof}
\end{thm}

\pagebreak
\section{Homotopy Fibers and Homotopy Cofibers}
\subsection{Basic Definitions}
\begin{defn}{Homotopy Fibers and Cofibers}{} Let $f:X\to Y$ be a map. Define the homotopy fiber of $f$ at $y\in Y$ to be $$\text{hofiber}_y(f)=\{(x,\phi)\in X\times\text{Map}(I,Y)\;|\;f(x)=\phi(0), \phi(1)=y\}=\text{Fib}_y(P_f\to Y)$$ Define the homotopy cofiber of $f$ to be $$\text{hocofiber}=\frac{(X\times I)\amalg Y}{(x,1)\sim f(x),(x,0)\sim(x',0)}=\text{Cofib}(X\to M_f)=C_f$$
\end{defn}

TBA: hofiber = pullback $P_f\to Y\leftarrow\ast$ (time $t=1$ and $\ast\mapsto y$). \\~\\

Since the map $P_f\to Y$ is a fibration, the fibers of $P_f\to Y$, and hence the homotopy fibers of $f$ are all homotopy equivalent. 

\begin{prp}{}{} Let $p:E\to B$ be a fibration. Then the there is a homotopy equivalence $$\text{Fib}_b(f)\simeq\text{Hofib}_b(f)$$ for each $b\in B$, given by the inclusion map $x\mapsto(x,e_x)$. \tcbline
\begin{proof}
Consider the following diagram \\~\\
\adjustbox{scale=1,center}{\begin{tikzcd}
	{\text{Fib}_b(f)} & {\text{Hofib}_b(f)} \\
	X & {P_f} \\
	Y & Y
	\arrow["\simeq", from=1-1, to=1-2]
	\arrow[hook, from=1-1, to=2-1]
	\arrow[hook, from=1-2, to=2-2]
	\arrow["{h, \simeq}", from=2-1, to=2-2]
	\arrow["f"', from=2-1, to=3-1]
	\arrow["q", from=2-2, to=3-2]
	\arrow["{\text{id}_Y}"', from=3-1, to=3-2]
\end{tikzcd}} \\~\\
and apply 2.4.6 to conclude. 
\end{proof}
\end{prp}

\begin{prp}{}{} Let $X,Y$ be spaces. Let $f,g:X\to Y$ be maps. If $f,g$ are homotopic, then for all $y\in Y$, there is a homotopy equivalence $$\text{Hofib}_y(f)\simeq\text{Hofib}_y(g)$$ induced by the homotopy from $f$ to $g$. 
\end{prp}

\begin{eg}{}{} Let $X,Y$ be spaces. Let $f:X\to Y$ be a map. 
\begin{itemize}
\item The homotopy fiber of $\{y\}\hookrightarrow Y$ is given by $\text{Hofib}_y(\{y\}\hookrightarrow Y)\cong\Omega Y$
\item If $f$ is null-homotopic, then the homotopy fiber of $f$ is given by $\text{Hofib}_y(f)\cong X\times\Omega Y$
\end{itemize}
\end{eg}

\subsection{The Fiber and Cofiber Sequences}
\begin{defn}{Path Spaces}{} Let $(X,x_0)$ be a pointed space. Define the path space of $(X,x_0)$ to be $$PX=\{\phi:(I,0)\to(X,x_0)\;|\;\phi(0)=x_0\}=\text{Map}_\ast((I,0),(X,x_0))$$ together with the topology of the mapping space. 
\end{defn}

\begin{thm}{}{} Let $X$ be a space. Then the following are true. 
\begin{itemize}
\item The map $\pi:PX\to X$ defined by $\pi(\phi)=\phi(1)$ is a fibration with fiber $\Omega X$
\item The map $\pi:X^I\to X$ defined by $\pi(\phi)=\phi(1)$ is a fibration with fiber homeomorphic to $PX$. 
\end{itemize}
\end{thm}

We now write a fibration as a sequence $F\to E\to B$ for $F$ the fiber of the fibration $p:E\to B$. This compact notation allows the following theorem to be formulated nicely. 

\begin{thm}{}{} Let $f:X\to Y$ be a fibration with homotopy fiber $F_f$. Let $\iota:\Omega Y\to F_f$ be the inclusion map and $\pi:F_f\to X$ the projection map. Then up to homotopy equivalence of spaces, there is a sequence \\~\\
\adjustbox{scale=1.0,center}{\begin{tikzcd}
	\cdots & {\Omega^2 X} & {\Omega^2Y} & {\Omega F_f} & {\Omega X} & {\Omega_Y} & {F_f} & X & Y
	\arrow[from=1-1, to=1-2]
	\arrow["{\Omega^2 f}", from=1-2, to=1-3]
	\arrow["{-\Omega\iota}", from=1-3, to=1-4]
	\arrow["{-\Omega\pi}", from=1-4, to=1-5]
	\arrow["{-\Omega f}", from=1-5, to=1-6]
	\arrow["\iota", from=1-6, to=1-7]
	\arrow["\pi", from=1-7, to=1-8]
	\arrow["f", from=1-8, to=1-9]
\end{tikzcd}}\\~\\
where any two consecutive maps form a fibration. Moreover, $-\Omega f:\Omega X\to\Omega Y$ is defined as $$(-\Omega f)(\zeta)(t)=(f\circ\zeta)(1-t)$$ for $\zeta\in\Omega X$. 
\end{thm}

There is then the dual notion of loop spaces and the corresponding sequence. Write a cofibration $f:A\to X$ with homotopy cofiber $B$ as $B\to A\to X$. 

\begin{thm}{}{} Let $f:X\to Y$ be a cofibration with homotopy cofiber $C_f$. Let $i:Y\to C_f$ be the inclusion map and $\pi:C_f\to C_f/Y\cong\Sigma X$ be the projection map. Then up to homotopy equivalence of spaces, there is a sequence \\~\\
\adjustbox{scale=1.0,center}{\begin{tikzcd}
	X & Y & {C_f} & {\Sigma X} & {\Sigma Y} & {\Sigma C_f} & {\Sigma^2X} & {\Sigma^2Y} & \cdots
	\arrow["f", from=1-1, to=1-2]
	\arrow["i", from=1-2, to=1-3]
	\arrow["\pi", from=1-3, to=1-4]
	\arrow["{-\Sigma f}", from=1-4, to=1-5]
	\arrow["{-\Sigma i}", from=1-5, to=1-6]
	\arrow["{-\Sigma\pi}", from=1-6, to=1-7]
	\arrow["{\Sigma^2 f}", from=1-7, to=1-8]
	\arrow[from=1-8, to=1-9]
\end{tikzcd}}\\~\\
where any two consecutive maps form a cofibration. Moreover, $-\Sigma f:\Sigma X\to\Sigma Y$ is defined by $$(-\Sigma f)(x\wedge t)=f(x)\wedge(1-t)$$
\end{thm}

\subsection{n-Connected Maps}
\begin{defn}{n-Connected Maps}{} Let $X,Y$ be spaces. Let $f:X\to Y$ be a map. We say that $f$ is $n$-connected if the induced map $$\pi_k(f):\pi_k(X)\to\pi_k(Y)$$ is an isomorphism for $0\leq k<n$ and a surjection for $k=n$. 
\end{defn}

We can rephrase some of the corner stone theorems of homotopy theory using $n$-connected maps. 
\begin{itemize}
\item The homotopy excision theorem can be rephrased into the following. For $X$ a CW-complex and $A,B$ sub complexes of $X$ such that $X=A\cup B$ and $A\cap B\neq\emptyset$. If $(A,A\cap B)$ is $m$-connected and $(B,A\cap B)$ is $n$-connected for $m,n\geq 0$, then the inclusion $$\iota:(A,A\cap B)\to(X,B)$$ is $(m+n)$-connected. 
\item The Freudenthal suspension theorem says that if $X$ is an $n$-connected CW complex, then the map $$\Omega\Sigma:X\to\Omega(\Sigma(X))$$ is a $(2n+1)$-connected map. 
\end{itemize}

\begin{prp}{}{} Let $X,Y$ be spaces. Let $f:X\to Y$ be a map. Then $f$ is $k$-connected if and only if $\text{Hofib}_y(f)$ is $(k-1)$-connected for all $y\in Y$. 
\end{prp}

\pagebreak
\section{Homotopy Pullbacks and Pushouts}
Homotopy pullbacks and pushouts are a special case of homotopy limits and colimits. It would be fruitful for us to first consider this case also because of how it is related to maps of spaces and (co)fibrations. 

\subsection{The Standard Homotopy Pullback and Pushout}
Consider the following diagram: \\~\\
\adjustbox{scale=1.0,center}{\begin{tikzcd}
	X & Z & Y \\
	{X'} & {Z'} & {Y'}
	\arrow["f", from=1-1, to=1-2]
	\arrow["{e_X}"', from=1-1, to=2-1]
	\arrow["{e_Z}"', from=1-2, to=2-2]
	\arrow["g"', from=1-3, to=1-2]
	\arrow["{e_Y}", from=1-3, to=2-3]
	\arrow["{f'}"', from=2-1, to=2-2]
	\arrow["{g'}", from=2-3, to=2-2]
\end{tikzcd}}\\~\\
We can form pullbacks of the upper and lower horizontals and obtain an induced map. However, when $e_X,e_Y,e_Z$ are homotopy equivalences, the induced map is not a homotopy equivalence. We remedy this by introducing a homotopic notion of pullbacks. 

\begin{defn}{The Standard Homotopy Pullback}{} Let $X,Y,Z\in\bold{CGWH}$ be spaces. Let $\mS$ denote the following diagram \\~\\
\adjustbox{scale=1.0,center}{\begin{tikzcd}
	X & Z & Y
	\arrow["f", from=1-1, to=1-2]
	\arrow["g"', from=1-3, to=1-2]
\end{tikzcd}}\\~\\
in $\bold{CGWH}$. Define the standard homotopy pullback $\text{holim}(X\overset{f}{\rightarrow}Z\overset{g}{\leftarrow}Y)$ of the diagram to be the subspace of $X\times\text{Map}(I,Z)\times Y$ consisting of $$\{(x,\alpha,y)\in X\times\text{Map}(I,Z)\times Y\;|\;\alpha(0)=f(x),\alpha(1)=g(y)\}$$
\end{defn}

The idea is that normally in pullbacks, we require that under $f$ and $g$ the elements of the pullback must arrive at the same point in $Z$. But here we relax the requirement by simply allowing elements of the homotopy pullback to arrive at the same path component of $Z$ (so up to the existence of an homotopy of the two points in $Z$). \\

Recall that the standard pullback need not be defined explicitly with a formula, but instead via a universal property guaranteeing that the pullback is unique up to unique homeomorphism. The homotopy pullback somewhat satisfies a homotopic version of the universal property. 

\begin{prp}{}{} Let the following be a diagram in $\bold{CGWH}$: \\~\\
\adjustbox{scale=1.0,center}{\begin{tikzcd}
	X & Z & Y
	\arrow["f", from=1-1, to=1-2]
	\arrow["g"', from=1-3, to=1-2]
\end{tikzcd}}\\~\\
Then the following diagram commutes up to homotopy: \\~\\
\adjustbox{scale=1.0,center}{\begin{tikzcd}
	{\text{holim}(X\overset{f}{\rightarrow}Z\overset{g}{\leftarrow}Y)} & X \\
	Y & Z
	\arrow["{\text{pr}}", from=1-1, to=1-2]
	\arrow["{\text{pr}}"', from=1-1, to=2-1]
	\arrow["f", from=1-2, to=2-2]
	\arrow["g"', from=2-1, to=2-2]
\end{tikzcd}}\\~\\
Moreover, for any $W\in\bold{CGWH}$ and any maps $a:W\to X$ and $b:W\to Y$ such that $f\circ a\simeq g\circ b$, there exists a map $u:W\to\text{holim}(X\overset{f}{\rightarrow}Z\overset{g}{\leftarrow}Y)$ such that the following diagram commutes up to homotopy: \\~\\
\adjustbox{scale=1.0,center}{\begin{tikzcd}
	A \\
	\\
	& \text{holim}(X\overset{f}{\rightarrow}Z\overset{g}{\leftarrow}Y) & X \\
	& Y & Z
	\arrow["{\exists u}"{description}, dashed, from=1-1, to=3-2]
	\arrow["a",bend left = 20, from=1-1, to=3-3]
	\arrow["b"', bend right = 30, from=1-1, to=4-2]
	\arrow["\text{pr}", from=3-2, to=3-3]
	\arrow["\text{pr}"', from=3-2, to=4-2]
	\arrow["f", from=3-3, to=4-3]
	\arrow["g"', from=4-2, to=4-3]
\end{tikzcd}}\\~\\
\end{prp}

The above proposition implies the existence of a map $$\lim(X\overset{f}{\rightarrow}Z\overset{g}{\leftarrow}Y)\to\text{holim}(X\overset{f}{\rightarrow}Z\overset{g}{\leftarrow}Y)$$ but it is not unique up to some commutativity restriction. However one such map is dubbed canonical. 

\begin{defn}{The Canonical Map of Homotopy Pullbacks}{} Let $X,Y,Z\in\bold{CGWH}$ be spaces such that \\~\\
\adjustbox{scale=1.0,center}{\begin{tikzcd}
	X & Z & Y
	\arrow["f", from=1-1, to=1-2]
	\arrow["g"', from=1-3, to=1-2]
\end{tikzcd}}\\~\\
is a diagram in $\bold{CGWH}$. Define the canonical map of the homotopy pullback of the diagram to be the map $$c:\lim(X\overset{f}{\rightarrow}Z\overset{g}{\leftarrow}Y)\to\text{holim}(X\overset{f}{\rightarrow}Z\overset{g}{\leftarrow}Y)$$ defined by $(x,y)\mapsto(x,c_{f(x)=g(y)},y)$. 
\end{defn}

Recall that we motivated the definition of a homotopy pullback from the fact that pullbacks does not work well with homotopy. We can now show that homotopy pullbacks remedy the issue. 

\begin{thm}{The Matching Lemma}{} Suppose that we have a commutative diagram of spaces \\~\\
\adjustbox{scale=1.0,center}{\begin{tikzcd}
	X & Z & Y \\
	{X'} & {Z'} & {Y'}
	\arrow["f", from=1-1, to=1-2]
	\arrow["{e_X}"', from=1-1, to=2-1]
	\arrow["{e_Z}"', from=1-2, to=2-2]
	\arrow["g"', from=1-3, to=1-2]
	\arrow["{e_Y}", from=1-3, to=2-3]
	\arrow["{f'}"', from=2-1, to=2-2]
	\arrow["{g'}", from=2-3, to=2-2]
\end{tikzcd}}\\~\\
in $\bold{CGWH}$. Define the map $$\phi_{X,Z,Y}^{X',Z',Y'}:\text{holim}(X\overset{f}{\rightarrow}Z\overset{g}{\leftarrow}Y)\to\text{holim}(X'\overset{f'}{\rightarrow}Z'\overset{g'}{\leftarrow}Y')$$ by the formula $(x,\gamma,y)\mapsto(e_X(x),e_Z\circ\gamma,e_Y(y))$. Then the following are true. 
\begin{itemize}
\item If each $e_X,e_Y,e_Z$ are homotopy equivalences, then $\phi$ is a homotopy equivalence. 
\item If each $e_X,e_Y,e_Z$ are weak equivalences, then $\phi$ is a weak equivalence. 
\end{itemize} \tcbline
\begin{proof}
We first prove the case for homotopy equivalence. Consider the following commutative diagram: \\~\\
\adjustbox{scale=1.0,center}{\begin{tikzcd}
	X & Z & Y \\
	X & {Z'} & Y \\
	{X'} & {Z'} & {Y'}
	\arrow["f", from=1-1, to=1-2]
	\arrow["{\text{id}_X}"', from=1-1, to=2-1]
	\arrow["{e_Z}", from=1-2, to=2-2]
	\arrow["g"', from=1-3, to=1-2]
	\arrow["{\text{id}_Y}", from=1-3, to=2-3]
	\arrow["{e_Z\circ f}", from=2-1, to=2-2]
	\arrow["{e_X}"', from=2-1, to=3-1]
	\arrow["{\text{id}_{Z'}}", from=2-2, to=3-2]
	\arrow["{e_Z\circ g}"', from=2-3, to=2-2]
	\arrow["{e_Y}", from=2-3, to=3-3]
	\arrow["{f'}"', from=3-1, to=3-2]
	\arrow["{g'}", from=3-3, to=3-2]
\end{tikzcd}}\\~\\
We prove that the homotopy pullback of the first row is homotopy equivalent to that of the second, and we prove that the homotopy pullback of the second row is homotopy equivalent to that of the third. \\~\\

Since $e_Z$ is a homotopy equivalence, we can find a homotopy inverse $k$ for $e_Z$ and a homotopy $H:Z\times I\to Z$ such that $H(-,0)=\text{id}_Z$ and $H(-,1)=k\circ e_Z$. Define a map $$\rho:\text{holim}(X\overset{f}{\rightarrow}Z'\overset{g}{\leftarrow}Y)\to\text{holim}(X\overset{e_Z\circ f}{\rightarrow}Z\overset{e_Z\circ g}{\leftarrow}Y)$$ by the formula $$(x,\gamma',y)\mapsto(x,H(f(x),-)\ast k(\gamma'(-))\ast\overline{H(g(y),-)}:I\to Z,y)$$ where $\ast$ denotes concatenation of paths. The path concatenation is well defined because we have that $H(f(x),1)=(k\circ e_Z\circ f)(x)=(k\circ\gamma')(0)$ and $k(\gamma'(1))=k(e_Z(g(y)))=H(g(y),1)$. This is well defined on the homotopy pullback because we have that 
\begin{itemize}
\item $H(f(x),-)\ast k(\gamma'(-))\ast\overline{H(g(y),-)}(0)=H(f(x),0)=\text{id}_Z(f(x))=f(x)$
\item $H(f(x),-)\ast k(\gamma'(-))\ast\overline{H(g(y),-)}(1)=H(g(y),0)=\text{id}_Z(g(y))=g(y)$
\end{itemize}
I claim that this map is inverse to the map $\phi=\phi_{X,Y,Z}^{X,Z',Y}$ where we take $e_X=\text{id}_X$ and $e_Y=\text{id}_Y$. We have that 
\begin{align*}
\rho(\phi(x,\gamma,y))&=\rho(x,e_Z\circ\gamma,y)\\
&=(x,H(f(x),-)\ast k(e_Z(\gamma(-))\ast\overline{H(g(y),-)},y)
\end{align*}
Now I claim that the middle path is homotopic to $\gamma$. For the first part, the path $H(f(x),t):I\to Z$ can be contracted to $H(f(x),0)=f(x)=\gamma(0)$ so you can homotope the traversal along $H(f(x),-)$ to the single point $f(x)=\gamma(0)$. For the third part, this is similar so we can homotope the traversal of $\overline{H(g(y),-)}$ to the single point $g(y)=\gamma(1)$. The middle part of the path is homotopic to $\gamma$ because $k\circ e_Z$ is homotopic to $\text{id}_Z$. Thus we conclude. 
\end{proof}
\end{thm}

Dually, we define the notion of standard homotopy pushouts. 

\begin{defn}{The Standard Homotopy Pushout}{} Let $X,Y,Z\in\bold{CGWH}$ be spaces. Let $\mS$ denote the following diagram \\~\\
\adjustbox{scale=1.0,center}{\begin{tikzcd}
	X & Z & Y
	\arrow["f"', from=1-2, to=1-1]
	\arrow["g", from=1-2, to=1-3]
\end{tikzcd}}\\~\\
in $\bold{CGWH}$. Define the standard homotopy pushout of the diagram to be the quotient space $$\hocolim(\mS)=\frac{X\amalg(Z\times I)\amalg Y}{\sim}$$ where $\sim$ is the equivalence relation generated by $f(z)\sim (z,0)$ and $g(z)\sim(z,1)$ for $z\in Z$. If $(Z,z_0)$ is a based space, then the equivalence relation is also generated by $(x_0,t)\sim(z_0,s)$ for $s,t\in I$. 
\end{defn}

TBA: Mimic the univ property up to homotopy

\begin{defn}{The Canonical Map of Homotopy Pushouts}{} Let $X,Y,Z\in\bold{CGWH}$ be spaces. Let $\mS$ denote the following diagram \\~\\
\adjustbox{scale=1.0,center}{\begin{tikzcd}
	X & Z & Y
	\arrow["f"', from=1-2, to=1-1]
	\arrow["g", from=1-2, to=1-3]
\end{tikzcd}}\\~\\
in $\bold{CGWH}$. Define the canonical map of the homotopy pushout of the diagram to be the map $$s:\hocolim(\mS)\to\colim(\mS)$$ given by the formula $$u\mapsto\begin{cases}
u & \text{ if }u\in X\\
f(z)=g(z) & \text{ if }u=(z,t)\in Z\times I\\
u & \text{ if }u\in Y
\end{cases}$$
\end{defn}

\begin{thm}{The Gluing Lemma}{} Suppose that we have a commutative diagram of spaces \\~\\
\adjustbox{scale=1.0,center}{\begin{tikzcd}
	X & Z & Y \\
	{X'} & {Z'} & {Y'}
	\arrow["f"', from=1-2, to=1-1]
	\arrow["{e_X}"', from=1-1, to=2-1]
	\arrow["{e_Z}"', from=1-2, to=2-2]
	\arrow["g", from=1-2, to=1-3]
	\arrow["{e_Y}", from=1-3, to=2-3]
	\arrow["{f'}", from=2-2, to=2-1]
	\arrow["{g'}"', from=2-2, to=2-3]
\end{tikzcd}}\\~\\
in $\bold{CGWH}$. If each $e_X,e_Y,e_Z$ are (homotopy) weak equivalences, then the induced map $$\hocolim(X\overset{f}{\leftarrow}Z\overset{g}{\rightarrow}Y)\to\hocolim(X'\overset{f'}{\leftarrow}Z'\overset{g'}{\rightarrow}Y')$$ defined by the formula $$u\mapsto\begin{cases}
e_X(u) & \text{ if }u\in X\\
(e_Z(v),t) & \text{ if }u=(v,t)\in Z\times I\\
e_Y(u) & \text{ if }u\in Y
\end{cases}$$ is a (homotopy) weak equivalence. 
\end{thm}

\subsection{Homotopy Pullbacks and Pushout Squares}
We use the standard homotopy pullback and pushout to recognize homotopy pullbacks and pushouts up to weak equivalence. 

\begin{defn}{Homotopy Pullback Squares}{} Let $W,X,Y,Z\in\bold{CGWH}$ be spaces such that there is a (not necessarily commutative) diagram \\~\\
\adjustbox{scale=1.0,center}{\begin{tikzcd}
	W & Y \\
	X & Z
	\arrow[from=1-1, to=1-2]
	\arrow[from=1-1, to=2-1]
	\arrow[from=1-2, to=2-2]
	\arrow[from=2-1, to=2-2]
\end{tikzcd}}\\~\\
\begin{itemize}
\item We say that the diagram is a homotopy pullback if the map $$\alpha:W\to\lim(X\overset{f}{\rightarrow}Z\overset{g}{\leftarrow}Y)\overset{c}{\longrightarrow}\text{holim}(X\overset{f}{\rightarrow}Z\overset{g}{\leftarrow}Y)$$ is a weak equivalence. 
\item We say that the diagram is $k$-cartesian if $\alpha$ is $k$-connected. 
\end{itemize}
\end{defn}

It is clear that the standard homotopy pullback indeed is a homotopy pullback. 

\begin{defn}{Homotopy Pushout Squares}{} Let $W,X,Y,Z\in\bold{CGWH}$ be spaces such that there is a (not necessarily commutative) diagram \\~\\
\adjustbox{scale=1.0,center}{\begin{tikzcd}
	W & Y \\
	X & Z
	\arrow[from=1-1, to=1-2]
	\arrow[from=1-1, to=2-1]
	\arrow[from=1-2, to=2-2]
	\arrow[from=2-1, to=2-2]
\end{tikzcd}}\\~\\
\begin{itemize}
\item We say that the square is a homotopy pushout square if the map $$\beta:\hocolim(X\overset{f}{\leftarrow}W\overset{g}{\rightarrow}Y)\overset{s}{\longrightarrow}\colim(X\overset{f}{\leftarrow}W\overset{g}{\rightarrow}Y)\to Z$$ is a weak equivalence. 
\item We say that the diagram is $k$-cocartesian if $\beta$ is $k$-connected. 
\end{itemize}
\end{defn}

Notice that this definition makes sense because they satisfy the universal property up to homotopy, in the following way. 

\begin{prp}{}{} Let $W,X,Y,Z\in\bold{CGWH}$ be spaces such that the following diagram is a homotopy pullback square: \\~\\
\adjustbox{scale=1.0,center}{\begin{tikzcd}
	W & Y \\
	X & Z
	\arrow["a", from=1-1, to=1-2]
	\arrow["b"', from=1-1, to=2-1]
	\arrow["c", from=1-2, to=2-2]
	\arrow["d"', from=2-1, to=2-2]
\end{tikzcd}}\\~\\
If $A\in\bold{CGWH}$ is a space and $p:A\to Y$ and $q:A\to X$ such that $c\circ p\simeq d\circ q$, then there exists $u:A\to W$ such that $a\circ u\simeq b\circ u$. In other words, the following diagram commutes up to homotopy: \\~\\
\adjustbox{scale=1.0,center}{\begin{tikzcd}
	A \\
	& W & X \\
	& Y & Z
	\arrow["{\exists u}"{description}, dashed, from=1-1, to=2-2]
	\arrow["p",bend left = 20, from=1-1, to=2-3]
	\arrow["q"', bend right = 20, from=1-1, to=3-2]
	\arrow["a", from=2-2, to=2-3]
	\arrow["b"', from=2-2, to=3-2]
	\arrow["c", from=2-3, to=3-3]
	\arrow["d"', from=3-2, to=3-3]
\end{tikzcd}}\\~\\
\end{prp}

\begin{prp}{}{} Let $W,X,Y,Z\in\bold{CGWH}$ be spaces such that the following diagram is a homotopy pushout square: \\~\\
\adjustbox{scale=1.0,center}{\begin{tikzcd}
	W & Y \\
	X & Z
	\arrow["a", from=1-1, to=1-2]
	\arrow["b"', from=1-1, to=2-1]
	\arrow["c", from=1-2, to=2-2]
	\arrow["d"', from=2-1, to=2-2]
\end{tikzcd}}\\~\\
If $A\in\bold{CGWH}$ is a space and $p:Y\to A$ and $q:X\to A$ such that $p\circ a\simeq q\circ b$, then there exists $u:Z\to A$ such that $u\circ c\simeq u\circ d$. In other words, the following diagram commutes up to homotopy: \\~\\
\adjustbox{scale=1.0,center}{\begin{tikzcd}
	W & X \\
	Y & Z \\
	&& A
	\arrow["a", from=1-1, to=1-2]
	\arrow["b"', from=1-1, to=2-1]
	\arrow["c", from=1-2, to=2-2]
	\arrow["p", bend left = 20, from=1-2, to=3-3]
	\arrow["d"', from=2-1, to=2-2]
	\arrow["q"',bend right = 20, from=2-1, to=3-3]
	\arrow["{\exists u}"{description}, dashed, from=2-2, to=3-3]
\end{tikzcd}}\\~\\
\end{prp}

\subsection{Recognizing Homotopy Pullbacks and Pushouts}
\begin{prp}{}{} Let $X,Y,Z\in\bold{CGWH}$ be spaces. Let $\mS$ denote the following diagram \\~\\
\adjustbox{scale=1.0,center}{\begin{tikzcd}
	X & Z & Y
	\arrow["f"', from=1-2, to=1-1]
	\arrow["g", from=1-2, to=1-3]
\end{tikzcd}}\\~\\
in $\bold{CGWH}$. Then the following spaces are homeomorphic. 
\begin{itemize}
\item $\hocolim(\mS)$
\item $\colim(M_f\leftarrow Z\rightarrow Y)$
\item $\colim(X\leftarrow Z\rightarrow M_g)$
\item $\colim(M_f\leftarrow Z\rightarrow M_g)$
\end{itemize}
\end{prp}

\begin{prp}{}{} Let $X,Y,Z\in\bold{CGWH}$ be spaces such that \\~\\
\adjustbox{scale=1.0,center}{\begin{tikzcd}
	X & Z & Y
	\arrow["f", from=1-1, to=1-2]
	\arrow["g"', from=1-3, to=1-2]
\end{tikzcd}}\\~\\
is a diagram in $\bold{CGWH}$. Then the following spaces are homeomorphic. 
\begin{itemize}
\item $\text{holim}(X\overset{f}{\rightarrow}Z\overset{g}{\leftarrow}Y)$
\item $\lim(P_f\rightarrow Z\overset{g}{\leftarrow}Y)$
\item $\lim(X\overset{f}{\rightarrow}Z\leftarrow P_g)$
\item $\lim(P_f\rightarrow Z\leftarrow P_g)$
\end{itemize}
\end{prp}

In model theoretic terms: the homotopy pullback can be computed by the standard pullback of fibrant replacements. \\

When one of the maps $f$ or $g$ is a fibration, then the notion of a pullback coincides with that of homotopy pullback. 

\begin{prp}{}{} Let $X,Y,Z\in\bold{CGWH}$ be spaces such that \\~\\
\adjustbox{scale=1.0,center}{\begin{tikzcd}
	X & Z & Y
	\arrow["f", from=1-1, to=1-2]
	\arrow["g"', from=1-3, to=1-2]
\end{tikzcd}}\\~\\
is a diagram in $\bold{CGWH}$. If $f$ or $g$ is a fibration, then the canonical map $$\lim(X\overset{f}{\rightarrow}Z\overset{g}{\leftarrow}Y)\to\text{holim}(X\overset{f}{\rightarrow}Z\overset{g}{\leftarrow}Y)$$ is a homotopy equivalence. 
\end{prp}

\subsection{Relation to Homotopy (Co)Fibers}
Recall that the mapping path space $P_f$ of a map $f:X\to Y$ is defined to be $$P_f=f^\ast(\text{Map}(I,Y))=\{(x,\phi)\subseteq X\times\text{Map}(I,Y)\;|\;f(x)=\pi_0(\phi)=\phi(0)\}$$ we can now prove that $P_f$ is a homotopy invariance. 

\begin{crl}{}{} Let $X,Y\in\bold{CGWH}$ be spaces. Let $f,g:X\to Y$ be maps. Then there is a homotopy equivalence $$P_f\simeq P_g$$ Moreover, there is a homotopy equivalence $$\text{hofiber}_y(f)\simeq\text{hofiber}_y(g)$$ for any $y\in Y$. 
\end{crl}

Recall that the fiber of a map $f:X\to Y$ behaves poorly because the fibers are not homeomorphic and not even homotopy equivalent. However, we can now prove that the homotopy fibers are the correct notion of a fiber to study because they are homotopy equivalent. 

\begin{crl}{}{} Let $X,Y\in\bold{CGWH}$ be space. Let $f:X\to Y$ be a map. If $y_1$ and $y_2$ lie in the same path component of $Y$ then there is a homotopy equivalence $$\text{hofiber}_{y_1}(f)=\text{hofiber}_{y_2}(f)$$
\end{crl}

\begin{crl}{}{} Let $X,Y\in\bold{CGWH}$ be spaces. Let $f,g:X\to Y$ be maps. Then there is a homotopy equivalence $$M_f\simeq M_g$$ Moreover, there is a homotopy equivalence $$\text{hocofiber}(f)\simeq\text{hocofiber}(g)$$ for any $y\in Y$. 
\end{crl}

We can we interpret homotopy pullbacks and pushouts using homotopy (co)fibers. 

\begin{prp}{}{} Let $W,X,Y,Z\in\bold{CGWH}$ be spaces such that following is a (not necessarily commutative) square \\~\\
\adjustbox{scale=1.0,center}{\begin{tikzcd}
	W & Y \\
	X & Z
	\arrow[from=1-1, to=1-2]
	\arrow[from=1-1, to=2-1]
	\arrow[from=1-2, to=2-2]
	\arrow[from=2-1, to=2-2]
\end{tikzcd}}\\~\\
Then the following are true. 
\begin{itemize}
\item The square is a homotopy pullback if and only if for all $x\in X$, the map $$\text{hofiber}_x(W\to X)\to\text{hofiber}_{f(x)}(Y\to Z)$$ is a weak equivalence. 
\item The square is $k$-carteisna if and only if for all $x\in X$, the map $$\text{hofiber}_x(W\to X)\to\text{hofiber}_{f(x)}(Y\to Z)$$ is $k$-connected. 
\end{itemize}
\end{prp}

\subsection{Connectedness of Homotopy Squares}

\pagebreak
\section{Blakers-Massey Theorem}
\subsection{The Blakers-Massey Theorem for Squares}
The Blakers-Massey theorem is a direct generalization of the homotopy excision theorem. Its proof takes a similar form to the homotopy excision theorem. Let us recall some definitions used. 

\begin{defn}{(Degenerative) Cubes}{} Let $a=(a_1,\dots,a_n)\in\R^n$. Let $\delta>0$. Let $L\subseteq\{1,\dots,n\}$. A cube in $\R^n$ is a set of the form $$W=W(a,\delta,L)=\{x\in\R^n\;|\;a_i\leq x\leq a_i+\delta\text{ for }i\in L\text{ and }x_i=a_i\text{ for }i\notin L\}$$
\end{defn}

\begin{lmm}{}{} Let $Y$ be a space. Let $B\subseteq Y$ be a subspace of $Y$. Let $W$ be a cube in $\R^n$. Let $f:W\to Y$ be a map. Let $j=1$ or $2$. Suppose that there exists some $p\leq\abs{L}$ such that $$f^{-1}(B)\cap C\subset K_p^j(C)=\left\{x\in C\;|\;\frac{\delta(j-1)}{2}+a_i<x_i<\frac{\delta j}{2}+a_i\text{ for at least }p\text{ values of }i\in L\right\}$$ for all cubes $C\subset\partial W$. Then there exists a map $g:W\to Y$ such that $g\overset{\partial W}{\simeq} f$ and $$g^{-1}(B)\subset K_p^j(C)$$
\end{lmm}

\begin{prp}{}{} Let $X$ be a space. Let $X_0,X_1,X_2\subseteq X$ be subspaces of $X$ such that $$X=X_1\amalg_{X_0}X_2$$ Let $f:I^n\to X$ be a map. Suppose that $W\subseteq I^n$ is any cube given by the Lebesgue covering lemma for which $f(W)\subseteq X_i$ for one of $i=0,1,2$. Assume that for each $i=1,2$, $(X_i,X_0)$ is $k_i$-connected with $k_i\geq 0$. Then there exists a homotopy $$H:I^n\times I\to X$$ from $f=H(-,0)$ such that the following are true. 
\begin{itemize}
\item If $f(W)\subset X_i$, then $H(W,t)\subset X_i$ for all $t\in I$. 
\item If $f(W)\subset X_0$, then $H(W,t)=f(W)$ for all $t\in I$. 
\item If $f(W)\subset X_i$, then $f^{-1}(X_i\setminus X_0)\cap W\subset K_{k_j+1}^j(W)$. 
\end{itemize}
\end{prp}

The reasons for the above setup is that we want to prove the following lemma. 

\begin{lmm}{}{} Let $X$ be a space. Let $e^{d_i}$ be a cell of dimension $d_i$ for $i=1,2$. Then the following diagram \\~\\
\adjustbox{scale=1.0,center}{\begin{tikzcd}
	X & X\cup e^{d_1} \\
	X\cup e^{d_2} & X\cup e^{d_1}\cup e^{d_2}
	\arrow[from=1-1, to=1-2]
	\arrow[from=1-1, to=2-1]
	\arrow[from=1-2, to=2-2]
	\arrow[from=2-1, to=2-2]
\end{tikzcd}}\\~\\
given by inclusion maps is $(d_1+d_2-3)$-cartesian. 
\end{lmm}

This lemma has a less geometric way of proving it that does not involve any of the preparative lemmas and propositions, however it does use significant material that have not been covered. 

\begin{thm}{Blakers-Massey Theorem for Squares}{} Let $X_0,X_1,X_2,X_{12}\in\bold{CGWH}$ be spaces such that the square \\~\\
\adjustbox{scale=1.0,center}{\begin{tikzcd}
	X_0 & X_1 \\
	X_2 & X_{12}
	\arrow[from=1-1, to=1-2]
	\arrow[from=1-1, to=2-1]
	\arrow[from=1-2, to=2-2]
	\arrow[from=2-1, to=2-2]
\end{tikzcd}}\\~\\
is a homotopy pushout. Suppose the map $X_0\to X_i$ is $k_i$-connected for $i=1,2$. Then the diagram is $(k_1+k_2-1)$-cartesian. Explicitly, this means that $$\alpha:X_0\to\text{holim}(X_1\rightarrow X_{12}\leftarrow X_2)$$ is $(k_1+k_2-1)$-connected. 
\end{thm}

This theorem directly generalizes the homotopy excision theorem in the following way. For $X$ a CW complex and $A,B$ two subcomplexes with non-empty intersection and $X=A\cup B$, consider the following square of inclusions: \\~\\
\adjustbox{scale=1.0,center}{\begin{tikzcd}
	A\cap B & A \\
	B & X
	\arrow[from=1-1, to=1-2]
	\arrow[from=1-1, to=2-1]
	\arrow[from=1-2, to=2-2]
	\arrow[from=2-1, to=2-2]
\end{tikzcd}}\\~\\
We have seen that such a square diagram is a homotopy pushout diagram. Now any inclusion map $W\hookrightarrow Z$ is $k$-connected if and only if $(Z,W)$ is $k$-connected. So $(A,A\cap B)$ is $k_1$-connected and $(X,B)$ is $k_2$-connected. Blaker's-Massey theorem implies that $$\text{hofiber}(A\cap B\to A)\to\text{hofiber}(B,X)$$ is $(k_1+k_2-1)$-connected. But by definition we have an isomorphism $\pi_k(\text{hofiber}(U\to V)\cong\pi_{k+1}(V,U)$. So we are really just saying that $\pi_k(A,A\cap B)\to\pi_k(X,B)$ given by the inclusion is $(k_1+k_2)$-connected. 

\subsection{The Dual Blakers-Massey Theorem for Squares}
\begin{thm}{Dual Blakers-Massey Theorem for Squares}{} Let $X_0,X_1,X_2,X_{12}\in\bold{CGWH}$ be spaces such that the square \\~\\
\adjustbox{scale=1.0,center}{\begin{tikzcd}
	X_0 & X_1 \\
	X_2 & X_{12}
	\arrow[from=1-1, to=1-2]
	\arrow[from=1-1, to=2-1]
	\arrow[from=1-2, to=2-2]
	\arrow[from=2-1, to=2-2]
\end{tikzcd}}\\~\\
is a homotopy pullback. Suppose the map $X_i\to X_{12}$ is $k_i$-connected for $i=1,2$. Then the diagram is $(k_1+k_2-1)$-cocartesian. Explicitly, this means that $$\beta:\hocolim(X_1\leftarrow X_0\rightarrow X_2)\to X_{12}$$ is $(k_1+k_2-1)$-connected.
\end{thm}

\pagebreak
\section{Homotopy n-Cubes}
In algebraic topology, we have learnt about spaces, maps of spaces and maps of maps of spaces. We can say this in a more compact way. Namely, if we think of maps of maps of space as a square (2-cube), we can think of spaces as 0-cubes and maps of spaces as 1-cube. We have studied 2-cubes extensively under the guise of homotopy pullbacks and pushouts. We can now take this further and consider general n-cubes. 

\subsection{Maps between Homotopy Squares}

\subsection{n-Cubes of Spaces}
\begin{defn}{n-Cubes of Spaces}{} Let $n\in\N$. Let $P(n)$ denote the category of posets of the set $\{1,\dots,n\}$. An $n$-cube of spaces is a functor $$X:P(n)\to\bold{CGWH}$$ An $n$-cube of based spaces is a functor $X:P(n)\to\bold{CGWH}_\ast$. 
\end{defn}

Explicitly, an $n$-cube of spaces $X:P(n)\to\bold{CGWH}$ consists of the following data. 
\begin{itemize}
\item For each $S\subseteq\{1,\dots,n\}$ a space $X_S$
\item For each $S\subseteq T$, a map $f_{S\subseteq T}:X_S\to X_T$ such that $f_{S\subseteq S}=1_{X_S}$ and for all $R\subseteq S\subseteq T$, we have a commutative diagram \\~\\
\adjustbox{scale=1.0,center}{\begin{tikzcd}
	{X_R} & {X_S} \\
	& {X_T}
	\arrow["{f_{R\subseteq S}}", from=1-1, to=1-2]
	\arrow["{f_{R\subseteq T}}"', from=1-1, to=2-2]
	\arrow["{f_{S\subseteq T}}", from=1-2, to=2-2]
\end{tikzcd}}\\~\\
\end{itemize}

Omit drawing composite arrows and omit drawing identities. \\
Also: punctured cubes def

\begin{defn}{Cube of Cubes}{} An $n$-cube of $m$-cubes is a functor $$X:P(n)\times P(m)\to\bold{CGWH}$$
\end{defn}

\begin{lmm}{}{} An $n$-cube of $m$-cubes $X$ is precisely an $(n+m)$-cube. 
\end{lmm}

\begin{defn}{Map of $n$-Cubes}{} Let $X,Y:P(n)\to\bold{CGWH}$ be $n$-cubes. A map of $n$-cubes is a natural transformation $F:X\to Y$ such that the assignment $Z:P(n+1)\to\bold{CGWH}$ given by $$Z(S)=\begin{cases}
X(S) & \text{ if } S\subseteq\{1,\dots,n\}\\
Y(S\setminus\{1,\dots, n+1\}) & \text{ if }\{1,\dots,n+1\}\subseteq S
\end{cases}$$
defines an $(n+1)$-cube. 
\end{defn}

objectwise (co)fibration, homotopy (weak) equivalence. homeomorphism

\begin{defn}{Strongly Homotopy Cartesian}{} Let $X$ be an $n$-cube of spaces. We say that $X$ is strongly homotopy cartesian if each of its faces of dimension $n\geq 2$ is homotopy cartesian. 
\end{defn}

\pagebreak
\section{Homotopy Limits and Colimits}

Let $X:\mJ\to\bold{Top}$ be a diagram of spaces. Denote the constant functor of the one point space by $\Delta\ast:\bold{X}\to\bold{Top}$. The data of the constant functor is given as follows. 
\begin{itemize}
\item For each $I\in\mJ$, $\Delta\ast(I)=\ast$
\item For each morphism $f:I\to J$ in $X$, define $\Delta\ast(f)=\text{id}_\ast$. 
\end{itemize}
Consider the set of all natural transformations $\Delta\ast\Rightarrow X$ denoted by $\text{Nat}(\Delta\ast,X)$. Now this set can inherit a subspace topology via the isomorphism (of sets) $$\text{Nat}(\Delta\ast,X)\cong\prod_{I\in\mJ}\Hom_\bold{Top}(\ast,X_I)\subset\prod_{I\in\mJ}X_I$$ There is in fact a canonical homeomorphism between the set of natural transformations and the limit of $X$. 

\begin{thm}{}{} Let $X:\mJ\to\bold{Top}$ be a diagram of spaces. Then there is a canonical homeomorphism $$\lim_\mJ X\cong\text{Nat}(\Delta\ast,X)$$
\end{thm}
Ref: Cubical diagrams

\begin{thm}{}{} Let $X:\mJ\to\bold{Top}$ be a diagram of spaces. Then there is a canonical homeomorphism $$\colim_\mJ X\cong\frac{\coprod_{I\in\mJ}X_I}{\sim}$$ where $x\in X_I\sim y\in X_J$ if and only if there exists $f:I\to J$ such that $X(f)(x)=y$. 
\end{thm}

\subsection{Homotopy Limits and Colimits}
Let $\mJ$ be a small diagram. Let $I\in\mJ$ and denote $\mJ/I$ to be the overcategory with the distinguished object $I$. We can turn it into a topological space by constructing its classifying space $$\mB(\mJ/I)=\abs{N(\mJ/I)}$$ which is the geometric realization of the nerve of $\mJ/I$. We aim to use the overcategory $\mJ/I$ to record homotopy information. Recall that the limit is canonically isomorphic to the equalizer of $$f,g:\prod_{J\in\Obj\mJ}X_J\to\prod_{(\alpha:J\to I)\in\mJ}X_I$$ as follows. 
\begin{itemize}
\item Define $f$ to be the unique map such that $\pi_I\circ f=\pi_{\text{cod}(\alpha)}$ where both $\pi$ are projections. 
\item Define $g$ to be the unique map such that $\pi_{\text{cod}(\alpha)}\circ g=F(\alpha)\circ\pi_{\text{dom}(\alpha)}$
\end{itemize}
We aim to replace each space $X_I$ by the space $\text{Map}(\mB(\mJ/I),X_I)$ (which is why we work with $\bold{CGWH}$). Indeed the hom space and $X_I$ are homotopy equivalent since $\mB(\mJ/I)$ is contractible. Notice that this is no longer a functor in $I$, but rather a bifunctor. This all works with a simplicial model category. 

\begin{defn}{Homotopy Limits}{} Let $\mC$ be a simplicial model category. Let $X:\mJ\to\mC$ be a diagram. Define two maps $$f,g:\prod_{J\in\Obj\mJ}\text{Map}(\mB(\mJ/J),X_J)\to\prod_{(\alpha:J\to I)\in\mJ}\text{Map}(\mB(\mJ/J),X_I)$$ as follows. 
\begin{itemize}
\item Define $f$ to be the unique map such that $$\pi_I\circ f=\text{Map}(\mB(\mJ/J),X(\alpha):X_J\to X_I)$$ for any $\alpha:J\to I$ a morphism in $\mJ$. 
\item Define $g$ to be the unique map such that $$\pi_I\circ g=\text{Map}\left(\mB\left(\mJ/X:\mJ/I\to\mJ/J\right),X_I\right)$$ for any $\alpha:J\to I$ a morphism in $\mJ$. 
\end{itemize}
Define the homotopy limit $\text{holim}X$ of $X$ to be the equalizer of the maps $$\text{holim}X=\text{Eq}(f,g)$$
\end{defn}

Let $\mJ$ be the small diagram consisting of two objects $A,B$ and two non-trivial morphisms $f,g:A\to B$. Let us illustrate the homotopy limit of $\mJ$. The slice category $\mJ/A$ consists only of a single object corresponding to the identity map $\text{id}_A:A\to A$. The slice category $\mJ/B$ consists of $3$ objects corresponding to $\text{id}_B$, $f$ and $g$. Non-trivial morphisms are given as follows: \\~\\
\adjustbox{scale=1.0,center}{\begin{tikzcd}
	f \\
	& {\text{id}_B} \\
	g
	\arrow[from=1-1, to=2-2]
	\arrow[from=3-1, to=2-2]
\end{tikzcd}}\\~\\
Passing to the classifying space, $\mB(\mJ/A)$ is just the one point space and $\mB(\mJ/B)$ becomes the interval $I$. 

Expanding things out show that $$f,g:\text{Map}(,A)\times\text{Map}(,B)$$ 

\begin{prp}{}{} Let $X:\mJ\to\bold{CGWH}$ be a diagram. Then there is a natural transformation $$\text{holim}_\mJ X\cong\text{Nat}(\mB(\mJ/-):\mJ\to\bold{CGWH},X)$$
\end{prp}

\begin{defn}{Homotopy Colimits}{} Let $\mC$ be a simplicial model category. Let $X:\mJ\to\mC$ be a diagram. Define two maps $$f,g:\coprod_{(\alpha:J\to I)\in\mJ}X_J\times\mB\left((I/\mJ)^\text{op}\right)\to\coprod_{J\in\Obj\mJ}X_J\times\mB\left((J/\mJ)^\text{op}\right)$$ as follows. 
\begin{itemize}
\item On each summand of the domain of $f$, define $f$ to be the map $$X(\alpha)\times\text{id}_{\mB\left((I/\mJ)^\text{op}\right)}:X_J\times\mB\left((I/\mJ)^\text{op}\right)\longrightarrow X_I\times\mB\left((I/\mJ)^\text{op}\right)$$ and then injecting the into the coproduct. 
\item On each summand of the domain of $g$, define $g$ to be the map $$\text{id}_{X_J}\times\mB(\mJ/X)^\text{op}:X_J\times\mB\left((I/\mJ)^\text{op}\right)\longrightarrow X_J\times\mB\left((J/\mJ)^\text{op}\right)$$ and then injecting into the coproduct. 
\end{itemize}
Define the homotopy colimit $\hocolim X$ of $X$ to be the coequalizer of the maps $$\hocolim X=\text{Coeq}(f,g)$$
\end{defn}


\end{document}
