\documentclass[a4paper]{article}

%=========================================
% Packages
%=========================================
\usepackage{mathtools}
\usepackage{amsfonts}
\usepackage{amsmath}
\usepackage{amssymb}
\usepackage{amsthm}
\usepackage[a4paper, total={6in, 8in}, margin=1in]{geometry}
\usepackage[utf8]{inputenc}
\usepackage{fancyhdr}
\usepackage[utf8]{inputenc}
\usepackage{graphicx}
\usepackage{physics}
\usepackage[listings]{tcolorbox}
\usepackage{hyperref}
\usepackage{tikz-cd}
\usepackage{adjustbox}
\usepackage{enumitem}
\usepackage[font=small,labelfont=bf]{caption}
\usepackage{subcaption}
\usepackage{wrapfig}
\usepackage{makecell}



\raggedright

\usetikzlibrary{arrows.meta}

\DeclarePairedDelimiter\ceil{\lceil}{\rceil}
\DeclarePairedDelimiter\floor{\lfloor}{\rfloor}

%=========================================
% Fonts
%=========================================
\usepackage{tgpagella}
\usepackage[T1]{fontenc}


%=========================================
% Custom Math Operators
%=========================================
\DeclareMathOperator{\adj}{adj}
\DeclareMathOperator{\im}{im}
\DeclareMathOperator{\nullity}{nullity}
\DeclareMathOperator{\sign}{sign}
\DeclareMathOperator{\dom}{dom}
\DeclareMathOperator{\lcm}{lcm}
\DeclareMathOperator{\ran}{ran}
\DeclareMathOperator{\ext}{Ext}
\DeclareMathOperator{\dist}{dist}
\DeclareMathOperator{\diam}{diam}
\DeclareMathOperator{\aut}{Aut}
\DeclareMathOperator{\inn}{Inn}
\DeclareMathOperator{\syl}{Syl}
\DeclareMathOperator{\edo}{End}
\DeclareMathOperator{\cov}{Cov}
\DeclareMathOperator{\vari}{Var}
\DeclareMathOperator{\cha}{char}
\DeclareMathOperator{\Span}{span}
\DeclareMathOperator{\ord}{ord}
\DeclareMathOperator{\res}{res}
\DeclareMathOperator{\Hom}{Hom}
\DeclareMathOperator{\Mor}{Mor}
\DeclareMathOperator{\coker}{coker}
\DeclareMathOperator{\Obj}{Obj}
\DeclareMathOperator{\id}{id}
\DeclareMathOperator{\GL}{GL}
\DeclareMathOperator*{\colim}{colim}

%=========================================
% Custom Commands (Shortcuts)
%=========================================
\newcommand{\CP}{\mathbb{CP}}
\newcommand{\GG}{\mathbb{G}}
\newcommand{\F}{\mathbb{F}}
\newcommand{\N}{\mathbb{N}}
\newcommand{\Q}{\mathbb{Q}}
\newcommand{\R}{\mathbb{R}}
\newcommand{\C}{\mathbb{C}}
\newcommand{\E}{\mathbb{E}}
\newcommand{\Prj}{\mathbb{P}}
\newcommand{\RP}{\mathbb{RP}}
\newcommand{\T}{\mathbb{T}}
\newcommand{\Z}{\mathbb{Z}}
\newcommand{\A}{\mathbb{A}}
\renewcommand{\H}{\mathbb{H}}
\newcommand{\K}{\mathbb{K}}

\newcommand{\mA}{\mathcal{A}}
\newcommand{\mB}{\mathcal{B}}
\newcommand{\mC}{\mathcal{C}}
\newcommand{\mD}{\mathcal{D}}
\newcommand{\mE}{\mathcal{E}}
\newcommand{\mF}{\mathcal{F}}
\newcommand{\mG}{\mathcal{G}}
\newcommand{\mH}{\mathcal{H}}
\newcommand{\mI}{\mathcal{I}}
\newcommand{\mJ}{\mathcal{J}}
\newcommand{\mK}{\mathcal{K}}
\newcommand{\mL}{\mathcal{L}}
\newcommand{\mM}{\mathcal{M}}
\newcommand{\mO}{\mathcal{O}}
\newcommand{\mP}{\mathcal{P}}
\newcommand{\mS}{\mathcal{S}}
\newcommand{\mT}{\mathcal{T}}
\newcommand{\mV}{\mathcal{V}}
\newcommand{\mW}{\mathcal{W}}

%=========================================
% Colours!!!
%=========================================
\definecolor{LightBlue}{HTML}{2D64A6}
\definecolor{ForestGreen}{HTML}{4BA150}
\definecolor{DarkBlue}{HTML}{000080}
\definecolor{LightPurple}{HTML}{cc99ff}
\definecolor{LightOrange}{HTML}{ffc34d}
\definecolor{Buff}{HTML}{DDAE7E}
\definecolor{Sunset}{HTML}{F2C57C}
\definecolor{Wenge}{HTML}{584B53}
\definecolor{Coolgray}{HTML}{9098CB}
\definecolor{Lavender}{HTML}{D6E3F8}
\definecolor{Glaucous}{HTML}{828BC4}
\definecolor{Mauve}{HTML}{C7A8F0}
\definecolor{Darkred}{HTML}{880808}
\definecolor{Beaver}{HTML}{9A8873}
\definecolor{UltraViolet}{HTML}{52489C}



%=========================================
% Theorem Environment
%=========================================
\tcbuselibrary{listings, theorems, breakable, skins}

\newtcbtheorem[number within = subsection]{thm}{Theorem}%
{	colback=Buff!3, 
	colframe=Buff, 
	fonttitle=\bfseries, 
	breakable, 
	enhanced jigsaw, 
	halign=left
}{thm}

\newtcbtheorem[number within=subsection, use counter from=thm]{defn}{Definition}%
{  colback=cyan!1,
    colframe=cyan!50!black,
	fonttitle=\bfseries, breakable, 
	enhanced jigsaw, 
	halign=left
}{defn}

\newtcbtheorem[number within=subsection, use counter from=thm]{axm}{Axiom}%
{	colback=red!5, 
	colframe=Darkred, 
	fonttitle=\bfseries, 
	breakable, 
	enhanced jigsaw, 
	halign=left
}{axm}

\newtcbtheorem[number within=subsection, use counter from=thm]{prp}{Proposition}%
{	colback=LightBlue!3, 
	colframe=Glaucous, 
	fonttitle=\bfseries, 
	breakable, 
	enhanced jigsaw, 
	halign=left
}{prp}

\newtcbtheorem[number within=subsection, use counter from=thm]{lmm}{Lemma}%
{	colback=LightBlue!3, 
	colframe=LightBlue!60, 
	fonttitle=\bfseries, 
	breakable, 
	enhanced jigsaw, 
	halign=left
}{lmm}

\newtcbtheorem[number within=subsection, use counter from=thm]{crl}{Corollary}%
{	colback=LightBlue!3, 
	colframe=LightBlue!60, 
	fonttitle=\bfseries, 
	breakable, 
	enhanced jigsaw, 
	halign=left
}{crl}

\newtcbtheorem[number within=subsection, use counter from=thm]{eg}{Example}%
{	colback=Beaver!5, 
	colframe=Beaver, 
	fonttitle=\bfseries, 
	breakable, 
	enhanced jigsaw, 
	halign=left
}{eg}

\newtcbtheorem[number within=subsection, use counter from=thm]{ex}{Exercise}%
{	colback=Beaver!5, 
	colframe=Beaver, 
	fonttitle=\bfseries, 
	breakable, 
	enhanced jigsaw, 
	halign=left
}{ex}

\newtcbtheorem[number within=subsection, use counter from=thm]{alg}{Algorithm}%
{	colback=UltraViolet!5, 
	colframe=UltraViolet, 
	fonttitle=\bfseries, 
	breakable, 
	enhanced jigsaw, 
	halign=left
}{alg}




%=========================================
% Hyperlinks
%=========================================
\hypersetup{
    colorlinks=true, %set true if you want colored links
    linktoc=all,     %set to all if you want both sections and subsections linked
    linkcolor=DarkBlue,  %choose some color if you want links to stand out
}


\pagestyle{fancy}
\fancyhf{}
\rhead{Labix}
\lhead{Algebraic Geometry 2}
\rfoot{\thepage}

\title{Algebraic Geometry 2}

\author{Labix}

\date{\today}
\begin{document}
\maketitle
\begin{abstract}
Algebraic Geometry is such a messy subject in a sense that a different books and lecture notes introduce different materials in a different orders, as well as having different prerequisites. After understanding a bit more in the subject, I believe that there is the need to give a clear distinction between traditional algebraic geometry and contemporary algebraic geometry. Although there are undoubtedly many overlapping between the two, I attempt to separate them to make clear their motivations as well as their results. \\~\\

This book will mainly cover traditional algebraic geometry in the sense that the construction of affine and projective varieties will be covered, as well as the Hilbert Nullstellensatz theorems, morphisms, tangent maps and smoothness as well as classical constructions of some varieties. Affine schemes and sheaf theory are left for another time where they attempt to reinvent the fundamentals of algebraic geometry. \\~\\

Knowledge on commutative algebra is required as a prerequisite. These set of notes make use of
\begin{itemize}
\item Algebraic Geometry I by I. R. Shafarevich and V. I. Danilov
\item Algebraic Geometry by R. Hartshorne
\item An Invitation to Algebraic Geometry by Karen. S, Pekka. K, Lauri .K, William .T
\end{itemize}
\end{abstract}
\pagebreak
\tableofcontents

\pagebreak
\section{The Tangent Space and Smooth Points}
\subsection{The Tangent Space of Affine Varieties}
\begin{defn}{The Tangent Space of an Affine Variety}{} Let $k$ be a field. Let $V=\V(f_1,\dots,f_r)$ be an affine variety over $k$. Define the tangent space of $V$ at $p\in V$ to be the zero set $$T_pV=\V\left(\sum_{k=1}^n\frac{\partial f_1}{\partial x_k}\bigg{|}_p(x_k-p_k),\dots,\sum_{k=1}^n\frac{\partial f_r}{\partial x_k}\bigg{|}_p(x_k-p_k)\right)$$
\end{defn}

It should first be made sense that the definition is independent of the choice of polynomials $f_1,\dots,f_r$ of the zero set. 

\begin{prp}{}{} Let $V$ be a closed affine variety over $\C$. Let $p\in V$. Let $m_p$ denote the corresponding maximal ideal. Then there is an isomorphism $$T_pV\cong\left(\frac{m_p}{m_p^2}\right)^\ast$$ given by ?????. In particular, we have the identity $$\dim(T_pV)=\dim_{\C[V]/m_p}(m_p/m_p^2)$$
\end{prp}

\begin{defn}{The Jacobian Matrix}{} Let $k$ be a field. Let $V=\V(f_1,\dots,f_m)\subseteq\A_k^n$ be an affine variety. Let $p\in V$. Define the Jacobian matrix of $V$ at $p$ to be the $m\times n$ matrix $$J_{V,p}=\begin{pmatrix}
\frac{\partial f_1}{\partial x_1}\bigg{|}_p & \cdots & \frac{\partial f_1}{\partial x_n}\bigg{|}_p\\
\vdots & \ddots & \vdots\\
\frac{\partial f_m}{\partial x_1}\bigg{|}_p & \cdots & \frac{\partial f_m}{\partial x_n}\bigg{|}_p
\end{pmatrix}$$
\end{defn}

\begin{prp}{}{} Let $k$ be a field. Let $V=\V(f_1,\dots,f_m)\subseteq\A_k^n$ be an affine variety. Let $p=(p_1,\dots,p_n)\in V$. Then $$T_pV=\left\{(x_1,\dots,x_n)\in\A_k^n\;\bigg{|}\;J_{V,p}\cdot\begin{pmatrix}x_1-p_1\\\vdots\\x_n-p_n\end{pmatrix}=0\right\}$$
\end{prp}

\subsection{Smooth Points of an Affine Variety}
We continue to restrict our attention to affine varieties. 

\begin{defn}{Smooth and Singular Points of Affine Varieties}{} Let $k$ be a field. Let $X$ be an irreducible affine variety over $k$. Let $p\in X$ be a point. We say that $p$ is a smooth point of $X$ if $$\dim(T_p(X))=\dim(X)$$ Otherwise, we say that $p$ is a singular point of $X$. 
\end{defn}

\begin{prp}{}{} Let $V=\V(f_1,\dots,f_m)\subseteq\A_\C^n$ be an irreducible affine variety. Let $p\in V$. Then the following are equivalent. 
\begin{itemize}
\item $p$ is a smooth point of $V$. 
\item $\rank(J_{V,p})=n-\dim(V)$. 
\item $\mO_{V,p}$ is a regular local ring. 
\end{itemize}
\end{prp}

In particular, this shows that smoothness is independent of the choice of generators of $V$, because we have given a characterization in terms of a property of the local ring $\mO_{V,p}$. \\

Hard to prove: smoothness is preserved by isomorphisms. 

\subsection{The Tangent Space of Varieties in General}
\begin{defn}{The Tangent Space of a Quasi-Projective Variety}{} Let $k$ be a field. Let $V$ be a quasi-projective variety over $k$. Let $p\in V$. Define the tangent space of $V$ at $p$ to be $$T_pV=\left(\frac{m_p}{m_p}\right)^\ast$$ where $m_p$ is the unique maximal ideal of the local ring $\mO_{V,p}$. 
\end{defn}

\subsection{Smooth Points of a Variety in General}
We can now motivate the definition of a smooth point using the purely algebraic characterization. 

\begin{defn}{Smooth and Singular Points of A General Variety}{} Let $X$ be a variety. We say $p\in X$ is a smooth point of $X$ if the local ring $\mO_{X,p}$ is a regular local ring. Otherwise, we say that $p$ is a singular point of $X$. 
\end{defn}

\begin{thm}{}{} Let $X$ be a variety. Then the set of singular points of $X$ is a proper closed subset of $X$. 
\end{thm}

\begin{prp}{}{} Let $X$ be a variety. If $p\in X$ is a smooth point, then $\mO_{X,p}$ is a UFD. 
\end{prp}

\begin{prp}{}{} Let $X$ be a variety and let $Y\subseteq X$ be an irreducible subvariety of $X$. If $p\in X$ is non-singular, then there exists an affine neighbourhood $U\subseteq X$ of $x$ together with $f_1,\dots,f_k\in k[U]$
\end{prp}

\pagebreak
\section{Birational Geometry}
\subsection{Rational Morphisms}
\begin{defn}{Equivalent Maps}{} Let $X,Y$ be irreducible varieties. Let $U_1,U_2\subseteq X$ be open. Let $f_1:U_1\to Y$ and $f_2:U_2\to Y$ be morphisms of varieties. We say that $f_1$ and $f_2$ are equivalent if there exists an open subset $W\subseteq U_1\cap U_2$ such that $$f_1|_W=f_2|_W:W\to Y$$
\end{defn}

\begin{defn}{Rational Maps}{} Let $X,Y$ be irreducible varieties. A rational map $f:X\to Y$ is an equivalent class of morphisms of varieties $f:U\to Y$ for some open subset $U\subseteq X$. 
\end{defn}

Since open subsets of a variety dense, rational maps are maps that are defined almost entirely on $X$. 

\begin{defn}{Dominant Maps}{} Let $X,Y$ be irreducible varieties. Let $f:X\to Y$ be a rational map defined on $U\subseteq X$. We say that $f$ is dominant if $f(U)$ contains an open subset. 
\end{defn}

It only makes sense to compose rational maps if the former one is dominant. 

\begin{prp}{}{} Let $X,Y,Z$ be irreducible varieties. Let $f:X\to Y$ and $g:Y\to Z$ be rational maps. If $f$ is dominant, then $g\circ f$ is rational. 
\end{prp}

\begin{prp}{}{} Let $X$ be an irreducible variety. Then the set of rational maps $X\to\A^1$ are in bijection with the function field $K(X)$. 
\end{prp}

\subsection{Birational Maps}
\begin{defn}{Birational Maps}{} Let $X,Y$ be irreducible varieties. Let $f:X\to Y$ be a dominant rational map defined on $U\subseteq X$. We say that $f$ is a birational map if there exists a dominant rational map $g:Y\to X$ such that $$g\circ f=\text{id}_X\;\;\;\;\text{ and }\;\;\;\;f\circ g=\text{id}_Y$$ In this case, we say that $X$ and $Y$ are birational. 
\end{defn}

\subsection{Categorical Equivalence with Finitely Generated Field Extensions}
\begin{prp}{}{} Let $\phi:X\to Y$ be a dominant rational map represented by $\langle U,\phi_U\rangle$. Let $f\in \C[Y]$ be a rational function represented by $\langle V,f\rangle$ where $V$ is an open set in $Y$ and $f$ regular function on $V$. Then $f\circ\phi_U$ is a homomorphism of $\C$-algebras from $\C[Y]$ to $\C[X]$. \tcbline
\begin{proof}
Notice that since $\phi_U(U)$ is dense in $Y$, $\phi_U^{-1}(V)$ is a nonempty open subset of $X$. Thus $f\circ\phi_U$ is a regular function on $\phi_U^{-1}(V)$. Thus $f\circ\phi_U$ is rational function on $X$. This means that $f\circ\phi_U\in\C[X]$. \\~\\
In particular, the map taking $f$ to $f\circ\phi_U$ is a $\C$-algebra homomorphism. 
\end{proof}
\end{prp}

\begin{thm}{}{} Let $X$ and $Y$ be two varieties. The above construction gives a bijection between the set of dominant rational maps from $X\to Y$ and the set of $\C$-algebra homomorphisms from $\C[Y]$ to $\C[X]$. \\~\\
In other words, this correspondence is a contravariant functor from the category of varieties and the category of finitely generated field extensions of $\C$. 
\end{thm}

\begin{crl}{}{} Let $X,Y$ be two varieties. The the following conditions are equivalent. 
\begin{itemize}
\item $X$ and $Y$ are birationally equivalent
\item There exists open subsets $U\subseteq X$ and $V\subseteq Y$ with $U$ isomorphic to $V$
\item $K(X)$ and $K(Y)$ are isomorphic $\C$-algebras
\end{itemize}
\end{crl}

\subsection{Blowing Ups}
\begin{defn}{Blowing Up at $\A^n$}{} Define the blowing up of $\A^n$ at the point $0$ to be the closed subset $X$ of $\A^n\times\Prj^{n-1}$ defined by the equations $\{x_iy_j=x_jy_i|0\leq i,j\leq n\}$. Restricting the projection $\A^n\times\Prj^{n-1}\to\A^n$ to the first factor gives a natural morphism $\phi:X\to\A^n$. 
\end{defn}

\begin{thm}{}{} The following are true with regards to blowing up at $\A^n$. 
\begin{itemize}
\item $X$ is a quasiprojective variety
\item $\phi$ is an isomorphism for the sets $X\setminus\phi^{-1}(0)$ and $\A^n\setminus\{0\}$
\item $\phi^{-1}(0)\cong\Prj^{n-1}$
\end{itemize}
\end{thm}

\begin{defn}{Blowing Up at a Point}{} Let $Y$ be a closed subvariety of $\A^n$ passing through $0$. Define the blowing up of $Y$ at $0$ to be the the closure of $Z=\phi^{-1}(Y\setminus\{0\})$, where $\phi:X\to\A^n$ is obtained from the above blowing up at $\A^n$. Also denote $\phi:\overline{Z}\to Z$ the morphism obtained by further restricting $\phi$ to $\overline{Z}$. \\~\\
To blow up any point other than $0$, perform a linear change in coordinates sending $P$ to $0$. 
\end{defn}

\begin{defn}{Blowup along an Ideal}{} Let $F_1,\dots,F_r$ be functions in the coordinate ring $\C[x]$ of an affine algebraic variety $X$, and let $I$ be the ideal they generate. Assume that $I$ is a proper nonzero ideal of $\C[x]$. The blowup of the variety $X$ along the ideal $I$ is the graph $B$ of the rational map $F:X\to\Prj^{r-1}$ defined by $$F(x)=[F_1(x):\dots:F_r(x)]$$ amd the natural projection $\pi:X\times\Prj^{r-1}\to X$. 
\end{defn}

\pagebreak
\section{Theory of Divisors}
\subsection{Divisors of a Variety}
\begin{defn}{Divisors of a Variety}{} Let $X$ be a variety. Let $C_1,\dots,C_r$ be irreducible closed subvarieties of $X$ of codimension $1$. A divisor of $X$ is of the form $$D=\sum_{i=1}^rk_iC_i$$ for $k_i\in\Z$. We say that $k_i$ is the multiplicity of $C_i$ in $D$. Define the free group of all divisors of $X$ by $$\text{Div}(X)=\Z\left\langle C\;|\;C\substack{\text{ is an irreducible closed}\\\text{subvariety of codimension }1}\right\rangle$$ Generators of $\text{Div}(X)$ are called prime divisors. 
\end{defn}

\begin{defn}{Effective Divisor}{} Let $X$ be a variety. We say that a divisor $$D=\sum_{i=1}^rk_iC_i$$ of $X$ is effective if $k_i\geq 0$ for all $i$. In this case we write $D>0$. 
\end{defn}

\begin{defn}{Divisor of a Function}{} Let $X$ be a variety such that the set of singular points of $X$ has codimension $\geq 2$. Let $f\in K(X)$. Let $C$ be a prime divisor of $X$. 
\end{defn}

\begin{defn}{Principal Divisors}{} Let $X$ be a variety. A divisor of the form $D=\text{div}(f)$ for some $f\in K(X)$ is called a principal divisor. \\~\\
Define the set of all principal divisors by $P(X)$. 
\end{defn}

\begin{prp}{}{} Let $X$ be a variety. The set of all principal divisors $P(X)$ is a group. 
\end{prp}

\begin{defn}{Divisor Class Group}{} Let $X$ be a variety. Define the divisor class group of $X$ to be $$\text{Cl}(X)=\frac{\text{Div}(X)}{P(X)}$$
We say that two divisors $D_1$ and $D_2$ are linearly equivalent if they lie in the same coset of $\text{Cl}(X)$, written as $D_1\sim D_2$. 
\end{defn}

\begin{defn}{Degree of a Divisor}{}
\end{defn}

\begin{prp}{}{} Let $X$ be a variety. Then $D$ is a principal divisor if and only if $\deg(D)=0$. 
\end{prp}

\subsection{The Linear System of a Divisor}
\begin{defn}{Associated Vector Space of a Divisor}{} Let $X$ be a nonsingular variety. Define the associated vector space of a divisor $D$ of $X$ to be $$\mathcal{L}(D)=\{f\in K(X)\;|\;\text{div}(f)+D\geq 0\}\cup\{0\}$$
\end{defn}

\begin{lmm}{}{} Let $X$ be a nonsingular variety. Then $\mathcal{L}(D)$ is a vector space over the field $k$. 
\end{lmm}

\begin{defn}{Dimension of the Associated Vector Space}{} Let $X$ be a nonsingular variety. Denote $\ell(D)$ the dimension of $\mathcal{L}(D)$, which is also called the dimension of $D$. 
\end{defn}

\begin{thm}{}{} Linearly equivalent divisors have the same dimension. 
\end{thm}

\pagebreak
\section{Intersection Theory}
















\end{document}
