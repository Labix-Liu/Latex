\documentclass[a4paper]{article}

%=========================================
% Packages
%=========================================
\usepackage{mathtools}
\usepackage{amsfonts}
\usepackage{amsmath}
\usepackage{amssymb}
\usepackage{amsthm}
\usepackage[a4paper, total={6in, 8in}, margin=1in]{geometry}
\usepackage[utf8]{inputenc}
\usepackage{fancyhdr}
\usepackage[utf8]{inputenc}
\usepackage{graphicx}
\usepackage{physics}
\usepackage[listings]{tcolorbox}
\usepackage{hyperref}
\usepackage{tikz-cd}
\usepackage{adjustbox}
\usepackage{enumitem}
\usepackage[font=small,labelfont=bf]{caption}
\usepackage{subcaption}
\usepackage{wrapfig}
\usepackage{makecell}



\raggedright

\usetikzlibrary{arrows.meta}

\DeclarePairedDelimiter\ceil{\lceil}{\rceil}
\DeclarePairedDelimiter\floor{\lfloor}{\rfloor}

%=========================================
% Fonts
%=========================================
\usepackage{tgpagella}
\usepackage[T1]{fontenc}


%=========================================
% Custom Math Operators
%=========================================
\DeclareMathOperator{\adj}{adj}
\DeclareMathOperator{\im}{im}
\DeclareMathOperator{\nullity}{nullity}
\DeclareMathOperator{\sign}{sign}
\DeclareMathOperator{\dom}{dom}
\DeclareMathOperator{\lcm}{lcm}
\DeclareMathOperator{\ran}{ran}
\DeclareMathOperator{\ext}{Ext}
\DeclareMathOperator{\dist}{dist}
\DeclareMathOperator{\diam}{diam}
\DeclareMathOperator{\aut}{Aut}
\DeclareMathOperator{\inn}{Inn}
\DeclareMathOperator{\syl}{Syl}
\DeclareMathOperator{\edo}{End}
\DeclareMathOperator{\cov}{Cov}
\DeclareMathOperator{\vari}{Var}
\DeclareMathOperator{\cha}{char}
\DeclareMathOperator{\Span}{span}
\DeclareMathOperator{\ord}{ord}
\DeclareMathOperator{\res}{res}
\DeclareMathOperator{\Hom}{Hom}
\DeclareMathOperator{\Mor}{Mor}
\DeclareMathOperator{\coker}{coker}
\DeclareMathOperator{\Obj}{Obj}
\DeclareMathOperator{\id}{id}
\DeclareMathOperator{\GL}{GL}
\DeclareMathOperator*{\colim}{colim}

%=========================================
% Custom Commands (Shortcuts)
%=========================================
\newcommand{\CP}{\mathbb{CP}}
\newcommand{\GG}{\mathbb{G}}
\newcommand{\F}{\mathbb{F}}
\newcommand{\N}{\mathbb{N}}
\newcommand{\Q}{\mathbb{Q}}
\newcommand{\R}{\mathbb{R}}
\newcommand{\C}{\mathbb{C}}
\newcommand{\E}{\mathbb{E}}
\newcommand{\Prj}{\mathbb{P}}
\newcommand{\RP}{\mathbb{RP}}
\newcommand{\T}{\mathbb{T}}
\newcommand{\Z}{\mathbb{Z}}
\newcommand{\A}{\mathbb{A}}
\renewcommand{\H}{\mathbb{H}}
\newcommand{\K}{\mathbb{K}}

\newcommand{\mA}{\mathcal{A}}
\newcommand{\mB}{\mathcal{B}}
\newcommand{\mC}{\mathcal{C}}
\newcommand{\mD}{\mathcal{D}}
\newcommand{\mE}{\mathcal{E}}
\newcommand{\mF}{\mathcal{F}}
\newcommand{\mG}{\mathcal{G}}
\newcommand{\mH}{\mathcal{H}}
\newcommand{\mI}{\mathcal{I}}
\newcommand{\mJ}{\mathcal{J}}
\newcommand{\mK}{\mathcal{K}}
\newcommand{\mL}{\mathcal{L}}
\newcommand{\mM}{\mathcal{M}}
\newcommand{\mO}{\mathcal{O}}
\newcommand{\mP}{\mathcal{P}}
\newcommand{\mS}{\mathcal{S}}
\newcommand{\mT}{\mathcal{T}}
\newcommand{\mV}{\mathcal{V}}
\newcommand{\mW}{\mathcal{W}}

%=========================================
% Colours!!!
%=========================================
\definecolor{LightBlue}{HTML}{2D64A6}
\definecolor{ForestGreen}{HTML}{4BA150}
\definecolor{DarkBlue}{HTML}{000080}
\definecolor{LightPurple}{HTML}{cc99ff}
\definecolor{LightOrange}{HTML}{ffc34d}
\definecolor{Buff}{HTML}{DDAE7E}
\definecolor{Sunset}{HTML}{F2C57C}
\definecolor{Wenge}{HTML}{584B53}
\definecolor{Coolgray}{HTML}{9098CB}
\definecolor{Lavender}{HTML}{D6E3F8}
\definecolor{Glaucous}{HTML}{828BC4}
\definecolor{Mauve}{HTML}{C7A8F0}
\definecolor{Darkred}{HTML}{880808}
\definecolor{Beaver}{HTML}{9A8873}
\definecolor{UltraViolet}{HTML}{52489C}



%=========================================
% Theorem Environment
%=========================================
\tcbuselibrary{listings, theorems, breakable, skins}

\newtcbtheorem[number within = subsection]{thm}{Theorem}%
{	colback=Buff!3, 
	colframe=Buff, 
	fonttitle=\bfseries, 
	breakable, 
	enhanced jigsaw, 
	halign=left
}{thm}

\newtcbtheorem[number within=subsection, use counter from=thm]{defn}{Definition}%
{  colback=cyan!1,
    colframe=cyan!50!black,
	fonttitle=\bfseries, breakable, 
	enhanced jigsaw, 
	halign=left
}{defn}

\newtcbtheorem[number within=subsection, use counter from=thm]{axm}{Axiom}%
{	colback=red!5, 
	colframe=Darkred, 
	fonttitle=\bfseries, 
	breakable, 
	enhanced jigsaw, 
	halign=left
}{axm}

\newtcbtheorem[number within=subsection, use counter from=thm]{prp}{Proposition}%
{	colback=LightBlue!3, 
	colframe=Glaucous, 
	fonttitle=\bfseries, 
	breakable, 
	enhanced jigsaw, 
	halign=left
}{prp}

\newtcbtheorem[number within=subsection, use counter from=thm]{lmm}{Lemma}%
{	colback=LightBlue!3, 
	colframe=LightBlue!60, 
	fonttitle=\bfseries, 
	breakable, 
	enhanced jigsaw, 
	halign=left
}{lmm}

\newtcbtheorem[number within=subsection, use counter from=thm]{crl}{Corollary}%
{	colback=LightBlue!3, 
	colframe=LightBlue!60, 
	fonttitle=\bfseries, 
	breakable, 
	enhanced jigsaw, 
	halign=left
}{crl}

\newtcbtheorem[number within=subsection, use counter from=thm]{eg}{Example}%
{	colback=Beaver!5, 
	colframe=Beaver, 
	fonttitle=\bfseries, 
	breakable, 
	enhanced jigsaw, 
	halign=left
}{eg}

\newtcbtheorem[number within=subsection, use counter from=thm]{ex}{Exercise}%
{	colback=Beaver!5, 
	colframe=Beaver, 
	fonttitle=\bfseries, 
	breakable, 
	enhanced jigsaw, 
	halign=left
}{ex}

\newtcbtheorem[number within=subsection, use counter from=thm]{alg}{Algorithm}%
{	colback=UltraViolet!5, 
	colframe=UltraViolet, 
	fonttitle=\bfseries, 
	breakable, 
	enhanced jigsaw, 
	halign=left
}{alg}




%=========================================
% Hyperlinks
%=========================================
\hypersetup{
    colorlinks=true, %set true if you want colored links
    linktoc=all,     %set to all if you want both sections and subsections linked
    linkcolor=DarkBlue,  %choose some color if you want links to stand out
}


\pagestyle{fancy}
\fancyhf{}
\rhead{Labix}
\lhead{Topics in (Co)Homology}
\rfoot{\thepage}

\title{Topics in (Co)Homology}

\author{Labix}

\date{\today}
\begin{document}
\maketitle
\begin{abstract}
\end{abstract}
\pagebreak
\tableofcontents

\pagebreak
\section{The Universal Coefficient Theorem for Homology}
\subsection{The Tor Functor}
\subsection{The Universal Coefficient Theorem}
\begin{thm}{}{} Let $C_\bullet$ be a chain complex of free abelian groups. Let $A$ be an abelian group. Then there exists a natural map $h:H_n(C_\bullet)\otimes A\to H_n(C_\bullet;A)$ such that $\text{coker}(h)\cong\text{Tor}_1^\Z(H_{n-1}(C_\bullet),A)$ and a split exact sequence (that is not natural) of the form \\~\\
\adjustbox{scale=1.0,center}{\begin{tikzcd}
	0 & {H_n(C_\bullet)\otimes A} & {H_n(C_\bullet;A)} & {\text{Tor}_1^\Z(H_{n-1}(C_\bullet),A)} & 0
	\arrow[from=1-1, to=1-2]
	\arrow["h", from=1-2, to=1-3]
	\arrow[from=1-3, to=1-4]
	\arrow[from=1-4, to=1-5]
\end{tikzcd}}\\~\\
for any $n\in\N$. In particular, split exactness implies that there is an isomorphism $$H_n(C_\bullet;A)\cong H_n(C_\bullet)\otimes A\oplus\text{Tor}_1^\Z(H_{n-1}(C_\bullet),A)$$ for any $n\in\N$. 
\end{thm}

\begin{crl}{}{} Let $(X,A)$ be a pair of space. Let $T$ be an abelian group. Then there exists a natural map $h:H_n(X,A)\otimes T\to H_n(X,A;T)$ such that $\text{coker}(h)\cong\text{Tor}_1^\Z(H_{n-1}(X,A),T)$ and a split exact sequence (that is not natural) of the form \\~\\
\adjustbox{scale=1.0,center}{\begin{tikzcd}
	0 & {H_n(X,A)\otimes T} & {H_n(X,A;T)} & {\text{Tor}_1^\Z(H_{n-1}(X,A),T)} & 0
	\arrow[from=1-1, to=1-2]
	\arrow["h", from=1-2, to=1-3]
	\arrow[from=1-3, to=1-4]
	\arrow[from=1-4, to=1-5]
\end{tikzcd}}\\~\\
for any $n\in\N$. In particular, split exactness implies that there is an isomorphism $$H_n(X,A;T)\cong H_n(X,A)\otimes T\oplus\text{Tor}_1^\Z(H_{n-1}(X,A),T)$$ for any $n\in\N$. 
\end{crl}

\subsection{The General Kunneth Theorem}
\begin{defn}{The Homological Cross Product}{}
\end{defn}

\begin{thm}{}{} Let $X$ and $Y$ be CW-complexes. Let $R$ be a principal ideal domain. Then there is a short exact sequence \\~\\
\adjustbox{scale=0.85,center}{\begin{tikzcd}
	0 & {\bigoplus_{i+j=n}H_i(X;R)\otimes_R H_j(Y;R)} & {H_n(X\times Y;R)} & {\bigoplus_{i+j=n}\text{Tor}_1^R(H_i(X;R),H_{j-1}(Y;R))} & 0
	\arrow[from=1-1, to=1-2]
	\arrow["\times", from=1-2, to=1-3]
	\arrow[from=1-3, to=1-4]
	\arrow[from=1-4, to=1-5]
\end{tikzcd}}\\~\\
induced by the cross product, that is natural in maps $f:X\to A$ and $g:Y\to B$. Moreover, this sequence splits. 
\end{thm}

\pagebreak
\section{The Cohomology of Some Topological Groups}

\pagebreak
\section{Cohomology Operations}

\pagebreak
\section{Spectral Sequences in Algebraic Topology}
\subsection{Spectral Sequences in Topology}
\begin{thm}{}{} Let $X$ be a space. Let the following be a sequence $$\emptyset\subset X_0\subset X_1\subset\cdots\subset X$$ of subspaces. Let $G$ be an abelian group. Then the following data
\begin{itemize}
\item $A_{p,q}=H_{p+q}(X_p;G)$
\item $E_{p,q}=H_{p+q}(X_p,X_{p-1};G)$
\item $i:H_{p+q}(X_p;G)=A_{p,q}\to H_{p+q}(X_{p+1};G)=A_{p+1,q-1}$ (degree $(1,-1)$)
\item $j:H_{p+q}(X_p;G)=A_{p,q}\to H_{p+q}(X_p,X_{p-1};G)=E_{p,q}$ (degree $(0,0)$)
\item $k:H_{p+q}(X_p,X_{p-1};G)=A_{p,q}\to H_{p+q-1}(X_{p-1};G)=A_{p-1,q}$ (degree $(-1,0)$)
\end{itemize}
defines an exact couple and hence a spectral sequence with $E^1$ page given by $$E_{p,q}^1=H_{p+q}(X_p,X_{p-1};G)$$ where the differential $d:E_{p,q}^1\to E_{p-1,q}^1$ is given by the composition $$H_{p+q}(X_p,X_{p-1};G)\overset{k}{\longrightarrow}H_{p+q-1}(X_{p-1};G)\overset{j}{\longrightarrow} H_{p+q-1}(X_{p-1},X_{p-2};G)$$
\end{thm}

The $E_1$ page of such a spectral sequence is given by \\~\\
\adjustbox{scale=1.0,center}{\begin{tikzcd}
	\cdots & \cdots & \cdots & \cdots \\
	{H_3(X_1,X_0;G)} & {H_4(X_2,X_1;G)} & {H_4(X_3,X_2;G)} & \cdots \\
	{H_2(X_1,X_0;G)} & {H_3(X_2,X_1;G)} & {H_4(X_3,X_2;G)} & \cdots \\
	{H_1(X_1,X_0;G)} & {H_2(X_2,X_1;G)} & {H_3(X_3,X_2;G)} & \cdots
	\arrow[from=1-2, to=1-1]
	\arrow[from=1-3, to=1-2]
	\arrow[from=1-4, to=1-3]
	\arrow[from=2-2, to=2-1]
	\arrow[from=2-3, to=2-2]
	\arrow[from=2-4, to=2-3]
	\arrow[from=3-2, to=3-1]
	\arrow[from=3-3, to=3-2]
	\arrow[from=3-4, to=3-3]
	\arrow[from=4-2, to=4-1]
	\arrow[from=4-3, to=4-2]
	\arrow[from=4-4, to=4-3]
\end{tikzcd}}\\~\\

Things get interesting when we choose $X$ to be a CW complex and we choose the filtration of $X$ by the skeleton of $X$. Recall that we have the formula $$H_{p+q}(X_p,X_{p-1};G)\cong\begin{cases}
C_p^\text{CW}(X;G) & \text{ if } q=0\\
0 & \text{otherwise}
\end{cases}$$ Thus the $E^1$ page is only left with a chain complex at $q=0$. 

Let us also compute the derived couple of this exact couple or in other words, the $E^2$ page of the spectral sequence. This is more intuitive then the one thinks about on the definition of the derived couple. The $E_{p,q}^2$ slot is simply the homology of the chain complex at the $(p,q)$th slot. The direction of the maps of the $E^2$ page depends not on the choice of spectral sequence at all (In fact, the direction only depends on the page). Now in our case, the homology can be given by a known construct: $$E_{p,q}^2=\frac{\ker(d:H_{p+q}(X_p,X_{p-1};G)\to H_{p+q-1}(X_{p-1},X_{p-2};G))}{\im(d:H_{p+q+1}(X_{p+1},X_p;G)\to H_{p+q}(X_p,X_{p-1};G))}=H_{p+q}^\text{CW}(X;G)$$ Since the direction of the maps are now diagonal and when $q\neq 0$ we have $E_{p,q}^2=0$, all maps in $E^2$ are $0$ and we are left with \\~\\
\adjustbox{scale=1.0,center}{\begin{tikzcd}
	{H_0^\text{CW}(X;G)} & {H_1^\text{CW}(X;G)} & {H_2^\text{CW}(X;G)} & {H_3^\text{CW}(X;G)} & \cdots
\end{tikzcd}}\\~\\

\begin{thm}{Leray-Serre Spectral Sequence}{} Let $p:E\to B$ be a Serre fibration with fibre $F$ and path connected $B$. Suppose that the action of $\pi_1(B)$ on $H_\ast(F;G)$ is trivial. Then there is a first quadrant homological spectral sequence starting with $E^2$ and weakly converging to $H_\bullet(E;\Z)$. Explicitly, there is a convergence $$E_{p,q}^2=H_p(B,H_1(F))\Rightarrow H_{p+q}(E;\Z)$$
\end{thm}

\subsection{Spectral Kunneth Theorem}

\end{document}