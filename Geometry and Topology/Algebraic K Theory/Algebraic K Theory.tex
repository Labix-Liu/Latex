\documentclass[a4paper]{article}

%=========================================
% Packages
%=========================================
\usepackage{mathtools}
\usepackage{amsfonts}
\usepackage{amsmath}
\usepackage{amssymb}
\usepackage{amsthm}
\usepackage[a4paper, total={6in, 8in}, margin=1in]{geometry}
\usepackage[utf8]{inputenc}
\usepackage{fancyhdr}
\usepackage[utf8]{inputenc}
\usepackage{graphicx}
\usepackage{physics}
\usepackage[listings]{tcolorbox}
\usepackage{hyperref}
\usepackage{tikz-cd}
\usepackage{adjustbox}
\usepackage{enumitem}
\usepackage[font=small,labelfont=bf]{caption}
\usepackage{subcaption}
\usepackage{wrapfig}
\usepackage{makecell}



\raggedright

\usetikzlibrary{arrows.meta}

\DeclarePairedDelimiter\ceil{\lceil}{\rceil}
\DeclarePairedDelimiter\floor{\lfloor}{\rfloor}

%=========================================
% Fonts
%=========================================
\usepackage{tgpagella}
\usepackage[T1]{fontenc}


%=========================================
% Custom Math Operators
%=========================================
\DeclareMathOperator{\adj}{adj}
\DeclareMathOperator{\im}{im}
\DeclareMathOperator{\nullity}{nullity}
\DeclareMathOperator{\sign}{sign}
\DeclareMathOperator{\dom}{dom}
\DeclareMathOperator{\lcm}{lcm}
\DeclareMathOperator{\ran}{ran}
\DeclareMathOperator{\ext}{Ext}
\DeclareMathOperator{\dist}{dist}
\DeclareMathOperator{\diam}{diam}
\DeclareMathOperator{\aut}{Aut}
\DeclareMathOperator{\inn}{Inn}
\DeclareMathOperator{\syl}{Syl}
\DeclareMathOperator{\edo}{End}
\DeclareMathOperator{\cov}{Cov}
\DeclareMathOperator{\vari}{Var}
\DeclareMathOperator{\cha}{char}
\DeclareMathOperator{\Span}{span}
\DeclareMathOperator{\ord}{ord}
\DeclareMathOperator{\res}{res}
\DeclareMathOperator{\Hom}{Hom}
\DeclareMathOperator{\Mor}{Mor}
\DeclareMathOperator{\coker}{coker}
\DeclareMathOperator{\Obj}{Obj}
\DeclareMathOperator{\id}{id}
\DeclareMathOperator{\GL}{GL}
\DeclareMathOperator*{\colim}{colim}

%=========================================
% Custom Commands (Shortcuts)
%=========================================
\newcommand{\CP}{\mathbb{CP}}
\newcommand{\GG}{\mathbb{G}}
\newcommand{\F}{\mathbb{F}}
\newcommand{\N}{\mathbb{N}}
\newcommand{\Q}{\mathbb{Q}}
\newcommand{\R}{\mathbb{R}}
\newcommand{\C}{\mathbb{C}}
\newcommand{\E}{\mathbb{E}}
\newcommand{\Prj}{\mathbb{P}}
\newcommand{\RP}{\mathbb{RP}}
\newcommand{\T}{\mathbb{T}}
\newcommand{\Z}{\mathbb{Z}}
\newcommand{\A}{\mathbb{A}}
\renewcommand{\H}{\mathbb{H}}
\newcommand{\K}{\mathbb{K}}

\newcommand{\mA}{\mathcal{A}}
\newcommand{\mB}{\mathcal{B}}
\newcommand{\mC}{\mathcal{C}}
\newcommand{\mD}{\mathcal{D}}
\newcommand{\mE}{\mathcal{E}}
\newcommand{\mF}{\mathcal{F}}
\newcommand{\mG}{\mathcal{G}}
\newcommand{\mH}{\mathcal{H}}
\newcommand{\mI}{\mathcal{I}}
\newcommand{\mJ}{\mathcal{J}}
\newcommand{\mK}{\mathcal{K}}
\newcommand{\mL}{\mathcal{L}}
\newcommand{\mM}{\mathcal{M}}
\newcommand{\mO}{\mathcal{O}}
\newcommand{\mP}{\mathcal{P}}
\newcommand{\mS}{\mathcal{S}}
\newcommand{\mT}{\mathcal{T}}
\newcommand{\mV}{\mathcal{V}}
\newcommand{\mW}{\mathcal{W}}

%=========================================
% Colours!!!
%=========================================
\definecolor{LightBlue}{HTML}{2D64A6}
\definecolor{ForestGreen}{HTML}{4BA150}
\definecolor{DarkBlue}{HTML}{000080}
\definecolor{LightPurple}{HTML}{cc99ff}
\definecolor{LightOrange}{HTML}{ffc34d}
\definecolor{Buff}{HTML}{DDAE7E}
\definecolor{Sunset}{HTML}{F2C57C}
\definecolor{Wenge}{HTML}{584B53}
\definecolor{Coolgray}{HTML}{9098CB}
\definecolor{Lavender}{HTML}{D6E3F8}
\definecolor{Glaucous}{HTML}{828BC4}
\definecolor{Mauve}{HTML}{C7A8F0}
\definecolor{Darkred}{HTML}{880808}
\definecolor{Beaver}{HTML}{9A8873}
\definecolor{UltraViolet}{HTML}{52489C}



%=========================================
% Theorem Environment
%=========================================
\tcbuselibrary{listings, theorems, breakable, skins}

\newtcbtheorem[number within = subsection]{thm}{Theorem}%
{	colback=Buff!3, 
	colframe=Buff, 
	fonttitle=\bfseries, 
	breakable, 
	enhanced jigsaw, 
	halign=left
}{thm}

\newtcbtheorem[number within=subsection, use counter from=thm]{defn}{Definition}%
{  colback=cyan!1,
    colframe=cyan!50!black,
	fonttitle=\bfseries, breakable, 
	enhanced jigsaw, 
	halign=left
}{defn}

\newtcbtheorem[number within=subsection, use counter from=thm]{axm}{Axiom}%
{	colback=red!5, 
	colframe=Darkred, 
	fonttitle=\bfseries, 
	breakable, 
	enhanced jigsaw, 
	halign=left
}{axm}

\newtcbtheorem[number within=subsection, use counter from=thm]{prp}{Proposition}%
{	colback=LightBlue!3, 
	colframe=Glaucous, 
	fonttitle=\bfseries, 
	breakable, 
	enhanced jigsaw, 
	halign=left
}{prp}

\newtcbtheorem[number within=subsection, use counter from=thm]{lmm}{Lemma}%
{	colback=LightBlue!3, 
	colframe=LightBlue!60, 
	fonttitle=\bfseries, 
	breakable, 
	enhanced jigsaw, 
	halign=left
}{lmm}

\newtcbtheorem[number within=subsection, use counter from=thm]{crl}{Corollary}%
{	colback=LightBlue!3, 
	colframe=LightBlue!60, 
	fonttitle=\bfseries, 
	breakable, 
	enhanced jigsaw, 
	halign=left
}{crl}

\newtcbtheorem[number within=subsection, use counter from=thm]{eg}{Example}%
{	colback=Beaver!5, 
	colframe=Beaver, 
	fonttitle=\bfseries, 
	breakable, 
	enhanced jigsaw, 
	halign=left
}{eg}

\newtcbtheorem[number within=subsection, use counter from=thm]{ex}{Exercise}%
{	colback=Beaver!5, 
	colframe=Beaver, 
	fonttitle=\bfseries, 
	breakable, 
	enhanced jigsaw, 
	halign=left
}{ex}

\newtcbtheorem[number within=subsection, use counter from=thm]{alg}{Algorithm}%
{	colback=UltraViolet!5, 
	colframe=UltraViolet, 
	fonttitle=\bfseries, 
	breakable, 
	enhanced jigsaw, 
	halign=left
}{alg}




%=========================================
% Hyperlinks
%=========================================
\hypersetup{
    colorlinks=true, %set true if you want colored links
    linktoc=all,     %set to all if you want both sections and subsections linked
    linkcolor=DarkBlue,  %choose some color if you want links to stand out
}


\pagestyle{fancy}
\fancyhf{}
\rhead{Labix}
\lhead{Algebraic K Theory}
\rfoot{\thepage}

\title{Algebraic K Theory}

\author{Labix}

\date{\today}
\begin{document}
\maketitle
\begin{abstract}
\end{abstract}
\pagebreak
\tableofcontents

\pagebreak
\section{The K${_0}$-Group}
\subsection{K${_0}$ of a Symmetric Monoidal Category}
\begin{defn}{The K${_0}$-Group of a Symmetric Monoidal Category}{} Let $(\mC,I,\oplus)$ be a symmetric monoidal category. Let $\mC^\text{iso}$ be the category consisting of isomorphism classes of objects, which is also an abelian monoid under the operation $\oplus$. Define the $K_0$ group of $\mC$ by the Grothendieck completion $$K_0(\mC,I,\oplus)=\left(\mC^\text{iso}\right)^{-1}\mC^\text{iso}$$
\end{defn}

\subsection{K${_0}$ of a Ring}
\begin{defn}{The Category of Finitely Generated Projective Modules over a Ring}{} Let $R$ be a ring. Define the category $\bold{FGP}(R)$ of projective modules over $R$ as follows. 
\begin{itemize}
\item The objects are the isomorphism classes of finitely generated projective modules $[M]$ over $R$
\item For two isomorphism classes of projective modules $[M],[N]$ over $R$, a morphism $[M]\to [N]$ is just an $R$-module homomorphism. 
\item Composition is given by the composition of functions. 
\end{itemize}
\end{defn}

\begin{lmm}{}{} Let $R$ be a ring. Then the category $\bold{FGP}(R)$ is a symmetric monoidal category with the following data. 
\begin{itemize}
\item The binary operator $\oplus:P(R)\times P(R)\to P(R)$ is given by $$[M]+[N]=[M\oplus N]$$ which is the direct sum. 
\item The unital object is the isomorphism class $[R]$ of the $R$-module $R$. 
\end{itemize}
\end{lmm}

\begin{defn}{The K${_0}$-Group of a Ring}{} Let $R$ be a ring. Define the $K_0$-group of $R$ by the Grothendieck completion of the abelain monoid: $$K_0(R)=P(R)^{-1}P(R)=K_0(P(R),R,\oplus)$$
\end{defn}

\begin{defn}{The K${_0}$ Functor}{} Define the $K_0$-functor $$K_0:\bold{Ring}\to\bold{Grp}$$ to consist of the following data. 
\begin{itemize}
\item For each ring $R$, define $K_0(R)$ to be the $K_0$-group of $R$
\item For each ring homomorphism $f:R\to S$, define $K_0(f):K_0(R)\to K_0(S)$ by the formula $$[P]\mapsto[S\otimes_RP]$$
\end{itemize}
\end{defn}

\begin{thm}{Universal Property of the K${_0}$-Group}{} Let $R$ be a ring. 
\end{thm}

Recall that a ring $R$ is a principal ideal domain if every ideal of $R$ is generated by one element. By the structure theorem of finitely generated $R$-modules over PID, we can immediately conclude that there is an isomorphism $$\Z\cong K_0(R)$$ given by $n\mapsto[R^n]$. 

\begin{defn}{Reduced K${_0}$-Group}{} Let $R$ be a ring. Define the reduced $K_0$-group $\widetilde{K}_0(R)$ of $R$ to be the quotient $$\widetilde{K}_0(R)=\frac{K_0(R)}{\{[R^m]-[R^n]\;|\;n,m\in\N\}}$$
\end{defn}

\begin{lmm}{}{} Let $R$ be a ring. Then the unique ring homomorphism $f:\Z\to R$ induces an isomorphism $$\widetilde{K_0}(R)\cong\frac{K_0(R)}{\im(K_0(f))}$$
\end{lmm}

Recall that a stably free module is an $R$-module $M$ such that there exists a finitely generated free $R$-module $T$ such that $M\oplus T$ is free. Now $[P]\in\widetilde{K}_0(R)$ is trivial if and only if $P$ is stably free and finitely generated. Thus the reduced $K_0$ of a ring measures how far away a finitely generated $R$-module from also being stably free. 

\begin{thm}{}{} Let $R$ be a ring. Let $n\geq 1$. Then there is an isomorphism $$\mu_R:K_0(R)\overset{\cong}{\longrightarrow}K_0(M_n(R))$$ given by $[P]\mapsto[R^n\oplus_RP]$, where $R^n$ here is considered as an $(M_n(R),R)$-bimodule. Moreover, the isomorphism is natural in the following sense. If $f:R\to S$ is a ring homomorphism, then the following diagram is commutative: \\~\\
\adjustbox{scale=1.0,center}{\begin{tikzcd}
	{K_0(R)} & {K_0(S)} \\
	{K_0(M_n(R))} & {K_0(M_n(S))}
	\arrow["{K_0(f)}", from=1-1, to=1-2]
	\arrow["{\mu_R}"', from=1-1, to=2-1]
	\arrow["{\mu_S}", from=1-2, to=2-2]
	\arrow["{K_0(M_n(f))}"', from=2-1, to=2-2]
\end{tikzcd}}\\~\\
\end{thm}

\begin{prp}{}{} Let $R,S$ be rings. Denote $p_1:R\times S\to R$ and $p_2:R\times S\to S$ the projection maps. Then the projection maps induce an isomorphism $$K_0(p_1)\times K_0(p_2):K_0(R\times S)\overset{\cong}{\longrightarrow}K_0(R)\times K_0(S)$$
\end{prp}

\begin{prp}{}{} Let $k$ be a field. Let $V$ be a vector space over $k$ with countable basis. Then $$K_0(\text{End}_k(V))\cong\{1\}$$
\end{prp}

\begin{lmm}{}{} Let $G$ be a group. Let $R$ be a commutative integral domain with quotient field $F$. Then there is an isomorphism $$K_0(R[G])\cong\widetilde{K}_0(R[G])\oplus\Z$$ given by $[P]\mapsto([P],\dim_F(F\otimes_{R[G]}P))$
\end{lmm}

\begin{cjt}{Farrell-Jones Conjecture}{} Let $G$ be a torsion-free group. Let $R$ be a regular ring. Then the map $\{1\}\hookrightarrow G$ induces an isomorphism $$K_0(R)\cong K_0(R[G])$$
\end{cjt}

\subsection{K${_0}$ of an Abelian Category}
\subsection{K${_0}$ of a Waldhaussen Category}

\pagebreak
\section{The K${_1}$-Group}
\subsection{K${_1}$ of a Ring}
\begin{defn}{The K${_1}$-Group of a Ring}{} Let $R$ be a ring. Define the $K_1$-group of $R$ to be the group $$K_1(R)=\frac{GL(R)}{[GL(R),GL(R)]}$$
\end{defn}

\begin{prp}{}{} Let $R$ and $S$ be two rings. Then there is an isomorphism $$K_1(R\times S)\cong K_1(R)\oplus K_1(S)$$
\end{prp}

\begin{prp}{}{} Let $R$ be a ring. Then there is an isomorphism $$K_1(R)\cong K_1(M_n(R))$$ for any $n\in\N$. 
\end{prp}

\subsection{The Fundamental Theorems for K${_1}$ and K${_0}$}

\pagebreak
\section{The Negative K-Groups}

\pagebreak
\section{The K${_2}$-Group}
\subsection{The Steinberg Group}
\begin{defn}{The $n$th Steinberg Group}{} Let $R$ be a ring. For $n\geq 3$, define the $n$th Steinberg group by $$\text{St}_n(R)=\frac{\langle x_{ij}(r)\text{ for }r\in R, 1\leq i,j\leq n\rangle}{R}$$ where $R$ is the relation generated by 
\begin{itemize}
\item For $r,s\in R$, $x_{ij}(r)x_{ij}(s)=x_{ij}(rs)$ for $1\leq i,j\leq n$
\item For $r,s\in R$, $$[x_{ij}(r),x_{kl}(s)]=\begin{cases}
1 & \text{ if } j\neq k\text{ and }i\neq l\\
x_{il}(rs) & \text{ if } j=k \text{ and }i\neq l\\
x_{kj}(-rs) & \text{ if } j\neq k\text{ and }i=l
\end{cases}$$
\end{itemize}
\end{defn}

\begin{lmm}{}{} Let $R$ be a ring. For any $n\geq 3$, the $n$th Steinberg group $\text{St}_n(R)$ of $R$ includes into the $(n+1)$th Steinberg group $\text{St}_{n+1}(R)$. 
\end{lmm}

\begin{prp}{}{} Let $R$ be a ring. Let $n\geq 3$. Then the universal property of free groups with relations induce a canonical group surjection $$\phi_n:\text{St}_n(R)\to[GL(R),GL(R)]$$ that sends $x_{ij}(r)$ to $e_{ij}(r)$. 
\end{prp}

\begin{defn}{The Steinberg Group of a Ring}{} Let $R$ be a ring. Define the Steinberg group of $R$ by the direct limit $$\text{St}(R)=\varinjlim_{n\in\N\setminus\{0,1,2\}}\text{St}_n(R)$$
\end{defn}

\begin{prp}{}{} Let $R$ be a ring. The universal property of the direct limit induces a canonical group surjection $$\phi:\text{St}(R)\to[GL(R),GL(R)]$$
\end{prp}

\subsection{K${_2}$ of a Ring}
\begin{defn}{The K${_2}$-Group of a Ring}{} Let $R$ be a ring. Define the $K_2$-group of $R$ to be the kernel $$K_2(R)=\ker\left(\phi:\text{St}(R)\to[GL(R),GL(R)]\right)$$
\end{defn}

\begin{lmm}{}{} Let $R$ be a ring. Then there is an exact sequence of groups \\~\\
\adjustbox{scale=1.0,center}{\begin{tikzcd}
	0 & {K_2(R)} & {\text{St}(R)} & {[GL(R),GL(R)]} & {K_1(R)} & 0
	\arrow[from=1-1, to=1-2]
	\arrow[from=1-2, to=1-3]
	\arrow[from=1-3, to=1-4]
	\arrow[from=1-4, to=1-5]
	\arrow[from=1-5, to=1-6]
\end{tikzcd}}\\~\\
\end{lmm}

\begin{thm}{(Stein)}{} For any ring $R$, the $K_2$-group $K_2(R)$ is an abelian group. Moreover, we have $$Z(\text{St}(R))=K_2(R)$$
\end{thm}

\pagebreak
\section{The K${_n}$-Group}
\subsection{Universal Definition}
\begin{defn}{The Plus Construction}{} Let $R$ be a ring. Define $BGL(R)^+$ to be any CW complex that has a distinguished map $BGL(R)\to BGL(R)^+$ such that the following are true. 
\begin{itemize}
\item There is an isomorphism $\pi_1(BGL(R)^+)\cong K_1(R)$ given by the induced map $\pi_1(BGL(R))\to\pi_1(BGL(R)^+)$, which is required to be surjective with kernel $[GL(R),GL(R)]$
\item For each $n\in\N$, there are isomorphisms $$H_n(BGL(R);M)\cong H_n(BGL(R)^+;M)$$ for any $R$-module $M$. 
\end{itemize}
\end{defn}

Intuitively, $BGL(R)^+$ is a modification of the classifying space of $GL(R)$ so that their homology remains the same while its fundamental group returns $K_1(R)$. The latter point is important because $K_n$ will be defined as the $n$th homotopy group. 

\begin{defn}{K${_n}$ of a Ring}{} Let $R$ be a ring. Define the $n$th $K$-group of $R$ to be $$K_n(R)=\pi_n(BGL(R)^+)$$ for $n\geq 1$. 
\end{defn}

Notice that $BGL(R)^+$ for a ring $R$ is not defined uniquely. However, we can prove that any two such plus constructions are homotopy equivalent so that $K_n(R)$ is well defined. \\~\\

In order to accommodate the $0$th $K$-group, we make the following amendments. 

\begin{defn}{K-Theory of a Ring}{} Let $R$ be a ring. Define the $K$-theory of $R$ by $$K(R)=K_0(R)\times BGL(R)^+$$ so that $\pi_n(BGL(R)^+)=K_n(R)$ for all $n\geq 0$. 
\end{defn}

\section{}
\subsection{}
\begin{thm}{Serre-Swan Theorem I}{} Let $M$ be a smooth manifold. Let $E$ be a smooth vector bundle over $M$. Then the space of smooth sections $\Gamma(E)$ of $E$ is finitely generated and projective over $C^\infty(M)$. \\~\\

If $M$ is connected, then the space of smooth section is one-to-one with the finitely generated and projective modules over $C^\infty(M)$. 
\end{thm}

\begin{thm}{}{} Let $M$ be a smooth and connected manifold. Then the category of smooth vector bundles $\text{SVect}(M)$ is equivalent to the category of finitely generated projective modules $\text{FinProj}{_{C^\infty(M)}\text{Mod}}$ via the global section functor $$\Gamma:\text{SVect}(M)\to\text{FinProj}{_{C^\infty(M)}\text{Mod}}$$ defined by $E\mapsto\Gamma(E)$
\end{thm}

\subsection{}










\end{document}
