\documentclass[a4paper]{article}

%=========================================
% Packages
%=========================================
\usepackage{mathtools}
\usepackage{amsfonts}
\usepackage{amsmath}
\usepackage{amssymb}
\usepackage{amsthm}
\usepackage[a4paper, total={6in, 8in}, margin=1in]{geometry}
\usepackage[utf8]{inputenc}
\usepackage{fancyhdr}
\usepackage[utf8]{inputenc}
\usepackage{graphicx}
\usepackage{physics}
\usepackage[listings]{tcolorbox}
\usepackage{hyperref}
\usepackage{tikz-cd}
\usepackage{adjustbox}
\usepackage{enumitem}
\usepackage[font=small,labelfont=bf]{caption}
\usepackage{subcaption}
\usepackage{wrapfig}
\usepackage{makecell}



\raggedright

\usetikzlibrary{arrows.meta}

\DeclarePairedDelimiter\ceil{\lceil}{\rceil}
\DeclarePairedDelimiter\floor{\lfloor}{\rfloor}

%=========================================
% Fonts
%=========================================
\usepackage{tgpagella}
\usepackage[T1]{fontenc}


%=========================================
% Custom Math Operators
%=========================================
\DeclareMathOperator{\adj}{adj}
\DeclareMathOperator{\im}{im}
\DeclareMathOperator{\nullity}{nullity}
\DeclareMathOperator{\sign}{sign}
\DeclareMathOperator{\dom}{dom}
\DeclareMathOperator{\lcm}{lcm}
\DeclareMathOperator{\ran}{ran}
\DeclareMathOperator{\ext}{Ext}
\DeclareMathOperator{\dist}{dist}
\DeclareMathOperator{\diam}{diam}
\DeclareMathOperator{\aut}{Aut}
\DeclareMathOperator{\inn}{Inn}
\DeclareMathOperator{\syl}{Syl}
\DeclareMathOperator{\edo}{End}
\DeclareMathOperator{\cov}{Cov}
\DeclareMathOperator{\vari}{Var}
\DeclareMathOperator{\cha}{char}
\DeclareMathOperator{\Span}{span}
\DeclareMathOperator{\ord}{ord}
\DeclareMathOperator{\res}{res}
\DeclareMathOperator{\Hom}{Hom}
\DeclareMathOperator{\Mor}{Mor}
\DeclareMathOperator{\coker}{coker}
\DeclareMathOperator{\Obj}{Obj}
\DeclareMathOperator{\id}{id}
\DeclareMathOperator{\GL}{GL}
\DeclareMathOperator*{\colim}{colim}

%=========================================
% Custom Commands (Shortcuts)
%=========================================
\newcommand{\CP}{\mathbb{CP}}
\newcommand{\GG}{\mathbb{G}}
\newcommand{\F}{\mathbb{F}}
\newcommand{\N}{\mathbb{N}}
\newcommand{\Q}{\mathbb{Q}}
\newcommand{\R}{\mathbb{R}}
\newcommand{\C}{\mathbb{C}}
\newcommand{\E}{\mathbb{E}}
\newcommand{\Prj}{\mathbb{P}}
\newcommand{\RP}{\mathbb{RP}}
\newcommand{\T}{\mathbb{T}}
\newcommand{\Z}{\mathbb{Z}}
\newcommand{\A}{\mathbb{A}}
\renewcommand{\H}{\mathbb{H}}
\newcommand{\K}{\mathbb{K}}

\newcommand{\mA}{\mathcal{A}}
\newcommand{\mB}{\mathcal{B}}
\newcommand{\mC}{\mathcal{C}}
\newcommand{\mD}{\mathcal{D}}
\newcommand{\mE}{\mathcal{E}}
\newcommand{\mF}{\mathcal{F}}
\newcommand{\mG}{\mathcal{G}}
\newcommand{\mH}{\mathcal{H}}
\newcommand{\mI}{\mathcal{I}}
\newcommand{\mJ}{\mathcal{J}}
\newcommand{\mK}{\mathcal{K}}
\newcommand{\mL}{\mathcal{L}}
\newcommand{\mM}{\mathcal{M}}
\newcommand{\mO}{\mathcal{O}}
\newcommand{\mP}{\mathcal{P}}
\newcommand{\mS}{\mathcal{S}}
\newcommand{\mT}{\mathcal{T}}
\newcommand{\mV}{\mathcal{V}}
\newcommand{\mW}{\mathcal{W}}

%=========================================
% Colours!!!
%=========================================
\definecolor{LightBlue}{HTML}{2D64A6}
\definecolor{ForestGreen}{HTML}{4BA150}
\definecolor{DarkBlue}{HTML}{000080}
\definecolor{LightPurple}{HTML}{cc99ff}
\definecolor{LightOrange}{HTML}{ffc34d}
\definecolor{Buff}{HTML}{DDAE7E}
\definecolor{Sunset}{HTML}{F2C57C}
\definecolor{Wenge}{HTML}{584B53}
\definecolor{Coolgray}{HTML}{9098CB}
\definecolor{Lavender}{HTML}{D6E3F8}
\definecolor{Glaucous}{HTML}{828BC4}
\definecolor{Mauve}{HTML}{C7A8F0}
\definecolor{Darkred}{HTML}{880808}
\definecolor{Beaver}{HTML}{9A8873}
\definecolor{UltraViolet}{HTML}{52489C}



%=========================================
% Theorem Environment
%=========================================
\tcbuselibrary{listings, theorems, breakable, skins}

\newtcbtheorem[number within = subsection]{thm}{Theorem}%
{	colback=Buff!3, 
	colframe=Buff, 
	fonttitle=\bfseries, 
	breakable, 
	enhanced jigsaw, 
	halign=left
}{thm}

\newtcbtheorem[number within=subsection, use counter from=thm]{defn}{Definition}%
{  colback=cyan!1,
    colframe=cyan!50!black,
	fonttitle=\bfseries, breakable, 
	enhanced jigsaw, 
	halign=left
}{defn}

\newtcbtheorem[number within=subsection, use counter from=thm]{axm}{Axiom}%
{	colback=red!5, 
	colframe=Darkred, 
	fonttitle=\bfseries, 
	breakable, 
	enhanced jigsaw, 
	halign=left
}{axm}

\newtcbtheorem[number within=subsection, use counter from=thm]{prp}{Proposition}%
{	colback=LightBlue!3, 
	colframe=Glaucous, 
	fonttitle=\bfseries, 
	breakable, 
	enhanced jigsaw, 
	halign=left
}{prp}

\newtcbtheorem[number within=subsection, use counter from=thm]{lmm}{Lemma}%
{	colback=LightBlue!3, 
	colframe=LightBlue!60, 
	fonttitle=\bfseries, 
	breakable, 
	enhanced jigsaw, 
	halign=left
}{lmm}

\newtcbtheorem[number within=subsection, use counter from=thm]{crl}{Corollary}%
{	colback=LightBlue!3, 
	colframe=LightBlue!60, 
	fonttitle=\bfseries, 
	breakable, 
	enhanced jigsaw, 
	halign=left
}{crl}

\newtcbtheorem[number within=subsection, use counter from=thm]{eg}{Example}%
{	colback=Beaver!5, 
	colframe=Beaver, 
	fonttitle=\bfseries, 
	breakable, 
	enhanced jigsaw, 
	halign=left
}{eg}

\newtcbtheorem[number within=subsection, use counter from=thm]{ex}{Exercise}%
{	colback=Beaver!5, 
	colframe=Beaver, 
	fonttitle=\bfseries, 
	breakable, 
	enhanced jigsaw, 
	halign=left
}{ex}

\newtcbtheorem[number within=subsection, use counter from=thm]{alg}{Algorithm}%
{	colback=UltraViolet!5, 
	colframe=UltraViolet, 
	fonttitle=\bfseries, 
	breakable, 
	enhanced jigsaw, 
	halign=left
}{alg}




%=========================================
% Hyperlinks
%=========================================
\hypersetup{
    colorlinks=true, %set true if you want colored links
    linktoc=all,     %set to all if you want both sections and subsections linked
    linkcolor=DarkBlue,  %choose some color if you want links to stand out
}


\pagestyle{fancy}
\fancyhf{}
\rhead{Labix}
\lhead{Higher Category Theory}
\rfoot{\thepage}

\title{Higher Category Theory}

\author{Labix}

\date{\today}
\begin{document}
\maketitle
\begin{abstract}
\begin{itemize}
\end{itemize}
\end{abstract}
\pagebreak
\tableofcontents

\pagebreak
\section{Introduction to Infinity Categories}
\subsection{Infinity Categories and Some Examples}
The foundations of infinity categories lay on the simpicial sets. Intuitively, any face $\partial_k\Delta$ of an $n$-simplex $\Delta$ captures a homotopy of the faces of $\partial_k\Delta$. 

\begin{defn}{Infinity Categories}{} An infinity category is a simplicial set $C$ such that each inner horn admits a filler. In other words, for all $0<i<n$, the following diagram commutes: \\~\\
\adjustbox{scale=1.0,center}{\begin{tikzcd}
	{\Lambda_i^n} & C \\
	{\Delta^n}
	\arrow["\forall", from=1-1, to=1-2]
	\arrow[hook, from=1-1, to=2-1]
	\arrow["\exists"', dashed, from=2-1, to=1-2]
\end{tikzcd}}\\~\\
\end{defn}

\begin{thm}{}{} Let $\mC$ be a category. Every inner horn of the nerve $N(C)$ of $\mC$ admits a filler and hence is an infinity category. 
\end{thm}

\subsection{Homotopy Infinity Categories}
Recall that for a simplicial set $X$, we defined the homotopy category $h(X)$ of $X$. Such an assignment is functorial. In the case of infinity categories, we can exhibit the structure of $h(X)$ more explicitly. 

\begin{defn}{Homotopic Morphisms}{} Let $\mC$ be an infinity category. Two morphisms $f,g:C\to D$ are said to be homotopic if there exists a $2$-simplex $\sigma$ such that 
\begin{itemize}
\item $d_0(\sigma)=\text{id}_D$
\item $d_1(\sigma)=g$
\item $d_2(\sigma)=f$
\end{itemize}
In this case we write $f\simeq g$. 
\end{defn}

\begin{lmm}{}{} Homotopy is an equivalence relation in any infinity category. 
\end{lmm}

\begin{prp}{}{} Let $C$ be an infinity category. Let $f,f':C\to D$ and $g,g':D\to E$ be morphisms in $C$. If $f\simeq f'$ and $g\simeq g'$, then $$g\circ f\simeq g'\circ f'$$
\end{prp}

\begin{defn}{Homotopy Category}{} Let $C$ be an infinity category. Define the homotopy category $h(C)$ of $C$ to consist of the following. 
\begin{itemize}
\item The objects are the objects of $C$
\item The morphisms are equivalent classes of morphisms $[f]$ for $f$ a morphism in $C$
\item Composition is defined by $$[g]\circ[f]=[g\circ f]$$ which is well defined by the above. 
\end{itemize}
\end{defn}

\begin{defn}{Isomorphisms in Infinity Categories}{} Let $C$ be an infinity category. Let $f:C\to D$ be a morphism. We say that $f$ is an isomorphism if there exists $g:D\to C$ such that $g\circ f\simeq\text{id}_C$ and $f\circ g\simeq\text{id}_D$. 
\end{defn}

\begin{lmm}{}{} Let $C$ be an infinity category. Let $f:C\to D$ be a morphism. Then $f$ is an isomorphism in $C$ if and only if $[f]$ is an isomorphism in $h(C)$. 
\end{lmm}

\pagebreak
\section{Infinity Categories in Topology}
\begin{lmm}{}{} Let $X$ be a space. Then applying the singular functor $S(X)$ gives an infinity category. 
\end{lmm}

\begin{prp}{}{} Let $X$ be a space. Then the homotopy category of the singular set of $X$ is equal to $h(S(X))=\prod_1(X)$ the fundamental groupoid of $X$. 
\end{prp}

\subsection{Kan Complexes}
\begin{defn}{Kan Complexes}{} A Kan complex is a simplicial set $C$ such that each horn (inner and outer) admits a filler. In other words, for all $0\leq i\leq n$, the following diagram commutes: \\~\\
\adjustbox{scale=1.0,center}{\begin{tikzcd}
	{\Lambda_i^n} & C \\
	{\Delta^n}
	\arrow["\forall", from=1-1, to=1-2]
	\arrow[hook, from=1-1, to=2-1]
	\arrow["\exists"', dashed, from=2-1, to=1-2]
\end{tikzcd}}\\~\\
\end{defn}

Since infinity catregories require only inner horns to admit a filler, we have the following inclusion relation: $$\substack{\text{Infinity}\\\text{Categories}}\subset\substack{\text{Kan}\\\text{Complexes}}$$

\begin{prp}{}{} Let $X$ be a space. Then $S(X)$ is a Kan complex. 
\end{prp}

\begin{thm}{}{} Let $\mC$ be a small category. Then the simplicial set $N(\mC)$ is a Kan complex if and only if $\mC$ is a groupoid. 
\end{thm}

More: Kan complexes = infinity groupoids (quillen equivalence in model category), and we should think of spaces as Kan complexes / infinity groupoids from now on. 




\end{document}
