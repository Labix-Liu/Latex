\documentclass[a4paper]{article}

%=========================================
% Packages
%=========================================
\usepackage{mathtools}
\usepackage{amsfonts}
\usepackage{amsmath}
\usepackage{amssymb}
\usepackage{amsthm}
\usepackage[a4paper, total={6in, 8in}, margin=1in]{geometry}
\usepackage[utf8]{inputenc}
\usepackage{fancyhdr}
\usepackage[utf8]{inputenc}
\usepackage{graphicx}
\usepackage{physics}
\usepackage[listings]{tcolorbox}
\usepackage{hyperref}
\usepackage{tikz-cd}
\usepackage{adjustbox}
\usepackage{enumitem}
\usepackage[font=small,labelfont=bf]{caption}
\usepackage{subcaption}
\usepackage{wrapfig}
\usepackage{makecell}



\raggedright

\usetikzlibrary{arrows.meta}

\DeclarePairedDelimiter\ceil{\lceil}{\rceil}
\DeclarePairedDelimiter\floor{\lfloor}{\rfloor}

%=========================================
% Fonts
%=========================================
\usepackage{tgpagella}
\usepackage[T1]{fontenc}


%=========================================
% Custom Math Operators
%=========================================
\DeclareMathOperator{\adj}{adj}
\DeclareMathOperator{\im}{im}
\DeclareMathOperator{\nullity}{nullity}
\DeclareMathOperator{\sign}{sign}
\DeclareMathOperator{\dom}{dom}
\DeclareMathOperator{\lcm}{lcm}
\DeclareMathOperator{\ran}{ran}
\DeclareMathOperator{\ext}{Ext}
\DeclareMathOperator{\dist}{dist}
\DeclareMathOperator{\diam}{diam}
\DeclareMathOperator{\aut}{Aut}
\DeclareMathOperator{\inn}{Inn}
\DeclareMathOperator{\syl}{Syl}
\DeclareMathOperator{\edo}{End}
\DeclareMathOperator{\cov}{Cov}
\DeclareMathOperator{\vari}{Var}
\DeclareMathOperator{\cha}{char}
\DeclareMathOperator{\Span}{span}
\DeclareMathOperator{\ord}{ord}
\DeclareMathOperator{\res}{res}
\DeclareMathOperator{\Hom}{Hom}
\DeclareMathOperator{\Mor}{Mor}
\DeclareMathOperator{\coker}{coker}
\DeclareMathOperator{\Obj}{Obj}
\DeclareMathOperator{\id}{id}
\DeclareMathOperator{\GL}{GL}
\DeclareMathOperator*{\colim}{colim}

%=========================================
% Custom Commands (Shortcuts)
%=========================================
\newcommand{\CP}{\mathbb{CP}}
\newcommand{\GG}{\mathbb{G}}
\newcommand{\F}{\mathbb{F}}
\newcommand{\N}{\mathbb{N}}
\newcommand{\Q}{\mathbb{Q}}
\newcommand{\R}{\mathbb{R}}
\newcommand{\C}{\mathbb{C}}
\newcommand{\E}{\mathbb{E}}
\newcommand{\Prj}{\mathbb{P}}
\newcommand{\RP}{\mathbb{RP}}
\newcommand{\T}{\mathbb{T}}
\newcommand{\Z}{\mathbb{Z}}
\newcommand{\A}{\mathbb{A}}
\renewcommand{\H}{\mathbb{H}}
\newcommand{\K}{\mathbb{K}}

\newcommand{\mA}{\mathcal{A}}
\newcommand{\mB}{\mathcal{B}}
\newcommand{\mC}{\mathcal{C}}
\newcommand{\mD}{\mathcal{D}}
\newcommand{\mE}{\mathcal{E}}
\newcommand{\mF}{\mathcal{F}}
\newcommand{\mG}{\mathcal{G}}
\newcommand{\mH}{\mathcal{H}}
\newcommand{\mI}{\mathcal{I}}
\newcommand{\mJ}{\mathcal{J}}
\newcommand{\mK}{\mathcal{K}}
\newcommand{\mL}{\mathcal{L}}
\newcommand{\mM}{\mathcal{M}}
\newcommand{\mO}{\mathcal{O}}
\newcommand{\mP}{\mathcal{P}}
\newcommand{\mS}{\mathcal{S}}
\newcommand{\mT}{\mathcal{T}}
\newcommand{\mV}{\mathcal{V}}
\newcommand{\mW}{\mathcal{W}}

%=========================================
% Colours!!!
%=========================================
\definecolor{LightBlue}{HTML}{2D64A6}
\definecolor{ForestGreen}{HTML}{4BA150}
\definecolor{DarkBlue}{HTML}{000080}
\definecolor{LightPurple}{HTML}{cc99ff}
\definecolor{LightOrange}{HTML}{ffc34d}
\definecolor{Buff}{HTML}{DDAE7E}
\definecolor{Sunset}{HTML}{F2C57C}
\definecolor{Wenge}{HTML}{584B53}
\definecolor{Coolgray}{HTML}{9098CB}
\definecolor{Lavender}{HTML}{D6E3F8}
\definecolor{Glaucous}{HTML}{828BC4}
\definecolor{Mauve}{HTML}{C7A8F0}
\definecolor{Darkred}{HTML}{880808}
\definecolor{Beaver}{HTML}{9A8873}
\definecolor{UltraViolet}{HTML}{52489C}



%=========================================
% Theorem Environment
%=========================================
\tcbuselibrary{listings, theorems, breakable, skins}

\newtcbtheorem[number within = subsection]{thm}{Theorem}%
{	colback=Buff!3, 
	colframe=Buff, 
	fonttitle=\bfseries, 
	breakable, 
	enhanced jigsaw, 
	halign=left
}{thm}

\newtcbtheorem[number within=subsection, use counter from=thm]{defn}{Definition}%
{  colback=cyan!1,
    colframe=cyan!50!black,
	fonttitle=\bfseries, breakable, 
	enhanced jigsaw, 
	halign=left
}{defn}

\newtcbtheorem[number within=subsection, use counter from=thm]{axm}{Axiom}%
{	colback=red!5, 
	colframe=Darkred, 
	fonttitle=\bfseries, 
	breakable, 
	enhanced jigsaw, 
	halign=left
}{axm}

\newtcbtheorem[number within=subsection, use counter from=thm]{prp}{Proposition}%
{	colback=LightBlue!3, 
	colframe=Glaucous, 
	fonttitle=\bfseries, 
	breakable, 
	enhanced jigsaw, 
	halign=left
}{prp}

\newtcbtheorem[number within=subsection, use counter from=thm]{lmm}{Lemma}%
{	colback=LightBlue!3, 
	colframe=LightBlue!60, 
	fonttitle=\bfseries, 
	breakable, 
	enhanced jigsaw, 
	halign=left
}{lmm}

\newtcbtheorem[number within=subsection, use counter from=thm]{crl}{Corollary}%
{	colback=LightBlue!3, 
	colframe=LightBlue!60, 
	fonttitle=\bfseries, 
	breakable, 
	enhanced jigsaw, 
	halign=left
}{crl}

\newtcbtheorem[number within=subsection, use counter from=thm]{eg}{Example}%
{	colback=Beaver!5, 
	colframe=Beaver, 
	fonttitle=\bfseries, 
	breakable, 
	enhanced jigsaw, 
	halign=left
}{eg}

\newtcbtheorem[number within=subsection, use counter from=thm]{ex}{Exercise}%
{	colback=Beaver!5, 
	colframe=Beaver, 
	fonttitle=\bfseries, 
	breakable, 
	enhanced jigsaw, 
	halign=left
}{ex}

\newtcbtheorem[number within=subsection, use counter from=thm]{alg}{Algorithm}%
{	colback=UltraViolet!5, 
	colframe=UltraViolet, 
	fonttitle=\bfseries, 
	breakable, 
	enhanced jigsaw, 
	halign=left
}{alg}




%=========================================
% Hyperlinks
%=========================================
\hypersetup{
    colorlinks=true, %set true if you want colored links
    linktoc=all,     %set to all if you want both sections and subsections linked
    linkcolor=DarkBlue,  %choose some color if you want links to stand out
}


\pagestyle{fancy}
\fancyhf{}
\rhead{Labix}
\lhead{Higher Category Theory}
\rfoot{\thepage}

\title{Higher Category Theory}

\author{Labix}

\date{\today}
\begin{document}
\maketitle
\begin{abstract}
\begin{itemize}
\end{itemize}
\end{abstract}
\pagebreak
\tableofcontents

\pagebreak
\section{Introduction to Infinity Categories}
\subsection{Infinity Categories as Simplicial Sets}
We recall some basic facts about simplicial sets. If $S:\Delta\to\bold{Set}$ is a simplicial set, then by Yoneda's emebdding we know that the $n$-simplices of $S$ are given by $$S([n])=\Hom_\bold{sSet}(\Delta^n,S)$$ In other words, specifying an $n$-simplex is the same as specifying a map of simplicial sets $$\Delta^n\to S$$



The foundations of infinity categories lay on the simpicial sets. Intuitively, any face $\partial_k\Delta$ of an $n$-simplex $\Delta$ captures a homotopy of the faces of $\partial_k\Delta$. 

\begin{defn}{Infinity Categories}{} An infinity category is a simplicial set $C$ such that each inner horn admits a filler. In other words, for all $0<i<n$, the following diagram commutes: \\~\\
\adjustbox{scale=1.0,center}{\begin{tikzcd}
	{\Lambda_i^n} & C \\
	{\Delta^n}
	\arrow["\forall", from=1-1, to=1-2]
	\arrow[hook, from=1-1, to=2-1]
	\arrow["\exists"', dashed, from=2-1, to=1-2]
\end{tikzcd}}\\~\\
\end{defn}

\begin{defn}{Objects and Morphisms}{} Let $\mC$ be an infinity category. Define the following notions for $\mC$. 
\begin{itemize}
\item Define the objects of $\mC$ to be the $0$-simplices of $\mC$. 
\item Define the morphisms of $\mC$ to be the $1$-simplices of $\mC$. 
\end{itemize}
\end{defn}

\begin{thm}{}{} Let $\mC$ be a category. Every inner horn of the nerve $N(C)$ of $\mC$ admits a filler and hence is an infinity category. 
\end{thm}

\subsection{The Homotopy Category of Infinite Categories}
Let $S$ be a simplicial set. Recall that we have functorially assigned a category $h(S)$ to $S$ called the homotopy category of $S$. This is given together with the universal functor $u:S\to N(h(S))$ by the universal property: For category $\mD$ and a functor $F:S\to N(\mD)$, there exists a unique morphism $F:h(S)\to\mD$ such that $F=N(G)\circ u$. When $S$ is an infinity category, compositions of morphisms forming $n$-simplexes can be shortened to one by the filler-admitting property. 

\begin{defn}{Homotopic Morphisms}{} Let $\mC$ be an infinity category. Two morphisms $f,g:C\to D$ are said to be homotopic if there exists a $2$-simplex $\sigma$ such that 
\begin{itemize}
\item $d_0(\sigma)=\text{id}_D$
\item $d_1(\sigma)=g$
\item $d_2(\sigma)=f$
\end{itemize}
In this case we write $f\simeq g$. 
\end{defn}

Pictorially, we denote the existence of such a $\sigma$ by \\~\\
\adjustbox{scale=1.0,center}{\begin{tikzcd}
	& D \\
	& \sigma \\
	C && D
	\arrow["{\text{id}_D}", from=1-2, to=3-3]
	\arrow["f", from=3-1, to=1-2]
	\arrow["g"', from=3-1, to=3-3]
\end{tikzcd}}\\~\\

This diagram here does not denote commutative, but instead denotes the existence of a $2$-simplex $\sigma$ that has the above as vertices and edges. Rewriting the above definition, we can say that $g\circ f:C\to E$ is homotopic to $h:C\to E$ if there exists a $2$-simplex of the form \\~\\
\adjustbox{scale=1.0,center}{\begin{tikzcd}
	& D \\
	& \sigma \\
	C && E
	\arrow["g", from=1-2, to=3-3]
	\arrow["f", from=3-1, to=1-2]
	\arrow["h"', from=3-1, to=3-3]
\end{tikzcd}}\\~\\
By definition of an infinity category, every inner horn admits a filler. This means that for any composable morphisms $f$ and $g$ giving $g\circ f$, we can always find a morphism $h$ such that $g\circ f$ is homotopic to $h$. However, this $h$ may not be unique, so we cannot conclude that infinity categories have a well defined notion of composition. 

\begin{prp}{}{} Let $C$ be an infinity category. Let $f,f':C\to D$ and $g,g':D\to E$ be morphisms in $C$. If $f\simeq f'$ and $g\simeq g'$, then $$g\circ f\simeq g'\circ f'$$
\end{prp}

\begin{lmm}{}{} Homotopy is an equivalence relation in any infinity category. 
\end{lmm}

We can explicitly write out the homotopy category of an infinity category as follows. 

\begin{prp}{}{} Let $\mC$ be an infinity category. Then the homotopy category $h(\mC)$ is isomorphic (as categories) to the category defined as follows. 
\begin{itemize}
\item The objects of $h(\mC)$ are the objects of $\mC$
\item For $A,B\in\mC$ two objects, the morphisms are equivalent classes of morphisms $[f]$ for $f\in\Hom_\mC(A,B)$. 
\item Composition is defined by $$[g]\circ[f]=[g\circ f]$$ which is well defined by .2
\end{itemize}
\end{prp}

\begin{defn}{Isomorphisms in Infinity Categories}{} Let $\mC$ be an infinity category. Let $f:X\to Y$ be a morphism in $\mC$. We say that $f$ is an isomorphism if $[f]$ is an isomorphism in $h(\mC)$. 
\end{defn}

\subsection{The Infinity Category of Morphisms}
Let $\mC$ and $\mD$ be infinity categories. Recall that the nerve functor is fully faithful. This means that there is a bijection $$\Hom_\bold{Cat}(\mC,\mD)\cong\Hom_\bold{sSet}(N(\mC),N(\mD))$$ We generalize this bijection to define functors for infinity categories. 

\begin{defn}{Functors between Infinity Categories}{} Let $\mC,\mD$ be infinity categories. A functor $F:\mC\to\mD$ is a morphism of simplicial sets. 
\end{defn}

In other words, there is no extra structure for morphisms between infinity categories and between simplicial sets. 

\begin{lmm}{}{} Let $\mC,\mD$ be infinity categories. Let $F:\mC\to\mD$ be a functor. Then the following are true. 
\begin{itemize}
\item $F$ sends an object of $\mC$ to an object of $\mD$. 
\item $F$ sends a morphism in $\mC$ to a morphism in $\mD$. 
\item $F$ sends the identity morphism of $X\in\mC$ to the identity morphism of $F(X)\in\mD$. 
\item If $f:X\to Y$ and $g:Y\to Z$ are morphisms in $\mC$, then $F(g\circ f)=F(g)\circ F(f)$
\end{itemize}
\end{lmm}

Explicitly, morphisms of infinity categories behave exactly what we want it to be like: A generalization of functors between ordinary categories. However, note that it is not enough to specify a morphism of infinity categories just from specifying it on objects. This is because we also need to tell the functor where to map the $n$-simplices. In other words, we need to tell the functor where to send the homotopy data. \\

Because the data of a functor between infinity categories carry $2$-simplicies to $2$-simplicies, we can easily deduce the following. 

\begin{lmm}{}{} Let $\mC,\mD$ be infinity category. Let $F:\mC\to\mD$ be a functor. Then the following are true. 
\begin{itemize}
\item If $f\simeq g$ are homotopic in $\mC$, then $F(f)\simeq F(g)$ are homotopic in $\mD$. 
\item If $f$ is an isomorphism in $\mC$, then $F(f)$ is an isomorphism in $\mD$. 
\end{itemize}
\end{lmm}

When $\mC,\mD$ are ordinary categories, we can talk about diagrams of shape $\mC$ in $\mD$. This just means that we only care about the shape of $\mC$, and we consider this shape inside $\mD$. This was the foundations for limits and colimits of a category. We can also do this for infinity categories, but recall that a functor between infinity categories carries much more data than just the shape of the domain infinity category: it also carries homotopy information. \\

Now recall that for $S,T$ two simplicial sets, we can canonically identify the internal hom $[S,T]$ with the external hom $\Hom_\bold{sSet}(S,T)$ (What is the identification?). This gives the structure of a simplicial set with $\Hom_\bold{sSet}(S,T)$. When $S$ and $T$ are infinity categories, we can show that the Hom set is also an infinity category. 

\begin{prp}{}{} Let $\mC,\mD$ be infinity categories. Then $$\text{Hom}_\bold{sSet}(\mC,\mD)$$ is an infinity category. 
\end{prp}

\subsection{Natural Transformations}
\begin{defn}{Natural Transformations}{} Let $\mC,\mD$ be infinity categories. Let $F,G\in\Hom_\bold{sSet}(\mC,\mD)$ be functors. A natural transformation $\alpha:F\Rightarrow G$ from $F$ to $G$ is a morphism in $\Hom_\bold{sSet}(\mC,\mD)$. 
\end{defn}

\begin{prp}{}{} Let $\mC,\mD$ be infinity categories. Let $F,G\in\Hom_\bold{sSet}(\mC,\mD)$ be functors. Then $\alpha:F\Rightarrow G$ is a natural transformation if and only if $\alpha$ is a homotopy of simplicial sets from $F$ to $G$. 
\end{prp}

\begin{lmm}{}{} Let $\mC,\mD$ be categories. Let $F,G:\mC\to\mD$ be functors. Then $\alpha:F\Rightarrow G$ is a natural transformation if and only if $N(\alpha):N(\mC)\to N(\mD)$ is a natural transformation of infinity categories. 
\end{lmm}

\begin{defn}{Natural Isomorphisms}{} Let $\mC,\mD$ be infinity categories. Let $F,G\in\Hom_\bold{sSet}(\mC,\mD)$ be functors. A natural isomorphism from $F$ to $G$ is a natural transformation $\alpha:F\Rightarrow G$ such that $\alpha$ is an isomorphism in $\Hom_\bold{sSet}(\mC,\mD)$. In this case, we say that $F$ and $G$ are naturally isomorphic. 
\end{defn}

\subsection{Equivalence of Infinity Categories}
\begin{defn}{Equivalence of Infinity Categories}{} Let $\mC,\mD$ be infinity categories. We say that $\mC$ and $\mD$ are equivalent infinity categories if there exists functors $F:\mC\to\mD$ and $G:\mD\to\mC$ such that the following are true. 
\begin{itemize}
\item $G\circ F$ is isomorphic to $\text{id}_\mC$ in $\Hom_\bold{sSet}(\mC,\mC)$
\item $F\circ G$ is isomorphic to $\text{id}_\mD$ in $\Hom_\bold{sSet}(\mD,\mD)$
\end{itemize}
\end{defn}

Recall that two objects in an infinity category $\mC$ is isomorphic if they are isomorphic in $h(\mC)$ in the ordinary sense. In our case, this means that we consider $G\circ F$ and $\text{id}_\mC$ to be objects of the infinity category $\Hom_\bold{sSet}(\mC,\mC)$, and they are isomorphic if $[G\circ F]=[\text{id}_\mC]$. This is the same as saying that $G\circ F$ and $\text{id}_\mC$ are homotopic. (It is also the same as saying $\mC$ and $\mD$ are homotopy equivalent as simplicial sets)

\begin{lmm}{}{} Let $\mC,\mD$ be infinity categories. If $\mC$ and $\mD$ are naturally isomorphic, then $\mC$ and $\mD$ are equivalent. 
\end{lmm}

\begin{prp}{}{} Let $\mC,\mD$ be ordinary categories. Let $F:\mC\to\mD$ be functor. Then $F:\mC\to\mD$ induces an equivalence of categories if and only if $N(F):N(\mC)\to N(\mD)$ induces an equivalence of categories. 
\end{prp}

\begin{prp}{}{} Let $\mC,\mD$ be infinity categories. Let $F:\mC\to\mD$ be a functor. If $F$ is an equivalence of infinity categories, then $h(F):h(\mC)\to h(\mD)$ is an equivalence of ordinary categories. 
\end{prp}

\begin{prp}{}{} Let $\mC,\mD$ be infinity categories. Let $F:\mC\to\mD$ be a functor. Then $F$ is an equivalence of infinity categories if and only if $$F\circ -:\Hom_\bold{sSet}(K,\mC)\to\Hom_\bold{sSet}(K,\mD)$$ is an equivalence of infinity categories for all simplicial sets $K$. 
\end{prp}

\pagebreak
\section{Simplicial Categories}
\subsection{Infinity Categories as Simplicial Categories}
\begin{defn}{Simplicial Categories}{} A simplicial category is a category $\mC$ enriched over $\bold{sSet}$. A simplicial functor is a functor $F:\mC\to\mD$ that is $\bold{sSet}$-enriched. Denote the category of simplicial categories by $$\bold{Cat}_\bold{sSet}$$
\end{defn}

\begin{prp}{}{} Let $\mC$ be a category. Then $\mC$ is a simplicial category if and only if $\mC$ is a simplicial object in $\bold{Cat}$ such that the underlying simplicial set of objects is constant. 
\end{prp}

1.1.4.2 HTT

\begin{defn}{Weakly Equivalent Simplicial Categories}{} Let $\mC,\mD$ be simplicial categories. Let $F:\mC\to\mD$ be a simplicial functor. We say that $F$ is a weak equivalence if the following are true. 
\begin{itemize}
\item For all $A,B\in\mC$, the induced map of simplicial sets $$F:\Hom_\mC(A,B)\to\Hom_\mC(F(A),F(B))$$ is weakly equivalent. 
\item For all $D\in\mD$, there exists some $C\in\mC$ such that $F(C)\cong D$
\end{itemize}
\end{defn}

Note: Markus land says this is weak equivalence, HTT says that this equivalence. 

\begin{defn}{Topological Categories}{} Let $\mC$ be a category. We say that $\mC$ is a topological category if $\mC$ is enriched over $\bold{CGWH}$. 
\end{defn}

Recall that two enriched categories are equivalent if $F:\mC\to\mD$ is fully faithful and essentially surjective. Being fully faithful as $\mS$-functor means that $F$ induces an isomorphism on Hom sets. However this notion is too strong for us because we only want to consider spaces up to homotopy equivalence. 


\pagebreak
\section{Kan Complexes}
\begin{lmm}{}{} Let $X$ be a space. Then applying the singular functor $S(X)$ gives an infinity category. 
\end{lmm}

\begin{prp}{}{} Let $X$ be a space. Then the homotopy category of the singular set of $X$ is equal to $h(S(X))=\prod_1(X)$ the fundamental groupoid of $X$. 
\end{prp}

\subsection{Kan Complexes}
\begin{defn}{Kan Complexes}{} A Kan complex is a simplicial set $C$ such that each horn (inner and outer) admits a filler. In other words, for all $0\leq i\leq n$, the following diagram commutes: \\~\\
\adjustbox{scale=1.0,center}{\begin{tikzcd}
	{\Lambda_i^n} & C \\
	{\Delta^n}
	\arrow["\forall", from=1-1, to=1-2]
	\arrow[hook, from=1-1, to=2-1]
	\arrow["\exists"', dashed, from=2-1, to=1-2]
\end{tikzcd}}\\~\\
\end{defn}

Since infinity catregories require only inner horns to admit a filler, we have the following inclusion relation: $$\substack{\text{Kan}\\\text{Complexes}}\subset\substack{\text{Infinity}\\\text{Categories}}$$

\begin{prp}{}{} Let $X$ be a space. Then $S(X)$ is a Kan complex. 
\end{prp}

\begin{thm}{}{} Let $\mC$ be a small category. Then the simplicial set $N(\mC)$ is a Kan complex if and only if $\mC$ is a groupoid. 
\end{thm}

More: Kan complexes = infinity groupoids (quillen equivalence in model category), and we should think of spaces as Kan complexes / infinity groupoids from now on. 

\pagebreak
\section{Infinity Categorical Constructions}
\subsection{Joins and Slices}
We begin by rewriting the definition of a simplex category as follows. Instead of having distinguished names $[n]$ for the objects, we instead just think of the simplex category with objects as finite and totally ordered sets. Indeed any of these sets will be in bijection to $[n]$ for some $n\in\N$. This language will help us define the join. 

\begin{defn}{}{} Let $J$ be a finite and totally ordered set. A cut of $J$ consists of two subsets $I,I'\subseteq J$ such that $$J=I\amalg I'$$ and $i<i'$ for all $i\in I$ and $i'<I'$. 
\end{defn}

\begin{defn}{Joins}{} Let $X,Y$ be simplicial sets. Define the join of $X$ and $Y$ to be the simplicial set $X\ast Y$ as follows. 
\begin{itemize}
\item Denote $J\neq\emptyset$ any finite and totally ordered set. Define $$X\ast Y(J)=\coprod_{\substack{I\amalg I'=J\\i<i'\text{ for }i\in I,i'\in I'}}X(I)\times Y(I')\coprod_{I,I'\text{ cuts of }J}X(I)\times Y(I')$$ where by convention, $X(\emptyset)=Y(\emptyset)=\ast$. 
\item For two finite and totally ordered sets $J$ and $J'$ and a morphism $J\to J'$ preserving order, the map $$(X\ast Y)[J']\to(X\ast Y)[J]$$ is defined as follows. Let $K,K'$ be a cut of $J'$. Then $\alpha$ restricts to two well defined maps $$\alpha|_{\alpha^{-1}(K)}:\alpha^{-1}(K)\to K\;\;\;\;\text{ and }\;\;\;\;\alpha|_{\alpha^{-1}(K')}:\alpha^{-1}(K')\to K'$$ In particular these are order preserving, and each are morphisms in the simplex category $\Delta$. Thus this gives us a unique morphism $$X(K)\times X(K')\to X(\alpha^{-1}(K))\times X(\alpha^{-1}(K'))$$ By taking the product of these maps, we thus obtain a morphism $(X\ast Y)[J']\to(X\ast Y)[J]$, turning the above definition into a simplicial set. 
\end{itemize}
\end{defn}

Concrete examples: 
\begin{itemize}
\item When $J=[0]$, we have that 
\begin{align*}
(X\ast Y)[0]&=X[0]\times Y(\emptyset)\amalg X(\emptyset)\times Y[0]\\
&=X_0\amalg Y_0
\end{align*}
which means that the vertices of $X\ast Y$ are the vertices of $X$ and $Y$ combined disjointly. 
\item When $J=[1]$, we have that 
\begin{align*}
(X\ast Y)[1]&=X[1]\times Y(\emptyset)\amalg X(\{0\})\times Y(\{1\}) \amalg X(\emptyset)\times Y[1]\\
&=X_1\amalg X_0\times Y_0\amalg Y_1
\end{align*}
\end{itemize}

TBA: The join of ordinary categories. 

\begin{lmm}{}{} Let $X$ and $Y$ be simplicial sets. Then $N(X\ast Y)\cong N(X)\ast N(Y)$
\end{lmm}

TBA: functoriality of join

\begin{prp}{}{} Let $X,Y$ be simplicial sets. Then $X\ast Y$ is an infinity category if and only if $X$ and $Y$ are infinity categories. 
\end{prp}

Recall that the over category $\mC/X$ consists of pairs $(Y,f:Y\to X)$ and morphism are given by commutative diagrams. Let us rephrase the definition as follows. The over category is the unique category such that if $\mD$ is another category, there is a bijection $$\Hom_\bold{CAT}(\mD,\mC/X)\cong\Hom_X(\mD\ast[0],\mC)$$ where the right hand side indicates that we only consider morphisms $\mD\ast[0]\to\mC$ in which $[0]$ is mapped to $X$. This characterization is due to the fact that a morphism $[0]\to\mC$ is essentially a choice of object in $\mC$, in which case we choose to be $X$. 

\begin{defn}{Over Category for Infinity Categories}{} Let $K,X$ be simplicial sets. Let $f:K\to X$ be a map. Define the over category (which is a simplicial set) $$f/X:\Delta\to\bold{Set}$$ as follows. 
\begin{itemize}
\item For each $n$, we have $$(f/X)_n=\Hom_{K/\bold{sSet}}(K\ast\Delta^n,X)$$
\end{itemize}
\end{defn}

TBA: Adjunction of join and slice. 

\subsection{Mapping Spaces}
\begin{defn}{Mapping Spaces}{} Let $\mC$ be an infinity category. Let $x,y\in\mC$ be objects. Define the mapping space from $x$ to $y$ to be the pullback $$\Hom_\mC(x,y)=\{x\}\times_{\Hom_\bold{sSet}(\{0\},\mC)}\times\Hom_\bold{sSet}(\Delta^1,\mC)\times_{\Hom_\bold{sSet}(\{1\},\mC)}\{y\}$$
\end{defn}

Note: $\Hom_\bold{sSet}(\Delta^0,\mC)\cong\mC$ via the map $\text{Ev}:\Hom_\bold{sSet}(\Delta^0,\mC)\times\Delta^0\to\mC$. \\

Note: Land 1.3.47, Kerodon 4.6

\subsection{Left and Right (Pinched) Mapping Spaces}
For an ordinary category $\mC$, we have the notion of Hom sets (at least for locally small categories). We would like to reproduce this notion for infinity categories. \\

Recall that a an $n$-simplex $x$ is degenerate if any two of its consecutive vertices are given by the same element. Explicitly, this means that $x$ lies in the image of some degeneracy map $s_k$. 

\begin{defn}{The Right Mapping Space}{} Let $\mC$ be an infinity category. Let $x,y\in\mC$ be objects. Define the right mapping space from $x$ to $y$ to be the simplicial set defined by $$\Hom_\mC^R(x,y)([n])=\left\{h\in\mC_{n+1}\;\bigg{|}\;d_{n+1}(h)=(\underbrace{s_0\circ\cdots\circ s_0}_{n\text{ times}})(x)\text{ and }(d_0\circ\cdots\circ d_n)(h)=y\right\}$$ for each $n\in\N$. 
\end{defn}

In plain English, the hom set from $x$ to $y$ on the $n$th level consists of $n+1$-simplices $h$ for which the face of $h$ with the first $n$-vertices are given by the $n$ simplex $[x,\dots,x]$, while the last vertex of $h$ is given by $y$. 

\begin{defn}{The Left Mapping Space}{} Let $\mC$ be an infinity category. Let $x,y\in\mC$ be objects. Define the left mapping space from $x$ to $y$ to be the simplicial set defined by $$\Hom_\mC^L(x,y)([n])=\left\{h\in\mC_{n+1}\;\bigg{|}\;d_{n+1}(h)=(\underbrace{s_0\circ\cdots\circ s_0}_{n\text{ times}})(y)\text{ and }(d_0\circ\cdots\circ d_n)(h)=x\right\}$$ for each $n\in\N$. 
\end{defn}

These two notions are equivalent up to homotopy (Land) Also pullbacks (Land)

\begin{prp}{}{} Let $\mC$ be an infinity category. Let $x,y\in\mC$. Then both mapping spaces $\Hom_\mC^R(x,y)$ and $\Hom_\mC^L(x,y)$ are Kan complexes. 
\end{prp}

\begin{prp}{}{} Let $\mC$ be an infinity category. Let $x,y\in\mC$. Then the following are true. 
\begin{itemize}
\item The right mapping space is isomorphic to the pullback $$\Hom_\mC^R(x,y)\cong\{x\}\times_{\Hom_\bold{sSet}(\{0\},\mC)}\mC/y$$
\item The left mapping space is isomorphic to the pullback $$\Hom_\mC^L(x,y)\cong x/\mC\times_{\Hom_\bold{sSet}(\{1\},\mC)}\{y\}$$
\end{itemize}
\end{prp}

\subsection{Composition of Morphisms in Infinity Categories}

\pagebreak
\section{Limits and Colimits}
\subsection{Terminal and Initial Objects}
\begin{defn}{Initial and Terminal Objects}{} Let $\mC$ be an infinity category. Let $x\in\mC$ be an object. 
\begin{itemize}
\item We say that $x$ is initial if for all objects $y\in\mC$, there is a homotopy equivalence $$\Hom_\mC(x,y)\simeq\Delta^0$$
\item Dually, we say that $x$ is terminal if for all objects $y\in\mC$, there is a homotopy equivalence $$\Hom_\mC(y,x)\simeq\Delta^0$$
\end{itemize}
\end{defn}

\begin{prp}{}{} Let $\mC$ be an infinity category. Let $x\in\mC$ be an object. Then the following are equivalent. 
\begin{itemize}
\item $x$ is terminal. 
\item For all $n\geq 1$, every lifting problem of the form \\~\\
\adjustbox{scale=1.0,center}{\begin{tikzcd}
	{\Delta^{\{n\}}} & {\partial\Delta^n} & \mC \\
	& {\Delta^n}
	\arrow[hook, from=1-1, to=1-2]
	\arrow[from=1-2, to=1-3]
	\arrow["x", bend left = 20, from=1-1, to=1-3]
	\arrow[from=1-2, to=2-2]
	\arrow[dashed, from=2-2, to=1-3]
\end{tikzcd}}\\~\\
has a solution. 
\end{itemize}
\end{prp}

initial / terminal carries over by equivalence\\

initial in i-cat imply initial in hCat

\subsection{Limits and Colimits}
\begin{defn}{Limits in Infinity Categories}{} Let $K,X$ be infinity categories. Let $F:K\to X$ be a map. Define the limit $$\lim_FX$$of $F$ over $X$ to be the terminal object of the slice category $X/F$ if it exists. 
\end{defn}


\pagebreak
\section{Relation to Model Categories}
\subsection{Inverting Morphisms in an Infinity Category}
\begin{defn}{}{} Let $\mC$ be an infinity category. Let $W$ be a collection of morphisms in $\mC$. Define the category $$\mC[W^{-1}]$$ together with its canonical functor $F:\mC\to\mC[W^{-1}]$ by the following universal property. \\~\\

For every infinity category $\mD$ together with a functor $G:\mC\to\mD$ such that $G(f)$ is an equivalence for $f\in W$, there exists a unique functor $H:\mC[W^{-1}]\to\mD$ such that the following diagram commutes: \\~\\
\adjustbox{scale=1.0,center}{\begin{tikzcd}
	\mC & {\mC[W^{-1}]} \\
	& \mD
	\arrow["F", from=1-1, to=1-2]
	\arrow["G"', from=1-1, to=2-2]
	\arrow["{\exists!H}", dashed, from=1-2, to=2-2]
\end{tikzcd}}\\~\\
\end{defn}

\begin{prp}{}{} Let $\mC$ be an infinity category. Let $W$ be a collection of morphisms in $\mC$. Then $\mC[W^{-1}]$ exists and is unique up to equivalence of infinity categories. 
\end{prp}

Given a category $\mC$ with weak equivalences $\mW$, we now have a way to systematically construct an infinity category associated to $\mC$. Namely, $$(\mC,\mW)\mapsto N(\mC)[\mW^{-1}]$$

\subsection{Exhibiting a Model Category as an Infinity Category}
Up until now, we have two ways of associating different types of categories with its homotopy category. Namely, if $\mC$ is a model category, then we can associate to it the homotopy category $\text{Ho}(\mC)$. Similarly, if $\mD$ is an infinity category, we can also associate to it a homotopy category $\text{Ho}(\mD)$. This constructions are highly related. In particular, there is a functor sending every model category to an infinity category such that the most important notions such as homotopy limits and colimits coincide. \\

Recall that for a model category $\mC$, we denote the full subcategory spanned by cofibrant objects by $\mC_c$. 

\begin{defn}{}{} Let $(\mC,W)$ be a model category. Let $\mD$ be an infinity category. Let $F:N(\mC_c)\to\mD$ be a functor. We say that $F$ exhibits the underlying category $\mC$ as $\mD$ if the functor induces an equivalence of categories $$N(\mC_c)[W^{-1}]\simeq\mD$$
\end{defn}

Ref:1.3.4.20 HA

\begin{thm}{[Dwyer-Kan]}{} Let $(\mC,\mW)$ be a model category. ??? determines a map $N(\mC_c)\to N(\mC_{cf})$ that induces an equivalence of infinity categories $$N(\mC_c)[\mW^{-1}]\simeq N(\mC_{cf})$$
\end{thm}

TBA: Left Quillen equivalence implies equivalence of infinity categories. 

\subsection{}
Presentable iff $\mD\simeq N(\mC_cf)$ where $\mC$ is a combinatorial simplicial model category. 


\end{document}
