\documentclass[a4paper]{article}

\input{C:/Users/liula/Desktop/Latex/Headers.tex}

\pagestyle{fancy}
\fancyhf{}
\rhead{Labix}
\lhead{The Topology of Fiber Bundles}
\rfoot{\thepage}

\title{The Topology of Fiber Bundles}

\author{Labix}

\date{\today}
\begin{document}
\maketitle
\begin{abstract}
\begin{itemize}
\item Notes on Algebraic Topology by Oscar Randal-Williams
\end{itemize}
\end{abstract}
\pagebreak
\tableofcontents

\pagebreak
\section{Fibrations}
\subsection{Fiber Bundles}
\begin{defn}{Fiber Bundles}{} Let $E,B,F$ be spaces with $B$ connected, and $p:E\to B$ a trivial map. We say that $p$ is a fiber bundle over $F$ if the following are true. 
\begin{itemize}
\item $p^{-1}(b)\cong F$ for all $b\in B$
\item $p:E\to B$ is surjective
\item For every $x\in B$, there is an open neighbourhood $U\subset B$ of $x$ and a fiber preserving homomorphism $\Psi_U:p^{-1}(U)\to U\times F$ that is a homeomormorphism such that the following diagram commutes: \\~\\
\adjustbox{scale=1.0,center}{\begin{tikzcd}
	{p^{-1}(U)} && {U\times F} \\
	& U
	\arrow["{\Psi_U}", from=1-1, to=1-3]
	\arrow["p"', from=1-1, to=2-2]
	\arrow["\pi", from=1-3, to=2-2]
\end{tikzcd}}\\~\\
where $\pi$ is the projection by forgetting the second variable. 
\end{itemize}
We say that $B$ is the base space, $E$ the total space. It is denoted as $(F,E,B)$
\end{defn}

\begin{defn}{Map of Fiber Bundles}{} Let $(F_1,E_1,B_1)$ and $(F_2,E_2,B_2)$ be fiber bundles. A morphism of fiber bundles is a pair of basepoint preserving continuous maps $(\tilde{f}:E_1\to E_2,f:B_1\to B_2)$ such that the following diagram commutes: \\~\\
\adjustbox{scale=1.0,center}{\begin{tikzcd}
	{E_1} & {E_2} \\
	{B_1} & {B_2}
	\arrow["{\tilde{f}}", from=1-1, to=1-2]
	\arrow["{p_1}"', from=1-1, to=2-1]
	\arrow["{p_2}", from=1-2, to=2-2]
	\arrow["f"', from=2-1, to=2-2]
\end{tikzcd}}\\~\\
Such a map of fibrations determine a continuous of the fibers $F_1\cong p_1^{-1}(b_1)\to p_2^{-1}(b_2)\cong F_2$. \\~\\

A map of fibrations $(\tilde{f},f)$ is said to be an isomorphism if there is a map $(\tilde{g}:E_2\to E_1,g:B_2\to B_1)$ such that $\tilde{g}$ is the inverse of $\tilde{f}$ and $g$ is the inverse of $f$. 
\end{defn}

\begin{defn}{Trivial Bundles}{} We say that a fiber bundle $(F,E,B)$ is trivial if $(F,E,B)$ is isomorphic to the trivial fibration $B\times F\to B$. 
\end{defn}

\begin{defn}{Sections}{} Let $(F,E,B)$ be a fiber bundle. A section on the fiber bundle is a map $s:B\to E$ such that $p\circ s=\text{id}_B$. Let $U\subset B$ be an open set. A local section of the fiber bundle on $U$ is a map $s:U\to B$ such that $p\circ s=\text{id}_U$. 
\end{defn}

\subsection{G-Bundles and the Structure Groups}
Notice that for non empty intersections $U_i\cap U_j$ for $U_i,U_j$ open sets in $B$, there is a well defined homeomorphism $$\varphi_j\circ\varphi_i^{-1}:(U_i\cap U_j)\times F\to(U_i\cap U_j)\times F$$ This is reminiscent of properties of an atlas on $M$. 

\begin{defn}{G-Atlas}{} Let $(F,E,B)$ be a fiber bundle. Let $G$ be topological group with a continuous faithful action on $F$. A $G$-atlas on $(F,E,B)$ is a set of local trivalization charts $\{(U_k,\varphi_k)\;|\;k\in I\}$ such that the following are true. 
\begin{itemize}
\item For $(U_k,\varphi_k)$ a chart, define $\varphi_{i,x}:F\to F$ by $f\mapsto\varphi_i(x,f)$. Then the homeomorphism $$\varphi_{j,x}\circ\varphi_{i,x}^{-1}:F\to F$$ for $x\in U_i\cap U_j\neq\emptyset$ is an element of $G$. 
\item For $i,j\in I$, the map $g_{ij}:U_i\cap U_j\to G$ defined by $$g_{ij}(x)=\varphi_{j,x}\circ\varphi_{i,x}^{-1}$$ is continuous. 
\end{itemize}
\end{defn}

If the $(F,E,B)$ is a fiber bundle with $F=\R$, then it is often seen that $G=GL(n,\R)$. Similarly, if $F=\C$ then the structure group is $G=GL(n,\C)$. 

\begin{defn}{Equivalent $G$-Atlas}{} Two $G$-atlases on a fiber bundle $(F,E,B)$ is said to be equivalent if their union is a $G$-atlas. 
\end{defn}

\begin{defn}{G-Bundle}{} Let $G$ be a topological group. A $G$-bundle is a fiber bundle $(F,E,B)$ together with an equivalence class of $G$-atlas. In this case, $G$ is said to be the structure group of the fiber bundle. 
\end{defn}

The structure group and the trivialization charts completely determine the isomorphism type of the fiber bundle. 

\subsection{Morphisms of G-Bundles}
\begin{defn}{Morphisms of $G$-Bundles}{} Let $G$ be a topological group. A morphism of $G$-bundles is a morphism of fiber bundles $(\tilde{h},h):(F,E_1,B_1)\to(F,E_2,B_2)$ where the two are $G$-bundles, such that the following are true. 
\begin{itemize}
\item Let $U_i$ be open in $B_1$ and $V_j$ be open in $B_2$. Let $x\in U_u\cap h^{-1}(V_j)$. Let $\widetilde{h_{(E_1)_x}}:(E_1)_x\to(E_2)_{f(x)}$ be the map induced by $\tilde{h}:E_1\to E_2$. Then the map $$\varphi_{j,x}\circ\widetilde{h_{(E_1)_x}}\circ\varphi_{i,x}^{-1}:F\to F$$ is an element of $G$. 
\item The map $\widetilde{g_{ij}}:U_i\cap h^{-1}(V_j)\to G$ defined by $$\widetilde{g_{ij}}(x)=\varphi_{j,x}\circ\widetilde{h_{(E_1)_x}}\circ\varphi_{i,x}^{-1}$$ is continuous. 
\end{itemize}
\end{defn}

It is easy to see that the mapping transformations $\widetilde{g_{ij}}$ satisfy the following two relations: 
\begin{itemize}
\item $\widetilde{g_{jk}}(x)\cdot g_{ij}(x)=\widetilde{g_{ik}}(x)$ for all $x\in U_i\cap U_j\cap h^{-1}(V_k)$
\item $g_{jk}'(h(x))\cdot\widetilde{g_{ij}}(x)=\widetilde{g_{ik}}(x)$ for all $x\in U_i\cap h^{-1}(V_j\cap V_k)$
\end{itemize}

$g_{jk}'$ here refers to the transition charts in $(F,E_2,B_2)$. \\~\\

Just as the structure groups and trivialization charts determine the isomorphism type of a fiber bundle, the $\widetilde{g_{ij}}$ and a map of base space $h:B_1\to B_2$ completes determines a morphism of $G$-bundle. 

\begin{lmm}{}{} Let $(F,E_1,B_1)$ and $(F,E_2,B_2)$ be two $G$-bundles for a topological group $G$ with the same fiber $F$. Suppose that we have the following. 
\begin{itemize}
\item A map $h:B_1\to B_2$ of base space
\item $\widetilde{g_{ij}}:U_i\cap h^{-1}(V_j)\to G$ a set of continuous maps such that \begin{gather*}
\widetilde{g_{jk}}(x)\cdot g_{ij}(x)=\widetilde{g_{ik}}(x)\;\;\;\;\text{ for all }\;\;\;\;x\in U_i\cap U_j\cap h^{-1}(V_k)\\
g_{jk}'(h(x))\cdot\widetilde{g_{ij}}(x)=\widetilde{g_{ik}}(x)\;\;\;\;\text{ for all }\;\;\;\;x\in U_i\cap h^{-1}(V_j\cap V_k)
\end{gather*}
\end{itemize}
Then there exists a unique $G$-bundle morphism having $h$ as the map of base space and having $\{\widetilde{g_{ij}}\;|\;i,j\in I\}$ as its mapping transformations. 
\end{lmm}

\subsection{Principal G-Bundles}
\begin{defn}{Principal Bundles}{} Let $G$ be a topological group. A principal $G$-bundle is a $G$-bundle $(F,E,B)$ together with a continuous group action $G$ on $E$ such that the following are true. 
\begin{itemize}
\item The action of $G$ preserves fibers. This means that $g\cdot x\in E_b$ if $x\in E_b$. (This also means that $G$ is a group action on each fiber)
\item The action of $G$ on each fiber is free and transitive
\item For each $x\in E_b$, the map $G\to E_b$ defined by $g\mapsto g\cdot x$ is homeomorphism. 
\item Local triviality condition: If $\Psi_U:p^{-1}(U)\to U\times F$ are the local triviality maps, then each $\Psi_U$ are $G$-equivariant maps. 
\end{itemize}
\end{defn}

Note that since $G$ is homeomorphic to each fiber $E_b$ of the total space, we can think of the action of $G$ on the fiber simply becomes left  multiplication. \\~\\

For those who know what homogenous spaces are, principal bundles are $G$-bundles such that $F$ is a principal homogenous space for the left action of $G$ itself. \\~\\

Conversely, given a continuous group action on a space, we can ask in what conditions will the space be a principal bundle over the orbit space. 

\begin{prp}{}{} Let $E$ be a space with a free $G$ action. Let $p:E\to E/G$ be the projection map to the orbit space. If for all $x\in E/G$, there is a neighbourhood $U$ of $x$ and a continuous map $s:U\to E$ such that $p\circ s=\text{id}_U$, then $(G,E,E/G)$ is a principal $G$-bundle. 
\end{prp}

This proposition essentially means that if for each point in $E/G$, there is a local section, then it is sufficient for $E$ to be a principal $G$ bundle over $E/G$. 

\begin{thm}{}{} A principal $G$-bundle is trivial if and only if it admits a global section. 
\end{thm}

This is entirely untrue for general bundles. For examples, the zero section of a fiber bundle is a global section. 

\subsection{Classifying Space}
\begin{defn}{Universal G-Bundles}{} Let $G$ be a topological group. A principal $G$-bundle $(F,E,B)$ is said to be universal if for any space $X$, the induced pullback map $$\psi:[X,B]\to\text{Prin}_G(X)$$ defined by $f\mapsto f^\ast(E)$ is a bijective correspondence. 
\end{defn}

\begin{thm}{}{} Let $(F,E,B)$ be a principal $G$-bundle. If $E$ is contractible then $(F,E,B)$ is a universal $G$-bundle. 
\end{thm}

\begin{thm}{}{} Let $(F,E_1,B_1)$ and $(F,E_2,B_2)$ be universal principal $G$-bundles. Then there exists a bundle map \\~\\
\adjustbox{scale=1.0,center}{\begin{tikzcd}
	{E_1} & {E_2} \\
	{B_1} & {B_2}
	\arrow["{\tilde{f}}", from=1-1, to=1-2]
	\arrow["{p_1}"', from=1-1, to=2-1]
	\arrow["{p_2}", from=1-2, to=2-2]
	\arrow["f"', from=2-1, to=2-2]
\end{tikzcd}}\\~\\
such that $f$ is a homotopy equivalence. 
\end{thm}

\begin{defn}{Classifying Space}{} Let $G$ be a topological group. The classifying space $BG$ of $G$ is the homotopy type of the universal principal $G$-bundle. Also denote $EG$ as the total space of the universal $G$-bundle. 
\end{defn}


\pagebreak
\section{Fibrations and Cofibrations}
\subsection{Fibrations}
\begin{defn}{Fibrations}{} We say that a map $p:E\to B$ is a fibration if it has the homotopy lifting property with respect to all topological spaces $X$. \\~\\

In other words, for any space $X$ together with a homotopy $H:X\times I\to B$ and a lift $\widetilde{H(-,0)}:X\to E$ of $H(-,0)$, there exists a homotopy $\widetilde{H}:X\times I\to E$ lifting $\widetilde{H}$ and extending $\widetilde{H(-,0)}$: \\~\\
\adjustbox{scale=1.0,center}{\begin{tikzcd}
	X && E \\
	\\
	{X\times I} && B
	\arrow["H"', from=3-1, to=3-3]
	\arrow["{\exists\widetilde{H}}"{description}, dashed, from=3-1, to=1-3]
	\arrow["p", from=1-3, to=3-3]
	\arrow["\iota"', hook, from=1-1, to=3-1]
	\arrow["{\widetilde{H(-,0)}}", from=1-1, to=1-3]
\end{tikzcd}}\\~\\
We call $B$ the base space and $E$ the total space. Define the fiber over $b\in B$ to be the subspace $$F_b=p^{-1}(b)\subseteq E$$
\end{defn}

\begin{defn}{Fibration Homomorphism}{} Let $p_1:E_1\to B$ and $p_2:E_2\to B$ be two fibrations. We say that a map $f:E_1\to E_2$ is a fibration homomorphism if $$p_2\circ f=p_1$$ In other words, the following diagram commutes: \\~\\
\adjustbox{scale=1.0,center}{\begin{tikzcd}
	{E_1} && {E_2} \\
	\\
	& B
	\arrow["{p_1}"', from=1-1, to=3-2]
	\arrow["{p_2}", from=1-3, to=3-2]
	\arrow["f", from=1-1, to=1-3]
\end{tikzcd}}\\~\\
\end{defn}

\begin{defn}{Fiber Homotopy Equivalence}{} We say that a fiber homomorphism $f:E_1\to E_2$ is a fiber homotopy equivalence if there exists a fiber homomorphism $g:E_2\to E_1$ such that $f\circ g$ and $g\circ f$ are homotopic by fibration homomorphisms to the identities $\text{id}_{E_2}$ and $\text{id}_{E_1}$ respectively. 
\end{defn}

\begin{defn}{Serre Fibration}{} We say that a map $p:E\to B$ is a Serre fibration if it has the homotopy lifting property with respect to all CW-complexes. 
\end{defn}

It is clear that every (Hurewicz) fibration is a Serre fibration. Moreover, every fiber bundle is also a Serre fibration. 

\begin{prp}{}{} Every (Hurewicz) fibration is a Serre fibration. Every fiber bundle is a Serre fibration. 
\end{prp}










\end{document}
