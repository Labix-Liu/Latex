\documentclass[a4paper]{article}

\input{C:/Users/liula/Desktop/Latex/Headers.tex}

\pagestyle{fancy}
\fancyhf{}
\rhead{Labix}
\lhead{Point Set Topology}
\rfoot{\thepage}

\title{Point Set Topology}

\author{Labix}

\date{\today}
\begin{document}
\maketitle
\begin{abstract}
Topology is the most general setting ever to study any properties of a so called space as it requires only set-theoretic notation to define. This allows this definition to encapsulate a wide range of different spaces as well as allowing some very weird sets to appear as topological space. \\~\\
This subject in maths is the main gateway to higher level mathematics: almost every other mathematical subject requires a space to work on. Therefore it is important to layout the foundations. \\~\\
As expected by some readers, this subject will be definition heavy, since it involves the classfication and identification of different types of spaces. We classify it so that our theories can work on some spaces that contain more sturcture, instead of the general setting. 
\end{abstract}
\textbf{References}
\begin{itemize}
\item A course in Point Set Topology by John B. Conway
\item Lecture Notes of MAT327 at the University of Toronto by Ivan Khatchatourian
\end{itemize}
\pagebreak
\tableofcontents
\pagebreak
\input{C:/Users/liula/Desktop/Latex/Geometry and Topology/Point Set Topology/Point Set Topology Content.tex}
\end{document}