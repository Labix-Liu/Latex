\documentclass[a4paper]{article}

\input{C:/Users/liula/Desktop/Latex/Headers V1.2.tex}

\pagestyle{fancy}
\fancyhf{}
\rhead{Labix}
\lhead{Stable Homotopy Theory}
\rfoot{\thepage}

\title{Stable Homotopy Theory}

\author{Labix}

\date{\today}
\begin{document}
\maketitle
\begin{abstract}
\begin{itemize}
\end{itemize}
\end{abstract}
\pagebreak
\tableofcontents

\pagebreak
\section{The Category of Spectra}
Recall the reduced suspension $\Sigma X$ of a space $X$. 

\subsection{Spectra and $\Omega$-Spectra}
\begin{defn}{Spectra}{} A spectrum $E$ is a collection $\{(E_n,\ast)\;|\;n\in\Z\}$ of pointed spaces together with continuous maps $e_n:\Sigma E_n\to E_{n+1}$. 
\end{defn}

By the adjunction formula between $\Sigma$ and $\Omega$, we can reformulate the map $e_n$ to mean $e_n:E_n\to\Omega E_{n+1}$. 

\begin{defn}{The Suspension Spectrum}{} Let $X$ be a compactly generated space. Define the suspension spectrum of $X$ to consist of the following data. 
\begin{itemize}
\item The collection $\{\Sigma^n X\;|\;n\in\N\}$ of spaces. 
\item The collection $\sigma_n:\Sigma(\Sigma^nX)\to\Sigma^{n+1}X$ of maps which is a homeomorphism. 
\end{itemize}
\end{defn}

\begin{defn}{The CW Spectrum}{} The CW spectrum $E$ is a collection $\{E_n\;|\;n\in\Z\}$ of CW-complexes with a chosen basepoint together with maps $e_n:\Sigma E_n\to E_{n+1}$ so that $\Sigma E_n$ is recognized as a subcomplex of $E_{n+1}$. 
\end{defn}

\begin{lmm}{}{} The suspension spectrum and the CW spectrum for the $n$-sphere $S^n$
\end{lmm}

\begin{defn}{$\Omega$-Spectra}{} Let $\{E_n\;|\;n\in\Z\}$ and $e_n:E_n\to\Omega E_{n+1}$ be a spectra. We say that it is an $\Omega$-spectra if the induced map $(e_n)_\ast$ is a weak homotopy equivalence. 
\end{defn}

\begin{prp}{}{} Let $G$ be an abelian group. Then there is an isomorphism $$\Omega K(G,n)\cong K(G,n-1)$$
\end{prp}

\begin{defn}{Eilenberg-MacLane Spectrum}{} Let $X$ be a space. Define the Elienberg-Maclane spectrum
\end{defn}

\subsection{Brown's Representability Theorem}
Given a spectrum, we can construct a (co)homology theory. The converse is given by Brown's representability theorem. 

\begin{thm}{}{} Let $\{T_n\;|\;n\in\Z\}$ be a CW spectrum such that $T_n$ is $(n-1)$-connected. Define $$\widetilde{E}_k(X)=\colim_{n\to\infty}\pi_{k+n}(X\wedge T_n)$$ Then the functors $\widetilde{E}_k$ for all $k$ defines a reduced homology theory on CW complexes with base point. 
\end{thm}

\begin{thm}{}{} Let $\{T_n\;|\;n\in\Z\}$ be a $\Omega$-spectrum consisting of CW complexes. For any space $X$, define $$\widetilde{E}_k(X)=[X,T_k]$$ for $k\in\Z$. Then the functors $\widetilde{E}_k$ for all $k$ defines a reduced cohomology theory on CW complexes with base point. 
\end{thm}

\begin{defn}{Homology with Coefficients in a Spectrum}{} Let $\K=\{T_n\;|\;n\in\Z\}$ be a spectrum. Define a functor $H_n(-;\K):\bold{CW}^2\to\bold{Ab}$ by $$H_n(X,A;\K)=\lim_{k\to\infty}\pi_{n+k}\left(\frac{X_+}{A_+}\wedge T_k\right)$$ where $X_+$ is the space $X$ together with a chosen base point. 
\end{defn}

\begin{thm}{Brown's Representability Theorem I}{} Let $(h_n,\delta_n)$ be a generalized homology theory. Then there exists a spectrum $\K$ and a natural isomorphism $$H_n(X,A)\cong H_n(X,A;\K)$$ for all CW pairs $(X,A)$. 
\end{thm}
\end{document}
