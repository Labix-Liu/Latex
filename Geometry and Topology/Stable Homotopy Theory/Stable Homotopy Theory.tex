\documentclass[a4paper]{article}

%=========================================
% Packages
%=========================================
\usepackage{mathtools}
\usepackage{amsfonts}
\usepackage{amsmath}
\usepackage{amssymb}
\usepackage{amsthm}
\usepackage[a4paper, total={6in, 8in}, margin=1in]{geometry}
\usepackage[utf8]{inputenc}
\usepackage{fancyhdr}
\usepackage[utf8]{inputenc}
\usepackage{graphicx}
\usepackage{physics}
\usepackage[listings]{tcolorbox}
\usepackage{hyperref}
\usepackage{tikz-cd}
\usepackage{adjustbox}
\usepackage{enumitem}
\usepackage[font=small,labelfont=bf]{caption}
\usepackage{subcaption}
\usepackage{wrapfig}
\usepackage{makecell}



\raggedright

\usetikzlibrary{arrows.meta}

\DeclarePairedDelimiter\ceil{\lceil}{\rceil}
\DeclarePairedDelimiter\floor{\lfloor}{\rfloor}

%=========================================
% Fonts
%=========================================
\usepackage{tgpagella}
\usepackage[T1]{fontenc}


%=========================================
% Custom Math Operators
%=========================================
\DeclareMathOperator{\adj}{adj}
\DeclareMathOperator{\im}{im}
\DeclareMathOperator{\nullity}{nullity}
\DeclareMathOperator{\sign}{sign}
\DeclareMathOperator{\dom}{dom}
\DeclareMathOperator{\lcm}{lcm}
\DeclareMathOperator{\ran}{ran}
\DeclareMathOperator{\ext}{Ext}
\DeclareMathOperator{\dist}{dist}
\DeclareMathOperator{\diam}{diam}
\DeclareMathOperator{\aut}{Aut}
\DeclareMathOperator{\inn}{Inn}
\DeclareMathOperator{\syl}{Syl}
\DeclareMathOperator{\edo}{End}
\DeclareMathOperator{\cov}{Cov}
\DeclareMathOperator{\vari}{Var}
\DeclareMathOperator{\cha}{char}
\DeclareMathOperator{\Span}{span}
\DeclareMathOperator{\ord}{ord}
\DeclareMathOperator{\res}{res}
\DeclareMathOperator{\Hom}{Hom}
\DeclareMathOperator{\Mor}{Mor}
\DeclareMathOperator{\coker}{coker}
\DeclareMathOperator{\Obj}{Obj}
\DeclareMathOperator{\id}{id}
\DeclareMathOperator{\GL}{GL}
\DeclareMathOperator*{\colim}{colim}

%=========================================
% Custom Commands (Shortcuts)
%=========================================
\newcommand{\CP}{\mathbb{CP}}
\newcommand{\GG}{\mathbb{G}}
\newcommand{\F}{\mathbb{F}}
\newcommand{\N}{\mathbb{N}}
\newcommand{\Q}{\mathbb{Q}}
\newcommand{\R}{\mathbb{R}}
\newcommand{\C}{\mathbb{C}}
\newcommand{\E}{\mathbb{E}}
\newcommand{\Prj}{\mathbb{P}}
\newcommand{\RP}{\mathbb{RP}}
\newcommand{\T}{\mathbb{T}}
\newcommand{\Z}{\mathbb{Z}}
\newcommand{\A}{\mathbb{A}}
\renewcommand{\H}{\mathbb{H}}
\newcommand{\K}{\mathbb{K}}

\newcommand{\mA}{\mathcal{A}}
\newcommand{\mB}{\mathcal{B}}
\newcommand{\mC}{\mathcal{C}}
\newcommand{\mD}{\mathcal{D}}
\newcommand{\mE}{\mathcal{E}}
\newcommand{\mF}{\mathcal{F}}
\newcommand{\mG}{\mathcal{G}}
\newcommand{\mH}{\mathcal{H}}
\newcommand{\mI}{\mathcal{I}}
\newcommand{\mJ}{\mathcal{J}}
\newcommand{\mK}{\mathcal{K}}
\newcommand{\mL}{\mathcal{L}}
\newcommand{\mM}{\mathcal{M}}
\newcommand{\mO}{\mathcal{O}}
\newcommand{\mP}{\mathcal{P}}
\newcommand{\mS}{\mathcal{S}}
\newcommand{\mT}{\mathcal{T}}
\newcommand{\mV}{\mathcal{V}}
\newcommand{\mW}{\mathcal{W}}

%=========================================
% Colours!!!
%=========================================
\definecolor{LightBlue}{HTML}{2D64A6}
\definecolor{ForestGreen}{HTML}{4BA150}
\definecolor{DarkBlue}{HTML}{000080}
\definecolor{LightPurple}{HTML}{cc99ff}
\definecolor{LightOrange}{HTML}{ffc34d}
\definecolor{Buff}{HTML}{DDAE7E}
\definecolor{Sunset}{HTML}{F2C57C}
\definecolor{Wenge}{HTML}{584B53}
\definecolor{Coolgray}{HTML}{9098CB}
\definecolor{Lavender}{HTML}{D6E3F8}
\definecolor{Glaucous}{HTML}{828BC4}
\definecolor{Mauve}{HTML}{C7A8F0}
\definecolor{Darkred}{HTML}{880808}
\definecolor{Beaver}{HTML}{9A8873}
\definecolor{UltraViolet}{HTML}{52489C}



%=========================================
% Theorem Environment
%=========================================
\tcbuselibrary{listings, theorems, breakable, skins}

\newtcbtheorem[number within = subsection]{thm}{Theorem}%
{	colback=Buff!3, 
	colframe=Buff, 
	fonttitle=\bfseries, 
	breakable, 
	enhanced jigsaw, 
	halign=left
}{thm}

\newtcbtheorem[number within=subsection, use counter from=thm]{defn}{Definition}%
{  colback=cyan!1,
    colframe=cyan!50!black,
	fonttitle=\bfseries, breakable, 
	enhanced jigsaw, 
	halign=left
}{defn}

\newtcbtheorem[number within=subsection, use counter from=thm]{axm}{Axiom}%
{	colback=red!5, 
	colframe=Darkred, 
	fonttitle=\bfseries, 
	breakable, 
	enhanced jigsaw, 
	halign=left
}{axm}

\newtcbtheorem[number within=subsection, use counter from=thm]{prp}{Proposition}%
{	colback=LightBlue!3, 
	colframe=Glaucous, 
	fonttitle=\bfseries, 
	breakable, 
	enhanced jigsaw, 
	halign=left
}{prp}

\newtcbtheorem[number within=subsection, use counter from=thm]{lmm}{Lemma}%
{	colback=LightBlue!3, 
	colframe=LightBlue!60, 
	fonttitle=\bfseries, 
	breakable, 
	enhanced jigsaw, 
	halign=left
}{lmm}

\newtcbtheorem[number within=subsection, use counter from=thm]{crl}{Corollary}%
{	colback=LightBlue!3, 
	colframe=LightBlue!60, 
	fonttitle=\bfseries, 
	breakable, 
	enhanced jigsaw, 
	halign=left
}{crl}

\newtcbtheorem[number within=subsection, use counter from=thm]{eg}{Example}%
{	colback=Beaver!5, 
	colframe=Beaver, 
	fonttitle=\bfseries, 
	breakable, 
	enhanced jigsaw, 
	halign=left
}{eg}

\newtcbtheorem[number within=subsection, use counter from=thm]{ex}{Exercise}%
{	colback=Beaver!5, 
	colframe=Beaver, 
	fonttitle=\bfseries, 
	breakable, 
	enhanced jigsaw, 
	halign=left
}{ex}

\newtcbtheorem[number within=subsection, use counter from=thm]{alg}{Algorithm}%
{	colback=UltraViolet!5, 
	colframe=UltraViolet, 
	fonttitle=\bfseries, 
	breakable, 
	enhanced jigsaw, 
	halign=left
}{alg}




%=========================================
% Hyperlinks
%=========================================
\hypersetup{
    colorlinks=true, %set true if you want colored links
    linktoc=all,     %set to all if you want both sections and subsections linked
    linkcolor=DarkBlue,  %choose some color if you want links to stand out
}


\pagestyle{fancy}
\fancyhf{}
\rhead{Labix}
\lhead{Stable Homotopy Theory}
\rfoot{\thepage}

\title{Stable Homotopy Theory}

\author{Labix}

\date{\today}
\begin{document}
\maketitle
\begin{abstract}
\begin{itemize}
\end{itemize}
\end{abstract}
\pagebreak
\tableofcontents

\pagebreak
\section{The Stable Phenomena}
\subsection{The Stable Homotopy Groups}
In Algebraic Topology, we refer to a property or invariant being stable if applying the suspension functor to the space does not change the property (up to possibly a shift in index). The first such example is one we have already encountered. \\~\\

Let $X$ be a space. We have seen in Algebraic Topology 3 that we can define the stable homotopy groups of $X$ as follows. 

\begin{defn}{Stable Homotopy Groups}{} Let $X$ be a space. Let $n\in\N$. Define the $n$th stable homotopy groups of $X$ to be $$\pi_n^s(X)=\colim_{k\to\infty}\pi_{n+k}(\Sigma^kX)$$
\end{defn}

This is well defined because of the Freudenthal suspension theorem, which states that the groups in the direct limit eventually stabilize. Indeed, the same theorem shows the following. 

\begin{prp}{}{} Let $X$ be a space. Then there is an isomorphism $$\pi_n^s(X)\cong\pi_{n+1}^s(\Sigma X)$$ induced by the suspension functor between stable homotopy groups for any $n\in\N$. 
\end{prp}

New: Graded ring structure on $\pi_\ast^s$. When $X$ and $Y$ are pointed CW complexes, the abelian groups $[\Sigma^n X,\Sigma Y]$ also become stable under suspensions. 

\begin{thm}{}{} Let $X$ and $Y$ be pointed CW complexes. Then the sequence of abelian groups, \\~\\
\adjustbox{scale=1,center}{\begin{tikzcd}
	{[\Sigma^nX,\Sigma^nY]} & {[\Sigma^{n+1}X,\Sigma^{n+1}Y]} & \cdots
	\arrow["\Sigma", from=1-1, to=1-2]
	\arrow[from=1-2, to=1-3]
\end{tikzcd}}\\~\\
are isomorphic under the suspension functor for $n>\dim(X)$. 
\end{thm}

\begin{defn}{Set of Stable Homotopy Classes of Maps}{} Let $X$ and $Y$ be pointed CW complexes. The set of stable homotopy classes of maps from $X$ to $Y$ is defined to be the abelian group $$[X,Y]^s=\colim_{n\in\N}[\Sigma^nX,\Sigma^nY]$$
\end{defn}

New: When $X$ is compact, $[X,QY]_\ast=[X,Y]_\ast^s$. 

\begin{thm}{}{} The stable homotopy groups define a collection of functors $\pi_n^s:\bold{CW}_\ast\to\bold{Ab}$ where $X$ is sent to $\pi_n^s(X)$ and a map $f:X\to Y$ is sent to the map $\pi_n^s(X)\to\pi_n^s(Y)$ defined by the colimit of the suspensions. 
\end{thm}

\begin{thm}{}{} The stable homotopy functors $\pi_n^s:\bold{CW}_\ast\to\bold{Ab}$ for each $n\in\N$ defines a reduced homology theory. 
\end{thm}

This is untrue for the unstable homotopy groups. In particular, the fundamental group is not necessarily abelian. 

\subsection{Spectra and Functions of Spectra}
The stable homotopy groups inputs a space and outputs a colimit of homotopy groups which stabilizes by the Freudenthal suspension theorem. Conversely, we can extract from this result the following. If $X$ is a space, we have a sequence of spaces $$X,\Sigma X,\Sigma^2X,\dots$$ For each $n$, the sequence $$\pi_n(X)\to\pi_{n+1}(\Sigma X)\to\pi_{n+2}(\Sigma^2X)\to\cdots$$ eventually stabilizes by the Freudenthal suspension theorem. This is the guiding result for the definition of a spectrum. 

\begin{defn}{Spectra}{} A spectrum $E$ is a collection $\{(E_n,\ast)\;|\;n\in\Z\}$ of pointed spaces in $\bold{CG}$ together with continuous maps $e_n:\Sigma E_n\to E_{n+1}$ or equivalently, continuous maps $e_n:E_n\to\Omega E_{n+1}$. 
\end{defn}

We relate the definition with the above as follows. A spectrum consists of a sequence of spaces (let us start index it with $\N$) $$E_0,E_1,E_2,\dots$$ For each $n$, we would like to have a sequence of maps $$\pi_n(E_0)\to\pi_{n+1}(E_1)\to\pi_{n+2}(E_2)\to\dots$$ similar to the initial digression. These maps are in fact in our hands. For each $k\in\N$, one has the maps $$\pi_{n+k}(E_k)\overset{\Sigma_\ast}{\to}\pi_{n+k+1}(\Sigma E_k)\overset{(e_k)_\ast}{\to}\pi_{n+k+1}(\Sigma E_{k+1})$$~\\

Notice that we have restricted our spaces to that in $\bold{CG}$, the category of compactly generated spaces. Most of the theorems only work under such an assumption, and there is little loss forgetting the rest of the spaces. 

\begin{defn}{Functions of Spectra}{} Let $E$ and $F$ be spectra. A function from $E$ to $F$ is a collection of maps $\varphi_n:E_n\to F_n$ such that the following diagrams (which are equivalent by adjunction) are commutative: \\~\\
\adjustbox{scale=1,center}{\begin{tikzcd}
	{\Sigma E_n} & {E_{n+1}} && {E_n} & {\Omega E_{n+1}} \\
	{\Sigma F_n} & {F_{n+1}} && {F_n} & {\Omega F_{n+1}}
	\arrow["{e_n}", from=1-1, to=1-2]
	\arrow["{\Sigma\varphi_n}"', from=1-1, to=2-1]
	\arrow["{\varphi_{n+1}}", from=1-2, to=2-2]
	\arrow["{e_n'}", from=1-4, to=1-5]
	\arrow["{\varphi_n}"', from=1-4, to=2-4]
	\arrow["{\Omega\varphi_{n+1}}", from=1-5, to=2-5]
	\arrow["{f_n}"', from=2-1, to=2-2]
	\arrow["{f_n'}"', from=2-4, to=2-5]
\end{tikzcd}}\\~\\
\end{defn}

\begin{defn}{Subspectra}{} Let $E$ and $F$ be spectra. We say that $F$ is a subspectra of $E$ if $F_n\subseteq E_n$ for all $n\in\N$. 
\end{defn}

\begin{defn}{The Suspension Spectrum}{} Let $X\in\bold{CG}_\ast$ be a space. Define the suspension spectrum $\Sigma^\infty X$ of $X$ to consist of the following data. 
\begin{itemize}
\item The collection $\{\Sigma^n X\;|\;n\in\N\}$ of spaces. 
\item The collection $\sigma_n:\Sigma(\Sigma^nX)\to\Sigma^{n+1}X$ of maps which is a homeomorphism. 
\end{itemize}
\end{defn}

\begin{defn}{Eilenberg-MacLane Spectrum}{} Let $G$ be an abelian group. Define the Elienberg-Maclane spectrum to be the spaces $$K(G,n)$$ together with maps $K(G,n)\to\Omega K(G,n+1)$ which are homeomorphisms. 
\end{defn}

\subsection{Brown's Representability Theorem}
Specific types of spectra are related to (co)homology theories. We will introduce the names of such spectra below. 

\begin{defn}{CW Spectra}{} A CW spectrum $E$ is a collection $\{E_n\;|\;n\in\Z\}$ of CW-complexes with a chosen basepoint together with maps $e_n:\Sigma E_n\to E_{n+1}$ so that $\Sigma E_n$ is recognized as a subcomplex of $E_{n+1}$. 
\end{defn}

\begin{defn}{$\Omega$-Spectra}{} Let $\{E_n\;|\;n\in\Z\}$ and $e_n:E_n\to\Omega E_{n+1}$ be a spectra. We say that it is an $\Omega$-spectra if the induced map $(e_n)_\ast$ is a weak homotopy equivalence. 
\end{defn}

Recall that if $Z$ is a group-like $H$-space (ref Concise J.P. May), then $[X,Z]$ has a group structure. 

\begin{thm}{}{} Let $\{T_n\;|\;n\in\Z\}$ be a $\Omega$-spectrum consisting of CW complexes. For any space $X$, define a functor $\widetilde{E}^k:\bold{CW}_\ast\to\bold{Ab}$ as follows. 
\begin{itemize}
\item On objects, a space $X$ is sent to $\widetilde{E}^k(X)=[X,T_k]$ for $k\in\Z$.
\item For $f:X\to Y$ a morphism, $\widetilde{E}^k(f):[Y,T_k]\to[X,T_k]$ is defined by pre composition.
\end{itemize}
Then the collection of functors $\widetilde{E}^k$ for all $k$ defines a reduced cohomology theory on CW complexes with base point. 
\end{thm}

\begin{thm}{Brown's Representability Theorem}{} Let $\widetilde{h}^n:\bold{hCW}_\ast\to\bold{Ab}$ be a reduced cohomology theory with chosen base points on the CW complexes. Then there exists a CW spectrum $\K=\{K_n\;|\;n\in\N\}$ and natural isomorphisms $$\widetilde{h}^n(X)\cong[X,K_n]_\ast$$ for all CW-complexes $X$. 
\end{thm}

It is related to representability in category theory in the following sense: Since $\widetilde{h}^n$ are functors that are homotopy equivalent, we can instead consider $\widetilde{h}^n$ as a functor from the homotopy category $\bold{hCW}_\ast$ of pointed CW-complexes to $\bold{Ab}$. Then Hom sets in $\bold{hCW}$ are precisely $[X,Y]_\ast$ which are the base point preserving homotopic maps from $X$ to $Y$. Then Brown's representability states that the functor $\widetilde{h}^n:\bold{hCW}_\ast\to\bold{Ab}$ is representable via $[X,K_n]_\ast$ and more over the $K_n$ assemble into a spectrum. 

\begin{thm}{}{} Every reduced cohomology theory determines and is determined by an $\Omega$ CW-spectrum. 
\end{thm}

We note here that every reduced cohomology theory induces a generalized cohomology theory, hence the above theorem hence a version for generalized cohomology theories and also pointed cohomology theories. \\~\\

Note: non homotopy equivalent spectra can represent the same cohomology theory. 

\begin{thm}{}{} The reduced singular cohomology theory $\widetilde{H}^k:\bold{CW}\to\bold{Ab}$ with coefficients in an abelian group $G$ is determined by the Eilenberg-maclane spectrum $\{K(G,n)\;|\;n\in\N\}$. 
\end{thm}

\begin{defn}{Homology with Coefficients in a Spectrum}{} Let $\K=\{T_n\;|\;n\in\Z\}$ be a spectrum. Define a functor $H_n(-;\K):\bold{CW}^2\to\bold{Ab}$ by $$H_n(X,A;\K)=\colim_{k\to\infty}\pi_{n+k}\left(\frac{X_+}{A_+}\wedge T_k\right)$$ where $X_+$ is the space $X$ together with a chosen base point. 
\end{defn}

\begin{thm}{}{} Let $(h_n,\delta_n)$ be a generalized homology theory. Then there exists a spectrum $\K$ and a natural isomorphism $$h_n(X,A)\cong H_n(X,A;\K)$$ for all CW pairs $(X,A)$. 
\end{thm}

\begin{thm}{}{} Let $\{T_n\;|\;n\in\Z\}$ be a CW spectrum such that $T_n$ is $(n-1)$-connected. Define $$\widetilde{E}_k(X)=\colim_{n\to\infty}\pi_{k+n}(X\wedge T_n)$$ Then the functors $\widetilde{E}_k$ for all $k$ defines a reduced homology theory on CW complexes with base point. (Concise J.P. May)
\end{thm}

\begin{thm}{}{} Any reduced homology theory determines and is determined by a CW spectrum. 
\end{thm}

\subsection{The Spanier-Whitehead Category}
We would like to define a category so that we can investigate the stable phenomena. Recall from Model Category Theory that the classical suspension functor $\Sigma:\bold{Top}_\ast\to\bold{Top}_\ast$ does not give an equivalence of categories. Thus $\bold{Top}$ is not a good category to investigate stable phenomena. Indeed we would like such a category to be stable (in the sense of equivalence of categories) with respect to the suspension and loopspace functor. More generally, we want such a category $\bold{SHC}$ to have the following properties: 
\begin{itemize}
\item There is an adjunction $$\Sigma^\infty:\bold{CG}_\ast\rightleftarrows\bold{SHC}:\Omega^\infty$$
\item If $A$ is a CW complex with finitely many cells, and $B$ is a CW complex, then there is an isomorphism $$[\Sigma^\infty A,\Sigma^\infty B]\cong[A,B]^s$$
\item ?
\end{itemize}

\begin{defn}{The Spanier-Whitehead Category}{} Define the Spanier-Whitehead category $\bold{SW}$ as follows. 
\begin{itemize}
\item The objects consists of a pair $(X,n)$ where $X$ is a pointed CW complex and $n\in\N$. 
\item For $(X,n)$ and $(Y,m)$ to objects, $$\Hom_{\bold{SW}}((X,n),(Y,m))=\colim_{r\to\infty}[\Sigma^{n+r}X,\Sigma^{m+r}Y]_\ast$$
\item Composition is given by the composition of maps. 
\end{itemize}
\end{defn}

\begin{prp}{}{} The category $\bold{SW}$ is additive and is a triangulated category. 
\end{prp}

$[-,X]_\ast^s$ almost defines a reduced cohomology theory. It fails at the wedge axiom. 

\pagebreak
\section{The Stable Homotopy Category}
\subsection{The Category of Spectra}
\begin{defn}{The Category of Spectra}{} Define the category $\mS^\N$ of spectra to consist of the following data. 
\begin{itemize}
\item The objects are spectra arising from sequences of spaces in $\bold{CG}_\ast$
\item The morphisms are given by functions of spectra (not maps)
\item Composition is given by the composition of functions
\end{itemize}
\end{defn}

\begin{defn}{Smash Tensoring}{} Let $E$ be a spectrum and let $X\in\bold{CG}_\ast$ be a pointed space. Define the smash tensoring of $E$ and $X$ to be the spectrum $E\wedge X$ given as follows. 
\begin{itemize}
\item For each $n\in\N$, $(E\wedge X)_n=E_n\wedge X$
\item For each map $e_n:\Sigma E\to E_{n+1}$, the structure map is given by $$e_n\wedge\text{id}_X:\Sigma(E\wedge X)_n\to(E\wedge X)_{n+1}$$ by wed
\end{itemize}
\end{defn}

\begin{defn}{Powering}{} Let $E$ be a spectrum and let $X\in\bold{CG}_\ast$ be a pointed space. Define the powering of $E$ and $X$ to be the spectrum $\text{Map}(X,E)_\ast$ given as follows. 
\begin{itemize}
\item For each $n\in\N$, $(\text{Map}(X,E)_\ast)_n=\text{Map}(X,E_n)_\ast$
\item For each map $e_n:\Sigma E\to E_{n+1}$, the structure map is given by $$\Sigma(\text{Map}(X,E)_\ast)_n\overset{(\text{const},\text{id})}{\longrightarrow}\text{Map}(X,\Sigma E_n)_\ast\overset{\text{Map}(X,e_n)_\ast}{\longrightarrow}(\text{Map}(X,E)_\ast)_{n+1}$$ by wed
\end{itemize}
\end{defn}

Next: Smash tensoring and powering are functorial (nLab), adjunction between smash tensoring and powering. 

\begin{defn}{Shifted Spectrum}{} Let $X\in\bold{CG}$ be a space. Define the $k$-fold shifted spectrum $F_k^\N X$ of $X$ to consists of 
\begin{itemize}
\item For each $n$, the space is given by $$(F_k^\N X)_n=\begin{cases}
S^{n-k}\wedge X & \text{ if } n\geq k\\
\ast & \text{ if } n<k
\end{cases}$$
\item The structure maps are given by the canonical maps and the unique map from $\ast$. 
\end{itemize}
\end{defn}

\subsection{The Level-wise Model Structure on the Category of Spectra}
Note: $X_+=(X\coprod\{\ast\},\ast)\in\bold{Top}_\ast$. 

\begin{thm}{}{} The category $\mS^\N$ of spectra has a pointed model structure with the following data. 
\begin{itemize}
\item The weak equivalences are the level-wise weak homotopy equivalences of spaces. 
\item The fibrations are the level-wise Serre fibrations
\item The cofibrations are the level-wise $q$-cofibrations. 
\end{itemize}
This model structure is cofibrantly generated with the generating sets given by $$I_\text{level}=\{F_d^\N(S_+^{a-1}\to D_+^a)\;|\;a,d\in\N\}\;\;\;\;\text{ and }\;\;\;\; J_\text{level}=\{F_d^\N(D_+^a\to(D^a\times I)_+)\;|\;a,d\in\N\}$$
\end{thm}

\begin{defn}{Level-wise Model Structure}{} The level wise model structure on the category $\mS^\N$ of spectra is the model structure generated by 
\begin{itemize}
\item The cofibrations $I_\text{level}=\{F_d^\N(S_+^{a-1}\to D_+^a)\;|\;a,d\in\N\}$
\item The ayclic cofibrations $J_\text{level}=\{F_d^\N(D_+^a\to(D^a\times I)_+)\;|\;a,d\in\N\}$
\end{itemize}
\end{defn}

Unfortunately, such a direct translation of model category structure from $\bold{Top}$ to $\mS^\N$ does not give the appropriate stable homotopy category. Therefore we need a new model structure. For this, we turn to the homotopy groups. 

\subsection{Homotopy Groups of a Spectrum}
\begin{defn}{Homotopy Groups of a Spectrum}{} Let $E$ be a spectrum. Define the $n$th (stable) homotopy group of $E$ to be $$\pi_n(E)=\colim_{k\to\infty}\pi_{n+k}(E_k)$$
\end{defn}

Notice that this is a generalization of the stable homotopy groups in Algebraic Topology 3. Indeed if one considers the suspension spectrum of space, then the homotopy groups of the given suspension spectrum are the stable homotopy groups in the sense of Algebraic Topology 3. 

Next: $\pi_k(-)$ is functorial. 

\begin{prp}{}{} Let $X$ be an $\Omega$-spectrum. Then the homotopy groups of $X$ are given as follows. $$\pi_k(X)=\begin{cases}
\pi_{k+n}(X_n) & \text{ if } k+n\geq 0\\
\pi_k(X_0) & \text{ if } k\geq 0\\
\pi_0(X_{\abs{k}}) & \text{ if } k<0
\end{cases}$$
\end{prp}

\begin{defn}{$\pi_\ast$-Equivalence}{} Let $f:X\to Y$ be a map of spectra. We say that $f$ is a $\pi_\ast$-equivalence if the induced map $$\pi_n(f):\pi_n(X)\to\pi_n(Y)$$ is an isomorphism for all $n$. In this case, we say that $X$ and $Y$ are $\pi_\ast$-isomorphic. 
\end{defn}

\subsection{The Stable Model Structure on the Category of Spectra}
\begin{defn}{Generating Sets of the Stable Model Structure}{} Define the generating sets of the stable model structure by
\begin{itemize}
\item $I_\text{stable}=I_\text{level}$
\item $J_\text{stable}=J_\text{level}\cup\{???\}$
\end{itemize}
\end{defn}

\begin{defn}{Stable Fibrations}{} Let $f:X\to Y$ be a map of spectra. We say that $f$ is a stable fibration if it has the right lifting property with respect to $J_\text{stable}$. 
\end{defn}

\begin{prp}{}{} Let $f:X\to Y$ be a map of spectra. Then $f$ is a stable fibration if and only if $f$ is a level-wise fibration of spaces and for each $n\in\N$, the map $$X_n\to Y_n\times_{\Omega Y_{n+1}}\Omega X_{n+1}$$ induced by $\widetilde{\sigma}_n^X$ and $f$ is a weak homotopy equivalence. 
\end{prp}

\begin{thm}{}{} The above generating sets cofibrantly generates a model structure on $\mS^\N$ with the following data. 
\begin{itemize}
\item The weak equivalence are precisely the $\pi_\ast$-isomorphisms
\item The cofibrations are the $q$-fibrations
\item The fibrations are precisely the stable fibrations
\end{itemize}
Moreover, the fibrant objects are precisely the $\Omega$-spectra. 
\end{thm}

\begin{defn}{The Stable Model Structure}{} The stable model structure on $\mS^\N$ is the model structure described above. Explicitly, 
\begin{itemize}
\item The weak equivalence are precisely the $\pi_\ast$-isomorphisms
\item The cofibrations are the $q$-fibrations
\item The fibrations are precisely the stable fibrations
\end{itemize}
\end{defn}

\begin{defn}{The Stable Homotopy Category}{} Define the stable homotopy category to be the homotopy category $$\mS\mH\mC=\text{Ho}(\mS^\N)$$ of the category of spectra. 
\end{defn}

Recall that a pointed model category implicitly has the notion of a suspension and loopspace functor. In our case it will prove to be useful to be able to construct it explicitly. In particular, one can see that such functors are reminiscent of the usual suspension and loopspace functors in classical algebraic topology. 

\begin{defn}{Alternative Suspension}{} Define the alternative suspension functor $\Sigma:\mS^\N\to\mS^\N$ by the following data. 
\begin{itemize}
\item For a spectrum $X\in\mS^\N$, define $\Sigma X$ to be the spectrum where $(\Sigma X)_n=S^1\wedge X_n$ and $$\sigma_n^{\Sigma X}=S^1\wedge(\sigma_n^X):S^1\wedge\Sigma X_n\to S^1\wedge X_{n+1}$$
\item For a map $f:X\to Y$ of spectra, define $\Sigma f:\Sigma X\to\Sigma Y$ level-wise by $(\Sigma f)_n=S^1\wedge(\sigma_n(f)):\Sigma X_n\to\Sigma Y_n$
\end{itemize}
\end{defn}

\begin{defn}{Alternative Looping}{} Define the alternative looping functor $\Sigma:\mS^\N\to\mS^\N$ by the following data. 
\begin{itemize}
\item For a spectrum $X\in\mS^\N$, define $\Omega X$ to be the spectrum where $(\Omega X)_n=\text{Map}_\ast(S^1,X_n)$ and $$\sigma_n^{\Omega X}=\text{Map}_\ast(S^1,\sigma_n^X):\text{Map}_\ast(S^1,\Sigma X_n)\to\text{Map}_\ast(S^1,X_{n+1})$$
\item For a map $f:X\to Y$ of spectra, define $\Omega f:\Omega X\to\Omega Y$ level-wise by $(\Omega f)_n=\text{Map}_\ast(S^1,\sigma_n(f)):\Omega X_n\to\Omega Y_n$
\end{itemize}
\end{defn}

\begin{thm}{}{} The suspension and looping functor of the stable model structure on $\mS^\N$ is precisely given by the alternative suspension and alternative looping. In particular, there is an adjunction $$\Sigma:\text{Ho}(\mS^\N)\rightleftarrows\text{Ho}(\mS^\N):\Omega$$
\end{thm}

\begin{thm}{}{} The category $\mS^\N$ is a stable model category. Explicitly, this means that both $\Sigma,\Omega:\text{Ho}(\mS^\N)\to\text{Ho}(\mS^\N)$ define equivalence of categories. 
\end{thm}

\subsection{Properties of the Stable Homotopy Category}
\begin{defn}{The Suspension Functor}{} Define the suspension functor $\Sigma^\infty:\bold{CG}_\ast\to\mS^\N$ to consist of the following. 
\begin{itemize}
\item For $X\in\bold{CG}_\ast$ a space, $\Sigma^\infty X$ is the suspension spectrum of $X$
\item For $f:X\to Y$ a map, $\Sigma^\infty f:\Sigma^\infty X\to\Sigma^\infty Y$ is the map induced by the $n$th suspension of $f$ for all $n\in\N$. 
\end{itemize}
\end{defn}

\begin{prp}{}{} Let $X\in\bold{CG}_\ast$ be a space. There is an isomorphism $$\mathbb{S}\wedge X\cong\Sigma^\infty X$$
\end{prp}

\begin{defn}{The Loopspace Functor}{} Define the loopspace functor $\Omega^\infty:\mS^\N\to\bold{CG}_\ast$ as follows. 
\begin{itemize}
\item For $X=\{X_n\;|\;n\in\N\}$ a spectrum, $\Omega^\infty X=X_0$ returns the first space in the sequence. 
\item For $f:X\to Y$ a morphism, $\Omega^\infty f:\Omega^\infty X\to\Omega^\infty Y$ is the map on the $0$th level in the function of spectra. 
\end{itemize}
\end{defn}

\begin{thm}{}{} There is an adjunction given by $$\Sigma^\infty:\bold{CG}_\ast\rightleftarrows\mS^\N :\Omega^\infty$$ Explicitly, this means that there are isomorphisms $$\Hom_{\mS^\N }(\Sigma^\infty X,Y)\cong\Hom_{\bold{CG}_\ast}(X,\Omega^\infty Y)$$ that are natural in $X\in\bold{CG}_\ast$ and $Y\in\mS^\N $. 
\end{thm}

\begin{thm}{}{} The category of $\Omega$ spectra is equivalent to the category of generalized cohomology theories on $\bold{Top}_\ast$. 
\end{thm}

\pagebreak
\section{Modern Categories of Spectra}
\subsection{The Category of Orthogonal Spectra}
\subsection{The Category of Symmetric Spectra}
\subsection{The Category of CW Spectra}
\begin{defn}{Category of CW Spectra}{} The category of CW spectra $\mS$ is defined via the following data. 
\begin{itemize}
\item The objects of $\mS$ are CW Spectra
\item For $E,F$ two CW spectra, a morphism from $E$ to $F$ is a map $\varphi:E\to F$. 
\item Composition is defined as the composition of maps of spectra. 
\end{itemize}
\end{defn}

\begin{defn}{Homotopy Category of CW Spectra}{} The homotopy category of CW spectra $\mH\mS$ is defined via the following data. 
\begin{itemize}
\item The objects of $\mH\mS$ are CW Spectra
\item For $E,F$ two CW spectra, a morphism from $E$ to $F$ is a homotopy class of maps of spectra $[\varphi]:E\to F$. 
\item Composition is defined as the composition of maps of spectra. 
\end{itemize}
Isomorphisms in the category are called equivalences. If $E$ and $F$ are equivalent spectra then we denote it by $E\simeq F$. 
\end{defn}

\begin{thm}{}{} Let $E,F$ be CW spectra and let $\varphi:E\to F$ be a morphism of spectra. Then $\varphi:E\to F$ is an equivalence if and only if the induced homomorphism $$\varphi_\ast:\pi_k(E)\to\pi_k(F)$$ is an isomorphism for all $k$. 
\end{thm}

\begin{thm}{}{} Every spectrum is homotopy equivalent to a CW spectrum. Explicitly, for every spectrum $X=\{X_n,t_n\}$, there exists a CW spectrum $E=\{E_n,s_n\}$ together with pointed homotopy equivalences $f_n:E_n\to X_n$ such that the following diagram commutes: \\~\\
\adjustbox{scale=1,center}{\begin{tikzcd}
	{SE_n} & {SX_n} \\
	{E_{n+1}} & {X_{n+1}}
	\arrow["{Sf_n}", from=1-1, to=1-2]
	\arrow["{s_n}"', from=1-1, to=2-1]
	\arrow["{t_n}", from=1-2, to=2-2]
	\arrow["{f_{n+1}}"', from=2-1, to=2-2]
\end{tikzcd}}\\~\\
\end{thm}

\begin{thm}{}{} Every CW spectrum is equivalent to some $\Omega$ CW spectrum. 
\end{thm}

\pagebreak
\section{Formal Group Laws}
\begin{defn}{Formal Group Laws}{} Let $R$ be a ring. A formal group law over $R$ is a power series $$f(x,y)\in R[[x,y]]$$ such that the following are true. 
\begin{itemize}
\item $f(x,0)=f(0,x)=x$
\item $f(x,y)=f(y,x)$
\item $f(x,f(y,z))=f(f(x,y),z)$
\end{itemize}
\end{defn}

\begin{defn}{The Formal Group Law Functor}{} Define the formal group law functor $$FGL:\bold{Ring}\to\bold{Set}$$ by the following data. 
\begin{itemize}
\item For each ring $R$, $FGL(R)$ is the set of all formal group laws over $R$
\item For each ring homomorphism $f:R\to S$, $FGL(f)$ sends each formal group law $\sum_{i,j=0}^\infty c_{i,j}x^iy^j$ over $R$ to the formal group law $\sum_{i,j=0}^\infty f(c_{i,j})x^iy^j$ over $S$. 
\end{itemize}
\end{defn}

\begin{defn}{The Lazard Ring of a Formal Group Law}{} Define the lazard ring by $$L=\frac{\Z[c_{i,j}]}{Q}$$ where $Q$ is the ideal generated as follows. Write $f=\sum_{i,j=0}^\infty c_{i,j}x^iy^j$. Then $Q$ is generated by the constraints on $c_{i,j}$ for which $f$ becomes a formal group law. 
\end{defn}

\begin{lmm}{}{} The Lazard ring $L=\Z[c_{i,j}]/Q$ has the structure of a graded ring where $c_{i,j}$ has degree $2(i+j-1)$. 
\end{lmm}

\begin{thm}{}{} The formal group law functor $FGL:\bold{Ring}\to\bold{Set}$ is representable $$FGL(R)\cong\Hom_\bold{Ring}(L,R)$$ There exists a universal element $f\in L$ such that the map $\Hom_\bold{Ring}(L,R)\to FGL(R)$ given by evaluation on $f$ is a bijection for any ring $R$. 
\end{thm}

\begin{thm}{}{} There is an isomorphism of the Lazard ring $$L\cong\Z[t_1,t_2,\dots]$$ where each $t_k$ has degree $2k$. 
\end{thm}



\end{document}
