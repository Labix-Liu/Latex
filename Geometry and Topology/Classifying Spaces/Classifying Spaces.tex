\documentclass[a4paper]{article}

\input{C:/Users/liula/Desktop/Latex/Headers V1.2.tex}

\pagestyle{fancy}
\fancyhf{}
\rhead{Labix}
\lhead{Classifying Spaces}
\rfoot{\thepage}

\title{Classifying Spaces}

\author{Labix}

\date{\today}
\begin{document}
\maketitle
\begin{abstract}
\begin{itemize}
\item Notes on Algebraic Topology by Oscar Randal-Williams
\end{itemize}
\end{abstract}
\pagebreak
\tableofcontents

\pagebreak
\section{The Category of Fiber Bundles}
\subsection{Fiber Bundles}
\begin{defn}{Fiber Bundles}{} Let $E,B,F$ be spaces with $B$ connected, and $p:E\to B$ a trivial map. We say that $p$ is a fiber bundle over $F$ if the following are true. 
\begin{itemize}
\item $p^{-1}(b)\cong F$ for all $b\in B$
\item $p:E\to B$ is surjective
\item For every $x\in B$, there is an open neighbourhood $U\subset B$ of $x$ and a fiber preserving homomorphism $\Psi_U:p^{-1}(U)\to U\times F$ that is a homeomormorphism such that the following diagram commutes: \\~\\
\adjustbox{scale=1.0,center}{\begin{tikzcd}
	{p^{-1}(U)} && {U\times F} \\
	& U
	\arrow["{\Psi_U}", from=1-1, to=1-3]
	\arrow["p"', from=1-1, to=2-2]
	\arrow["\pi", from=1-3, to=2-2]
\end{tikzcd}}\\~\\
where $\pi$ is the projection by forgetting the second variable. 
\end{itemize}
We say that $B$ is the base space, $E$ the total space. It is denoted as $(F,E,B)$
\end{defn}

\begin{lmm}{}{} Every vector bundle is a fiber bundle. 
\end{lmm}

\begin{prp}{}{} Every fiber bundle is a Serre fibration. 
\end{prp}

We can provide a partial converse for the fact that every fiber bundle is a Serre fibration. 

\begin{prp}{}{} Let $p:E\to B$ be a fiber bundle. If $B$ is paracompact, then $p$ is a (Hurewicz) fibration. 
\end{prp}

\begin{defn}{Map of Fiber Bundles}{} Let $(F_1,E_1,B_1)$ and $(F_2,E_2,B_2)$ be fiber bundles. A morphism of fiber bundles is a pair of basepoint preserving continuous maps $(\tilde{f}:E_1\to E_2,f:B_1\to B_2)$ such that the following diagram commutes: \\~\\
\adjustbox{scale=1.0,center}{\begin{tikzcd}
	{E_1} & {E_2} \\
	{B_1} & {B_2}
	\arrow["{\tilde{f}}", from=1-1, to=1-2]
	\arrow["{p_1}"', from=1-1, to=2-1]
	\arrow["{p_2}", from=1-2, to=2-2]
	\arrow["f"', from=2-1, to=2-2]
\end{tikzcd}}\\~\\
Such a map of fibrations determine a continuous of the fibers $F_1\cong p_1^{-1}(b_1)\to p_2^{-1}(b_2)\cong F_2$. \\~\\

A map of fibrations $(\tilde{f},f)$ is said to be an isomorphism if there is a map $(\tilde{g}:E_2\to E_1,g:B_2\to B_1)$ such that $\tilde{g}$ is the inverse of $\tilde{f}$ and $g$ is the inverse of $f$. 
\end{defn}

\begin{defn}{Trivial Bundles}{} We say that a fiber bundle $(F,E,B)$ is trivial if $(F,E,B)$ is isomorphic to the trivial fibration $B\times F\to B$. 
\end{defn}

\begin{defn}{Sections}{} Let $(F,E,B)$ be a fiber bundle. A section on the fiber bundle is a map $s:B\to E$ such that $p\circ s=\text{id}_B$. Let $U\subset B$ be an open set. A local section of the fiber bundle on $U$ is a map $s:U\to B$ such that $p\circ s=\text{id}_U$. 
\end{defn}

\begin{defn}{The Pullback Bundle}{} Let $p:E\to B$ be a fiber bundle with fiber $F$. Let $f:B'\to B$ be a continuous function. Define the pullback of $p$ by $f$ to be the space $$f^\ast(E)=\{(b',e)\in B'\times E\;|\;p(e)=f(b')\}$$
\end{defn}

\subsection{G-Bundles and the Structure Groups}
Notice that for non empty intersections $U_i\cap U_j$ for $U_i,U_j$ open sets in $B$, there is a well defined homeomorphism $$\varphi_j\circ\varphi_i^{-1}:(U_i\cap U_j)\times F\to(U_i\cap U_j)\times F$$ This is reminiscent of properties of an atlas on $M$. 

\begin{defn}{G-Atlas}{} Let $(F,E,B)$ be a fiber bundle. Let $G$ be topological group with a continuous faithful action on $F$. A $G$-atlas on $(F,E,B)$ is a set of local trivalization charts $\{(U_k,\varphi_k)\;|\;k\in I\}$ such that the following are true. 
\begin{itemize}
\item For $(U_k,\varphi_k)$ a chart, define $\varphi_{i,x}:F\to F$ by $f\mapsto\varphi_i(x,f)$. Then the homeomorphism $$\varphi_{j,x}\circ\varphi_{i,x}^{-1}:F\to F$$ for $x\in U_i\cap U_j\neq\emptyset$ is an element of $G$. 
\item For $i,j\in I$, the map $g_{ij}:U_i\cap U_j\to G$ defined by $$g_{ij}(x)=\varphi_{j,x}\circ\varphi_{i,x}^{-1}$$ is continuous. 
\end{itemize}
\end{defn}

If $(F,E,B)$ is a fiber bundle with $F=\R$, then it is often seen that $G=GL(n,\R)$. Similarly, if $F=\C$ then the structure group is $G=GL(n,\C)$. 

\begin{defn}{Equivalent $G$-Atlas}{} Two $G$-atlases on a fiber bundle $(F,E,B)$ is said to be equivalent if their union is a $G$-atlas. 
\end{defn}

\begin{defn}{G-Bundle}{} Let $G$ be a topological group. A $G$-bundle is a fiber bundle $(F,E,B)$ together with an equivalence class of $G$-atlas. In this case, $G$ is said to be the structure group of the fiber bundle. 
\end{defn}

The structure group and the trivialization charts completely determine the isomorphism type of the fiber bundle. 

\begin{defn}{Morphisms of $G$-Bundles}{} Let $G$ be a topological group. A morphism of $G$-bundles is a morphism of fiber bundles $(\tilde{h},h):(F,E_1,B_1)\to(F,E_2,B_2)$ where the two are $G$-bundles, such that the following are true. 
\begin{itemize}
\item Let $U_i$ be open in $B_1$ and $V_j$ be open in $B_2$. Let $x\in U_u\cap h^{-1}(V_j)$. Let $\widetilde{h_{(E_1)_x}}:(E_1)_x\to(E_2)_{f(x)}$ be the map induced by $\tilde{h}:E_1\to E_2$. Then the map $$\varphi_{j,x}\circ\widetilde{h_{(E_1)_x}}\circ\varphi_{i,x}^{-1}:F\to F$$ is an element of $G$. 
\item The map $\widetilde{g_{ij}}:U_i\cap h^{-1}(V_j)\to G$ defined by $$\widetilde{g_{ij}}(x)=\varphi_{j,x}\circ\widetilde{h_{(E_1)_x}}\circ\varphi_{i,x}^{-1}$$ is continuous. 
\end{itemize}
\end{defn}

It is easy to see that the mapping transformations $\widetilde{g_{ij}}$ satisfy the following two relations: 
\begin{itemize}
\item $\widetilde{g_{jk}}(x)\cdot g_{ij}(x)=\widetilde{g_{ik}}(x)$ for all $x\in U_i\cap U_j\cap h^{-1}(V_k)$
\item $g_{jk}'(h(x))\cdot\widetilde{g_{ij}}(x)=\widetilde{g_{ik}}(x)$ for all $x\in U_i\cap h^{-1}(V_j\cap V_k)$
\end{itemize}

$g_{jk}'$ here refers to the transition charts in $(F,E_2,B_2)$. \\~\\

Just as the structure groups and trivialization charts determine the isomorphism type of a fiber bundle, the $\widetilde{g_{ij}}$ and a map of base space $h:B_1\to B_2$ completes determines a morphism of $G$-bundle. 

\begin{lmm}{}{} Let $(F,E_1,B_1)$ and $(F,E_2,B_2)$ be two $G$-bundles for a topological group $G$ with the same fiber $F$. Suppose that we have the following. 
\begin{itemize}
\item A map $h:B_1\to B_2$ of base space
\item $\widetilde{g_{ij}}:U_i\cap h^{-1}(V_j)\to G$ a set of continuous maps such that \begin{gather*}
\widetilde{g_{jk}}(x)\cdot g_{ij}(x)=\widetilde{g_{ik}}(x)\;\;\;\;\text{ for all }\;\;\;\;x\in U_i\cap U_j\cap h^{-1}(V_k)\\
g_{jk}'(h(x))\cdot\widetilde{g_{ij}}(x)=\widetilde{g_{ik}}(x)\;\;\;\;\text{ for all }\;\;\;\;x\in U_i\cap h^{-1}(V_j\cap V_k)
\end{gather*}
\end{itemize}
Then there exists a unique $G$-bundle morphism having $h$ as the map of base space and having $\{\widetilde{g_{ij}}\;|\;i,j\in I\}$ as its mapping transformations. 
\end{lmm}

\subsection{Principal G-Bundles}
\begin{defn}{Principal Bundles}{} Let $G$ be a topological group. A principal $G$-bundle is a $G$-bundle $(F,E,B)$ together with a continuous group action $G$ on $E$ such that the following are true. 
\begin{itemize}
\item The action of $G$ preserves fibers. This means that $g\cdot x\in E_b$ if $x\in E_b$. (This also means that $G$ is a group action on each fiber)
\item The action of $G$ on each fiber is free and transitive
\item For each $x\in E_b$, the map $G\to E_b$ defined by $g\mapsto g\cdot x$ is homeomorphism. 
\item Local triviality condition: If $\Psi_U:p^{-1}(U)\to U\times F$ are the local triviality maps, then each $\Psi_U$ are $G$-equivariant maps. 
\end{itemize}
\end{defn}

Note that since $G$ is homeomorphic to each fiber $E_b$ of the total space, we can think of the action of $G$ on the fiber simply becomes left  multiplication. \\~\\

For those who know what homogenous spaces are, principal bundles are $G$-bundles such that $F$ is a principal homogenous space for the left action of $G$ itself. \\~\\

Conversely, given a continuous group action on a space, we can ask in what conditions will the space be a principal bundle over the orbit space. 

\begin{prp}{}{} Let $E$ be a space with a free $G$ action. Let $p:E\to E/G$ be the projection map to the orbit space. If for all $x\in E/G$, there is a neighbourhood $U$ of $x$ and a continuous map $s:U\to E$ such that $p\circ s=\text{id}_U$, then $(G,E,E/G)$ is a principal $G$-bundle. 
\end{prp}

This proposition essentially means that if for each point in $E/G$, there is a local section, then it is sufficient for $E$ to be a principal $G$ bundle over $E/G$. 

\begin{thm}{}{} A principal $G$-bundle is trivial if and only if it admits a global section. 
\end{thm}

This is entirely untrue for general bundles. For examples, the zero section of a fiber bundle is a global section. 

\subsection{Classifying Space}
\begin{thm}{}{} Let $X,Y$ be spaces and let $f,g:X\to Y$ be homotopic maps. If $p:E\to B$ is a fiber bundle, then there is an isomorphism $$f^\ast(E)\cong g^\ast(E)$$
\end{thm}

This allows the principal bundles functor, defined below, to be well defined in homotopy classes of maps. 

\begin{defn}{Principal Bundle Functor}{} Let $G$ be a topological group and $X$ a space. Define a contravariant functor $\text{Prin}_G:\bold{hTop}\to\bold{Set}$ as follows. 
\begin{itemize}
\item For $X$ a topological space, $\text{Prin}_G(X)$ is the set of isomorphism classes of principal $G$-bundles over $X$. 
\item If $[f:X\to Y]$ is a homotopy class of continuous maps, $\text{Prin}_G([f]):\text{Prin}_G(Y)\to\text{Prin}_G(X)$ is defined as follows. If $[p:E\to Y]$ is an isomorphism class of principal $G$-bundles over $Y$, then it is sent to $[f^\ast(E)]$ the isomorphism class of the pullback of $p$. 
\end{itemize}
\end{defn}

\begin{thm}{}{} Let $G$ be ta topological group. Then the principal bundle functor is representable. Explicitly, this means that there exists a principal $G$-bundle $EG\to BG$ together with a natural isomorphism $$\psi:[X,BG]\to\text{Prin}_G(X)$$ This natural isomorphism is defined by $f\mapsto[f^\ast(EG)]$. 
\end{thm}

\begin{defn}{Universal G-Bundles}{} Let $G$ be a topological group. A principal $G$-bundle $(F,E,B)$ is said to be universal if it represents the principal bundle functor. 
\end{defn}

\begin{thm}{}{} Let $(F,E,B)$ be a principal $G$-bundle. If $E$ is contractible then $(F,E,B)$ is a universal $G$-bundle. 
\end{thm}

A surprising thing is that $BG$ is not determined by its isomorphism type but instead by the weaker condition of its homotopy type. 

\begin{thm}{}{} Let $(F,E_1,B_1)$ and $(F,E_2,B_2)$ be universal principal $G$-bundles. Then there exists a bundle map \\~\\
\adjustbox{scale=1.0,center}{\begin{tikzcd}
	{E_1} & {E_2} \\
	{B_1} & {B_2}
	\arrow["{\tilde{f}}", from=1-1, to=1-2]
	\arrow["{p_1}"', from=1-1, to=2-1]
	\arrow["{p_2}", from=1-2, to=2-2]
	\arrow["f"', from=2-1, to=2-2]
\end{tikzcd}}\\~\\
such that $f$ is a homotopy equivalence. In particular, this means that any two universal principal $G$-bundles are homotopy equivalent. 
\end{thm}

\begin{defn}{Classifying Space}{} Let $G$ be a topological group. The classifying space $BG$ of $G$ is the homotopy type of the universal principal $G$-bundle. Denote the total space of $BG$ by $EG$. For a principal $G$-bundle $f:Y\to X\in\text{Prin}_G(X)$, define the classifying map to be the associated map $X\to BG$ given in 1.5.3. 
\end{defn}

TBA: Functoriality of $B:\bold{Grp}\to\bold{Top}$. 

\pagebreak
\section{Vector Bundles as Principal Bundles}
\subsection{The Frame Bundle}
\begin{defn}{Frame Bundle}{}
\end{defn}

\begin{thm}{}{} Let $X$ be a space. Then there is a natural bijection $$\phi:\text{Prin}_{\text{GL}(n,\R)}(X)\overset{\cong}{\longrightarrow}\text{Vect}_n^\R(X)$$ given by mapping $p:E\to B$ to the frame bundle $F(E)$. Similarly, there is a natural bijection $$\phi:\text{Prin}_{\text{GL}(n,\C)}(X)\overset{\cong}{\longrightarrow}\text{Vect}_n^\C(X)$$
\end{thm}

\begin{thm}{}{} Let $n\in\N$, then there is an isomorphism in the classifying spaces $$B\text{GL}(n,\R)\cong BO(n)\cong\text{GL}_n(\R^\infty)$$
\end{thm}

\begin{thm}{}{} Let $n\in\N$, then there is an isomorphism in the classifying spaces $$B\text{GL}(n,\C)\cong BU(n)$$
\end{thm}

\begin{thm}{}{} Let $X$ be a paracompact space. Then there is a natural bijection $$\phi:\text{Prin}_{O(n)}(X)\overset{\cong}{\longrightarrow}\text{Vect}_n^\R(X)$$ given by mapping $p:E\to B$ to the frame bundle $F(E)$. Similarly, there is a natural bijection $$\phi:\text{Prin}_{U(n)}(X)\overset{\cong}{\longrightarrow}\text{Vect}_n^\C(X)$$
\end{thm}

\subsection{The Tautological Bundle}

\subsection{The Thom Isomorphism}
\begin{defn}{Unit Sphere and Unit Disc Bundle}{} Let $p:E\to B$ be an $n$-dimensional vector bundle over $\R$. Let $\langle-,-\rangle:E\times E\to\R$ be a smoothly varying inner product on $E$. Define the disc bundle to be $$D(E)=\{e\in E\;|\;\langle e,e\rangle\leq 1\}$$ together with the map $p|_{D(E)}:D(E)\to B$. Define the sphere bundle to be $$S(E)=\{e\in E\;|\;\langle e,e\rangle=1\}$$ together with the map $p|_{S(E)}:S(E)\to B$. 
\end{defn}

\begin{defn}{Thom Space}{} Let $p:E\to B$ be an $n$-dimensional vector bundle over $\R$ such that $B$ is paracompact. Define the Thom space of $E$ to be $$\frac{D(E)}{S(E)}$$ The base point is taken as the equivalent class $S(E)$ if needed. 
\end{defn}

\begin{thm}{The Thom Isomorphism}{} Let $p:E\to B$ be an $n$-dimensional vector bundle over $\R$. Let $E_0$ denote the zero section of $E$. Then there exists a unique $u\in H^n(E,E\setminus E_0;\Z/2\Z)$ such that $$u|_{(F_b,F_b\setminus\{0\})}\in H^n(F_b,F_b\setminus\{0\};\Z/2\Z)$$ is non-zero for all $b\in B$. Moreover, there is an isomorphism $$\Phi:H^k(E;\Z/2\Z)\to\widetilde{H}^{k+n}(E,E\setminus E_0;\Z/2\Z)$$ given by $y\mapsto y\smile u$ for all $k\in\Z$. 
\end{thm}

Ref: Milnor

\subsection{Orientation of a Bundle}
\begin{defn}{Orientation of a Vector Space}{} Let $V$ be a finite dimensional vector space over $F$. An orientation on $V$ is an equivalence class of bases, where we say that two ordered bases $\{v_1,\dots,v_n\}$ and $\{w_1,\dots,w_n\}$ are equivalent if the matrix defined by the equations $$w_i=\sum_{k=0}^na_kv_k$$ has positive determinant. 
\end{defn}

\begin{lmm}{}{} Let $V$ be a finite dimensional vector space. Then there are only two possible orientations on $V$. 
\end{lmm}

\begin{defn}{}{} Let $p:E\to B$ be a vector bundle with fiber $F$. An orientation on $E$ is an assignment of an orientation to each fiber of $E$ such that the following local compatibility condition is satisfied. \\~\\

For every $b\in B$, there exists a local coordinate system $(U,\varphi)$ of $b$ and $\varphi:U\times\R^n\to p^{-1}(U)$ such that for all $x\in U$, the homomorphism $\varphi(b,-):\R^n\to F$ is orientation preserving. 
\end{defn}

\begin{thm}{}{} Let $p:E\to B$ be a vector bundle with fiber $F$. An orientation on $E$ is equivalent to the following data. To each $b\in B$ there is assignment $$u_b\in H^n(F_b,F_b\setminus\{0\};\Z)$$ called the orientation class of $F_b$, such that for every $b\in B$, there exists a neighbourhood $U$ of $b$ and a cohomology class $$u\in H^n(p^{-1}(U),p^{-1}(U)\setminus 0;)$$ where $0$ is the zero section such that for every $x\in U$, $$u|_{(F_x,F_x\setminus\{0\})}\in H^n(F_x,F_x\setminus\{0\}\;\Z)$$ is equal to $u_b$. 
\end{thm}

\begin{thm}{The Thom Isomorphism}{} Let $p:E\to B$ be an orientable $n$-dimensional vector bundle over $\R$. Let $R$ be a ring. Let $E_0$ denote the zero section of $E$. Then there exists a unique $u\in H^n(E,E\setminus E_0;R)$ such that $$u|_{(F_b,F_b\setminus\{0\})}\in H^n(F_b,F_b\setminus\{0\};R)$$ gives precisely the orientation class on $F_b$ for all $b\in B$. Moreover, there is an isomorphism $$\Phi:H^k(E;R)\to\widetilde{H}^{k+n}(E,E\setminus E_0;R)$$ given by $y\mapsto y\smile u$ for all $k\in\Z$. 
\end{thm}

\pagebreak
\section{Characteristic Classes}
\subsection{Characteristic Classes as a Ring}
\begin{defn}{Characteristic Classes}{} Let $G$ be a topological group and $X$ a space. Denote $\text{Prin}_G(X)$ the isomorphism classes of principal $G$-bundles over $X$. Let $H^\ast(-)$ be a cohomology functor. A characteristic class for $G$ is a natural transformation $c$ from $\text{Prin}_G(-)$ to $H^\ast(-)$. \\~\\

Explicitly, if $p:E\to X$ is a principal $G$-bundle, then $c$ assigns $p$ to the collection of cohomology groups $c(p)\in H^\ast(X)$. 
\end{defn}

Here cohomology can be taken for example singular cohomology with coefficients in a fixed group. 

\begin{lmm}{}{} Let $G$ be a topological group. Let $c$ be a characteristic class for $G$. If $e$ is the trivial $G$-bundle, then $c(e)=0$. 
\end{lmm}

\begin{defn}{Ring of Characteristic Classes}{} Let $G$ be a topological group. Let $R$ be a commutative ring. Define $\text{Char}_G(R)$ to be the set of all characteristic classes for principal $G$-bundles that take values in $H^\ast(-;R)$. 
\end{defn}

\begin{prp}{}{} Let $G$ be a topological group. Let $R$ be a commutative ring. Then $\text{Char}_G(R)$ is a ring with unit the constant characteristic class. 
\end{prp}

\begin{thm}{}{} Let $G$ be a topological group and let $R$ be a commutative ring. Then there is an isomorphism $$\text{Char}_G(R)\cong H^\ast(BG;R)$$
\end{thm}

\subsection{The Stiefel-Whiteny Class}
\begin{defn}{The Stiefel-Whitney Class}{} Consider the group $O(n)$. We say that a characteristic class $w:\text{Prin}_{O(n)}(-)\to H^\ast(-,\Z/2\Z)$ for $O(n)$ is a Stiefel-Whitney Class if the following are satisfied. 
\begin{enumerate}
\item Rank: If $E$ is a principal $O(n)$-bundle, then $w_0(E)=1$ and $w_i(E)=0$ for $i>\rank(E)$. 
\item Naturality: Let $p:E\to X$ be a principal $O(n)$-bundle and let $f:Y\to X$ be a map. Then $$w_i(f^\ast(E))=f^\ast(w_i(E))$$
\item Whitney Product Formula: If $E_1,E_2$ are principal $O(n)$-bundles, then $$w_k(E_1\oplus E_2)=\sum_{i=0}^kw_i(E_1)\smile w_{k-i}(E_2)$$
\item Normalization: If $\gamma$ is the tautological line bundle over $\Prj^1(\R)$, then $w_1(\gamma)$ is non-zero. 
\end{enumerate}
\end{defn}

TBA: Existence and uniqueness

\begin{prp}{}{} The following are true regarding the Stiefel-Whitney class. 
\begin{itemize}
\item If $p_1:E_1\to B_1$ and $p_2:E_2\to B_2$ are isomorphic principal $O(n)$-bundles, then $w(E_1)=w(E_2)$
\item If $e=B\otimes\R^n$ is the trivial bundle, then $w(e\oplus E)=w(E)$ for any principal $O(n)$-bundle $E$. 
\item 
\end{itemize}
\end{prp}

\begin{thm}{}{} Let $n\in\N$, then the ring of characteristic classes of $O(n)$ is isomorphic to $$\text{Char}_{O(n)}(\Z/2\Z)\cong\Z/2\Z[c_1,\dots,c_n]$$ a polynomial ring in $n$ variables for $w_i\in H^i(BO(n),\Z/2\Z)$. 
\end{thm}

\subsection{The Chern Class}


\begin{thm}{}{} Let $E$ be an $n$-dimensional complex vector bundle over $X$. Then $c_1(E)=0$ if and only if $E$ has an $SU(n)$-structure. 
\end{thm}

TBA: First chern class is complete invariant of complex line bundles. First Stiefel-Whitney class is a complete invariant of real line bundle. 

\begin{thm}{}{} Let $n\in\N$, then the ring of characteristic classes of $U(n)$ is isomorphic to $$\text{Char}_{U(n)}(\Z)\cong\Z[c_1,\dots,c_n]$$ a polynomial ring in $n$ variables for $c_i\in H^{2i}(BU(n),\Z)$. 
\end{thm}

\subsection{The Euler Class}
\begin{defn}{The Euler Class}{} Let $p:E\to B$ be an $n$-dimensional orientable vector bundle over $\R$. Let $E_0\subseteq E$ denote the zero section. Consider the inclusion $B\hookrightarrow E$ as $E_0$. Let $u\in H^n(E,E\setminus E_0;\Z)$ be the orientation class. Define the euler class of $E$ $$e(E)\in H^n(B;\Z)$$ to be the image of $u$ under the compositions \\~\\
\adjustbox{scale=1.0,center}{\begin{tikzcd}
	{H^n(E,E\setminus E_0;\Z)} & {H^n(E,\Z)} & {H^n(B;\Z)}
	\arrow[from=1-1, to=1-2]
	\arrow[from=1-2, to=1-3]
\end{tikzcd}}\\~\\
that is induced by the sequence of inclusions $(B,\emptyset)\hookrightarrow(E,\emptyset)\hookrightarrow(E,E\setminus E_0)$.
\end{defn}

\begin{prp}{}{} Let $p:E\to B$ be an $n$-dimensional orientable vector bundle over $\R$. Then the following are true regarding the Euler class. 
\begin{itemize}
\item If $f:C\to B$ is a map, then $e(f^\ast(E))=f^\ast(e(E))$
\item If the orientation of $E$ is reversed, then $e(E)$ changes sign. 
\item If $F$ is another orientable vector bundle, then $e(E\oplus F)=e(E)\smile e(F)$. 
\end{itemize}
\end{prp}

\begin{prp} Let $p:E\to B$ be an orientable vector bundle over $\R$. If the dimension of the bundle is odd, then $2e(E)=0$. 
\end{prp}

\begin{prp}{}{} Let $p:E\to B$ be an $n$-dimensional orientable vector bundle over $\R$. The natural homomorphism $$H^n(B;\Z)\to H^n(B;\Z/2\Z)$$ sends the Euler class $e(E)$ to the top Stiefel-Whitney class $w_n(E)$. 
\end{prp}

\begin{prp}{}{} Let $p:E\to B$ be an $n$-dimensional orientable vector bundle over $\R$. If $E$ possess a nowhere $0$ section, then $e(E)=0$. 
\end{prp}

\subsection{The Pontrjagin Class}

\pagebreak
\section{Obstruction Theory}

\end{document}
