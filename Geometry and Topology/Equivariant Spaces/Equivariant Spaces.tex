\documentclass[a4paper]{article}

%=========================================
% Packages
%=========================================
\usepackage{mathtools}
\usepackage{amsfonts}
\usepackage{amsmath}
\usepackage{amssymb}
\usepackage{amsthm}
\usepackage[a4paper, total={6in, 8in}, margin=1in]{geometry}
\usepackage[utf8]{inputenc}
\usepackage{fancyhdr}
\usepackage[utf8]{inputenc}
\usepackage{graphicx}
\usepackage{physics}
\usepackage[listings]{tcolorbox}
\usepackage{hyperref}
\usepackage{tikz-cd}
\usepackage{adjustbox}
\usepackage{enumitem}
\usepackage[font=small,labelfont=bf]{caption}
\usepackage{subcaption}
\usepackage{wrapfig}
\usepackage{makecell}



\raggedright

\usetikzlibrary{arrows.meta}

\DeclarePairedDelimiter\ceil{\lceil}{\rceil}
\DeclarePairedDelimiter\floor{\lfloor}{\rfloor}

%=========================================
% Fonts
%=========================================
\usepackage{tgpagella}
\usepackage[T1]{fontenc}


%=========================================
% Custom Math Operators
%=========================================
\DeclareMathOperator{\adj}{adj}
\DeclareMathOperator{\im}{im}
\DeclareMathOperator{\nullity}{nullity}
\DeclareMathOperator{\sign}{sign}
\DeclareMathOperator{\dom}{dom}
\DeclareMathOperator{\lcm}{lcm}
\DeclareMathOperator{\ran}{ran}
\DeclareMathOperator{\ext}{Ext}
\DeclareMathOperator{\dist}{dist}
\DeclareMathOperator{\diam}{diam}
\DeclareMathOperator{\aut}{Aut}
\DeclareMathOperator{\inn}{Inn}
\DeclareMathOperator{\syl}{Syl}
\DeclareMathOperator{\edo}{End}
\DeclareMathOperator{\cov}{Cov}
\DeclareMathOperator{\vari}{Var}
\DeclareMathOperator{\cha}{char}
\DeclareMathOperator{\Span}{span}
\DeclareMathOperator{\ord}{ord}
\DeclareMathOperator{\res}{res}
\DeclareMathOperator{\Hom}{Hom}
\DeclareMathOperator{\Mor}{Mor}
\DeclareMathOperator{\coker}{coker}
\DeclareMathOperator{\Obj}{Obj}
\DeclareMathOperator{\id}{id}
\DeclareMathOperator{\GL}{GL}
\DeclareMathOperator*{\colim}{colim}

%=========================================
% Custom Commands (Shortcuts)
%=========================================
\newcommand{\CP}{\mathbb{CP}}
\newcommand{\GG}{\mathbb{G}}
\newcommand{\F}{\mathbb{F}}
\newcommand{\N}{\mathbb{N}}
\newcommand{\Q}{\mathbb{Q}}
\newcommand{\R}{\mathbb{R}}
\newcommand{\C}{\mathbb{C}}
\newcommand{\E}{\mathbb{E}}
\newcommand{\Prj}{\mathbb{P}}
\newcommand{\RP}{\mathbb{RP}}
\newcommand{\T}{\mathbb{T}}
\newcommand{\Z}{\mathbb{Z}}
\newcommand{\A}{\mathbb{A}}
\renewcommand{\H}{\mathbb{H}}
\newcommand{\K}{\mathbb{K}}

\newcommand{\mA}{\mathcal{A}}
\newcommand{\mB}{\mathcal{B}}
\newcommand{\mC}{\mathcal{C}}
\newcommand{\mD}{\mathcal{D}}
\newcommand{\mE}{\mathcal{E}}
\newcommand{\mF}{\mathcal{F}}
\newcommand{\mG}{\mathcal{G}}
\newcommand{\mH}{\mathcal{H}}
\newcommand{\mI}{\mathcal{I}}
\newcommand{\mJ}{\mathcal{J}}
\newcommand{\mK}{\mathcal{K}}
\newcommand{\mL}{\mathcal{L}}
\newcommand{\mM}{\mathcal{M}}
\newcommand{\mO}{\mathcal{O}}
\newcommand{\mP}{\mathcal{P}}
\newcommand{\mS}{\mathcal{S}}
\newcommand{\mT}{\mathcal{T}}
\newcommand{\mV}{\mathcal{V}}
\newcommand{\mW}{\mathcal{W}}

%=========================================
% Colours!!!
%=========================================
\definecolor{LightBlue}{HTML}{2D64A6}
\definecolor{ForestGreen}{HTML}{4BA150}
\definecolor{DarkBlue}{HTML}{000080}
\definecolor{LightPurple}{HTML}{cc99ff}
\definecolor{LightOrange}{HTML}{ffc34d}
\definecolor{Buff}{HTML}{DDAE7E}
\definecolor{Sunset}{HTML}{F2C57C}
\definecolor{Wenge}{HTML}{584B53}
\definecolor{Coolgray}{HTML}{9098CB}
\definecolor{Lavender}{HTML}{D6E3F8}
\definecolor{Glaucous}{HTML}{828BC4}
\definecolor{Mauve}{HTML}{C7A8F0}
\definecolor{Darkred}{HTML}{880808}
\definecolor{Beaver}{HTML}{9A8873}
\definecolor{UltraViolet}{HTML}{52489C}



%=========================================
% Theorem Environment
%=========================================
\tcbuselibrary{listings, theorems, breakable, skins}

\newtcbtheorem[number within = subsection]{thm}{Theorem}%
{	colback=Buff!3, 
	colframe=Buff, 
	fonttitle=\bfseries, 
	breakable, 
	enhanced jigsaw, 
	halign=left
}{thm}

\newtcbtheorem[number within=subsection, use counter from=thm]{defn}{Definition}%
{  colback=cyan!1,
    colframe=cyan!50!black,
	fonttitle=\bfseries, breakable, 
	enhanced jigsaw, 
	halign=left
}{defn}

\newtcbtheorem[number within=subsection, use counter from=thm]{axm}{Axiom}%
{	colback=red!5, 
	colframe=Darkred, 
	fonttitle=\bfseries, 
	breakable, 
	enhanced jigsaw, 
	halign=left
}{axm}

\newtcbtheorem[number within=subsection, use counter from=thm]{prp}{Proposition}%
{	colback=LightBlue!3, 
	colframe=Glaucous, 
	fonttitle=\bfseries, 
	breakable, 
	enhanced jigsaw, 
	halign=left
}{prp}

\newtcbtheorem[number within=subsection, use counter from=thm]{lmm}{Lemma}%
{	colback=LightBlue!3, 
	colframe=LightBlue!60, 
	fonttitle=\bfseries, 
	breakable, 
	enhanced jigsaw, 
	halign=left
}{lmm}

\newtcbtheorem[number within=subsection, use counter from=thm]{crl}{Corollary}%
{	colback=LightBlue!3, 
	colframe=LightBlue!60, 
	fonttitle=\bfseries, 
	breakable, 
	enhanced jigsaw, 
	halign=left
}{crl}

\newtcbtheorem[number within=subsection, use counter from=thm]{eg}{Example}%
{	colback=Beaver!5, 
	colframe=Beaver, 
	fonttitle=\bfseries, 
	breakable, 
	enhanced jigsaw, 
	halign=left
}{eg}

\newtcbtheorem[number within=subsection, use counter from=thm]{ex}{Exercise}%
{	colback=Beaver!5, 
	colframe=Beaver, 
	fonttitle=\bfseries, 
	breakable, 
	enhanced jigsaw, 
	halign=left
}{ex}

\newtcbtheorem[number within=subsection, use counter from=thm]{alg}{Algorithm}%
{	colback=UltraViolet!5, 
	colframe=UltraViolet, 
	fonttitle=\bfseries, 
	breakable, 
	enhanced jigsaw, 
	halign=left
}{alg}




%=========================================
% Hyperlinks
%=========================================
\hypersetup{
    colorlinks=true, %set true if you want colored links
    linktoc=all,     %set to all if you want both sections and subsections linked
    linkcolor=DarkBlue,  %choose some color if you want links to stand out
}


\pagestyle{fancy}
\fancyhf{}
\rhead{Labix}
\lhead{Equivariant Spaces}
\rfoot{\thepage}

\title{Equivariant Spaces}

\author{Labix}

\date{\today}
\begin{document}
\maketitle
\begin{abstract}
\end{abstract}

\pagebreak
\tableofcontents
\pagebreak

\pagebreak
\section{Topological Groups and G-Spaces}
The above four chapters has established deep connections between three properties of a space. Namely, the fundamental group, the fibers of the covering space and the group of homeomorphisms of the covering space. Such a deep connection between algebra and topology is not unique to covering space theory, nor to fundamental groups. In this section we will take a step back and look at the big picture. 

\subsection{Topological Groups and the Coset Space}
\begin{defn}{Topological Groups}{} Let $G$ be a group. We say that $G$ is a topological group if $G$ is also a topological space and that the following are true. 
\begin{itemize}
\item The multiplication map $\cdot:G\times G\to G$ defined by $(g,h)\mapsto gh$ is continuous. 
\item The inverse map $(-)^{-1}:G\to G$ defined by $g\mapsto g^{-1}$ is continuous. 
\end{itemize}
\end{defn}

Notice that every group can be given the discrete topology, and so every group is trivially a topological group. But of course there is no guarantee that anything interesting theorems will occur in this case. We call these topological groups discrete. 

\begin{defn}{Discrete Group}{} Let $G$ be a topological group. We say that $G$ is a discrete group if it has the discrete topology. 
\end{defn}

\begin{prp}{}{} Let $G$ be a topological group. Let $H$ be a subgroup of $G$. Then $H$ and $\overline{H}$ are both topological groups. Moreover, if $H$ is normal, then $\overline{H}$ is normal. 
\end{prp}

\begin{prp}{}{} Let $G$ be a topological group. Let $H$ be a subgroup of $G$. Then the normalizer $N_G(H)$ and the centralizer $C_G(H)$ are closed subgroups of $G$. 
\end{prp}

\begin{defn}{The Coset Space}{} Let $G$ be a topological group and $H$ a closed subgroup of $G$. Define the coset space of $H$ in $G$ to be the quotient space $$G/H=\{gH\;|\;g\in G\}$$ together with the (topological) quotient map $p:G\to G/H$ such that $U\subseteq G/H$ is open if and only if $p^{-1}(U)$ is open. 
\end{defn}

\begin{thm}{}{} Let $G$ be a topological group. Let $H$ be a subgroup of $G$. Then the (topological) quotient map $$p:G\to G/H$$ is an open map. Moreover, the following are true regarding the quotient. 
\begin{itemize}
\item $G/H$ is Hausdorff if and only if $H$ is closed in $G$
\item $G/H$ is discrete if and only if $H$ is open in $G$. 
\item If $H$ is normal and closed in $G$, then $G/H$ is a topological group. 
\end{itemize}
\end{thm}

\subsection{Morphisms Between Topological Groups}
\begin{defn}{Continuous Homomorphisms}{} Let $G$ and $H$ be topological groups. A function $f:G\to H$ is said to be a continuous homomorphism if it is continuous and a group homomorphism. 
\end{defn}

\begin{prp}{}{} Let $G,H$ be topological groups. Let $\varphi:G\to H$ be a surjective continuous homomorphism. Then $\ker(\varphi)$ is a closed subgroup of $G$ and $\varphi$ is a continuous bijection. 
\end{prp}

When the topological group $G$ is compact, the first isomorphism theorem in fact gives a homeomorphism. 

\begin{prp}{}{} Let $G,H$ be topological groups. Let $\varphi:G\to H$ be a surjective continuous homomorphism. If $G$ is compact, then $$\overline{\varphi}:\frac{G}{\ker(\varphi)}\to H$$ is a homeomomorphism. 
\end{prp}

\begin{prp}{}{} Let $G$ be a compact topological group. Let $g\in G$. Then $$A=\overline{\{g^n\;|\;n\in\N\}}$$ is a subgroup of $G$. 
\end{prp}

\subsection{Some Real Topological Groups}
\begin{prp}{The Group of Matrices as a Topological Group}{} The group of real $n\times n$ matrices $$M_n(\R)=\{(a_{ij})_{n\times n}\;|\;a_{ij}\in\R\}$$ with group structure given by matrix addition can be given the structure of a topological group whose topology is given by the topology of $\R^{n^2}$ and a choice of bijection of sets $\R^{n^2}\cong M_n(\R)$. 
\end{prp}

\begin{prp}{}{} Let $n\in\N\setminus\{0\}$. Then $\GL(n,\R)$ is an open subset of $M_n(\R)$. 
\end{prp}

Beware that $\GL(n,\R),O(n,\R),SL(n,\R),SO(n,\R)$ are not topological groups as subsets of $M_n(\R)$ because they are not subgroups of $M_n(\R)$ in the first place. 

\begin{prp}{The General Linear Group as a Topological Group}{} The general linear group $$\GL(n,\R)=\{M\in M_n(\R)\;|\;\det(M)\neq 0\}$$ with group structure given by matrix multiplication can be given the structure of a topological group whose topology is given by the subspace topology as a subset of $M_n(\R)$. 
\end{prp}

\begin{prp}{}{} Let $n\in\N\setminus\{0\}$. Then the following are all topological groups with group structure and topology inherited from $\GL(n,\R)$. 
\begin{itemize}
\item The orthogonal group $O(n,\R)$. 
\item The special linear group $SL(n,\R)$. 
\item The special orthogonal group $SO(n,\R)$. 
\end{itemize}
Moreover, they are closed subsets of $\GL(n,\R)$. 
\end{prp}

\subsection{Some Complex Topological Groups}

$\GL(n,\C), U(n), SU(n)$

\pagebreak
\section{Equivariant Spaces}
\subsection{G-Spaces and G-Equivariant Maps}
In algebraic topology, we have the results of considering groups acting on spaces. We can in fact consider topological groups acting on spaces. 

\begin{defn}{Continuous Group Actions}{} Let $G$ be a topological group and $X$ a space. We say that $G$ is a continuous group action if $G$ is a group acting on $X$ such that the group action map $$\cdot:G\times X\to X$$ is continuous. In this case we say that $X$ is a $G$-space. 
\end{defn}

Frequently a continuous group action is also called a (topological) transformation group, for example in Milnor's Topology of Fiber Bundles or Introduction to Compact Topological Groups. 

\begin{prp}{}{} Let $G$ be a continuous group action of $X$. Then for each $g\in G$, the left action map $x\mapsto g\cdot x$ is a homeomorphism of $X$. \tcbline
\begin{proof}
Every element of $g$ has an inverse $g^{-1}$ which are both continuous and are bijections on $X$. 
\end{proof}
\end{prp}

\begin{prp}{}{} Let $G$ be a topological group and $(X,\mathcal{T})$ a topological space. Then $G$ is a continuous group action on $X$ if and only if $G$ acts on $\mathcal{T}$. \tcbline
\begin{proof}
Suppose that $G$ is a continuous group action on $X$. Then for each $g\in G$, $g\cdot U=\{g\cdot x\;|\; x\in U\}$ for $U\in\mathcal{T}$ is open since $A_g$ as above is a homeomorphism. Now suppose that $G$ acts on $\mathcal{T}$. Then for each open set $U$ of $X$, $g^{-1}\cdot U$ is open. Thus $G$ is a continuous group action. 
\end{proof}
\end{prp}

In particular, some authors would assume one knows this fact, so it is always nice to see it spelled out. It is also standard to denote this action just by the element $g$ instead of $A_g$. 

\begin{defn}{Group of Homeomorphisms}{} Let $X$ be a space. Define the group of homeomorphisms of $X$ to be $$\text{Homeo}(X)=\{f:X\to X\;|\;f\text{ is a homeomorphism}\}$$ together with composition of functions. We say that a group $A$ is a subgroup of homeomorphisms of $X$ if $A$ is isomorphic to a subgroup of $\text{Homeo}(X)$. 
\end{defn}

\begin{lmm}{}{} Let $G$ be a topological group. Let $X$ be a $G$-space. Then there is a group homomorphism $\varphi:G\to\text{Homeo}(X)$ defined by $$g\mapsto\left(x\mapsto g\cdot x\right)$$ \tcbline
\begin{proof}
We have already seen that for any $g\in G$, the map $x\mapsto g\cdot x$ is a homeomorphism. Thus the above mapping is well defined. Now we have that 
\begin{align*}
\varphi(gh)(x)&=gh\cdot x\\
&=g\cdot(h\cdot x)\\
&=(\varphi(g)\circ\varphi(h))(x)
\end{align*}
and so $\varphi$ is a group homomorphism. 
\end{proof}
\end{lmm}

Notice that if the above group homomorphism is injective, then the structure group $G$ is a subgroup of homeomorphisms of $G$. 

\begin{defn}{G-Equivariant Maps}{} Let $G$ be a topological group and let $X,Y$ be $G$-spaces. A $G$-equivariant map is a continuous map $f:X\to Y$ such that $f$ is equivariant. In other words, we require that $$f(g\cdot x)=g\cdot f(x)$$ for all $x\in X$ and all $g\in G$. 
\end{defn}

\begin{defn}{Isomorphic G-Spaces}{} Let $G$ be a topological group and let $X,Y$ be $G$-space. We say that $G$ and $H$ are isomorphic $G$-spaces if there exists a $G$-equivariant map such that $f$ is a homeomorphism. 
\end{defn}

\begin{thm}{}{} Let $G$ be a topological group and let $X$ be a $G$-space. Then the map $$p:\frac{G}{\text{Stab}_G(x_0)}\to Gx_0\subseteq X$$ induced by the map $g\mapsto g\cdot x_0$ is well defined. Moreover, it is isomorphic to the left $G$-space $Gx_0$. \tcbline
\begin{proof}
To show that it s well defined, we want to show that if $g\in\text{Stab}_G(x_0)$, then $g\cdot x_0=x_0$. But this is true by definition of the stabilizer. By definition of the induced map, it is continuous. Also, the orbit of $x_0$ is precisely $Gx_0$ and hence $p$ is a bijection. It remains to show that $p$ is an open map. \\~\\

To show isomorphism, we also need to show that $p$ is a $G$-equivariant map. We have that 
\begin{align*}
p(g\cdot(h\text{Stab}_G(x_0)))&=p(gh\text{Stab}_G(x_0))\\
&=(gh)\cdot x_0\\
&=g\cdot(h\cdot x_0)\\
&=g\cdot p(h\text{Stab}_G(x_0))
\end{align*}
so that $p$ is $G$-equivariant. 
\end{proof}
\end{thm}

\begin{defn}{The Category of G-Spaces}{} Let $G$ be a topological space. Define the category of $G$-spaces $${_G}\bold{Top}$$ to consist of the following data. 
\begin{itemize}
\item The objects are the $G$-spaces
\item The morphisms are the $G$-equivariant spaces
\item Composition is given by the composition of functions. 
\end{itemize}
\end{defn}

There is an obvious forgetful functor ${_G}\bold{Top}\to\bold{Top}$. One of its adjoint should assign the space to a trivial $G$-action. 

\begin{defn}{The Trivial G-Space Functor}{} Let $G$ be a topological group. Define the trivial $G$-space functor $$\text{Triv}:\bold{Top}\to{_G\bold{Top}}$$ by the following. 
\begin{itemize}
\item For each space $X$, define a group action on $X$ by $g\cdot x=x$ for all $g\in G$ and $x\in X$. 
\item For each map $f:X\to Y$, $\text{Triv}(f)=f$ because $f$ is trivially equivariant. 
\end{itemize}
\end{defn}

\subsection{Induced and Restricted G-Spaces}
\begin{defn}{Induced G-Spaces}{} Let $G$ be a topological group. Let $H\leq G$ be a subgroup. Let $X$ be an $H$-space. Define the induced $G$-space of $X$ to be the space $$\text{Ind}_H^GX=G\times_HX=\frac{G\times X}{\sim}$$ where the relation is generated by $(g\cdot h,x)\sim(g,h\cdot x)$ for $g\in G$, $h\in H$ and $x\in X$. 
\end{defn}

Pushout?

\begin{lmm}{}{} Let $G$ be a topological group. Let $H\leq G$ be a subgroup. Let $X$ be an $H$-space. Then the following are true. 
\begin{itemize}
\item If $X=\ast$, then $G\times_H X\cong\frac{G}{H}$. 
\item If $h\cdot x=x$ for all $h\in H$ (the action of $H$ is trivial), then $G\times_HX\cong\frac{G}{H}\times X$. 
\end{itemize}
\end{lmm}

\begin{defn}{Restricted G-Spaces}{} Let $G$ be a topological group. Let $H\leq G$ be a subgroup. Let $X$ be a $G$-space. Define the restriction $$\text{Res}_H^GX$$ of $X$ to $H$ to be the space $X$ considered as an $H$-space by the group action of $G$. 
\end{defn}

\begin{prp}{}{} Let $G$ be a topological group. Let $H\leq G$ be a subgroup. Then there is an adjunction $$\text{Ind}_H^G:{_H\bold{Top}}\rightleftarrows{_G\bold{Top}}:\text{Res}_H^G$$ This means that there is an isomorphism $$\Hom_{_G\bold{Top}}(\text{Ind}_H^GX,Y)\cong\Hom_{_H\bold{Top}}(X,\text{Res}_H^GY)$$ that is natural in $X$ and $Y$. 
\end{prp}

\subsection{Fixed Points and Orbit Spaces}
\begin{defn}{The Fixed Points Functor}{} Let $G$ be a topological group. Define the fixed points functor $$(-)^G:{_G\bold{Top}}\to\bold{Top}$$ by the following. 
\begin{itemize}
\item For each $G$-space, $X$, $X^G=\{x\in X\;|\;g\cdot x=x\text{ for all }g\in G\}$ is the subset of fixed points of $G$ equipped with the subspace topology. 
\item For each $G$-equivariant map $f:X:\to Y$, $(f)^G:X^G\to Y^G$ is the restriction of $f$ to $X^G$. 
\end{itemize}
\end{defn}

Check: it is well defined.

\begin{prp}{}{} Let $G$ be a topological group. There is an adjunction $$\text{Triv}:\bold{Top}\rightleftarrows{_G\bold{Top}}:(-)^G$$ This means that there is an isomorphism $$\Hom_{_G\bold{Top}}(\text{Triv}X,Y)\cong\Hom_\bold{Top}(X,Y^G)$$ that is natural in $X$ and $Y$. 
\end{prp}

\begin{defn}{The Orbit Space}{} Let $X$ be a space and $G$ be a group acting on $X$. Define the orbit space of $X$ and $G$ to be $$\frac{X}{G}=\{\text{Orb}_G(x)\;|\; x\in X\}$$ the set of all orbits of $G$ on $X$, inherited with the quotient topology of the equivalence relation of orbits. 
\end{defn}

This has the quotient topology because recall from groups and rings that $\text{Orb}_G(x)$ defines an equivalence relation on $X$. 

\subsection{The Pointed Analogue}
\begin{defn}{Pointed G-Spaces}{} Let $G$ be a topological group. A pointed $G$-space is a space $X$ together with a $G$-equivariant map $\ast\to X$. We denote it by $(X,x_0)$ where $x_0$ is the image of the map $\ast\to X$. 
\end{defn}

\begin{lmm}{}{} Let $G$ be a topological group. Let $X$ be a $G$-space. Let $(X,x_0)$ be a pointed space. Then $(X,x_0)$ is a pointed $G$-space if and only if $x_0$ is a $G$-fixed point of $X$. 
\end{lmm}

\begin{defn}{Induced Pointed G-Spaces}{} Let $G$ be a topological group. Let $H\leq G$ be a subgroup. Let $(X,x_0)$ be a pointed $H$-space. Define the induced pointed $G$-space of $(X,x_0)$ to be the space $$\text{Ind}_H^GX=G_+\wedge_HX=\frac{G_+\wedge X}{\sim}$$ where the relation is generated by $(g\cdot h,x)\sim(g,h\cdot x)$ for all $g\in G$, $h\in H$ and $x\in X$. 
\end{defn}

Pushout?

\begin{prp}{}{} Let $G$ be a topological group. Let $H\leq G$ be a subgroup. Then there is an adjunction $$\text{Ind}_H^G:{_H\bold{Top}_\ast}\rightleftarrows{_G\bold{Top}_\ast}:\text{Res}_H^G$$ This means that there is an isomorphism $$\Hom_{_G\bold{Top}_\ast}(\text{Ind}_H^GX,Y)\cong\Hom_{_H\bold{Top}_\ast}(X,\text{Res}_H^GY)$$ that is natural in $X$ and $Y$. 
\end{prp}

\pagebreak
\section{Types of Actions on G-Spaces}
\subsection{Homogenous G-Spaces}
Recall that a group action $G$ on $X$ is said to be transitive if for any $x,y\in X$, there exists $g\in G$ such that $g\cdot x=y$. 

\begin{defn}{Homogenous G-Space}{} Let $G$ be a topological group and let $X$ be a $G$-space. We say that $X$ is a Homogenous $G$-space if $G$ acts transitively on $X$. 
\end{defn}

Much of the theorem we considered in covering space theory was in fact on homogenous $G$-spaces. For instance, when $\tilde{X}$ is path connected, prp2.4.7 says that the fibers $p^{-1}(x_0)$ of the covering space is a homogenous $\pi_1(X,x_0)$-space. The following corollary proves thm 2.4.9

\begin{crl}{}{} Let $G$ be a topological group and let $X$ be a homogenous $G$-space. Then there is an isomorphism of $G$-spaces $$\frac{G}{\text{Stab}_G(x_0)}\cong X$$ induced by the map $g\mapsto g\cdot x_0$. \tcbline
\begin{proof}
Since $G$ is transitive on $X$, we have that $Gx_0=X$. Hence by the above theorem, we obtain the desired isomorphism. 
\end{proof}
\end{crl}

Indeed, if $p:\tilde{X}\to X$ is a covering space, $p^{-1}(x_0)$ is a homogenous $\pi_1(X,x_0)$-space and it follows that $$\frac{\pi_1(X,x_0)}{p_\ast(\pi_1(\tilde{X},\tilde{x}_0))}\cong p^{-1}(x_0)$$ Notice that $\text{Stab}_{\pi_1(X,x_0)}=\im(p_\ast)$ is a non-trivial fact that was proven in prp 2.4.7. 

\begin{crl}{}{} Let $G$ be a topological group and let $X$ be a homogenous $G$-space. If $G$ is more over a free action on $X$, there is an isomorphism of $G$-spaces $$G\cong X$$ given by the map $g\mapsto g\cdot x_0$. \tcbline
\begin{proof}
If $G$ is free, then the stabilizer is trivial. By the above corollary, we obtain the desired isomorphism. 
\end{proof}
\end{crl}

\begin{thm}{}{} Let $G$ be a topological group and let $X$ be a homogenous $G$-space. Then the following are true. 
\begin{itemize}
\item For any $\varphi\in\text{Homeo}(X)$, $x\in X$ and $\varphi(x)$ has the same stabilizers
\item If $x,y\in X$ has the same stabilizers, then there exists $\varphi\in\text{Homeo}(X)$ such that $\varphi(x)=y$. 
\end{itemize}
\end{thm}

\begin{lmm}{}{} Let $G$ be a topological group and let $X$ be a homogenous $G$-space. Let $A$ be a subgroup of $\text{Homeo}(X)$. Then $A=\text{Homeo}(X)$ if and only if for any $x,y\in X$ such that $\text{Stab}_G(x)=\text{Stab}_G(y)$, there exists $\varphi\in A$ such that $\varphi(x)=y$. 
\end{lmm}

\begin{thm}{}{} Let $G$ be a topological group and let $X$ be a homogenous $G$-space. Then there is a isomorphism of left $G$-spaces $$\frac{N(\text{Stab}_G(x_0))}{\text{Stab}_G(x_0)}\cong\text{Homeo}(X)$$
\end{thm}

\subsection{Properly Discontinuous Group Actions}
\begin{defn}{Proper Group Actions}{} Let $G$ be a topological group acting continuously on a topological space $X$. The action is said to be proper if the map $G\times X\to X\times X$ defined by $$(g,x)\mapsto(x,g\cdot x)$$ is a proper map. 
\end{defn}

\begin{defn}{Properly Discontinuous Group Actions}{} Let $G$ be a group acting on a space $X$. Then we say that $G$ is a properly discontinuous group action if for every compact set $K\subseteq X$, we have $$(g\cdot K)\cap K\neq\emptyset$$ for finitely many $g\in G$. 
\end{defn}

\begin{prp}{}{} Every properly discontinuous group action is a wandering action. 
\end{prp}

\begin{prp}{}{} If $G$ is a proper group action on a space $X$, then the action is properly discontinuous. 
\end{prp}

The converse is not true in general, unless we assume that $X$ is locally compact. \\~\\

Recall the notion of a covering space action. $G$ is a covering space action on $X$ if $g\cdot U\cap U\neq\emptyset$ implies $g=1$. This is also related to properly discontinuous group actions. In fact, properly discontinuous group actions are in general stronger than covering space actions. 

\begin{prp}{}{} Let $G$ be a covering space action on $X$. If $X$ is locally compact and Hausdorff, then $G$ is a properly discontinuous group action on $X$. 
\end{prp}

\subsection{Covering Space Actions}
\begin{defn}{Wandering Actions}{} Let $X$ be a space. Let $G$ be a group acting on $X$. We say that $G$ is a wandering action on $X$ if for all $x\in X$, there exists a neighbourhood $U$ of $x$ such that $$(g\cdot U)\cap U\neq\emptyset$$ for finitely many $g\in G$. 
\end{defn}

Algebraic topologists are primarily interested in the following type of group actions. 

\begin{defn}{Covering Space Action}{} Let $X$ be a space and $G$ be a group acting on $X$. We say that $G$ is a covering space action if for each $x\in X$, there is a neighbourhood $U$ of $x$ such that $$(g_1\cdot U)\cap(g_2\cdot U)=\emptyset$$ for all $g_1,g_2\in G$. 
\end{defn}

\begin{lmm}{}{} Every covering space action is wandering and free. \tcbline
\begin{proof}
Let $G$ be a covering space action. Then $(g\cdot U)\cap U=\emptyset$ for all $g\in G$ implies that $g\cdot x$ cannot be equal to $x$. Thus $G$ is a free action on $X$. It is clear that $G$ is a wandering action on $X$ since we require all actions of $g\in G$ to be disjoint while for wandering actions, we only require a finite amount of actions of $g\in G$ to be disjoint. 
\end{proof}
\end{lmm}

The following proposition will show where the name of covering space actions comes from. In particular, we will see that if $X$ is path connected, then there is one unique covering space action on $X$, namely via the deck group $\text{Deck}(p)$. 

\begin{prp}{}{} Let $X$ be a space and $p:\tilde{X}\to X$ be a covering of $X$. Then the action of $\text{Deck}(p)$ on $\tilde{X}$ is a covering space action. \tcbline
\begin{proof}
Suppose that $\tilde{x}\in(\tau_1\cdot U)\cap(\tau_2\cdot U)$. Then this means that $\tau_1(\tilde{x}_1)=\tau_2(\tilde{x}_2)$ for some $\tilde{x}_1,\tilde{x}_2\in U$. But we have that $p\circ\tau_1=p\circ\tau_2$ which implies that $\tilde{x}_1$ and $\tilde{x}_2$ lie in the same fiber $p^{-1}(x)$ for some $x\in X$. By definition of covering spaces, the fiber $p^{-1}(x_0)$ intersects $U$ at exactly one point so that $\tilde{x}_1=\tilde{x_2}$. But this implies that $\tau_2^{-1}\tau_1$ fixes one point in $\tilde{X}$ so that $\tau_2^{-1}\tau_1=1$ and $tau_1=\tau_2$. 
\end{proof}
\end{prp}

\begin{lmm}{}{} Let $X$ be a space and let $p:\tilde{X}\to X$ be a regular covering space of $X$. Then the orbit space of the deck group $$\frac{\tilde{X}}{\text{Deck}(p)}\cong X$$ is isomorphic to the base space. 
\end{lmm}

\begin{prp}{}{} Let $X$ be a space and let $G$ be a covering space action on $X$. Then the quotient map $p:X\to X/G$ defined by $p(x)=\text{Orb}_G(x)$ is a regular covering space of $X/G$. 
\end{prp}

\begin{thm}{}{} Let $X$ be a path connected space. Let $G$ be a covering space action on $X$. Then $G\cong\text{Deck}(p)$ where $p:X\to X/G$ is the regular covering space of $X/G$. 
\end{thm}

\begin{crl}{}{} Let $X$ be a path connected and locally path connected space. Then there is a group isomorphism $$\text{Deck}(p)\cong\frac{\pi_1\left(\frac{X}{\text{Deck}(p)},x_0\right)}{p_\ast(\pi_1\left(X,p(x_0))\right)}$$ for any $x_0\in X$. 
\end{crl}

\pagebreak
\section{Equivariant Homotopy Theory}
\subsection{G-Homotopy}
\begin{defn}{G-Homotopy}{} Let $G$ be a topological group and let $X,Y$ be $G$-spaces. Let $f,g:X\to Y$ be $G$-equivariant maps. A $G$-homotopy from $f$ to $g$ is a homotopy $H:X\times I\to Y$ from $f$ to $g$ such that for each $t\in I$, the map $$H(-,t):X\to Y$$ is $G$-equivariant map. 
\end{defn}














\end{document}
