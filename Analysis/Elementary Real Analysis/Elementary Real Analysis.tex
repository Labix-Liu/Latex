\documentclass[a4paper]{article}

\input{C:/Users/liula/Desktop/Latex/Headers.tex}

\pagestyle{fancy}
\fancyhf{}
\rhead{Labix}
\lhead{Elementary Real Analysis}
\rfoot{\thepage}

\title{Elementary Real Analysis}

\author{Labix}

\date{\today}
\begin{document}
\maketitle
\begin{abstract}
Real analysis is all about functions from $\R$ to $\R$. It involves continuity, differentiability and integration as its central notions. However, to better characterize and prove stuff easier, we use and apply what we know about sequences and series. They are crucial not only in real analysis, but also in sequences of functions. Real analysis is often treated as the most simple case of general analysis. \\~\\
Often when dealing with functions of multivariable, or complex functions, or even spaces with different properties, the methods and techniques are still the core of it. It is expected that when dealing with analysis on a different space, similar methods and ideas could be applied. \\~\\
Real analysis is the best course for first year undergraduate students trying to get the hang of university mathematics. Its deployment of a wide range of ideas in proofs as well as a fair sense of abstraction invites challenging yet rewarding questions and theorems. 
\end{abstract}
\textbf{References}
\begin{itemize}
\item Principles of Mathematical Analysis Third Edition by Walter Rudin
\item University of Warwick MA131 Analysis Lecture Notes by Keith Ball
\item University of Warwick MA244 Analysis Lecture Notes by Jose Rodrigo
\item Imperial College London Math40002 Lecture Notes
\end{itemize}
\pagebreak
\tableofcontents
\pagebreak
\input{C:/Users/liula/Desktop/Latex/Analysis/Elementary Real Analysis/Elementary Real Analysis Content.tex}
\end{document}