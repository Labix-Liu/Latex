\documentclass[a4paper]{article}

\input{C:/Users/liula/Desktop/Latex/Headers.tex}

\pagestyle{fancy}
\fancyhf{}
\rhead{Labix}
\lhead{Fourier Analysis}
\rfoot{\thepage}

\title{Fourier Analysis}

\author{Labix}

\date{\today}
\begin{document}
\maketitle
\begin{abstract}
These notes will act as a an introductory text with a collection of theorems and definitions for differential equations. 
\end{abstract}
\tableofcontents
\pagebreak

\section{Fourier Series}
\subsection{Fourier Series}
\begin{defn}{Fourier Series}{} Let $n\in\N$. Fourier polynomials of degree $2n$ on $[-\pi,\pi]$ are complex polynomials of the form $$\sum_{k=-n}^nc_ke^{ikx}$$ where $c_k\in\C$ and $x\in[-\pi,\pi]$. When $n=\infty$, we have the fourier series $$\sum_{k=-\infty}^\infty c_ke^{ikx}$$ where $c_k\in\C$ and $x\in[-\pi,\pi]$. 
\end{defn}

\begin{prp}{}{} Every fourier polynomial $\sum_{k=-n}^nc_ke^{ikx}$ can be written in real variable $$a_0+\sum_{k=1}^n(a_k\cos(kx)+b_k\sin(kx))$$ where $$c_k=\begin{cases}
\frac{1}{2}(a_k-ib_k) & k>0\\
\frac{1}{2}(a_{-k}+ib_{-k}) & k<0\\
a_0 & k=0
\end{cases}$$
\end{prp}

\begin{lmm}{Orthogonality Property}{} The monomials of the fourier polynomial satisfy the following property. $$\int_{-\pi}^\pi e^{ikx}e^{-ilx}\,dx=\begin{cases}
2\pi & \text{if }k=l\\
0 & \text{otherwise}
\end{cases}$$ where $k,l\in\Z$
\end{lmm}

\subsection{Approximation of Functions}
\begin{prp}{Fourier Coefficients}{} Suppose that $\phi:[-\pi,\pi]\to\R$ can be represented by a fourier series. Then the fourier coefficients are given by $$c_k=\frac{1}{2\pi}\int_{-\pi}^\pi\phi(x)e^{-ikx}\,dx$$ for $k\in\Z$. 
\end{prp}

\begin{defn}{$n$th Fourier Polynomial}{} The $n$th fourier polynomial of $\phi$ that can be represented by a fourier series is given by $$S_n(\phi)(x)=\sum_{k=-n}^nc_ke^{ikx}$$ where $c_k$ are the fourier coefficients of $\phi$. 
\end{defn}

\begin{lmm}{}{} Let $\phi$ be able to be represnted by a fourier series. Then $$c_{-k}=\overline{c_k}$$
\end{lmm}

\begin{lmm}{Riemann-Lebesgue}{} For any continuous function $\phi\in C^0([-\pi,\pi],\C)$, the fourier coefficients converge to $0$. Menaing $$\lim_{k\to\pm\infty}c_k=\lim_{k\to\pm\infty}\frac{1}{2\pi}\int_{-\pi}^\pi\phi(x)e^{-ikx}\,dx=0$$
\end{lmm}

\begin{lmm} Let $\phi:\R\to\R$ be a $2\pi$-periodic function and suppose that its fourier series exists. 
\begin{itemize}
\item If $\phi$ is odd then $\phi(x)=2i\sum_{k=1}^nc_k\sin(kx)$
\item If $\phi$ is even then $\phi(x)=c_0+2\sum_{k=1}^nc_k\cos(kx)$
\end{itemize}
\end{lmm}

\subsection{Convergence of Fourier Series}
\begin{defn}{Dirichlet Kernel}{} The function $$K_n(\theta)=\frac{1}{2\pi}\sum_{k=-n}^ne^{ik\theta}=\frac{1}{2\pi}\frac{\sin\left(\left(n+\frac{1}{2}\right)\theta\right)}{\sin\left(\frac{1}{2}\theta\right)}$$
\end{defn}

\begin{lmm}{}{} Let $\phi:[-\pi,\pi]\to\R$. Then $$S_n(\phi)(x)=\int_{-\pi}^\pi K_n(x-z)\phi(z)\,dz$$
\end{lmm}

\begin{thm}{Pointwise Convergence}{} Let $\phi\in C^1(\R)$ be $2\pi$ periodic.  Then $$S_n(\phi)(x)\to\phi(x)$$ for all $x\in[-\pi,\pi]$. 
\end{thm}

\begin{lmm}{Decay of Fourier Coefficients}{} Assume that $\phi\in C^s(\R)$ is $2\pi$ periodic where $s\in\N$. Then for $k\neq 0$, $$\frac{\abs{c_k}}{\abs{k}^s}\leq\|\phi^{(s)}(x)\|_\infty$$
\end{lmm}

\begin{thm}{Uniform Convergence}{} Let $\phi\in C^2(\R)$ be $2\pi$ periodic.  Then $$\lim_{n\to\infty}S_n(\phi)\to\phi$$ uniformly as $n\to\infty$. 
\end{thm}

























\end{document}