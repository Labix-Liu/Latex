\documentclass[a4paper]{article}

%=========================================
% Packages
%=========================================
\usepackage{mathtools}
\usepackage{amsfonts}
\usepackage{amsmath}
\usepackage{amssymb}
\usepackage{amsthm}
\usepackage[a4paper, total={6in, 8in}, margin=1in]{geometry}
\usepackage[utf8]{inputenc}
\usepackage{fancyhdr}
\usepackage[utf8]{inputenc}
\usepackage{graphicx}
\usepackage{physics}
\usepackage[listings]{tcolorbox}
\usepackage{hyperref}
\usepackage{tikz-cd}
\usepackage{adjustbox}
\usepackage{enumitem}


\hypersetup{
    colorlinks=true, %set true if you want colored links
    linktoc=all,     %set to all if you want both sections and subsections linked
    linkcolor=black,  %choose some color if you want links to stand out
}
\usetikzlibrary{arrows.meta}

\DeclarePairedDelimiter\ceil{\lceil}{\rceil}
\DeclarePairedDelimiter\floor{\lfloor}{\rfloor}

%=========================================
% Custom Math Operators
%=========================================
\DeclareMathOperator{\adj}{adj}
\DeclareMathOperator{\im}{im}
\DeclareMathOperator{\nullity}{nullity}
\DeclareMathOperator{\sign}{sign}
\DeclareMathOperator{\dom}{dom}
\DeclareMathOperator{\lcm}{lcm}
\DeclareMathOperator{\ran}{ran}
\DeclareMathOperator{\ext}{Ext}
\DeclareMathOperator{\dist}{dist}
\DeclareMathOperator{\diam}{diam}
\DeclareMathOperator{\aut}{Aut}
\DeclareMathOperator{\inn}{Inn}
\DeclareMathOperator{\syl}{Syl}
\DeclareMathOperator{\edo}{End}
\DeclareMathOperator{\cov}{Cov}
\DeclareMathOperator{\vari}{Var}
\DeclareMathOperator{\cha}{char}
\DeclareMathOperator{\Span}{span}
\DeclareMathOperator{\ord}{ord}
\DeclareMathOperator{\res}{res}
\DeclareMathOperator{\Hom}{Hom}
\DeclareMathOperator{\Mor}{Mor}
\DeclareMathOperator{\coker}{coker}
\DeclareMathOperator{\Obj}{Obj}
\DeclareMathOperator{\id}{id}
\DeclareMathOperator{\GL}{GL}
\DeclareMathOperator*{\colim}{colim}

%=========================================
% Custom Commands (Shortcuts)
%=========================================
\newcommand{\CP}{\mathbb{CP}}
\newcommand{\GG}{\mathbb{G}}
\newcommand{\F}{\mathbb{F}}
\newcommand{\N}{\mathbb{N}}
\newcommand{\Q}{\mathbb{Q}}
\newcommand{\R}{\mathbb{R}}
\newcommand{\C}{\mathbb{C}}
\newcommand{\E}{\mathbb{E}}
\newcommand{\Prj}{\mathbb{P}}
\newcommand{\RP}{\mathbb{RP}}
\newcommand{\T}{\mathbb{T}}
\newcommand{\Z}{\mathbb{Z}}
\newcommand{\A}{\mathbb{A}}
\renewcommand{\H}{\mathbb{H}}

\newcommand{\mA}{\mathcal{A}}
\newcommand{\mB}{\mathcal{B}}
\newcommand{\mC}{\mathcal{C}}
\newcommand{\mD}{\mathcal{D}}
\newcommand{\mE}{\mathcal{E}}
\newcommand{\mF}{\mathcal{F}}
\newcommand{\mG}{\mathcal{G}}
\newcommand{\mH}{\mathcal{H}}
\newcommand{\mJ}{\mathcal{J}}
\newcommand{\mO}{\mathcal{O}}
\newcommand{\mS}{\mathcal{S}}

%=========================================
% Theorem Environment
%=========================================
\newcommand\todoin[2][]{\todo[backgroundcolor=white!20!white, inline, caption={2do}, #1]{
\begin{minipage}{\textwidth-4pt}#2\end{minipage}}}

\tcbuselibrary{listings, theorems, breakable, skins}

\newtcbtheorem[number within=subsection]{thm}{Theorem}%
{colback=gray!5, colframe=gray!65!black, fonttitle=\bfseries, breakable, enhanced jigsaw, halign=left}{th}
\newtcbtheorem[number within=subsection, use counter from=thm]{defn}{Definition}%
{colback=gray!5, colframe=gray!65!black, fonttitle=\bfseries, breakable, enhanced jigsaw, halign=left}{th}
\newtcbtheorem[number within=subsection, use counter from=thm]{axm}{Axiom}%
{colback=gray!5, colframe=gray!65!black, fonttitle=\bfseries, breakable, enhanced jigsaw, halign=left}{th}
\newtcbtheorem[number within=subsection, use counter from=thm]{prp}{Proposition}%
{colback=gray!5, colframe=gray!65!black, fonttitle=\bfseries, breakable, enhanced jigsaw, halign=left}{th}
\newtcbtheorem[number within=subsection, use counter from=thm]{lmm}{Lemma}%
{colback=gray!5, colframe=gray!65!black, fonttitle=\bfseries, breakable, enhanced jigsaw, halign=left}{th}
\newtcbtheorem[number within=subsection, use counter from=thm]{crl}{Corollary}%
{colback=gray!5, colframe=gray!65!black, fonttitle=\bfseries, breakable, enhanced jigsaw, halign=left}{th}
\newtcbtheorem[number within=subsection, use counter from=thm]{eg}{Example}%
{colback=gray!5, colframe=gray!65!black, fonttitle=\bfseries, breakable, enhanced jigsaw, halign=left}{th}
\newtcbtheorem[number within=subsection, use counter from=thm]{ex}{Exercise}%
{colback=gray!5, colframe=gray!65!black, fonttitle=\bfseries, breakable, enhanced jigsaw, halign=left}{th}
\newtcbtheorem[number within=subsection, use counter from=thm]{alg}{Algorithm}%
{colback=gray!5, colframe=gray!65!black, fonttitle=\bfseries, breakable, enhanced jigsaw, halign=left}{th}

\newcounter{qtnc}
\newtcolorbox[use counter=qtnc]{qtn}%
{colback=gray!5, colframe=gray!65!black, fonttitle=\bfseries, breakable, enhanced jigsaw, halign=left}




\raggedright

\pagestyle{fancy}
\fancyhf{}
\rhead{Labix}
\lhead{Functional Analysis}
\rfoot{\thepage}

\title{Functional Analysis}

\author{Labix}

\date{\today}
\begin{document}
\maketitle
\begin{abstract}
\end{abstract}
\tableofcontents
\pagebreak

\section{Banach Spaces}
\subsection{Normed Spaces}
Recall from linear algebra that a normed space is a vector space equipped with a norm. 

\begin{defn}{Isometrically Isomorphic Normed Spaces}{} Two normed spaces $V$ and $W$ are isometrically isomorphic if there is a surjective linear isometry $L:V\to W$. IN this case, we write $V\cong W$. 
\end{defn}

\begin{prp}{}{} Let $(X,\|\cdot\|_X)$ be a normed space. Let $V$ be a vector space over the same field and $L:V\to X$ a linear isomorphism. Then the pullback norm $$\|v\|_V=\|L(v)\|_X$$ defines a norm on $V$.  In particular, $L:(V,\|\cdot\|_V)\to(X,\|\cdot\|_X)$ is a linear isometry. 
\end{prp}

\begin{thm}{}{} If $V$ is a finite dimensional vector space then all norms on $V$ are equivalent. 
\end{thm}

Recall that every normed space is a vector space by defining $d(x,y)=\|x-y\|$ for $x,y$ in a normed space $X$. We thus have the notion of convergence in normed spaces. 

\subsection{Condition for Finite-Dimensional}
\begin{thm}{}{} A compact set in a normed space is closed and bounded. 
\end{thm}

Recall in topology that compactness is preserved by continuity. This allows us to construct an argument proving the following. 

\begin{thm}{}{} Let $X$ be a finite dimensional normed space. Then a subset $U$ of $X$ is compact if and only if it is closed and bounded. 
\end{thm}

\begin{lmm}{Riesz's Lemma}{} Let $(X,\|\cdot\|)$ be a normed space and $Y$ a non-empty closed subspace of $X$ not equal to $X$. Then there exists $x\in X$ with $\|x\|=1$ such that $\|x-y\|\geq \frac{1}{2}$ for every $y\in Y$. 
\end{lmm}

\begin{prp}{}{} Every finite dimensional subspace of a normed space is closed. 
\end{prp}

\begin{thm}{}{} A normed space $X$ is finite dimensional if and only if its closed unit ball is compact. 
\end{thm}

\subsection{Banach Spaces}
\begin{defn}{Complete Spaces}{} A normed space $(X,\|\cdot\|)$ is complete if any Cauchy sequence in $X$ converges to some $x\in X$. 
\end{defn}

\begin{defn}{Banach Spaces}{} A complete normed space is called a Banach space.
\end{defn}

\begin{lmm}{}{} Suppose that $(X,\|\cdot\|_X)\cong(Y,\|\cdot\|_Y)$ are isometrically isomorphic. Then $X$ is complete if and only if $Y$ is complete. 
\end{lmm}

\begin{lmm}{}{} If $\|\cdot\|_1$ and $\|\cdot\|_2$ are equivalent norms on a vector space $X$ then $(X,\|\cdot\|_1)$ is complete if and only if $(X,\|\cdot\|_2)$ is complete. 
\end{lmm}

\begin{prp}{}{} Both $\R^n$ and $\C^n$ are complete. 
\end{prp}

\begin{crl}{}{} Every finite dimensional normed space $(V,\|\cdot\|)$ is complete. 
\end{crl}

\subsection{Separability of Banach Spaces}
Recall the notion of separability: A space is separable if it has a countably dense subset. 

\begin{lmm}{}{} Let $X$ be a normed space. Then the following are equivalent. 
\begin{itemize}
\item $X$ is separable
\item The set $\{x\in X|\|x\|=1\}$ is separable
\item $X$ contains a sequence $(x_n)_{n\in\N}$ whose linear span is dense. 
\end{itemize}
\end{lmm}

\pagebreak
\section{Linear Maps Between Banach Spaces}
\subsection{Boundedness and Continuity}
\begin{defn}{Bounded Linear Maps}{} A linear map $A$ from a normed space $(X,\|\cdot\|_X)$ to $(Y,\|\cdot\|_Y)$ is bounded if there exists a constant $M$ such that $$\|Ax\|_Y\leq M\|x\|_X$$ for all $x\in X$. \\~\\
Denote the space of all bounded linear operators by $B(X,Y)$. 
\end{defn}

\begin{lmm}{}{} If $X$ is a finite dimensional space then any linear map $T:(X,\|\cdot\|_X)\to(Y,\|\cdot\|_Y)$ is bounded. 
\end{lmm}

\begin{lmm}{}{} A linear map $T:X\to Y$ is continuous if and only if it is bounded. 
\end{lmm}

\begin{thm}{}{} Let $X$ be a normed space and $Y$ a Banach space. Then $B(X,Y)$ is Banach space. 
\end{thm}

\subsection{Invertibility}
\begin{crl}{}{} Let $T\in B(X,Y)$ for $X,Y$ vector spaces. Then $\ker(T)$ is a closed linear subspace of $X$. 
\end{crl}

\begin{defn}{Bounded Invertible}{} A linear map $T\in B(X,Y)$ is bounded invertible if there exists $S\in B(X,Y)$ such that $S\circ T$ and $T\circ S$ are the identity. 
\end{defn}

\begin{lmm}{}{} Suppose that $X$ and $Y$ are normed spaces. Then for any $T\in B(X,Y)$ the following are equivalent. 
\begin{itemize}
\item $T$ is bounded invertible
\item $T$ is a bijection and $T^{-1}\in B(X,Y)$
\item $T$ is surjective and for some $c>0$, $\|T(x)\|_Y\geq c\|x\|_X$ for every $x\in X$. 
\end{itemize}
\end{lmm}

\begin{crl}{}{} If $X$ is finite dimensional then a linear operator $T:X\to X$ is invertible if and only if $\ker(T)=\{0\}$. 
\end{crl}

\subsection{The Hahn Banach Theorem}
\begin{defn}{Continuous Dual Space}{} Let $X$ be a normed space. Denote the continuous dual space of $X$ to be the the subspace $$X'=B(X,K)$$ of the dual space $X^\ast$. 
\end{defn}

Notice that since continuity is the same as boundedness, this notation of $B(X,K)$ coincides with the set of all bounded linear operators. 

\begin{lmm}{}{} Let $X$ be a Banach space. Then $X'$ is also a Banach space. \tcbline
\begin{proof}
Since we know that $B(X,Y)$ is a Banach space for any normed space $X$ and Banach space $Y$, choose $Y=\R$ and we are done. 
\end{proof}
\end{lmm}

\begin{thm}{The Real Hahn-Banach Theorem}{} Let $X$ be a real vector space. Let $p$ be a convex function on $X$. Let $X_0$ be a linear subspace of $X$ and let $f$ be a linear functional on $Y$ satisfying $$f(x)\leq p(x)$$ for all $x\in Y$. Then $f$ can be extended to a linear functional $F$ on all of $X$ satisfying the condition $$F(x)\leq p(x)$$ for all $x\in X$ and $F|_Y=f$. 
\end{thm}

\pagebreak
\section{Hilbert Spaces}
\subsection{Hilbert Spaces}
\begin{defn}{Hilbert Spaces}{} A Hilbert space is a complete inner product space, where complete means complete using the norm induced by the inner product. 
\end{defn}

\subsection{Orthogonality in Hilbert Spaces}
\begin{defn}{Orthonormal Set}{} A subset $E$ of a Hilbert space is orthonormal if $\|e\|=$ for all $e\in E$ and $\langle e_1,e_2\rangle=0$ for all $e_1,e_2\in E$ with $e_1\neq e_2$. 
\end{defn}

\begin{lmm}{Bessel's Inequality}{} Let $V$ be an inner product space and $(e_n)_{n\in\N}$ be an orthonormal sequence. Then for any $x\in V$ we have $$\sum_{k=1}^\infty\abs{\langle x,e_n\rangle}^2\leq \|x\|^2$$
\end{lmm}

\begin{lmm}{}{} Let $H$ be a Hilbert space and $(e_n)_{n\in\N}$ an orthonormal sequence in $H$. Then the series $\sum_{k=1}^\infty a_ne_n$ converges if and only if $$\sum_{k=1}^\infty\abs{a_n}^2<\infty$$ and $$\left\|\sum_{k=1}^\infty a_ne_n\right\|^2=\sum_{k=1}^\infty\abs{a_n}^2$$
\end{lmm}

\begin{crl}{}{} Let $H$ be a Hilbert space and $(e_n)_{n\in\N}$ an orthonormal sequence in $H$. Then for any $x$ the sequence $$\sum_{k=1}^\infty\langle x,e_n\rangle e_n$$ converges. 
\end{crl}

\begin{prp}{}{} Let $E=(e_n)_{n\in\N}$ be an orthonormal sequence in a Hilbert space $H$. Then the following are equivalent. 
\begin{itemize}
\item $E$ is a basis for $H$
\item For any $x\in H$ we have $x=\sum_{k=1}^\infty\langle x,e_n\rangle e_n$
\item $\|x\|^2=\sum_{k=1}^\infty\abs{\langle x,e_n\rangle}^2$ for all $x\in H$
\item $\langle x,e_n\rangle=0$ for all $n$ implies $x=0$
\item $\overline{\text{span}(E)}=H$
\end{itemize}
\end{prp}

\subsection{Separability of Hilbert Spaces}
\begin{prp}{}{} An infinite-dimensional Hilbert space is separable if and only if it has a countable orthonormal basis. 
\end{prp}

\begin{thm}{}{} Any infinite dimensional separable Hilbert space $H$ is isometric to $l^2(K)$ for some field $K$. 
\end{thm}

\subsection{The Adjoint of Maps of Hilbert Spaces}
\begin{thm}{Riesz Representation Theorem}{}
\end{thm}

\begin{thm}{}{} Let $H,K$ be Hilbert spaces. Let $A:H\to K$ be a bounded linear map. Then there exists a unique bounded linear map $A^\ast:K\to H$ such that $$\langle A(x),y\rangle=\langle x,A^\ast(y)\rangle$$ for all $x\in H$ and $y\in K$. 
\end{thm}

\pagebreak
\section{Spectral Theory}
\subsection{Spectrum}
\begin{defn}{Resolvent and Spectrum}{} Let $X$ be a complex Banach space and $T\in B(X)$. Define the resolvent set of $T$ to be $$\rho(T)=\{\lambda\in\C|T-\lambda I\text{ is invertible }\}$$ Define the spectrum of $T$ to be $$\sigma(T)=\C\setminus\rho(T)=\{\lambda\in\C|T-\lambda I\text{ is not invertible }\}$$
\end{defn}

Notice that in the case that $X$ is finite dimensional, the resolvent set of any linear operator is just its set of eigenvalues. 

\begin{lmm}{}{} Suppose that $T\in B(X)$ and that $(\lambda_n)_{n\in\N}$ are distinct eigenvalues of $T$. Then any set $(e_n)_{n\in\N}$ of corresponding eigenvectors is linearly independent. 
\end{lmm}

\begin{thm}{}{} Let $X$ be a Banach space and $T\in B(X)$ such that $T^{-1}\in B(X)$. Then for any $U\in B(X)$, with $\|U\|<\|T^{-1}\|^{-1}$, we have $$\|(T+U)^{-1}\|\leq\frac{\|T^{-1}\|}{1-\|U\|\|T^{-1}\|}$$
\end{thm}

\begin{lmm}{}{} Let $X$ be a Banach space. Let $T\in B(X)$. Then $\sigma(T)$ is a closed subset of $\{\lambda\in\C|\abs{\lambda}\leq\|T\|\}$. 
\end{lmm}














\end{document}