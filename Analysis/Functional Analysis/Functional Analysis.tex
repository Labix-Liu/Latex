\documentclass[a4paper]{article}

%=========================================
% Packages
%=========================================
\usepackage{mathtools}
\usepackage{amsfonts}
\usepackage{amsmath}
\usepackage{amssymb}
\usepackage{amsthm}
\usepackage[a4paper, total={6in, 8in}, margin=1in]{geometry}
\usepackage[utf8]{inputenc}
\usepackage{fancyhdr}
\usepackage[utf8]{inputenc}
\usepackage{graphicx}
\usepackage{physics}
\usepackage[listings]{tcolorbox}
\usepackage{hyperref}
\usepackage{tikz-cd}
\usepackage{adjustbox}
\usepackage{enumitem}
\usepackage[font=small,labelfont=bf]{caption}
\usepackage{subcaption}
\usepackage{wrapfig}
\usepackage{makecell}



\raggedright

\usetikzlibrary{arrows.meta}

\DeclarePairedDelimiter\ceil{\lceil}{\rceil}
\DeclarePairedDelimiter\floor{\lfloor}{\rfloor}

%=========================================
% Fonts
%=========================================
\usepackage{tgpagella}
\usepackage[T1]{fontenc}


%=========================================
% Custom Math Operators
%=========================================
\DeclareMathOperator{\adj}{adj}
\DeclareMathOperator{\im}{im}
\DeclareMathOperator{\nullity}{nullity}
\DeclareMathOperator{\sign}{sign}
\DeclareMathOperator{\dom}{dom}
\DeclareMathOperator{\lcm}{lcm}
\DeclareMathOperator{\ran}{ran}
\DeclareMathOperator{\ext}{Ext}
\DeclareMathOperator{\dist}{dist}
\DeclareMathOperator{\diam}{diam}
\DeclareMathOperator{\aut}{Aut}
\DeclareMathOperator{\inn}{Inn}
\DeclareMathOperator{\syl}{Syl}
\DeclareMathOperator{\edo}{End}
\DeclareMathOperator{\cov}{Cov}
\DeclareMathOperator{\vari}{Var}
\DeclareMathOperator{\cha}{char}
\DeclareMathOperator{\Span}{span}
\DeclareMathOperator{\ord}{ord}
\DeclareMathOperator{\res}{res}
\DeclareMathOperator{\Hom}{Hom}
\DeclareMathOperator{\Mor}{Mor}
\DeclareMathOperator{\coker}{coker}
\DeclareMathOperator{\Obj}{Obj}
\DeclareMathOperator{\id}{id}
\DeclareMathOperator{\GL}{GL}
\DeclareMathOperator*{\colim}{colim}

%=========================================
% Custom Commands (Shortcuts)
%=========================================
\newcommand{\CP}{\mathbb{CP}}
\newcommand{\GG}{\mathbb{G}}
\newcommand{\F}{\mathbb{F}}
\newcommand{\N}{\mathbb{N}}
\newcommand{\Q}{\mathbb{Q}}
\newcommand{\R}{\mathbb{R}}
\newcommand{\C}{\mathbb{C}}
\newcommand{\E}{\mathbb{E}}
\newcommand{\Prj}{\mathbb{P}}
\newcommand{\RP}{\mathbb{RP}}
\newcommand{\T}{\mathbb{T}}
\newcommand{\Z}{\mathbb{Z}}
\newcommand{\A}{\mathbb{A}}
\renewcommand{\H}{\mathbb{H}}
\newcommand{\K}{\mathbb{K}}

\newcommand{\mA}{\mathcal{A}}
\newcommand{\mB}{\mathcal{B}}
\newcommand{\mC}{\mathcal{C}}
\newcommand{\mD}{\mathcal{D}}
\newcommand{\mE}{\mathcal{E}}
\newcommand{\mF}{\mathcal{F}}
\newcommand{\mG}{\mathcal{G}}
\newcommand{\mH}{\mathcal{H}}
\newcommand{\mI}{\mathcal{I}}
\newcommand{\mJ}{\mathcal{J}}
\newcommand{\mK}{\mathcal{K}}
\newcommand{\mL}{\mathcal{L}}
\newcommand{\mM}{\mathcal{M}}
\newcommand{\mO}{\mathcal{O}}
\newcommand{\mP}{\mathcal{P}}
\newcommand{\mS}{\mathcal{S}}
\newcommand{\mT}{\mathcal{T}}
\newcommand{\mV}{\mathcal{V}}
\newcommand{\mW}{\mathcal{W}}

%=========================================
% Colours!!!
%=========================================
\definecolor{LightBlue}{HTML}{2D64A6}
\definecolor{ForestGreen}{HTML}{4BA150}
\definecolor{DarkBlue}{HTML}{000080}
\definecolor{LightPurple}{HTML}{cc99ff}
\definecolor{LightOrange}{HTML}{ffc34d}
\definecolor{Buff}{HTML}{DDAE7E}
\definecolor{Sunset}{HTML}{F2C57C}
\definecolor{Wenge}{HTML}{584B53}
\definecolor{Coolgray}{HTML}{9098CB}
\definecolor{Lavender}{HTML}{D6E3F8}
\definecolor{Glaucous}{HTML}{828BC4}
\definecolor{Mauve}{HTML}{C7A8F0}
\definecolor{Darkred}{HTML}{880808}
\definecolor{Beaver}{HTML}{9A8873}
\definecolor{UltraViolet}{HTML}{52489C}



%=========================================
% Theorem Environment
%=========================================
\tcbuselibrary{listings, theorems, breakable, skins}

\newtcbtheorem[number within = subsection]{thm}{Theorem}%
{	colback=Buff!3, 
	colframe=Buff, 
	fonttitle=\bfseries, 
	breakable, 
	enhanced jigsaw, 
	halign=left
}{thm}

\newtcbtheorem[number within=subsection, use counter from=thm]{defn}{Definition}%
{  colback=cyan!1,
    colframe=cyan!50!black,
	fonttitle=\bfseries, breakable, 
	enhanced jigsaw, 
	halign=left
}{defn}

\newtcbtheorem[number within=subsection, use counter from=thm]{axm}{Axiom}%
{	colback=red!5, 
	colframe=Darkred, 
	fonttitle=\bfseries, 
	breakable, 
	enhanced jigsaw, 
	halign=left
}{axm}

\newtcbtheorem[number within=subsection, use counter from=thm]{prp}{Proposition}%
{	colback=LightBlue!3, 
	colframe=Glaucous, 
	fonttitle=\bfseries, 
	breakable, 
	enhanced jigsaw, 
	halign=left
}{prp}

\newtcbtheorem[number within=subsection, use counter from=thm]{lmm}{Lemma}%
{	colback=LightBlue!3, 
	colframe=LightBlue!60, 
	fonttitle=\bfseries, 
	breakable, 
	enhanced jigsaw, 
	halign=left
}{lmm}

\newtcbtheorem[number within=subsection, use counter from=thm]{crl}{Corollary}%
{	colback=LightBlue!3, 
	colframe=LightBlue!60, 
	fonttitle=\bfseries, 
	breakable, 
	enhanced jigsaw, 
	halign=left
}{crl}

\newtcbtheorem[number within=subsection, use counter from=thm]{eg}{Example}%
{	colback=Beaver!5, 
	colframe=Beaver, 
	fonttitle=\bfseries, 
	breakable, 
	enhanced jigsaw, 
	halign=left
}{eg}

\newtcbtheorem[number within=subsection, use counter from=thm]{ex}{Exercise}%
{	colback=Beaver!5, 
	colframe=Beaver, 
	fonttitle=\bfseries, 
	breakable, 
	enhanced jigsaw, 
	halign=left
}{ex}

\newtcbtheorem[number within=subsection, use counter from=thm]{alg}{Algorithm}%
{	colback=UltraViolet!5, 
	colframe=UltraViolet, 
	fonttitle=\bfseries, 
	breakable, 
	enhanced jigsaw, 
	halign=left
}{alg}




%=========================================
% Hyperlinks
%=========================================
\hypersetup{
    colorlinks=true, %set true if you want colored links
    linktoc=all,     %set to all if you want both sections and subsections linked
    linkcolor=DarkBlue,  %choose some color if you want links to stand out
}


\pagestyle{fancy}
\fancyhf{}
\rhead{Labix}
\lhead{Functional Analysis}
\rfoot{\thepage}

\title{Functional Analysis}

\author{Labix}

\date{\today}
\begin{document}
\maketitle
\begin{abstract}
\end{abstract}
\tableofcontents
\pagebreak

\section{Vector Spaces with a Topological Structure}
\subsection{Topological Vector Spaces}
\begin{defn}{Topological Field}{} Let $k$ be a field equipped with a topology. We say that $k$ is a topological field if the following are true. 
\begin{itemize}
\item The addition map $+:k\times k\to k$ is continuous. 
\item The multiplication map $\cdot:k\times k\to k$ is continuous. 
\item The inverse map $(\cdot)^{-1}:k\to k$ is continuous. 
\end{itemize}
\end{defn}

\begin{defn}{Topological Vector Space}{} Let $k$ be a topological field. Let $V$ be a vector space over a field $k$ that is also a topological space. We say that $V$ is a topological vector space if the following are true. 
\begin{itemize}
\item The addition map $+:V\times V\to V$ is continuous. 
\item The scalar multiplication map $\cdot:k\times V\to V$ is continuous. 
\end{itemize}
\end{defn}

\subsection{Normed Spaces}
Let $\F$ be a field. Let $V$ be a vector space over $\F$. Recall that a norm on $V$ is a function $\|\cdot\|:V\to\F$ satisfying the following rules: 
\begin{itemize}
\item $\|x\|\geq 0$ with equality if and only if $x=0$. 
\item $\|\lambda x\|=\abs{\lambda}\|x\|$ for any $\lambda\in\F$ and $x\in V$. 
\item $\|x+y\|\leq\|x\|+\|y\|$ for all $x,y\in V$. 
\end{itemize}

\begin{lmm}{}{} Let $(V,\|\cdot\|)$ be a normed space. Then the induced topology given by the induced metric space from the norm gives $V$ the structure of a topological space. 
\end{lmm}

\begin{prp}{}{} Let $V$ be a finite dimensional normed space. Let $U\subseteq X$ be a subset of $V$. Then $U$ is compact if and only if $V$ is closed and bounded. 
\end{prp}

\begin{lmm}{Riesz's Lemma}{} Let $(X,\|\cdot\|)$ be a normed space and $Y$ a non-empty closed subspace of $X$ not equal to $X$. Then there exists $x\in X$ with $\|x\|=1$ such that $\|x-y\|\geq \frac{1}{2}$ for every $y\in Y$. 
\end{lmm}

\begin{prp}{}{} Let $V$ be a normed space. Let $U\subseteq V$ be a finite dimensional vector subspace of $V$. Then $U$ is closed. 
\end{prp}

\begin{prp}{}{} Let $V$ be a normed space. Then $V$ is finite dimensional if and only if $B_1(0)$ is compact. 
\end{prp}

\subsection{Isomorphisms that Preserve Distance}
\begin{defn}{Isometrically Isomorphic Normed Spaces}{} Two normed spaces $V$ and $W$ are isometrically isomorphic if there is a surjective linear isometry $L:V\to W$. In this case, we write $V\cong W$. 
\end{defn}

\begin{prp}{}{} Let $(X,\|\cdot\|_X)$ be a normed space. Let $V$ be a vector space over the same field and $L:V\to X$ a linear isomorphism. Then the pullback norm $$\|v\|_V=\|L(v)\|_X$$ defines a norm on $V$.  In particular, $L:(V,\|\cdot\|_V)\to(X,\|\cdot\|_X)$ is a linear isometry. 
\end{prp}

\begin{thm}{}{} If $V$ is a finite dimensional vector space then all norms on $V$ are equivalent. 
\end{thm}

Recall that every normed space is a vector space by defining $d(x,y)=\|x-y\|$ for $x,y$ in a normed space $X$. We thus have the notion of convergence in normed spaces. 

\pagebreak
\section{Banach Spaces}
\subsection{Banach Spaces}
\begin{defn}{Banach Spaces}{} Let $(V,\|\cdot\|)$ be a normed space. We say that $V$ is a Banach space if $V$ is complete as a metric space. 
\end{defn}

\begin{prp}{}{} Let $V$ be a finite dimensional normed space. Then $V$ is a Banach space. 
\end{prp}

\begin{lmm}{}{} Suppose that $(X,\|\cdot\|_X)\cong(Y,\|\cdot\|_Y)$ are isometrically isomorphic. Then $X$ is a Banach space if and only if $Y$ is a Banach space. 
\end{lmm}

\begin{lmm}{}{} If $\|\cdot\|_1$ and $\|\cdot\|_2$ are equivalent norms on a vector space $X$ then $(X,\|\cdot\|_1)$ is complete if and only if $(X,\|\cdot\|_2)$ is complete. 
\end{lmm}

\begin{prp}{}{} Both $\R^n$ and $\C^n$ are complete. 
\end{prp}

\subsection{Separability of Banach Spaces}
Recall the notion of separability: A space is separable if it has a countably dense subset. 

\begin{lmm}{}{} Let $X$ be a normed space. Then the following are equivalent. 
\begin{itemize}
\item $X$ is separable
\item The set $\{x\in X|\|x\|=1\}$ is separable
\item $X$ contains a sequence $(x_n)_{n\in\N}$ whose linear span is dense. 
\end{itemize}
\end{lmm}

\subsection{The Completion of a Normed Space}

\pagebreak
\section{The Space of Bounded Linear Maps}
\subsection{Basic Definitions}
\begin{defn}{Bounded Linear Maps}{} Let $V,W$ be normed space over a field $k$. Let $T:V\to W$ be a linear map. We say that $T$ is bounded if there exists $M>0$ such that $$\|T(x)\|_W\leq M\|x\|_V$$ for all $x\in V$. 
\end{defn}

\begin{lmm}{}{} Let $V,W$ be normed spaces over a field $k$. Let $T:V\to W$ be a linear map. Then $T$ is bounded if and only if $T$ is continuous. 
\end{lmm}

\begin{defn}{The Space of Bounded Linear Maps}{} Let $V,W$ be normed spaces over a field $k$. Define the space of bounded linear maps to be the vector subspace $$B(V,W)=\{T\in\Hom_k(V,W)\;|\;T\text{ is bounded }\}$$ together with the operator norm $\|\cdot\|_{\text{op}}:B(V,W)\to k$ defined by $$\|T\|_{\text{op}}=\sup\{\|T(v)\|_W\;|\;v\in V\text{ such that }\|v\|_V=1\}$$
\end{defn}

\begin{lmm}{}{} Let $V,W$ be normed spaces. If $V$ is finite dimensional, then we have $$B(V,W)=\Hom_k(V,W)$$
\end{lmm}

\begin{lmm}{}{} Let $V,W$ be normed spaces over a field $k$. Then $B(V,W)$ is a Banach space. 
\end{lmm}

\subsection{Invertibility}
\begin{crl}{}{} Let $T\in B(X,Y)$ for $X,Y$ vector spaces. Then $\ker(T)$ is a closed linear subspace of $X$. 
\end{crl}

\begin{defn}{Bounded Invertible}{} A linear map $T\in B(X,Y)$ is bounded invertible if there exists $S\in B(X,Y)$ such that $S\circ T$ and $T\circ S$ are the identity. 
\end{defn}

\begin{lmm}{}{} Suppose that $X$ and $Y$ are normed spaces. Then for any $T\in B(X,Y)$ the following are equivalent. 
\begin{itemize}
\item $T$ is bounded invertible
\item $T$ is a bijection and $T^{-1}\in B(X,Y)$
\item $T$ is surjective and for some $c>0$, $\|T(x)\|_Y\geq c\|x\|_X$ for every $x\in X$. 
\end{itemize}
\end{lmm}

\begin{crl}{}{} If $X$ is finite dimensional then a linear operator $T:X\to X$ is invertible if and only if $\ker(T)=\{0\}$. 
\end{crl}

\subsection{The Hahn Banach Theorem}
\begin{defn}{Continuous Dual Space}{} Let $X$ be a normed space. Denote the continuous dual space of $X$ to be the the subspace $$X'=B(X,K)$$ of the dual space $X^\ast$. 
\end{defn}

Notice that since continuity is the same as boundedness, this notation of $B(X,K)$ coincides with the set of all bounded linear operators. 

\begin{lmm}{}{} Let $X$ be a Banach space. Then $X'$ is also a Banach space. \tcbline
\begin{proof}
Since we know that $B(X,Y)$ is a Banach space for any normed space $X$ and Banach space $Y$, choose $Y=\R$ and we are done. 
\end{proof}
\end{lmm}

\begin{thm}{The Real Hahn-Banach Theorem}{} Let $X$ be a real vector space. Let $p$ be a convex function on $X$. Let $X_0$ be a linear subspace of $X$ and let $f$ be a linear functional on $Y$ satisfying $$f(x)\leq p(x)$$ for all $x\in Y$. Then $f$ can be extended to a linear functional $F$ on all of $X$ satisfying the condition $$F(x)\leq p(x)$$ for all $x\in X$ and $F|_Y=f$. 
\end{thm}

\pagebreak
\section{Hilbert Spaces}
\subsection{Hilbert Spaces}
\begin{defn}{Hilbert Spaces}{} Let $V$ be an inner product space. We say that $V$ is a Hilbert space if $V$ is a Banach space. 
\end{defn}

\subsection{Orthogonality in Hilbert Spaces}
\begin{defn}{Orthonormal Set}{} A subset $E$ of a Hilbert space is orthonormal if $\|e\|=$ for all $e\in E$ and $\langle e_1,e_2\rangle=0$ for all $e_1,e_2\in E$ with $e_1\neq e_2$. 
\end{defn}

\begin{lmm}{Bessel's Inequality}{} Let $V$ be an inner product space and $(e_n)_{n\in\N}$ be an orthonormal sequence. Then for any $x\in V$ we have $$\sum_{k=1}^\infty\abs{\langle x,e_n\rangle}^2\leq \|x\|^2$$
\end{lmm}

\begin{lmm}{}{} Let $H$ be a Hilbert space and $(e_n)_{n\in\N}$ an orthonormal sequence in $H$. Then the series $\sum_{k=1}^\infty a_ne_n$ converges if and only if $$\sum_{k=1}^\infty\abs{a_n}^2<\infty$$ and $$\left\|\sum_{k=1}^\infty a_ne_n\right\|^2=\sum_{k=1}^\infty\abs{a_n}^2$$
\end{lmm}

\begin{crl}{}{} Let $H$ be a Hilbert space and $(e_n)_{n\in\N}$ an orthonormal sequence in $H$. Then for any $x$ the sequence $$\sum_{k=1}^\infty\langle x,e_n\rangle e_n$$ converges. 
\end{crl}

\begin{prp}{}{} Let $E=(e_n)_{n\in\N}$ be an orthonormal sequence in a Hilbert space $H$. Then the following are equivalent. 
\begin{itemize}
\item $E$ is a basis for $H$
\item For any $x\in H$ we have $x=\sum_{k=1}^\infty\langle x,e_n\rangle e_n$
\item $\|x\|^2=\sum_{k=1}^\infty\abs{\langle x,e_n\rangle}^2$ for all $x\in H$
\item $\langle x,e_n\rangle=0$ for all $n$ implies $x=0$
\item $\overline{\text{span}(E)}=H$
\end{itemize}
\end{prp}

\subsection{Separability of Hilbert Spaces}
\begin{prp}{}{} An infinite-dimensional Hilbert space is separable if and only if it has a countable orthonormal basis. 
\end{prp}

\begin{thm}{}{} Any infinite dimensional separable Hilbert space $H$ is isometric to $l^2(K)$ for some field $K$. 
\end{thm}

\subsection{The Adjoint of Maps of Hilbert Spaces}
\begin{thm}{Riesz Representation Theorem}{}
\end{thm}

\begin{thm}{}{} Let $H,K$ be Hilbert spaces. Let $A:H\to K$ be a bounded linear map. Then there exists a unique bounded linear map $A^\ast:K\to H$ such that $$\langle A(x),y\rangle=\langle x,A^\ast(y)\rangle$$ for all $x\in H$ and $y\in K$. 
\end{thm}

\pagebreak
\section{Spectral Theory}
\subsection{Spectrum}
\begin{defn}{Resolvent and Spectrum}{} Let $X$ be a complex Banach space and $T\in B(X)$. Define the resolvent set of $T$ to be $$\rho(T)=\{\lambda\in\C|T-\lambda I\text{ is invertible }\}$$ Define the spectrum of $T$ to be $$\sigma(T)=\C\setminus\rho(T)=\{\lambda\in\C|T-\lambda I\text{ is not invertible }\}$$
\end{defn}

Notice that in the case that $X$ is finite dimensional, the resolvent set of any linear operator is just its set of eigenvalues. 

\begin{lmm}{}{} Suppose that $T\in B(X)$ and that $(\lambda_n)_{n\in\N}$ are distinct eigenvalues of $T$. Then any set $(e_n)_{n\in\N}$ of corresponding eigenvectors is linearly independent. 
\end{lmm}

\begin{thm}{}{} Let $X$ be a Banach space and $T\in B(X)$ such that $T^{-1}\in B(X)$. Then for any $U\in B(X)$, with $\|U\|<\|T^{-1}\|^{-1}$, we have $$\|(T+U)^{-1}\|\leq\frac{\|T^{-1}\|}{1-\|U\|\|T^{-1}\|}$$
\end{thm}

\begin{lmm}{}{} Let $X$ be a Banach space. Let $T\in B(X)$. Then $\sigma(T)$ is a closed subset of $\{\lambda\in\C|\abs{\lambda}\leq\|T\|\}$. 
\end{lmm}














\end{document}