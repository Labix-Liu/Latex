\documentclass[a4paper]{article}

%=========================================
% Packages
%=========================================
\usepackage{mathtools}
\usepackage{amsfonts}
\usepackage{amsmath}
\usepackage{amssymb}
\usepackage{amsthm}
\usepackage[a4paper, total={6in, 8in}, margin=1in]{geometry}
\usepackage[utf8]{inputenc}
\usepackage{fancyhdr}
\usepackage[utf8]{inputenc}
\usepackage{graphicx}
\usepackage{physics}
\usepackage[listings]{tcolorbox}
\usepackage{hyperref}
\usepackage{tikz-cd}
\usepackage{adjustbox}
\usepackage{enumitem}
\usepackage[font=small,labelfont=bf]{caption}
\usepackage{subcaption}
\usepackage{wrapfig}
\usepackage{makecell}



\raggedright

\usetikzlibrary{arrows.meta}

\DeclarePairedDelimiter\ceil{\lceil}{\rceil}
\DeclarePairedDelimiter\floor{\lfloor}{\rfloor}

%=========================================
% Fonts
%=========================================
\usepackage{tgpagella}
\usepackage[T1]{fontenc}


%=========================================
% Custom Math Operators
%=========================================
\DeclareMathOperator{\adj}{adj}
\DeclareMathOperator{\im}{im}
\DeclareMathOperator{\nullity}{nullity}
\DeclareMathOperator{\sign}{sign}
\DeclareMathOperator{\dom}{dom}
\DeclareMathOperator{\lcm}{lcm}
\DeclareMathOperator{\ran}{ran}
\DeclareMathOperator{\ext}{Ext}
\DeclareMathOperator{\dist}{dist}
\DeclareMathOperator{\diam}{diam}
\DeclareMathOperator{\aut}{Aut}
\DeclareMathOperator{\inn}{Inn}
\DeclareMathOperator{\syl}{Syl}
\DeclareMathOperator{\edo}{End}
\DeclareMathOperator{\cov}{Cov}
\DeclareMathOperator{\vari}{Var}
\DeclareMathOperator{\cha}{char}
\DeclareMathOperator{\Span}{span}
\DeclareMathOperator{\ord}{ord}
\DeclareMathOperator{\res}{res}
\DeclareMathOperator{\Hom}{Hom}
\DeclareMathOperator{\Mor}{Mor}
\DeclareMathOperator{\coker}{coker}
\DeclareMathOperator{\Obj}{Obj}
\DeclareMathOperator{\id}{id}
\DeclareMathOperator{\GL}{GL}
\DeclareMathOperator*{\colim}{colim}

%=========================================
% Custom Commands (Shortcuts)
%=========================================
\newcommand{\CP}{\mathbb{CP}}
\newcommand{\GG}{\mathbb{G}}
\newcommand{\F}{\mathbb{F}}
\newcommand{\N}{\mathbb{N}}
\newcommand{\Q}{\mathbb{Q}}
\newcommand{\R}{\mathbb{R}}
\newcommand{\C}{\mathbb{C}}
\newcommand{\E}{\mathbb{E}}
\newcommand{\Prj}{\mathbb{P}}
\newcommand{\RP}{\mathbb{RP}}
\newcommand{\T}{\mathbb{T}}
\newcommand{\Z}{\mathbb{Z}}
\newcommand{\A}{\mathbb{A}}
\renewcommand{\H}{\mathbb{H}}
\newcommand{\K}{\mathbb{K}}

\newcommand{\mA}{\mathcal{A}}
\newcommand{\mB}{\mathcal{B}}
\newcommand{\mC}{\mathcal{C}}
\newcommand{\mD}{\mathcal{D}}
\newcommand{\mE}{\mathcal{E}}
\newcommand{\mF}{\mathcal{F}}
\newcommand{\mG}{\mathcal{G}}
\newcommand{\mH}{\mathcal{H}}
\newcommand{\mI}{\mathcal{I}}
\newcommand{\mJ}{\mathcal{J}}
\newcommand{\mK}{\mathcal{K}}
\newcommand{\mL}{\mathcal{L}}
\newcommand{\mM}{\mathcal{M}}
\newcommand{\mO}{\mathcal{O}}
\newcommand{\mP}{\mathcal{P}}
\newcommand{\mS}{\mathcal{S}}
\newcommand{\mT}{\mathcal{T}}
\newcommand{\mV}{\mathcal{V}}
\newcommand{\mW}{\mathcal{W}}

%=========================================
% Colours!!!
%=========================================
\definecolor{LightBlue}{HTML}{2D64A6}
\definecolor{ForestGreen}{HTML}{4BA150}
\definecolor{DarkBlue}{HTML}{000080}
\definecolor{LightPurple}{HTML}{cc99ff}
\definecolor{LightOrange}{HTML}{ffc34d}
\definecolor{Buff}{HTML}{DDAE7E}
\definecolor{Sunset}{HTML}{F2C57C}
\definecolor{Wenge}{HTML}{584B53}
\definecolor{Coolgray}{HTML}{9098CB}
\definecolor{Lavender}{HTML}{D6E3F8}
\definecolor{Glaucous}{HTML}{828BC4}
\definecolor{Mauve}{HTML}{C7A8F0}
\definecolor{Darkred}{HTML}{880808}
\definecolor{Beaver}{HTML}{9A8873}
\definecolor{UltraViolet}{HTML}{52489C}



%=========================================
% Theorem Environment
%=========================================
\tcbuselibrary{listings, theorems, breakable, skins}

\newtcbtheorem[number within = subsection]{thm}{Theorem}%
{	colback=Buff!3, 
	colframe=Buff, 
	fonttitle=\bfseries, 
	breakable, 
	enhanced jigsaw, 
	halign=left
}{thm}

\newtcbtheorem[number within=subsection, use counter from=thm]{defn}{Definition}%
{  colback=cyan!1,
    colframe=cyan!50!black,
	fonttitle=\bfseries, breakable, 
	enhanced jigsaw, 
	halign=left
}{defn}

\newtcbtheorem[number within=subsection, use counter from=thm]{axm}{Axiom}%
{	colback=red!5, 
	colframe=Darkred, 
	fonttitle=\bfseries, 
	breakable, 
	enhanced jigsaw, 
	halign=left
}{axm}

\newtcbtheorem[number within=subsection, use counter from=thm]{prp}{Proposition}%
{	colback=LightBlue!3, 
	colframe=Glaucous, 
	fonttitle=\bfseries, 
	breakable, 
	enhanced jigsaw, 
	halign=left
}{prp}

\newtcbtheorem[number within=subsection, use counter from=thm]{lmm}{Lemma}%
{	colback=LightBlue!3, 
	colframe=LightBlue!60, 
	fonttitle=\bfseries, 
	breakable, 
	enhanced jigsaw, 
	halign=left
}{lmm}

\newtcbtheorem[number within=subsection, use counter from=thm]{crl}{Corollary}%
{	colback=LightBlue!3, 
	colframe=LightBlue!60, 
	fonttitle=\bfseries, 
	breakable, 
	enhanced jigsaw, 
	halign=left
}{crl}

\newtcbtheorem[number within=subsection, use counter from=thm]{eg}{Example}%
{	colback=Beaver!5, 
	colframe=Beaver, 
	fonttitle=\bfseries, 
	breakable, 
	enhanced jigsaw, 
	halign=left
}{eg}

\newtcbtheorem[number within=subsection, use counter from=thm]{ex}{Exercise}%
{	colback=Beaver!5, 
	colframe=Beaver, 
	fonttitle=\bfseries, 
	breakable, 
	enhanced jigsaw, 
	halign=left
}{ex}

\newtcbtheorem[number within=subsection, use counter from=thm]{alg}{Algorithm}%
{	colback=UltraViolet!5, 
	colframe=UltraViolet, 
	fonttitle=\bfseries, 
	breakable, 
	enhanced jigsaw, 
	halign=left
}{alg}




%=========================================
% Hyperlinks
%=========================================
\hypersetup{
    colorlinks=true, %set true if you want colored links
    linktoc=all,     %set to all if you want both sections and subsections linked
    linkcolor=DarkBlue,  %choose some color if you want links to stand out
}


\pagestyle{fancy}
\fancyhf{}
\rhead{Labix}
\lhead{Complex Analysis 1}
\rfoot{\thepage}

\title{Complex Analysis 1}

\author{Labix}

\date{\today}
\begin{document}
\maketitle
\begin{abstract}
\end{abstract}
\pagebreak
\tableofcontents
\pagebreak

\section{Limits, Continuity and Differentiation}
\subsection{Limits}
\begin{defn}{Limits}{} Let $f:\Omega\subseteq\C\to\C$ be a function. We say that $$\lim_{z\to z_0}f(z)=L$$ if for every $\epsilon>0$, there exists $\delta>0$ such that $$0<\abs{z-z_0}<\delta\implies\abs{f(z)-L}<\epsilon$$
\end{defn}

\begin{prp}{}{} The limit of a function is unqiue if it exists. 
\end{prp}

\begin{thm}{Sequential Limits}{} Let $f:\Omega\subseteq\C\to\C$ be a function. $\lim_{z\to z_0}f(z)=l$ if and only if for every sequence such that $z_n\to z_0$, $(f(z_n))\to l$. 
\end{thm}

\begin{prp}{}{} Let $\lim_{z\to z_0}f(z)=L$ and $\lim_{z\to z_0}g(z)=K$. Then
\begin{itemize}
\item $\lim_{z\to z_0}(f(z)\pm g(z))=L\pm K$
\item $\lim_{z\to z_0}(f(z)g(z))=LK$
\item $\lim_{z\to z_0}\frac{f(z)}{g(z)}=\frac{L}{K}$ if $K\neq0$
\end{itemize}
\end{prp}

\begin{prp}{}{} Let $z_0=a+bi$. Let $f(x+yi)=u(x,y)+v(x,y)i$. $$\lim_{z\to z_0}f(z)=L=\alpha+\beta i$$ if and only if $u(x,y)\to\alpha$ and $v(x,y)\to\beta$ as $(x,y)\to(a,b)$
\end{prp}

\subsection{Continuity}
\begin{defn}{Continuity}{} Let $f:\Omega\subseteq\C\to\C$ be a function. We say that $f(z)$ is continuous at $z_0\in\Omega$ if for every $\epsilon>0$, there exists $\delta>0$ such that for all $z\in\Omega$, $$\abs{z-z_0}<\delta\implies\abs{f(z)-f(z_0)}<\epsilon$$
\end{defn}

\begin{thm}{Limits and Continuity}{} Let $f:\Omega\subseteq\C\to\C$ be a function. $f(z)$ is continuous at $z_0$ if and only if $$\lim_{z\to z_0}f(z)=f(z_0)$$
\end{thm}

\begin{thm}{Sequential Continuity}{} Let $f:\Omega\subseteq\C\to\C$ be a function. Then $f$ is continuous if and only if for every sequence such that $z_n\to z_0$, $f(z_n)\to f(z_0)$. 
\end{thm}

\begin{prp}{}{} If $f$ is continuous at $z_0\in\Omega$ and $g$ is continuous at $f(z_0)$, then $g\circ f$ is continuous at $z_0$. 
\end{prp}

\begin{prp}{}{} Let $\lim_{z\to z_0}f(z)=f(z_0)$ and $\lim_{z\to z_0}g(z)=g(z_0)$. Then
\begin{itemize}
\item $\lim_{z\to z_0}(f(z)\pm g(z))=f(z_0)\pm g(z_0)$
\item $\lim_{z\to z_0}(f(z)g(z))=f(z_0)g(z_0)$
\item $\lim_{z\to z_0}\frac{f(z)}{g(z)}=\frac{f(z_0)}{g(z_0)}$ if $g(z_0)\neq0$
\end{itemize}
\end{prp}

\begin{prp}{}{} Let $f:\Omega\subset\C\to\C$. Then $f(x+yi)=u(x,y)+v(x,y)i$ is continuous at $z_0=x_0+y_0i$ if and only if $u,v$ are continuous at $(x_0,y_0)$
\end{prp}

\subsection{Holomorphic Functions}
\begin{defn}{Complex Differentiability}{} Let $U\subseteq\C$ be open. A complex function $f:U\to\C$ is complex differentiable at $z_0\in\Omega$ if $$\lim_{z\to z_0}\frac{f(z)-f(z_0)}{z-z_0}$$ exists. In this case we denote the limit as $f'(z_0)$. 
\end{defn}

\begin{prp}{}{} If $f$ is complex differentiable at $z_0$ then $f$ is continuous at $z_0$. 
\end{prp}

\begin{prp}{}{} Let $f$ and $g$ be complex differentiable at $z_0$ then the following are complex differentiable. 
\begin{itemize}
\item $(f\pm g)'=f'\pm g'$
\item $(fg)'=f'g+g'f$
\item $\left(\frac{f}{g}\right)'=\frac{f'g-fg'}{g^2}$ provided that $g(z_0)\neq0$
\end{itemize}
\end{prp}

\begin{defn}{Holomorphic Functions}{} A function $f$ is said to be holomorphic on $\Omega\subseteq\C$ if it is complex differentiable at every $z\in\Omega$. 
\end{defn}

\begin{defn}{Biholomorphic Functions}{} Let $U_1,U_2\subseteq\C$ be open. A function $f:U_1\to U_2$ is said to be biholomorphic if it is a bijection such that both $f$ and $f^{-1}$ are holomorphic. 
\end{defn}

Notice that while $f$ can be written as a function of two real variables, the notion of complex differentiable is stronger than that in multivariable calculus since we can approach the limit in complex differentiability through say $h=i\alpha$ and $\alpha\to 0$. 

\begin{thm}{Cauchy Riemann Equations}{} Let $U\subseteq\C$ be open. Let $f(x+iy)=u(x,y)+iv(x,y)$ be real differentiable on $U$. Then $f$ is holomorphic on $U$ if and only if $$\frac{\partial u(x,y)}{\partial x}=\frac{\partial v(x,y)}{\partial y}$$ and $$\frac{\partial v(x,y)}{\partial x}=-\frac{\partial u(x,y)}{\partial y}$$\tcbline
\begin{proof}
Suppose first that $f$ is complex differentiable at $z_0$. Then in particular, the limit exists when $h$ is purely real. This means that $$\frac{\partial f}{\partial x}\bigg{|}_{z_0}=\lim_{\substack{h\to 0\\h\in\R}}\frac{f(z_0+h)-f(z_0)}{h}$$ and similarly along the imaginary axis, $$\frac{1}{i}\frac{\partial f}{\partial y}\bigg{|}_{z_0}=\lim_{\substack{ih\to 0\\h\in\R}}\frac{f(z_0+h)-f(z_0)}{h}$$ Since the two are required to be compatible, we have $$i\frac{\partial f}{\partial x}\bigg{|}_{z_0}=\frac{\partial f}{\partial y}\bigg{|}_{z_0}$$ Recall that $\frac{\partial f}{\partial x}=\frac{\partial u}{\partial x}+i\frac{\partial v}{\partial x}$ and similar for $y$, then this is precisely the Cauchy Riemann Equations. \\~\\
Now suppose that $f$ is real differentiable and the Cauchy Riemann equations are satisfied. 
\end{proof}
\end{thm}

The following definition is purely by convention. There is nothing analytic or related to derivatives in their definition. 

\begin{defn}{Partial Derivatives in z}{} Let $f:\Omega\subseteq\C\to\C$ be complex differentiable. Define $$f_z=\frac{\partial f}{\partial z}=\frac{1}{2}\left(\frac{\partial f}{\partial x}-i\frac{\partial f}{\partial y}\right)\;\;\;\;\text{ and }\;\;\;\;f_{\overline{z}}=\frac{\partial f}{\partial \overline{z}}=\frac{1}{2}\left(\frac{\partial f}{\partial x}+i\frac{\partial f}{\partial y}\right)$$
\end{defn}

Under these definition, the Cauchy Riemann equations becomes simple. 

\begin{lmm}{}{} Let $f:\Omega\subseteq\C\to\C$ be real differentiable. Then $f$ is complex differentiable if and only if the following alternate version of Cauchy Riemann Equations are satisfied: $$f_{\overline{z}}=0$$ In this case, the complex derivative at $z\in\Omega$ is precisely $$f'(z)=f_z$$ \tcbline
\begin{proof}
Notice that the condition that $\frac{\partial u}{\partial x}=\frac{\partial v}{\partial y}$ and $\frac{\partial u}{\partial y}=-\frac{\partial v}{\partial x}$ is equivalent to saying $\frac{\partial f}{\partial x}=-i\frac{\partial f}{\partial y}$. In this case, it is easy to see that we have $f_{\overline{z}}=0$ and $f'(x)=f_z$. 
\end{proof}
\end{lmm}

The product rule and the chain rule also hold for partial derivatives: 

\begin{lmm}{}{} Let $U,V\subseteq\C$ be open. Let $f:U\to\C$ and $g:V\to\C$ be holomorphic functions. Then the following are true. 
\begin{itemize}
\item Product rule: $(fg)_z=f\cdot g_z+g\cdot f_z$ on $U\cap V$
\item Chain rule: $(f\circ g)_z(z)=f_z(g(z))g_z(z)$ if $g(V)\subseteq U$
\end{itemize}
\end{lmm}

\subsection{Conformal Maps}
\begin{defn}{Conformal Maps}{} Let $U\subseteq\C$ be open. A function $f:U\to\C$ is said to be a conformal map if $f$ is holomorphic and $f'(z)\neq 0$ for all $z\in U$. 
\end{defn}

Intuitively, we would like conformal maps to preserve straight angles in the complex plane. 

\begin{lmm}{}{} Let $U_1,U_2$ be open. If $f:U_1\to U_2$ is a biholomorphic map then both $f$ and $f^{-1}$ are conformal. \tcbline
\begin{proof}
Let $z\in U_1$ and $w=f(z)$. It is clear that $$\frac{d}{dw}\left(f(f^{-1}(w))\right)=1$$ By the chain rule, we have that $f'(f^{-1}(w))\cdot\frac{d}{dw}(f^{-1}(w))=1$ which implies $f'(f^{-1}(w))\neq 0$. This means that $f'(z)\neq 0$. The case for $f^{-1}$ is similar by considering $f^{-1}\circ f$. 
\end{proof}
\end{lmm}

\begin{defn}{Conformally Equivalent}{} Two open and connected sets $\Omega_1,\Omega_2\subseteq\C$ are said to be formally equivalent if there exists a biholomorphic function $\varphi:\Omega_1\to\Omega_2$. 
\end{defn}

\subsection{Stereographic Projection}
\begin{defn}{The Extended Complex Numbers}{} The extended complex numbers consists of the complex numbers $\C$ and $\{\infty\}$, denoted $\C_\infty$. We define addition, subtraction and multiplication with this number as $z\pm\infty=\infty$, $z\cdot\infty=\infty$. We also define $\frac{1}{\infty}=0$ and $\frac{1}{0}=\infty$. 
\end{defn}

\begin{thm}{}{} The one point compactification of $\C$ into $\C_\infty$ turns $\C_\infty$ into a compact topological space. 
\end{thm}

\begin{defn}{Stereographic Projection}{} Denote $S^2\subset\R^3$ the unit sphere. The stereographic projection from $S^2$ to $\C$ is the map $\pi:S^2\to\C\infty$ defined by $$\pi(x_1,x_2,x_3)=\frac{x_1+ix_2}{1-x_3}$$
\end{defn}

\begin{prp}{}{} The stereographic projection $\pi$ is bijective. In particular, the inverse map is given by $$\pi^{-1}(x+iy)=\left(\frac{2x}{1+\abs{x+iy}^2},\frac{2y}{1+\abs{x+iy}^2},\frac{\abs{x+iy}^2-1}{1+\abs{x+iy}^2}\right)$$
\end{prp}

\begin{prp}{}{} The stereographic projection is a homeomorphism. 
\end{prp}

\pagebreak
\section{Complex Analytic Functions}
\subsection{Power Series}
Recall the notion of power series in real analysis. We can extend that notion into the realm of complex analysis. 
\begin{defn}{Power Series}{} We say that a function $f:\C\to\C$ is a power series if $f$ is of the form $$f(z)=\sum_{k=0}^{\infty}a_k(z-z_0)^k$$ where $a_n\in\C$ for all $n\in\N$ and $z_0\in\C$ is a fixed complex number. 
\end{defn}

In particular, just as there is a radius of convergence for real power series, there is an identical one with that for complex power series. 

\begin{thm}{}{} Let $f:\C\to\C$ be a power series where $f(z)=\sum_{k=0}^\infty a_kz^k$, the radius of convergence is $$R=\frac{1}{\limsup_{n\to\infty}\abs{a_n}^{\frac{1}{n}}}$$ The power series converges for $\abs{x}<R$. The power series diverges for $\abs{x}>R$
\end{thm}

\begin{thm}{}{} Let $f:\C\to\C$ be a power series with radius of convergence $R\in(0,\infty]$, then $f$ is holomorphic within the ball $B_R(0)$, the derivative is given by $$f'(z)=\sum_{k=1}^\infty ka_kz^{k-1}$$ and also has radius of convergence $R$. 
\end{thm}

Repeatedly applying the above result to the derivative gives the following corollary. 

\begin{crl}{}{} Let $f:\C\to\C$ be a power series. If $f$ has radius of convergence $R\in(0,\infty]$, then $f$ is infinitely differentiable in the ball $B_R(0)$, and the $n$-th derivative of $f$ at $0\in\C$ is given by $$f^{(n)}(0)=a_nn!$$
\end{crl}


\begin{thm}{}{} Let $f:\C\to\C$ be a power series of the form $$f(z)=\sum_{k=0}^\infty a_kz^k$$ Define $s_n=\sum_{k=1}^na_kz^k$ the partial sum. If $f$ has radius of convergence $R\in(0,\infty]$, then for all $r\in R$, $s_n\to f$ uniformly within the ball $B_r(0)$ as $n\to\infty$. 
\end{thm}

\subsection{Analytic Functions}
\begin{defn}{Analytic Functions}{} Let $U\subseteq\C$ be open. Let $f:U\to\C$ be a function. We say that $f$ is analytic if for each $w\in U$, we can write $f$ as a power series $$f(z)=\sum_{k=0}^\infty a_k(z-w)^k$$ for all $z\in B_r(w)$ for some $r>0$. 
\end{defn}

\begin{prp}{}{} Let $U,V\subseteq\C$ be open. Let $f:U\to\C$ and $g:V\to\C$ be analytic functions. Then the following are also analytic functions: 
\begin{itemize}
\item The sum $f\pm g:U\cap V\to\C$
\item The product $f\cdot g:U\cap V\to\C$
\item The quotient $\frac{f}{g}:U\cap V\to\C$ provided that $g\neq 0$
\item The composition $f\circ g:V\to\C$ provided that $g(V)\subseteq U$
\end{itemize}
\end{prp}

\begin{prp}{}{} Let $f:\C\to\C$ be a power series whose radius of convergence is $R$. Then $f$ is analytic on $B_R(0)$
\end{prp}

\begin{crl}{}{} Let $U\subseteq\C$ be open. If $f:U\to\C$ is analytic on $U$, then $f$ is holomorphic on $U$. 
\end{crl}

\subsection{The Exponential and Logarithm}
\begin{defn}{Exponential Function}{} The exponential function is defined to be the function $$e^z=\sum_{k=0}^\infty\frac{z^k}{k!}$$ for all $z\in\C$. 
\end{defn}

\begin{thm}{}{} The exponential function $f(z)=e^z$ is convergent for all $z\in\C$. 
\end{thm}

\begin{prp}{}{} Let $z,z_1,z_2\in\C$. Then
\begin{itemize}
\item $e^{z_1+z_2}=e^{z_1}e^{z_2}$
\item $e^{mn}=(e^{m})^n$
\item $\frac{d}{dz}e^z=e^z$
\item $e^{i\theta}=\cos(\theta)+i\sin(\theta)$ for $\theta\in\R$. 
\end{itemize}
\end{prp}

\begin{thm}{}{} Every $z\in\C$ can be written in the form $$z=re^{i\theta}$$ where $r=\abs{z}>0$ and $\theta\in\arg(z)$. 
\end{thm}

\begin{lmm}{}{} Suppose that $\theta$ and $\phi$ are real numbers.Then $e^{i\theta}=e^{i\phi}$ if and only if there exists some $k\in\Z$ such that $\phi=\theta+2\pi k$. 
\end{lmm}

From now on we denote the logarithm for the real numbers by $\text{Log}:\R\to\R$. 

\begin{defn}{Logarithm of a Complex Number}{} Define the logarithm of $z\in\C$ to be $\log(z)=\text{Log}(\abs{z})+i\arg(z)$. In terms of the $\text{Arg}(z)$ function, we have $$\log(z)=\text{Log}(\abs{z})+i\text{Arg}(z)+2i\pi k$$ for $k\in\Z$. Define the principal branch of the Logarithm to be $$\log(z)=\text{Log}(z)+i\text{Arg}(z)$$
\end{defn}

\begin{defn}{Principal Branch}{} Let $z=re^{i\theta}$. Define the principal branch of the logarithm to be $\log(z)=\log(r)+i(\theta)$ where $-\pi<\theta\leq\pi$
\end{defn}


\pagebreak
\section{Integration of Complex Functions}
\subsection{Curves and Paths}
Note that different authors have different definitions for paths and curves. Be sure to stick with the correct wording of the book in question. 
\begin{defn}{Paths and Curves}{} A path in the complex plane is a function $\gamma:[a,b]\to\C$ such that $\gamma$ is continuous. A curve in the complex plane is a continuous $\mathcal{C}^1$ function $\gamma:[a,b]\to\C$. We further classifty the curves as following. 
\begin{itemize}
\item We say that a path $\gamma:[a,b]\to\C$ is simple if $t_1,t_2\in[a,b]$ and $\gamma(t_1)\neq\gamma(t_2)$ implies $t_1\neq t_2$ except possibly $\gamma(a)=\gamma(b)$. In other words, simple paths does not intersect themselves. 
\item We say that a path is closed if $\gamma(a)=\gamma(b)$
\end{itemize}
\end{defn}

\begin{defn}{Contour}{} A contour in the complex plane is a sequence of curves $$\gamma=\{\gamma_1,\dots,\gamma_n\}$$ such that the end point of $\gamma_j$ is equal to the begining point of $\gamma_{j+1}$ and each $\gamma_j$ is a curve. It is also called a piecewise smooth curve. \\~\\
We stay that a contour is a step path if $\gamma=\{\gamma_1,\dots,\gamma_n\}$, and $\gamma_k$ has the property that either the real part or the imaginary part is constant for each individual $\gamma_k$
\end{defn}

\begin{defn}{Reparametrization}{} Let $\gamma:[a,b]\to\C$ be a path and $\rho:[c,d]\to[a,b]$ be continuous and satisfies $\rho(c)=a$ and $\rho(d)=b$. Then $\gamma\circ\rho$ is also a path and is called a reparametrization of $\gamma$. 
\end{defn}

\begin{defn}{Equivalent Parametrization}{} Let $\gamma_1:[a,b]\to\C$ and $\gamma_2:[c,d]\to\C$ be paths. They are said to have equivalent parametrization if there exists a strictly increasing and continuously differentiable function $\phi:[c,d]\to[a,b]$ such that $$\phi(c)=a, \phi(d)=b$$ and $$\gamma_2(t)=\gamma_1(\phi(t))$$ for all $t\in[c,d]$
\end{defn}

\subsection{Integration on Paths}
\begin{defn}{Integral of Complex Valued Functions}{} Let $f:[a,b]\to\C$ and $f(t)=u(t)+iv(t)$. Define $$\int_a^bf(t)\,dt=\int_a^bu(t)\,dt+i\int_a^bv(t)\,dt$$
\end{defn}

This is just a natural extension of the definition given by real analysis. Notice that in particular, we only discuss complex integration from a real interval to $\C$. To extend this definition to a complex function with complex domain, we use paths on the domain so that the composition of the path with the function gives an integral from a real intervel to $\C$. Which we then can define through the above natural extension. 

\begin{defn}{Contour Integral}{} Let $f:\Omega\subseteq\C\to\C$ be continuous. Let $\gamma:[a,b]\to\Omega$ be a curve, then define the integral of $f$ along $\gamma$ to be $$\int_{\gamma}f(z)\,dz=\int_a^bf(\gamma(t))\gamma'(t)\,dt$$~\\
Now let $\gamma=\{\gamma_1,\dots,\gamma_n\}$ be a contour. Define the contour integral along $f$ to be $$\int_{\gamma}f=\sum_{k=1}^n\int_{\gamma_k}f$$
\end{defn}

\begin{prp}{}{} Let $f:\Omega\subseteq\C\to\C$ be continuous. Let $\gamma:[a,b]\to\Omega$. Let $\gamma^-$ represent $\gamma$ in the opposite direction. Then $$\int_{\gamma^-}f(z)\,dz=-\int_\gamma f(z)\,dz$$ \tcbline
\begin{proof} Apply the previous theorem with $\gamma+\gamma^-$. 
\end{proof}
\end{prp}

\begin{lmm}{Estimation Lemma}{} Let $f:U\subseteq\C\to\C$ be continuous such that there exists $M<\infty$ for which $\abs{f(z)}\leq M$ for all $z\in\gamma$. Let $\gamma:[a,b]\to\C$ be a $C^1$ curve. Then $$\abs{\int_{\gamma}f(z)\,dz}\leq M\int_a^b\abs{\gamma'(t)}\,dt$$
\end{lmm}

\begin{prp}{Independence of Equivalent Parametrization}{} Suppose that $\gamma_1:[a,b]\to\Omega\subseteq\C$ and $\gamma_2:[c,d]\to\Omega$ are equivalent parametrizations. Let $f:\Omega\to\C$ be a continuous function. Then $$\int_{\gamma_1}f(z)\,dz=\int_{\gamma_2}f(z)\,dz$$ \tcbline
\begin{proof} Since $\gamma_1,\gamma_2$ are equivalent parametrizations, there exists a strictly increasing and continuously differentiable function $\phi$ such that $\gamma_2=\gamma_1\circ\phi$. Then
\begin{align*}
\int_{\gamma_2}f(z)\,dz&=\int_c^df(\gamma_2(t))\gamma_2'(t)\,dt\\
&=\int_c^df(\gamma_1(\phi(t)))\gamma_1'(\phi(t))\phi'(t)\,dt\tag{Let $u=\phi(t)$}\\
&=\int_a^bf(\gamma_1(u))\gamma_1'(u)\,du\\
&=\int_{\gamma_1}f(z)\,dz
\end{align*}
and so we conclude. 
\end{proof}
\end{prp}

\begin{defn}{Antiderivative}{} Let $f$ be continuous on $\Omega\subseteq\C$. We say that $F$ is an antiderivative of $f$ if $F$ is holomorphic on $\Omega$ and $F'(z)=f(z)$ for all $z\in\Omega$
\end{defn}

We postpone the requirements for the existence of antiderivatives and first see an important property the intergral exhibits if the function has an antiderivative. 

\begin{thm}{}{} Let $f:\Omega\subseteq\C\to\C$ be continuous. Suppose that $f$ has an antiderivative $F$. Let $a,b\in U$. Let $\gamma$ be a path joining $a$ to $b$. Then $$\int_{\gamma}f=F(\gamma(b))-F(\gamma(a))$$ \tcbline
\begin{proof} Note that we have $$\frac{d}{dt}F(\gamma(t))=f(\gamma(t))\gamma'(t)$$ from chain rule. Thus
\begin{align*}
\int_{\gamma}f(z)\,dz&=\int_a^bf(\gamma(t))\gamma'(t)\,dt\\
&=\int_a^b\frac{d}{dt}F(\gamma(t))\,dt\\
&=F(\gamma(b))-F(\gamma(a))\tag{By the FTC}
\end{align*}
\end{proof}
\end{thm}

\pagebreak
\section{Cauchy's Theorem}
\subsection{Winding Number}
We begin with a special case of the Path Lifting Property given in algebraic topology 1. 

\begin{thm}{Lifting Lemma}{} Let $\gamma:[a,b]\to\C\setminus\{z_0\}$ be a closed continuous path. Then there exists a function $\theta:[a,b]\to\R$ such that $\theta$ is continuous and for every $t\in[a,b]$, the argument of $\gamma(t)$ is equal to $\theta(t)$. 
\end{thm}

This produces a well defined integer, called the winding number for each continuous path. 

\begin{defn}{Winding Number}{} Define the winding number of a closed continuous path $\gamma:[a,b]\to\C\setminus\{z_0\}$ around $z_0$ to be $$I(\gamma,z_0)=\frac{\theta(b)-\theta(a)}{2\pi}$$
\end{defn}

\begin{prp}{}{} Let $\gamma:[a,b]\to\C\setminus\{z_0\}$ be a closed continuous path. Then the following are true regarding the winding number of $\gamma$. 
\begin{itemize}
\item Translation Invariance: Let $\tau:[a,b]\to\C\setminus\{0\}$ be defined as $\tau(t)=\gamma(t)-z_0$ for $t\in[a,b]$. Then $I(\gamma,z_0)=I(\tau,0)$
\item Linearity: Suppose that $\gamma=\gamma_1+\gamma_2$ be piecewise. Then $I(\gamma_1+\gamma_2,z_0)=I(\gamma_1,z_0)+I(\gamma_2,z_0)$
\end{itemize}\tcbline
\begin{proof}
Translation invariance: \\
Linearity: \\
Suppose that $\gamma_1$ and $\gamma_2$ has common end point $c$. Then we have
\begin{align*}
I(\gamma_1,z_0)+I(\gamma_2,z_0)&=\frac{\theta(b)-\theta(c)}{2\pi}+\frac{\theta(c)-\theta(a)}{2\pi}\\
&=\frac{\theta(b)-\theta(a)}{2\pi}\\
&=I(\gamma_1+\gamma_2,z_0)
\end{align*}
and so we conclude. 
\end{proof}
\end{prp}

\begin{lmm}{}{} Let $\gamma:[a,b]\to\C$ be a closed continuous path that avoids a radial line for some point $w\in\C$ to $\infty$, then $I(\gamma,w)=0$. \tcbline
\begin{proof}
We can take $\theta(t)$ to be $\text{arg}(\gamma(t))$ since the argument in this case also avoids the radial line so that it is continuous. Then we have that $I(\gamma,w)=\frac{\text{arg}(\gamma(b)-\text{arg}(\gamma(a)}{2\pi }=0$. 
\end{proof}
\end{lmm}

\begin{lmm}{Dog Walking Lemma}{} Let $\gamma_1,\gamma_2:[a,b]\to\C\setminus\{0\}$ be closed continuous paths with $\abs{\gamma_1(t)-\gamma_2(t)}<\abs{\gamma(t)}$ for all $t\in[a,b]$. Then $$I(\gamma_1,0)=I(\gamma_2,0)$$
\end{lmm}

\begin{lmm}{}{} Suppose that $\gamma:[a,b]\to\C$ is a closed continuous path. Then on each connected component of $\C\setminus\gamma([a,b])$, the function $z\to I(\gamma,z)$ is constant. 
\end{lmm}

\begin{thm}{}{} Let $\gamma$ be a closed continuous path. Then $$I(\gamma,0)=\frac{1}{2\pi i}\int_{\gamma}\frac{1}{z}\,dz$$ More specifically, $$I(\gamma,z_0)=\frac{1}{2\pi i}\int_{\gamma}\frac{1}{z-z_0}\,dz$$ \tcbline
\begin{proof}
Without loss of generality, we may assume that $w=0$ by lemma 4.1.3 and that $\gamma$ is a $C^1$ curve. Let $\theta:[a,b]\to\R$ be a lift from theorem 4.1.1 so that $\gamma(t)=\abs{\gamma(t)}e^{i\theta(t)}$. Notice that since $\gamma$ is $C^1$ and is away from $0$, $\theta(t)$ is also $C^1$. We can then compute $$\gamma'(t)=e^{i\theta(t)}\frac{d}{dt}\abs{\gamma(t)}+\abs{\gamma(t)}i\theta'(t)e^{i\theta(t)}$$ which implies $$\frac{\gamma'(t)}{\gamma(t)}=\left(\frac{d}{dt}\log(\abs{\gamma(t)})\right)+i\theta'(t)$$ Integrating gives 
\begin{align*}
\int_\gamma\frac{1}{z}\,dz&=\int_a^b\left(\frac{d}{dt}\log(\abs{\gamma(t)})\right)+i\theta'(t)\,dt\\
&=0+i(\theta(b)-\theta(a))\\
&=2\pi iI(\gamma,0)
\end{align*}
and so we conclude. 
\end{proof}
\end{thm}

\begin{prp}{}{} Let $\gamma[a,b]\to\C$ be a closed continuous path. Then the set of points $w\in\C\setminus\gamma([a,b])$ such that $I(\gamma,w)\neq 0$ is bounded. \tcbline
\begin{proof}
By continuity and compactness of $[a,b]$, the image $\gamma([a,b])$ must lie within some ball $B_R(0)$ for some $R>0$. Denote $$A=\{w\in\C\setminus\gamma([a,b])\;|\;I(\gamma,w)\neq 0\}$$ Our goal now is to show that $A$ is bounded in $B_R(0)$. Let $w\notin B_R(0)$. Then $\gamma$ avoids the radical line from $w$ to infinity. By lemma 4.1.4, $I(\gamma,w)=0$ which implies that $w\notin A$. Thus $A\subseteq B_R(0)$ and so it is bounded. 
\end{proof}
\end{prp}

\subsection{Goursat's Theorem and Cauchy's Theorem}
Goursat's theorem markes the start of our quest of finding anti-derivatives. Any proof of existence of anti-derivatives comes fundamentally from Goursat's theorem. 

\begin{thm}{Goursat's Theorem}{} Let $U\subseteq\C$ be open and let $R$ be a triangle contained in $U$. Let $f:U\to\C$ be holomorphic. Then $$\int_{\partial R}f=0$$ \tcbline
\begin{proof} Denote our triangle $T_0$. Say that our triangle has perimeter and diameter $p_0$ and $d_0$. Divide $T_0$ into four smaller triangles by bisecting each side of $T_0$ and connecting it. Denote it $T_1^1,\dots,T_1^4$. Then $$\int_{\partial T_0}f(z)\,dz=\sum_{k=1}^4\int_{\partial T_1^k}f(z)\,dz$$ Then for some $k\in\{1,2,3,4\}$ we have $$\abs{\int_{\partial T_0}f(z)\,dz}\leq 4\abs{\int_{\partial T_1^k}f(z)\,dz}$$ Thus we can drop the arbitrary triangle and write it as $$\int_{\partial T_1}f(z)\,dz$$ Since the four new triangles are equivalent, we have $p_1=\frac{1}{2}p_0$ and $d_1=\frac{1}{2}d_0$. Repeat this procedure $n$ times will result in $$\abs{\int_{\partial T_0}f(z)\,dz}\leq 4^n\abs{\int_{\partial T_n}f(z)\,dz}$$ and $d_n=\frac{1}{2^n}d_0$ and $p_n=\frac{1}{2^n}p_0$. The triangles form a sequence of compact sets with $T_0\supseteq T_1\supseteq\dots$ By Cantor's Intersection Theorem, the intersection of $T_0,T_1,\dots$ is non-empty. Let $z_0\in\bigcap_{k=0}^\infty T_k$, then since $f$ is analytic, $$f(z)=f(z_0)+f'(z_0)(z-z_0)+\eta(z-z_0)$$ where $\lim_{z\to z_0}\eta(z_0)=0$. We have that
\begin{align*}
\int_{\partial T_n}f(z)\,dz&=\int_{\partial T_n}f(z_0)\,dz+\int_{\partial T_n}f'(z_0)(z-z_0)\,dz+\int_{\partial T_n}\eta(z)(z-z_0)\,dz\\
&=\int_{\partial T_n}\eta(z-z_0)\,dz
\end{align*} Since $\lim_{z\to z_0}\eta(z)=0$, then for any $\epsilon>0$ there exists $\delta>0$ such that $$\abs{z-z_0}<\delta\implies\abs{\eta(z)}<\frac{2}{d_0^2}\epsilon$$ Choose $n$ such that $T_k^n\subset B_\delta(z_0)$. We also have that $z\in T_n$ implies that $\abs{z-z_0}\leq d_{n+1}$. Now we have
\begin{align*}
\abs{\int_{\partial T_0}f(z)\,dz}&\leq4^n\abs{\int_{\partial T_n}f(z)\,dz}\\
&\leq4^n\abs{\int_{\partial T_n}\eta(z)(z-z_0)\,dz}\\
&\leq4^n\frac{2}{d_0^2}\epsilon d_{n+1}d_n\\
&=\epsilon
\end{align*} Thus we have that $\int_{\partial T_0}f(z)\,dz=0$
\end{proof}
\end{thm}

\begin{defn}{Star-Shaped Domain}{} Let $\Omega\subseteq\C$ be open. It is called a star shaped domain if there exists $z_0\in\Omega$ such that for all $z\in\Omega$, the line segment $[z_0,z]\subseteq\Omega$. 
\end{defn}

Notice that star-shaped domains are necessarily connected since it is path connected via the central point $z_0$. 

\begin{thm}{}{} Suppose that $\Omega$ is a star-shaped domain, and $f:\Omega\to\C$ is a continuous function. Suppose that for every closed triangle $T\subset\Omega$ we have $$\int_{\partial T}f(z)\,dz=0$$ Then there exists a holomorphic function $F:\Omega\to\C$ such that $F'(z)=f(z)$. Specifically, we can choose $$F(z)=\int_{[z_0,z]}f(w)\,dw$$ \tcbline
\begin{proof}
Let $r>0$ be sufficiently small so that $B_r(z)\subset\Omega$. For each $h\in B_r(0)$, the point $z+h$ and indeed the whole segment $[z,z+h]$ lies in $\Omega$. Since $\Omega$ is star-shaped with respect to $z_0$, the entire closed triangle $T$ with vertices $z_0,z$ and $z+h$ must lie in $\Omega$. By hypothesis, we have that $$\int_{\partial T}f(w)\,dw=0$$ This means that $$F(z+h)-\int_{[z,z+h]}f(w)\,dw-F(z)=0$$ Since we also have $$\int_{[z,z+h]}\,dw=\int_0^1\gamma'(t)\,dt=\gamma(1)-\gamma(0)=h$$ we can combine this result to have 
\begin{align*}
\lim_{h\to 0}\abs{\frac{F(z+h)-F(z)}{h}-f(z)}&=\lim_{h\to 0}\abs{\frac{1}{h}\int_{[z,z+h]}(f(w)-f(z))\,dw}\\
&=\lim_{h\to 0}\max_{w\in[z,z+h]}\abs{f(w)-f(z)}\\
&=0
\end{align*}
because $f$ is continuous. This shows that $F$ is complex differentiable with $F'(z)=f(z)$. 
\end{proof}
\end{thm}

This theorem will be used in the proof of Morera theorem later. 

\begin{crl}{}{} Suppose that $\Omega$ is a star-shaped domain, and $f:\Omega\to\C$ is a holomorphic function. Then there exists a holomorphic function $F:\Omega\to\C$ such that $F'(z)=f(z)$. \tcbline
\begin{proof}
Let $\Omega$ be a star-shaped domain. Then by Goursat's theorem, every triangle $T$ in $\Omega$ is such that $\int_\gamma f(z)\,dz=0$. Then this satisfies the hypoethesis of the above theorem and so we conclude. 
\end{proof}
\end{crl}

\begin{thm}{Cauchy's Theorem on Star-Shaped Domain}{} Let $\Omega\subseteq\C$ be star-shaped. Let $f:\Omega\to\C$ be holomorphic. Let $\gamma$ be closed in $\Omega$. Then $$\int_{\gamma}f(z)\,dz=0$$ \tcbline
\begin{proof}
By the above theorem, $f$ has an antiderivative. We conclude by theorem 3.2.8. 
\end{proof}
\end{thm}

\begin{crl}{Cauchy's Theorem on Annuli}{} Suppose that $0\leq r_1<R_1<R_2<r_2$, and that $f$ is a holomorphic function on the annulus $A_{r_1,r_2}=\{z\in\C\;|\; r_1<\abs{z}<r_2\}$. Then writing $A=\{z\in\C\;|\; R_1<\abs{z}<R_2\}$, we have $$\int_{\partial A}f(z)\,dz=0$$ \tcbline
\begin{proof}
When $r_1=0$, we can divide the annulus $A_{R_1,R_2}$ into four quarter. Call their interior $A_1,A_2,A_3,A_4$ respectively. We can write $$\int_{\partial A}f(z)\,dz=\sum_{j=1}^4\int_{\partial A_j}f(z)\,dz$$ Each quarter $A_j$ is contained in a half disc, such as $A_1$ is contained in $D_1=\left\{z=re^{i\theta}\in A_{r_1,r_2}\;|\; \frac{\pi}{2}+\frac{\pi}{4}\leq\theta\leq 2\pi-\frac{\pi}{4}\right\}$. These discs are star-shaped, we can apply Cauchy's theorem on star-shaped domain to deduce that each term in the sum $\sum_{j=1}^4\int_{\partial A_j}f(z)\,dz$ is $0$. \\~\\

If $r_1>0$, then there exists narrow enough sectors $\{z\in\C\;|\;\text{arg}(z)\in(0,\delta)\}$ that cover $A$ so that we can apply Cauchy's theorem on star-shaped domain. 
\end{proof}
\end{crl}

\subsection{Taylor's Theorem and Cauchy's Integral Formula}
Taylor's theorem is one of the milestones in complex analysis. Many important results follow from Taylor's theorem. In particular, the immediate consequence is that the term holomorphic is the same as analytic: they both are just power series. \\~\\

Before we state Taylor's theorem, we need another very important theorem. 

\begin{thm}{Cauchy's Integral Formula on a Disc}{} Let $U\subseteq\C$ be open. Let $\overline{B_r(a)}\subset U$. Let $f:U\to\C$ be holomorphic. Then for every $w\in B_r(a)$, we have $$f(w)=\frac{1}{2\pi i}\oint_{\partial B_r(a)}\frac{f(z)}{z-w}\,dz$$ \tcbline
\begin{proof}
Consider the function $\frac{f(z)-f(w)}{z-w}$. Since $f$ is holomorphic, the new function is well defined and holomorphic on $U\setminus\{w\}$. Let $0<\epsilon<r-\abs{w-z_0}$. Then the circle $\partial B_\epsilon(w)$ lies in $B_r(a)$. Define a path $\gamma_1$ as follows. Take the boundary of the semicircle $B_r(a)$ with diameter going through $a$ and $z_0$. Instead of going through $z_0$, the path walks along the upper boundary $\partial B_\epsilon(w)$. $\gamma_2$ is defined similarly for the semicircle in the lower half. \\~\\

Applying Cauchy's theorem for star-shaped domains, we have $$\int_{\partial B_r(a)}\frac{f(z)-f(w)}{z-w}\,dz=\int_{\partial B_\epsilon(w)}\frac{f(z)-f(w)}{z-w}\,dz$$ By the estimation lemma, we have, $$\abs{\int_{\partial B_r(z)}\frac{f(z)-f(w)}{z-w}\,dz}=\abs{\int_{\partial B_\epsilon(w)}\frac{f(z)-f(w)}{z-w}\,dz}\leq M\cdot 2\pi\epsilon$$ As $\epsilon$ tends to $0$, we have that $$\int_{\partial B_r(a)}\frac{f(z)-f(w)}{z-w}\,dz=0$$ Thus we have that 
\begin{align*}
\int_{\partial B_r(a)}\frac{f(z)}{z-w}\,dz&=\int_{\partial B_r(z)}\frac{f(w)}{z-w}\,dz\\
&=f(w)\int_{\partial B_r(a)}\frac{1}{z-w}\,dz\\
&=f(w)\cdot 2\pi i
\end{align*} This gives $$f(w)=\frac{1}{2\pi i}\int_{\partial B_r(a)}\frac{f(z)}{z-w}\,dz$$ and so we conclude
\end{proof}
\end{thm}

Finally we can prove Taylor's theorem. 

\begin{thm}{Taylor's Theorem}{} Let $\Omega\subseteq\C$ be open. Let $f:\Omega\subseteq\C$ be holomorphic. Let $z_0\in\Omega$ and $r>0$ and $B_r(z_0)\subseteq\Omega$. Then all higher derivatives of $f$ exists and for all $z\in B_r(z_0)$ we can write $$f(z)=\sum_{k=0}^\infty a_k(z-z_0)^k$$ where $$a_k=\frac{1}{2\pi i}\int_{\partial B_r(z_0)}\frac{f(w)}{(w-z_0)^{k+1}}\,dw$$ \tcbline
\begin{proof} Using Cauchy's Integral Formula, we have that $$f(z)=\frac{1}{2\pi i}\int_{\partial B_r(z_0)}\frac{f(w)}{w-z}\,dw$$ Now since $\abs{z-z_0}<r=\abs{w-z_0}$, we can we write $\frac{1}{w-z}$ as 
\begin{align*}
\frac{1}{w-z}&=\frac{1}{w-z_0}\frac{1}{1-\frac{z-z_0}{w-z_0}}\\
&=\frac{1}{w-z_0}\sum_{k=0}^\infty\left(\frac{z-z_0}{w-z_0}\right)^k
\end{align*}
So we have that 
\begin{align*}
f(z)&=\frac{1}{2\pi i}\int_{\partial B_r(z_0)}\frac{f(w)}{w-z}\,dw\\
&=\frac{1}{2\pi i}\int_{\partial B_r(z_0)}\frac{f(w)}{w-z_0}\sum_{k=0}^\infty\left(\frac{z-z_0}{w-z_0}\right)^k\,dw\\
&=\frac{1}{2\pi i}\sum_{k=0}^\infty\int_{\partial B_r(z_0)}\frac{f(w)}{(w-z_0)^{k+1}}\,dw(z-z_0)^k\\
&=\sum_{k=0}^\infty a_k(z-z_0)^k
\end{align*}
\end{proof}
\end{thm}

\begin{crl}{}{} A function $f:\Omega\to\C$ is holomorphic on an open set $U$ if and only if it is analytic. \tcbline
\begin{proof} If $f$ is analytic, then it can be represented in a power function and power functions are holomorphic. If $f$ is holomorphic then by the above $f$ can be represented by a power function by replacing $z=z_0+h$ and thus is analytic. 
\end{proof}
\end{crl}

\begin{thm}{Cauchy's Integral Formula}{} Let $U\subseteq\C$ be open. Let $f:U\to\C$ be holomorphic. Then $f$ is infinitely complex differentiable. In particular, if $\overline{B_r(z_0)}\subset U$, then we have the formula $$f^{(n)}(z_0)=\frac{n!}{2\pi i}\int_{\partial B_r(z_0)}\frac{f(w)}{(w-z_0)^{n+1}}\,dw$$ for each $n\in\N$. \tcbline
\begin{proof}
We have seen that holomorphic functions are analytic and analytic functions are infinitely complex differentiable. Moreover, by Taylor's theorem, we have the formula $$f^{(n)}(z_0)=a_nn!=\frac{n!}{2\pi i}\int_{\partial B_r(z_0)}\frac{f(w)}{(w-z_0)^{n+1}}\,dw$$ and so we conclude. 
\end{proof}
\end{thm}

We will see two more generalized versions of Cauchy's Integral formula once we have developed sufficient theory for singularities. 

\subsection{Consequences of Taylor's Theorem}
Two immediate applications are due. 

\begin{thm}{Mean Value Theorem}{} Let $U\subseteq\C$ be open and $\overline{B_r(z_0)}\subset U$ for some $r>0$. Let $z_0\in U$ and $f:U\to\C$ be holomorphic. Then $$f(z_0)=\frac{1}{2\pi}\int_0^{2\pi}f(z_0+re^{i\theta})\,d\theta$$ \tcbline
\begin{proof}
By Cauchy's integral formula, we have 
\begin{align*}
f(z_0)&=\frac{1}{2\pi i}\int_{\partial B_r(z_0)}\frac{f(z)}{z-z_0}\,dz\\
&=\frac{1}{2\pi i}\int_0^{2\pi}\frac{f(z_0+re^{i\theta})ire^{i\theta}}{re^{i\theta}}\,d\theta\\
&=\frac{1}{2\pi}\int_0^{2\pi}f(z_0+re^{i\theta})\,d\theta
\end{align*}
\end{proof}
\end{thm}

\begin{crl}{Cauchy's Estimate}{} Suppose that $f(z)=\sum_{k=0}^\infty a_kz^k$ is holomorphic on $B_r(0)$ for some $r>0$ and that for all $z\in B_r(0)$ we have $\abs{f(z)}\leq M$. Then $$\abs{a_k}\leq\frac{M}{r^k}$$ for all $k$. \tcbline
\begin{proof}
Using Taylor's theorem, we have that 
\begin{align*}
\abs{a_k}&\leq\frac{1}{2\pi}\abs{\int_{\partial B_r(z_0)}\frac{f(w)}{w^{k+1}}\,dw}\\
&\leq\frac{1}{2\pi}2\pi r\frac{M}{r^{k+1}}\\
&=\frac{M}{r^k}
\end{align*}
Then allowing $r\to R$ gives the result. 
\end{proof}
\end{crl}

Cauchy's estimate gives a short proof for Liouville's theorem, though it can be proven independent of Cauchy's estimate entirely. 

\begin{crl}{Liouville's Theorem}{} Any bounded holomorphic function is constant. \tcbline
\begin{proof}
By applying Taylor's theorem with $z_0=0$ and arbitrary large $R$, we can write the function $f:\C\to\C$ as a Taylor series $$f(z)=\sum_{k=0}^\infty a_kz^k$$ Since $f$ is bounded, say $\abs{f(z)}\leq M$ for all $z\in\C$, by Cauchy's estimate we have that $$\abs{a_k}\leq\frac{M}{R^k}$$ for every $k\in\N$ and every $R>0$. By taking $R\to\infty$, we see that $a_k=0$ for all $k\geq 1$. 
\end{proof}
\end{crl}

The direct consequence of Lioville's theorem would be the fundamental theorem of algebra. Albeit the name of the theorem, most of the proofs are rather analytic in flavour. 

\begin{thm}{Fundamental Theorem of Algebra}{} Every non-constant polynomial has at least one zero in $\C$. \tcbline
\begin{proof}
Suppose that $f$ is a polynomial such that $\abs{f(z)}\neq 0$ for all $z\in\C$. Define $g:\C\to\C$ by $g(z)=\frac{1}{f(z)}$. Then since $f$ does not vanish, $g(z)$ is analytic for all $z\in\C$. Now assume $f(z)=\sum_{k=0}^nc_kz^k$ with $c_n\neq 0$. Then $\abs{f(z)}\to\infty$ as $z\to\infty$ and satisfies $\abs{f(z)}>1$ for all $\abs{z}>r$ for some $r>0$. This means that $g(z)$ is bounded in $\C$ and less than $1$ for all $\abs{z}>r$. \\~\\
By Liouville's Theorem, $g$ is constant thus $f$ is constant, which is a contradiction. 
\end{proof}
\end{thm}

\begin{prp}{}{} Let $f:\C\to\C$ be a polynomial such that $f$ has degree $n$. Then $f$ has exactly $n$ roots in $\C$ counting repeated roots. \tcbline
\begin{proof}
Suppose that $f$ has degree $1$. Then $f$ has at least one zero by the fundamental theorem of algebra. Thus we are done. Now supoose that the theorem is true for all polynomials with degree less than $n$. Let $f$ has degree $n$. Then by the fundamental theorem of algerba, $f$ has a root, say $w$. Then by division algorithm we have that $f(z)=(z-w)g(z)$ for some $g$ that has degree $n-1$. Applying the induction hypothesis gives that $g$ has $n-1$ roots, and we are done. 
\end{proof}
\end{prp}

Morera's theorem is an inverse to Goursat's theorem. It gives a criteria of differentiability in terms of integrability! Indeed we have seen that holomorphic functions are infinitely differentiable. So if we can exhibit a holomorphic anti-derivative then its derivative is also holomorphic. 

\begin{thm}{Morera's Theorem}{} Suppose $U\subseteq\C$ is open and $f:U\to\C$ is continuous. Suppose that for all closed triangles $T\subset\Omega$ we have $$\int_{\partial T}f(z)\,dz=0$$ Then $f$ is holomorphic on $\Omega$. \tcbline
\begin{proof}
Let $a\in U$. Pick $r>0$ sufficiently small so that $B_r(a)\subset U$. By theorem 4.3.2, we can construct a holomorphic function $F:B_r(a)\to\C$ such that $F'(z)=f(z)$ for all $z\in B_r(a)$. Since $F$ is holomorphic, it is infinitely complex differentiable and thus $f$ is also holomorphic. 
\end{proof}
\end{thm}

\begin{lmm}{Local Invertibility Lemma}{} Let $U\subseteq\C$ be open. Let $f:U\to\C$ is holomorphic with $f'(z_0)\neq 0$ at some $z_0\in U$. Then there exists a neighbourhood $V\subset U$ of $z_0$ and a neighbourhood $W\subset\C$ of $f(z_0)$ such that the restriction $f|_V:V\to W$ is biholomorphic. \tcbline
\begin{proof}
Let $f$ be holomorphic. Then $f$ is infinitely complex differentiable. In particular, it is $C^1$ when considered as a real function. The hypothesis $f'(z_0)\neq 0$ together combined shows that the real derivative of $f$ is invertible because the derivative of $f$ is a linear map that is a rotation and dilation. We can now apply the inverse function theorem to obtain the existence of $V$ and $W$ and the invertibility of the restriction $f|_V$. Moreover, $f^{-1}$ is also $C^1$. By shrinking the neighbourhoods if necessary, we may assume that $f'(z)\neq 0$ for all $z\in V$ because $f'(z_0)\neq 0$ and $f'$ is continuous. \\~\\

It remains to show that $f^{-1}$ is holomorphic. Applying the chain rule, we deduce that $$0=\frac{\partial z}{\partial\overline{z}}=\frac{\partial (f\circ f^{-1})(z)}{\partial\overline{z}}=f'(f^{-1}(z))\cdot\frac{\partial(f^{-1}(z))}{\partial\overline{z}}$$ which shows that $(f^{-1})_{\overline{z}}=0$ since $f'\neq 0$. Similarly, we have that $$1=\frac{\partial z}{\partial z}=\frac{\partial (f\circ f^{-1})(z)}{\partial z}=f'(f^{-1}(z))\cdot\frac{\partial f^{-1}(z)}{\partial z}$$ In particular, because $f'\neq 0$, we have $$\frac{\partial f^{-1}(z)}{\partial z}=\frac{1}{f'(f^{-1}(z))}\neq 0$$ for all $z\in V$. And so we conclude. 
\end{proof}
\end{lmm}

\subsection{Homology Version of Cauchy's Theorem}
\begin{defn}{Cycles in $\C$}{} Let $U\subseteq\C$ be open. A cycle in $U$ is a formal linear combination $$\gamma=a_1\gamma_1+\dots+a_n\gamma_n$$ with each $\gamma_i$ a closed, piecewise $C^1$ curve in $U$ and $a_1,\dots,a_n\in\Z$. 
\end{defn}

Using definition 3.2.2, we may extend the definition of integrals for cycles to be formal linear combinations of the integrals: If $\gamma=a_1\gamma_1+\dots+a_n\gamma_n$ is a cycle in $U\subseteq\C$, then define $$\int_\gamma f(z)\,dz=\sum_{k=1}^na_k\int_{\gamma_k}f(z)\,dz$$ Indeed, it makes sense because $a\gamma$ for $a\in\Z$ is just a concatenation of $a$ copies of $\gamma$ so that the integral now makes sense using definition 3.2.2. Important think to note here is that formal linear combination only makes sense when each $\gamma_1,\dots,\gamma_n$ are closed loops so that they can be concatenated. For general curves as in definition 3.2.2, formal linear combination would not have made geometric sense, albeit you can still define it algebraically. 

\begin{defn}{Homologous to $0$}{} Let $U\subseteq\C$ be open. We say that a cycle $\gamma$ in $U$ is homologous to $0$ if $I(\gamma,a)=0$ for all $a\in\C\setminus\{U\}$. 
\end{defn}

\begin{thm}{Cauchy's Theorem - Homology Version}{} Let $U\subseteq\C$ be open and $f:U\to\C$ holomorphic. Let $\gamma$ be a cycle that is homologous to $0$. Then we have $$\int_\gamma f(z)\,dz=0$$ \tcbline
\begin{proof}
Without loss of generality, suppose that $U$ is bounded. If not, replace $\Omega$ by $\Omega\cap B_R(0)$ where $R>0$ is chosen large enough such that all curves making up $\gamma$ lies in $B_R(0)$. By compactness, the distance of the image of $\gamma$ to $\C\setminus\Omega$ is strictly positive. Denote it by $2\delta>0$. \\~\\

Consider a grid of width $\delta$ on $\C$ made up of closed squares $$\{x+iy\;|\;x\in[k\delta,(k+1)\delta]\text{ and }y\in[l\delta,(l+1)\delta]\}$$ for $k,l\in\Z$. Denote by $\{Q_j\}_{j\in J}$ the finitely many such closed squares that are fully contained in $\Omega$. They combine to make up an open set $$\Omega_\delta=\left(\bigcup_{j\in J}Q_j\right)^\circ$$ that is a slight shrinking of $\Omega$. Let $w$ be a point in the interior of some $Q_{j_0}$. By Cauchy's integral formula, we have that $$f(w)=\frac{1}{2\pi i}\int_{\partial Q_{j_0}}\frac{f(z)}{z-w}\,dz$$ For any other $Q_j$ with $j\neq j_0$, we have that $$\frac{1}{2\pi i}\int_{\partial Q_j}\frac{f(z)}{z-w}\,dz=0$$ by Cauchy's theorem on star-shaped domains. Summing all these identities over all squares making up $\Omega_\delta$, then all integrals over interior edges cancel because all interior edges are traversed twice with opposite directions, and we obtain $$f(w)=\frac{1}{2\pi i}\sum_{j\in J}\int_{\partial Q_j}\frac{f(z)}{z-w}\,dz=\frac{1}{2\pi i}\int_{\partial \Omega_\delta}\frac{f(z)}{z-w}\,dz$$ By continuity, this formula holds for all $w\in\Omega_\delta$. \\~\\

By definition of $\delta$, the image of $\gamma$ is fully contained in $\Omega_\delta$. Also for every $z\in\C\setminus\Omega_\delta$ and for every $z\in\partial\Omega_\delta$, we have that $I(\gamma,z)=0$ since $\gamma$ is homologous to $0$. Integrating the above formula, we have that 
\begin{align*}
\int_\gamma f(w)\,dw&=\int_{\gamma}\frac{1}{2\pi i}\left(\int_{\partial\Omega_\delta}\frac{f(z)}{z-w}\,dz\right)\,dw\\
&=\int_{\partial\Omega_\delta}f(z)\left(\frac{1}{2\pi i}\int_\gamma\frac{1}{z-w}\,dw\right)\,dz\\
&=\int_{\partial\Omega_\delta}f(z)(-I(\gamma,z))\,dz\\
&=0
\end{align*}
And so we conclude. 
\end{proof}
\end{thm}

Notice that if $\gamma=k_1\gamma_1+\dots+k_n\gamma_n$ is a cycle, then $I(\gamma,a)=0$ for all $a\in\C\setminus U$ is the same as saying each individual loop $\gamma_1,\dots,\gamma_n$ is such that $k_1I(\gamma_1,a)+\dots+k_nI(\gamma_n,a)=0$. 

\pagebreak
\section{Zeroes and Singularities}
In this section we will study the zeroes and the singularities of holomorphic functions. There will be a lot of useful tools that we will end up collecting for use in other chapters, such as studying biholomorphic functions. 

\subsection{Zeroes of Holomorphic Functions}
\begin{defn}{Zeros of a Function}{} Let $U\subseteq\C$ be open. A zero of a differentiable function $f:U\subseteq\C\to\C$ is a point $z_0\in U$ for which $f(z_0)=0$. 
\end{defn}

\begin{defn}{Order of Zero at $f$}{} Let $U\subseteq\C$ be open and let $f:U\subseteq\C$ be a holomorphic function with $z_0\in U$ a zero of $f$. Define the order of zero of $f$ at $z_0$ to be $$\ord(f,z_0)=\begin{cases}
\infty & \text{if $f^{(k)}(z_0)=0$} \text{ for all $k\in\N$}\\
\min\{k\in\N:f^{(k)}(z_0)\neq0\} & \text{otherwise}
\end{cases}$$
\end{defn}

The following two theorems is the core of this section. With a zero of finite order, we can factor out the linear factors of the zero, with a zero of infinite order in a connected set, the function is identically $0$. 

\begin{thm}{}{} Let $U\subseteq\C$ be open. Let $f:U\to\C$ be holomorphic with $z_0\in U$ a zero of $f$ with order $n\in\N$. Then there exists a holomorphic function $g:U\to\C$ such that $$f(z)=(z-z_0)^ng(z)$$ and $g$ is non-zero in a neighbourhood of $z_0$. \\~\\

In particular, each zero of finite order is an isolated point of the set of zeros. \tcbline
\begin{proof}
Suppose that $\overline{B_r(z_0)}\subset U$, we can use Taylor's theorem to write $$f(z)=\sum_{k=0}^\infty a_k(z-z_0)^k$$ where $a_k=\frac{f^{(k)}(z_0)}{k!}$. Since $f$ has a zero of order $n$ at $z_0$, we must have that $a_k=0$ for $k<n$ and $a_n\neq 0$. Then for all $z\in B_r(z_0)$, we can write $$f(z)=\sum_{k=n}^\infty a_n(z-z_0^k)=(z-z_0)^n\sum_{k=0}^\infty a_{k+n}(z-z_0)^k$$ It is clear that $g(z)=\sum_{k=0}^\infty a_{k+n}(z-z_0)^k$ gives us the formula since $g(z_0)\neq 0$ and it is well defined on $B_r(z_0)$. By construction, the power series defining $g$ must converge pointwise for each $z\in B_r(z_0)$. So its radius of convergence must be at least $r$ by theorem 2.1.3. By theorem 2.1.4, $f$ is holomorphic. \\~\\

We can extend $g$ to the rest of $U$ be setting $g(z)=f(z)(z-z_0)^{-n}$. It is clear that this definition of $g$ agrees on $B_r(z_0)\setminus\{z_0\}$, and is holomorphic everywhere by the product rule. Because $g$ is continuous and $g(z_0)\neq 0$, $g$ is non-zero in some neighbourhood of $z_0$. 
\end{proof}
\end{thm}

Now notice that some of the theorems below require that the domain of $f$ to be connected in addition to being open. 

\begin{thm}{}{} Suppose that $\Omega\subset\C$ is open and connected. Let $f:\Omega\to\C$ be a holomorphic function that has a zero of infinite order at some $z_0\in\Omega$. Then $f$ is identically $0$. \tcbline
\begin{proof}
Consider the set $S=\{z\in\Omega|\; f\text{ has a zero of infinite order at }z\}$. By assumption, $S$ is non-empty. Let $w\in S$. By Taylor's theorem, we can write $f$ as a power series in any ball $B_r(w)$ so that $f$ has a radius of convergence $r$. By corollary 2.1.5, we have $a_k=\frac{f^{(k)}(w)}{k!}=0$. This means that $f$ is identically $0$ in any open ball $B_r(w)\subset S$. Thus $S$ is open. Now suppose that $(z_n)_{n\in\N}$ is a sequence in $S$ that converges to some $w\in\Omega$, then $f(w)=0$ by continuity of $f$. But $w$ cannot be a zero of finite order since such zeroes are isolated within the set of all zeroes. Thus $w\in S$ and we conclude that $S$ is closed. \\~\\

Since $\Omega$ is connected and $S$ is open and closed and non-empty, we have that $\Omega=S$ and thus $f$ is identically $0$ on $\Omega$. 
\end{proof}
\end{thm}

An immediate application follows from the two theorems, namely the identity theorem. 

\begin{thm}{The Identity Theorem}{} Let $\Omega\subset\C$ be open and connected. Let $f,g:\Omega\to\C$ be holomorphic. If the set $$\{z\in\Omega|f(z)=g(z)\}$$ has at least one accumulation point in $\Omega$, then $f(z)=g(z)$ for all $z\in\Omega$. \tcbline
\begin{proof}
By the hypothesis, $h=f-g$ is holomorphic and has a non-isolated zero. By theorem 5.1.3, the zero must be of infinite order. By theorem 5.1.4, we must have that $h$ is identically $0$ so that $f=g$. 
\end{proof}
\end{thm}

In easier way to think of the identity theorem is to set $g$ to be the zero function. Then the accumulation point condition translates to the following: some point $z\in\Omega$ has some sequence in the zero set that converges to $z$. By continuity of $f$, this implies that $f(z)=0$. \\~\\

Our next goal is to refine our result for zeroes of finite order. We will need to develop some machinery on logarithmic functions for this. 

\begin{lmm}{}{} Suppose $\Omega\subset\C$ is open and connected and $g:\Omega\to\C\setminus\{0\}$ is a holomorphic function such that there exists $F:\Omega\to\C$ for which $$F'(z)=\frac{g'(z)}{g(z)}$$ Then there exists $w_0\in\C$ such that for any $l:\Omega\to\C$ defined by $l(z)=F(z)+w_0$, we have $$g(z)=e^{l(z)}$$ for all $z\in\Omega$. The function $l$ is unique up to ad additive constant $2\pi i n$ for $n\in\Z$. \tcbline
\begin{proof}
Let $z_0\in\Omega$, by assumption, $g(z_0)\neq 0$ so there exists $w_0\in\C$ such that $$e^{w_0}=g(z_0)e^{-F(z_0)}$$ Indeed $g(z_0)$ and $e^{-F(z_0)}$ are both non-zero functions (but $w_0$ is not uniquely defined). Then the function $z\mapsto g(z)e^{-l(z)}$ is holomorphic. Moreover, the derivative is given by 
\begin{align*}
\frac{\partial}{\partial z}\left(g(z)e^{-l(z)}\right)&=g'(z)e^{-l(z)}-g(z_0)e^{-l(z)}l'(z)\\
&=e^{-l(z)}\left(g'(z)e^{-l(z)}-g(z)F'(z)\right)\\
&=0
\end{align*}
Thus $g(z)e^{-l(z)}$ is a constant. But $g(z_0)e^{-F(z_0)}e^{-w_0}=1$. Thus $$g(z)=e^{l(z)}$$ and we are done. 
\end{proof}
\end{lmm}

This means that applying any Cauchy's theorem of your liking allows us to take the logarithm of a function. 

\begin{crl}{}{} Let $U\subseteq\C$ be star-shaped and open. Let $g:U\to\C\setminus\{0\}$ be holomorphic. Then there exists a holomorphic function $l:U\to\C$, unique up to an integer multiple of $2\pi i$, such that $$g(z)=e^{l(z)}$$ In particular, for $k\in\N$, the function $z\mapsto e^{l(z)/k}$ gives a holomorphic function on $U$ whose $k$th power of $g$. \tcbline
\begin{proof}
By Cauchy's theorem, $U$ is star-shaped and so it admits and antiderivative. By the above lemma we conclude. 
\end{proof}
\end{crl}

We conclude the section with a refined version of the main theorem of this section. 

\begin{thm}{}{} Let $\Omega\subset\C$ be open. Let $f:\Omega\to\C$ be holomorphic with $z_0$ a zero of order $1\leq n<\infty$. Then there exists $z_0\in V\subset\Omega$ and a biholomorphic function $h:V\to B_r(0)$ for some $r>0$ such that $$f(z)=\left(h(z)\right)^k$$ for all $z\in V$. This means that $f$ is locally $k$-to-one near $z_0$. \tcbline
\begin{proof}
By theorem 5.1.3, we have shown that $$f(z)=(z-z_0)^kg(z)$$ where $g$ is holomorphic and non-zero in $B_\delta(z_0)\subset\Omega$. By the above corollary, $B_\delta(z_0)$ is star-shaped so that we have a holomorphic function $l:B_\delta(z_0)\to\C$ such that $g(z)=e^{l(z)}$. Now defined $h(z)=(z-z_0)e^{\frac{1}{k}l(z)}$. Then $\left(h(z)\right)^k=f(z)$ and $h'(z_0)=e^{\frac{1}{k}l(z_0)}\neq 0$. \\~\\

The local invertibility lemma implies that there exists $V_0\in B_\delta(z_0)$ a neighbourhood of $z_0$ and $W$ a neighbourhood of $h(z_0)=0$ such that $h|_{V_0}:V_0\to W$ is biholomorphic. Replace $V_1$ by $B_r(0)\subset V_1$. Then $h$ is biholomorphic from $h^{-1}(B_r(0))$ to $B_r(0)$. \\~\\

Pick $w\in B_{r^k}(0)\setminus\{0\}$. Then there exists $\xi_1,\dots,\xi_k\in B_r(0)$ such that $\xi_j^k=w$ for all $j=1,\dots,k$. Then $\{h^{-1}(\xi_j)\}$ are mapped to $f$. 
\end{proof}
\end{thm}

Two major results arises from the study of zeroes: 

\begin{thm}{Open Mapping Theorem}{} Suppose $\Omega\subset\C$ is open and connected. Let $f:\Omega\to\C$ be holomorphic but not constant. Then the image $f(\Omega)$ is also open and connected. \tcbline
\begin{proof}
We have seen that continuous maps take connected sets to connected sets in point set topology. Let $w=f(z_0)\in f(\Omega)$. The function $g(z)=f(z)-w$ has a zero at $z_0$. This zero must be of finite order, say $k\in\N$, since other wise theorem 5.1.4 would imply that $f$ is constant. By theorem 5.1.8, we locally have $f(z)=w_0+(h(z))^k$ where $h$ is a biholomorphic map from some neighbourhood $V$ of $z_0$ to $B_r(0)$. Therefore the image of $f$ contains the ball $B_{r^k}(w)$. 
\end{proof}
\end{thm}

\begin{crl}{Maximum Modulus Principle}{} Let $\Omega\subset\C$ be open and connected. Let $f:\Omega\to\C$ be a holomorphic but not constant function. Then $\abs{f}$ does not have any local maxima. \tcbline
\begin{proof}
Suppose that $\abs{f}$ attains a local maximum at $z_0\in\Omega$. Notice that this means $f(z_0)\in\partial B_{\abs{f}}(0)$. By the open mapping theorem, the image of any neighbourhood of $z_0$ is a neighbourhood of $f(z_0)$, and so $f(\Omega)$ is a open set on the boundary of $B_{\abs{f}}(0)$. By assumption $f(\Omega)\subseteq B_{\abs{f}}(0)$ and so we reached a contradiction by definition of a boundary point. 
\end{proof}
\end{crl}

\subsection{Isolated Singularities of Holomorphic Functions}
\begin{defn}{Isolated Singularities}{} Let $f:B_r(a)\setminus\{a\}\to\C$ be a holomorphic function. Then we say that $a$ is an isolated singularity of $f$. 
\end{defn}

\begin{defn}{Classification of Isolated Singularities}{} Let $f:B_r(a)\setminus\{a\}\subseteq\C\to\C$ be a holomorphic function and $a$ an isolated singularity. We say $a$ is a
\begin{itemize}
\item Removable singularity if $\lim_{z\to a}f(z)$ exists and is finite
\item Pole if $\lim_{z\to a}f(z)=\infty$. 
\item Essential singularity otherwise
\end{itemize}
\end{defn}

\begin{lmm}{}{} Let $f:B_r(a)\setminus\{a\}\to\C$ be a holomorphic function with isolated singularity $a$. Then $a$ is a removable singularity if and only if there exists $M\in\R$ such that $$\abs{f(z)}\leq M$$ for all $z\in B_r(a)\setminus\{a\}$. 
\end{lmm}

\begin{lmm}{}{} Let $f:B_r(a)\setminus\{a\}\to\C$ be a holomorphic function with isolated singularity $a$. Then $a$ is a pole of $f$ if and only if $a$ is a zero of $\frac{1}{f(z)}$. 
\end{lmm}

\begin{defn}{Order of a Pole}{} Let $f:B_r(a)\setminus\{a\}\to\C$ be a holomorphic function with a pole at $a$. Define the order of $a$ to be the order of zero of $\frac{1}{f(z)}$ at $a$. 
\end{defn}

We will briefly study all three types of isolated singularities, the first one being the removable type. It is called removable because Riemann has a theorem which states that functions encompassing these singularities can be extended holomorphically to include the singularity. 

\begin{thm}{Riemann's Removable Singularity Theorem}{} Let $f:B_r(a)\setminus\{a\}\to\C$ be a holomorphic function. Suppose that $z_0$ is a removable singularity of $f$. Then $f$ can be extended to a holomorphic function $f:B_r(a)\to\C$. \tcbline
\begin{proof}
Define a function $g:B_r(a)\to\C$ by $g(z)=(z-a)^2f(z)$ for all $z\in B_r(a)\setminus\{a\}$ and $g(a)=0$. By the product rule, $g$ is holomorphic at $B_r(a)$. Let $z\in B_r(a)$, then we have $$\lim_{z\to a}\frac{g(z)-g(a)}{z-a}=\lim_{z\to a}(z-a)f(z)=0$$ which shows that $g$ is complex differentiable at $a$, with $g'(a)=0$. This means that $g$ is holomorphic throughout all of $B_r(a)$ with a zero of order at least $2$ at $a$. \\~\\

Case 1: $a$ is of order $\infty$. \\
Then both $f$ and $g$ must be identically zero by theorem 5.1.4. Then by setting $f(a)=0$, we are done. \\~\\

Case 2: $a$ is of finite order. \\
Suppose $a$ has order $n\geq 2$. Then we can apply theorem 5.1.3 to $g$ to obtain $$g(z)=(z-a)^nh(z)$$ for holomorphic $h:B_r(a)\to\C$ with $h(a)\neq 0$. But then $(z-a)^{n-2}h(z)$ is holomorphic on all of $B_r(a)$ and agrees with $B_r(a)\setminus\{a\}$, and so is an extension of $f$. 
\end{proof}
\end{thm}

Poles work rather similarly to zeroes of holomorphic functions. The following is an analogue to 5.1.3: 

\begin{prp}{}{} Let $f:B_r(a)\setminus\{a\}\to\C$ be holomorphic function with $a$ a pole. Then there exists $n\in\N$ and a holomorphic function $g:B_r(a)\to\C$ such that $$f(z)=\frac{g(z)}{(z-a)^n}$$ and $g$ is non-zero in a neighbourhood of $a$. Moreover, $n$ is the order of $a$. \tcbline
\begin{proof}
Let $f:B_r(z_0)\setminus\{z_0\}\to\C$ be holomorphic and has a pole of order $n$. Then $\lim_{z\to z_0}f(z)=\infty$ means that there exists $R\in(0,r)$ such that for all $z\in B_r(z_0)\setminus\{z_0\}$ we have that $\frac{1}{f(z)}$ is bounded and holomorphic on $B_r(z_0)\setminus\{z_0\}$. By Riemann's removable singularity theorem, $\frac{1}{f}$ is the restriction of some holomorphic function $F:B_r(z_0)\to\C$ with $F(z_0)=0$. \\~\\

The zero of $F$ cannot be of infinite order otherwise the identity theorem implies that $F=0$. But $F=\frac{1}{f}$ so that is is non zero everywhere in $B_r(z_0)$. Thus $F$ has a zero of some finite order $n\in\N$. By theorem 5.1.3, we can write $F(z)=(z-z_0)^nG(z)$ where $G$ is holomorphic and non zero on $B_r(z_0)$. Define $g(z)=\frac{1}{G(z)}$. Then $g$ is another holomorphic non zero function on $B_r(z_0)$ and so we are done. 
\end{proof}
\end{prp}

The last type of singularities are dealt with through the following theorem. 

\begin{thm}{Casorati-Weierstrass Theorem}{} Suppose that a holomorphic function $f:B_r(z_0)\setminus\{z_0\}\to\C$ has an essential singularity at $z_0$. Then for any $\delta\in(0,r)$, the image of the set $B_\delta(z_0)\setminus\{z_0\}$ under $f$ is dense in $\C$. \tcbline
\begin{proof}
Suppose that this is not true for every $\delta$ that the image of $f$ is dense. Then there exists $\delta\in(0,r)$ and $\epsilon>0$ and $w\in\C$ such that $\abs{f(z)-w}>\epsilon$ for all $z\in B_\delta(z_0)\setminus\{z_0\}$. Consider the holomorphic function $h(z)=\frac{1}{f(z)-w}$. By assumption, $h$ is bounded and is non zero throughout $B_\delta(z_0)\setminus\{z_0\}$. By Riemann's removable singularity theorem, $h$ can be extended to a holomorphic function on $B_\delta(z_0)$. If $h(z_0)\neq 0$, then we can write $$f(z)=\frac{1}{h(z)}+w$$ to give a holomorphic extension of $f$ to the whole of $B_\delta(z_0)$ and we see that $f$ has a removable singularity at $z_0$. If instead $h(z_0)=0$, then $h$ has a zero of finite order otherwise by the identity theorem we have $h=0$, and we can write $$h(z)=\frac{g(z)}{(z-z_0)^n}$$ for some holomorphic $g:B_\delta(z_0)\to\C$ that is non zero. Then we can write $$f(z)=(z-z_0)^{-n}\frac{1}{g(z)}+w$$ so that $f$ has a pole at $z_0$. 
\end{proof}
\end{thm}

\subsection{Line Singularities}
\begin{thm}{}{} Let $U\subseteq\C$ be open with at least one real number. Let $f:U\to\C$ be continuous and the restriction of $f$ to $U\setminus\R$ is holomorphic. Then $f$ is holomorphic throughout $U$. 
\end{thm}

\begin{thm}{Schwarz Reflection Principle}{} Let $U\subseteq\C$ be open. Assume that $U$ is symmetric with respect to the real line. Suppose that $f:U\cap H_{\text{Im}(z)>0}\to\C$ is a continuous function such that 
\begin{itemize}
\item $f$ is holomorphic on $U\cap H_{\text{Im}(z)>0}$
\item $f$ only attains real values on $U\cap\R$. 
\end{itemize}
If we extend $f$ to a function on the whole of $U$ be setting $f(z)=\overline{f(z)}$ for all $z\in U\cap H_{\text{Im}(z)<0}$, then $f$ is holomorphic on all of $U$. 
\end{thm}

\subsection{Laurent Series}
Laurent series plays a major role in the study of zeroes and singularities. 

\begin{defn}{Laurent Series}{} A laurent series is a series of the form $$\sum_{k=-\infty}^\infty a_kz^k$$ for $a_k\in\C$. 
\end{defn}

\begin{defn}{Convergence of Laurent Series}{} A double-ended series $\sum_{k=-\infty}^\infty a_k$ is said to converge to $L\in\C$ if $$\sum_{k=0}^\infty a_k=L_+\;\;\;\;\text{ and }\;\;\;\;\sum_{k=1}^\infty a_{-k}=L_-$$ such that $L=L_++L_-$ for $L_+,L_-\in\C$. 
\end{defn}

Before we arrive at the main theorem for Laurent series, we need a new version of Cauchy's integral formula. Specifically, for annuli. 

\begin{thm}{Cauchy's Integral Formula for Annuli}{} Let $U\subseteq\C$ be open. Let $f:U\to\C$ be holomorphic. Suppose that for some $a\in\C$ and radii $0<R_1<R_2<\infty$, the closure of the annulus $$A=\{z\in\C\;|\; R_1<\abs{z-a}<R_2\}$$ is contained in $U$. Then for any $w\in A$, we have that $$f(w)=\frac{1}{2\pi i}\int_{\partial A}\frac{f(z)}{z-w}\,dz$$ \tcbline
\begin{proof}
Fix $w\in A$. Define a function $g:U\setminus\{w\}\to\C$ by $$g(z)=\frac{f(z)-f(w)}{z-w}$$ By Riemann's removable singularity theorem, $g$ can be holomorphically extended to $U$. Using this function for Cauchy's theorem on annuli, we have that $$\int_{\partial A}\frac{f(z)-f(w)}{z-w}=0$$ and so we have that 
\begin{align*}
\int_{\partial A}\frac{f(z)}{z-w}\,dz&=\int_{\partial A}\frac{f(w)}{z-w}\,dz\\
&=f(w)\left(\int_{\partial B_{R_2}(a)}\frac{1}{z-w}\,dz-\int_{\partial B_{R_1}(a)}\frac{1}{z-w}\,dz\right)\\
&=2\pi if(w)\left(I(\partial B_{R_2}(a),w)-I(\partial B_{R_1}(a),w)\right)\\
&=2\pi if(w)(1-0)\\
&=w\pi i f(w)
\end{align*}
Notice that $I(\partial B_{R_2}(a),w)=1$ because $I(\partial B_{R_2}(a),w)=I(\partial B_{R_2}(a),a)$ by lemma 4.1.7. Then by a direct computation via the taking the lift to be $\theta(t)=t$, we obtain $I(\partial B_{R_2}(a),a)\frac{2\pi-0}{2\pi}$. By lemma 4.1.4, we obtain that $I(\partial B_{R_1}(a),w)=0$
\end{proof}
\end{thm}

This prove relies Riemann's removable singularity theorem, which is why we delayed the theorem until here. 

\begin{thm}{Laurent's Theorem}{} Suppose $0\leq r_1<r_2$, $a\in\C$ and $f$ is holomorphic on $A=\{z\in\C|r_1<\abs{z-a}<r_2\}$ Then for every $z\in A$, we have $$f(z)=\sum_{k=-\infty}^\infty a_k(z-a)^k$$ where $$a_k=\frac{1}{2\pi i}\int_{\partial B_s(a)}\frac{f(w)}{(w-a)^{k+1}}\,dw$$ for all $s\in(r_1,r_2)$. \tcbline
\begin{proof}
Without loss of generality, assume that $a=0$. Fix $z\in A$. Choose numbers, $R_1,R_2$ such that $r_1<R_1<\abs{z}<R_2<r_2$. By Cauchy's theorem for annuli, the integral of a holomorphic function on $A$ around $\partial B_s(0)$ does not depend on $s\in(r_1,r_2)$. In particular, the formula for $a_k$ is independent of $s$. Cauchy's integral formula for annuli implies that $$
f(z)=\frac{1}{2\pi i}\int_{\partial B_{R_2}(0)}\frac{f(w)}{w-z}\,dw-\frac{1}{2\pi i}\int_{\partial B_{R_1}(0)}\frac{f(w)}{w-z}\,dw$$
The first term of the right hand side can be handled as in the proof of Taylor's theorem: 
\begin{align*}
\frac{1}{2\pi i}\int_{\partial B_{R_2}(0)}\frac{f(w)}{w-z}\,dz&=\sum_{k=0}^\infty\left(\frac{1}{2\pi i}\int_{\partial B_{R_2}(0)}\frac{f(w)}{w^{k+1}}\,dw\right)z^k\\
&=\sum_{k=0}^\infty a_kz^k
\end{align*}
Now notice that for any $w\in\partial B_{R_1}(0)$, we have that $$-\frac{1}{w-z}=\frac{1}{z}\frac{1}{1-\frac{w}{z}}=\frac{1}{z}\sum_{k=0}^\infty\left(\frac{w}{z}\right)^k$$ where the power series converges because $\abs{z}>R_1=\abs{w}$ and so that $\abs{\frac{w}{z}}<1$. Hence we have that 
\begin{align*}
-\frac{1}{2\pi i}\int_{\partial B_{R_1}(0)}\frac{f(w)}{w-z}\,dw&=\frac{1}{2\pi i}\int_{\partial B_{R_1}(0)}\sum_{k=0}^\infty\frac{f(w)}{z^{k+1}}w^k\,dw\\
&=\sum_{k=-\infty}^{-1}\left(\frac{1}{2\pi i}\int_{\partial B_{R_1}(0)}\frac{f(w)}{w^{k+1}}\right)z^k\tag{Uniform Convergence}\\
&=\sum_{k=-\infty}^{-1}a_kz^k
\end{align*}
and so we conclude. 
\end{proof}
\end{thm}

Just as the coefficients of a Taylor series is unique, the coefficients of Laurent series are uniquely determined by the function. 

With our new machinery, we can now classify all injective entire functions. 

\begin{thm}{Injective Entire Functions}{} Let $f:\C\to\C$ be injective and entire. Then $$f(z)=\alpha z+\beta$$ for some $\alpha\in\C\setminus\{0\}$ and $\beta\in\C$. \tcbline
\begin{proof}
Define $g(z)=f\left(\frac{1}{z}\right)$. This function is clearly holomorphic on $\C\setminus\{0\}$ and injective since both $f$ and $z\mapsto\frac{1}{z}$ are injective. We will show that $g$ can only have poles and not other singularities. \\~\\

Suppose that $g$ has a removable singularity. Then $g$ would be bounded in say $\overline{D}\setminus\{0\}$ by lemma 5.2.3. This implies that $f$ would also be bounded in $\C\setminus D$. But $f$ is continuous and thus bounded on $\overline{D}$ so that $f$ would be bounded on the whole of $\C$. By Liouville's theorem, $f$ is constant, which is impossible for an injective function. Now suppose instead that $g$ has an essential singularity. Then the Casorati-Weierstrass theorem implies that the image $g(\overline{D}\setminus\{0\})$ would be dense. This implies that the image $f(\C\setminus\overline{D})$ is dense in $\C$. But $f(D)$ is an open set by the open mapping theorem, and there must be some intersection of $f(\C\setminus\overline{D})$ and $f(D)$. This is a contradiction since $f$ is injective. \\~\\

Thus $0$ is a pole of $g$. Suppose that $f$ has Taylor expansion $$f(z)=\sum_{k=0}^\infty a_kz^k$$ Then the unique Laurent expansion for $g$ on $\C\setminus\{0\}$ must be $$g(z)=\frac{1}{f(z)}=\sum_{k=-\infty}^0 a_{-k}z^k$$ Since $g$ has a pole, say of order $n$, we must have that $a_k=0$ for $k>n$. This forces $f$ to be a polynomial. By the fundamental theorem of algebra, the only injective polynomials are of the form claimed in the theorem. Indeed, by injectivity, only one point, say $z=a$ can be mapped to zero so the factorization of the polynomial must be of the form $f(z)=\alpha(z-a)^n$. But this polynomial is only injective for $n=1$, and so we conclude. 
\end{proof}
\end{thm}

\subsection{Residue Theory}
In conjunction with the previous Cauchy theorems, residue theorem is a slight generalization of the homology version of Cauchy's theorem. Residues are greatly related to singularities. 

\begin{defn}{Residue}{} Let $f:B_\delta(z_0)\setminus\{z_0\}\to\C$ is holomorphic for some $\delta>0$, $z_0\in\C$. Define the residue of $f$ at $z_0$ to be $$\res(f,z_0)=\frac{1}{2\pi i}\int_{\partial B_{\epsilon}(z_0)}f(z)\,dz$$ for any $\epsilon\in(0,\delta)$. It is equivalent to $b_1$ at the laurent series of $f$ at $z_0$. 
\end{defn}

For Residue theorem, we need an even more powerful version of Cauchy's integral formula, through homology. 

\begin{thm}{Cauchy's Integral Formula - General Version}{} Let $U\subset\C$ be open and let $\gamma$ be a cycle in $U$ that is homologous to $0$. Then for any holomorphic function $f:U\to\C$ and for any $w\in\Omega$ not lying in the image of $\gamma$, we have $$f(w)I(\gamma,w)=\frac{1}{2\pi i}\int_\gamma\frac{f(z)}{z-w}\,dz$$ \tcbline
\begin{proof}
By Riemann's removable singularity theorem, the function $$g(z)=\frac{f(z)-f(w)}{z-w}$$ which is holomorphic initially on $\Omega\setminus\{w\}$, can be extended to a holomorphic function on $\Omega$. By Homology version of Cauchy's theorem, we have that $$\int_\gamma g(z)\,dz=0$$ which implies that $$\frac{1}{2\pi i}\int_\gamma\frac{f(z)}{z-w}\,dz=\frac{1}{2\pi i}\int_\gamma\frac{f(w)}{z-w}\,dz=f(w)I(\gamma,w)$$ and so we are done. 
\end{proof}
\end{thm}

Once again, this version of Cauchy's integral formula relies on Riemann's removable singularity theorem relies on Riemann's removable singularity theorem. 

\begin{thm}{Residue Theorem}{} Let $U\subseteq\C$ be open. Let $f:U\setminus S\to\C$ be holomorphic where $S$ is the set of isolated singularities. Let $\gamma$ be a cycle in $U\setminus S$ that is homologous to $0$. Then we have $$\int_\gamma f(z)\,dz=2\pi i\sum_{a\in S}I(\gamma,a)\text{res}(f,a)$$ In particular, the sum is finite in the sense that there are finitely many $a\in S$ for which $I(\gamma,a)\neq 0$. \tcbline
\begin{proof} 
We begin by first showing that the sum is finite. Suppose that this is not the case. Then we can pick a sequence $(a_n)_{n\in\N}$ in $$A=\{a\in S|I(\gamma,a)\neq 0\}$$ with pairwise distinct elements. By proposition 4.1.8, $A$ is bounded an so we can pass to a subsequence $b_k=a_{n_k}$ so that $b_k\to b_\infty\in\overline{U}$. I claim that $b_\infty$ cannot lie in $U$. If it does, then since $S$ is closed, $b_\infty\in S$. We now have a sequence in $S$ that converges to an element in $S$. This is a contradiction since the set is assumed to be isolated. Therefore we must have $b_\infty\in\partial U$. In particular, $b_\infty\in\C\setminus U$ implies that $I(\gamma, b_\infty)=0$ since $\gamma$ is homologous to $0$. Since the winding number is constant on connected components, there is a neighbourhood of $b_\infty$ for which its points have winding number $0$. In particular, $I(\gamma, b_n)=0$ for all $n$ sufficiently large, which is a contradiction. \\~\\

We can now write $A=\{a_1,\dots,a_n\}$. Choose $\varepsilon>0$ small so that $B_{2\varepsilon}(a_k)\setminus\{a_k\}\subset U\setminus S$ for every $1\leq k\leq n$. Write $\gamma_k:[0,1]\to U\setminus S$ for the curve $\gamma_k(t)=a_k+\varepsilon e^{i2\pi t}$. Notice that $I(\gamma_k,a_k)=1$, while $I(\gamma_k,a)=0$ for all $a\in S\setminus\{a_k\}$. \\~\\

Consider the cycle $$\Gamma=\gamma-\sum_{k=1}^nI(\gamma,a_k)\gamma_k$$ By construction, the cycle $\Gamma$ does not wind around any point in $S$, meaning that $I(\Gamma,a)=0$ for all $a\in S$. Moreover, $I(\Gamma,a)=0$ for all $a\in\C\setminus U$. Hence by the general Cauchy's theorem applied on $U\setminus S$, we have that $\int_\Gamma f(z)\,dz=0$. This means that $$\int_\gamma f(z)\,dz=2\pi i\sum_{k=1}^nI(\gamma,a_k)\text{res}(f,a_k)$$ and so we conclude. 
\end{proof}
\end{thm}

Notice that when $S=\emptyset$, one recovers the homology version of Cauchy's theorem. 

\begin{prp}{}{} Let $U\subseteq\C$ be open. Let $f:U\setminus\{z_0\}\to\C$ be holomorphic with $z_0$ an isolated singularity. We have the following formulas for the residue of $z_0$ regarding the type of singularity of $z_0$. 
\begin{itemize}
\item If $z_0$ is a removable singularity, then $\text{res}(f,z_0)=0$
\item If $z_0$ is a simple pole, then $\res(f,z_0)=\lim_{z\to z_0}(z-z_0)f(z)$
\item If $f(z)=\frac{g(z)}{h(z)}$ for $z_0$ a zero of order $1$ of $h$, then $\res(f,z_0)=\frac{g(z_0)}{h'(z_0)}$
\item If $z_0$ is a pole of order $m$, then $$\res(f,z_0)=\lim_{z\to z_0}\left(\frac{1}{(m-1)!}\frac{d^{m-1}}{dz^{m-1}}(z-z_0)^mf(z)\right)$$
\item If $z_0$ is an essential singularity, then $\text{res}(f,z_0)=a_{-1}$, where $a_k$ is the Laurent coefficient of $f$. 
\end{itemize}
\end{prp}

\subsection{Meromorphic Functions \& The Argument Principle}
In this section we develop some theory for meromorphic functions. In particular, the argument principles gives us an easy method of computing the integral of a logarithmic derivative. 

\begin{defn}{Meromorphic Functions}{} Suppose $\Omega\subseteq\C$ is open and connected. Let $f:\Omega\to\C_{\infty}$ that is not identically equal to $\infty$. We say that $f$ is meromorphic if $f$ is complex differentiable at every point $z_0\in\Omega$ with $f(z_0)\neq\infty$ and $\frac{1}{f}$ is complex differentiable at every point $z_0\in\Omega$ with $f(z_0)=\infty$. 
\end{defn}

\begin{lmm}{}{} Let $\Omega\subseteq\C$ be open. A function $f:\Omega\to\C$ is meromorphic if and only if $f$ is a ratio $$f(z)=\frac{g(z)}{h(z)}$$ of two functions $g$ and $h$ that are holomorphic except for a set of isolated points which are all poles of $f$. 
\end{lmm}

\begin{defn}{Number of Zeroes and Poles}{} Let $U$ be open. Let $f:U\to\C$ be holomorphic. For every $A\subseteq U$, define $$Z_A(f)=\sum_{\substack{z\in A\\ z\text{ is a zero of }f}}\text{ord}(f,z)$$ to be the number of zeroes of $f$ in $A$, and $$P_A(f)=\sum_{\substack{z\in A\\ z\text{ is a pole of }f}}-\text{ord}(f,z)$$ to be the number of poles of $f$ in $A$. 
\end{defn}

\begin{defn}{Curves Bounding a Set}{} Let $\gamma:[a,b]\to\C$ be a closed continuous path. We say that $\gamma$ bounds an open set $A\subset\C$ in a positive direction if $\C\setminus\gamma([a,b])$ has two connected components, one of which is $A$, and $I(\gamma,z)=1$ for all $z\in A$. 
\end{defn}

\begin{thm}{The Argument Principle}{} Let $\Omega\subseteq\C$ be a open and connected and let $f:\Omega\to\C\cup\{\infty\}$ be a meromorphic function that it not identically $0$. Denote $S\subset\Omega$ the set of all poles and zeroes of $f$. Let $\gamma:[a,b]\to\Omega\setminus S$ be a piecewise $C^1$ simple closed curve that bounds an open set $A\subset\Omega$ positively. Then $$\frac{1}{2\pi i}\int_\gamma\frac{f'(z)}{f(z)}dz=Z_A(f)-P_A(f)$$ Moreover, we have the formula $$Z_A(f)-P_A(f)=I(f\circ\gamma,0)$$ \tcbline
\begin{proof}
Consider the logarithmic derivative $\frac{f'(z)}{f(z)}$ near a pole or a zero of $f$ at $z_0$. In either case, we can write $$f(z)=(z-z_0)^ng(z)$$ for $z$ in a neighbourhood of $z_0$, $g$ is holomorphic with $g(z_0)\neq 0$, and $n\in\Z\setminus\{0\}$ is equal to $\text{ord}(f,z_0)$. For $z$ sufficiently close to $z_0$ such that $g(z)\neq 0$, the logarithmic derivative implies that 
\begin{align*}
\frac{f'(z)}{f(z)}&=\frac{1}{f(z)}\left(n(z-z_0)^{n-1}g(z)+(z-z_0)^ng'(z)\right)\\
&=\frac{n}{z-z_0}+\frac{g'(z)}{g(z)}
\end{align*}
As $g(z_0)\neq 0$, we can conclude that $\frac{f'}{f}$ has a simple pole with residue $n$ at $z_0$. Thus $$\text{res}\left(\frac{f'(z)}{f(z)},z_0\right)=\text{ord}(f,z_0)$$ Our result follows immediately by the Residue theorem. 
\end{proof}
\end{thm}

An application of the argument principle is Rouche's theorem, which is just as powerful in its own right. 

\begin{thm}{Rouche's Theorem}{} Let $f,g:\Omega\to\C$ be two holomorphic functions and $\Omega$ a open and connected set. Let $\gamma:[a,b]\to\Omega$ be a piecewise $C^1$ simple closed curve that bounds an open set $A\subset\Omega$ in a positive direction. Suppose that $\abs{f(z)}<\abs{g(z)}$ for all $z\in\gamma([a,b])$. Then $$Z_A(g)=Z_A(f+g)$$ has the same number of zeroes in $A$. \tcbline
\begin{proof}
Because of the hypothesis, we can apply the dog walking lemma with $f\circ\gamma$ and $(f+g)\circ\gamma$. This implies that $$I(f\circ\gamma,0)=I\left((f+g)\circ\gamma,0\right)$$ By the argument principle, the left hand side $Z_A(f)$ and the right hand side is $Z_A(f+g)$ and so we are done. 
\end{proof}
\end{thm}

\pagebreak
\section{Möbius Transformations \& Biholomorphism}
\subsection{Möbius Transformations}
\begin{defn}{Möbius Transformations}{} We say that a map $f:\C_\infty\to\C_\infty$ is a Möbius Transformations if $f$ is a function of the form $$f(z)=\frac{az+b}{cz+d}$$ where $a,b,c,d\in\C$ such that $ad-bc\neq 0$. 
\end{defn}

\begin{prp}{}{} Möbius Transformations are continuous and bijective. In particular, the inverse of a Möbius Transformation is also a Möbius Transformation given by the formula $$f^{-1}(z)=\frac{dz-b}{-cz+a}$$ \tcbline
\begin{proof}
It is easy to compute that $f\circ f^{-1}=z$ and $f^{-1}\circ f=z$. 
\end{proof}
\end{prp}

\begin{prp}{}{} The composition of two Möbius Transformations is also a Möbius Transformation so that the set of all Möbius Transformations form a group under functional composition. Moreover, the group is isomorphic to $$\text{PSL}(2,\C)\cong\frac{\text{SL}(2,\C)}{\{\pm I\}}$$
\end{prp}

\begin{defn}{Elementary Möbius Transformations}{} We say that a Möbius Transformation $f(z)=\frac{az+b}{cz+d}$ is elementary if it is of one of the following forms. 
\begin{itemize}
\item Translations: $f(z)=z+b$ for some $b\in\C$
\item Rotations: $f(z)=e^{i\theta}z$ for some $\theta\in\R$
\item Dilations: $f(z)=\lambda z$ for some $\lambda>0$
\item Complex Inversion: $f(z)=\frac{1}{z}$
\end{itemize}
\end{defn}

\begin{lmm}{}{} Every Möbius Transformation can be written as the composition of elementary Möbius Transformations. \tcbline
\begin{proof}
We consider two cases. For $c=0$ the Möbius Transformation is exactly $f(z)=\frac{az+b}{d}$. This is given by $f_1(z)=az$, $f_2(z)=z+\frac{b}{a}$, $f_3(z)=\frac{z}{d}$ and then composing them in order. \\~\\
For $c\neq 0$, write $$\frac{az+b}{cz+d}=\frac{a}{c}+\frac{b-\frac{ad}{c}}{cz+d}$$ Then the Möbius Transformation is given by $f_1(z)=cz$, $f_2(z)=z+\frac{d}{c}$, $f_3(z)=\frac{1}{z}$, $f_4(z)=\frac{1}{cz+d}b-\frac{ad}{c}$, $f_5(z)=z+\frac{a}{c}$ and then composing them in order. 
\end{proof}
\end{lmm}

\begin{lmm}{}{} Every Mobius transformation is a biholomorphic function from $C_\infty$ to $C_\infty$. 
\end{lmm}

\subsection{The Fixed Points of a Möbius Transformation}
\begin{lmm}{}{} Every Möbius Transformation $f:\C_\infty\to\C_\infty$ other than the identity has at least one but at most two fixed points. \tcbline
\begin{proof}
Assume that $f(z)=\frac{az+b}{cz+d}$ is not the identity. This means that $a\neq d$ and $b\neq 0$ or $c\neq 0$. There are two cases to consider. \\~\\

Case 1: $c=0$. \\
Then $f$ is just a linear transformation $$f(z)=\frac{a}{d}z+\frac{b}{d}$$ where $a,d\neq 0$ by non degeneracy. Then it is clear that $\infty$ and $\frac{b}{d-a}$ are the only fixed points since $a\neq d$. \\~\\

Case 2: $c\neq 0$. \\
Then we have that 
\begin{align*}
f(z)=z&\iff (az+b)=(cz+d)z\\
&\iff 0=cz^2+(d-a)z-b
\end{align*}
The quadratic formula yields two solutions, which might coincide. 
\end{proof}
\end{lmm}

This lemma implies that if there is at least three fixed points for a Möbius Transformation, then the Möbius Transformation is the identity. 

\begin{prp}{}{} Let $z_1,z_2,z_3\in\C_\infty$. Then there exists a Möbius Transformation $f$ that maps $z_1,z_2,z_3$ to $1,0,\infty$ respectively. Moreover, 
\begin{itemize}
\item If $z_1,z_2,z_3\neq\infty$ then $f(z)=\frac{(z-z_2)(z_1-z_3)}{(z-z_3)(z_1-z_2)}$
\item If $z_1=\infty$ then $f(z)=\frac{(z-z_2)}{(z-z_3)}$
\item If $z_2=\infty$ then $f(z)=\frac{(z_1-z_3)}{(z-z_3)}$
\item If $z_3=\infty$ then $f(z)=\frac{(z-z_2)}{(z_1-z_2)}$
\end{itemize} \tcbline
\begin{proof}
It is clear that all of the above are Möbius transformations. It is also easy to see that they  send $z_i$ to their required image. 
\end{proof}
\end{prp}

\begin{thm}{}{} Let $z_1,z_2,z_3\in\C_\infty$ be distinct. Also let $w_1,w_2,w_3\in\C_\infty$ be distinct, then there exists a unique Möbius Transformation $f$ such that $f(z_1)=w_1$, $f(z_2)=w_2$ and $f(z_3)=w_3$. \tcbline
\begin{proof}
By the above proposition, we can find maps $f$ and $g$ that send $z_1,z_2,z_3$ to $1,0,\infty$ and $w_1,w_2,w_3$ to $1,0,\infty$ respectively. Then $g^{-1}\circ f$ satisfies the requirements of the theorem. \\~\\

For uniqueness, suppose that $h$ and $k$ both send $z_1,z_2,z_3$ to $w_1,w_2,w_3$ correspondingly. Then $k^{-1}\circ h$ is a Möbius transformation that has fixed points $z_1,z_2,z_3$. Thus it must be the identity and $h=k$. 
\end{proof}
\end{thm}

The cross ratio is useful in explicitly determining the unique map sending $z_i$ to $w_i$. 

\begin{defn}{Cross Ratio}{} Let $z_0,z_1,z_2,z_3\in\C_\infty$. The cross ratio of the four points is the image of $z_0$ under the unique Möbius transformation that sends $z_1,z_2,z_3$ to $1,0,\infty$ respectively. It is denoted as $$(z_0,z_1,z_2,z_3)$$
\end{defn}

\begin{thm}{}{} The cross ratio is invariant under Möbius transformations. 
\end{thm}

\begin{thm}{}{} The cross ratio is real valued if and only if $z_0,z_1,z_2,z_3$ all lie on a common circle in $\C_\infty$. 
\end{thm}

This is particularly useful for determining the existence of a Möbius transformation a function specifying the image of four points. This is because invariance of the cross ratio implies that $$(z,z_1,z_2,z_3)=(h(z),w_1,w_2,w_3)$$ given a Möbius transformation $h$. Then using the formula for the Möbius transformation of the two cross ratios, we have that $$\frac{(z_0-z_2)(z_1-z_3)}{(z_0-z_3)(z_1-z_2)}\;\;\;\;\text{ and }\;\;\;\;\frac{w_0-w_2)(w_1-w_3)}{(w_0-w_3)(w_1-w_2)}$$ It is then easy to see that if they are not the same, then since cross ratio is an invariant of the Möbius transformation there cannot be a Möbius transformation sending $z_0,z_1,z_2,z_3$ to $w_0,w_1,w_2,w_3$. \\~\\

Alternatively, given three points $z_1,z_2,z_3$ on the domain and three points $w_1,w_2,w_3$ on the codomain, we can use the same formula to compute the unique Möbius transformation $f$ sending $z_i$ to $w_i$. This is done by $$\frac{(z-z_2)(z_1-z_3)}{(z-z_3)(z_1-z_2)}=\frac{(f(z)-w_2)(w_1-w_3)}{(f(z)-w_3)(w_1-w_2)}$$ and then expanding this on out to get a formula for $f$. 

We end the section with some examples. 

\begin{thm}{}{} Let $w\in\C$ with $\abs{w}<1$. Then the Möbius transformation $$f(z)=\frac{z-w}{\overline{w}z-1}$$ is a bijective map from $D$ to $D$ that sends $\partial D$ to $\partial D$. \tcbline
\begin{proof}
Firstly, notice that we have the identity $$\abs{z-w}^2=\abs{\overline{w}z-1}^2-(1-\abs{z}^2)(1-\abs{w}^2)$$ Now we can compute $$\abs{f(z)}^2=\frac{\abs{z-w}^2}{\abs{\overline{w}z-1}^2}=1-\frac{(1-\abs{z}^2)(1-\abs{w}^2)}{\abs{\overline{w}z-1}^2}$$ Now we are given that $1-\abs{w}^2>0$. This means that $\abs{f(z)}<1$ if and only if $\abs{z}<1$ and $\abs{f(z)}=1$ if and only if $\abs{z}=1$. \\~\\

For bijectivity, notice that $f\circ f=\text{id}$ which means that $f$ is its own inverse. 
\end{proof}
\end{thm}

\begin{prp}{Cayley Transformation}{} The upper half plane $$H=\{z=x+iy\in\C\;|\;y>0\}$$ is conformally equivalent to the unit disc by the biholomorphic function $f:\C\to\C$ defined by $$f(z)=\frac{z-i}{z+i}$$ \tcbline
\begin{proof}
We have that 
\begin{align*}
f(z)\in D&\iff \abs{f(z)}<1\\
&\iff \abs{z-i}<\abs{z+i}\\
&\iff\text{Im}(z)>0
\end{align*}
and so we are done. 
\end{proof}
\end{prp}

\begin{prp}{}{} The upper right quarter of the complex plane $$Q=\{z=x+iy\in\C\;|\;x>0,y>0\}$$ is conformally equivalent to the unit disc $D$. \tcbline
\begin{proof}
Notice that $Q$ is conformally equivalent to the upper half plane via the biholomorphic function $z\mapsto z^2$. Then apply the above proposition. 
\end{proof}
\end{prp}

\begin{prp}{}{} The upper half disc $$D^+=\{z=x+iy\in\C\;|\;\abs{z}<1\text{ and }y>0\}$$ is conformally equivalent to the whole disc $D$. \tcbline
\begin{proof}
The Mobius transformation sending $-1,0,1$ to $0,1,\infty$ is a biholomorphic function which gives conformal equivalence. 
\end{proof}
\end{prp}

\subsection{Biholomorphic Functions as Mobius Transformations}
The following theorem shows that relatively weak hypothesis are sufficient for biholomorphism. 

\begin{thm}{}{} Suppose $\Omega\subset\C$ is open and connected and $f:\Omega\to\C$ is both injective and holomorphic. Then $f(\Omega)$ is a open and connected set and $f$ is a biholomorphic map onto its image. \tcbline
\begin{proof}
By the open mapping theorem, $f(\Omega)$ is open and connected. Suppose for a contradiction that we could find some $z_0\in\Omega$ such that $f'(z_0)=0$. Then the function $F(z)=f(z)-f(z_0)$ would satisfy $F(z_0)=F'(z_0)=0$. This means that $F$ would have a zero of order $k\geq 2$ at $z_0$. Theorem 5.1.8 would imply that the function $F$ could not be injective and thus $f$ cannot be injective. \\~\\

Now we have that $f'(z)\neq 0$ for all $z\in\Omega$. The local invertibility lemma implies that $f$ is locally biholomorphic. Because $f$ is bijective, it must then extend to a global biholomorphism. 
\end{proof}
\end{thm}

\begin{thm}{Schwarz Lemma}{} Let $f:D\to D$ be a holomorphic function on $D$ with $f(0)=0$. Then the following are true. 
\begin{itemize}
\item $\abs{f'(0)}\leq 1$
\item $\abs{f(z)}\leq\abs{z}$
\end{itemize}
If the first equality holds, or the second equality holds for some $z\in D$, then $f$ is a rotation in the sense that $f(z)=e^{i\theta}z$ for some $\theta\in\R$. \tcbline
\begin{proof}
The zero of $f$ can be assume to be of finite order. Otherwise by theorem 5.1.4, $f$ is identically $0$ in which case both inequalities hold. By theorem 5.1.3, there exists a holomorphic function $g:D\to\C$ such that $f(z)=zg(z)$ for all $z\in D$. \\~\\

Suppose $r\in(0,1)$. By assumption, for all $z$ with $\abs{z}=r$, we have $$1>\abs{f(z)}=\abs{z}\abs{g(z)}=r\abs{g(z)}$$ and hence $\abs{g(z)}<\frac{1}{r}$. By the maximum modulo principle, $\abs{g(z)}$ must attain its maximum over the boundary $\partial\overline{B_r(0)}$. This means that $\abs{g(z)}<\frac{1}{r}$ for all $\abs{z}\leq r$. By taking the limit as $r$ tends to $1$, we have that $\abs{g(z)}\leq 1$ throughout $D$. Now since $f'(0)=g'(0)$, we have that $\abs{f'(0)}\leq 1$. Since $\abs{f(z)}=\abs{z}\abs{g(z)}$, we have that $\abs{f(z)}\leq\abs{z}$. \\~\\

Now notice that $\abs{f'(0)}=1$ if and only if $\abs{g(0)}=1$ and $\abs{f(z)}=\abs{z}$ if and only if $\abs{g(z)}=1$. Either way, we need to consider the case $\abs{g(z_0)}=1$ for some $z_0\in D$. In this case, $\abs{g}$ attains a local maximum at $z_0$. By the maximum modulus principle, we have that $g$ must be constant and of magnitude $1$. This means that we can write $g(z)=e^{i\theta}$ for some fixed $\theta\in\R$, and so $f(z)=e^{i\theta}z$. 
\end{proof}
\end{thm}

\begin{thm}{}{} Every biholomorphic function $f:D\to D$ is a Mobius transformation of the form $$f(z)=e^{i\theta}\left(\frac{z-a}{\overline{a}z-1}\right)$$ for $\abs{a}<1$ and $\theta\in[-\pi,\pi)$. \tcbline
\begin{proof}
Suppose first that $f(0)=0$. We can then apply the Schwarz lemma to deduce that $\abs{f(z)}\leq\abs{z}$ for all $z\in D$. But we can also apply it to $f^{-1}$ to obtain $\abs{f^{-1}(f(z))}\leq\abs{f(z)}$. This implies that $\abs{z}=\abs{f(z)}$ and so Schwarz's lemma implies that $f$ is rotation. \\~\\

Now for the general case, set $a=f^{-1}(0)$ and define $$\varphi(z)=\frac{z-a}{\overline{a}z-1}$$ It is clear that $\varphi$ is a biholomorphic map from $D$ to $D$. that maps $0$ to $a$ by theorem 6.2.7. This means that $f\circ\varphi$ is a biholomorphic map from $D$ to itself that maps $0$ to $0$. Thus $f\circ\varphi$ is a rotation $z\mapsto e^{i\theta}z$ by the first part of the proof. By theorem 6.2.7, $\varphi$ is its own inverse. Therefore $$f(z)=(f\circ\varphi\circ\varphi)(z)=e^{i\theta}\frac{z-a}{\overline{a}z-1}$$ and so we are done. 
\end{proof}
\end{thm}

































\end{document}