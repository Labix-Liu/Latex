\documentclass[a4paper]{article}

%=========================================
% Packages
%=========================================
\usepackage{mathtools}
\usepackage{amsfonts}
\usepackage{amsmath}
\usepackage{amssymb}
\usepackage{amsthm}
\usepackage[a4paper, total={6in, 8in}, margin=1in]{geometry}
\usepackage[utf8]{inputenc}
\usepackage{fancyhdr}
\usepackage[utf8]{inputenc}
\usepackage{graphicx}
\usepackage{physics}
\usepackage[listings]{tcolorbox}
\usepackage{hyperref}
\usepackage{tikz-cd}
\usepackage{adjustbox}
\usepackage{enumitem}


\hypersetup{
    colorlinks=true, %set true if you want colored links
    linktoc=all,     %set to all if you want both sections and subsections linked
    linkcolor=black,  %choose some color if you want links to stand out
}
\usetikzlibrary{arrows.meta}

\DeclarePairedDelimiter\ceil{\lceil}{\rceil}
\DeclarePairedDelimiter\floor{\lfloor}{\rfloor}

%=========================================
% Custom Math Operators
%=========================================
\DeclareMathOperator{\adj}{adj}
\DeclareMathOperator{\im}{im}
\DeclareMathOperator{\nullity}{nullity}
\DeclareMathOperator{\sign}{sign}
\DeclareMathOperator{\dom}{dom}
\DeclareMathOperator{\lcm}{lcm}
\DeclareMathOperator{\ran}{ran}
\DeclareMathOperator{\ext}{Ext}
\DeclareMathOperator{\dist}{dist}
\DeclareMathOperator{\diam}{diam}
\DeclareMathOperator{\aut}{Aut}
\DeclareMathOperator{\inn}{Inn}
\DeclareMathOperator{\syl}{Syl}
\DeclareMathOperator{\edo}{End}
\DeclareMathOperator{\cov}{Cov}
\DeclareMathOperator{\vari}{Var}
\DeclareMathOperator{\cha}{char}
\DeclareMathOperator{\Span}{span}
\DeclareMathOperator{\ord}{ord}
\DeclareMathOperator{\res}{res}
\DeclareMathOperator{\Hom}{Hom}
\DeclareMathOperator{\Mor}{Mor}
\DeclareMathOperator{\coker}{coker}
\DeclareMathOperator{\Obj}{Obj}
\DeclareMathOperator{\id}{id}
\DeclareMathOperator{\GL}{GL}
\DeclareMathOperator*{\colim}{colim}

%=========================================
% Custom Commands (Shortcuts)
%=========================================
\newcommand{\CP}{\mathbb{CP}}
\newcommand{\GG}{\mathbb{G}}
\newcommand{\F}{\mathbb{F}}
\newcommand{\N}{\mathbb{N}}
\newcommand{\Q}{\mathbb{Q}}
\newcommand{\R}{\mathbb{R}}
\newcommand{\C}{\mathbb{C}}
\newcommand{\E}{\mathbb{E}}
\newcommand{\Prj}{\mathbb{P}}
\newcommand{\RP}{\mathbb{RP}}
\newcommand{\T}{\mathbb{T}}
\newcommand{\Z}{\mathbb{Z}}
\newcommand{\A}{\mathbb{A}}
\renewcommand{\H}{\mathbb{H}}

\newcommand{\mA}{\mathcal{A}}
\newcommand{\mB}{\mathcal{B}}
\newcommand{\mC}{\mathcal{C}}
\newcommand{\mD}{\mathcal{D}}
\newcommand{\mE}{\mathcal{E}}
\newcommand{\mF}{\mathcal{F}}
\newcommand{\mG}{\mathcal{G}}
\newcommand{\mH}{\mathcal{H}}
\newcommand{\mJ}{\mathcal{J}}
\newcommand{\mO}{\mathcal{O}}
\newcommand{\mS}{\mathcal{S}}

%=========================================
% Theorem Environment
%=========================================
\newcommand\todoin[2][]{\todo[backgroundcolor=white!20!white, inline, caption={2do}, #1]{
\begin{minipage}{\textwidth-4pt}#2\end{minipage}}}

\tcbuselibrary{listings, theorems, breakable, skins}

\newtcbtheorem[number within=subsection]{thm}{Theorem}%
{colback=gray!5, colframe=gray!65!black, fonttitle=\bfseries, breakable, enhanced jigsaw, halign=left}{th}
\newtcbtheorem[number within=subsection, use counter from=thm]{defn}{Definition}%
{colback=gray!5, colframe=gray!65!black, fonttitle=\bfseries, breakable, enhanced jigsaw, halign=left}{th}
\newtcbtheorem[number within=subsection, use counter from=thm]{axm}{Axiom}%
{colback=gray!5, colframe=gray!65!black, fonttitle=\bfseries, breakable, enhanced jigsaw, halign=left}{th}
\newtcbtheorem[number within=subsection, use counter from=thm]{prp}{Proposition}%
{colback=gray!5, colframe=gray!65!black, fonttitle=\bfseries, breakable, enhanced jigsaw, halign=left}{th}
\newtcbtheorem[number within=subsection, use counter from=thm]{lmm}{Lemma}%
{colback=gray!5, colframe=gray!65!black, fonttitle=\bfseries, breakable, enhanced jigsaw, halign=left}{th}
\newtcbtheorem[number within=subsection, use counter from=thm]{crl}{Corollary}%
{colback=gray!5, colframe=gray!65!black, fonttitle=\bfseries, breakable, enhanced jigsaw, halign=left}{th}
\newtcbtheorem[number within=subsection, use counter from=thm]{eg}{Example}%
{colback=gray!5, colframe=gray!65!black, fonttitle=\bfseries, breakable, enhanced jigsaw, halign=left}{th}
\newtcbtheorem[number within=subsection, use counter from=thm]{ex}{Exercise}%
{colback=gray!5, colframe=gray!65!black, fonttitle=\bfseries, breakable, enhanced jigsaw, halign=left}{th}
\newtcbtheorem[number within=subsection, use counter from=thm]{alg}{Algorithm}%
{colback=gray!5, colframe=gray!65!black, fonttitle=\bfseries, breakable, enhanced jigsaw, halign=left}{th}

\newcounter{qtnc}
\newtcolorbox[use counter=qtnc]{qtn}%
{colback=gray!5, colframe=gray!65!black, fonttitle=\bfseries, breakable, enhanced jigsaw, halign=left}




\raggedright

\pagestyle{fancy}
\fancyhf{}
\rhead{Labix}
\lhead{Multilinear Algebra}
\rfoot{\thepage}

\title{Multilinear Algebra}

\author{Labix}

\date{\today}
\begin{document}
\maketitle
\begin{abstract}
\end{abstract}
\pagebreak
\tableofcontents
\pagebreak

\section{Tensor products of Vector Spaces}
\subsection{Tensor Products}
\begin{defn}{Bilinear Mappings}{} Let $V_1,V_2,W$ be vector spaces over $\F$. Let $\phi:V_1\times V_2\to W$ be a mapping. Then $\phi$ is called bilinear if
\begin{itemize}
\item $\phi(\lambda x_1+\mu x_2,y)=\lambda\phi(x_1,y)+\mu\phi(x_2,y)$ for all $x_1,x_2\in V_1$, $y\in V_2$ and $\lambda,\mu\in\F$
\item $\phi(x,\lambda y_1+\mu y_2)=\lambda\phi(x,y_1)+\mu\phi(x,y_2)$ for all $x\in V_1$, $y_1,y_2\in V_2$ and $\lambda,\mu\in\F$
\end{itemize}
If $W$ is the ground field $\F$, we say that $\phi$ is a bilinear function. 
\end{defn}

The tensor product is defined through a universal property. The following universal property essentially determines the uniqueness for a tensor product, but not the existence part. To complete the definition and show that tensor products indeed exists, we explicitly define one such of tensor products (one such because it is defined up to canonical isomorphism). 

\begin{defn}{Tensor Product and Universal Property}{} The tensor product of two vector spaces $V_1,V_2$ is a vector space denoted $V_1\otimes V_2$, toegther with a bilinear map $\phi:V_1\times V_2\to V_1\otimes V_2$ defined by $\phi(v_1,v_2)=v_1\otimes v_2$ such that for every bilinear map $h:V_1\times V_2\to W$, there is a unique linear map $\overline{h}:V_1\otimes V_2\to W$ such that $h=\overline{h}\circ\phi$. In other words, the following diagram commutes: \\~\\
\adjustbox{scale=1.1,center}{\begin{tikzcd}
V_1\times V_2\arrow[r, "\phi"]\arrow[rd, "h"] & V_1\otimes V_2\arrow[d, "\overline{h}"]\\
& W
\end{tikzcd}}
\end{defn}

This universal property allows canonical isomorphism since if we have two tensor products $V_1\otimes_1 V_2$ and $V_1\otimes_2 V_2$, we can apply the universal property to the both of them to obtain an isomorphism between the two. One way of explicitly calculating the elements are as follows: 

\begin{prp}{}{} Let $V,W$ be vector spaces over a field $\F$. Define $L$ to be the vector space that has $V\times W$ as a basis. Define $R$ to be the linear subspace of $L$ spanned by $$\{(v_1+v_2,w)-(v_1,w)-(v_2,w),(v,w_1+w_2)-(v,w_1)-(v,w_2),(sv,w)-s(v,w),(v,sw)-s(v,w)\}$$ Constuct the quotient space $V\otimes W$ to be $$V\otimes W=\frac{L}{R}$$ Then $V\otimes W$ is the tensor product of $V$ and $W$. 
\end{prp}

This gives the existence of tensor products. In practise no one explicitly finds out the elements in this way, and would simply use $v\otimes w$ to denote the tensor product. 

\begin{prp}{}{} Let $U,V,W$ be vector spaces. Then the following are true for tensor products. 
\begin{itemize}
\item Commutativity: There is a unique isomorphism from $V\otimes W$ to $W\otimes V$
\item Associativity: There is a unique isomorphism from $(U\otimes V)\otimes W$ to $U\otimes(V\otimes W)$
\item $\dim(V\otimes W)=\dim(V)\cdot\dim(W)$. 
\end{itemize}
\end{prp}

\subsection{Tensor Algebra}
\begin{defn}{$k$th Tensor Power}{} Let $V$ be a vector space over a field $\F$. Let $k\in\N$. Define the $k$th tensor power of $V$ to be the tensor product $$V^{\otimes k}=V\otimes V\dots\otimes V$$ where the tensor product over $V$ is taken $k$ times. \\~\\
By convention, define $V^{\otimes k}$ to be $\F$. 
\end{defn}

\begin{defn}{Tensor Algebra}{} Let $V$ be a vector space over $\F$. Define the tensor algebra over $V$ to be the direct sum $$T(V)=\bigoplus_{k=0}^\infty V^{\otimes k}$$ Define multiplication in $T(V)$ to be the determined by the canonical isomorphism $V^{\otimes k}\otimes V^{\otimes l}\to V^{\otimes k+l}$, which is extended by linearity to all of $T(V)$. 
\end{defn}

\begin{prp}{}{} Let $V$ be a vector space over $\F$. Then $T(V)$ is a graded algebra with the above defined multiplication rule. 
\end{prp}

\begin{prp}{Universal Property}{}
\end{prp}

\subsection{Exterior Algebra}
\begin{defn}{Exterior Algebra}{} Let $V$ be a vector space over $\F$. Let $I$ be the ideal generated by all elements of the form $v\otimes v$ for $v\in V$. Define the exterior algebra of $V$ to be the quotien $$\Lambda=T(V)/I$$ Elements of the form $v_1\otimes v_2$ are written as $v_1\wedge v_2$ by convention. 
\end{defn}

\begin{lmm}{}{} Let $V$ be a vector space over $\F$ where the characteristic of $\F$ is not equal to $2$. Then the ideals $$\{v\otimes v|v\in V\}=\{v\otimes w+w\otimes v\}$$ are equal and thus they both form the exterior algebra over $V$. 
\end{lmm}

\begin{prp}{}{} Let $V$ be a vector space over $\F$. Then the following are true for $\Lambda(V)$. 
\begin{itemize}
\item $\Lambda(V)$ is a graded algebra
\item $\Lambda(V)$ is alternating
\item $\Lambda(V)$ is anticommutative, meaning that $x\wedge y=-y\wedge x$
\end{itemize}
\end{prp}

\begin{defn}{$k$th Graded Component}{} Let $\Lambda(V)$ be an exterior algebra. Define the $k$th graded component of $\Lambda(V)$ to be the graded component $\Lambda^k(V)$. 
\end{defn}

\begin{prp}{Universal Property}{}
\end{prp}

\begin{prp}{}{} Let $\{e_1,\dots,e_n\}$ be a basis of the vector space $V$. Then $$\{e_{i_1}\wedge\dots\wedge e_{i_r}|1\leq i_1<\dots<i_r\leq n\}$$ is a basis of $\Lambda^r(V)$ and $$\dim(\Lambda^r(V))=\binom{n}{r}$$
\end{prp}



















\end{document}