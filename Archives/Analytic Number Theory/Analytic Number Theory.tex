\documentclass[a4paper]{article}

%=========================================
% Packages
%=========================================
\usepackage{mathtools}
\usepackage{amsfonts}
\usepackage{amsmath}
\usepackage{amssymb}
\usepackage{amsthm}
\usepackage[a4paper, total={6in, 8in}, margin=1in]{geometry}
\usepackage[utf8]{inputenc}
\usepackage{fancyhdr}
\usepackage[utf8]{inputenc}
\usepackage{graphicx}
\usepackage{physics}
\usepackage[listings]{tcolorbox}
\usepackage{hyperref}
\usepackage{tikz-cd}
\usepackage{adjustbox}
\usepackage{enumitem}


\hypersetup{
    colorlinks=true, %set true if you want colored links
    linktoc=all,     %set to all if you want both sections and subsections linked
    linkcolor=black,  %choose some color if you want links to stand out
}
\usetikzlibrary{arrows.meta}

\DeclarePairedDelimiter\ceil{\lceil}{\rceil}
\DeclarePairedDelimiter\floor{\lfloor}{\rfloor}

%=========================================
% Custom Math Operators
%=========================================
\DeclareMathOperator{\adj}{adj}
\DeclareMathOperator{\im}{im}
\DeclareMathOperator{\nullity}{nullity}
\DeclareMathOperator{\sign}{sign}
\DeclareMathOperator{\dom}{dom}
\DeclareMathOperator{\lcm}{lcm}
\DeclareMathOperator{\ran}{ran}
\DeclareMathOperator{\ext}{Ext}
\DeclareMathOperator{\dist}{dist}
\DeclareMathOperator{\diam}{diam}
\DeclareMathOperator{\aut}{Aut}
\DeclareMathOperator{\inn}{Inn}
\DeclareMathOperator{\syl}{Syl}
\DeclareMathOperator{\edo}{End}
\DeclareMathOperator{\cov}{Cov}
\DeclareMathOperator{\vari}{Var}
\DeclareMathOperator{\cha}{char}
\DeclareMathOperator{\Span}{span}
\DeclareMathOperator{\ord}{ord}
\DeclareMathOperator{\res}{res}
\DeclareMathOperator{\Hom}{Hom}
\DeclareMathOperator{\Mor}{Mor}
\DeclareMathOperator{\coker}{coker}
\DeclareMathOperator{\Obj}{Obj}
\DeclareMathOperator{\id}{id}
\DeclareMathOperator{\GL}{GL}
\DeclareMathOperator*{\colim}{colim}

%=========================================
% Custom Commands (Shortcuts)
%=========================================
\newcommand{\CP}{\mathbb{CP}}
\newcommand{\GG}{\mathbb{G}}
\newcommand{\F}{\mathbb{F}}
\newcommand{\N}{\mathbb{N}}
\newcommand{\Q}{\mathbb{Q}}
\newcommand{\R}{\mathbb{R}}
\newcommand{\C}{\mathbb{C}}
\newcommand{\E}{\mathbb{E}}
\newcommand{\Prj}{\mathbb{P}}
\newcommand{\RP}{\mathbb{RP}}
\newcommand{\T}{\mathbb{T}}
\newcommand{\Z}{\mathbb{Z}}
\newcommand{\A}{\mathbb{A}}
\renewcommand{\H}{\mathbb{H}}

\newcommand{\mA}{\mathcal{A}}
\newcommand{\mB}{\mathcal{B}}
\newcommand{\mC}{\mathcal{C}}
\newcommand{\mD}{\mathcal{D}}
\newcommand{\mE}{\mathcal{E}}
\newcommand{\mF}{\mathcal{F}}
\newcommand{\mG}{\mathcal{G}}
\newcommand{\mH}{\mathcal{H}}
\newcommand{\mJ}{\mathcal{J}}
\newcommand{\mO}{\mathcal{O}}
\newcommand{\mS}{\mathcal{S}}

%=========================================
% Theorem Environment
%=========================================
\newcommand\todoin[2][]{\todo[backgroundcolor=white!20!white, inline, caption={2do}, #1]{
\begin{minipage}{\textwidth-4pt}#2\end{minipage}}}

\tcbuselibrary{listings, theorems, breakable, skins}

\newtcbtheorem[number within=subsection]{thm}{Theorem}%
{colback=gray!5, colframe=gray!65!black, fonttitle=\bfseries, breakable, enhanced jigsaw, halign=left}{th}
\newtcbtheorem[number within=subsection, use counter from=thm]{defn}{Definition}%
{colback=gray!5, colframe=gray!65!black, fonttitle=\bfseries, breakable, enhanced jigsaw, halign=left}{th}
\newtcbtheorem[number within=subsection, use counter from=thm]{axm}{Axiom}%
{colback=gray!5, colframe=gray!65!black, fonttitle=\bfseries, breakable, enhanced jigsaw, halign=left}{th}
\newtcbtheorem[number within=subsection, use counter from=thm]{prp}{Proposition}%
{colback=gray!5, colframe=gray!65!black, fonttitle=\bfseries, breakable, enhanced jigsaw, halign=left}{th}
\newtcbtheorem[number within=subsection, use counter from=thm]{lmm}{Lemma}%
{colback=gray!5, colframe=gray!65!black, fonttitle=\bfseries, breakable, enhanced jigsaw, halign=left}{th}
\newtcbtheorem[number within=subsection, use counter from=thm]{crl}{Corollary}%
{colback=gray!5, colframe=gray!65!black, fonttitle=\bfseries, breakable, enhanced jigsaw, halign=left}{th}
\newtcbtheorem[number within=subsection, use counter from=thm]{eg}{Example}%
{colback=gray!5, colframe=gray!65!black, fonttitle=\bfseries, breakable, enhanced jigsaw, halign=left}{th}
\newtcbtheorem[number within=subsection, use counter from=thm]{ex}{Exercise}%
{colback=gray!5, colframe=gray!65!black, fonttitle=\bfseries, breakable, enhanced jigsaw, halign=left}{th}
\newtcbtheorem[number within=subsection, use counter from=thm]{alg}{Algorithm}%
{colback=gray!5, colframe=gray!65!black, fonttitle=\bfseries, breakable, enhanced jigsaw, halign=left}{th}

\newcounter{qtnc}
\newtcolorbox[use counter=qtnc]{qtn}%
{colback=gray!5, colframe=gray!65!black, fonttitle=\bfseries, breakable, enhanced jigsaw, halign=left}




\raggedright

\pagestyle{fancy}
\fancyhf{}
\rhead{Labix}
\lhead{Analytic Number Theory}
\rfoot{\thepage}

\title{Analytic Number Theory}

\author{Labix}

\date{\today}
\begin{document}
\maketitle
\begin{abstract}
\end{abstract}
\tableofcontents
\pagebreak

\section{Arithmetic Functions}
\subsection{Mobius Function and Euler Totient Function}
\begin{defn}[Arithmetical Functions] A function $f:\N\to\C$ is an arithmetical function. 
\end{defn}

\begin{defn}[Mobius Function] Let $n=\prod_{i=1}^kp_i^{\alpha_i}$. Define the mobius function as $$\mu(n)=(-1)^k$$ if $\alpha_1=\dots=\alpha_k=1$. And $0$ otherwise.  
\end{defn}

\begin{thm} For $n\in\N$, we have $$\sum_{d|n}\mu(d)=\floor*{\frac{1}{n}}=\begin{cases}
1 & \text{ if }n=1\\
0 & \text{ otherwise }
\end{cases}$$
\end{thm}

\begin{defn}[Euler's Totient Function] Let $\phi(n)$ denote the numeber of positive integers less than $n$ and relatively prime to $n$. 
\end{defn}

\begin{thm} Let $n\geq 1$. $$n=\sum_{d|n}\phi(d)$$
\end{thm}

\begin{thm} For $n\in\N$ we have $$\phi(n)=\sum_{d|n}\mu(d)\frac{n}{d}$$
\end{thm}

\begin{thm} For $n\in\N$ we have $$\phi(n)=n\prod_{p|n}\left(1-\frac{1}{p}\right)$$
\end{thm}

\begin{prp} The Euler's Totient Function has the following properties. 
\begin{itemize}
\item $\phi(p^n)=p^{n-1}(p-1)$
\item $\phi(mn)=\phi(m)\phi(n)\left(\frac{d}{\phi(d)}\right)$, where $d=\gcd(m,n)$
\item $a|b\implies\phi(a)|\phi(b)$
\item $\phi(n)$ is even for $n\geq 3$. Moreover, if $n$ has $r$ distinct odd prime factors, then $2^r|\phi(n)$
\end{itemize}
\end{prp}

\subsection{Dirichlet Functions}
\begin{defn}[Dirichlet Product] If $f$ and $g$ are two arithmetical functions we define their dirichlet product to be the arithmetical function $h$ defined by the equation $$h(n)=\sum_{d|n}f(d)g\left(\frac{n}{d}\right)$$
\end{defn}

\begin{thm} Dirichlet multiplication is commutative and associative. 
\end{thm}

\begin{defn} The arithmetical function $I$ given by $$I(n)=\floor*{\frac{1}{n}}=\begin{cases}
1 & \text{ if }n=1\\
0 & \text{ otherwise }
\end{cases}$$ is called the identity function. 
\end{defn}

\begin{prp} For all $f$ we have $I\ast f=f\ast I=f$
\end{prp}

\begin{thm} Let $f$ be an arithmetical function with $f(1)\neq0$. There is a unique arithemetical function $f^{-1}$, called the Dirichlet inverse of $f$ such that $$f\ast f^{-1}=f^{-1}\ast f=I$$ Moreover $f^{-1}$ is given by the recursion formulas $$f^{-1}(1)=\frac{1}{f(1)}$$ and $$f^{-1}(n)=\frac{-1}{f(n)}\sum_{d|n\text{ and }d<n}f\left(\frac{n}{d}\right)f^{-1}(d)$$ for $n>1$. 
\end{thm}

\begin{defn}[Unit Function] Define the unit function $u$ to be the arithmetical function such that $u(n)=1$ for all $n$. 
\end{defn}

\begin{prp} The Dirichlet Inverse of the mobius function is the unit function. 
\end{prp}

\begin{thm}[Mobius Inversion Formula] Let $f,g$ be arithmetical functions. $$f(n)=\sum_{d|n}g(d)$$ if and only if $$g(n)=\sum_{d|n}f(d)\mu\left(\frac{n}{d}\right)$$
\end{thm}

\subsection{Mangoldt Function}
\begin{defn}[Mangoldt's Function] For $n\in\N$ define $$\Lambda(n)=\begin{cases}
\ln(p) & \text{ if } n=p^m \text{ for some prime } p \text{ and some } m\geq 1\\
0 & \text{ otherwise }
\end{cases}$$
\end{defn}

\begin{thm} For $n\in\N$, $$\ln(n)=\sum_{d|n}\Lambda(d)$$
\end{thm}

\begin{thm} For $n\in\N$ we have $$\Lambda(n)=\sum_{d|n}\mu(d)\ln\left(\frac{n}{d}\right)=-\sum_{d|n}\mu(d)\ln(d)$$
\end{thm}

\subsection{Multiplicative Functions}
\begin{defn}[Multiplicative Functions] An arithmetical function $f$ is called multiplicative if $f$ is not identically $0$ and if $$f(mn)=f(m)f(n)$$ when $\gcd(m,n)=1$ It is completely multiplicative if it is multiplicative regardless of the condition. 
\end{defn}

\begin{prp} If $f$ is multiplicative then $f(1)=1$. 
\end{prp}

\begin{prp} Let $f(1)=1$ be an arithmetical function. $f$ is multiplicative if and only if $$f\left(\prod_{i=1}^kp_i^{\alpha_i}\right)=\prod_{i=1}^kf\left(p_i^{\alpha_i}\right)$$
\end{prp}

\begin{prp} Let $f(1)=1$ be an arithmetical function. $f$ is completely multiplicative if and only if $f(p)^\alpha=f(p)^\alpha$ for all primes $p$ and all integers $\alpha\geq 1$. 
\end{prp}

\begin{prp} If $f$ and $g$ are multiplicative, so is their Dirichlet product. 
\end{prp}

\begin{prp} If $f$ and $f\ast g$ are multiplicative, then $g$ is multiplicative. 
\end{prp}

\begin{prp} If $f$ is multiplicative, so is $f^{-1}$. 
\end{prp}

\subsection{Completely Multiplicative Functions}
\begin{thm} Let $f$ be multiplicative. Then $f$ is completely multiplicative if and only if $$f^{-1}(n)=\mu(n)f(n)$$ for all $n\geq 1$. 
\end{thm}

\begin{thm} If $f$ is multiplicative we have $$\sum_{d|n}\mu(d)f(d)=\prod_{p|n}(1-f(p))$$
\end{thm}

\subsection{Liouville's Function}
\begin{defn}[Liouville's Function] Define $\lambda(1)=1$ and if $$n=\prod_{i=1}^kp_i^{\alpha_i}$$ define $$\lambda(n)=(-1)^{\alpha_1+\dots+\alpha_k}$$
\end{defn}

\begin{prp} $\lambda(n)$ is completely mutiplicative. 
\end{prp}

\begin{thm} For $n\in\N$ we have $$\sum_{d|n}\lambda(d)=\begin{cases}
1 & \text{ if } n \text{ is a square }\\
0 & \text{ otherwise }
\end{cases}$$ Also $\lambda^{-1}(n)=\abs{\mu(n)}$ for all $n$. 
\end{thm}

\subsection{The Divisor Function}
\begin{defn} For real and complex $\alpha$ and any $n\in\N$ define $$\sigma_\alpha(n)=\sum_{d|n}d^\alpha$$
\end{defn}

\begin{prp} $\sigma_\alpha(n)$ is multiplicative. 
\end{prp}

\begin{thm} For $n\in\N$ we have $$\sigma_\alpha^{-1}(n)=\sum_{d|n}d^\alpha\mu(d)\mu\left(\frac{n}{d}\right)$$
\end{thm}

\subsection{Bell Series}
\begin{defn}[Bell Series] Let $f$ be an arithmetical function and $p$ a prime. Denote $$f_p(x)=\sum_{n=0}^\infty f(p^n)x^n$$ the bell series of $f$ modulo $p$. 
\end{defn}

\begin{thm} Let $f,g$ be multiplicative functions. Then $f=g$ if and only if $f_p(x)=g_p(x)$ for all primes $p$. 
\end{thm}

\begin{thm} Let $f,g$ be arithmetical functions and let $h=f\ast g$. Then for every prime $p$ we have $$h_p(x)=f_p(x)g_p(x)$$
\end{thm}

\subsection{Derivatives}
\begin{defn}[Derivatives of Arithmetical Functions] For any arithmetical function $f$ define $f'$ to be its derivative where $$f'(n)=f(n)\ln(n)$$ for $n\geq 1$. 
\end{defn}

\begin{thm} Let $f,g$ be arithmetical functions. 
\begin{itemize}
\item $(f+g)'=f'+g'$
\item $(f\ast g)'=f'\ast g+f\ast g'$
\item $(f^{-1})'=-f'\ast(f\ast f)^{-1}$ whenever $f(1)\neq0$
\end{itemize}
\end{thm}

\subsection{Selberg Identity}
\begin{thm}[Selberg Identity] For $n\in\N$ we have $$\Lambda(n)\ln(n)+\sum_{d|n}\Lambda(d)\Lambda\left(\frac{n}{d}\right)=\sum_{d|n}\mu(d)\ln^2\left(\frac{n}{d}\right)$$
\end{thm}






\end{document}