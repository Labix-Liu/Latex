\documentclass[a4paper]{article}

%=========================================
% Packages
%=========================================
\usepackage{mathtools}
\usepackage{amsfonts}
\usepackage{amsmath}
\usepackage{amssymb}
\usepackage{amsthm}
\usepackage[a4paper, total={6in, 8in}, margin=1in]{geometry}
\usepackage[utf8]{inputenc}
\usepackage{fancyhdr}
\usepackage[utf8]{inputenc}
\usepackage{graphicx}
\usepackage{physics}
\usepackage[listings]{tcolorbox}
\usepackage{hyperref}
\usepackage{tikz-cd}
\usepackage{adjustbox}
\usepackage{enumitem}


\hypersetup{
    colorlinks=true, %set true if you want colored links
    linktoc=all,     %set to all if you want both sections and subsections linked
    linkcolor=black,  %choose some color if you want links to stand out
}
\usetikzlibrary{arrows.meta}

\DeclarePairedDelimiter\ceil{\lceil}{\rceil}
\DeclarePairedDelimiter\floor{\lfloor}{\rfloor}

%=========================================
% Custom Math Operators
%=========================================
\DeclareMathOperator{\adj}{adj}
\DeclareMathOperator{\im}{im}
\DeclareMathOperator{\nullity}{nullity}
\DeclareMathOperator{\sign}{sign}
\DeclareMathOperator{\dom}{dom}
\DeclareMathOperator{\lcm}{lcm}
\DeclareMathOperator{\ran}{ran}
\DeclareMathOperator{\ext}{Ext}
\DeclareMathOperator{\dist}{dist}
\DeclareMathOperator{\diam}{diam}
\DeclareMathOperator{\aut}{Aut}
\DeclareMathOperator{\inn}{Inn}
\DeclareMathOperator{\syl}{Syl}
\DeclareMathOperator{\edo}{End}
\DeclareMathOperator{\cov}{Cov}
\DeclareMathOperator{\vari}{Var}
\DeclareMathOperator{\cha}{char}
\DeclareMathOperator{\Span}{span}
\DeclareMathOperator{\ord}{ord}
\DeclareMathOperator{\res}{res}
\DeclareMathOperator{\Hom}{Hom}
\DeclareMathOperator{\Mor}{Mor}
\DeclareMathOperator{\coker}{coker}
\DeclareMathOperator{\Obj}{Obj}
\DeclareMathOperator{\id}{id}
\DeclareMathOperator{\GL}{GL}
\DeclareMathOperator*{\colim}{colim}

%=========================================
% Custom Commands (Shortcuts)
%=========================================
\newcommand{\CP}{\mathbb{CP}}
\newcommand{\GG}{\mathbb{G}}
\newcommand{\F}{\mathbb{F}}
\newcommand{\N}{\mathbb{N}}
\newcommand{\Q}{\mathbb{Q}}
\newcommand{\R}{\mathbb{R}}
\newcommand{\C}{\mathbb{C}}
\newcommand{\E}{\mathbb{E}}
\newcommand{\Prj}{\mathbb{P}}
\newcommand{\RP}{\mathbb{RP}}
\newcommand{\T}{\mathbb{T}}
\newcommand{\Z}{\mathbb{Z}}
\newcommand{\A}{\mathbb{A}}
\renewcommand{\H}{\mathbb{H}}

\newcommand{\mA}{\mathcal{A}}
\newcommand{\mB}{\mathcal{B}}
\newcommand{\mC}{\mathcal{C}}
\newcommand{\mD}{\mathcal{D}}
\newcommand{\mE}{\mathcal{E}}
\newcommand{\mF}{\mathcal{F}}
\newcommand{\mG}{\mathcal{G}}
\newcommand{\mH}{\mathcal{H}}
\newcommand{\mJ}{\mathcal{J}}
\newcommand{\mO}{\mathcal{O}}
\newcommand{\mS}{\mathcal{S}}

%=========================================
% Theorem Environment
%=========================================
\newcommand\todoin[2][]{\todo[backgroundcolor=white!20!white, inline, caption={2do}, #1]{
\begin{minipage}{\textwidth-4pt}#2\end{minipage}}}

\tcbuselibrary{listings, theorems, breakable, skins}

\newtcbtheorem[number within=subsection]{thm}{Theorem}%
{colback=gray!5, colframe=gray!65!black, fonttitle=\bfseries, breakable, enhanced jigsaw, halign=left}{th}
\newtcbtheorem[number within=subsection, use counter from=thm]{defn}{Definition}%
{colback=gray!5, colframe=gray!65!black, fonttitle=\bfseries, breakable, enhanced jigsaw, halign=left}{th}
\newtcbtheorem[number within=subsection, use counter from=thm]{axm}{Axiom}%
{colback=gray!5, colframe=gray!65!black, fonttitle=\bfseries, breakable, enhanced jigsaw, halign=left}{th}
\newtcbtheorem[number within=subsection, use counter from=thm]{prp}{Proposition}%
{colback=gray!5, colframe=gray!65!black, fonttitle=\bfseries, breakable, enhanced jigsaw, halign=left}{th}
\newtcbtheorem[number within=subsection, use counter from=thm]{lmm}{Lemma}%
{colback=gray!5, colframe=gray!65!black, fonttitle=\bfseries, breakable, enhanced jigsaw, halign=left}{th}
\newtcbtheorem[number within=subsection, use counter from=thm]{crl}{Corollary}%
{colback=gray!5, colframe=gray!65!black, fonttitle=\bfseries, breakable, enhanced jigsaw, halign=left}{th}
\newtcbtheorem[number within=subsection, use counter from=thm]{eg}{Example}%
{colback=gray!5, colframe=gray!65!black, fonttitle=\bfseries, breakable, enhanced jigsaw, halign=left}{th}
\newtcbtheorem[number within=subsection, use counter from=thm]{ex}{Exercise}%
{colback=gray!5, colframe=gray!65!black, fonttitle=\bfseries, breakable, enhanced jigsaw, halign=left}{th}
\newtcbtheorem[number within=subsection, use counter from=thm]{alg}{Algorithm}%
{colback=gray!5, colframe=gray!65!black, fonttitle=\bfseries, breakable, enhanced jigsaw, halign=left}{th}

\newcounter{qtnc}
\newtcolorbox[use counter=qtnc]{qtn}%
{colback=gray!5, colframe=gray!65!black, fonttitle=\bfseries, breakable, enhanced jigsaw, halign=left}




\raggedright

\pagestyle{fancy}
\fancyhf{}
\rhead{Labix}
\lhead{Fourier Analysis}
\rfoot{\thepage}

\title{Fourier Analysis}

\author{Labix}

\date{\today}
\begin{document}
\maketitle
\begin{abstract}
\end{abstract}
\tableofcontents
\pagebreak

\section{The Fourier Series}
\subsection{Introduction to the Fourier Series}
\begin{defn}{The $n$th Fourier Coefficient}{} Let $f$ be an integrable function on an interval $[a,b]$ with $b-a=L$, then the $n$th Fourier coefficient of $f$ is defined by $$f_n=\frac{1}{L}\int_a^bf(x)e^{-\frac{2\pi in}{L}x}\,dx$$ where $n\in\Z$
\end{defn}

\begin{defn}{Fourier Series}{} Define the Fourier series of $f$ is given by $$f_F(x)=\sum_{n=-\infty}^\infty f_ne^{\frac{2\pi in}{L}x}$$
\end{defn}

\begin{defn}{$N$th partial sum}{} The $N$th partial sum of the Fourier series of $f$, for $N\in\N$ is given by $$S_N(f)(x)=\sum_{n=-N}^N f_ne^{\frac{2\pi in}{L}x}$$
\end{defn}

\begin{thm}{}{} Suppose that $f$ is an integrable function on the circle with $f_n=0$ for all $n\in\N$, then $f(\theta_0)=0$ whenever $f$ is continuous at the point $\theta_0$. 
\end{thm}

\begin{crl}{}{} Let $f$ be a continuous function on the circle and that the Fourier series of $f$ is absolutely convergent, then the Fourier series converges uniformly to $f$. That is $$\lim_{N\to\infty}S_N(f)(\theta)=f(\theta)$$
\end{crl}

\begin{crl}{}{} Let $f$ be twice differentiable defined on the circle. Then $$f_n=O\left(\frac{1}{\abs{n}^2}\right)$$ as $\abs{n}\to\infty$. 
\end{crl}

\begin{defn}{Convolution}{} Let $f,g$ be $2\pi$-periodic integrable functions on $\R$. Define their convolution $f\ast g$ on $[-\pi,\pi]$ by $$(f\ast g)(x)=\frac{1}{2\pi}\int_{-\pi}^\pi f(y)g(x-y)\,dy$$
\end{defn}

\begin{prp}{}{} Let $f,g,h$ be $2\pi$-periodic integrable functions. Then
\begin{itemize}
\item $f\ast(g+h)=(f\ast g)+(f\ast h)$
\item $(cf)\ast g=c(f\ast g)=f\ast(cg)$ for any $c\in\C$
\item $f\ast g=g\ast f$
\item $(f\ast g)\ast h=f\ast(g\ast h)$
\item $f\ast g$ is continuous
\item $(f\ast g)_n=f_ng_n$
\end{itemize}
\end{prp}

\begin{defn}{Family of Kernels}{} A family of kernels $\{K_n(x)\}_{n=1}^\infty$ on the circle is said to be a family of good kernels if 
\begin{itemize}
\item For all $n\in\N$, $\frac{1}{2\pi}\int_{-\pi}^\pi K_n(x)\,dx=1$
\item There exists $M>0$ such that for all $n\in\N$, $\int_{-\pi}^\pi\abs{K_n(x)}\,dx\leq M$
\item For every $\delta>0$. $\int_{\delta\leq\abs{x}\leq\pi}\abs{K_n(x)}\,dx\to0$
\end{itemize}
\end{defn}























\end{document}