\documentclass[a4paper]{article}

%=========================================
% Packages
%=========================================
\usepackage{mathtools}
\usepackage{amsfonts}
\usepackage{amsmath}
\usepackage{amssymb}
\usepackage{amsthm}
\usepackage[a4paper, total={6in, 8in}, margin=1in]{geometry}
\usepackage[utf8]{inputenc}
\usepackage{fancyhdr}
\usepackage[utf8]{inputenc}
\usepackage{graphicx}
\usepackage{physics}
\usepackage[listings]{tcolorbox}
\usepackage{hyperref}
\usepackage{tikz-cd}
\usepackage{adjustbox}
\usepackage{enumitem}


\hypersetup{
    colorlinks=true, %set true if you want colored links
    linktoc=all,     %set to all if you want both sections and subsections linked
    linkcolor=black,  %choose some color if you want links to stand out
}
\usetikzlibrary{arrows.meta}

\DeclarePairedDelimiter\ceil{\lceil}{\rceil}
\DeclarePairedDelimiter\floor{\lfloor}{\rfloor}

%=========================================
% Custom Math Operators
%=========================================
\DeclareMathOperator{\adj}{adj}
\DeclareMathOperator{\im}{im}
\DeclareMathOperator{\nullity}{nullity}
\DeclareMathOperator{\sign}{sign}
\DeclareMathOperator{\dom}{dom}
\DeclareMathOperator{\lcm}{lcm}
\DeclareMathOperator{\ran}{ran}
\DeclareMathOperator{\ext}{Ext}
\DeclareMathOperator{\dist}{dist}
\DeclareMathOperator{\diam}{diam}
\DeclareMathOperator{\aut}{Aut}
\DeclareMathOperator{\inn}{Inn}
\DeclareMathOperator{\syl}{Syl}
\DeclareMathOperator{\edo}{End}
\DeclareMathOperator{\cov}{Cov}
\DeclareMathOperator{\vari}{Var}
\DeclareMathOperator{\cha}{char}
\DeclareMathOperator{\Span}{span}
\DeclareMathOperator{\ord}{ord}
\DeclareMathOperator{\res}{res}
\DeclareMathOperator{\Hom}{Hom}
\DeclareMathOperator{\Mor}{Mor}
\DeclareMathOperator{\coker}{coker}
\DeclareMathOperator{\Obj}{Obj}
\DeclareMathOperator{\id}{id}
\DeclareMathOperator{\GL}{GL}
\DeclareMathOperator*{\colim}{colim}

%=========================================
% Custom Commands (Shortcuts)
%=========================================
\newcommand{\CP}{\mathbb{CP}}
\newcommand{\GG}{\mathbb{G}}
\newcommand{\F}{\mathbb{F}}
\newcommand{\N}{\mathbb{N}}
\newcommand{\Q}{\mathbb{Q}}
\newcommand{\R}{\mathbb{R}}
\newcommand{\C}{\mathbb{C}}
\newcommand{\E}{\mathbb{E}}
\newcommand{\Prj}{\mathbb{P}}
\newcommand{\RP}{\mathbb{RP}}
\newcommand{\T}{\mathbb{T}}
\newcommand{\Z}{\mathbb{Z}}
\newcommand{\A}{\mathbb{A}}
\renewcommand{\H}{\mathbb{H}}

\newcommand{\mA}{\mathcal{A}}
\newcommand{\mB}{\mathcal{B}}
\newcommand{\mC}{\mathcal{C}}
\newcommand{\mD}{\mathcal{D}}
\newcommand{\mE}{\mathcal{E}}
\newcommand{\mF}{\mathcal{F}}
\newcommand{\mG}{\mathcal{G}}
\newcommand{\mH}{\mathcal{H}}
\newcommand{\mJ}{\mathcal{J}}
\newcommand{\mO}{\mathcal{O}}
\newcommand{\mS}{\mathcal{S}}

%=========================================
% Theorem Environment
%=========================================
\newcommand\todoin[2][]{\todo[backgroundcolor=white!20!white, inline, caption={2do}, #1]{
\begin{minipage}{\textwidth-4pt}#2\end{minipage}}}

\tcbuselibrary{listings, theorems, breakable, skins}

\newtcbtheorem[number within=subsection]{thm}{Theorem}%
{colback=gray!5, colframe=gray!65!black, fonttitle=\bfseries, breakable, enhanced jigsaw, halign=left}{th}
\newtcbtheorem[number within=subsection, use counter from=thm]{defn}{Definition}%
{colback=gray!5, colframe=gray!65!black, fonttitle=\bfseries, breakable, enhanced jigsaw, halign=left}{th}
\newtcbtheorem[number within=subsection, use counter from=thm]{axm}{Axiom}%
{colback=gray!5, colframe=gray!65!black, fonttitle=\bfseries, breakable, enhanced jigsaw, halign=left}{th}
\newtcbtheorem[number within=subsection, use counter from=thm]{prp}{Proposition}%
{colback=gray!5, colframe=gray!65!black, fonttitle=\bfseries, breakable, enhanced jigsaw, halign=left}{th}
\newtcbtheorem[number within=subsection, use counter from=thm]{lmm}{Lemma}%
{colback=gray!5, colframe=gray!65!black, fonttitle=\bfseries, breakable, enhanced jigsaw, halign=left}{th}
\newtcbtheorem[number within=subsection, use counter from=thm]{crl}{Corollary}%
{colback=gray!5, colframe=gray!65!black, fonttitle=\bfseries, breakable, enhanced jigsaw, halign=left}{th}
\newtcbtheorem[number within=subsection, use counter from=thm]{eg}{Example}%
{colback=gray!5, colframe=gray!65!black, fonttitle=\bfseries, breakable, enhanced jigsaw, halign=left}{th}
\newtcbtheorem[number within=subsection, use counter from=thm]{ex}{Exercise}%
{colback=gray!5, colframe=gray!65!black, fonttitle=\bfseries, breakable, enhanced jigsaw, halign=left}{th}
\newtcbtheorem[number within=subsection, use counter from=thm]{alg}{Algorithm}%
{colback=gray!5, colframe=gray!65!black, fonttitle=\bfseries, breakable, enhanced jigsaw, halign=left}{th}

\newcounter{qtnc}
\newtcolorbox[use counter=qtnc]{qtn}%
{colback=gray!5, colframe=gray!65!black, fonttitle=\bfseries, breakable, enhanced jigsaw, halign=left}




\raggedright

\pagestyle{fancy}
\fancyhf{}
\rhead{Labix}
\lhead{Classical Mechanics}
\rfoot{\thepage}

\title{Classical Mechanics}

\author{Labix}

\date{\today}
\begin{document}
\maketitle
\begin{abstract}
\end{abstract}
\pagebreak
\tableofcontents

\pagebreak
\section{Newton's Laws of Motion}
\subsection{Space and Time}
\begin{defn}{Position}{} The position of a particle is defined in relation to a coordinate system centered on an arbitrary fixed reference point in space. 
\end{defn}

\begin{defn}{Velocity}{} The velocity of a particle with position $\vb{r}$ is given by $$\vb{v}=\frac{d\vb{r}}{dt}$$
\end{defn}

\begin{defn}{Acceleration}{} The acceleration of a particle with position $\vb{r}$ is given by $$\vb{a}=\frac{d\vb{v}}{dt}=\frac{d^2\vb{r}}{dt^2}$$
\end{defn}

\begin{defn}{Frame of Reference}{} A frame of reference is an abstract coordinate system whose origin, orientation and scale are specified by a set of reference points. 
\end{defn}

\begin{defn}{Inertial Frame of Reference}{} An intertial frame of reference is a frame of reference not undergoing accleration. 
\end{defn}

\subsection{Newton's Laws of Motion}
\begin{defn}{Mass}{} Mass is the quantity of matter, measured in $kg$. 
\end{defn}

\begin{defn}{Momentum}{} Momentum is the quantity of motion, defined to be $$\vb{p}=m\vb{v}$$
\end{defn}

\begin{defn}{Force}{} A force is an influence that can change the motion of an object. 
\end{defn}

\begin{axm}{Newton's First Law}{} In the absense of forces, a particle moves with constant velocity $\vb{v}$. 
\end{axm}

\begin{axm}{Newton's Second Law}{} For any particle of mass $m$, the net force $\vb{F}$ on the particle is always equal to the mass $m$ times the particle's accleration. In other words, $$\vb{F}=m\vb{a}=\frac{d\vb{p}}{dt}$$
\end{axm}

\begin{axm}{Newton's Third Law}{} If object $1$ exerts a force $\vb{F}_{21}$ on onject $2$, then the object $2$ always exerts a reaction force $\vb{F}_{12}$ on object $1$ given by $$\vb{F}_{12}=-\vb{F}_{21}$$
\end{axm}

\begin{thm}{Conservation of Momentum}{} If the net external force $\vb{F}^\text{ext}$ on an $N$-particle system if $0$, then then system's total momentum $\vb{P}$ is constant. \tcbline
\begin{proof} Suppose we have $n$ particles named $\alpha_1,\dots,\alpha_n$. For each $i\in\{1,\dots,n\}$, $$\vb{F}_{\alpha_i}=\sum_{k=1, k\neq i}^n\vb{F}_{\alpha_i\alpha_k}+\vb{F}_{\alpha_i}^\text{ext}$$ Now the total momentum of the system is given by $$\vb{P}=\sum_{k=1}^n\vb{p}_{\alpha_k}$$ Differentiating it gives $$\frac{d\vb{P}}{dt}=\sum_{k=1}^n\frac{d\vb{p}_{\alpha_k}}{dt}$$ Substituting $\vb{F}_{\alpha_i}=\frac{d\vb{p}_{\alpha_i}}{dt}$, we have 
\begin{align*}
\frac{d\vb{P}}{dt}&=\sum_{k=1}^n\sum_{j=1,j\neq k}^n\vb{F}_{\alpha_k\alpha_j}+\sum_{k=1}^n\vb{F}_{\alpha_k}^\text{ext}\\
&=\sum_{k=1}^n\sum_{j=k+1}^n(\vb{F}_{\alpha_k\alpha_j}+\vb{F}_{\alpha_j\alpha_k})+\sum_{k=1}^n\vb{F}_{\alpha_k}^\text{ext}\\
&=\sum_{k=1}^n\vb{F}_{\alpha_k}^\text{ext}\\
&=\vb{F}^\text{ext}
\end{align*} Thus if the net external force $\vb{F}^{ext}=0$, $\vb{P}$ is a constant. 
\end{proof}
\end{thm}

\begin{thm}{Law of Gravity}{} Suppose two mass $m_1,m_2$ are in play and they are $r$ distance apart. Newton's law of gravity states that there is a force acting on $m_2$ by $m_1$ and vice versa given by $$F=\frac{Gm_1m_2}{r^2}$$
\end{thm}

\subsection{Friction}
\begin{defn}{Static Friction}{} Friction arises when one onject is in contact with another. Denote $f_{\text{max}}$ the maximal of static friction before it changes into kinetic friction. If $\vb{F}<f_{\text{max}}$, then the static friction $\vb{f}=-\vb{F}$. We also have $f_{\text{max}}=\mu\abs{\vb{N}}$, meaning the max friction is proportional to $\vb{N}$. 
\end{defn}

\begin{defn}{Coefficient of Friction}{} $\mu$ is called the coefficient of friction. We have $0<\mu\leq 1$. 
\end{defn}

\begin{defn}{Kinetic Friction}{} When motion occurs, friction is still roughly proportional to the normal force, but the coefficient changes $$f_k=\mu_kN$$ where $\mu_k$ is the coefficient of kinetic or dynamic friction and $f_k$ is the kinetic friction. 
\end{defn}

\subsection{Drag Force}
\begin{defn}{Drag Force}{} Drag force is experienced when moaing fluid or gas, the force that hinders your movement. $$D=\frac{1}{2}C\rho Av^2$$ where $\rho$ is the air density, $A$ is the effective cross-sectional area, $C$ the drag coefficient. 
\end{defn}

\begin{thm}{Falling}{} When free falling, the total force acting on the body is $D-F_g$. Drag force reaches maximum $F_g$ when $a=0$. The velocity required thus is given by $$v_t=\sqrt{\frac{2F_g}{C\rho A}}$$
\end{thm}

\subsection{Circular Motion}
\begin{thm}{}{} If a particle moves in a circle or a circular arc of radius $R$ at constant speed $v$, the particle experiences centripetal acceleration $\vb{a}$ with $$\abs{\vb{a}}=\frac{v^2}{R}$$
\end{thm}

\begin{thm}{}{} Thus the force experienced by the particle is given by $$F=m\frac{v^2}{R}$$
\end{thm}

\pagebreak
\section{Work and Energy}
\subsection{Kinetic Energy}
\begin{axm}{Principle of Energy Conservation}{} Energy in a system is conserved. 
\end{axm}

\begin{defn}{Kinetic Energy}{} Kinetic Energy $K$ is energy associated with the state of motion of an object. For an object of mass $m$ whose speed $v$ is well below the speed of light, $$K=\frac{1}{2}mv^2$$
\end{defn}

\subsection{Work}
\begin{defn}{Work}{} Work $W$ is energy transferred to or from an object by means of a force acting on the object. 
\end{defn}

\begin{thm}{}{} $$W=Fd\cos(\theta)$$ where $\theta$ is the angle between $\vb{F}$ and $\vb{d}$. 
\end{thm}

\begin{defn}{}{}$$W=\vb{F}\cdot\vb{d}$$ where $d$ is the displacement of the object. 
\end{defn}

\begin{thm}{Work Kinetic Energy Theorem}{} The change in kinectic energy is equal to the network done. $$W=T_2-T_1$$ where $T_1$ is the initial kinetic energy and $T_2$ its final. 
\end{thm}

\begin{thm}{Work Done by Gravitational Force}{} $$W_g=mgd\cos(\theta)$$
\end{thm}

\begin{thm}{Hooke's Law}{} $$\vb{F}_s=-k\vb{d}$$ where $k$ is called the spring constant, $\vb{d}$ is the displacement. 
\end{thm}

\begin{thm}{}{} Work done by a spring force $$W_s=\frac{1}{2}k(x_i^2-x_f^2)$$
\end{thm}

\subsection{Work done by a Force as a Variable}
\begin{defn}{}{} The work done by $F$ on the $x$ component is given by $$W=\int_{x_i}^{x_f}F(x)\,dx$$
\end{defn}

\begin{thm}{}{} The work done $$W=\int_{x_i}^{x_f}F_x\,dx+\int_{y_i}^{y_f}F_y\,dy+\int_{z_i}^{z_f}F_z\,dz$$
\end{thm}

\subsection{Power}
\begin{defn}{}{} power is defined as the rate at which work is done. $$P=\frac{dW}{dt}$$
\end{defn}

\begin{defn}{}{} The instantaneous power at a point in time is given by $P=\vb{F}\cdot\vb{v}$
\end{defn}

\pagebreak
\section{Simple Harmonic Motion}
\subsection{Mass on a Spring}
\begin{thm}{}{} Springs produce a force $F$ that is linear to the displacement of $x$. $$F=-kx$$
\end{thm}










\end{document}