\documentclass[a4paper]{article}

\input{Headers}

\pagestyle{fancy}
\fancyhf{}
\rhead{Labix}
\lhead{Measure Theory}
\rfoot{\thepage}

\title{Measure Theory}

\author{Labix}

\date{\today}
\begin{document}
\maketitle
\begin{abstract}
\end{abstract}
\pagebreak
\tableofcontents
\pagebreak

\section{Measure Theory}
\subsection{Sigma Fields}
\begin{defn}[Sigma Fields] A $\sigma$-field on a non-empty set $S$ is a collection $\mathcal{F}$ of subsets of $S$ such that
\begin{itemize}
\item $S,\emptyset\in\mathcal{F}$
\item $A\in\mathcal{F}$ implies $A^c\in\mathcal{F}$
\item $A_k\in\mathcal{F}$, $k\in\N$ implies $\bigcup_{k=1}^\infty A_k\in\mathcal{F}$
\end{itemize}
\end{defn}

\begin{defn}[Measure] A measure on a $\sigma$-field $\mathcal{F}$ of subsets of $S$ is a function $\mu:\mathcal{F}\to[0,+\infty]$ such that 
\begin{itemize}
\item $\mu(\emptyset)=$
\item $\mu\left(\bigcup_{k=1}^\infty A_k\right)=\sum_{k=1}^\infty\mu\left(A_k\right)$ where $A_k\in\mathcal{F}$ are pairwise disjoint. 
\end{itemize}
\end{defn}

\begin{prp} Let $\mu$ be a measure on a $\sigma$-field $\mathcal{F}$ and $A_1,A_2,\dots\in\mathcal{F}$
\begin{itemize}
\item If $A_1\subseteq A_2$, then $\mu(A_1)\leq\mu(A_2)$
\item $\mu(\bigcup_{k=1}^\infty A_k)\leq\sum_{k=1}^\infty\mu(A_k)$
\item $\mu(A_1)+\mu(A_2)=\mu(A_1\cup A_2)+\mu(A_1\cap A_2)$
\end{itemize}
\end{prp}

\pagebreak
\section{Lebesgue Measure}
\subsection{Elementary Measure}
\begin{defn}[Intervals] An interval $I$ is a subset of $\R$ of the form 
\begin{itemize}
\item $[a,b]=\{x\in\R|a\leq x\leq b\}$
\item $(a,b]=\{x\in\R|a<x\leq b\}$
\item $[a,b)=\{x\in\R|a\leq x<b\}$
\item $(a,b)=\{x\in\R|a<x<b\}$
\end{itemize} Define the measure of an interval to be its length, $m(I)=b-a$
\end{defn}

\begin{defn}[Boxes] A box in $\R^n$ is a cartesian product $$B=I_1\times I_2\times\dots\times I_n$$ of $n$ intervals. Define the measure of a box to be its "volume", $$m(B)=\abs{B}=\abs{I_1}\cdots\abs{I_n}$$
\end{defn}

\begin{defn}[Elementary Sets] An elementary set is a finite union of boxes. Denote the set of all Elementary sets to be $\mathcal{M}_E$. 
\end{defn}

\begin{prp} Let $E,F\in\mathcal{M}_E$. $E\cap F$, $E\cup F$, $E/F\in\mathcal{M}_E$. 
\end{prp}

\begin{prp} Let $E$ be an elementary set of $\R^n$. 
\begin{itemize}
\item $E$ can be partitioned into finite union of disjoint boxes. 
\item The measure of $E$ is independent of the partition. 
\end{itemize}
\end{prp}

\begin{prp} Let $E,F$ be an elementary set. Elementary measures have the following properties. 
\begin{itemize}
\item $m(E)\geq 0$
\item $m(E\cup F)=m(E)+m(F)$ if $E\cap F=\emptyset$
\item $m(\emptyset)=0$
\item $E\subset F\implies m(E)\leq m(F)$
\item $m(E\cup F)\leq m(E)+m(F)$
\end{itemize}
\end{prp}

\begin{thm} The elementary measure function is unqiue up to a constant multiple. 
\end{thm}

\subsection{Jordan Measure}
\begin{defn}[Jordan Inner Measure] Let $E\subset\R^n$ be a bounded set. Define the Jordan Inner Measure as $$m_{J\ast}(E)=\sup_{A\subset E,A\in\mathcal{M}_E}m(A)$$
\end{defn}

\begin{defn}[Jordan Outer Measure] Let $E\subset\R^n$ be a bounded set. Define the Jordan Outer Measure as $$m^{J\ast}(E)=\inf_{E\subset A,A\in\mathcal{M}_E}m(A)$$
\end{defn}

\begin{defn}[Jordan Measurable] We say that $E$ is Jordan mesurable if $$m_{J\ast}(E)=m^{J\ast}(E)$$ Denote the set of all Jordan measurable sets to be $\mathcal{M}_J$. 
\end{defn}

\begin{lmm} $\mathcal{M}_E\subseteq\mathcal{M}_J$ and $m(E)$ is consistent if $E\in\mathcal{M}_E$
\end{lmm}

\begin{prp} Let $E,F\in\mathcal{M}_J$. 
\begin{itemize}
\item $E\cap F\in\mathcal{M}_J$
\item $E\cup F\in\mathcal{M}_J$
\item $E/F\in\mathcal{M}_J$
\end{itemize}
\end{prp}

\begin{prp} A set $E\subset\R^n$ is Jordna measurable if and only if for every $\epsilon>0$ there exists $A\subset E\subset B$ such that $m(B/A)<\epsilon$
\end{prp}

\begin{prp} Let $E,F\in\mathcal{M}_J$. Elementary measures have the following properties. 
\begin{itemize}
\item $m(E)\geq 0$
\item $m(E\cup F)=m(E)+m(F)$ if $M\cap F=\emptyset$
\item $m(\emptyset)=0$
\item $E\subset F\implies m(E)\leq m(F)$
\item $m(E\cup F)\leq m(E)+m(F)$
\end{itemize}
\end{prp}

\begin{thm} The Jordan measure function is unqiue up to a constant multiple. 
\end{thm}

\begin{prp} Let $E\subset\R^n$ and $F\subset\R^m$ be Jordan measurable. $E\times F$ is also Jordan Measurable and $m(E\times F)=m(E)\times m(F)$. 
\end{prp}

\begin{prp} $B_r(x)\subset\R^n$ and $\overline{B}_r(x)\subset\R^n$ are Jordan Measurable with Jordan Measure $c_nr^n$ for some $c_n>0$ depending on $n$. 
\end{prp}

\begin{prp} Let $E\in\mathcal{M}_J$. 
\begin{itemize}
\item $m^{J\ast}(E)=m^{J\ast}(\overline{E})$
\item $m_{J\ast}(E)=m^{J\ast}(E^\circ)$
\item $E\in\mathcal{M}_J$ if and only if $m^{J\ast}(\partial E)=0$
\end{itemize}
\end{prp}

\begin{prp}[Jordan Measurable implies Riemann Integrable] Let $E\in\mathcal{M}_J$ and $E\subset[a,b]$. Let $$1_E=\begin{cases}
1 & x\in E\\
0 & x\notin E
\end{cases}$$ Then $1_E$ is Riemann Integrable and $\int_a^b 1_E(x)\,dx=m(E)$
\end{prp}

\begin{prp} Let $f:[a,b]\to\R$ be bounded. $f$ is Riemann integrable if and only if $E_\ast=\{(x,t)|x\in[a,b],f(x)\leq t\leq 0\}$ and $E^\ast=\{(x,t)|x\in[a,b],0\leq t\leq f(x)\}$ are Jordan Measurable. In this case, $$\int_a^bf(x)\,dx=m(E^\ast)-m(E_\ast)$$
\end{prp}

\begin{lmm} Let $E_n\in\mathcal{M}_J$ for all $n$. $\bigcup_{n=1}^\infty E_n$ and $\bigcap_{n=1}^\infty E_n$ may not be Jordan Measurable. 
\end{lmm}

\subsection{Lebesgue Outer Measure}
\begin{defn}[Lebesgue Outer Measure] Let $E\subset\R^n$ be a bounded set. Define the Lebesgue Outer Measure as $$m^\ast(E)=\inf_{E\subset \bigcup_{n=1}^\infty A_n,A_n\in\mathcal{M}_E}\sum_{n=1}^\infty\abs{A_n}$$
\end{defn}

\begin{prp} $m^\ast(E)\leq m^{J\ast}(E)$
\end{prp}

\begin{prp} Let $E,F\subset\R^n$. 
\begin{itemize}
\item $m^\ast(\emptyset)=0$
\item $E\subset F$ implies $m^\ast(E)\leq m^\ast(F)$
\item Let $E_n\subset\R^n$ for all $n$. Then $$m^\ast\left(\bigcup_{n=1}^\infty E_n\right)\leq\sum_{n=1}^\infty m^\ast(E_n)$$
\end{itemize}
\end{prp}

\begin{lmm} Let $E,F\subset\R^n$ such that $\dist(E,F)>0$. Then $$m^\ast(E\cup F)=m^\ast(E)+m^\ast(F)$$
\end{lmm}

\begin{defn}[Almost Disjoint Sets] Two boxes $E,F$ are almost disjoint if $$E^\circ\cap F^\circ=\emptyset$$
\end{defn}

\begin{lmm} Let $E=\bigcup_{n=1}^\infty B_n$ be a countable union of almost disjoint boxes. Then $$m^\ast(E)=\sum_{n=1}^\infty\abs{B_n}$$
\end{lmm}

\begin{lmm} If $E\subset\R^n$ is expressible as a countable union of almost disjoint boxes, then $m^\ast(E)=m_{J\ast}(E)$
\end{lmm}

\begin{lmm} Open sets can be expressed as a countable union of almost disjoint boxes. 
\end{lmm}

\begin{lmm} Let $E\subset\R^n$ be an arbitrary set. Then $$m^\ast(E)=\inf_{E\subset U,U\text{ open}}m^\ast(U)$$
\end{lmm}

\subsection{Lebesque Measure}
\begin{defn}[Lebesgue Measurability] A set $E\subset\R^n$ is Lebesgue mesurable if for every $\epsilon>0$, there exists an open set $U\subset\R^n$ with $E\subset U$ such that $m^\ast(U/E)<\epsilon$. Denote the set of all Lebesgue measurable sets to be $\mathcal{M}_L$. 
\end{defn}

\begin{prp} There exists Lebesgue mesurable sets. 
\begin{itemize}
\item Every open set is Lebesgue measurable
\item Every closed set is Lebesgue measurable
\item Every set of Lebesgue outer measure $0$ is measurable
\item $\emptyset\in\mathcal{M}_L$
\item If $E\subset\R^n\in\mathcal{M}_L$ implies $\R^n/E\in\mathcal{M}_L$
\item Let $E_n\in\mathcal{M}_L$ for all $n$ and $E_n\subset\R^n$. Then $\bigcup_{n=1}^\infty E_n\in\mathcal{M}_L$ and $\bigcap_{n=1}^\infty E_n\in\mathcal{M}_L$
\end{itemize}
\end{prp}

\begin{thm}[Criteria for Measurability] Let $E\subset\R^n$. The following are equivalent
\begin{itemize}
\item $E\in\mathcal{M}_L$
\item For every $\epsilon>0$ there exists an open set $U$ such that $E\subset U$ with $m^\ast(U/E)<\epsilon$
\item For every $\epsilon>0$ there exists an open set $U$ such that $m^\ast(U\bigtriangleup E)<\epsilon$
\item For every $\epsilon>0$, there exists a closed set $F$ such that $F\subset E$ with $m^\ast(E/F)<\epsilon$
\item For every $\epsilon>0$ there exists a closed set $F$ such that $m^\ast(F\bigtriangleup E)<\epsilon$
\item For every $\epsilon>0$ there exists $L\in\mathcal{M}_L$ such that $m^\ast(L\bigtriangleup E)<\epsilon$. 
\end{itemize}
\end{thm}

\begin{lmm} $\mathcal{M}_J\subset\mathcal{M}_L$ and $m(E)$ is consistent if $E\in\mathcal{M}_J$. 
\end{lmm}

\begin{prp}[Measure Axioms] Let $E,F\subset\R^n$. 
\begin{itemize}
\item $m(\emptyset)=0$
\item Let $E_n\subset\R^n$ for all $n$. Then $$m\left(\bigcup_{n=1}^\infty E_n\right)=\sum_{n=1}^\infty m(E_n)$$
\end{itemize}
\end{prp}

\begin{thm}[Monotone Convergence Theorem] Let $E_n$ be measurable for all $n$. 
\begin{itemize}
\item Suppose $E_1\subset E_2\subset\dots\subset\R^d$. Then $m\left(\bigcup_{n=1}^\infty E_n\right)=\lim_{n\to\infty}m(E_n)$
\item Suppose $\R^d\supset E_1\supset E_2\supset\dots$. Then $m\left(\bigcap_{n=1}^\infty E_n\right)=\lim_{n\to\infty}m(E_n)$
\end{itemize}
\end{thm}

\begin{defn}[Set Convergence] We say that $\{E_n\}$ converges to $E$ if the indicator functionsd $1_{E_n}$ converges pointwise to $E$. 
\end{defn}

\begin{thm} Suppose that $\{E_n\}\subset\mathcal{M}_L$ and $\{E_n\to E\}$ pointwise. Then $E\in\mathcal{M}_L$. 
\end{thm}

\begin{thm}[Dominated Convergence Theorem] Suppose that $E_n$ are all contained in another Lebesgue measurable set $F$ of finite measure. Then $$\lim_{n\to\infty}m(E_n)=m(E)$$
\end{thm}

\subsection{Lebesque Integral}
\begin{defn}[Simple Functions] A simple function $f:\R^d\to\C$ is a finite linear combination $$f=c_11_{E_1}+\dots+c_k1_{E_k}$$ of indicator functions $1_{E_i}$ that are from Lesbegue measurable sets $E_i\subset\R^d$ for $i\in\{1,\dots,k\}$, where $k\in\N$ and $c_1,\dots,c_k\in\C$. 
\end{defn}

\begin{defn}[Unsigned Simple Functions] A simple function is unsigned if $f:\R^d\to[0,+\infty]$ and $c_i$ is a function mapping to the positive reals. 
\end{defn}

\begin{defn} Let $f$ be an unsigned simple function. Then define $$S\left(\int_{R^d}f(x)\,dx\right)=\sum_{k=1}^nc_km(E_k)$$
\end{defn}















\end{document}