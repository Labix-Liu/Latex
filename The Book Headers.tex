\usepackage{mathtools}
\usepackage{amsfonts}
\usepackage{amsmath}
\usepackage{amssymb}
\usepackage{amsthm}
\usepackage[a4paper, total={6in, 8in}, margin=1in]{geometry}
\usepackage[utf8]{inputenc}
\usepackage{fancyhdr}
\usepackage[utf8]{inputenc}
\usepackage{graphicx}
\usepackage{physics}
\usepackage[listings]{tcolorbox}
\usepackage{hyperref}
\usepackage{tikz-cd}
\usepackage{adjustbox}
\usepackage{enumitem}
\hypersetup{
    colorlinks=true, %set true if you want colored links
    linktoc=all,     %set to all if you want both sections and subsections linked
    linkcolor=black,  %choose some color if you want links to stand out
}
\usetikzlibrary{arrows.meta}

\DeclarePairedDelimiter\ceil{\lceil}{\rceil}
\DeclarePairedDelimiter\floor{\lfloor}{\rfloor}

\DeclareMathOperator{\adj}{adj}
\DeclareMathOperator{\im}{im}
\DeclareMathOperator{\nullity}{nullity}
\DeclareMathOperator{\sign}{sign}
\DeclareMathOperator{\dom}{dom}
\DeclareMathOperator{\lcm}{lcm}
\DeclareMathOperator{\ran}{ran}
\DeclareMathOperator{\ext}{Ext}
\DeclareMathOperator{\dist}{dist}
\DeclareMathOperator{\diam}{diam}
\DeclareMathOperator{\aut}{Aut}
\DeclareMathOperator{\inn}{Inn}
\DeclareMathOperator{\syl}{Syl}
\DeclareMathOperator{\homo}{Hom}
\DeclareMathOperator{\edo}{End}
\DeclareMathOperator{\cov}{Cov}
\DeclareMathOperator{\vari}{Var}
\DeclareMathOperator{\cha}{char}
\DeclareMathOperator{\Span}{span}
\DeclareMathOperator{\ord}{ord}
\DeclareMathOperator{\res}{res}
\DeclareMathOperator{\Hom}{Hom}
\DeclareMathOperator{\Mor}{Mor}
\DeclareMathOperator{\coker}{coker}

\newcommand{\C}{\mathbb{C}}
\newcommand{\CP}{\mathbb{CP}}
\newcommand{\GG}{\mathbb{G}}
\newcommand{\F}{\mathbb{F}}
\newcommand{\N}{\mathbb{N}}
\newcommand{\Q}{\mathbb{Q}}
\newcommand{\R}{\mathbb{R}}
\newcommand{\E}{\mathbb{E}}
\newcommand{\Prj}{\mathbb{P}}
\newcommand{\RP}{\mathbb{RP}}
\newcommand{\T}{\mathbb{T}}
\newcommand{\Z}{\mathbb{Z}}
\newcommand{\A}{\mathbb{A}}
\renewcommand{\H}{\mathbb{H}}

\newcommand\todoin[2][]{\todo[backgroundcolor=white!20!white, inline, caption={2do}, #1]{
\begin{minipage}{\textwidth-4pt}#2\end{minipage}}}

\tcbuselibrary{listings, theorems, breakable, skins}

\newtcbtheorem[number within=section]{thm}{Theorem}%
{colback=gray!5, colframe=gray!65!black, fonttitle=\bfseries, breakable, enhanced jigsaw, halign=left}{th}
\newtcbtheorem[number within=section, use counter from=thm]{defn}{Definition}%
{colback=gray!5, colframe=gray!65!black, fonttitle=\bfseries, breakable, enhanced jigsaw, halign=left}{th}
\newtcbtheorem[number within=section, use counter from=thm]{axm}{Axiom}%
{colback=gray!5, colframe=gray!65!black, fonttitle=\bfseries, breakable, enhanced jigsaw, halign=left}{th}
\newtcbtheorem[number within=section, use counter from=thm]{prp}{Proposition}%
{colback=gray!5, colframe=gray!65!black, fonttitle=\bfseries, breakable, enhanced jigsaw, halign=left}{th}
\newtcbtheorem[number within=section, use counter from=thm]{lmm}{Lemma}%
{colback=gray!5, colframe=gray!65!black, fonttitle=\bfseries, breakable, enhanced jigsaw, halign=left}{th}
\newtcbtheorem[number within=section, use counter from=thm]{crl}{Corollary}%
{colback=gray!5, colframe=gray!65!black, fonttitle=\bfseries, breakable, enhanced jigsaw, halign=left}{th}
\newtcbtheorem[number within=section, use counter from=thm]{eg}{Example}%
{colback=gray!5, colframe=gray!65!black, fonttitle=\bfseries, breakable, enhanced jigsaw, halign=left}{th}
\newtcbtheorem[number within=section, use counter from=thm]{ex}{Exercise}%
{colback=gray!5, colframe=gray!65!black, fonttitle=\bfseries, breakable, enhanced jigsaw, halign=left}{th}
\newtcbtheorem[number within=section, use counter from=thm]{alg}{Algorithm}%
{colback=gray!5, colframe=gray!65!black, fonttitle=\bfseries, breakable, enhanced jigsaw, halign=left}{th}

\newcounter{qtnc}
\newtcolorbox[use counter=qtnc]{qtn}%
{colback=gray!5, colframe=gray!65!black, fonttitle=\bfseries, breakable, enhanced jigsaw, halign=left}




\raggedright