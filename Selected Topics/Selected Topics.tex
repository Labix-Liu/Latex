\documentclass[a4paper]{article}

\input{C:/Users/liula/Desktop/Latex/Headers V1.2.tex}

\pagestyle{fancy}
\fancyhf{}
\rhead{Labix}
\lhead{Selected Topics}
\rfoot{\thepage}

\title{Selected Topics}

\author{Labix}

\date{\today}
\begin{document}
\maketitle
\begin{abstract}
\end{abstract}

References: 
\pagebreak
\tableofcontents

\pagebreak
\section{Symmetric Polynomials}
\subsection{Symmetric Polynomials}
The theory of symmetric functions are important in combinatorics, representation theory, Galois theory and the theory of $\lambda$-rings. \\

Requirements: Groups and Rings\\
Books: Donald Yau: Lambda Rings

\begin{defn}{Symmetric Group Action on Polynomial Rings}{} Let $R$ be a ring. Define a group action of $S_n$ on $R[x_1,\dots,x_n]$ by $$\sigma\cdot f(x_1,\dots,x_n)=f(x_{\sigma(1)},\dots,x_{\sigma(n)})$$
\end{defn}

It is easy to check that this defines a group action. 

\begin{defn}{Symmetric Polynomials}{} Let $R$ be a ring. We say that a polynomial $f\in R[x_1,\dots,x_n]$ is symmetric if $$\sigma\cdot f=f$$ for all $\sigma\in S_n$. 
\end{defn}

\begin{defn}{The Ring of Symmetric Polynomials}{} Let $R$ be a ring. Define the ring of symmetric polynomials in $n$ variables over $R$ to be the set $$\Sigma=\{f\in R[x_1,\dots,x_n]\;|\;\sigma\text{ is a symmetric polynomial }\}$$
\end{defn}

\begin{defn}{Elementary Symmetric Polynomials}{} Let $R$ be a ring. Define the elementary symmetric polynomials to be the elements $s_1,\dots,s_n\in R[x_1,\dots,x_n]$ given by the formula $$s_k(x_1,\dots,x_n)=\sum_{1\leq i_1\leq\cdots\leq i_k\leq n}x_{i_1}\cdots x_{i_k}$$
\end{defn}

\begin{thm}{The Fundamental Theorem of Symmetric Polynomials}{} Let $R$ be a ring. Then $s_1,\dots,s_n$ are algebraically independent over $R$. Moreover, $$\Sigma=R[s_1,\dots,s_n]$$
\end{thm}

\pagebreak
\section{$\lambda$-Rings}
\subsection{$\lambda$-Rings}
Complex representation of a group is a $\lambda$-ring. Topological $K$ theory is a $\lambda$-ring. \\

Requirements: Category Theory, Groups and Rings, Symmetric Functions\\
Books: Donald Yau: Lambda Rings\\

We need the theory of symmetric polynomials before defining $\lambda$-structures. 

\begin{defn}{$\lambda$-Structures}{} Let $R$ be a commutative ring. A $\lambda$-structure on $R$ consists of a sequence of maps $\lambda^n:R\to R$ for $n\geq 0$ such that the following are true. 
\begin{itemize}
\item $\lambda^0(r)=1$ for all $r\in R$
\item $\lambda^1=\text{id}_R$
\item $\lambda^n(1)=0$ for all $n\geq 2$
\item $\lambda^n(r+s)=\sum_{k=0}^n\lambda^k(r)\lambda^{n-k}(s)$ for all $r,s\in R$
\item $\lambda^n(rs)=P_n(\lambda^1(r),\dots,\lambda^n(r),\lambda^1(s),\dots,\lambda^n(s))$ for all $r,s\in R$
\item $\lambda^m(\lambda^n(r))=P_{m,n}(\lambda^1(r),\dots,\lambda^{mn}(r))$ for all $r\in R$
\end{itemize}
Here $P_n$ and $P_{m,n}$ are defined as follows. 
\begin{itemize}
\item The coefficient of $t^n$ in the polynomial $$h(t)=\prod_{i,j=1}^n(1+x_iy_jt)$$ is a symmetric polynomial in $x_i$ and $y_j$ with coefficients in $\Z$. $P_n$ is precisely this polynomial written in terms of the elementary polynomials $e_1,\dots,e_n$ and $f_1,\dots,f_n$ of $x_i$ and $y_j$ respectively. 
\item The coefficient of $t^n$ in the polynomial $$g(t)=\prod_{1\leq i_1\leq\cdots\leq i_m\leq nm}(1+x_{i_1}\cdots x_{i_m}t)$$ is a symmetric polynomial in $x_i$ with coefficients in $\Z$. $P_{m,n}$ is precisely this polynomial written in terms of the elementary polynomials $e_1,\dots,e_n$ of $x_i$. 
\end{itemize}
In this case, we call $R$ a $\lambda$-ring. 
\end{defn}

Note that we do not require that the $\lambda^n$ are ring homomorphisms. 

\begin{defn}{Associated Formal Power Series}{} Let $R$ be a $\lambda$-ring. Define the associated formal power series to be the function $\lambda_t:R\to R[[t]]$ given by $$\lambda_t(r)=\sum_{k=0}^\infty\lambda^k(r)t^k$$ for all $r\in R$
\end{defn}

\begin{prp}{}{} Let $R$ be a $\lambda$-ring. Then the following are true regarding $\lambda_t(r)$. 
\begin{itemize}
\item $\lambda_t(1)=1+t$
\item $\lambda_t(0)=1$
\item $\lambda_t(r+s)=\lambda_t(r)\lambda_t(s)$
\item $\lambda_t(-r)=\lambda(r)^{-1}$
\end{itemize}
\end{prp}

\begin{prp}{}{} The ring $\Z$ has a unique $\lambda$-structure given by $$\lambda_t(n)=(1+t)^n$$
\end{prp}

\begin{prp}{}{} Let $R$ be a $\lambda$-ring. Then $R$ has characteristic $0$. 
\end{prp}

\begin{defn}{Dimension of an Element}{} Let $R$ be a $\lambda$-ring and let $r\in R$. We say that $r$ has dimension $n$ if $\deg(\lambda_t(r))=n$. In this case, we write $\dim(r)=n$. 
\end{defn}

\begin{prp}{}{} Let $R$ be a $\lambda$-ring. Then the following are true regarding the dimension of $n$. 
\begin{itemize}
\item $\dim(r+s)\leq\dim(r)+\dim(s)$ for all $r,s\in R$
\item If $r$ and $s$ both has dimension $1$, then so is $rs$. 
\end{itemize}
\end{prp}

\subsection{$\lambda$-Ring Homomorphisms and Ideals}
\begin{defn}{$\lambda$-Ring Homomorphisms}{} Let $R$ and $S$ be $\lambda$-rings. A $\lambda$-ring homomorphism from $R$ to $S$ is a ring homomorphism $f:R\to S$ such that $$\lambda^n\circ f=f\circ\lambda^n$$ for all $n\in\N$. 
\end{defn}

\begin{defn}{$\lambda$-Ideals}{} Let $R$ be a $\lambda$-ring. A $\lambda$-ideal of $R$ is an ideal $I$ of $R$ such that $$\lambda^n(i)\in I$$ for all $i\in I$ and $n\geq 1$. 
\end{defn}

TBA:$\lambda$-ideal and subring. Ker, Im, Quotient Product, Tensor, Inverse Limit are $\lambda$-rings

\begin{prp}{}{} Let $R$ be a $\lambda$-ring. Let $I=\langle z_i\;|\;i\in I\rangle$ be an ideal in $R$. Then $I$ is a $\lambda$-ideal if and only if $\lambda^n(z_i)\in I$ for all $n\geq 1$ and $i\in I$. 
\end{prp}

\begin{prp}{}{} Every $\lambda$-ring $R$ contains a $\lambda$-subring isomorphic to $\Z$. 
\end{prp}

\subsection{Augmented $\lambda$-Rings}
\begin{defn}{Augmented $\lambda$-Rings}{} Let $R$ be a $\lambda$-ring. We say that $R$ is an augmented $\lambda$-ring if it comes with a $\lambda$-homomorphism $$\varepsilon:R\to\Z$$ called the augmentation map. 
\end{defn}

TBA: tensor of augmented is augmented

\begin{prp}{}{} Let $R$ a $\lambda$-ring. Then $R$ is augmented if and only if there exists a $\lambda$-ideal $I$ such that $$R=\Z\oplus I$$ as abelian groups. 
\end{prp}

\subsection{Extending $\lambda$-Structures}
\begin{prp}{}{} Let $R$ be a $\lambda$-ring. Then there exists a unique $\lambda$-structure on $R[x]$ such that $\lambda_t(r)=1+rt$. Moreover, if $R$ is augmented, then so is $R[x]$ and $\varepsilon(r)=0$ or $1$. 
\end{prp}

\begin{prp}{}{} Let $R$ be a $\lambda$-ring. Then there exists a unique $\lambda$-structure on $R[[x]]$ such that $\lambda_t(r)=1+rt$. Moreover, if $R$ is augmented, then so is $R[[x]]$ and $\varepsilon(r)=0$ or $1$. 
\end{prp}

\subsection{Free $\lambda$-Rings}

\subsection{The Universal $\lambda$-Ring}

\subsection{Adams Operations}

\pagebreak
\section{Witt Vectors}
\subsection{Fundamentals of the Ring of Big Witt Vectors}
Prelim: Symm Functions, Lambda Rings, Category theory, Frobenius endomorphism (Galois), Rings and Modules, Kaehler differentials (commutative algebra 2)\\
Leads to: K theory\\
Books: Donald Yau: Lambda Rings

\begin{defn}{Truncation Sets}{} Let $S\subseteq\N$. We say that $S$ is a truncation set if for all $n\in S$ and $d|n$, then $d\in S$. For $n\in\N$ and $S$ a truncation set, define $$S/n=\{d\in\N\;|\;nd\in S\}$$
\end{defn}

For instance, $\N\setminus\{0\}$ is a truncation set. We will also use $\{1,\dots,n\}$. 

\begin{thm}{Dwork's Theorem}{} Let $R$ be a ring and let $S$ be a truncation set. Suppose that for all primes $p$, there exists a ring endomorphism $\sigma_p:R\to R$ such that $\sigma_p(r)\equiv r^p\;(\bmod\;pR)$ for some $s\in R$. Then the following are equivalent. 
\begin{itemize}
\item Every element $(b_i)_{i\in S}\in\prod_{i\in S} R$ has the form $$(b_i)_{i\in S}=(w_i(a))_{i\in S}$$ for some $a\in R$
\item For all primes $p$ and all $n\in S$ such that $p|n$, we have $$b_n\equiv\sigma_p(b_{n/p})\;(\bmod\;p^nR)$$ 
\end{itemize}
In this case, $a$ is unique, and $a_n$ depends solely on all the $b_k$ for $1\leq k\leq n$ and $k\in S$. 
\end{thm}

We wish to equip $\prod_{i\in S}R$ with a non-standard addition and multiplication to make it into a ring. 

\begin{prp}{}{} Consider the ring $R=\Z[x_i,y_i\;|\;i\in S]$. There exists unique polynomials $$\xi_n(x_1,\dots,x_n,y_1,\dots,y_n), \pi_n(x_1,\dots,x_n,y_1,\dots,y_n),\iota_n(x_1,\dots,x_n)$$ for $n\in S$ such that 
\begin{itemize}
\item $w_n(\xi_1,\dots,\xi_n)=w_n((x_i)_{i\in S})+w_n((y_i)_{i\in S})$
\item $w_n(\pi_1,\dots,\pi_n)=w_n((x_i)_{i\in S})\cdot w_n((y_i)_{i\in S})$
\item $w_n(\iota_1,\dots,\iota_n)=-w_n((x_i)_{i\in S})$
\end{itemize}
for all $n\in S$. 
\end{prp}

Note that the polynomials $\xi_n$, $\pi_n$ have variables $x_k$ and $y_k$ for $k\leq n$ and $k\in S$. This is similar for the variables of $\iota$. From now on, this will be the convention: For $S$ a truncation set, the sequence $a_1,\dots,a_n$ actually refers to the sequence $a_1,a_{d_1},\dots,a_{d_k},a_n$ where $1\leq d_1\leq\cdots\leq d_k\leq n$ and $d_1,\dots,d_k$ are all divisors of $n$. The result of this is that sequences in $\N$ are now restricted to $S$. 

\begin{defn}{The Ring of Truncated Witt Vector}{} Let $R$ be a ring. Let $S$ be a truncation set. Define the ring of big Witt vectors $W_S(R)$ of $R$ to consist of the following. 
\begin{itemize}
\item The underlying set $\prod_{i\in S}R$
\item Addition defined by $(a_n)_{n\in S}+(b_n)_{n\in S}=(\xi_n(a_1,\dots,a_n,b_1,\dots,b_n))_{n\in\N}$
\item Multiplication defined by $(a_n)_{n\in S}\times(b_n)_{n\in S}=(\pi_n(a_1,\dots,a_n,b_1,\dots,b_n))_{n\in\N}$
\end{itemize}
\end{defn}

\begin{thm}{}{} Let $R$ be a ring. Let $S$ be a truncation set. Then the ring of big Witt vectors $W_S(R)$ of $R$ is a ring with additive identity $(0,0,\dots)$ and multiplicative identity $(1,0,0,\dots)$. Moreover, for $(a_n)_{n\in S}\in W(R)$, its additive inverse is given by $(\iota_n(a_1,\dots,a_n))_{n\in\N}$.
\end{thm}

\begin{prp}{}{} Let $\phi:R\to R'$ be a ring homomorphism. Then the induced map $W_S(\phi):W_S(R)\to W_S(R')$ defined by $$W(\phi)((a_n)_{n\in S})=(\phi(a_n))_{n\in S}$$ is a ring homomorphism. 
\end{prp}

\begin{defn}{The Witt Functor}{} Define the Witt functor $W_S:\bold{Ring}\to\bold{Ring}$ to consist of the following data. 
\begin{itemize}
\item For each ring $R$, $W_S(R)$ is the ring of big Witt vectors
\item For a ring homomorphism $\phi:R\to R'$, $W_S(\phi):W_S(R)\to W_S(R')$ is the induced ring homomorphism defined by $$W_S(\phi)((a_n)_{n\in S})=(\phi(a_n))_{n\in S}$$
\end{itemize}
\end{defn}

\begin{prp}{}{} Let $S$ be a truncation set. The Witt functor is indeed a functor. 
\end{prp}

\begin{defn}{The Ghost Map}{} Let $R$ be a ring. Let $S$ be a truncation set. Define the ghost map to be the map $$w:W_S(R)\to\prod_{k\in S}R$$ by the formula $$w((a_n)_{n\in S})=(w_n(a_1,\dots,a_n))_{n\in S}$$
\end{defn}

Remember, by the sequence $a_1,\dots,a_n$ we mean the sequence $a_1,a_{d_1},\dots,a_{d_k},a_n$ where $1\leq d_1\leq\cdots\leq d_k\leq n$ and $d_1,\dots,d_k$ the complete collection of divisors of $n$. 

\begin{prp}{}{} Let $S$ be a truncation set. Then the following are true. 
\begin{itemize}
\item For each $n\in S$, the collection of maps $w_n:W_S(R)\to R$ for a ring $R$ defines a natural transformation $w_n:W_S\rightarrow\text{id}$. 
\item The collection of ghost maps $w_R:W_S(R)\to\prod_{k\in S}R$ for $R$ a ring defines a natural transformation $w:W_S\rightarrow (-)^S$. 
\end{itemize}
\end{prp}

\begin{prp}{}{} Let $S$ be a truncation set. The truncated Witt functor $W_S:\bold{Ring}\to\bold{Ring}$ is uniquely characterized by the following conditions. 
\begin{itemize}
\item The underlying set of $W_S(R)$ is given by $\prod_{k\in S}R$
\item For a ring homomorphism $\phi:R\to S$, $W(\phi):W(R)\to W(S)$ is the induced ring homomorphism defined by $$W(\phi)((a_n)_{n\in\N})=(\phi(a_n))_{n\in\N}$$
\item For each $n\in S$, $w_n:W_S(R)\to R$ defines a natural transformation $w_n:W\rightarrow\text{id}$
\end{itemize}
This means that if there is another functor $V$ satisfying the above, then $W$ and $V$ are naturally isomorphic. 
\end{prp}

Note that the above theorem implies that the ring structure on $\prod_{k\in S}R$ is unique under the above conditions. 

\subsection{Important Maps of Witt Vectors}
\begin{defn}{The Forgetful Map}{} Let $R$ be a ring. Let $T\subseteq S$ be truncation sets. Define the forgetful map $R_T^S:W_S(R)\to W_T(R)$ to be the ring homomorphism given by forgetting all elements $s\in S$ but $s\notin T$. 
\end{defn}

\begin{defn}{The $n$th Verschiebung Map}{} Let $R$ be a ring. Let $S$ be a truncation set. For $n\in\N$, define the $n$th Verschiebung map $V_n:W_{S/n}(R)\to W_S(R)$ by $$V_n((a_d)_{d\in S/n})_m=\begin{cases}
a_d & \text{ if } m=nd\\
0 & \text{otherwise}
\end{cases}$$
\end{defn}

Note that this is not a ring homomorphism. However, it is additive. 

\begin{lmm}{}{} Let $R$ be a ring. Let $S$ be a truncation set. Then for all $a,b\in W_{S/n}(R)$, we have that $$V_n(a+b)=V_n(a)+V_n(b)$$
\end{lmm}

\begin{defn}{Frobenius Map}{} Let $S$ be a truncation set. Let $R$ be a ring. Define the Frobenius map to be a natural ring homomorphism $F_n:W_S(R)\to W_{S/n}(R)$ such that the following diagram commutes: \\~\\
\adjustbox{scale=1.0,center}{\begin{tikzcd}
	{W_S(R)} & {\prod_{k\in S}R} \\
	{W_{S/n}(R)} & {\prod_{k\in S/n}R}
	\arrow["w", from=1-1, to=1-2]
	\arrow["{F_n}"', from=1-1, to=2-1]
	\arrow["{F_n^w}", from=1-2, to=2-2]
	\arrow["w"', from=2-1, to=2-2]
\end{tikzcd}}\\~\\
if it exists. 
\end{defn}

\begin{lmm}{}{} Let $S$ be a truncation set. Let $R$ be a ring. Then the Frobenius map exists and is unique. 
\end{lmm}

The following lemma relates this notion of Frobenius map to that in ring theory. 

\begin{lmm}{}{} Let $A$ be an $F_p$ algebra. Let $S$ be a truncation set. Let $\varphi_p:A\to A$ denote the Frobenius homomorphism given by $a\mapsto a^p$. Then $$F_p=R_{S/p}^S\circ W_S(\varphi):W_S(A)\to W_{S/p}(A)$$
\end{lmm}

\begin{defn}{The Teichmuller Representative}{} Let $R$ be a ring. Let $S$ be a truncation set. Define the Teichmuller representative to be the map $[-]_S:R\to W_S(R)$ defined by $$([a]_S)_n=\begin{cases}
a & \text{ if } n=1\\
0 &b \text{ otherwise }
\end{cases}$$
\end{defn}

The Teichmuller representative is in general not a ring homomorphism, but it is still multiplicative. 

\begin{lmm}{}{} Let $R$ be a ring. Let $S$ be a truncation set. The for all $a,b\in R$, we have that $$[ab]_S=[a]_S\cdot [b]_S$$
\end{lmm}

The three maps introduced are related as follows. 

\begin{prp}{}{} Let $R$ be a ring. Let $S$ be a truncated set. Then the following are true. 
\begin{itemize}
\item $r=\sum_{n\in S}V_n([r_n]_{S/n})$ for all $r\in W_S(R)$
\item $F_n(V_n(a))=na$ for all $a\in W_{S/n}(R)$
\item $r\cdot V_n(a)=V_n(F_n(r)\cdot a)$ for all $r\in W_S(R)$ and all $a\in W_{S/n}(R)$
\item $F_m\circ V_n=V_n\circ F_m$ if $\gcd(m,n)=1$
\end{itemize}
\end{prp}

The remaining section is dedicated to the example of $R=\Z$. 

\begin{prp}{}{} Let $S$ be a truncation set. Then the ring of big Witt vectors of $\Z$ is given by $$W_S(\Z)=\prod_{n\in S}\Z\cdot V_n([1]_{S/n})$$ with multiplication given by $$V_m([1]_{S/m})\cdot V_n([1]_{S/n})=\gcd(m,n)\cdot V_d([1]_{S/d})$$ and $d=\lcm(m,n)$. 
\end{prp}

\subsection{The Ring of p-Typical Witt Vectors}
For the ring of $p$-typical Witt vectors, we consider the truncation set $P=\{1,p,p^2,\dots\}\subseteq\N$ for a prime $p$. 

\begin{defn}{The Ring of p-Typical Witt Vectors}{} Let $R$ be a ring. Let $p$ be a prime. Let $P=\{1,p,p^2,\dots\}\subseteq\N$. Define the ring of $p$-typical Witt vectors to be $$W_p(R)=W_P(R)$$ Define the ring of $p$-typical Witt vectors of length $n$ to be $$W_n(R)=W_{\{1,p,\dots,p^{n-1}\}}(R)$$ when the prime $p$ is understood. 
\end{defn}

\begin{thm}{}{} Let $R$ be a ring. Let $p$ be a prime number. Let $S$ be a truncation set. Write $I(S)=\{k\in S\;|\;k\text{ does not divide }p\}$. Suppose that all $k\in I(S)$ are invertible in $R$. Then there is a decomposition $$W_S(R)=\prod_{k\in I(S)}W_S(R)\cdot e_k$$ where $$e_k=\prod_{t\in I(S)\setminus\{1\}}\left(\frac{1}{k}V_k([1]_{S/k})-\frac{1}{kt}V-{kt}([1]_{S/kt})\right)$$ Moreover, the composite map given by \\~\\
\adjustbox{scale=1.0,center}{\begin{tikzcd}
	{W_S(R)\cdot e_k} & {W_S(R)} & {W_{S/k}R} & {W_{S/k\cap P}(R)}
	\arrow[hook, from=1-1, to=1-2]
	\arrow["{F_k}", from=1-2, to=1-3]
	\arrow["{R_{S/k\cap P}^{S/k}}", from=1-3, to=1-4]
\end{tikzcd}}\\~\\
is an isomorphism. 
\end{thm}

\subsection{The $\lambda$-structure on W(R)}
\begin{lmm}{}{} Let $R$ be a ring. Then every $f\in\Lambda(R)$ can be written uniquely as $$f=\prod_{k=1}^\infty(1-(-1)^na_nt^n)$$
\end{lmm}

\begin{thm}{The Artin-Hasse Exponential}{} There is a natural isomorphism $E:\Lambda\rightarrow W$ given as follows. For a ring $R$, $E_R:\Lambda(R)\to W(R)$ is defined by $$E_R\left(\prod_{k=1}^\infty(1-(-1)^na_nt^n)\right)=(a_n)_{n\in\N}$$
\end{thm}

\begin{crl}{}{} Let $R$ be a ring. Then $W(R)$ has a canonical $\lambda$-structure inherited from $\Lambda(R)$. 
\end{crl}

TBA: The forgetful functor $U:\Lambda\bold{Ring}\to\bold{CRing}$ has a left adjoint $\text{Symm}$ and has a right adjoint $W$. 


\pagebreak
\section{Formal Group Laws}
\begin{defn}{Formal Group Laws}{} Let $R$ be a ring. A formal group law over $R$ is a power series $$f(x,y)\in R[[x,y]]$$ such that the following are true. 
\begin{itemize}
\item $f(x,0)=f(0,x)=x$
\item $f(x,y)=f(y,x)$
\item $f(x,f(y,z))=f(f(x,y),z)$
\end{itemize}
\end{defn}

\begin{defn}{The Formal Group Law Functor}{} Define the formal group law functor $$FGL:\bold{Ring}\to\bold{Set}$$ by the following data. 
\begin{itemize}
\item For each ring $R$, $FGL(R)$ is the set of all formal group laws over $R$
\item For each ring homomorphism $f:R\to S$, $FGL(f)$ sends each formal group law $\sum_{i,j=0}^\infty c_{i,j}x^iy^j$ over $R$ to the formal group law $\sum_{i,j=0}^\infty f(c_{i,j})x^iy^j$ over $S$. 
\end{itemize}
\end{defn}

\begin{defn}{The Lazard Ring of a Formal Group Law}{} Define the lazard ring by $$L=\frac{\Z[c_{i,j}]}{Q}$$ where $Q$ is the ideal generated as follows. Write $f=\sum_{i,j=0}^\infty c_{i,j}x^iy^j$. Then $Q$ is generated by the constraints on $c_{i,j}$ for which $f$ becomes a formal group law. 
\end{defn}

\begin{lmm}{}{} The Lazard ring $L=\Z[c_{i,j}]/Q$ has the structure of a graded ring where $c_{i,j}$ has degree $2(i+j-1)$. 
\end{lmm}

\begin{thm}{}{} The formal group law functor $FGL:\bold{Ring}\to\bold{Set}$ is representable $$FGL(R)\cong\Hom_\bold{Ring}(L,R)$$ There exists a universal element $f\in L$ such that the map $\Hom_\bold{Ring}(L,R)\to FGL(R)$ given by evaluation on $f$ is a bijection for any ring $R$. 
\end{thm}

\begin{thm}{}{} There is an isomorphism of the Lazard ring $$L\cong\Z[t_1,t_2,\dots]$$ where each $t_k$ has degree $2k$. 
\end{thm}


\pagebreak
\section{Calculus of Functors}
\subsection{Excisive Functors}
\begin{defn}{Homotopy Functors}{} Let $\mC,\mD$ be categories with a notion of weak equivalence. We say that a functor $F:\mC\to\mD$ is a homotopy functor if $F$ preserves weak equivalences. 
\end{defn}

\begin{defn}{n-Excisive Functors}{} Let $F$ be a homotopy functor. We say that $F$ is $n$-excisive if it takes strongly homotopy cocartesian $(n+1)$-cubes to homotopy cartesian $(n+1)$-cubes. 
\end{defn}

\subsection{The Taylor Tower}
\begin{defn}{Fiberwise Join}{} Let $X,Y,U$ be spaces. Let $f:X\to Y$ be a map. Define the fiberwise join of $X$ and $U$ along $f$ to be the space $$X\ast_YU=\hocolim(X\longleftarrow X\times U\longrightarrow Y\times U)$$
\end{defn}

\begin{lmm}{}{} Let $X,Y,U,V$ be spaces. Let $f:X\to Y$ be a map. Then there is a natural isomorphism $$(X\ast_YU)\ast_YV\cong X\ast_Y(U\ast V)$$
\end{lmm}

\begin{prp}{}{} Let $\mP(n)$ denote the category of posets. Let $X$ be a space over $Y$. Then the assignment $$U\mapsto X\ast_YU$$ defines an $n$-dimensional cubical diagram in $\bold{Top}$. Moreover, it is strongly cocartesian. 
\end{prp}

\begin{defn}{}{} Let $Y$ be a space. Let $F:\bold{Top}_Y\to\bold{Top}$ be a homotopy functor. Define the functor $$T_nF:\bold{Top}_Y\to\bold{Top}$$ to consist of the following data. 
\begin{itemize}
\item For each $X\in\bold{Top}$, consider the functor $\mX:\mP(n+1)\to\bold{Top}$ given by $U\mapsto F(X\ast_YU)$. Define $$T_nF(X)=\holim(\mX)=\holim_{U\in\mP(n+1)}(F(X\ast_YU))$$
\item For each $f:X\to Z$ a morphism of spaces over $Y$, define a map $T_nF(X)\to T_nF(Y)$ to be the map $$F(f\ast_Y\text{id})\circ\mX$$
\end{itemize}
\end{defn}

\begin{lmm}{}{} Let $Y$ be a space. Let $X$ be a space over $Y$. Let $F$ be a homotopy functor. Then $T_nF$ is a homotopy functor.   
\end{lmm}

\begin{prp}{}{} Let $F$ be a homotopy functor. Then there exists a natural map $t_nF:F\Rightarrow T_nF$ given by the canonical map of homotopy limits. Moreover, $t_nF$ is natural in the following sense. If $G$ is another homotopy functor and $\lambda:\mF\Rightarrow\mG$ is a natural transformation, then the following diagram commutes: \\~\\
\adjustbox{scale=1.0,center}{\begin{tikzcd}
	F & G \\
	{T_nF} & {T_nG}
	\arrow["\lambda", from=1-1, to=1-2]
	\arrow["{t_nF}"', from=1-1, to=2-1]
	\arrow["{t_nG}", from=1-2, to=2-2]
	\arrow["{T_n\lambda}"', from=2-1, to=2-2]
\end{tikzcd}}\\~\\
\end{prp}

\begin{defn}{}{} Let $Y$ be a space. Let $F:\bold{Top}_Y\to\bold{Top}$ be a homotopy functor. Define the functor $$P_nF:\bold{Top}_Y\to\bold{Top}$$ to consist of the following data. 
\begin{itemize}
\item For each space $X$ over $Y$, define $P_nF(X)$ to be the homotopy limit $$P_nF(X)=\text{holim}(F(X)\to T_nF(X)\to(T_n(T_nF))(X)\to\dots)$$
\item For each morphism $f:X\to Z$ of spaces over $Y$, define $P_nF(f):P_nF(X)\to P_nF(Z)$ to be the map ????
\end{itemize}
\end{defn}

\begin{lmm}{}{} Let $Y$ be a space. Let $X$ be a space over $Y$. Let $F$ be a homotopy functor. Then $P_nF$ is a homotopy functor.   
\end{lmm}

\begin{prp}{}{} Let $F$ be a homotopy functor. Then there exists a natural map $p_nF:F\Rightarrow P_nF$ given by the canonical map of homotopy limits. Moreover, $p_nF$ is natural in the following sense. If $G$ is another homotopy functor and $\lambda:\mF\Rightarrow\mG$ is a natural transformation, then the following diagram commutes: \\~\\
\adjustbox{scale=1.0,center}{\begin{tikzcd}
	F & G \\
	{P_nF} & {P_nG}
	\arrow["\lambda", from=1-1, to=1-2]
	\arrow["{p_nF}"', from=1-1, to=2-1]
	\arrow["{p_nG}", from=1-2, to=2-2]
	\arrow["{P_n\lambda}"', from=2-1, to=2-2]
\end{tikzcd}}\\~\\
\end{prp}

\begin{defn}{n-Reduced Functors}{} Let $F$ be a homotopy functor. We say that $F$ is $n$-reduced if $P_{n-1}F\simeq\ast$. 
\end{defn}

\begin{defn}{n-Homogenous Functor}{} Let $F$ be a homotopy functor. We say that $F$ is $n$-homogenous if $F$ is $n$-excisive and $n$-reduced. 
\end{defn}

\subsection{Linear Functors}
\begin{defn}{Linear Functors}{} Let $F$ be a homotopy functor. We say that $F$ is linear if $F$ is $1$-homogenous. Explicitly, this means that 
\begin{itemize}
\item $F$ sends homotopy pushouts to homotopy pullbacks
\item $F(X)$ is homotopy equivalent to $\ast$
\end{itemize}
\end{defn}

Let us consider the case $n=1$ and $Y=\ast$. Now $\mP_0(2)$ is the small category given in a diagram as follows: \\~\\
\adjustbox{scale=1.0,center}{\begin{tikzcd}
	& {\{1\}} \\
	{\{0\}} & {\{0,1\}}
	\arrow[from=1-2, to=2-2]
	\arrow[from=2-1, to=2-2]
\end{tikzcd}}\\~\\
Now $T_1F$ sends every space $X$ to the homotopy limit of the following diagram: \\~\\
\adjustbox{scale=1.0,center}{\begin{tikzcd}
	& {F(X\ast\{1\})} \\
	{F(X\ast\{0\})} & {F(X\ast\{0,1\})}
	\arrow[from=1-2, to=2-2]
	\arrow[from=2-1, to=2-2]
\end{tikzcd}}\\~\\
But we know that $X\ast\{0\}$ is the cone $CX$ and $X\ast\{0,1\}$ is the reduced suspension. This means that we can simplify the above diagram into \\~\\
\adjustbox{scale=1.0,center}{\begin{tikzcd}
	& {F(CX)} \\
	{F(CX)} & {F(\Sigma X)}
	\arrow[from=1-2, to=2-2]
	\arrow[from=2-1, to=2-2]
\end{tikzcd}}\\~\\
Now $CX\simeq\ast$ and $F$ is a reduced functor. Thus we can further simplify the diagram into \\~\\
\adjustbox{scale=1.0,center}{\begin{tikzcd}
	& \ast \\
	\ast & {F(\Sigma X)}
	\arrow[from=1-2, to=2-2]
	\arrow[from=2-1, to=2-2]
\end{tikzcd}}\\~\\
We recognize this as the homotopy pullback, and so $T_1F(X)\simeq\Omega F(\Sigma X)$. Now recall that $$P_1F(X)=\hocolim(F(X)\overset{t_1F(X)}{\longrightarrow} T_1F(X)\overset{t_1(T_1F)}{\longrightarrow}(T_1(T_1F))(X)\longrightarrow)$$ Again because we know that $T_1F(X)\simeq\Omega F(\Sigma X)$ and we care about everything only up to homotopy, we can write $P_1F$ as $$P_1F(X)=\hocolim(F(X)\overset{t_1F(X)}{\longrightarrow} \Omega F(\Sigma X)\overset{t_1(T_1F)}{\longrightarrow}\Omega(T_1F)(\Sigma X)\longrightarrow)$$ which further simplifies to $$P_1F(X)=\hocolim(F(X)\to\Omega F(\Sigma X)\to\Omega^2 F(\Sigma^2 X)\longrightarrow)$$ Thus in general, $$P_1F(X)=\hocolim_{n\to\infty}(\Omega^n F(\Sigma^n X))$$




We are considering the case $n=2$. Now $\mP_0(3)$ is the small category given in a diagram as follows: \\~\\
\adjustbox{scale=1.0,center}{\begin{tikzcd}
	& {\{2\}} \\
	&&& {\{1,2\}} \\
	{\{0,2\}} &&& {\{1\}} \\
	{\{0\}} && {\{0,1,2\}} \\
	&& {\{0,1\}}
	\arrow[from=1-2, to=2-4]
	\arrow[from=1-2, to=3-1]
	\arrow[from=2-4, to=4-3]
	\arrow[from=3-1, to=4-3]
	\arrow[from=3-4, to=2-4]
	\arrow[from=3-4, to=5-3]
	\arrow[from=4-1, to=3-1]
	\arrow[from=4-1, to=5-3]
	\arrow[from=5-3, to=4-3]
\end{tikzcd}}\\~\\
If we plug it into the definition of $T_nF$ and choose $Y=\ast$, we obtain a functor $$T_2F:\bold{Top}_\ast\to\bold{Top}$$ that consists of the following data. For each $X\in\bold{Top}$, $T_2F(X)$ is precisely the homotopy limit of the diagram \\~\\
\adjustbox{scale=1.0,center}{\begin{tikzcd}
	& {F(X\ast\{2\})} \\
	&&& {F(X\ast\{1,2\})} \\
	{F(X\ast\{0,2\})} &&& {F(X\ast\{1\})} \\
	{F(X\ast\{0\})} && {F(X\ast\{0,1,2\})} \\
	&& {F(X\ast\{0,1\})}
	\arrow[from=1-2, to=2-4]
	\arrow[from=1-2, to=3-1]
	\arrow[from=2-4, to=4-3]
	\arrow[from=3-1, to=4-3]
	\arrow[from=3-4, to=2-4]
	\arrow[from=3-4, to=5-3]
	\arrow[from=4-1, to=3-1]
	\arrow[from=4-1, to=5-3]
	\arrow[from=5-3, to=4-3]
\end{tikzcd}}\\~\\
which simplifies to the diagram: \\~\\
\adjustbox{scale=1.0,center}{\begin{tikzcd}
	& {F(CX)} \\
	&&& {F(\Sigma X)} \\
	{F(\Sigma X)} &&& {F(CX)} \\
	{F(CX)} && {F(X\ast\{0,1,2\})} \\
	&& {F(\Sigma X)}
	\arrow[from=1-2, to=2-4]
	\arrow[from=1-2, to=3-1]
	\arrow[from=2-4, to=4-3]
	\arrow[from=3-1, to=4-3]
	\arrow[from=3-4, to=2-4]
	\arrow[from=3-4, to=5-3]
	\arrow[from=4-1, to=3-1]
	\arrow[from=4-1, to=5-3]
	\arrow[from=5-3, to=4-3]
\end{tikzcd}}\\~\\
Now since $F$ is reduced and $CX\simeq\ast$, we can further simplify it into \\~\\
\adjustbox{scale=1.0,center}{\begin{tikzcd}
	& \ast \\
	&&& {F(\Sigma X)} \\
	{F(\Sigma X)} &&& \ast \\
	\ast && {F(X\ast\{0,1,2\})} \\
	&& {F(\Sigma X)}
	\arrow[from=1-2, to=2-4]
	\arrow[from=1-2, to=3-1]
	\arrow[from=2-4, to=4-3]
	\arrow[from=3-1, to=4-3]
	\arrow[from=3-4, to=2-4]
	\arrow[from=3-4, to=5-3]
	\arrow[from=4-1, to=3-1]
	\arrow[from=4-1, to=5-3]
	\arrow[from=5-3, to=4-3]
\end{tikzcd}}\\~\\ 
(what does the maps look like?)





\begin{defn}{The Category of Linear Functors}{} Define the category $$\mH_1(\mC,\mD)$$ of linear functors to be the full subcategory of $\mD^\mC$ consisting of linear functors. 
\end{defn}

\begin{thm}{}{} There is an equivalence of categories $$\mH_1(\bold{CGWH}_\ast,\bold{CGWH}_\ast)\cong\Omega\text{Sp}^\N(\bold{CGWH}_\ast)$$ given as follows. For a linear functor $F$, we associate to it the sequence of spaces $\{F(S^n)\;|\;n\in\N\}$, and this defines a spectra. 
\end{thm}

\subsection{Important Theorems}
Denote $\text{Sp}$ by the category of spectra. Define a map $\mL(\bold{Top}_\ast,\text{Sp})\to\text{Sp}$ that sends $F:\bold{Top}_\ast\to\text{Sp}$ to the spectra $F(S^0)$. Conversely, define a map $\text{Sp}\to\mL(\bold{Top}_\ast,\text{Sp})$ by sending each spectra $X$ to the functor $X\wedge -$. \\~\\

Now define a map $\mL(\bold{Top}_\ast)\to\text{Sp}$ as follows. For each $F:\bold{Top}_\ast\to\bold{Top}_\ast$, $F(S^n)$ is a collection of spaces indexed by $\N$. As for the bonding maps $F(S^n)\wedge S^1\to F(S^{n+1})$, this is defined as follows: 
\begin{enumerate}
\item Consider the identity map $\text{id}:X\wedge Y\to X\wedge Y$. 
\item By the smash-hom adjunction, this corresponds to a map $Y\to\text{Map}(X,X\wedge Y)$. 
\item Now composing with $F$ gives a map $$Y\to\text{Map}(X,X\wedge Y)\to\text{Map}(F(X),F(X\wedge Y))$$ (Why is the latter map continuous?)
\item By the smash-hom adjunction, this corresponds to a map $F(X)\wedge Y\to F(X\wedge Y)$
\item Taking $X=S^n$ and $Y=S^1$ gives the desired results. 
\end{enumerate}

At the same time, we can do the following: 

\begin{enumerate}
\item We begin by noticing that \\~\\
\adjustbox{scale=1.0,center}{\begin{tikzcd}
	X & \ast \\
	\ast & {\Sigma X}
	\arrow[from=1-1, to=1-2]
	\arrow[from=1-1, to=2-1]
	\arrow[from=1-2, to=2-2]
	\arrow[from=2-1, to=2-2]
\end{tikzcd}}\\~\\
is a homotopy pushout. 
\item Applying $F$ sends the homotopy pushout to a homotopy pullback: \\~\\
\adjustbox{scale=1.0,center}{\begin{tikzcd}
	{F(X)} & {F(\ast)} \\
	{F(\ast)} & {F(\Sigma X)}
	\arrow[from=1-1, to=1-2]
	\arrow[from=1-1, to=2-1]
	\arrow[from=1-2, to=2-2]
	\arrow[from=2-1, to=2-2]
\end{tikzcd}}\\~\\
\item Since $F$ is reduced, the diagram can be simplified into \\~\\
\adjustbox{scale=1.0,center}{\begin{tikzcd}
	{F(X)} & \ast \\
	\ast & {F(\Sigma X)}
	\arrow[from=1-1, to=1-2]
	\arrow[from=1-1, to=2-1]
	\arrow[from=1-2, to=2-2]
	\arrow[from=2-1, to=2-2]
\end{tikzcd}}\\~\\
\item Now recall that $\Omega(F(\Sigma X))$ is the homotopy pullback of $\ast\rightarrow F(\Sigma X)\leftarrow\ast$. 
\item We obtain maps $F(X)\to\text{holim}(\ast\rightarrow F(\Sigma X)\leftarrow\ast)$ and $\Omega F(\Sigma X)\to\text{holim}(\ast\rightarrow F(\Sigma X)\leftarrow\ast)$ which are both weak 
\end{enumerate}

Now take the first map constructed $f:F(X)\wedge Y\to F(X\wedge Y)$ and substitute $X$ and $Y$ with our wanted values to get a map $f:F(S^n)\wedge S^1\to F(S^{n+1})$. Adjunct it to the map $f:F(S^n)\to\Omega(F(S^{n+1}))$. Using the weak equivalences we obtained, we conclude that there is a diagram \\~\\
\adjustbox{scale=1.0,center}{\begin{tikzcd}
	{F(S^n)} && {\Omega F(S^{n+1})} \\
	& {\text{Holim}}
	\arrow["f", from=1-1, to=1-3]
	\arrow["\simeq"', from=1-1, to=2-2]
	\arrow["\simeq", from=1-3, to=2-2]
\end{tikzcd}}\\~\\
which we can prove to be commutative. By the two out of three property we easily conclude that $f$ is a weak equivalences. This is exactly where the bonding maps come from. 


We now have maps $\mL(\bold{Top}_\ast)\rightleftarrows\text{Sp}$. This actually gives an equivalence of categories. In fact, one can find out that it is a two step process: $$\mL(\bold{Top}_\ast)\rightleftarrows\mL(\bold{Top}_\ast,\text{Sp})\rightleftarrows\text{Sp}$$





\pagebreak
\section{Algebras and Coalgebras}
\subsection{Coalgebras}
There is a need to revisit the definition of an algebra (over a field)

\begin{prp}{}{} A vector space $V$ over a field $k$ is an algebra if and only if there is a following collection of data: 
\begin{itemize}
\item A $k$-linear map $m:V\otimes V\to V$ called the multiplication map
\item An $k$-linear map $u:k\to V$ called the unital map
\end{itemize}
such that the following two diagrams are commutative: \\~\\
\adjustbox{scale=1.0,center}{\begin{tikzcd}
	{V\otimes V\otimes V} & {V\otimes V} && {k\otimes V} & {V\otimes V} \\
	{V\otimes V} & V && V & {V\otimes k}
	\arrow["{\text{id}\otimes m}", from=1-1, to=1-2]
	\arrow["{m\otimes\text{id}}"', from=1-1, to=2-1]
	\arrow["m", from=1-2, to=2-2]
	\arrow["{u\otimes\text{id}}", from=1-4, to=1-5]
	\arrow["\cong"', from=1-4, to=2-4]
	\arrow["m"{description}, from=1-5, to=2-4]
	\arrow["m"', from=2-1, to=2-2]
	\arrow["{\text{id}\otimes u}"', from=2-5, to=1-5]
	\arrow["\cong", from=2-5, to=2-4]
\end{tikzcd}}\\~\\
where the unnamed maps is the canonical isomorphisms. 
\end{prp}

Evidently, the map $\mu$ gives a multiplicative structure for $V$ and $\Delta$ gives the unitary structure of an algebra. The diagram on the left then represent associativity of multiplication. Notice that such additional structure on $V$ formally lives in the category $\bold{Vect}_k$ of vector spaces over a fixed field $k$. \\~\\

Therefore we can formally dualize all arrows to obtain a new object. 

\begin{defn}{Coalgebra}{} Let $V$ be a vector space over a field $k$. We say that $V$ is a coalgebra over $k$ if there is a collection of data: 
\begin{itemize}
\item A $k$-linear map $\Delta:V\to V\otimes V$ called the comultiplication map
\item An $k$-linear map $\varepsilon:V\to k$ called the counital map
\end{itemize} 
such that the following diagrams are commutative: \\~\\
\adjustbox{scale=1.0,center}{\begin{tikzcd}
	{V\otimes V\otimes V} & {V\otimes V} && {k\otimes V} & {V\otimes V} \\
	{V\otimes V} & V && V & {V\otimes k}
	\arrow["{\varepsilon\otimes\Delta}"', from=1-2, to=1-1]
	\arrow["{\varepsilon\otimes\text{id}}"', from=1-5, to=1-4]
	\arrow["{\text{id}\otimes\varepsilon}", from=1-5, to=2-5]
	\arrow["{\Delta\otimes\varepsilon}", from=2-1, to=1-1]
	\arrow["\Delta"', from=2-2, to=1-2]
	\arrow["\Delta", from=2-2, to=2-1]
	\arrow["\cong", from=2-4, to=1-4]
	\arrow["\Delta"{description}, from=2-4, to=1-5]
	\arrow["\cong"', from=2-4, to=2-5]
\end{tikzcd}}\\~\\
where the unnamed maps is the canonical isomorphisms. 
\end{defn}

\begin{lmm}{}{} Every vector space $V$ over a field $k$ can be given the structure of a coalgebra where 
\begin{itemize}
\item $\Delta:V\to V\otimes V$ is defined by $\Delta(v)=v\otimes v$
\item $\varepsilon:V\to k$ is defined by $\varepsilon(v)=1_k$
\end{itemize}

\end{lmm}

We would like to formally invert the definitions of algebra homomorphisms in order to define coalgebra homomorphisms. 

\subsection{Bialgebras}
\begin{defn}{Bialgebras}{} Let $V$ be a vector space over a field $k$. We say that $V$ is a bialgebra if there is a collection of data: 
\begin{itemize}
\item A $k$-linear map $m:V\otimes V\to V$ called the multiplication map
\item An $k$-linear map $u:k\to V$ called the unital map
\item A $k$-linear map $\Delta:V\to V\otimes V$ called the comultiplication map
\item An $k$-linear map $\varepsilon:V\to k$ called the counital map
\end{itemize}
such that $(V,m,u)$ is an algebra over $k$ and $(V,\Delta,\varepsilon)$ is a coalgebra over $k$ and that the following diagrams are commutative: \\~\\
\adjustbox{scale=1.0,center}{\begin{tikzcd}
	{V\otimes V} & V & {V\otimes V} && k && k \\
	{V\otimes V\otimes V\otimes V} && {V\otimes V\otimes V\otimes V} &&& V \\
	\\
	{V\otimes V} && V &&& {k\otimes k\cong k} \\
	& {k\otimes k\cong k} &&& {V\otimes V} && V
	\arrow["m", from=1-1, to=1-2]
	\arrow["{\Delta\otimes\Delta}"', from=1-1, to=2-1]
	\arrow["\Delta", from=1-2, to=1-3]
	\arrow["{\text{id}}", from=1-5, to=1-7]
	\arrow["u"', from=1-5, to=2-6]
	\arrow["{\text{id}\otimes\tau\otimes\text{id}}"', from=2-1, to=2-3]
	\arrow["{m\otimes m}"{description}, from=2-3, to=1-3]
	\arrow["\varepsilon"', from=2-6, to=1-7]
	\arrow["m", from=4-1, to=4-3]
	\arrow["{\varepsilon\otimes\varepsilon}"', from=4-1, to=5-2]
	\arrow["\varepsilon", from=4-3, to=5-2]
	\arrow["{u\otimes u}"', from=4-6, to=5-5]
	\arrow["u", from=4-6, to=5-7]
	\arrow["\Delta", from=5-7, to=5-5]
\end{tikzcd}}\\~\\
where $\tau:V\otimes V\to V\otimes V$ is the commutativity map defined by $\tau(x\otimes y)=y\otimes x$. 
\end{defn}

\begin{thm}{}{} Let $V$ be a vector space over $k$. Suppose that $(V,m,u)$ is an algebra and $(V,\Delta,\varepsilon)$ is a coalgebra. Then the following conditions are equivalent. 
\begin{itemize}
\item $(V,m,u,\Delta,\varepsilon)$ is a bialgebra
\item $m:V\otimes V\to V$ and $u:k\to V$ are coalgebra homomorphisms
\item $\Delta:V\to V\otimes V$ and $\varepsilon:V\to k$ are algebra homomorphisms
\end{itemize}
\end{thm}

\pagebreak
\section{Hopf Algebras}
\subsection{Hopf Algebras}
\begin{defn}{Hopf Algebra}{} Let $(H,m,u,\Delta,\varepsilon)$ be a bialgebra. We say that $H$ is a Hopf algebra if there is a $k$-linear map $S:H\to H$ called the antipode such that the following diagram commutes: \\~\\
\adjustbox{scale=1.0,center}{\begin{tikzcd}
	& {H\otimes H} && {H\otimes H} \\
	H && k && H \\
	& {H\otimes H} && {H\otimes H}
	\arrow["{S\otimes\text{id}}", from=1-2, to=1-4]
	\arrow["m", from=1-4, to=2-5]
	\arrow["\Delta", from=2-1, to=1-2]
	\arrow["\varepsilon", from=2-1, to=2-3]
	\arrow["\Delta"', from=2-1, to=3-2]
	\arrow["u", from=2-3, to=2-5]
	\arrow["{\text{id}\otimes S}"', from=3-2, to=3-4]
	\arrow["m"', from=3-4, to=2-5]
\end{tikzcd}}\\~\\
\end{defn}

\pagebreak

\section{Differential Graded Algebra}
\subsection{Basic Definitions}
Similar to how chain complexes and cochain complexes are two names of the same object, we can define differential graded algebra using either the chain complex notation or cochain complex notation. For our purposes, we will use the cochain version. This means that differentials will go up in index. \\~\\

A differential graded algebra equips a graded algebra with a differential so that the algebra in the grading form a cochain complex. 

\begin{defn}{Differential Graded Algebra}{} A differential graded algebra is a graded algebra $A_\bullet$ together with a map $d:A\to A$ that has degree $1$ such that the following are true. 
\begin{itemize}
\item $d\circ d=0$
\item For $a\in A_n$ and $b\in A_m$, we have $d(ab)=(da)b+(-1)^na(db)$
\end{itemize}
\end{defn}

\begin{lmm}{}{} Let $(A,d)$ be a differential graded algebra. Then $(A,d)$ is also a cochain complex. 
\end{lmm}

Recall that a graded commutative algebra $A$ is a collection of algebra over some ring $A_0$, graded in $\N$ together with a multiplication $A_n\times A_m\to A_{m+n}$ such that $$a\cdot b=(-1)^{nm}b\cdot a$$ Such a multiplication rule is said to be graded commutative. 

\begin{defn}{Commutative Differential Graded Algebra}{} A differential graded algebra $A$ is said to be a commutative differential graded algebra (CDGA) if $A$ is also graded commutative. 
\end{defn}

We will often be concerned of differential graded algebra over a field $\Q$, $\R$ or $\C$. In particular this means that the algebra has the structure of a vector space. 

\pagebreak
\section{Introduction to Group Homology and Cohomology}
\subsection{G-Modules}
\begin{defn}{G-Modules}{} Let $G$ be a group. A $G$-module is an abelian group $A$ together with a group action of $G$ on $A$. 
\end{defn}

\begin{defn}{Morphisms of G-Modules}{} Let $G$ be a group. Let $M$ and $N$ be $G$-modules. A function $f:M\to N$ is said to be a $G$-module homomorphism if it is an equivariant group homomorphism. This means that $$f(g\cdot m)=g\cdot f(m)$$ for all $m\in M$ and $g\in G$. 
\end{defn}

\subsection{Invariants and Coinvariants}
\begin{defn}{The Group of Invariants}{} Let $G$ be a group and let $M$ be a $G$-module. Define the group of invariants of $G$ in $M$ to be the subgroup $$M^G=\{m\in M\;|\;gm=m\text{ for all }g\in G\}$$
\end{defn}

This is the largest subgroup of $M$ for which $G$ acts trivially. 

\begin{defn}{Functor of Invariants}{} Let $G$ be a group. Define the functor of invariants by $$(-)^G:{_G}\bold{Mod}\to\bold{Ab}$$ as follows. 
\begin{itemize}
\item For each $G$-module $M$, $M^G$ is the group of invariants
\item For each morphism $f:M\to N$ of $G$-modules, $f^G:M^G\to N^G$ is the restriction of $f$ to $M^G$. 
\end{itemize}
\end{defn}

\begin{thm}{}{} Let $G$ be a group. The functor of invariants $(-)^G:{_G}\bold{Mod}\to\bold{Ab}$ is left exact. 
\end{thm}

\begin{defn}{The Group of Coinvariants}{} Let $G$ be a group and let $M$ be a $G$-module. Define the group of coinvariants of $G$ in $M$ to be the quotient group $$M_G=\frac{M}{\langle gm-m\;|\;g\in G, m\in M\rangle}$$
\end{defn}

This is the largest quotient of $M$ for which $G$ acts trivially. 

\subsection{Group Cohomology and its Equivalent Forms}
\begin{defn}{The nth Cohomology Group}{} Let $G$ be a group. Define the $n$th cohomology group of $G$ with coefficients in a $G$-module $M$ to be $$H_n(G;M)=(L_n(-)_G)(M)$$ the $n$th left derived functor of $(-)_G:{_G\bold{Mod}}\to\bold{Ab}$. 
\end{defn}

\begin{thm}{}{} Let $G$ be a group and let $M$ be a $G$-module. Then there is an isomorphism $$H^n(G;M)\cong\text{Ext}^n_{\Z[G]}(\Z,M)$$ that is natural in $M$. 
\end{thm}

Recall that there are two descriptions of $\text{Ext}$ by considering it as a functor of the first or second variable. Since the above theorem exhibits an isomorphism that is natural in the second variable, let us consider $\text{Ext}$ as the right derived functor of the functor $\Hom_{\Z[G]}(-,M)$ applied to $\Z$ as a $\Z[G]$-module. 

\begin{prp}{}{} Let $G$ be a group and let $M$ be a $G$-module. Let $P_\bullet\to\Z$ be a projective resolution of $\Z$ with $\Z[G]$-modules. Then there is an isomorphism $$H^n(G;M)\cong H^n(\Hom_{\Z[G]}(P_\bullet,M))$$ that is natural in $M$. 
\end{prp}

For any group $G$, there is always the trivial choice of projective resolution. 
In the following lemma, we use the notation $(g_0,\dots,\hat{g_i},\dots,g_n)$ as a shorthand for writing the element in $G^n$ but with the $i$th term omitted. 

\begin{lmm}{}{} Let $G$ be a group. Then the cochain complex \\~\\
\adjustbox{scale=1,center}{\begin{tikzcd}
	\cdots & {\Z[G^{n+1}]} & {\Z[G^n]} & {\Z[G^{n-1}]} & \cdots & {\Z[G]} & \Z & 0
	\arrow[from=1-1, to=1-2]
	\arrow["{f_n}", from=1-2, to=1-3]
	\arrow["{f_{n-1}}", from=1-3, to=1-4]
	\arrow[from=1-4, to=1-5]
	\arrow[from=1-5, to=1-6]
	\arrow[from=1-6, to=1-7]
	\arrow[from=1-7, to=1-8]
\end{tikzcd}}\\~\\ where $f_n:\Z[G^{n+1}]\to\Z[G^n]$ is defined by $$(g_0,\dots,g_n)\mapsto\sum_{i=0}^n(-1)^i(g_0,\dots,\hat{g_i},\dots,g_n)$$ is a projective resolution of $\Z$ lying in ${_{\Z[G]}}\bold{Mod}$. 
\end{lmm}

Let $A$ be an $R$-algebra and let $M$ be an $A$-module. Recall that the bar resolution is defined to be the chain complex consisting of $M\otimes A^{\otimes n}$ for each $n\in\N$ together with the boundary maps defined by multiplying the $i$the element to the $i+1$th element. Now let $G$ be a group. By considering $\Z[G]$ as a $\Z$-algebra and that and ring is a module over itself, it makes sense to talk about the bar resolution of $\Z[G]$. 

\begin{thm}{}{} Let $G$ be a group. Consider the bar resolution \\~\\
\adjustbox{scale=1.0,center}{\begin{tikzcd}
	\cdots & {\Z[G^{n+1}]} & {\Z[G^n]} & {\Z[G^{n-1}]} & \cdots & \Z[G] & \Z & 0
	\arrow[from=1-1, to=1-2]
	\arrow[from=1-2, to=1-3]
	\arrow[from=1-3, to=1-4]
	\arrow[from=1-4, to=1-5]
	\arrow[from=1-5, to=1-6]
	\arrow[from=1-6, to=1-7]
	\arrow[from=1-7, to=1-8]
\end{tikzcd}}\\~\\
of $\Z[G]$. Then it is a free resolution, and hence a projective resolution of $\Z$ with $\Z[G]$-modules. 
\end{thm}

Thus, given a group $G$ and a $G$-module $M$, the group cohomology of $G$ with coefficients in $M$ can be thought of in the following way: 
\begin{itemize}
\item It is the right derived functor of the functor of invariants $(-)^G:{_G\bold{Mod}}\to\bold{Ab}$
\item It is the extension group $\text{Ext}_{\Z[G]}^n(\Z,M)$ (which is computable by the obvious projective resolution $\Z[G^\bullet]$, or the bar resolution)
\end{itemize}

\subsection{Group Homology and its Equivalent Forms}
\begin{defn}{The nth Cohomology Group}{} Let $G$ be a group. Define the $n$th cohomology group of $G$ with coefficients in a $G$-module $M$ to be $$H^n(G;M)=(R^n(-)^G)(M)$$ the $n$th right derived functor of $(-)^G:{_G\bold{Mod}}\to\bold{Ab}$. 
\end{defn}

\begin{thm}{}{} Let $G$ be a group and let $M$ be a $G$-module. Then there is an isomorphism $$H_n(G;M)\cong\text{Tor}_n^{\Z[G]}(\Z,M)$$ that is natural in $M$. 
\end{thm}

\subsection{Low Degree Interpretations}
\begin{thm}{}{} Let $G$ be a group and let $M$ be a $G$-module. Then there are natural isomorphisms $$H^0(G,M)=M^G\;\;\;\;\text{ and }\;\;\;H_0(G;M)=M_G$$
\end{thm}

\begin{thm}{}{} Let $G$ be a group and let $M$ be a $G$-module. Then there is an isomorphism $$H_1(G,M)\cong\frac{G}{[G,G]}=G_\text{ab}$$
\end{thm}

\begin{thm}{}{} Let $G$ be a group and let $M$ be a trivial $G$-module. Then there is a natural isomorphism $$H^1(G;M)=\frac{(\{f:G\to M\;|\;f(ab)=f(a)+af(b)\},+)}{\langle f:G\to M\;|\;f(g)=gm-m\text{ for some fixed }m\rangle}$$
\end{thm}

\begin{crl}{}{} Let $G$ be a group and let $M$ be a trivial $G$-module. Then there is a natural isomorphism $$H^1(G;M)\cong\Hom_\bold{Grp}(G,M)$$
\end{crl}
\pagebreak

\section{Hochschild Homology}
\subsection{Hochschild Homology}
\begin{defn}{Hochschild Complex}{} Let $M$ be an $R$-module. Define the Hoschild complex to be the chain complex $C(R,M)$ given as follows. \\~\\
\adjustbox{scale=0.95,center}{\begin{tikzcd}
	\cdots & {M\otimes R^{\otimes n+1}} & {M\otimes R^{\otimes n}} & {M\otimes R^{\otimes n-1}} & \cdots & {M\otimes R} & M & 0
	\arrow[from=1-1, to=1-2]
	\arrow["d", from=1-2, to=1-3]
	\arrow["d", from=1-3, to=1-4]
	\arrow[from=1-4, to=1-5]
	\arrow[from=1-5, to=1-6]
	\arrow[from=1-6, to=1-7]
	\arrow[from=1-7, to=1-8]
\end{tikzcd}}\\~\\
The map $d$ is defined by $d=\sum_{i=0}^n(-1)^id_i$ where $d_i:M\otimes R^{\otimes n}\to M\otimes R^{\otimes n-1}$ is given by the following formula. 
\begin{itemize}
\item If $i=0$, then $d_0(m\otimes r_1\otimes\cdots\otimes r_n)=mr_1\otimes r_2\otimes\cdots\otimes r_n$
\item If $i=n$, then $d_n(m\otimes r_1\otimes\cdots\otimes r_n)=r_nm\otimes r_1\otimes\cdots\otimes r_{n-1}$
\item Otherwise, then $d_i(m\otimes r_1\otimes\cdots\otimes r_n)=m\otimes r_1\otimes\cdots\otimes r_ir_{i+1}\otimes \cdots\otimes r_{n-1}$
\end{itemize}
\end{defn}

\begin{defn}{Hochschild Homology}{} Let $M$ be an $R$-module. Define the Hochschild homology of $M$ to be the homology groups of the Hochschild complex $C(R,M)$: $$H_n(R,M)=\frac{\ker(d:M\otimes R^{\otimes n}\to M\otimes R^{\otimes n-1})}{\im(d:M\otimes R^{\otimes n+1}\to M\otimes R^{\otimes n})}=H_n(C(R,M))$$ If $M=R$ then we simply write $$HH_n(R)=H_n(R,R)=H_n(C(R,R))$$
\end{defn}

TBA: Functoriality. 

\begin{prp}{}{} Let $A$ be an $R$-algebra. Then $HH_n(A)$ is a $Z(A)$-module. 
\end{prp}

\begin{prp}{}{} Let $A$ be an $R$-algebra. Then the following are true regarding the $0$th Hochschild homology. 
\begin{itemize}
\item Let $M$ be an $A$-module. Then $H_0(A,M)=\frac{M}{\{am-ma\;|\;a\in A, m\in M\}}$
\item The $0$th Hochschild homology of $A$ is given by $HH_0(A)=\frac{A}{[A,A]}$
\item If $A$ is commutative, then the $0$th Hochschild homology is given by $HH_0(A)=A$. 
\end{itemize}
\end{prp}

\begin{thm}{}{} Let $A$ be a commutative $R$-algebra. Then there is a canonical isomorphism $$HH_1(A)\cong\Omega_{A/R}^1$$
\end{thm}

\subsection{Bar Complex}
\begin{defn}{Enveloping Algebra}{} Let $A$ be an $R$-algebra. Define the enveloping algebra of $A$ to be $$A^e=A\otimes A^\text{op}$$
\end{defn}

\begin{prp}{}{} Let $A$ be an $R$-algebra. Then any $A,A$-bimodule $M$ equal to a left (right) $A^e$-module. 
\end{prp}

\begin{defn}{Bar Complex}{}
\end{defn}

\begin{prp}{}{} Let $A$ be an $R$-algebra. The bar complex of $A$ is a resolution of the $A$ viewed as an $A^e$-module. 
\end{prp}

\begin{thm}{}{} Let $A$ be an $R$-algebra that is projective as an $R$-module. If $M$ is an $A$-bimodule, then there is an isomorphism $$H_n(A,M)=\text{Tor}_n^{A^e}(M,A)$$
\end{thm}

\subsection{Relative Hochschild Homology}

\subsection{The Trace Map}
\begin{defn}{The Generalized Trace Map}{} Let $R$ be a ring and let $M$ be an $R$-module. Define the generalized trace map $$\text{tr}:M_r(M)\otimes M_r(A)^{\oplus n}\to M\otimes A^{\otimes n}$$ by the formula $$\text{tr}((m_{i,j})\otimes (a_{i,j})_1\otimes\cdots\otimes(a_{i,j})_n)=\sum_{0\leq i_0,\dots,i_n\leq r}m_{i_0,i_1}\otimes (a_{i_1,i_2})_1\otimes\cdots\otimes (a_{i_n,i_0})_n$$
\end{defn}

\begin{thm}{}{} The trace map defines a morphism of chain complex $$\text{tr}:C_\bullet(M_r(A),M_r(M))\to C_\bullet(A,M)$$
\end{thm}

\subsection{Morita Equivalence and Morita Invariance}
\begin{defn}{}{} Let $R$ and $S$ be rings. We say that $R$ and $S$ are Morita equivalent if there is an equivalence of categories $$\bold{Mod}_R\cong\bold{Mod}_S$$
\end{defn}

\begin{thm}{Morita Invariance for Matrices}{}
\end{thm}

\pagebreak
\section{Group Structures on Maps of Spaces}

Req: AT3

$H$-spaces is a natural generalization of topological groups in the direction of homotopy theory. 

\begin{defn}{$H$-Spaces}{} Let $(X,x_0)$ be a pointed space. Let $\mu:(X,x_0)\times(X,x_0)\to(X,x_0)$ be a map. Let $e:(X,x_0)\to(X,x_0)$ be the constant map $x\mapsto x_0$. We say that $(X,x_0,\mu)$ is an $H$-space if the following diagram: \\~\\
\adjustbox{scale=0.95,center}{\begin{tikzcd}
	X & {X\times X} \\
	{X\times X} & X
	\arrow["{(e,\text{id}_X)}", from=1-1, to=1-2]
	\arrow["{(\text{id}_X,e)}"', from=1-1, to=2-1]
	\arrow["{\text{id}_X}"{description}, from=1-1, to=2-2]
	\arrow["\mu", from=1-2, to=2-2]
	\arrow["\mu"', from=2-1, to=2-2]
\end{tikzcd}}\\~\\
is commutative up to homotopy. The map $\mu$ is called $H$-multiplication. 
\end{defn}

\begin{defn}{$H$-Associative Spaces}{} Let $(X,x_0,\mu)$ be an $H$-space. We say that $(X,x_0,\mu)$ is an $H$-associative space if the following diagram: \\~\\
\adjustbox{scale=0.95,center}{\begin{tikzcd}
	{X\times X\times X} & {X\times X} \\
	{X\times X} & X
	\arrow["{\mu\times\text{id}_X}", from=1-1, to=1-2]
	\arrow["{\text{id}_X\times\mu}"', from=1-1, to=2-1]
	\arrow["\mu", from=1-2, to=2-2]
	\arrow["\mu"', from=2-1, to=2-2]
\end{tikzcd}}\\~\\
is commutative up to homotopy. 
\end{defn}

\begin{defn}{$H$-Group}{} Let $(X,x_0,\mu)$ be an $H$-space. Let $j:(X,x_0)\to(X,x_0)$ be a map. We say that $(X,x_0,\mu,j)$ is an $H$-group if the following diagram: \\~\\
\adjustbox{scale=0.95,center}{\begin{tikzcd}
	X & {X\times X} \\
	{X\times X} & X
	\arrow["{(j,\text{id}_X)}", from=1-1, to=1-2]
	\arrow["{(\text{id}_X,j)}"', from=1-1, to=2-1]
	\arrow["e"{description}, from=1-1, to=2-2]
	\arrow["\mu", from=1-2, to=2-2]
	\arrow["\mu"', from=2-1, to=2-2]
\end{tikzcd}}\\~\\
is commutative up to homotopy. The map $j$ is called $H$-inverse. 
\end{defn}

\begin{eg}{}{} Let $X$ be a pointed space. Then the loopspace $\Omega X$ is an $H$-group. 
\end{eg}

\begin{defn}{$H$-Abelian}{} Let $(X,x_0,\mu,j)$ be an $H$-group. Let $T:(X,x_0)\times(X,x_0)\to(X,x_0)$ be the map $T(x,y)=T(y,x)$. We say that $(X,x_0,\mu,j)$ is an $H$-abelian if the following diagram: \\~\\
\adjustbox{scale=0.95,center}{\begin{tikzcd}
	{X\times X} & {X\times X} \\
	& X
	\arrow["T", from=1-1, to=1-2]
	\arrow["\mu"', from=1-1, to=2-2]
	\arrow["\mu", from=1-2, to=2-2]
\end{tikzcd}}\\~\\
is commutative up to homotopy. 
\end{defn}

\begin{defn}{Natural Group Structure}{} Let $(X,x_0)$ be pointed spaces. We say that $[Z,X]_\ast$ has a natural group structure for all spaces $(Z,z_0)$ if the following are true. 
\begin{itemize}
\item $[Z,X]_\ast$ has a group structure such that the constant map $[e]$ is the identity of the group. 
\item For every map $f:A\to B$, the induced function $$f^\ast:[B,X]_\ast\to[A,X]_\ast$$ is a group homomorphism. 
\end{itemize}
\end{defn}


\end{document}