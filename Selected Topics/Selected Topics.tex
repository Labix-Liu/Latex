\documentclass[a4paper]{article}

%=========================================
% Packages
%=========================================
\usepackage{mathtools}
\usepackage{amsfonts}
\usepackage{amsmath}
\usepackage{amssymb}
\usepackage{amsthm}
\usepackage[a4paper, total={6in, 8in}, margin=1in]{geometry}
\usepackage[utf8]{inputenc}
\usepackage{fancyhdr}
\usepackage[utf8]{inputenc}
\usepackage{graphicx}
\usepackage{physics}
\usepackage[listings]{tcolorbox}
\usepackage{hyperref}
\usepackage{tikz-cd}
\usepackage{adjustbox}
\usepackage{enumitem}
\usepackage[font=small,labelfont=bf]{caption}
\usepackage{subcaption}
\usepackage{wrapfig}
\usepackage{makecell}



\raggedright

\usetikzlibrary{arrows.meta}

\DeclarePairedDelimiter\ceil{\lceil}{\rceil}
\DeclarePairedDelimiter\floor{\lfloor}{\rfloor}

%=========================================
% Fonts
%=========================================
\usepackage{tgpagella}
\usepackage[T1]{fontenc}


%=========================================
% Custom Math Operators
%=========================================
\DeclareMathOperator{\adj}{adj}
\DeclareMathOperator{\im}{im}
\DeclareMathOperator{\nullity}{nullity}
\DeclareMathOperator{\sign}{sign}
\DeclareMathOperator{\dom}{dom}
\DeclareMathOperator{\lcm}{lcm}
\DeclareMathOperator{\ran}{ran}
\DeclareMathOperator{\ext}{Ext}
\DeclareMathOperator{\dist}{dist}
\DeclareMathOperator{\diam}{diam}
\DeclareMathOperator{\aut}{Aut}
\DeclareMathOperator{\inn}{Inn}
\DeclareMathOperator{\syl}{Syl}
\DeclareMathOperator{\edo}{End}
\DeclareMathOperator{\cov}{Cov}
\DeclareMathOperator{\vari}{Var}
\DeclareMathOperator{\cha}{char}
\DeclareMathOperator{\Span}{span}
\DeclareMathOperator{\ord}{ord}
\DeclareMathOperator{\res}{res}
\DeclareMathOperator{\Hom}{Hom}
\DeclareMathOperator{\Mor}{Mor}
\DeclareMathOperator{\coker}{coker}
\DeclareMathOperator{\Obj}{Obj}
\DeclareMathOperator{\id}{id}
\DeclareMathOperator{\GL}{GL}
\DeclareMathOperator*{\colim}{colim}

%=========================================
% Custom Commands (Shortcuts)
%=========================================
\newcommand{\CP}{\mathbb{CP}}
\newcommand{\GG}{\mathbb{G}}
\newcommand{\F}{\mathbb{F}}
\newcommand{\N}{\mathbb{N}}
\newcommand{\Q}{\mathbb{Q}}
\newcommand{\R}{\mathbb{R}}
\newcommand{\C}{\mathbb{C}}
\newcommand{\E}{\mathbb{E}}
\newcommand{\Prj}{\mathbb{P}}
\newcommand{\RP}{\mathbb{RP}}
\newcommand{\T}{\mathbb{T}}
\newcommand{\Z}{\mathbb{Z}}
\newcommand{\A}{\mathbb{A}}
\renewcommand{\H}{\mathbb{H}}
\newcommand{\K}{\mathbb{K}}

\newcommand{\mA}{\mathcal{A}}
\newcommand{\mB}{\mathcal{B}}
\newcommand{\mC}{\mathcal{C}}
\newcommand{\mD}{\mathcal{D}}
\newcommand{\mE}{\mathcal{E}}
\newcommand{\mF}{\mathcal{F}}
\newcommand{\mG}{\mathcal{G}}
\newcommand{\mH}{\mathcal{H}}
\newcommand{\mI}{\mathcal{I}}
\newcommand{\mJ}{\mathcal{J}}
\newcommand{\mK}{\mathcal{K}}
\newcommand{\mL}{\mathcal{L}}
\newcommand{\mM}{\mathcal{M}}
\newcommand{\mO}{\mathcal{O}}
\newcommand{\mP}{\mathcal{P}}
\newcommand{\mS}{\mathcal{S}}
\newcommand{\mT}{\mathcal{T}}
\newcommand{\mV}{\mathcal{V}}
\newcommand{\mW}{\mathcal{W}}

%=========================================
% Colours!!!
%=========================================
\definecolor{LightBlue}{HTML}{2D64A6}
\definecolor{ForestGreen}{HTML}{4BA150}
\definecolor{DarkBlue}{HTML}{000080}
\definecolor{LightPurple}{HTML}{cc99ff}
\definecolor{LightOrange}{HTML}{ffc34d}
\definecolor{Buff}{HTML}{DDAE7E}
\definecolor{Sunset}{HTML}{F2C57C}
\definecolor{Wenge}{HTML}{584B53}
\definecolor{Coolgray}{HTML}{9098CB}
\definecolor{Lavender}{HTML}{D6E3F8}
\definecolor{Glaucous}{HTML}{828BC4}
\definecolor{Mauve}{HTML}{C7A8F0}
\definecolor{Darkred}{HTML}{880808}
\definecolor{Beaver}{HTML}{9A8873}
\definecolor{UltraViolet}{HTML}{52489C}



%=========================================
% Theorem Environment
%=========================================
\tcbuselibrary{listings, theorems, breakable, skins}

\newtcbtheorem[number within = subsection]{thm}{Theorem}%
{	colback=Buff!3, 
	colframe=Buff, 
	fonttitle=\bfseries, 
	breakable, 
	enhanced jigsaw, 
	halign=left
}{thm}

\newtcbtheorem[number within=subsection, use counter from=thm]{defn}{Definition}%
{  colback=cyan!1,
    colframe=cyan!50!black,
	fonttitle=\bfseries, breakable, 
	enhanced jigsaw, 
	halign=left
}{defn}

\newtcbtheorem[number within=subsection, use counter from=thm]{axm}{Axiom}%
{	colback=red!5, 
	colframe=Darkred, 
	fonttitle=\bfseries, 
	breakable, 
	enhanced jigsaw, 
	halign=left
}{axm}

\newtcbtheorem[number within=subsection, use counter from=thm]{prp}{Proposition}%
{	colback=LightBlue!3, 
	colframe=Glaucous, 
	fonttitle=\bfseries, 
	breakable, 
	enhanced jigsaw, 
	halign=left
}{prp}

\newtcbtheorem[number within=subsection, use counter from=thm]{lmm}{Lemma}%
{	colback=LightBlue!3, 
	colframe=LightBlue!60, 
	fonttitle=\bfseries, 
	breakable, 
	enhanced jigsaw, 
	halign=left
}{lmm}

\newtcbtheorem[number within=subsection, use counter from=thm]{crl}{Corollary}%
{	colback=LightBlue!3, 
	colframe=LightBlue!60, 
	fonttitle=\bfseries, 
	breakable, 
	enhanced jigsaw, 
	halign=left
}{crl}

\newtcbtheorem[number within=subsection, use counter from=thm]{eg}{Example}%
{	colback=Beaver!5, 
	colframe=Beaver, 
	fonttitle=\bfseries, 
	breakable, 
	enhanced jigsaw, 
	halign=left
}{eg}

\newtcbtheorem[number within=subsection, use counter from=thm]{ex}{Exercise}%
{	colback=Beaver!5, 
	colframe=Beaver, 
	fonttitle=\bfseries, 
	breakable, 
	enhanced jigsaw, 
	halign=left
}{ex}

\newtcbtheorem[number within=subsection, use counter from=thm]{alg}{Algorithm}%
{	colback=UltraViolet!5, 
	colframe=UltraViolet, 
	fonttitle=\bfseries, 
	breakable, 
	enhanced jigsaw, 
	halign=left
}{alg}




%=========================================
% Hyperlinks
%=========================================
\hypersetup{
    colorlinks=true, %set true if you want colored links
    linktoc=all,     %set to all if you want both sections and subsections linked
    linkcolor=DarkBlue,  %choose some color if you want links to stand out
}


\pagestyle{fancy}
\fancyhf{}
\rhead{Labix}
\lhead{Cohomology of Schemes}
\rfoot{\thepage}

\title{Cohomology of Schemes}

\author{Labix}

\date{\today}
\begin{document}
\maketitle
\begin{abstract}
\end{abstract}

References: 
\pagebreak
\tableofcontents

\pagebreak
\section{Symmetric Polynomials}
\subsection{Symmetric Polynomials}
The theory of symmetric functions are important in combinatorics, representation theory, Galois theory and the theory of $\lambda$-rings. \\

Requirements: Groups and Rings\\
Books: Donald Yau: Lambda Rings

\begin{defn}{Symmetric Group Action on Polynomial Rings}{} Let $R$ be a ring. Define a group action of $S_n$ on $R[x_1,\dots,x_n]$ by $$\sigma\cdot f(x_1,\dots,x_n)=f(x_{\sigma(1)},\dots,x_{\sigma(n)})$$
\end{defn}

It is easy to check that this defines a group action. 

\begin{defn}{Symmetric Polynomials}{} Let $R$ be a ring. We say that a polynomial $f\in R[x_1,\dots,x_n]$ is symmetric if $$\sigma\cdot f=f$$ for all $\sigma\in S_n$. 
\end{defn}

\begin{defn}{The Ring of Symmetric Polynomials}{} Let $R$ be a ring. Define the ring of symmetric polynomials in $n$ variables over $R$ to be the set $$\Sigma=\{f\in R[x_1,\dots,x_n]\;|\;\sigma\text{ is a symmetric polynomial }\}$$
\end{defn}

\begin{defn}{Elementary Symmetric Polynomials}{} Let $R$ be a ring. Define the elementary symmetric polynomials to be the elements $s_1,\dots,s_n\in R[x_1,\dots,x_n]$ given by the formula $$s_k(x_1,\dots,x_n)=\sum_{1\leq i_1\leq\cdots\leq i_k\leq n}x_{i_1}\cdots x_{i_k}$$
\end{defn}

\begin{thm}{The Fundamental Theorem of Symmetric Polynomials}{} Let $R$ be a ring. Then $s_1,\dots,s_n$ are algebraically independent over $R$. Moreover, $$\Sigma=R[s_1,\dots,s_n]$$
\end{thm}

\pagebreak
\section{$\lambda$-Rings}
\subsection{$\lambda$-Rings}
Complex representation of a group is a $\lambda$-ring. Topological $K$ theory is a $\lambda$-ring. \\

Requirements: Category Theory, Groups and Rings, Symmetric Functions\\
Books: Donald Yau: Lambda Rings\\

We need the theory of symmetric polynomials before defining $\lambda$-structures. 

\begin{defn}{$\lambda$-Structures}{} Let $R$ be a commutative ring. A $\lambda$-structure on $R$ consists of a sequence of maps $\lambda^n:R\to R$ for $n\geq 0$ such that the following are true. 
\begin{itemize}
\item $\lambda^0(r)=1$ for all $r\in R$
\item $\lambda^1=\text{id}_R$
\item $\lambda^n(1)=0$ for all $n\geq 2$
\item $\lambda^n(r+s)=\sum_{k=0}^n\lambda^k(r)\lambda^{n-k}(s)$ for all $r,s\in R$
\item $\lambda^n(rs)=P_n(\lambda^1(r),\dots,\lambda^n(r),\lambda^1(s),\dots,\lambda^n(s))$ for all $r,s\in R$
\item $\lambda^m(\lambda^n(r))=P_{m,n}(\lambda^1(r),\dots,\lambda^{mn}(r))$ for all $r\in R$
\end{itemize}
Here $P_n$ and $P_{m,n}$ are defined as follows. 
\begin{itemize}
\item The coefficient of $t^n$ in the polynomial $$h(t)=\prod_{i,j=1}^n(1+x_iy_jt)$$ is a symmetric polynomial in $x_i$ and $y_j$ with coefficients in $\Z$. $P_n$ is precisely this polynomial written in terms of the elementary polynomials $e_1,\dots,e_n$ and $f_1,\dots,f_n$ of $x_i$ and $y_j$ respectively. 
\item The coefficient of $t^n$ in the polynomial $$g(t)=\prod_{1\leq i_1\leq\cdots\leq i_m\leq nm}(1+x_{i_1}\cdots x_{i_m}t)$$ is a symmetric polynomial in $x_i$ with coefficients in $\Z$. $P_{m,n}$ is precisely this polynomial written in terms of the elementary polynomials $e_1,\dots,e_n$ of $x_i$. 
\end{itemize}
In this case, we call $R$ a $\lambda$-ring. 
\end{defn}

Note that we do not require that the $\lambda^n$ are ring homomorphisms. 

\begin{defn}{Associated Formal Power Series}{} Let $R$ be a $\lambda$-ring. Define the associated formal power series to be the function $\lambda_t:R\to R[[t]]$ given by $$\lambda_t(r)=\sum_{k=0}^\infty\lambda^k(r)t^k$$ for all $r\in R$
\end{defn}

\begin{prp}{}{} Let $R$ be a $\lambda$-ring. Then the following are true regarding $\lambda_t(r)$. 
\begin{itemize}
\item $\lambda_t(1)=1+t$
\item $\lambda_t(0)=1$
\item $\lambda_t(r+s)=\lambda_t(r)\lambda_t(s)$
\item $\lambda_t(-r)=\lambda(r)^{-1}$
\end{itemize}
\end{prp}

\begin{prp}{}{} The ring $\Z$ has a unique $\lambda$-structure given by $$\lambda_t(n)=(1+t)^n$$
\end{prp}

\begin{prp}{}{} Let $R$ be a $\lambda$-ring. Then $R$ has characteristic $0$. 
\end{prp}

\begin{defn}{Dimension of an Element}{} Let $R$ be a $\lambda$-ring and let $r\in R$. We say that $r$ has dimension $n$ if $\deg(\lambda_t(r))=n$. In this case, we write $\dim(r)=n$. 
\end{defn}

\begin{prp}{}{} Let $R$ be a $\lambda$-ring. Then the following are true regarding the dimension of $n$. 
\begin{itemize}
\item $\dim(r+s)\leq\dim(r)+\dim(s)$ for all $r,s\in R$
\item If $r$ and $s$ both has dimension $1$, then so is $rs$. 
\end{itemize}
\end{prp}

\subsection{$\lambda$-Ring Homomorphisms and Ideals}
\begin{defn}{$\lambda$-Ring Homomorphisms}{} Let $R$ and $S$ be $\lambda$-rings. A $\lambda$-ring homomorphism from $R$ to $S$ is a ring homomorphism $f:R\to S$ such that $$\lambda^n\circ f=f\circ\lambda^n$$ for all $n\in\N$. 
\end{defn}

\begin{defn}{$\lambda$-Ideals}{} Let $R$ be a $\lambda$-ring. A $\lambda$-ideal of $R$ is an ideal $I$ of $R$ such that $$\lambda^n(i)\in I$$ for all $i\in I$ and $n\geq 1$. 
\end{defn}

TBA:$\lambda$-ideal and subring. Ker, Im, Quotient Product, Tensor, Inverse Limit are $\lambda$-rings

\begin{prp}{}{} Let $R$ be a $\lambda$-ring. Let $I=\langle z_i\;|\;i\in I\rangle$ be an ideal in $R$. Then $I$ is a $\lambda$-ideal if and only if $\lambda^n(z_i)\in I$ for all $n\geq 1$ and $i\in I$. 
\end{prp}

\begin{prp}{}{} Every $\lambda$-ring $R$ contains a $\lambda$-subring isomorphic to $\Z$. 
\end{prp}

\subsection{Augmented $\lambda$-Rings}
\begin{defn}{Augmented $\lambda$-Rings}{} Let $R$ be a $\lambda$-ring. We say that $R$ is an augmented $\lambda$-ring if it comes with a $\lambda$-homomorphism $$\varepsilon:R\to\Z$$ called the augmentation map. 
\end{defn}

TBA: tensor of augmented is augmented

\begin{prp}{}{} Let $R$ a $\lambda$-ring. Then $R$ is augmented if and only if there exists a $\lambda$-ideal $I$ such that $$R=\Z\oplus I$$ as abelian groups. 
\end{prp}

\subsection{Extending $\lambda$-Structures}
\begin{prp}{}{} Let $R$ be a $\lambda$-ring. Then there exists a unique $\lambda$-structure on $R[x]$ such that $\lambda_t(r)=1+rt$. Moreover, if $R$ is augmented, then so is $R[x]$ and $\varepsilon(r)=0$ or $1$. 
\end{prp}

\begin{prp}{}{} Let $R$ be a $\lambda$-ring. Then there exists a unique $\lambda$-structure on $R[[x]]$ such that $\lambda_t(r)=1+rt$. Moreover, if $R$ is augmented, then so is $R[[x]]$ and $\varepsilon(r)=0$ or $1$. 
\end{prp}

\subsection{Free $\lambda$-Rings}

\subsection{The Universal $\lambda$-Ring}

\subsection{Adams Operations}

\pagebreak
\section{Witt Vectors}
\subsection{Fundamentals of the Ring of Big Witt Vectors}
Prelim: Symm Functions, Lambda Rings, Category theory, Frobenius endomorphism (Galois), Rings and Modules, Kaehler differentials (commutative algebra 2)\\
Leads to: K theory\\
Books: Donald Yau: Lambda Rings

\begin{defn}{Truncation Sets}{} Let $S\subseteq\N$. We say that $S$ is a truncation set if for all $n\in S$ and $d|n$, then $d\in S$. For $n\in\N$ and $S$ a truncation set, define $$S/n=\{d\in\N\;|\;nd\in S\}$$
\end{defn}

For instance, $\N\setminus\{0\}$ is a truncation set. We will also use $\{1,\dots,n\}$. 

\begin{thm}{Dwork's Theorem}{} Let $R$ be a ring and let $S$ be a truncation set. Suppose that for all primes $p$, there exists a ring endomorphism $\sigma_p:R\to R$ such that $\sigma_p(r)\equiv r^p\;(\bmod\;pR)$ for some $s\in R$. Then the following are equivalent. 
\begin{itemize}
\item Every element $(b_i)_{i\in S}\in\prod_{i\in S} R$ has the form $$(b_i)_{i\in S}=(w_i(a))_{i\in S}$$ for some $a\in R$
\item For all primes $p$ and all $n\in S$ such that $p|n$, we have $$b_n\equiv\sigma_p(b_{n/p})\;(\bmod\;p^nR)$$ 
\end{itemize}
In this case, $a$ is unique, and $a_n$ depends solely on all the $b_k$ for $1\leq k\leq n$ and $k\in S$. 
\end{thm}

We wish to equip $\prod_{i\in S}R$ with a non-standard addition and multiplication to make it into a ring. 

\begin{prp}{}{} Consider the ring $R=\Z[x_i,y_i\;|\;i\in S]$. There exists unique polynomials $$\xi_n(x_1,\dots,x_n,y_1,\dots,y_n), \pi_n(x_1,\dots,x_n,y_1,\dots,y_n),\iota_n(x_1,\dots,x_n)$$ for $n\in S$ such that 
\begin{itemize}
\item $w_n(\xi_1,\dots,\xi_n)=w_n((x_i)_{i\in S})+w_n((y_i)_{i\in S})$
\item $w_n(\pi_1,\dots,\pi_n)=w_n((x_i)_{i\in S})\cdot w_n((y_i)_{i\in S})$
\item $w_n(\iota_1,\dots,\iota_n)=-w_n((x_i)_{i\in S})$
\end{itemize}
for all $n\in S$. 
\end{prp}

Note that the polynomials $\xi_n$, $\pi_n$ have variables $x_k$ and $y_k$ for $k\leq n$ and $k\in S$. This is similar for the variables of $\iota$. From now on, this will be the convention: For $S$ a truncation set, the sequence $a_1,\dots,a_n$ actually refers to the sequence $a_1,a_{d_1},\dots,a_{d_k},a_n$ where $1\leq d_1\leq\cdots\leq d_k\leq n$ and $d_1,\dots,d_k$ are all divisors of $n$. The result of this is that sequences in $\N$ are now restricted to $S$. 

\begin{defn}{The Ring of Truncated Witt Vector}{} Let $R$ be a ring. Let $S$ be a truncation set. Define the ring of big Witt vectors $W_S(R)$ of $R$ to consist of the following. 
\begin{itemize}
\item The underlying set $\prod_{i\in S}R$
\item Addition defined by $(a_n)_{n\in S}+(b_n)_{n\in S}=(\xi_n(a_1,\dots,a_n,b_1,\dots,b_n))_{n\in\N}$
\item Multiplication defined by $(a_n)_{n\in S}\times(b_n)_{n\in S}=(\pi_n(a_1,\dots,a_n,b_1,\dots,b_n))_{n\in\N}$
\end{itemize}
\end{defn}

\begin{thm}{}{} Let $R$ be a ring. Let $S$ be a truncation set. Then the ring of big Witt vectors $W_S(R)$ of $R$ is a ring with additive identity $(0,0,\dots)$ and multiplicative identity $(1,0,0,\dots)$. Moreover, for $(a_n)_{n\in S}\in W(R)$, its additive inverse is given by $(\iota_n(a_1,\dots,a_n))_{n\in\N}$.
\end{thm}

\begin{prp}{}{} Let $\phi:R\to R'$ be a ring homomorphism. Then the induced map $W_S(\phi):W_S(R)\to W_S(R')$ defined by $$W(\phi)((a_n)_{n\in S})=(\phi(a_n))_{n\in S}$$ is a ring homomorphism. 
\end{prp}

\begin{defn}{The Witt Functor}{} Define the Witt functor $W_S:\bold{Ring}\to\bold{Ring}$ to consist of the following data. 
\begin{itemize}
\item For each ring $R$, $W_S(R)$ is the ring of big Witt vectors
\item For a ring homomorphism $\phi:R\to R'$, $W_S(\phi):W_S(R)\to W_S(R')$ is the induced ring homomorphism defined by $$W_S(\phi)((a_n)_{n\in S})=(\phi(a_n))_{n\in S}$$
\end{itemize}
\end{defn}

\begin{prp}{}{} Let $S$ be a truncation set. The Witt functor is indeed a functor. 
\end{prp}

\begin{defn}{The Ghost Map}{} Let $R$ be a ring. Let $S$ be a truncation set. Define the ghost map to be the map $$w:W_S(R)\to\prod_{k\in S}R$$ by the formula $$w((a_n)_{n\in S})=(w_n(a_1,\dots,a_n))_{n\in S}$$
\end{defn}

Remember, by the sequence $a_1,\dots,a_n$ we mean the sequence $a_1,a_{d_1},\dots,a_{d_k},a_n$ where $1\leq d_1\leq\cdots\leq d_k\leq n$ and $d_1,\dots,d_k$ the complete collection of divisors of $n$. 

\begin{prp}{}{} Let $S$ be a truncation set. Then the following are true. 
\begin{itemize}
\item For each $n\in S$, the collection of maps $w_n:W_S(R)\to R$ for a ring $R$ defines a natural transformation $w_n:W_S\rightarrow\text{id}$. 
\item The collection of ghost maps $w_R:W_S(R)\to\prod_{k\in S}R$ for $R$ a ring defines a natural transformation $w:W_S\rightarrow (-)^S$. 
\end{itemize}
\end{prp}

\begin{prp}{}{} Let $S$ be a truncation set. The truncated Witt functor $W_S:\bold{Ring}\to\bold{Ring}$ is uniquely characterized by the following conditions. 
\begin{itemize}
\item The underlying set of $W_S(R)$ is given by $\prod_{k\in S}R$
\item For a ring homomorphism $\phi:R\to S$, $W(\phi):W(R)\to W(S)$ is the induced ring homomorphism defined by $$W(\phi)((a_n)_{n\in\N})=(\phi(a_n))_{n\in\N}$$
\item For each $n\in S$, $w_n:W_S(R)\to R$ defines a natural transformation $w_n:W\rightarrow\text{id}$
\end{itemize}
This means that if there is another functor $V$ satisfying the above, then $W$ and $V$ are naturally isomorphic. 
\end{prp}

Note that the above theorem implies that the ring structure on $\prod_{k\in S}R$ is unique under the above conditions. 

\subsection{Important Maps of Witt Vectors}
\begin{defn}{The Forgetful Map}{} Let $R$ be a ring. Let $T\subseteq S$ be truncation sets. Define the forgetful map $R_T^S:W_S(R)\to W_T(R)$ to be the ring homomorphism given by forgetting all elements $s\in S$ but $s\notin T$. 
\end{defn}

\begin{defn}{The $n$th Verschiebung Map}{} Let $R$ be a ring. Let $S$ be a truncation set. For $n\in\N$, define the $n$th Verschiebung map $V_n:W_{S/n}(R)\to W_S(R)$ by $$V_n((a_d)_{d\in S/n})_m=\begin{cases}
a_d & \text{ if } m=nd\\
0 & \text{otherwise}
\end{cases}$$
\end{defn}

Note that this is not a ring homomorphism. However, it is additive. 

\begin{lmm}{}{} Let $R$ be a ring. Let $S$ be a truncation set. Then for all $a,b\in W_{S/n}(R)$, we have that $$V_n(a+b)=V_n(a)+V_n(b)$$
\end{lmm}

\begin{defn}{Frobenius Map}{} Let $S$ be a truncation set. Let $R$ be a ring. Define the Frobenius map to be a natural ring homomorphism $F_n:W_S(R)\to W_{S/n}(R)$ such that the following diagram commutes: \\~\\
\adjustbox{scale=1.0,center}{\begin{tikzcd}
	{W_S(R)} & {\prod_{k\in S}R} \\
	{W_{S/n}(R)} & {\prod_{k\in S/n}R}
	\arrow["w", from=1-1, to=1-2]
	\arrow["{F_n}"', from=1-1, to=2-1]
	\arrow["{F_n^w}", from=1-2, to=2-2]
	\arrow["w"', from=2-1, to=2-2]
\end{tikzcd}}\\~\\
if it exists. 
\end{defn}

\begin{lmm}{}{} Let $S$ be a truncation set. Let $R$ be a ring. Then the Frobenius map exists and is unique. 
\end{lmm}

The following lemma relates this notion of Frobenius map to that in ring theory. 

\begin{lmm}{}{} Let $A$ be an $F_p$ algebra. Let $S$ be a truncation set. Let $\varphi_p:A\to A$ denote the Frobenius homomorphism given by $a\mapsto a^p$. Then $$F_p=R_{S/p}^S\circ W_S(\varphi):W_S(A)\to W_{S/p}(A)$$
\end{lmm}

\begin{defn}{The Teichmuller Representative}{} Let $R$ be a ring. Let $S$ be a truncation set. Define the Teichmuller representative to be the map $[-]_S:R\to W_S(R)$ defined by $$([a]_S)_n=\begin{cases}
a & \text{ if } n=1\\
0 &b \text{ otherwise }
\end{cases}$$
\end{defn}

The Teichmuller representative is in general not a ring homomorphism, but it is still multiplicative. 

\begin{lmm}{}{} Let $R$ be a ring. Let $S$ be a truncation set. The for all $a,b\in R$, we have that $$[ab]_S=[a]_S\cdot [b]_S$$
\end{lmm}

The three maps introduced are related as follows. 

\begin{prp}{}{} Let $R$ be a ring. Let $S$ be a truncated set. Then the following are true. 
\begin{itemize}
\item $r=\sum_{n\in S}V_n([r_n]_{S/n})$ for all $r\in W_S(R)$
\item $F_n(V_n(a))=na$ for all $a\in W_{S/n}(R)$
\item $r\cdot V_n(a)=V_n(F_n(r)\cdot a)$ for all $r\in W_S(R)$ and all $a\in W_{S/n}(R)$
\item $F_m\circ V_n=V_n\circ F_m$ if $\gcd(m,n)=1$
\end{itemize}
\end{prp}

The remaining section is dedicated to the example of $R=\Z$. 

\begin{prp}{}{} Let $S$ be a truncation set. Then the ring of big Witt vectors of $\Z$ is given by $$W_S(\Z)=\prod_{n\in S}\Z\cdot V_n([1]_{S/n})$$ with multiplication given by $$V_m([1]_{S/m})\cdot V_n([1]_{S/n})=\gcd(m,n)\cdot V_d([1]_{S/d})$$ and $d=\lcm(m,n)$. 
\end{prp}

\subsection{The Ring of p-Typical Witt Vectors}
For the ring of $p$-typical Witt vectors, we consider the truncation set $P=\{1,p,p^2,\dots\}\subseteq\N$ for a prime $p$. 

\begin{defn}{The Ring of p-Typical Witt Vectors}{} Let $R$ be a ring. Let $p$ be a prime. Let $P=\{1,p,p^2,\dots\}\subseteq\N$. Define the ring of $p$-typical Witt vectors to be $$W_p(R)=W_P(R)$$ Define the ring of $p$-typical Witt vectors of length $n$ to be $$W_n(R)=W_{\{1,p,\dots,p^{n-1}\}}(R)$$ when the prime $p$ is understood. 
\end{defn}

\begin{thm}{}{} Let $R$ be a ring. Let $p$ be a prime number. Let $S$ be a truncation set. Write $I(S)=\{k\in S\;|\;k\text{ does not divide }p\}$. Suppose that all $k\in I(S)$ are invertible in $R$. Then there is a decomposition $$W_S(R)=\prod_{k\in I(S)}W_S(R)\cdot e_k$$ where $$e_k=\prod_{t\in I(S)\setminus\{1\}}\left(\frac{1}{k}V_k([1]_{S/k})-\frac{1}{kt}V-{kt}([1]_{S/kt})\right)$$ Moreover, the composite map given by \\~\\
\adjustbox{scale=1.0,center}{\begin{tikzcd}
	{W_S(R)\cdot e_k} & {W_S(R)} & {W_{S/k}R} & {W_{S/k\cap P}(R)}
	\arrow[hook, from=1-1, to=1-2]
	\arrow["{F_k}", from=1-2, to=1-3]
	\arrow["{R_{S/k\cap P}^{S/k}}", from=1-3, to=1-4]
\end{tikzcd}}\\~\\
is an isomorphism. 
\end{thm}

\subsection{The $\lambda$-structure on W(R)}
\begin{lmm}{}{} Let $R$ be a ring. Then every $f\in\Lambda(R)$ can be written uniquely as $$f=\prod_{k=1}^\infty(1-(-1)^na_nt^n)$$
\end{lmm}

\begin{thm}{The Artin-Hasse Exponential}{} There is a natural isomorphism $E:\Lambda\rightarrow W$ given as follows. For a ring $R$, $E_R:\Lambda(R)\to W(R)$ is defined by $$E_R\left(\prod_{k=1}^\infty(1-(-1)^na_nt^n)\right)=(a_n)_{n\in\N}$$
\end{thm}

\begin{crl}{}{} Let $R$ be a ring. Then $W(R)$ has a canonical $\lambda$-structure inherited from $\Lambda(R)$. 
\end{crl}

TBA: The forgetful functor $U:\Lambda\bold{Ring}\to\bold{CRing}$ has a left adjoint $\text{Symm}$ and has a right adjoint $W$. 


\pagebreak
\section{Formal Group Laws}
\begin{defn}{Formal Group Laws}{} Let $R$ be a ring. A formal group law over $R$ is a power series $$f(x,y)\in R[[x,y]]$$ such that the following are true. 
\begin{itemize}
\item $f(x,0)=f(0,x)=x$
\item $f(x,y)=f(y,x)$
\item $f(x,f(y,z))=f(f(x,y),z)$
\end{itemize}
\end{defn}

\begin{defn}{The Formal Group Law Functor}{} Define the formal group law functor $$FGL:\bold{Ring}\to\bold{Set}$$ by the following data. 
\begin{itemize}
\item For each ring $R$, $FGL(R)$ is the set of all formal group laws over $R$
\item For each ring homomorphism $f:R\to S$, $FGL(f)$ sends each formal group law $\sum_{i,j=0}^\infty c_{i,j}x^iy^j$ over $R$ to the formal group law $\sum_{i,j=0}^\infty f(c_{i,j})x^iy^j$ over $S$. 
\end{itemize}
\end{defn}

\begin{defn}{The Lazard Ring of a Formal Group Law}{} Define the lazard ring by $$L=\frac{\Z[c_{i,j}]}{Q}$$ where $Q$ is the ideal generated as follows. Write $f=\sum_{i,j=0}^\infty c_{i,j}x^iy^j$. Then $Q$ is generated by the constraints on $c_{i,j}$ for which $f$ becomes a formal group law. 
\end{defn}

\begin{lmm}{}{} The Lazard ring $L=\Z[c_{i,j}]/Q$ has the structure of a graded ring where $c_{i,j}$ has degree $2(i+j-1)$. 
\end{lmm}

\begin{thm}{}{} The formal group law functor $FGL:\bold{Ring}\to\bold{Set}$ is representable $$FGL(R)\cong\Hom_\bold{Ring}(L,R)$$ There exists a universal element $f\in L$ such that the map $\Hom_\bold{Ring}(L,R)\to FGL(R)$ given by evaluation on $f$ is a bijection for any ring $R$. 
\end{thm}

\begin{thm}{}{} There is an isomorphism of the Lazard ring $$L\cong\Z[t_1,t_2,\dots]$$ where each $t_k$ has degree $2k$. 
\end{thm}

\pagebreak
\section{Homotopy Pullbacks and Pushouts}
Homotopy pullbacks and pushouts are a special case of homotopy limits and colimits. It would be fruitful for us to first consider this case also because of how it is related to maps of spaces and (co)fibrations. 

\subsection{Homotopy Pullbacks}
\begin{defn}{Homotopy Pullbacks}{} Let $X,Y,Z\in\bold{CGWH}$ be spaces. Let $\mS$ denote the following diagram \\~\\
\adjustbox{scale=1.0,center}{\begin{tikzcd}
	X & Z & Y
	\arrow["f", from=1-1, to=1-2]
	\arrow["g"', from=1-3, to=1-2]
\end{tikzcd}}\\~\\
in $\bold{CGWH}$. Define the homotopy pullback $\text{holim}(X\overset{f}{\rightarrow}Z\overset{g}{\leftarrow}Y)$ of the diagram to be the subspace of $X\times\text{Map}(I,Z)\times Y$ consisting of $$\{(x,\alpha,y)\in X\times\text{Map}(I,Z)\times Y\;|\;\alpha(0)=f(x),\alpha(1)=g(y)\}$$
\end{defn}

The idea is that normally in pullbacks, we require that under $f$ and $g$ the elements of the pullback must arrive at the same point in $Z$. But here we relax the requirement by simply allowing elements of the homotopy pullback to arrive at the same path component of $Z$ (so up to the existence of an homotopy of the two points in $Z$). 

\begin{defn}{The Canonical Map of Homotopy Pullbacks}{} Let $X,Y,Z\in\bold{CGWH}$ be spaces such that \\~\\
\adjustbox{scale=1.0,center}{\begin{tikzcd}
	X & Z & Y
	\arrow["f", from=1-1, to=1-2]
	\arrow["g"', from=1-3, to=1-2]
\end{tikzcd}}\\~\\
is a diagram in $\bold{CGWH}$. Define the canonical map of the homotopy pullback of the diagram to be the map $$c:\lim(X\overset{f}{\rightarrow}Z\overset{g}{\leftarrow}Y)\to\text{holim}(X\overset{f}{\rightarrow}Z\overset{g}{\leftarrow}Y)$$ defined by $(x,y)\mapsto(x,c_{f(x)=g(y)},y)$. 
\end{defn}

\begin{thm}{The Matching Lemma}{} Suppose that we have a commutative diagram of spaces \\~\\
\adjustbox{scale=1.0,center}{\begin{tikzcd}
	X & Z & Y \\
	{X'} & {Z'} & {Y'}
	\arrow["f", from=1-1, to=1-2]
	\arrow["{e_X}"', from=1-1, to=2-1]
	\arrow["{e_Z}"', from=1-2, to=2-2]
	\arrow["g"', from=1-3, to=1-2]
	\arrow["{e_Y}", from=1-3, to=2-3]
	\arrow["{f'}"', from=2-1, to=2-2]
	\arrow["{g'}", from=2-3, to=2-2]
\end{tikzcd}}\\~\\
in $\bold{CGWH}$. Define the map $$\phi_{X,Z,Y}^{X',Z',Y'}:\text{holim}(X\overset{f}{\rightarrow}Z\overset{g}{\leftarrow}Y)\to\text{holim}(X'\overset{f'}{\rightarrow}Z'\overset{g'}{\leftarrow}Y')$$ by the formula $(x,\gamma,y)\mapsto(e_X(x),e_Z\circ\gamma,e_Y(y))$. Then the following are true. 
\begin{itemize}
\item If each $e_X,e_Y,e_Z$ are homotopy equivalences, then $\phi$ is a homotopy equivalence. 
\item If each $e_X,e_Y,e_Z$ are weak equivalences, then $\phi$ is a weak equivalence. 
\end{itemize} \tcbline
\begin{proof}
We first prove the case for homotopy equivalence. Consider the following commutative diagram: \\~\\
\adjustbox{scale=1.0,center}{\begin{tikzcd}
	X & Z & Y \\
	X & {Z'} & Y \\
	{X'} & {Z'} & {Y'}
	\arrow["f", from=1-1, to=1-2]
	\arrow["{\text{id}_X}"', from=1-1, to=2-1]
	\arrow["{e_Z}", from=1-2, to=2-2]
	\arrow["g"', from=1-3, to=1-2]
	\arrow["{\text{id}_Y}", from=1-3, to=2-3]
	\arrow["{e_Z\circ f}", from=2-1, to=2-2]
	\arrow["{e_X}"', from=2-1, to=3-1]
	\arrow["{\text{id}_{Z'}}", from=2-2, to=3-2]
	\arrow["{e_Z\circ g}"', from=2-3, to=2-2]
	\arrow["{e_Y}", from=2-3, to=3-3]
	\arrow["{f'}"', from=3-1, to=3-2]
	\arrow["{g'}", from=3-3, to=3-2]
\end{tikzcd}}\\~\\
We prove that the homotopy pullback of the first row is homotopy equivalent to that of the second, and we prove that the homotopy pullback of the second row is homotopy equivalent to that of the third. \\~\\

Since $e_Z$ is a homotopy equivalence, we can find a homotopy inverse $k$ for $e_Z$ and a homotopy $H:Z\times I\to Z$ such that $H(-,0)=\text{id}_Z$ and $H(-,1)=k\circ e_Z$. Define a map $$\rho:\text{holim}(X\overset{f}{\rightarrow}Z'\overset{g}{\leftarrow}Y)\to\text{holim}(X\overset{e_Z\circ f}{\rightarrow}Z\overset{e_Z\circ g}{\leftarrow}Y)$$ by the formula $$(x,\gamma',y)\mapsto(x,H(f(x),-)\ast k(\gamma'(-))\ast\overline{H(g(y),-)}:I\to Z,y)$$ where $\ast$ denotes concatenation of paths. The path concatenation is well defined because we have that $H(f(x),1)=(k\circ e_Z\circ f)(x)=(k\circ\gamma')(0)$ and $k(\gamma'(1))=k(e_Z(g(y)))=H(g(y),1)$. This is well defined on the homotopy pullback because we have that 
\begin{itemize}
\item $H(f(x),-)\ast k(\gamma'(-))\ast\overline{H(g(y),-)}(0)=H(f(x),0)=\text{id}_Z(f(x))=f(x)$
\item $H(f(x),-)\ast k(\gamma'(-))\ast\overline{H(g(y),-)}(1)=H(g(y),0)=\text{id}_Z(g(y))=g(y)$
\end{itemize}
I claim that this map is inverse to the map $\phi=\phi_{X,Y,Z}^{X,Z',Y}$ where we take $e_X=\text{id}_X$ and $e_Y=\text{id}_Y$. We have that 
\begin{align*}
\rho(\phi(x,\gamma,y))&=\rho(x,e_Z\circ\gamma,y)\\
&=(x,H(f(x),-)\ast k(e_Z(\gamma(-))\ast\overline{H(g(y),-)},y)
\end{align*}
Now I claim that the middle path is homotopic to $\gamma$. For the first part, the path $H(f(x),t):I\to Z$ can be contracted to $H(f(x),0)=f(x)=\gamma(0)$ so you can homotope the traversal along $H(f(x),-)$ to the single point $f(x)=\gamma(0)$. For the third part, this is similar so we can homotope the traversal of $\overline{H(g(y),-)}$ to the single point $g(y)=\gamma(1)$. The middle part of the path is homotopic to $\gamma$ because $k\circ e_Z$ is homotopic to $\text{id}_Z$. Thus we conclude. 
\end{proof}
\end{thm}

When one of the maps $f$ or $g$ is a fibration, then the notion of a pullback coincides with that of homotopy pullback. 

\begin{prp}{}{} Let $X,Y,Z\in\bold{CGWH}$ be spaces such that \\~\\
\adjustbox{scale=1.0,center}{\begin{tikzcd}
	X & Z & Y
	\arrow["f", from=1-1, to=1-2]
	\arrow["g"', from=1-3, to=1-2]
\end{tikzcd}}\\~\\
is a diagram in $\bold{CGWH}$. Then the following spaces are homeomorphic. 
\begin{itemize}
\item $\text{holim}(X\overset{f}{\rightarrow}Z\overset{g}{\leftarrow}Y)$
\item $\lim(P_f\rightarrow Z\overset{g}{\leftarrow}Y)$
\item $\lim(X\overset{f}{\rightarrow}Z\leftarrow P_g)$
\item $\lim(P_f\rightarrow Z\leftarrow P_g)$
\end{itemize}
\end{prp}

\begin{prp}{}{} Let $X,Y,Z\in\bold{CGWH}$ be spaces such that \\~\\
\adjustbox{scale=1.0,center}{\begin{tikzcd}
	X & Z & Y
	\arrow["f", from=1-1, to=1-2]
	\arrow["g"', from=1-3, to=1-2]
\end{tikzcd}}\\~\\
is a diagram in $\bold{CGWH}$. If $f$ or $g$ is a fibration, then the canonical map $$\lim(X\overset{f}{\rightarrow}Z\overset{g}{\leftarrow}Y)\to\text{holim}(X\overset{f}{\rightarrow}Z\overset{g}{\leftarrow}Y)$$ is a homotopy equivalence. 
\end{prp}

\subsection{Homotopy Pushouts}
We now want to measure how far away is a square diagram from being a homotopy pullback and dually, how far away is a square diagram from being a homotopy pushout. 

\begin{defn}{Homotopy Pushouts}{} Let $X,Y,Z\in\bold{CGWH}$ be spaces. Let $\mS$ denote the following diagram \\~\\
\adjustbox{scale=1.0,center}{\begin{tikzcd}
	X & Z & Y
	\arrow["f"', from=1-2, to=1-1]
	\arrow["g", from=1-2, to=1-3]
\end{tikzcd}}\\~\\
in $\bold{CGWH}$. Define the homotopy pushout of the diagram to be the quotient space $$\text{hocolim}(\mS)=\frac{X\amalg(Z\times I)\amalg Y}{\sim}$$ where $\sim$ is the equivalence relation generated by $f(z)\sim (z,0)$ and $g(z)\sim(z,1)$ for $z\in Z$. If $(Z,z_0)$ is a based space, then the equivalence relation is also generated by $(x_0,t)\sim(z_0,s)$ for $s,t\in I$. 
\end{defn}

\begin{defn}{The Canonical Map of Homotopy Pushouts}{} Let $X,Y,Z\in\bold{CGWH}$ be spaces. Let $\mS$ denote the following diagram \\~\\
\adjustbox{scale=1.0,center}{\begin{tikzcd}
	X & Z & Y
	\arrow["f"', from=1-2, to=1-1]
	\arrow["g", from=1-2, to=1-3]
\end{tikzcd}}\\~\\
in $\bold{CGWH}$. Define the canonical map of the homotopy pushout of the diagram to be the map $$s:\text{hocolim}(\mS)\to\colim(\mS)$$ given by the formula $$u\mapsto\begin{cases}
u & \text{ if }u\in X\\
f(z)=g(z) & \text{ if }u=(z,t)\in Z\times I\\
u & \text{ if }u\in Y
\end{cases}$$
\end{defn}

\begin{thm}{The Gluing Lemma}{} Suppose that we have a commutative diagram of spaces \\~\\
\adjustbox{scale=1.0,center}{\begin{tikzcd}
	X & Z & Y \\
	{X'} & {Z'} & {Y'}
	\arrow["f"', from=1-2, to=1-1]
	\arrow["{e_X}"', from=1-1, to=2-1]
	\arrow["{e_Z}"', from=1-2, to=2-2]
	\arrow["g", from=1-2, to=1-3]
	\arrow["{e_Y}", from=1-3, to=2-3]
	\arrow["{f'}", from=2-2, to=2-1]
	\arrow["{g'}"', from=2-2, to=2-3]
\end{tikzcd}}\\~\\
in $\bold{CGWH}$. If each $e_X,e_Y,e_Z$ are (homotopy) weak equivalences, then the induced map $$\text{hocolim}(X\overset{f}{\leftarrow}Z\overset{g}{\rightarrow}Y)\to\text{hocolim}(X'\overset{f'}{\leftarrow}Z'\overset{g'}{\rightarrow}Y')$$ defined by the formula $$u\mapsto\begin{cases}
e_X(u) & \text{ if }u\in X\\
(e_Z(v),t) & \text{ if }u=(v,t)\in Z\times I\\
e_Y(u) & \text{ if }u\in Y
\end{cases}$$ is a (homotopy) weak equivalence. 
\end{thm}

\begin{prp}{}{} Let $X,Y,Z\in\bold{CGWH}$ be spaces. Let $\mS$ denote the following diagram \\~\\
\adjustbox{scale=1.0,center}{\begin{tikzcd}
	X & Z & Y
	\arrow["f"', from=1-2, to=1-1]
	\arrow["g", from=1-2, to=1-3]
\end{tikzcd}}\\~\\
in $\bold{CGWH}$. Then the following spaces are homeomorphic. 
\begin{itemize}
\item $\text{hocolim}(\mS)$
\item $\colim(M_f\leftarrow Z\rightarrow Y)$
\item $\colim(X\leftarrow Z\rightarrow M_g)$
\item $\colim(M_f\leftarrow Z\rightarrow M_g)$
\end{itemize}
\end{prp}

\subsection{Homotopy Squares}
\begin{defn}{Homotopy Pullback and Pushout Squares}{} Let $W,X,Y,Z\in\bold{CGWH}$ be spaces such that there is a (not necessarily commutative) diagram \\~\\
\adjustbox{scale=1.0,center}{\begin{tikzcd}
	W & Y \\
	X & Z
	\arrow[from=1-1, to=1-2]
	\arrow[from=1-1, to=2-1]
	\arrow[from=1-2, to=2-2]
	\arrow[from=2-1, to=2-2]
\end{tikzcd}}\\~\\
\begin{itemize}
\item We say that the diagram is a homotopy pullback if the map $$\alpha:W\to\lim(X\overset{f}{\rightarrow}Z\overset{g}{\leftarrow}Y)\overset{c}{\longrightarrow}\text{holim}(X\overset{f}{\rightarrow}Z\overset{g}{\leftarrow}Y)$$ is a weak equivalence. 
\item We say that the diagram is $k$-cartesian if $\alpha$ is $k$-connected. 
\item Dually, we say that the square is a homotopy pushout square if the map $$\beta:\text{hocolim}(X\overset{f}{\leftarrow}W\overset{g}{\rightarrow}Y)\overset{s}{\longrightarrow}\colim(X\overset{f}{\leftarrow}W\overset{g}{\rightarrow}Y)\to Z$$ is a weak equivalence. 
\item Also dually, we say that the diagram is $k$-cocartesian if $\beta$ is $k$-connected. 
\end{itemize}
\end{defn}

We can rephrase 5.1.5 in the following way: if \\~\\
\adjustbox{scale=1.0,center}{\begin{tikzcd}
	W & Y \\
	X & Z
	\arrow[from=1-1, to=1-2]
	\arrow[from=1-1, to=2-1]
	\arrow[from=1-2, to=2-2]
	\arrow[from=2-1, to=2-2]
\end{tikzcd}}\\~\\
is a commutative square diagram and either $X\to Z$ or $Y\to Z$ is a fibration, then the square is a homotopy pullback square. 

\begin{lmm}{}{} Let $X$ be a space. Let $A,B\subseteq X$ be subspace of $X$ such that $X=A\cup B$. Then the following commutative square \\~\\
\adjustbox{scale=1.0,center}{\begin{tikzcd}
	A\cap B & A \\
	B & X
	\arrow[from=1-1, to=1-2]
	\arrow[from=1-1, to=2-1]
	\arrow[from=1-2, to=2-2]
	\arrow[from=2-1, to=2-2]
\end{tikzcd}}\\~\\
given by inclusions is a homotopy pushout square. 
\end{lmm}

\begin{thm}{Seifert-van Kampen Theorem}{} Let $W,X,Y,Z\in\bold{CGWH}$ be spaces such that the following square \\~\\
\adjustbox{scale=1.0,center}{\begin{tikzcd}
	W & Y \\
	X & Z
	\arrow[from=1-1, to=1-2]
	\arrow[from=1-1, to=2-1]
	\arrow[from=1-2, to=2-2]
	\arrow[from=2-1, to=2-2]
\end{tikzcd}}\\~\\
is a homotopy pullback. Suppose that $W,X,Y$ are path connected. Then the square \\~\\
\adjustbox{scale=1.0,center}{\begin{tikzcd}
	\pi_1(W) & \pi_1(Y) \\
	\pi_1(X) & \pi_1(Z)
	\arrow[from=1-1, to=1-2]
	\arrow[from=1-1, to=2-1]
	\arrow[from=1-2, to=2-2]
	\arrow[from=2-1, to=2-2]
\end{tikzcd}}\\~\\
is a pushout in $\bold{Grp}$. In other words, there is a canonical isomorphism $$\pi_1(X)\ast_{\pi_1(W)}\pi_1(Y)\overset{\cong}{\longrightarrow}\pi_1(Z)$$
\end{thm}

\subsection{Relation to Homotopy (Co)Fibers}
Recall that the mapping path space $P_f$ of a map $f:X\to Y$ is defined to be $$P_f=f^\ast(\text{Map}(I,Y))=\{(x,\phi)\subseteq X\times\text{Map}(I,Y)\;|\;f(x)=\pi_0(\phi)=\phi(0)\}$$ we can now prove that $P_f$ is a homotopy invariance. 

\begin{crl}{}{} Let $X,Y\in\bold{CGWH}$ be spaces. Let $f,g:X\to Y$ be maps. Then there is a homotopy equivalence $$P_f\simeq P_g$$ Moreover, there is a homotopy equivalence $$\text{hofiber}_y(f)\simeq\text{hofiber}_y(g)$$ for any $y\in Y$. 
\end{crl}

Recall that the fiber of a map $f:X\to Y$ behaves poorly because the fibers are not homeomorphic and not even homotopy equivalent. However, we can now prove that the homotopy fibers are the correct notion of a fiber to study because they are homotopy equivalent. 

\begin{crl}{}{} Let $X,Y\in\bold{CGWH}$ be space. Let $f:X\to Y$ be a map. If $y_1$ and $y_2$ lie in the same path component of $Y$ then there is a homotopy equivalence $$\text{hofiber}_{y_1}(f)=\text{hofiber}_{y_2}(f)$$
\end{crl}

\begin{crl}{}{} Let $X,Y\in\bold{CGWH}$ be spaces. Let $f,g:X\to Y$ be maps. Then there is a homotopy equivalence $$M_f\simeq M_g$$ Moreover, there is a homotopy equivalence $$\text{hocofiber}(f)\simeq\text{hocofiber}(g)$$ for any $y\in Y$. 
\end{crl}

We can we interpret homotopy pullbacks and pushouts using homotopy (co)fibers. 

\begin{prp}{}{} Let $W,X,Y,Z\in\bold{CGWH}$ be spaces such that following is a (not necessarily commutative) square \\~\\
\adjustbox{scale=1.0,center}{\begin{tikzcd}
	W & Y \\
	X & Z
	\arrow[from=1-1, to=1-2]
	\arrow[from=1-1, to=2-1]
	\arrow[from=1-2, to=2-2]
	\arrow[from=2-1, to=2-2]
\end{tikzcd}}\\~\\
Then the following are true. 
\begin{itemize}
\item The square is a homotopy pullback if and only if for all $x\in X$, the map $$\text{hofiber}_x(W\to X)\to\text{hofiber}_{f(x)}(Y\to Z)$$ is a weak equivalence. 
\item The square is $k$-carteisna if and only if for all $x\in X$, the map $$\text{hofiber}_x(W\to X)\to\text{hofiber}_{f(x)}(Y\to Z)$$ is $k$-connected. 
\end{itemize}
\end{prp}

\subsection{Connectedness of Homotopy Squares}

\pagebreak
\section{}
\subsection{The Blakers-Massey Theorem}
The Blakers-Massey theorem is a direct generalization of the homotopy excision theorem. Its proof takes a similar form to the homotopy excision theorem. Let us recall some definitions used. 

\begin{defn}{(Degenerative) Cubes}{} Let $a=(a_1,\dots,a_n)\in\R^n$. Let $\delta>0$. Let $L\subseteq\{1,\dots,n\}$. A cube in $\R^n$ is a set of the form $$W=W(a,\delta,L)=\{x\in\R^n\;|\;a_i\leq x\leq a_i+\delta\text{ for }i\in L\text{ and }x_i=a_i\text{ for }i\notin L\}$$
\end{defn}

\begin{lmm}{}{} Let $Y$ be a space. Let $B\subseteq Y$ be a subspace of $Y$. Let $W$ be a cube in $\R^n$. Let $f:W\to Y$ be a map. Let $j=1$ or $2$. Suppose that there exists some $p\leq\abs{L}$ such that $$f^{-1}(B)\cap C\subset K_p^j(C)=\left\{x\in C\;|\;\frac{\delta(j-1)}{2}+a_i<x_i<\frac{\delta j}{2}+a_i\text{ for at least }p\text{ values of }i\in L\right\}$$ for all cubes $C\subset\partial W$. Then there exists a map $g:W\to Y$ such that $g\overset{\partial W}{\simeq} f$ and $$g^{-1}(B)\subset K_p^j(C)$$
\end{lmm}

\begin{prp}{}{} Let $X$ be a space. Let $X_0,X_1,X_2\subseteq X$ be subspaces of $X$ such that $$X=X_1\amalg_{X_0}X_2$$ Let $f:I^n\to X$ be a map. Suppose that $W\subseteq I^n$ is any cube given by the Lebesgue covering lemma for which $f(W)\subseteq X_i$ for one of $i=0,1,2$. Assume that for each $i=1,2$, $(X_i,X_0)$ is $k_i$-connected with $k_i\geq 0$. Then there exists a homotopy $$H:I^n\times I\to X$$ from $f=H(-,0)$ such that the following are true. 
\begin{itemize}
\item If $f(W)\subset X_i$, then $H(W,t)\subset X_i$ for all $t\in I$. 
\item If $f(W)\subset X_0$, then $H(W,t)=f(W)$ for all $t\in I$. 
\item If $f(W)\subset X_i$, then $f^{-1}(X_i\setminus X_0)\cap W\subset K_{k_j+1}^j(W)$. 
\end{itemize}
\end{prp}

The reasons for the above setup is that we want to prove the following lemma. 

\begin{lmm}{}{} Let $X$ be a space. Let $e^{d_i}$ be a cell of dimension $d_i$ for $i=1,2$. Then the following diagram \\~\\
\adjustbox{scale=1.0,center}{\begin{tikzcd}
	X & X\cup e^{d_1} \\
	X\cup e^{d_2} & X\cup e^{d_1}\cup e^{d_2}
	\arrow[from=1-1, to=1-2]
	\arrow[from=1-1, to=2-1]
	\arrow[from=1-2, to=2-2]
	\arrow[from=2-1, to=2-2]
\end{tikzcd}}\\~\\
given by inclusion maps is $(d_1+d_2-3)$-cartesian. 
\end{lmm}

This lemma has a less geometric way of proving it that does not involve any of the preparative lemmas and propositions, however it does use significant material that have not been covered. 

\begin{thm}{Blakers-Massey Theorem for Squares}{} Let $X_0,X_1,X_2,X_{12}\in\bold{CGWH}$ be spaces such that the square \\~\\
\adjustbox{scale=1.0,center}{\begin{tikzcd}
	X_0 & X_1 \\
	X_2 & X_{12}
	\arrow[from=1-1, to=1-2]
	\arrow[from=1-1, to=2-1]
	\arrow[from=1-2, to=2-2]
	\arrow[from=2-1, to=2-2]
\end{tikzcd}}\\~\\
is a homotopy pushout. Suppose the map $X_0\to X_i$ is $k_i$-connected for $i=1,2$. Then the diagram is $(k_1+k_2-1)$-cartesian. Explicitly, this means that $$\alpha:X_0\to\text{holim}(X_1\rightarrow X_{12}\leftarrow X_2)$$ is $(k_1+k_2-1)$-connected. 
\end{thm}

This theorem directly generalizes the homotopy excision theorem in the following way. For $X$ a CW complex and $A,B$ two subcomplexes with non-empty intersection and $X=A\cup B$, consider the following square of inclusions: \\~\\
\adjustbox{scale=1.0,center}{\begin{tikzcd}
	A\cap B & A \\
	B & X
	\arrow[from=1-1, to=1-2]
	\arrow[from=1-1, to=2-1]
	\arrow[from=1-2, to=2-2]
	\arrow[from=2-1, to=2-2]
\end{tikzcd}}\\~\\
We have seen that such a square diagram is a homotopy pushout diagram. Now any inclusion map $W\hookrightarrow Z$ is $k$-connected if and only if $(Z,W)$ is $k$-connected. So $(A,A\cap B)$ is $k_1$-connected and $(X,B)$ is $k_2$-connected. Blaker's-Massey theorem implies that $$\text{hofiber}(A\cap B\to A)\to\text{hofiber}(B,X)$$ is $(k_1+k_2-1)$-connected. But by definition we have an isomorphism $\pi_k(\text{hofiber}(U\to V)\cong\pi_{k+1}(V,U)$. So we are really just saying that $\pi_k(A,A\cap B)\to\pi_k(X,B)$ given by the inclusion is $(k_1+k_2)$-connected. 

\begin{thm}{Dual Blakers-Massey Theorem for Squares}{} Let $X_0,X_1,X_2,X_{12}\in\bold{CGWH}$ be spaces such that the square \\~\\
\adjustbox{scale=1.0,center}{\begin{tikzcd}
	X_0 & X_1 \\
	X_2 & X_{12}
	\arrow[from=1-1, to=1-2]
	\arrow[from=1-1, to=2-1]
	\arrow[from=1-2, to=2-2]
	\arrow[from=2-1, to=2-2]
\end{tikzcd}}\\~\\
is a homotopy pullback. Suppose the map $X_i\to X_{12}$ is $k_i$-connected for $i=1,2$. Then the diagram is $(k_1+k_2-1)$-cocartesian. Explicitly, this means that $$\beta:\text{hocolim}(X_1\leftarrow X_0\rightarrow X_2)\to X_{12}$$ is $(k_1+k_2-1)$-connected.
\end{thm}

\pagebreak
\section{n-Cubes}
In algebraic topology, we have learnt about spaces, maps of spaces and maps of maps of spaces. We can say this in a more compact way. Namely, if we think of maps of maps of space as a square (2-cube), we can think of spaces as 0-cubes and maps of spaces as 1-cube. We have studied 2-cubes extensively under the guise of homotopy pullbacks and pushouts. We can now take this further and consider general n-cubes. 

\begin{defn}{n-Cubes of Spaces}{} Let $n\in\N$. Let $P(n)$ denote the category of posets of the set $\{1,\dots,n\}$. An $n$-cube of spaces is a functor $$X:P(n)\to\bold{CGWH}$$ An $n$-cube of based spaces is a functor $X:P(n)\to\bold{CGWH}_\ast$. 
\end{defn}

Explicitly, an $n$-cube of spaces $X:P(n)\to\bold{CGWH}$ consists of the following data. 
\begin{itemize}
\item For each $S\subseteq\{1,\dots,n\}$ a space $X_S$
\item For each $S\subseteq T$, a map $f_{S\subseteq T}:X_S\to X_T$ such that $f_{S\subseteq S}=1_{X_S}$ and for all $R\subseteq S\subseteq T$, we have a commutative diagram \\~\\
\adjustbox{scale=1.0,center}{\begin{tikzcd}
	{X_R} & {X_S} \\
	& {X_T}
	\arrow["{f_{R\subseteq S}}", from=1-1, to=1-2]
	\arrow["{f_{R\subseteq T}}"', from=1-1, to=2-2]
	\arrow["{f_{S\subseteq T}}", from=1-2, to=2-2]
\end{tikzcd}}\\~\\
\end{itemize}

Omit drawing composite arrows and omit drawing identities. \\
Also: punctured cubes def

\begin{defn}{Cube of Cubes}{} An $n$-cube of $m$-cubes is a functor $$X:P(n)\times P(m)\to\bold{CGWH}$$
\end{defn}

\begin{lmm}{}{} An $n$-cube of $m$-cubes $X$ is precisely an $(n+m)$-cube. 
\end{lmm}

\begin{defn}{Map of $n$-Cubes}{} Let $X,Y:P(n)\to\bold{CGWH}$ be $n$-cubes. A map of $n$-cubes is a natural transformation $F:X\to Y$ such that the assignment $Z:P(n+1)\to\bold{CGWH}$ given by $$Z(S)=\begin{cases}
X(S) & \text{ if } S\subseteq\{1,\dots,n\}\\
Y(S\setminus\{1,\dots, n+1\}) & \text{ if }\{1,\dots,n+1\}\subseteq S
\end{cases}$$
defines an $(n+1)$-cube. 
\end{defn}

objectwise (co)fibration, homotopy (weak) equivalence. homeomorphism

\begin{defn}{Strongly Homotopy Cartesian}{} Let $X$ be an $n$-cube of spaces. We say that $X$ is strongly homotopy cartesian if each of its faces of dimension $n\geq 2$ is homotopy cartesian. 
\end{defn}

\pagebreak
\section{Homotopy Limits and Colimits}

Let $X:\mJ\to\bold{Top}$ be a diagram of spaces. Denote the constant functor of the one point space by $\Delta\ast:\bold{X}\to\bold{Top}$. The data of the constant functor is given as follows. 
\begin{itemize}
\item For each $I\in\mJ$, $\Delta\ast(I)=\ast$
\item For each morphism $f:I\to J$ in $X$, define $\Delta\ast(f)=\text{id}_\ast$. 
\end{itemize}
Consider the set of all natural transformations $\Delta\ast\Rightarrow X$ denoted by $\text{Nat}(\Delta\ast,X)$. Now this set can inherit a subspace topology via the isomorphism (of sets) $$\text{Nat}(\Delta\ast,X)\cong\prod_{I\in\mJ}\Hom_\bold{Top}(\ast,X_I)\subset\prod_{I\in\mJ}X_I$$ There is in fact a canonical homeomorphism between the set of natural transformations and the limit of $X$. 

\begin{thm}{}{} Let $X:\mJ\to\bold{Top}$ be a diagram of spaces. Then there is a canonical homeomorphism $$\lim_\mJ X\cong\text{Nat}(\Delta\ast,X)$$
\end{thm}
Ref: Cubical diagrams

\begin{thm}{}{} Let $X:\mJ\to\bold{Top}$ be a diagram of spaces. Then there is a canonical homeomorphism $$\colim_\mJ X\cong\frac{\coprod_{I\in\mJ}X_I}{\sim}$$ where $x\in X_I\sim y\in X_J$ if and only if there exists $f:I\to J$ such that $X(f)(x)=y$. 
\end{thm}

\subsection{Homotopy Limits and Colimits}
Let $\mJ$ be a small diagram. Let $I\in\mJ$ and denote $\mJ/I$ to be the overcategory with the distinguished object $I$. We can turn it into a topological space by constructing its classifying space $$\mB(\mJ/I)=\abs{N(\mJ/I)}$$ which is the geometric realization of the nerve of $\mJ/I$. We aim to use the overcategory $\mJ/I$ to record homotopy information. Recall that the limit is canonically isomorphic to the equalizer of $$f,g:\prod_{J\in\Obj\mJ}X_J\to\prod_{(\alpha:J\to I)\in\mJ}X_I$$ as follows. 
\begin{itemize}
\item Define $f$ to be the unique map such that $\pi_I\circ f=\pi_{\text{cod}(\alpha)}$ where both $\pi$ are projections. 
\item Define $g$ to be the unique map such that $\pi_{\text{cod}(\alpha)}\circ g=F(\alpha)\circ\pi_{\text{dom}(\alpha)}$
\end{itemize}
We aim to replace each space $X_I$ by the space $\text{Map}(\mB(\mJ/I),X_I)$ (which is why we work with $\bold{CGWH}$). Indeed the hom space and $X_I$ are homotopy equivalent since $\mB(\mJ/I)$ is contractible. Notice that this is no longer a functor in $I$, but rather a bifunctor. This all works with a simplicial model category. 

\begin{defn}{Homotopy Limits}{} Let $\mC$ be a simplicial model category. Let $X:\mJ\to\mC$ be a diagram. Define two maps $$f,g:\prod_{J\in\Obj\mJ}\text{Map}(\mB(\mJ/J),X_J)\to\prod_{(\alpha:J\to I)\in\mJ}\text{Map}(\mB(\mJ/J),X_I)$$ as follows. 
\begin{itemize}
\item Define $f$ to be the unique map such that $$\pi_I\circ f=\text{Map}(\mB(\mJ/J),F(\alpha):X_J\to X_I)\circ\pi_J$$ for any $\alpha:J\to I$ a morphism in $\mJ$. 
\item Define $g$ to be the unique map such that $$\pi_I\circ g=\text{Map}\left(\mB\left(F^\ast(\alpha):\mJ/I\to\mJ/J\right),X_I\right)\circ\pi_I$$ for any $\alpha:J\to I$ a morphism in $\mJ$. 
\end{itemize}
Define the homotopy limit $\text{holim}X$ of $X$ to be the equalizer of the maps $$\text{holim}X=\text{Eq}(f,g)$$
\end{defn}

\begin{prp}{}{} Let $X:\mJ\to\bold{CGWH}$ be a diagram. Then there is a natural transformation $$\text{holim}_\mJ X\cong\text{Nat}(\mB(\mJ/-):\mJ\to\bold{CGWH},X)$$
\end{prp}


\pagebreak
\section{Calculus of Functors}
\begin{defn}{Homotopy Functors}{} Let $\mC,\mD$ be categories with a notion of weak equivalence. We say that a functor $F:\mC\to\mD$ is a homotopy functor if $F$ preserves weak equivalences. 
\end{defn}

\end{document}